% path0 <- "c:/data/GUK/"; path <- paste0(path0, "analysis/"); setwd(pathprogram <- paste0(path, "program/")); pathsource.mar <- paste0(path, "source/mar/"); pathreceived.mar <- paste0(path0, "received/mar/")
%  path0 <- "c:/data/GUK/"; path <- paste0(path0, "analysis/"); setwd(pathprogram <- paste0(path, "program/")); pathsource.mar <- paste0(path, "source/mar/"); pathreceived.mar <- paste0(path0, "received/mar/"); library(knitr); knit("ImpactEstimationOriginal1600.rnw", "ImpactEstimationOriginal1600.tex"); system("platex ImpactEstimationOriginal1600"); system("dvipdfmx ImpactEstimationOriginal1600")
%   path0 <- "c:/data/GUK/"; path <- paste0(path0, "analysis/"); setwd(pathprogram <- paste0(path, "program/")); system("recycle c:/data/GUK/analysis/program/cache/ImpactEstimationOriginal1600/"); library(knitr); knit("ImpactEstimationOriginal1600.rnw", "ImpactEstimationOriginal1600.tex"); system("platex ImpactEstimationOriginal1600"); system("dvipdfmx ImpactEstimationOriginal1600")

\input{c:/data/knitr_preamble.rnw}
\renewcommand\Routcolor{\color{gray30}}
\newtheorem{finding}{Finding}[section]
\makeatletter
\g@addto@macro{\UrlBreaks}{\UrlOrds}
\newcommand\gobblepars{%
    \@ifnextchar\par%
        {\expandafter\gobblepars\@gobble}%
        {}}
\newenvironment{lightgrayleftbar}{%
  \def\FrameCommand{\textcolor{lightgray}{\vrule width 1zw} \hspace{10pt}}% 
  \MakeFramed {\advance\hsize-\width \FrameRestore}}%
{\endMakeFramed}
\newenvironment{palepinkleftbar}{%
  \def\FrameCommand{\textcolor{palepink}{\vrule width 1zw} \hspace{10pt}}% 
  \MakeFramed {\advance\hsize-\width \FrameRestore}}%
{\endMakeFramed}
\makeatother
\usepackage{caption}
\usepackage{setspace}
\usepackage{framed}
\captionsetup[figure]{font={stretch=.6}} 
\def\pgfsysdriver{pgfsys-dvipdfm.def}
\usepackage{tikz}
\usetikzlibrary{calc, arrows, decorations, decorations.pathreplacing, backgrounds}
\usepackage{adjustbox}
\tikzstyle{toprow} =
[
top color = gray!20, bottom color = gray!50, thick
]
\tikzstyle{maintable} =
[
top color = blue!1, bottom color = blue!20, draw = white
%top color = green!1, bottom color = green!20, draw = white
]
\tikzset{
%Define standard arrow tip
>=stealth',
%Define style for different line styles
help lines/.style={dashed, thick},
axis/.style={<->},
important line/.style={thick},
connection/.style={thick, dotted},
}


\begin{document}
\setlength{\baselineskip}{12pt}






\begin{Schunk}
\begin{Sinput}
# Only change the trimming conditions to switch between "1 or 4" to "NoFlood"
ThisIsNoFlood <- F
\end{Sinput}
\end{Schunk}

\hfil Estimating lending impacts using original 1600 households\\

\hfil\MonthDY\\
\hfil{\footnotesize\currenttime}\\

\hfil Seiro Ito

\setcounter{tocdepth}{3}
\tableofcontents
\newpage

\setlength{\parindent}{1em}
\vspace{2ex}


\begin{Schunk}
\begin{Sinput}
setwd(path1234)
foldername <- list.dirs(path = ".", recursive = T, full.names = T)
fn <- list.files(path = foldername, pattern = ".dta$", 
	recursive = T, full.names = T)
fn <- fn[!grepl("orking|Live.*p.dta", fn)]
fn <- unique(fn)
fnd <- tolower(gsub(" ", "\\_", gsub("^.*\\/(.*).dta", "\\1", fn)))
\end{Sinput}
\end{Schunk}

\section{Summary}

\subsection{Definitions}

\begin{description}
\vspace{1.0ex}\setlength{\itemsep}{1.0ex}\setlength{\baselineskip}{12pt}
\item[Traditional]	A cash loan of Tk. 560 with one year maturity.
\item[Large]	A cash loan of Tk. 16800 with three year maturity.
\item[Large Grace]	A cash loan of Tk. 16800 with a one year grace period and three year maturity.
\item[Cow]	An in-kind loan of a cow worth Tk. 16800 with a one year grace period and three year maturity.
\item[LargeSize]	An indicator variable takes the value of 1 if the arm is Large, Large Grace, or Cow.
\item[WithGrace]	An indicator variable takes the value of 1 if the arm is Large Grace or Cow.
\item[InKind]	Same as Cow.
\end{description}
When one uses covariates \textsf{Large, Large Grace, Cow} in estimation, their estimates represent each arm's characteristics relative to \textsf{Traditional}. When one uses covariates \textsf{LargeSize, WithGrace, InKind}, their estimates represent their labeled names.

\subsection{Findings}

\begin{description}
\vspace{1.0ex}\setlength{\itemsep}{1.0ex}\setlength{\baselineskip}{12pt}
\item[Net saving and repayments]	 Sample uses all administrative records available. Smaller net saving increments for \textsf{traditional} arm. Period of rd 2-3 saw a decline in net saving, even further for \textsf{LargeGrace}, but remain in positive values (\textsc{Table \ref{tab FD saving original HH}}). \textsc{\normalsize Table \ref{tab FD saving attributes original HH}} reveals \textsf{LargeSize} have larger net saving changes while both \textsf{WithGrace} and \textsf{NonCash} hav smaller changes. Repayment changes are larger with \textsf{LargeSize} and \textsf{WithGrace} but smaller with \textsf{NonCash} in (4). Repayment is positively autocorrelated and is negatively correlated with previous net saving. Repayment is also positively correlated with other member's previous repayment, which can be explained by common shocks and/or strategic cooperative behaviours. 
The ultra poor repaid just as much as the moderately poor (\textsc{Table \ref{tab FD saving2 original HH}}). This is evidence against the popular belief that the ultra poor are riskier.  
\item[Schooling]	Enrollment changes are larger for primary school girls in \textsf{Large} and \textsf{Cow} arms for primary but smaller for junior in rd 1 vs rd 4 comparisons (\textsc{Table \ref{tab FD enroll5 original HH}}). When seen by attributes in \textsc{\normalsize Table \ref{tab FD enroll5 attributes original HH}}, \textsf{LargeSize} shows smaller changes especially for primary school boys. Primary school girls in \textsf{LargeSize} and \textsf{NonCash} show larger changes, while junior and high school girls in \textsf{LargeSize} show smaller changes than boys. This indicates that large sized arms have detrimetal impacts on older girls' schooling but promotional impacts on primary school aged girls. No decline in enrollment changes when repaying for the arms of \textsf{WithGrace}, despite the larger installments.
\item[Assets]	Household assets increased in all arms. Initially increased then decreased. There might have been liquidation of assets to repay the loans. Productive assets declined consecutively. Flood in rd 1 makes the increase in household assets smaller. Productive assets see a major decline among \textsf{Large} during rd 3-4 period (\textsc{\normalsize Table \ref{tab FD assets original HH}}). Comparison by attributes (\textsc{\normalsize Table \ref{tab FD assets attributes original HH}}) or of rd 2 and rd 4 gives the same picture (\textsc{\normalsize Table \ref{tab FD assets rd24 grace original HH}}). Comparison against the loan non-recipients shows that they also experience a similar, increase-increase-decrease pattern. This indicates that the pattern observed among the loan recipients may be a systemic pattern of the area, not necessarily reflecting the repayment burdern (\textsc{\normalsize Table \ref{tab FD assets pure control original HHs}}).
\item[Livestock]	Larger increases in holding values in rd 1-2, smaller increases in rd 2-3, no change in rd 3-4. Previous cow owners show a smaller increase in rd 1-2 while not rd 3-4 or rd 2-3 in the \textsf{Cow} arm (\textsc{\normalsize Table \ref{tab FD livestock original HH}}). Figures show that cow ownership increased for all arms but the \textsf{traditional} arm. \textsc{\normalsize Table \ref{tab FD livestock attributes original HH}} shows baseline trend is a large increse in rd 1-2, a small increase in rd 2-3, a small decline in rd 3-4, while \textsf{LargeSize} sees an even larger increase in rd 1-2 and similar trend as baseline afterwards. This shows that member who received a larger sized disbursement could hold on to its level of livestock accumulation. \textsc{\normalsize Table \ref{tab FD livestock poor original HH}} shows, albeit at $p$ values around 10\%, the ultra poor has a larger increase relative to the moderately poor, which is another manifestation against popular notion that the ultra poor are riskier.
\item[Total asset values]	Similar resulsts as assets.
\item[Labour incomes]	Small sample. Increased during rd 2-3 in all arms (\textsc{\normalsize Table \ref{tab FD incomes original HH}}). 
\item[Consumption]	Increased during rd 2-3 in all arms, a decrese in rd 3-4 (\textsc{\normalsize Table \ref{tab FD consumption original HH}}). Another notable result is that \textsf{NonCash} reduced the consumption in rd 3-4 even further than the baseline loan (\textsc{\normalsize Table \ref{tab FD consumption attributes original HH}}).
\item[IGAs]	Multiple IGAs for \textsf{Tradtional} arm. Everyone else chose to invest in cows, suggesting entrepreneurship does not seem to matter in the uptake of loans. It is consistent with the presence of a poverty trap induced by a liquidity constraint and convexity in livestock production technology.
\end{description}

One sees changes in investment choices when one compares \textsf{traditional} and all other arms. However, consumption does not seem to differ. Repayments and asset holding are greater in all other arms. These are consistent with households are enforcing the repayment disciplines and reinvesting the proceeds rather than increasing consumption. 


A more detailed summary:
\begin{description}
\vspace{1.0ex}\setlength{\itemsep}{1.0ex}\setlength{\baselineskip}{12pt}
\item[Low repayment rates]	Repayment was poor. Net saving was forfeit for repayment. Mean raw loan recovery rate (counting only repayments) measured at the end of third year was 0.67 overall, and was lowest for \textsf{traditional} at 0.48. Counting also net saving, these numbers change to 0.85, 0.59, respectively.
\item[Large-sized or grace period loans resulted in higher repayment rates]	Controlling for the loan size, larger initial lending resulted in larger repayment and net saving. As opposed to GUK's anxiety, lending was relatively less risky with large loans and loans with a grace period.
\item[No difference in repayment risk by poverty status] Raw loan recovery rates are 0.67, 0.67, respectively, for ultra poor and moderately poor. Also no statitically meaningful difference is found for cumulative repayment plus cumulative net saving.
\item[No difference in household assets]	Household assets increased in rd 1 - 3, then reduced in rd 4 (possibly liquidating for repayment purpose), with the overall impact of increased household asset values yet no statistically significant difference between arms. 
\item[No difference in labour incomes, per member consumption, marriage rates]	Per member consumption increased in all arms with no difference between arms. Marriage rates do not difffer between arms. A greater swing in labour incomes for \textsf{large}.
\end{description}

\subsection{Inference}

\begin{itemize}
\vspace{1.0ex}\setlength{\itemsep}{1.0ex}\setlength{\baselineskip}{12pt}
\item	First-difference estimators are used. This can be seen as an extension of DID to multi-periods (although historically the latter precedes the former). FD is used also for a binary indicator such as schooling.
\item	All the standard errors are clustered at the group (char) level.
\item	To aid the understanding if the data is more suited to the assumption of first-difference rather than fixed-effects, I used a check suggested by Wooldridge 10.71. It is an AR(1) regression of FD residuals. Most of results show low autocorrelations which is consistent with an assumption of FD estimator.
\end{itemize}

\section{Read files}


\subsection{Read from a list}

In reading raw files, I added ID information (\textsf{\footnotesize ./ID/ID\_Updated\_received\_from\_Abu.dta}) to all pages. I will further add HH ID information from the admin file if possible.

\begin{Schunk}
\begin{Sinput}
sch1 <- readRDS(paste0(path1234, "schooling_Age6-18InRd1.rds"))
ZB <- readRDS(paste0(path1234, "data_read_in_a_list_with_baseline_patched.rds"))
# roster
ros <- ZB[[1]]
# relocation
loc <- ZB[[grep("relo", names(ZB))]]
#loc[, .(hhid, survey, IntDate, duration_year, current_loc)]
# poverty
pov <- ZB[[grep("poverty$", names(ZB))]]
pov <- ZB[[grep("pov.*up", names(ZB))]]
# shocks
shk <- ZB[[grep("shock", names(ZB))]]
# asset (HH and productive
ass <- ZB[[grep("Mer", names(ZB))]]
# livestock ownership
lvo <- ZB[[grep("liv.*own", names(ZB))]]
# labour income
lab <- ZB[[grep("la.*come", names(ZB))]]
# farm income
far <- ZB[[grep("fa.*ion", names(ZB))]]
# HH consumption
con <- ZB[[grep("hh.con", names(ZB))]]
\end{Sinput}
\end{Schunk}
\begin{Schunk}
\begin{Sinput}
xid <- readRDS(paste0(path1234, "ID.rds"))
\end{Sinput}
\end{Schunk}
\begin{Schunk}
\begin{Sinput}
# fill in original arm assignment of drop outs, forced drop outs
probgp <- fread(paste0(path0, "received/CharRandomization2012.prn"))
probgp[, randomization0 := randomization]
probgp[grepl("grace", randomization0), randomization0 := "large grace"]
probgp[grepl("credit$", randomization0), randomization0 := "large"]
probgp[grepl("con", randomization0), randomization0 := "traditional"]
probgp[grepl("pack", randomization0), randomization0 := "cow"]
probgp <- probgp[, .(group.id, randomization0, comment)]
xid[, ObsPattern := "1111"]
xid[grepl("^Fi", missing_followup), ObsPattern := "0111"]
xid[grepl("^Se", missing_followup), ObsPattern := "1011"]
xid[grepl("^En", missing_followup), ObsPattern := "1110"]
xid[grepl("^2nd and 4", missing_followup), ObsPattern := "1010"]
xid[grepl("^3rd and 4", missing_followup), ObsPattern := "1100"]
xid[grepl("^2.*3.*4", missing_followup), ObsPattern := "1000"]
\end{Sinput}
\end{Schunk}

\begin{Schunk}
\begin{Sinput}
ass <- ass[, .(gid, hhid, Mstatus, AssignRegression, AssignOriginal, survey, DistDate1, IntDate, HAssetAmount, PAssetAmount, AssetAmount)]
ros <- ros[, .(AssignRegression, AssignOriginal, 
  gid, hhid, Mstatus, Mgroup, ObPattern, AttritIn,
  IntDate, year, survey, 
  mid, rel_hhh, fmid, mmid, sex, Age_1, AgeComputed, 
  current, marital, stay, nonstaym, reasons, literacy, edu, 
  HeadLiteracy, HeadAge, HHsize, randomization)]
ros[, HeadLiteracy := as.numeric(HeadLiteracy)]
sch1[, groupid := as.integer(as.numeric(as.character(gid)))]
ros[, groupid := as.integer(as.numeric(as.character(gid)))]
ros[grepl("2004", IntDate), IntDate := 
  strptime(gsub("2004", "2014", as.character(IntDate)), format = "%Y-%m-%d")]
ros[grepl("2005", IntDate), IntDate := 
  strptime(gsub("2005", "2015", as.character(IntDate)), format = "%Y-%m-%d")]
ros[, Year := as.numeric(format(as.Date(IntDate), "%Y"))]
ros[Year <= 2010, Year := Year + 10]
ros[, Month := format(as.Date(IntDate), "%B")]
setorder(ros, gid, hhid, IntDate, -Age_1, mid)
ros[, ChildAgeOrderAtRd1 := as.integer(NA)]
ros[grepl("^son", rel_hhh), ChildAgeOrderAtRd1 := 1:.N, by = .(IntDate, hhid)]
# ChildAgeOrderAtRd1 is complete and no child is left unordered.
if (any(nrow(ros[is.na(ChildAgeOrderAtRd1) & Age_1 <= 18 & grepl("^son", rel_hhh)]) > 0))
  ros[is.na(ChildAgeOrderAtRd1) & Age_1 <= 18 & grepl("^son", rel_hhh)]
ros[, c("EldestSon", "EldestDaughter") := 0L]
ros[grepl("^son", rel_hhh) & ChildAgeOrderAtRd1 == 1 & sex == "Male", EldestSon := 1L]
ros[grepl("^son", rel_hhh) & ChildAgeOrderAtRd1 == 1 & sex == "Female", EldestDaughter := 1L]
ros[, MarriedOff := 
  any(grepl("marr", .SD[, reasons])) & 
  !any(grepl("dea|job", .SD[, reasons])), 
  .SDcols = "reasons", by = .(hhid, mid)]
# Why below?
# drop head/spouse or adults (age > 18) but keep married off
#ros <- ros[MarriedOff | !(grepl("head|spo", rel_hhh) | Age_1 >18), ]
# drop married off
ros <- ros[!(MarriedOff), ]
# livestock
lvo <- lvo[, .(gid, hhid, survey, IntDate, 
  dummyHadCows, NumCows, NumCowsOwnedAtRd1, 
  sales_cow, sales_ox, sales_sheep, dead, born, 
  nowned_cow, nowned_ox,
  LivestockCode, number_owned, mrkt_value, total_cost, 
  TotalImputedValue, TotalSelfEvaluatedValue)]
# labour income
lab <- lab[, .(gid, hhid, survey, IntDate, mid, 
  code_1, duration_1, income1, code_2, duration_2, income2, 
  totalincome, TotalHHLabourIncome)]
# farm income
far <- far[, .(gid, hhid, survey, IntDate, 
  area_1, crop_code_1, total_production_1, Revenue1, 
  area_2, crop_code_2, total_production_2, Revenue2, 
  area_3, crop_code_3, total_production_3, Revenue3, TotalRevenue)]
# HH consumption
con = copy(ZB[[grep("hh.con", names(ZB))]])
con <- a2b.data.table(con, NA, 0)
setnames(con, "pulse_total", "pulses_total")
setnames(con, "pgarlic_exp", "garlic_exp")
setnames(con, "bettel_total", "bettle_total")
setnames(con, grepout("other_pulse", colnames(con)), 
  gsub("other\\_pu", "otherpu", grepout("other_pulse", colnames(con))))
setnames(con, grepout("chew_tob", colnames(con)), 
  gsub("chew\\_to", "chewto", grepout("chew_tob", colnames(con))))
items <- unique(unlist(strsplit(grepout("bought", colnames(con)), "_")))
items <- items[!grepl("bought", items)]
for (i in items) {
  con[, paste0(i, "_UPrice") := 
    eval(parse(text = paste0(i, "_exp"))) / eval(parse(text = paste0(i, "_bought")))]
}
# Set time unit to annual.
TimeUnitForCon <- rep(7, length(items))
TimeUnitForCon[grep("oil", items):length(items)] <- 30
con[, paste0(items[TimeUnitForCon == 7], "_AnnTotal") := 
  eval(parse(text = paste0(items[TimeUnitForCon == 7], "_total"))) * 4.5 * 12]
con[, paste0(items[TimeUnitForCon == 30], "_AnnTotal") := 
  eval(parse(text = paste0(items[TimeUnitForCon == 30], "_total"))) * 12]
con[, paste0(items[TimeUnitForCon == 7], "_AnnBought") := 
  eval(parse(text = paste0(items[TimeUnitForCon == 7], "_bought"))) * 4.5 * 12]
con[, paste0(items[TimeUnitForCon == 30], "_AnnBought") := 
  eval(parse(text = paste0(items[TimeUnitForCon == 30], "_bought"))) * 12]
# Inf -> NA (so median price to stay finite)
con <- a2b.data.table(con, Inf, NA)
for (i in items) {
  con[, paste0(i, "_MedianUPrice") := 
    median(eval(parse(text = paste0(i, "_UPrice"))), na.rm = T), by = year]
  con[, paste0(i, "_ImputedValue") := 
    eval(parse(text = paste0(i, "_MedianUPrice"))) * 
      eval(parse(text = paste0(i, "_AnnTotal")))]
  # errors: total < bought. => use bought as total.
  con[eval(parse(text = paste0(i, "_AnnTotal"))) < eval(parse(text = paste0(i, "_AnnBought"))), 
    paste0(i, "_ImputedValue") := 
      eval(parse(text = paste0(i, "_MedianUPrice"))) * 
      eval(parse(text = paste0(i, "_AnnBought")))]
}
# NA -> 0 (so total becomes nonNA)
con <- a2b.data.table(con, NA, 0)
con[, HygieneExpenditure := cloth_exp + soap_exp + haircut_exp + cosmetic_exp]
con[, SocialExpenditure := fest_exp + mosque_exp + contra_exp + social_exp]
con[, EnergyExpenditure := fuel_wood_exp + transport_exp + communication_exp + other_exp]
items <- items[items != "tea"]
con[, FoodExpenditure := 
  eval(parse(text = paste(grepout("AnnTotal", colnames(con)), collapse = "+")))]
con <- con[, .(gid, hhid, 
  IntDate, survey, FoodExpenditure, HygieneExpenditure, 
  SocialExpenditure, EnergyExpenditure)]
# shocks
# shk[!is.na(code_1) | !is.na(code_2), .(hhid, survey, IntDate, code_1, damage_amount_1, 
#   code_2, damage_amount_2)]
shk[, damage_amount_1 := as.numeric(damage_amount_1)]
shk[, damage_amount_2 := as.numeric(damage_amount_2)]
shkoverview <- shk[, .(Shock1 = round(sum(!is.na(code_1))/.N, 3), 
  MedianD1 = median(damage_amount_1, na.rm = T), 
  MeanD1 = round(mean(damage_amount_1, na.rm = T), 0), 
  Shock2 = round(sum(!is.na(code_2))/.N, 3), 
  MedianD2 = median(damage_amount_2, na.rm = T),
  MeanD2 = round(mean(damage_amount_2, na.rm = T), 0),
  AvgNumShocks = round((sum(!is.na(code_1)) + sum(!is.na(code_2)) + 
    sum(!is.na(code_3)) + sum(!is.na(code_4)))/.N, 3)), 
  by = .(survey, AssignOriginal)]
setkey(shkoverview, survey, AssignOriginal)
#shkoverview
#table0(shk[, .(survey, code_1)])
shk <- ZB[[grep("shock", names(ZB))]]
codecols <- grepout("code", colnames(shk))
shk[, (paste0("Code", 1:4)) := lapply(.SD, function(x) !is.na(x)), .SDcols = codecols]
shk[, (paste0("Code", 1:4)) := lapply(.SD, as.numeric), .SDcols = paste0("Code", 1:4)]
shk[, NumberOfShocks := eval(parse(text = paste(paste0("Code", 1:4), collapse = "+")))]
setkey(shk, hhid, survey)
shk[, FloodInRd1 := 0L]
shk[survey == 1, FloodInRd1 := as.integer(grepl("Fl", code_1))]
shk[, FloodInRd1 := FloodInRd1[1], by = hhid]
shk <- shk[, .(gid, hhid, survey, IntDate, Hhidyear, FloodInRd1, code_1, code_2, 
  damage_amount_1, damage_amount_2, NumberOfShocks)]
\end{Sinput}
\end{Schunk}

Description of data:
\begin{description}
\vspace{1.0ex}\setlength{\itemsep}{1.0ex}\setlength{\baselineskip}{12pt}
\item[ad]	Administrative data: Up to [-24, 48] months after first loan disbursement. This file has not been used in \textsf{read\_cleaned\_data.rnw}.
\item[sch1]	Schooling panel with attrition. Aged 6-18 in rd1. \textsf{Enrolled=\{0,1\}} is defined for children aged 6-18 in rd1 by referencing to \textsf{currently\_enrolled} and age information.
\item[ros]	 \textsf{roster} to condition the initial status prior to participation.
\item[ass]	 Assets. Household assets (houses, durables) and productive assets (machines, tools). 
\item[lvo]	Livestock holding. Rd 3 data is not entered yet.
\item[lab]	Labour incomes.
\item[far]	Farming revenues (no costs reported).
\item[con]	Household consumption. Food expenditure asks both bought and consumed volumes and prices. We impute consumption values by using median prices. All quantity is set to annualised quantity.
\item[shk]	Shocks. 
\end{description}

\subsection{Sample selection and treament assignment}

\subsubsection{Admin info}

\begin{Schunk}
\begin{Sinput}
adw2 <- readRDS(paste0(path1234, "admin_data_wide2.rds"))
idfu <- readRDS(paste0(pathsave, "idfu.rds"))
\end{Sinput}
\end{Schunk}
Redefine arms to include \textsf{DropOuts} in original arms.
\begin{Schunk}
\begin{Sinput}
idfu[, ArmInidfu := unique(arm[!is.na(arm) & arm !="before intervention"]), by = hhid]
idfu <- unique(idfu[, .(hhid, ArmInidfu)])
\end{Sinput}
\end{Schunk}
\begin{Schunk}
\begin{Sinput}
setkey(idfu, hhid)
setkey(adw2, hhid)
adw3 <- idfu[adw2]
adw3[, MemNum := 1:.N, by = .(hhid, Year)]
#table0(adw3[MemNum==1, .(ArmInidfu, randomArm)])
adw3[, RArm := Arm]
adw3[grepl("^drop", Arm) & grepl("con", randomArm), RArm := "traditional"]
adw3[grepl("^drop", Arm) & grepl("^La.*t$", randomArm), RArm := "large"]
adw3[grepl("^drop", Arm) & grepl("^La.*gr", randomArm), RArm := "large grace"]
adw3[grepl("^drop", Arm) & grepl("^pack", randomArm), RArm := "cow"]
ad0 <- adw3[, 
  .(RArm, Arm, randomArm, groupid, hhid, TradGroup, 
    creditstatus, Mem, povertystatus,
    Date, Year, Month, DisDate1, MonthsElapsed, MonthsRepaid, LoanYear,
    EffectiveRepayment, value.repay, value.NetSaving, value.missw,
    OtherRepaid, OtherNetSaving, OtherMisses, CumOtherMisses,
    CumRepaid, CumEffectiveRepayment, CumNetSaving, CumPlannedInstallment,
    CumOtherRepaid, CumOtherNetSaving, CumMisses, EffectivelyFullyRepaid,
    CumRepaidRate, CumEffectiveRepaidRate)]
\end{Sinput}
\end{Schunk}
\begin{Schunk}
\begin{Sinput}
#table0(ad0[, .(Arm, MonthsElapsedNA = is.na(MonthsElapsed))])
if (nrow(ad0[is.na(EffectiveRepayment)]) > 0) ad0[is.na(EffectiveRepayment)]
if (nrow(ad0[CumPlannedInstallment != 0L & is.na(CumEffectiveRepaidRate)]) > 0) 
  ad0[is.na(CumEffectiveRepaidRate)]
ad0[,  c("CumRepaidRate", "CumEffectiveRepaidRate") := 
  .(round(CumRepaid/CumPlannedInstallment, 3), 
    round(CumEffectiveRepayment/CumPlannedInstallment, 3))]
ad0[CumPlannedInstallment == 0L, c("CumRepaidRate", "CumEffectiveRepaidRate") := NA]
ad0[, MeanYearlyCERR := mean(CumEffectiveRepaidRate, na.rm = T), by = .(hhid, LoanYear)]
\end{Sinput}
\end{Schunk}
\begin{Schunk}
\begin{Sinput}
# add rolling means
library(zoo)
\end{Sinput}
\begin{Soutput}
Warning: package 'zoo' was built under R version 3.5.2
\end{Soutput}
\begin{Soutput}

Attaching package: 'zoo'
\end{Soutput}
\begin{Soutput}
The following objects are masked from 'package:base':

    as.Date, as.Date.numeric
\end{Soutput}
\begin{Sinput}
rollvars <- c("value.missw", "value.repay", "value.NetSaving", "OtherNetSaving", "OtherRepaid")
ad0[, (paste0("RM", rollvars)) := lapply(.SD, rollmean, k = 6, na.pad = TRUE), 
   by = hhid, .SDcols = rollvars]
  # lag rolling means by 3 months to get previous 6 month averages	
ad0[, (paste0("RM", rollvars)) := shift(.SD, n=3, type = "lag"), 
   by = hhid, .SDcols = paste0("RM", rollvars)]
ad0[, RMvalue.repay := RMvalue.repay/1000]
ad0[, RMvalue.NetSaving := RMvalue.NetSaving/1000]
ad0[, RMOtherRepaid := RMOtherRepaid/1000]
ad0[, RMOtherNetSaving := RMOtherNetSaving/1000]
#ad <- ad0[MonthsElapsed == 12 | MonthsElapsed == 24 | MonthsElapsed == 36 | MonthsElapsed == 48]
ad0[, c("EffectiveRepayment", "value.repay", "value.NetSaving", "value.missw") := NULL] 
\end{Sinput}
\end{Schunk}

\subsubsection{Merge admin and roster files}

How I combined between pages: First, merge time-invariant portion of admin data \textsf{adbase} with roster data \textsf{ros} with \textsf{hhid} as a key. Then it is merged with time-variant portion of admin data \textsf{adrest} with \textsf{hhid, Year, Month} as keys. %Keep only dates when survey data match. 
Second, merge \textsf{adbase+adrest+ros} with other data \textsf{sch1}, \textsf{ass}, ... 

By merging in this way, I have \textsf{RArm} information for each HH in survey 1:
\begin{Schunk}
\begin{Sinput}
table0(ar.0[, .(teenum =1:.N, RArm), by = .(hhid, survey)][
  survey == 1 & teenum == 1, RArm])
\end{Sinput}
\begin{Soutput}

traditional       large large grace         cow        <NA> 
        485         464         467         487         220 
\end{Soutput}
\begin{Sinput}
table0(ar.1[, .(Arm, AssignOriginal)])
\end{Sinput}
\begin{Soutput}
Error in table0(ar.1[, .(Arm, AssignOriginal)]): object 'ar.1' not found
\end{Soutput}
\end{Schunk}
Observations with no \textsf{povertystatus} are drop outs and rejecters.
\begin{Schunk}
\begin{Sinput}
table0(ar.0[, povertystatus])
\end{Sinput}
\begin{Soutput}

   Ultra Poor Moderate Poor          <NA> 
        21203          9255          2765 
\end{Soutput}
\begin{Sinput}
table0(ar.0[is.na(povertystatus), .(Mstatus, survey)])
\end{Sinput}
\begin{Soutput}
              survey
Mstatus          1   2   3   4
  gErosion     344 229 233   0
  gRejection   560 487 466   0
  iRejection     0   0   0 446
  iReplacement   0   0   0   0
  newGroup       0   0   0   0
  oldMember      0   0   0   0
\end{Soutput}
\begin{Sinput}
summary(ar.0[hhid %in% hhid[is.na(povertystatus)], 
  .(hhid, Mstatus, survey, creditstatus)])
\end{Sinput}
\begin{Soutput}
      hhid                  Mstatus         survey              creditstatus 
 Min.   : 7020501   gErosion    : 806   Min.   :1.00   Yes            :   0  
 1st Qu.: 7031914   gRejection  :1513   1st Qu.:1.00   No             :   0  
 Median : 7085811   iRejection  : 446   Median :2.00   Replaced Member:   0  
 Mean   :13884824   iReplacement:   0   Mean   :2.25   NA's           :2765  
 3rd Qu.: 8148314   newGroup    :   0   3rd Qu.:3.00                         
 Max.   :81710220   oldMember   :   0   Max.   :4.00                         
\end{Soutput}
\end{Schunk}


There are 46 members (\textsf{newGroup} in \textsf{Mstatus}) who did not borrow but only saved. 
\begin{Schunk}
\begin{Sinput}
summary(ar.1[is.na(DisDate1) & survey == 1 & MemNum == 1, 
  .(survey, DisDate1, creditstatus, Mstatus)] )
\end{Sinput}
\begin{Soutput}
     survey     DisDate1            creditstatus         Mstatus   
 Min.   :1   Min.   :NA    Yes            :  0   gErosion    : 80  
 1st Qu.:1   1st Qu.:NA    No             :208   gRejection  :140  
 Median :1   Median :NA    Replaced Member:  0   iRejection  :159  
 Mean   :1   Mean   :NA    NA's           :220   iReplacement:  3  
 3rd Qu.:1   3rd Qu.:NA                          newGroup    : 20  
 Max.   :1   Max.   :NA                          oldMember   : 26  
             NA's   :428                                           
\end{Soutput}
\end{Schunk}
So are the same with 104 \textsf{oldMember} in \textsf{Mstatus}:
\begin{Schunk}
\begin{Sinput}
summary(ar.1[is.na(DisDate1) & MemNum == 1 & grepl("old", Mstatus), 
  .(groupid = factor(groupid), survey, DisDate1, creditstatus, 
    Mstatus, CumRepaid, CumNetSaving, Arm)])
\end{Sinput}
\begin{Soutput}
  groupid       survey        DisDate1            creditstatus
 70425:20   Min.   :1.00   Min.   :NA    Yes            :  0  
 70650:12   1st Qu.:1.75   1st Qu.:NA    No             :104  
 70861:28   Median :2.50   Median :NA    Replaced Member:  0  
 71166: 8   Mean   :2.50   Mean   :NA                         
 71372:12   3rd Qu.:3.25   3rd Qu.:NA                         
 81693:24   Max.   :4.00   Max.   :NA                         
                           NA's   :104                        
         Mstatus      CumRepaid      CumNetSaving            Arm     
 gErosion    :  0   Min.   :    0   Min.   :-2780   traditional:104  
 gRejection  :  0   1st Qu.:    0   1st Qu.:    0   large      :  0  
 iRejection  :  0   Median :    0   Median :  462   large grace:  0  
 iReplacement:  0   Mean   :  844   Mean   :  487   cow        :  0  
 newGroup    :  0   3rd Qu.:    0   3rd Qu.:  958                    
 oldMember   :104   Max.   :16800   Max.   : 1804                    
                    NA's   :26      NA's   :26                       
\end{Soutput}
\end{Schunk}
There are 12 members (\textsf{iReplacement} in \textsf{Mstatus}) who did not borrow but only saved. 
\begin{Schunk}
\begin{Sinput}
summary(ar.1[is.na(DisDate1) & MemNum == 1 & grepl("Rep", Mstatus), 
  .(groupid = factor(groupid), survey, DisDate1, creditstatus, 
    Mstatus, CumRepaid, CumNetSaving, Arm)])
\end{Sinput}
\begin{Soutput}
  groupid       survey        DisDate1           creditstatus         Mstatus  
 70650:12   Min.   :1.00   Min.   :NA   Yes            : 0    gErosion    : 0  
            1st Qu.:1.75   1st Qu.:NA   No             :12    gRejection  : 0  
            Median :2.50   Median :NA   Replaced Member: 0    iRejection  : 0  
            Mean   :2.50   Mean   :NA                         iReplacement:12  
            3rd Qu.:3.25   3rd Qu.:NA                         newGroup    : 0  
            Max.   :4.00   Max.   :NA                         oldMember   : 0  
                           NA's   :12                                          
   CumRepaid  CumNetSaving           Arm    
 Min.   :0   Min.   :  60   traditional:12  
 1st Qu.:0   1st Qu.: 150   large      : 0  
 Median :0   Median : 220   large grace: 0  
 Mean   :0   Mean   : 481   cow        : 0  
 3rd Qu.:0   3rd Qu.: 585                   
 Max.   :0   Max.   :1415                   
                                            
\end{Soutput}
\end{Schunk}
Create \textsf{BorrowerStatus} to indicate these guys. \gobblepars
\begin{Schunk}
\begin{Sinput}
ar.1[, BorrowerStatus := "borrower"]
ar.1[is.na(DisDate1) & MemNum == 1 & grepl("old|new|Rep", Mstatus), 
  BorrowerStatus := "pure saver"]
ar.1[, BorrowerStatus := factor(BorrowerStatus)]
\end{Sinput}
\end{Schunk}
Set \textsf{No} in \textsf{creditstatus} if NA in \textsf{DisDate1}.
\begin{Schunk}
\begin{Sinput}
ar.1[is.na(DisDate1), creditstatus := "No"]
summary(ar.1[is.na(DisDate1) & survey == 1 & MemNum == 1, 
  .(survey, DisDate1, creditstatus, Mstatus, BorrowerStatus)] )
\end{Sinput}
\begin{Soutput}
     survey     DisDate1            creditstatus         Mstatus   
 Min.   :1   Min.   :NA    Yes            :  0   gErosion    : 80  
 1st Qu.:1   1st Qu.:NA    No             :428   gRejection  :140  
 Median :1   Median :NA    Replaced Member:  0   iRejection  :159  
 Mean   :1   Mean   :NA                          iReplacement:  3  
 3rd Qu.:1   3rd Qu.:NA                          newGroup    : 20  
 Max.   :1   Max.   :NA                          oldMember   : 26  
             NA's   :428                                           
    BorrowerStatus
 borrower  :379   
 pure saver: 49   
                  
                  
                  
                  
                  
\end{Soutput}
\end{Schunk}
\begin{Schunk}
\begin{Sinput}
ar.1[, YearMonthOfIntDate := paste0(Year, "-", Month)]
ar.1[, YearMonthOfIntDate := factor(YearMonthOfIntDate, 
  levels = unique(YearMonthOfIntDate[order(IntDate)]), ordered = T)]
YearMonthMatchTable <- table(ros[, paste0(hhid, "-", Year, "-", Month)] %in% 
    ad0[, paste0(hhid, "-", Year, "-", Month)])
\end{Sinput}
\end{Schunk}
Need to merge in 2 steps: Merge admin (time-invariant) with roster with \textsf{hhid} as a key, then merge to admin (time-variant {\footnotesize [e.g., OtherRepaid, OtherNetSaving, OtherMisses, CumOtherMisses, CumRepaid, CumEffectiveRepayment, CumNetSaving, CumPlannedInstallment, CumOtherRepaid, CumOtherNetSaving, CumMisses, CumRepaidRate, CumEffectiveRepaidRate, RMOtherNetSaving, RMOtherRepaid]}) with \textsf{hhid, Year, Month} as keys. This is because there are 8398 non-matching cases if we merge using \textsf{Year, Month} of \textsf{IntDate} in roster data and \textsf{Year, Month} of \textsf{Date} in admin data. This is inevitable because survey precedes the first meeting of borrowers: The admin data starts from 2013-05-01 while survey data starts from 2011-10-09 and rd 1 ends at 2013-10-12 for \textsf{oldMember}s with the median date 2012-10-20. Below gives \textsf{Year, Month} in roster data in rd 1 with no match in admin data.
\begin{Schunk}
\begin{Sinput}
setkey(ros, groupid, hhid, Year, Month)
setkey(ad0, groupid, hhid, Year, Month)
ar00 <- ad0[ros]
ar00[, MemNum := 1:.N, by = .(hhid, Date)]
ar00[, YearMonthOfIntDate := paste0(Year, "-", Month)]
ar00[, YearMonthOfIntDate := factor(YearMonthOfIntDate, 
  levels = unique(YearMonthOfIntDate[order(IntDate)]), ordered = T)]
table0(ar00[is.na(MonthsElapsed) & MemNum == 1, 
  YearMonthOfIntDate])
\end{Sinput}
\begin{Soutput}

  2011-October  2011-November   2012-January   2012-October  2012-November 
             6              1             19           1146            327 
 2012-December 2013-September   2013-October   2014-January   2014-October 
            79              6             19             12             83 
 2014-November  2014-December  2015-November  2015-December   2016-January 
            43             36            111             40             26 
  2017-January  2017-February     2017-March     2017-April          NA-NA 
            44             97             17             17             21 
\end{Soutput}
\end{Schunk}
After 2014, it is mostly drop out members who do not match with admin data because they do not attend the meeting.
\begin{Schunk}
\begin{Sinput}
table0(ar00[is.na(MonthsElapsed) & MemNum == 1 & Year >= 2014, 
  Mgroup])
\end{Sinput}
\begin{Soutput}

   continued    drop outs    new group replacements 
          78          381           58            9 
\end{Soutput}
\end{Schunk}
No additional match if matching only with \textsf{Year}. 
\begin{Schunk}
\begin{Sinput}
ros[, MemNum := 1:.N, by = .(hhid, IntDate)]
rbind(YearMonthMatch = table(ros[MemNum == 1, paste0(hhid, "-", Year, "-", Month)] %in% 
    ad0[, paste0(hhid, "-", Year, "-", Month)]), 
  YearMatch = table(ros[MemNum == 1, paste0(hhid, "-", Year)] %in% 
    ad0[, paste0(hhid, "-", Year)]))
\end{Sinput}
\begin{Soutput}
               FALSE TRUE
YearMonthMatch  2055 5958
YearMatch       2055 5958
\end{Soutput}
\begin{Sinput}
# iiNotInAdminData <- !(ros[, paste0(hhid, "-", Year, "-", Month)] %in% 
#   ad0[, paste0(hhid, "-", Year, "-", Month)])
# iiNotInAdminData <- ros[(iiNotInAdminData), hhid]
# UiiNotInAdminData <- unique(iiNotInAdminData)
#table(ros[hhid %in% iiNotInAdminData & MemNum == 1, 
#  .(survey, AssignRegression)])
setorder(ar.1, hhid, survey, IntDate, mid)
ar.1[, MemNum := 1:.N, by = .(hhid, survey, IntDate)]
\end{Sinput}
\end{Schunk}
In \textsf{roster + admin} (base: roster): Tabulate \textsf{hhid} observations by \textsf{survey} round and \textsf{RArm} before supplementing with \textsf{AssignOriginal} and \textsf{VArm}. Note: 220 observations with NA are also pointed in \textsf{read\_cleaned\_data.rnw} and are going to be dealt with in the next subsection.
\begin{Schunk}
\begin{Sinput}
ar.1[, YearMonthOfIntDate := NULL]
table0(ar.1[MemNum == 1, .(survey, RArm)])
\end{Sinput}
\begin{Soutput}
      RArm
survey traditional large large grace cow <NA>
     1         485   464         467 487  220
     2         472   445         447 446  173
     3         472   448         452 453  168
     4         465   444         447 444  114
\end{Soutput}
\begin{Sinput}
table0(ar.1[MemNum == 1, .(survey, AssignOriginal)])
\end{Sinput}
\begin{Soutput}
      AssignOriginal
survey traditional large large grace cow <NA>
     1         485   464         467 487  220
     2         472   445         447 446  173
     3         472   448         452 453  168
     4         465   444         447 444  114
\end{Soutput}
\begin{Sinput}
ar.1[, MemNum := NULL]
# droplevels do not work... it does not copy NAs.
#ar.1[, RArm := droplevels(RArm)]
\end{Sinput}
\end{Schunk}

\subsubsection{Merge village level info}

\begin{Schunk}
\begin{Sinput}
library(readstata13)
vr <- read.dta13(paste0(pathcleaned, "RCT_village.dta"), 
  generate.factors = T, nonint.factors = T)
vr <- data.table(vr)
vr[, GroupStatus := "accepted"]
vr[grepl("De", comment), GroupStatus := "group rejection"]
vr[grepl("Ero", comment), GroupStatus := "erosion"]
setnames(vr, c("comment", "randomization"), c("GroupComment", "VArm"))
\end{Sinput}
\end{Schunk}
\begin{Schunk}
\begin{Sinput}
vr[grepl("con", VArm), VArm := "traditional"]
vr[grepl("lar.*t$", VArm), VArm := "large"]
vr[grepl("Lar", VArm), VArm := "large grace"]
vr[grepl("pac", VArm), VArm := "cow"]
vr[, VArm := factor(VArm, levels = c("traditional", "large", "large grace", "cow"))]
vr <- vr[, .(groupid, VArm, GroupStatus, GroupComment)]
\end{Sinput}
\end{Schunk}
\begin{Schunk}
\begin{Sinput}
setkey(vr, groupid)
setkey(ar.1, groupid)
ar <- vr[ar.1]
# individual replacing members: GroupStatus: NA => accepted
ar[grepl("Rep", Mstatus), GroupStatus := "accepted"]
\end{Sinput}
\end{Schunk}
Tabulation of \textsf{AssignOriginal} against \textsf{VArm}. It shows complementarity so I can use one variable to fill in NAs in another.
\begin{Schunk}
\begin{Sinput}
setorder(ar, hhid, survey, IntDate, mid)
ar[, MemNum := 1:.N, by = .(hhid, survey, IntDate)]
table0(ar[MemNum == 1, .(AssignOriginal, VArm)])
\end{Sinput}
\begin{Soutput}
              VArm
AssignOriginal traditional large large grace  cow <NA>
   traditional        1244     0           0    0  650
   large                 0  1423           0    0  378
   large grace           0     0        1437    0  376
   cow                   0     0           0 1631  199
   <NA>                418   158          40   59    0
\end{Soutput}
\end{Schunk}
Tabulation of \textsf{RArm} after supplementing with \textsf{AssignOriginal} and \textsf{VArm}.
\begin{Schunk}
\begin{Sinput}
ar[is.na(RArm) & !is.na(AssignOriginal), RArm := AssignOriginal]
ar[is.na(RArm) & !is.na(VArm), RArm := VArm]
\end{Sinput}
\end{Schunk}
\begin{Schunk}
\begin{Sinput}
table0(ar[MemNum == 1, .(survey, RArm)])
\end{Sinput}
\begin{Soutput}
      RArm
survey traditional large large grace cow
     1         605   504         507 507
     2         585   485         447 466
     3         582   487         452 472
     4         540   483         447 444
\end{Soutput}
\end{Schunk}
Below is what is supplemented from \textsf{VArm} of village level information to the 220 NAs.
\begin{Schunk}
\begin{Sinput}
#table0(ar[MemNum == 1 & survey == 1 & is.na(AssignOriginal), RArm])
ar[MemNum == 1 & survey == 1 & is.na(AssignOriginal), 
  BorrowerStatus := "quit membership"]
table0(ar[MemNum == 1 & survey == 1 & is.na(AssignOriginal), 
  .(BorrowerStatus, RArm)])
\end{Sinput}
\begin{Soutput}
                 RArm
BorrowerStatus    traditional large large grace cow
  borrower                  0     0           0   0
  pure saver                0     0           0   0
  quit membership         120    40          40  20
\end{Soutput}
\end{Schunk}

\subsection{Merge admin-roster with other files}

\subsubsection{Choosing sample in admin-roster}

Tabulation of \textsf{RArm} when dropping \textsf{twice, double} in traditional arm.
\begin{Schunk}
\begin{Sinput}
tb <- table0(ar[MemNum == 1 & !grepl("tw|dou", TradGroup), .(survey, RArm)])
cbind(tb, total = apply(tb, 1, sum))
\end{Sinput}
\begin{Soutput}
  traditional large large grace cow total
1         441   504         507 507  1959
2         319   485         447 466  1717
3         316   487         452 472  1727
4         278   483         447 444  1652
\end{Soutput}
\end{Schunk}
Tabulation of \textsf{RArm} when dropping \textsf{twice} in traditional arm. This may make most sense but a large attrition between rd 1 and 2.
\begin{Schunk}
\begin{Sinput}
tb <- table0(ar[MemNum == 1 & !grepl("tw", TradGroup), .(survey, RArm)])
cbind(tb, total = apply(tb, 1, sum))
\end{Sinput}
\begin{Soutput}
  traditional large large grace cow total
1         505   504         507 507  2023
2         430   485         447 466  1828
3         426   487         452 472  1837
4         388   483         447 444  1762
\end{Soutput}
\end{Schunk}
Tabulation of \textsf{RArm} when dropping dirbursement after 2015-01-01. This has less attrition but includes heterogenous treatment among traditional.
\begin{Schunk}
\begin{Sinput}
tb <- table0(ar[MemNum == 1 & as.Date(DisDate1) < as.Date("2015-01-01") , .(survey, RArm)])
cbind(tb, total = apply(tb, 1, sum))
\end{Sinput}
\begin{Soutput}
  traditional large large grace cow total
1         328   385         359 328  1400
2         323   371         350 316  1360
3         323   372         349 318  1362
4         321   370         345 312  1348
\end{Soutput}
\begin{Sinput}
#table0(ar[MemNum == 1, .(Arm, RArm)])
#summary(ar[MemNum == 1 & grepl("rop", Arm), ])
#summary(ar[is.na(RArm), 1:10, with = F])
\end{Sinput}
\end{Schunk}
In \textsf{roster + admin} 1: Tabulate observations after keeping only observations used in estimation: Keep if \textsf{Mstatus} includes strings old, iRej, gEro, gRej, \& \textsf{DisDate1} is before 2015-01-01, \& \textsf{TradGroup} does not include strings tw.
\begin{Schunk}
\begin{Sinput}
tb <- table0(ar[grepl("old|iRej|^g", Mstatus) 
  & as.Date(DisDate1) < as.Date("2015-01-01") & !grepl("tw", TradGroup) 
  & MemNum == 1, .(survey, RArm)])
cbind(tb, total = apply(tb, 1, sum))
\end{Sinput}
\begin{Soutput}
  traditional large large grace cow total
1         170   296         278 248   992
2         137   285         270 240   932
3         137   286         270 239   932
4         136   284         266 235   921
\end{Soutput}
\end{Schunk}
In \textsf{roster + admin} 2: Keep if \textsf{Mstatus} includes strings old, iRej, gEro, gRej, \& \textsf{TradGroup} does not include strings tw (relaxing  \textsf{DisDate1} is before 2015-01-01). \textcolor{blue}{This the data used in this note.} This also shows a lower attrition rate for \textsf{large} arm. \gobblepars
\begin{Schunk}
\begin{Sinput}
cbind(table0(ar[grepl("old|iRej|^g", Mstatus) & 
  !grepl("tw", TradGroup) & MemNum == 1, .(survey, RArm)]), 
  total = apply(table0(ar[grepl("old|iRej|^g", Mstatus) & 
    !grepl("tw", TradGroup) & MemNum == 1, .(survey, RArm)]), 1, sum))
\end{Sinput}
\begin{Soutput}
  traditional large large grace cow total
1         400   400         400 400  1600
2         327   384         342 366  1419
3         324   386         348 366  1424
4         287   382         343 342  1354
\end{Soutput}
\begin{Sinput}
ar[, o1600 := 0L]
ar[grepl("old|iRej|^g", Mstatus) & !grepl("tw|dou", TradGroup), 
  o1600 := 1L]
\end{Sinput}
\end{Schunk}
Create \textsf{o1600} to indicate the original 1600 HHs.
\begin{Schunk}
\begin{Sinput}
# tabulation of total by o1600 and round
table0(ar[, .(o1600, survey)])
\end{Sinput}
\begin{Soutput}
     survey
o1600    1    2    3    4
    0 2101 2510 2543 2457
    1 6532 5817 5843 5420
\end{Soutput}
\begin{Sinput}
# tabulation of 1 obs per HH by o1600 and round. o1600 == 0 is added HHs through newGroup and iReplacement.
table0(ar[MemNum==1, .(o1600, survey)])
\end{Sinput}
\begin{Soutput}
     survey
o1600    1    2    3    4
    0  523  611  616  607
    1 1600 1372 1377 1307
\end{Soutput}
\end{Schunk}
\begin{Schunk}
\begin{Sinput}
ar[, c("BeforeJan2015", "Year2015", "Year2016", "AfterJan2017") :=
   .(as.Date(DisDate1) < as.Date("2015-01-01"),
     as.Date(DisDate1) >= as.Date("2015-01-01") & 
       as.Date(DisDate1) < as.Date("2016-01-01"),
     as.Date(DisDate1) >= as.Date("2016-01-01") & 
       as.Date(DisDate1) < as.Date("2017-01-01"),
    as.Date(DisDate1) >= as.Date("2017-01-01"))]
ar[, FirstDisPeriod := as.character(NA)]
ar[as.Date(DisDate1) < as.Date("2015-01-01"), 
  FirstDisPeriod := "BeforeJan2015"]
ar[as.Date(DisDate1) >= as.Date("2015-01-01") & 
    as.Date(DisDate1) < as.Date("2016-01-01"), 
  FirstDisPeriod := "Year2015"]
ar[as.Date(DisDate1) >= as.Date("2016-01-01") & 
    as.Date(DisDate1) < as.Date("2017-01-01"), 
  FirstDisPeriod := "Year2016"]
ar[as.Date(DisDate1) >= as.Date("2017-01-01"), 
  FirstDisPeriod := "After2017"]
\end{Sinput}
\end{Schunk}
Tabulate observations without disbursement date info. Note: \textsf{iReplacement} are borrower in \textsf{BorrowerStatus}. (Did they remain as a member?)
\begin{Schunk}
\begin{Sinput}
summary(ar[is.na(FirstDisPeriod) & survey == 1 & MemNum == 1, 
  .(survey, DisDate1, creditstatus, Mstatus, BorrowerStatus)] )
\end{Sinput}
\begin{Soutput}
     survey     DisDate1            creditstatus         Mstatus   
 Min.   :1   Min.   :NA    Yes            :  0   gErosion    : 80  
 1st Qu.:1   1st Qu.:NA    No             :428   gRejection  :140  
 Median :1   Median :NA    Replaced Member:  0   iRejection  :159  
 Mean   :1   Mean   :NA                          iReplacement:  3  
 3rd Qu.:1   3rd Qu.:NA                          newGroup    : 20  
 Max.   :1   Max.   :NA                          oldMember   : 26  
             NA's   :428                                           
         BorrowerStatus
 borrower       :159   
 pure saver     : 49   
 quit membership:220   
                       
                       
                       
                       
\end{Soutput}
\end{Schunk}
These are people who rejected loans. Add \textsf{RejectedLoans} to \textsf{FirstDisPeriod}. \gobblepars
\begin{Schunk}
\begin{Sinput}
ar[is.na(FirstDisPeriod), FirstDisPeriod := "RejectedLoans"]
ar[, FirstDisPeriod := factor(FirstDisPeriod, levels = 
  c("BeforeJan2015", "Year2015", "Year2016", "AfterJan2017", "RejectedLoans"))]
summary(ar[survey == 1 & MemNum == 1, 
  .(survey, DisDate1, FirstDisPeriod, creditstatus, Mstatus, BorrowerStatus)] )
\end{Sinput}
\begin{Soutput}
     survey     DisDate1                         FirstDisPeriod
 Min.   :1   Min.   :2013-05-01 00:00:00   BeforeJan2015:1400  
 1st Qu.:1   1st Qu.:2013-07-01 00:00:00   Year2015     : 295  
 Median :1   Median :2013-11-01 00:00:00   Year2016     :   0  
 Mean   :1   Mean   :2014-03-23 17:07:57   AfterJan2017 :   0  
 3rd Qu.:1   3rd Qu.:2014-12-01 00:00:00   RejectedLoans: 428  
 Max.   :1   Max.   :2015-12-01 00:00:00                       
             NA's   :428                                       
          creditstatus          Mstatus             BorrowerStatus
 Yes            :1695   gErosion    :  80   borrower       :1854  
 No             : 428   gRejection  : 140   pure saver     :  49  
 Replaced Member:   0   iRejection  : 160   quit membership: 220  
                        iReplacement: 115                         
                        newGroup    : 408                         
                        oldMember   :1220                         
                                                                  
\end{Soutput}
\end{Schunk}
%This tabulation of \textsf{survey} vs. \textsf{Arm} shows addition from \textsf{roster+admin} 1 is mostly in round 1 for \textsf{traditional} but in all rds for other arms. \textsf{FirstDisPeriod} gives the period of first disbursement, and all credit reeceivers received loans by the end of 2015.
\begin{Schunk}
\begin{Sinput}
table0(ar[is.na(FirstDisPeriod) & MemNum == 1 & survey == 1, .(DisDate1, creditstatus)])
\end{Sinput}
\end{Schunk}
Breakdown of first disbursement by \textsf{RArm} at rd 1 in \textsf{roster + admin} 2.
\begin{Schunk}
\begin{Sinput}
tb <- table0(ar[grepl("old|iRej|^g", Mstatus) 
  & !grepl("tw|dou", TradGroup) & survey == 1 & MemNum == 1,
  .(FirstDisPeriod, RArm)])
rbind(tb, total = apply(tb, 2, sum))
\end{Sinput}
\begin{Soutput}
              traditional large large grace cow
BeforeJan2015         170   296         278 248
Year2015               31    52          60  60
Year2016                0     0           0   0
AfterJan2017            0     0           0   0
RejectedLoans         199    52          62  92
total                 400   400         400 400
\end{Soutput}
\end{Schunk}
%Same tabulation if no conditioning on \textsf{Mstatus} or  \textsf{TradGroup}.
\begin{Schunk}
\begin{Sinput}
table0(ar[survey == 1 & MemNum == 1,  .(FirstDisPeriod, RArm)])
\end{Sinput}
\end{Schunk}
Tabulation of membership status against \textsf{GroupStatus} from \textsf{"RCT\_village.dta"}.
\begin{Schunk}
\begin{Sinput}
ar[grepl("new", Mstatus), GroupStatus := "accepted"]
table0(ar[MemNum == 1, .(Mstatus, GroupStatus)])
\end{Sinput}
\begin{Soutput}
              GroupStatus
Mstatus        accepted erosion group rejection
  gErosion            0     189               0
  gRejection          0       0             372
  iRejection        543       0             114
  iReplacement      445       0               0
  newGroup         1603       0               0
  oldMember        4747       0               0
\end{Soutput}
\end{Schunk}
There are 114 cases of group rejections in \textsf{GroupStatus} classified as individual rejections in \textsf{Mstatus}. Overwrite \textsf{Mstatus} with \textsf{GroupStatus} in these cases.
\begin{Schunk}
\begin{Sinput}
ar[grepl("iR", Mstatus) & grepl("rej", GroupStatus), Mstatus := "gRejection"]
table0(ar[MemNum == 1, .(Mstatus, GroupStatus)])
\end{Sinput}
\begin{Soutput}
              GroupStatus
Mstatus        accepted erosion group rejection
  gErosion            0     189               0
  gRejection          0       0             486
  iRejection        543       0               0
  iReplacement      445       0               0
  newGroup         1603       0               0
  oldMember        4747       0               0
\end{Soutput}
\begin{Sinput}
tb <- table0(ar[MemNum == 1, .(GroupStatus, RArm)])
cbind(tb, total = apply(tb, 1, sum))
\end{Sinput}
\begin{Soutput}
                traditional large large grace  cow total
accepted               1894  1801        1813 1830  7338
erosion                 110     0          20   59   189
group rejection         308   158          20    0   486
\end{Soutput}
\end{Schunk}
As one can see below, \textsf{gRejection} is more frequent in \textsf{traditional} and \textsf{large}, while there is none in \textsf{cow}. \textsf{traditional, cow} have more frequent \textsf{iRejection}. So \textsf{traditional} was disliked both at group and individual levels, \textsf{large} was disliked as a group, \textsf{cow} was disliked at an individual level, and \textsf{large grace} were well received at both group and individual levels. This indicates attractiveness of a grace period at least at the group level, and a large cash form (over small cash or in-kind) at the individual level.
\begin{Schunk}
\begin{Sinput}
tb <- table0(ar[MemNum == 1 & survey == 1, .(Mstatus, RArm)])
tb
\end{Sinput}
\begin{Soutput}
              RArm
Mstatus        traditional large large grace cow
  gErosion              40     0          20  20
  gRejection            80    40          20   0
  iRejection            54    12          22  72
  iReplacement          39     8          11  57
  newGroup             166    96          96  50
  oldMember            226   348         338 308
\end{Soutput}
\end{Schunk}
\begin{Schunk}
\begin{Sinput}
round((tb/apply(tb, 2, sum))*1, 2)
\end{Sinput}
\begin{Soutput}
              RArm
Mstatus        traditional large large grace  cow
  gErosion            0.07  0.00        0.03 0.04
  gRejection          0.16  0.08        0.04 0.00
  iRejection          0.11  0.02        0.04 0.12
  iReplacement        0.08  0.02        0.02 0.11
  newGroup            0.27  0.19        0.16 0.10
  oldMember           0.45  0.69        0.67 0.61
\end{Soutput}
\end{Schunk}
\begin{Schunk}
\begin{Sinput}
ar[, RArm := RArm[!is.na(RArm)][1], by = groupid]
\end{Sinput}
\end{Schunk}
Create time-invariant \textsf{HHinfo} from \textsf{ar}. \gobblepars
\begin{Schunk}
\begin{Sinput}
HHinfo <- ar[, c("hhid", "IntDate", "Mstatus", "BorrowerStatus", "Mgroup",
  adbasevars), with = F]
for (i in c("IntDate", "Mstatus", "BorrowerStatus", "Mgroup",
  adbasevars))
  HHinfo[, (i) := eval(parse(text=
      paste0(i, "[!is.na(", i, ")][1]")
    )), by = hhid]
HHinfo <- HHinfo[!duplicated(HHinfo[, c("hhid", 
  "RArm", "Arm", "randomArm", 
  "Mstatus", "BorrowerStatus", "Mgroup", "creditstatus", "Mem", "povertystatus")]), ]
\end{Sinput}
\end{Schunk}
Create roster member total \textsf{RosterMemTotal}. 
\begin{Schunk}
\begin{Sinput}
ar[, RosterMemTotal := .N, by = .(hhid, survey, IntDate)]
# HH member orders
table0(ar[, .(MemNum, survey)])
\end{Sinput}
\begin{Soutput}
      survey
MemNum    1    2    3    4
    1  2123 1983 1993 1914
    2  2061 1930 1930 1841
    3  1874 1781 1782 1691
    4  1414 1408 1415 1324
    5   744  778  803  734
    6   290  302  311  260
    7    88   96   98   79
    8    32   38   40   29
    9     6   10   10    4
    10    1    1    3    1
    11    0    0    1    0
\end{Soutput}
\begin{Sinput}
# HH size distribution
table0(ar[MemNum == RosterMemTotal, .(MemNum, survey)])
\end{Sinput}
\begin{Soutput}
      survey
MemNum   1   2   3   4
    1   62  53  63  73
    2  187 149 148 150
    3  460 373 367 367
    4  670 630 612 590
    5  454 476 492 474
    6  202 206 213 181
    7   56  58  58  50
    8   26  28  30  25
    9    5   9   7   3
    10   1   1   2   1
    11   0   0   1   0
\end{Soutput}
\begin{Sinput}
# single member HHs
ar[hhid %in% hhid[RosterMemTotal == 1], 
  .(hhid, mid, survey, IntDate, sex, Age_1, marital, HeadAge)]
\end{Sinput}
\begin{Soutput}
            hhid mid survey    IntDate    sex Age_1  marital HeadAge
  1:     7020405   1      1 2012-10-07 Female    55  widowed      55
  2:     7020405   1      2 2014-10-14 Female    55        3      55
  3:     7020405   1      3 2015-12-31 Female    55        3      55
  4:     7020405   1      4 2017-04-26 Female    55  widowed      55
  5:     7020413   1      1 2012-10-10 Female    55  widowed      55
 ---                                                                
485: 99081912103   1      4 2017-03-30 Female    20  married      20
486: 99081912103   2      4 2017-03-30   Male    24  married      20
487: 99081912406   1      1 2013-09-08 Female    50 divorced      50
488: 99081912406   1      3 2016-01-11 Female    50        3      50
489: 99081912406   1      4 2017-04-05 Female    50  widowed      50
\end{Soutput}
\end{Schunk}
Save roster-admin data.
\begin{Schunk}
\begin{Sinput}
saveRDS(ar, paste0(pathsaveOriginal1600, "RosterAdminData.rds"))
fwrite(ar, paste0(pathsaveOriginal1600, "RosterAdminData.prn"), sep = "\t", quote = F)
\end{Sinput}
\end{Schunk}


Schooling. \gobblepars
\begin{Schunk}
\begin{Sinput}
sch1[, Spattern := paste(as.character(.SD[, Enrolled]), collapse = ""), 
  by = .(hhid, mid), .SDcols = "Enrolled"]
sch1[, EnrollPattern := Spattern]
sch1[, en := 1:.N, by = .(hhid, mid)]
sch1[, Attrit := paste(as.character(.SD[, en]), collapse = ""), 
  by = .(hhid, mid), .SDcols = "en"]
sch1[Attrit == "123", c("Spattern", "EnrollPattern") := 
  .(paste0(Spattern, "n"), paste0(Spattern, "0"))]
sch1[Attrit == "124", c("Spattern", "EnrollPattern") := 
  .(paste0(substr(Spattern, 1, 2), "n", substr(Spattern, 1, 3)), 
    paste0(substr(Spattern, 1, 2), "0", substr(Spattern, 1, 3)))]
sch1[Attrit == "12", c("Spattern", "EnrollPattern") := 
  .(paste0(Spattern, "nn"), paste0(Spattern, "00"))]
sch1[Attrit == "13", c("Spattern", "EnrollPattern") := 
  .(paste0(substr(Spattern, 1, 1), "n", substr(Spattern, 2, 2), "n"), 
    paste0(substr(Spattern, 1, 1), "0", substr(Spattern, 2, 2), "0"))]
sch1[Attrit == "14", c("Spattern", "EnrollPattern") := 
  .(paste0(substr(Spattern, 1, 3), "n"), 
    paste0(substr(Spattern, 1, 3), "0"))]
sch1[Attrit == "23", c("Spattern", "EnrollPattern") := 
  .(paste0("n", Spattern, "n"), 
    paste0("0", Spattern, "0"))]
sch1[Attrit == "24", c("Spattern", "EnrollPattern") := 
  .(paste0("n", substr(Spattern, 1, 1), "n", substr(Spattern, 2, 2)), 
    paste0("0", substr(Spattern, 1, 1), "0", substr(Spattern, 2, 2)))]
sch1[Attrit == "1", c("Spattern", "EnrollPattern") := 
  .(paste0(Spattern, "nnn"), 
    paste0(Spattern, "000"))]
sch1[, Attrit := factor(Attrit)]
sch1[, Spattern := factor(Spattern)]
sch1[, EnrollPattern := factor(EnrollPattern)]
\end{Sinput}
\end{Schunk}
Schooling pattern in sch1.
\begin{Schunk}
\begin{Sinput}
table(sch1[, .(Spattern)])
\end{Sinput}
\begin{Soutput}

0000 0001 000n 0011 001n 00nn 0100 0101 010n 0111 011n 01nn 0nnn 1000 1001 100n 
 208   36  216  152   33  192   16    4    9  840  105   70  316   64    8   45 
1011 101n 10nn 1100 1101 110n 1110 1111 111n 11nn 1nnn 
  56   24   86   48   16   84   28 5172  654  326  199 
\end{Soutput}
\end{Schunk}


\begin{Schunk}
\begin{Sinput}
setkey(ar, groupid, hhid, mid, sex, AgeComputed, year)
setkey(sch1, groupid, hhid, mid, sex, AgeComputed, year)
s1 <- ar[sch1]
s1[, Tee := 1:.N, by = .(hhid, mid)]
s1[, Year := format(as.yearmon(IntDate), "%Y")]
\end{Sinput}
\end{Schunk}
In \textsf{sch1}: Number of unique \textsf{hhid}s by \textsf{year} (original entry) or \textsf{Year} (extracted from \textsf{IntDate}).
\begin{Schunk}
\begin{Sinput}
s1[, SVYHH := 1:.N, by = .(hhid, survey)]
table0(s1[SVYHH == 1, .(year, Year)])
\end{Sinput}
\begin{Soutput}
      Year
year   2011 2012 2013 2014 2015 2016 2017 <NA>
  2012    4 1069    1    0    0    0    0  168
  2013    0    0  359    0    0    0    0  100
  2014    0    0    0 1251    0    0    0    0
  2015    0    0    0    1  849  358    0    2
  2017    0    0    0    0    0    1 1118    8
\end{Soutput}
\end{Schunk}
In \textsf{sch1}: Number of observations tabulated by \textsf{year} (original entry) and round (\textsf{survey}).
\begin{Schunk}
\begin{Sinput}
table(s1[, .(year, survey)])
\end{Sinput}
\begin{Soutput}
      survey
year      1    2    3    4
  2012 1931    0    0    0
  2013  651    0    0    0
  2014    0 2059    0    0
  2015    0    0 1911    0
  2017    0    0    0 1696
\end{Soutput}
\end{Schunk}
In \textsf{sch1}: RoundOrder is 1 if individual is observed for the first time in data, 2 if for the second time, ...
\begin{Schunk}
\begin{Sinput}
table(s1[, .(year, RoundOrder = Tee)])
\end{Sinput}
\begin{Soutput}
      RoundOrder
year      1    2    3    4
  2012 2098    0    0    0
  2013  806    0    0    0
  2014    0 2282    0    0
  2015    0   79 1945    0
  2017    0   28  107 1662
\end{Soutput}
\end{Schunk}
In \textsf{sch1}: Number of observations tabulated by \textsf{year} (original entry) and age (\textsf{AgeComputed}).
\begin{Schunk}
\begin{Sinput}
table(s1[, .(year, AgeComputed)])
\end{Sinput}
\begin{Soutput}
      AgeComputed
year     6   7   8   9  10  11  12  13  14  15  16  17  18  19  20  21  22  23
  2012 168 264 279 114 333  77 237 109 104 173 103  43  94   0   0   0   0   0
  2013  48  93  90  61 118  60  79  55  46  58  46  14  38   0   0   0   0   0
  2014   0  43 222 317 298 211 346 131 234 121 124 152  62  15   6   0   0   0
  2015   0   0  42 225 311 291 198 302 118 192 100  93  95  38  11   8   0   0
  2017   0   0   0   0  40 218 289 279 186 272 110 171  90  64  51  22   4   1
\end{Soutput}
\end{Schunk}

\subsubsection{Attach variables from admin-roster to other files}


Attach \textsf{RArm, Arm, TradGroup, Mem, ObPattern, AttritIn, o1600, Mstatus, BorrowerStatus, creditstatus, povertystatus, RMvalue.repay, RMvalue.NetSaving, RMOtherNetSaving, RMOtherRepaid, HHsize, HeadLiteracy, IntDate, DisDate1} from \textsf{ar}.
\begin{Schunk}
\begin{Sinput}
ar <- readRDS(paste0(pathsaveOriginal1600, "RosterAdminData.rds"))
ar0 <- ar[, c("groupid", "hhid", "survey", vartoattach), with= F]
for (i in 2:4) {
  ar0[, paste0("Time.", i) := 0L ]
  ar0[grepl(i, survey), paste0("Time.", i) := 1L ]
}
ar0[, num := 1:.N, by = .(hhid, survey)]
ar0 <- ar0[num == 1, ]
ar0[, num := NULL]
ar0[, Year :=  as.numeric(format(as.Date(IntDate), "%Y"))]
ar0[, Month := as.character(format(as.Date(IntDate), "%B"))]
setkey(ar0, groupid, hhid, survey, Year, Month)
setkey(ar0, groupid, hhid, survey)
\end{Sinput}
\end{Schunk}
\begin{Schunk}
\begin{Sinput}
vartoattach <- c("RArm", "Arm", "TradGroup", "Mem", 
  "ObPattern", "AttritIn", "o1600", "Mstatus", "BorrowerStatus", 
  "creditstatus", "povertystatus", "RMvalue.repay", 
  "RMvalue.NetSaving", "RMOtherNetSaving", "RMOtherRepaid",
  "HHsize", "HeadLiteracy", "IntDate", "DisDate1")
dfiles <- c("ass", "s1", "lvo", "lab", "far", "con", "shk")
for (j in 1:length(dfiles)) {
  dd <- get(dfiles[j])
  dd[, groupid := as.integer(as.numeric(as.character(gid)))]
  dd[, gid := NULL]
  dd[, Year :=  as.numeric(format(as.Date(IntDate), "%Y"))]
  dd[, Month := as.character(format(as.Date(IntDate), "%B"))]
  dd[Year <= 2010, Year := Year + 10]
  # drop all variables in each page before copying from ar0
  dd[, (vartoattach) := NULL]
  setorder(dd, groupid, hhid, survey, Year, Month)
  setkey(dd, groupid, hhid, survey)
  if (j < length(dfiles)) dd <- ar0[dd]
  assign(dfiles[j], dd)
}
\end{Sinput}
\end{Schunk}
Create \textsf{Arm*HadCows}, \textsf{Arm*HadCows*Time} interactions in \textsf{lvo}. \gobblepars
\begin{Schunk}
\begin{Sinput}
lvo[, .Arm := paste0(toupper(substr(RArm, 1, 1)), substr(RArm, 2, 100))]
lvo[, .Arm := gsub(" g", "G", .Arm)]
lvo[grepl("NA", .Arm), .Arm := NA]
lvo[, .Arm := factor(.Arm, levels = c("Traditional", "Large", "LargeGrace", "Cow", NA))]
lvo[, 
  c(paste0("dummyHadCows.Time", 3:4), 
     paste0("dummy", levels(lvo[, .Arm]), ".dummyHadCows"),
     paste0("dummy", levels(lvo[, .Arm]), ".dummyHadCows.Time3"), 
     paste0("dummy", levels(lvo[, .Arm]), ".dummyHadCows.Time4")) := as.integer(0L)]
set(lvo, i = which(lvo[["dummyHadCows"]] == 1L & lvo[["Time.3"]] == 1L), 
  j = grep("^dummyHadCows.*3", colnames(lvo)), value = 1L)
set(lvo, i = which(lvo[["dummyHadCows"]] == 1L & lvo[["Time.4"]] == 1L), 
  j = grep("^dummyHadCows.*4", colnames(lvo)), value = 1L)
for (a in levels(lvo[, .Arm])) {
  # dummyHadCows*Arm
  set(lvo, i = which(lvo[["dummyHadCows"]] == 1L & grepl(a, lvo[[".Arm"]])), 
    j = grep(paste0(a, ".dummyHadCows$"), colnames(lvo)), 
    value = 1L)
  # dummyHadCows*Arm*Time3
  set(lvo, i = which(lvo[["dummyHadCows"]] == 1L & grepl(a, lvo[[".Arm"]]) & lvo[["Time.3"]] == 1L), 
    j = grep(paste0(a, ".dummyHadCows.*3"), colnames(lvo)), 
    value = 1L)
  # dummyHadCows*Arm*Time4
  set(lvo, i = which(lvo[["dummyHadCows"]] == 1L & grepl(a, lvo[[".Arm"]]) & lvo[["Time.4"]] == 1L), 
    j = grep(paste0(a, ".dummyHadCows.*4"), colnames(lvo)), 
    value = 1L)
}
#grepout("Had", colnames(lvo))
\end{Sinput}
\end{Schunk}
Check number of HHs in assets by \textsf{o1600}:
\begin{Schunk}
\begin{Sinput}
table(ass[, .(creditstatus, survey, o1600)])
\end{Sinput}
\begin{Soutput}
, , o1600 = 0

                 survey
creditstatus         1    2    3    4
  Yes              478  588  593  586
  No                23   23   23   21
  Replaced Member    0    0    0    0

, , o1600 = 1

                 survey
creditstatus         1    2    3    4
  Yes             1192 1047 1054 1039
  No               403  323  323  268
  Replaced Member    0    0    0    0
\end{Soutput}
\begin{Sinput}
#table0(ass[o1600 == 0L, .(creditstatus, survey)])
\end{Sinput}
\end{Schunk}
Save all data.
\begin{Schunk}
\begin{Sinput}
saveRDS(s1, paste0(pathsaveOriginal1600, "RosterAdminSchoolingData.rds"))
saveRDS(ass, paste0(pathsaveOriginal1600, "AssetAdminData.rds"))
saveRDS(lvo, paste0(pathsaveOriginal1600, "LivestockAdminData.rds"))
saveRDS(lab, paste0(pathsaveOriginal1600, "LabourIncomeAdminData.rds"))
saveRDS(far, paste0(pathsaveOriginal1600, "FarmRevenueAdminData.rds"))
saveRDS(con, paste0(pathsaveOriginal1600, "ConsumptionAdminData.rds"))
saveRDS(shk, paste0(pathsaveOriginal1600, "Shocks.rds"))
\end{Sinput}
\end{Schunk}
\begin{Schunk}
\begin{Sinput}
fwrite(s1, paste0(pathsaveOriginal1600, "RosterAdminSchoolingData.prn"), sep = "\t", quote = F)
fwrite(ass, paste0(pathsaveOriginal1600, "AssetAdminData.prn"), sep = "\t", quote = F)
fwrite(lvo, paste0(pathsaveOriginal1600, "LivestockAdminData.prn"), sep = "\t", quote = F)
fwrite(lab, paste0(pathsaveOriginal1600, "LabourIncomeAdminData.prn"), sep = "\t", quote = F)
fwrite(far, paste0(pathsaveOriginal1600, "FarmRevenueAdminData.prn"), sep = "\t", quote = F)
fwrite(con, paste0(pathsaveOriginal1600, "ConsumptionAdminData.prn"), sep = "\t", quote = F)
fwrite(shk, paste0(pathsaveOriginal1600, "Shocks.prn"), sep = "\t", quote = F)
\end{Sinput}
\end{Schunk}
\begin{Schunk}
\begin{Sinput}
flnames <- c("Roster", "Asset", "Livestock", "LabourIncome", "FarmRevenue", "Consumption", "Shocks")
\end{Sinput}
\end{Schunk}



Further data preparations (trimming, adding shocks, round numbering, creating dummy vectors, interaction terms) for estimation. Produces files: \textsf{\footnotesize RosterAdminDataUsedForEstimation.prn, AssetAdminDataUsedForEstimation.prn, LivestockAdminDataUsedForEstimation.prn, LabourIncomeAdminDataUsedForEstimation.prn, FarmRevenueAdminDataUsedForEstimation.prn, ConsumptionAdminDataUsedForEstimation.prn, ShocksAdminDataUsedForEstimation.prn}.
\begin{Schunk}
\begin{Sinput}
# Name it as sch1, sch2 rather than s1, s2 (as in other files) to display "s1" in TrimmingNumObsTable
# Following files are created in ImpactEstimatin_body1.rnw using paste0(path1234, "data_read_in_a_list_with_baseline_patched.rds")
sch1 <- readRDS(paste0(pathsaveOriginal1600, "RosterAdminSchoolingData.rds"))
ar <- readRDS(paste0(pathsaveOriginal1600, "RosterAdminData.rds"))
ass <- readRDS(paste0(pathsaveOriginal1600, "AssetAdminData.rds"))
lvo <- readRDS(paste0(pathsaveOriginal1600, "LivestockAdminData.rds"))
lab <- readRDS(paste0(pathsaveOriginal1600, "LabourIncomeAdminData.rds"))
far <- readRDS(paste0(pathsaveOriginal1600, "FarmRevenueAdminData.rds"))
con <- readRDS(paste0(pathsaveOriginal1600, "ConsumptionAdminData.rds"))
shk <- readRDS(paste0(pathsaveOriginal1600, "Shocks.rds"))
flnames <- c("RosterSchooling", "Roster", "Asset", 
  "Livestock", "LabourIncome", "FarmRevenue", "Consumption")
dfiles <- c("sch1", "ar", "ass", "lvo", "lab", "far", "con")
interterms <- c("Time.2", "Time.3", "Time.4")
Arms <- c("Traditional", "Large", "LargeGrace", "Cow")
povertystatus <- c("UltraPoor", "ModeratelyPoor")
Obs <- NULL
shk <- shk[survey == 1, ]
shk[, grepout("gid|Dat|Ye|Mo|surv|code", colnames(shk)) := NULL]
setkey(shk, groupid, hhid)
if (ThisIsNoFlood) 
  pathsaveHere <- pathsaveNoFlood else 
  pathsaveHere <- pathsaveOriginal1600
# shk[, Month := factor(Month, levels = 
#   c("January", "February", "March", "April", 
#   #"May", "June", "July",
#   "August", "September", "November", "October",   "December"))]
dimchange <- dimchangeRd1 <- NULL
for (j in 1:length(dfiles)) {
#  if (j == 1) print0(paste("old|iRej|^g in Mstatus", "==>", "con|^dro|^rep in Mgroup", "==>", "No tw|dou in TradGroup"))
  dd <- get(dfiles[j])
  if (!any(grepl("^tee$", colnames(dd)))) dd[, tee := 1:.N, by = hhid]
  # show trimming results
  dimchange <- rbind(dimchange, paste(dfiles[j], ":", nrow(dd), 
    "==>", 
    nrow(dd[grepl("old|iRej|^g", Mstatus), ]),
#    "==>", 
#    nrow(dd[grepl("old|iRej|^g", Mstatus), ][grepl("con|dro", Mgroup), ]), 
    "==>", 
    nrow(dd[grepl("old|iRej|^g", Mstatus), ][!grepl("tw|dou", TradGroup), ])
    ))
  dimchangeRd1 <- rbind(dimchangeRd1, paste(dfiles[j], ":", nrow(dd[tee == 1, ]), 
    "==>", 
    nrow(dd[tee == 1 & grepl("old|iRej|^g", Mstatus), ]),
#    "==>", 
#    nrow(dd[grepl("old|iRej|^g", Mstatus), ][grepl("con|dro", Mgroup), ]), 
    "==>", 
    nrow(dd[tee == 1 & grepl("old|iRej|^g", Mstatus), ][!grepl("tw|dou", TradGroup), ])
    ))
}
dmch <- gsub("==>", " & $\\\\Rightarrow$ &", dimchange)
dmch <- gsub(":", " & ", dmch)
#dmch <- rbind("file & & old$|$iRej$|$\\^{}g in \\textsf{Mstatus} && con$|$dro in \\textsf{Mgroup} && No tw$|$dou in \\textsf{TradGroup} &", dmch)
dmch <- rbind("file & & old$|$iRej$|$\\^{}g in \\textsf{Mstatus} && No tw$|$dou in \\textsf{TradGroup} &", 
  "\\makebox[1.5cm]{\\footnotesize all rounds}&&&&&",
  dmch)
dmch <- gsub("$", " \\\\\\\\ ", dmch)

dmchRd1 <- gsub("==>", " & $\\\\Rightarrow$ &", dimchangeRd1)
dmchRd1 <- gsub(":", " & ", dmchRd1)
dmchRd1 <- rbind("\\makebox[1.5cm]{\\footnotesize round 1 only}&&&&&",
  dmchRd1)
#dmchRd1 <- rbind("file & & old$|$iRej$|$\\^{}g in \\textsf{Mstatus} && No tw$|$dou in \\textsf{TradGroup} &", dmchRd1)
dmchRd1 <- gsub("$", " \\\\\\\\ ", dmchRd1)

hleft = c("\\sf", c(rbind(rep("\\hfill", 2), rep("\\hfil", 2)), "\\hfill"))
hcenter = c(1.5, c(rbind(rep(1, 2), rep(1.5, 2)), 1))
write.tablev(
  rbind(paste("\\begin{tabular}{", 
    paste(paste0(">{\\footnotesize ", hleft, "}", "p{", hcenter, "cm}", "<{}"), collapse = ""), "}\\rowcolor{paleblue}"),
  dmch, 
  dmchRd1, 
  "\\end{tabular}"), 
  paste0(pathsaveHere, "TrimmingNumObsTable.tex"), colnamestrue = F)
#print0(rbind(paste("(old|iRej|^g in Mstatus)", "==>", "(con|^dro|^rep in Mgroup)", "==>", "(No tw|dou in TradGroup)"), dimchange))
for (j in 1:length(dfiles)) {
  dd <- get(dfiles[j])
  setkey(dd, hhid, Year, Month)
  if (!any(grepl("^tee$", colnames(dd)))) dd[, tee := 1:.N, by = hhid]
  dd[, Arm := droplevels(Arm)]
  dd[, Year :=  as.integer(strftime(IntDate, format = "%Y"))]
  # 1. Keep only membership = 1 or 4, which corresponds to 
  # Mstatus old, iRej, gR, gE
  dd <- dd[grepl("old|iRej|^g", Mstatus), ]
  # 2. Keep only continuing, dropouts members in Mgroup.
  #dd <- dd[grepl("con|dro", Mgroup), ]
  # Rejecters do not receive loans. So I need to relax creditstatus = yes condition.
  # Remark out the following:
  # dd <- dd[grepl("Yes", creditstatus), ]
  #  dd <- dd[as.Date(DisDate1) < as.Date("2015-01-01"), ]
  dd <- dd[!grepl("tw|dou", TradGroup), ]
#grepl("es", creditstatus) & as.Date(DisDate1) <= as.Date("2015-01-01") & !grepl("tw|dou", TradGroup)
  setkey(dd, groupid, hhid)
  # merge shock module
  dd <- shk[dd]
  dd[, c("en") := NULL]
  dd[, teeyr := 1]
  dd[Year == 2014, teeyr := 2]
  dd[Year == 2015, teeyr := 3]
  dd[Year == 2016, teeyr := 3]
  dd[Year == 2017, teeyr := 4]
  dd[, Time := teeyr]
  setkey(dd, hhid, Year, teeyr)
  # Replace Arm with RArm
  dd[, ArmUsedPreviously := Arm]; dd[, Arm := RArm]
  dd <- data.table(dd, 
    makeDummyFromFactor(dd[, Arm], reference = NULL))
  if (any(grepl("dummyLarge grace", colnames(dd)))) 
    setnames(dd, grepout("dummyLarge grace", colnames(dd)), 
      gsub("dummyLarge g", "dummyLargeG", 
        grepout("dummyLarge grace", colnames(dd))))
  if (any(grepl("dummyNANA", colnames(dd)))) 
    dd[, dummyNANA := NULL]
  #dd[, dummyDropOuts := NULL]
  dd <- data.table(dd, 
    makeDummyFromFactor(dd[, povertystatus], reference = NULL))
  setnames(dd, c("dummyUltra Poor", "dummyModerate Poor"),
    c("dummyUltraPoor", "dummyModeratelyPoor"))
  dd[, c("Size", "Grace", "InKind") := .("SmallSize", "WithoutGrace", "Cash")]
  dd[!grepl("tra", Arm), Size := "LargeSize"]
  dd[grepl("gr|cow", Arm), Grace := "WithGrace"]
  dd[grepl("cow", Arm), InKind := "NonCash"]
  dd[, c("Grace", "Size", "InKind") := 
    .(factor(Grace), factor(Size, levels = c("LargeSize", "SmallSize")), 
      factor(InKind))]
  dd <- data.table(dd, 
    makeDummyFromFactor(dd[, Size], reference = NULL),
    makeDummyFromFactor(dd[, Grace], reference = NULL),
    makeDummyFromFactor(dd[, InKind], reference = NULL))
  # create demeaned dummies
  tobe.interacted <- c(Arms, povertystatus, 
    c("SmallSize", "LargeSize", "WithoutGrace", "WithGrace", "Cash", "NonCash"))
  for (k in tobe.interacted)
    dd[, paste0("DemeanedDummy", k) := 
      eval(parse(text = 
        paste0("dummy", k)
      )) - 
      mean(
        eval(parse(text = 
          paste0("dummy", k)
        ))
        , na.rm = T)
    ]
  for (i in interterms) {
    i1 <- unlist(strsplit(i, "\\."))
    i2 <- i1[2]; i1 <- i1[1]
    i0 <- gsub("\\.", "", i)
    dd[, (i) := as.numeric(eval(parse(text=i1)) == i2)]
    dd[, paste0("Demeaned", i0) := 
      eval(parse(text=i)) - mean(eval(parse(text=i)), na.rm = T)] 
    for (k in tobe.interacted)
      dd[, paste0("dummy", k, ".", i0) := 
        eval(parse(text=paste0("Demeaned", i0))) * 
        eval(parse(text=paste0("DemeanedDummy", k)))]
    # undemeand (UD) interactions
    for (k in tobe.interacted)
      dd[, paste0("UDdummy", k, ".", i0) := 
        eval(parse(text=i)) * 
        eval(parse(text = paste0("dummy", k)))]
  }
  # Only for livestock to create demeand Arm*HadCows, Arm*HadCows*Time interactions 
  if (grepl("lvo", dfiles[j])) {
    # demean HadCows
    dd[, "demeanedHadCows" := dummyHadCows - mean(dummyHadCows)]
    dd[, paste0("dummyHadCows.", "dummy", levels(dd[, .Arm])) := 0L]
    dd[, paste0(rep(paste0("dummyHadCows.", "dummy", levels(dd[, .Arm])), 2),  
      rep(paste0(".Time", 3:4), each = 4)) := 0L]
    for (a in levels(dd[, .Arm])) {
      dd[, paste0("dummyHadCows.dummy", a) := 
        eval(parse(text=paste0("DemeanedDummy", a))) * demeanedHadCows]
      dd[, paste0("dummyHadCows.dummy", a, ".Time", 3:4) := 
        .(eval(parse(text = paste0("dummyHadCows.dummy", a))) * DemeanedTime3,
          eval(parse(text = paste0("dummyHadCows.dummy", a))) * DemeanedTime4)]
    }
  }
  dd[, grepout("Demea|demeanedHad", colnames(dd)) := NULL]
  Obs <- rbind(Obs, cbind(dfiles[j], dd[, .(obs = .N), by = .(Arm, tee)]))
  assign(dfiles[j], dd)
  saveRDS(dd, paste0(pathsaveHere, flnames[j], "AdminDataUsedForEstimation.rds"))
  fwrite(dd, paste0(pathsaveHere, flnames[j], "AdminDataUsedForEstimation.prn"), 
    sep = "\t", quote = F)
}
\end{Sinput}
\end{Schunk}
\hspace{-1cm}\begin{minipage}[t]{14cm}
\hfil\textsc{\normalsize Table \refstepcounter{table}\thetable: Data trimming results\label{tab trim}}\\
\setlength{\tabcolsep}{1pt}
\setlength{\baselineskip}{8pt}
\renewcommand{\arraystretch}{.48}
\hfil\begin{tikzpicture}
\node (tbl) {\input{c:/data/GUK/analysis/save/Original1600/TrimmingNumObsTable.tex}};
%\input{c:/dropbox/data/ramadan/save/tablecolortemplate.tex}
\end{tikzpicture}\\
\renewcommand{\arraystretch}{.8}
\setlength{\tabcolsep}{1pt}
\begin{tabular}{>{\hfill\scriptsize}p{1cm}<{}>{\hfill\scriptsize}p{.25cm}<{}>{\scriptsize}p{12cm}<{\hfill}}
Source:& \multicolumn{2}{l}{\scriptsize GUK survey data.}\\
Notes: & 1. & Top panel is observations for all rounds. Bottom panel is observations for round 1 only. We aim for ITT estimates and need to retain original sampled individuals. old$|$iRej$|$\^{}g in \textsf{Mstatus} are strings for old members, individual rejecters, group rejecters, group erosion. con$|$\^{}dro$|$\^{}rep in \textsf{Mgroup} indicates continuing, dropouts, replacing members. tw$|$dou in \textsf{TradGroup} are members who received loans twice and double amount in the 2nd loans. They are omitted from analysis because they are under a different treatment arm.\\
& 2. & 
\end{tabular}
\end{minipage}

Number of observations after trimming: 1. Keep only membership = 1 or 4, which corresponds to Mstatus old, iRej, gR, gE; 2. Keep only continuing, dropouts members in Mgroup.
\begin{Schunk}
\begin{Sinput}
setnames(Obs, "V1", "file")
Obs[, Arm := factor(Arm, levels = c("traditional", "large", "large grace", "cow"))]
# from long to wide: Arm1, Arm2, ... with rows in fileX * teeY
Obs <- reshape(Obs, direction = "wide", idvar = c("file", "tee"),
  timevar = "Arm", v.names = "obs")
setnames(Obs, grepout("obs", colnames(Obs)), 
  gsub("obs.", "", grepout("obs", colnames(Obs))))
setcolorder(Obs, c("file", "tee", "traditional", "large", "large grace", "cow"))
setkey(Obs, file, tee)
Obs
\end{Sinput}
\begin{Soutput}
    file tee traditional large large grace cow
 1:   ar   1         400   400         400 400
 2:   ar   2         398   400         400 398
 3:   ar   3         379   399         398 394
 4:   ar   4         347   395         389 387
 5:   ar   5         307   378         369 370
 6:   ar   6         289   376         355 369
 7:   ar   7         270   374         340 360
 8:   ar   8         267   371         337 359
 9:   ar   9         248   351         317 335
10:   ar  10         241   350         316 330
11:   ar  11         232   338         311 322
12:   ar  12         225   334         300 318
13:   ar  13         187   287         254 269
14:   ar  14         183   283         250 267
15:   ar  15         173   274         234 251
16:   ar  16         156   250         217 229
17:   ar  17          99   169         147 166
18:   ar  18          94   162         142 159
19:   ar  19          86   146         126 138
20:   ar  20          77   131         110 120
21:   ar  21          41    65          64  61
22:   ar  22          39    64          60  57
23:   ar  23          33    55          50  44
24:   ar  24          28    48          39  39
25:   ar  25          12    25          18  18
26:   ar  26          11    25          14  16
27:   ar  27           9    24          13  10
28:   ar  28           8    19          12   8
29:   ar  29           5    12           8   2
30:   ar  30           5    12           7   1
31:   ar  31           4     8           6  NA
32:   ar  32           4     6           4  NA
33:   ar  33           2     2           2  NA
34:   ar  34           2     2           2  NA
35:   ar  35           1     1           1  NA
36:   ar  36           1    NA          NA  NA
37:   ar  37           1    NA          NA  NA
38:   ar  38           1    NA          NA  NA
39:   ar  39           1    NA          NA  NA
40:   ar  40           1    NA          NA  NA
41:  ass   1         398   400         399 399
42:  ass   2         283   389         353 378
43:  ass   3         276   384         349 365
44:  ass   4         238   378         330 330
45:  con   1         283   388         352 378
46:  con   2         276   383         349 365
47:  con   3         238   377         331 331
48:  far   1          21    96          52  57
49:  far   2           5    51          28  27
50:  far   3           2    22          17  12
51:  far   4          NA     2          NA   1
52:  lab   1         398   400         399 399
53:  lab   2         396   400         400 397
54:  lab   3         378   399         398 394
55:  lab   4         351   394         387 385
56:  lab   5         305   374         366 363
57:  lab   6         258   347         327 332
58:  lab   7         191   283         250 271
59:  lab   8         119   187         173 170
60:  lab   9          71   121         104 100
61:  lab  10          39    86          67  64
62:  lab  11          29    57          44  45
63:  lab  12          21    40          27  29
64:  lab  13          14    21          19  18
65:  lab  14           9    13          15   8
66:  lab  15           8    10           9   6
67:  lab  16           5     8           5   3
68:  lab  17           3     3           3   1
69:  lab  18           1     1           1  NA
70:  lab  19           1    NA           1  NA
71:  lab  20           1    NA           1  NA
72:  lvo   1         398   399         379 398
73:  lvo   2         283   390         373 379
74:  lvo   3         276   384         348 365
75:  lvo   4         238   377         330 328
76: sch1   1         460   479         505 487
77: sch1   2         300   396         369 403
78: sch1   3         266   356         340 351
79: sch1   4         204   306         282 277
    file tee traditional large large grace cow
\end{Soutput}
\end{Schunk}

\begin{Schunk}
\begin{Sinput}
s1 <- readRDS(paste0(pathsaveOriginal1600, "RosterAdminSchoolingData.rds"))
s2 <- readRDS(paste0(pathsaveOriginal1600, "RosterAdminSchoolingAugmentedData.rds"))
ass <- readRDS(paste0(pathsaveOriginal1600, "AssetAdminData.rds"))
lvo <- readRDS(paste0(pathsaveOriginal1600, "LivestockAdminData.rds"))
lab <- readRDS(paste0(pathsaveOriginal1600, "LabourIncomeAdminData.rds"))
far <- readRDS(paste0(pathsaveOriginal1600, "FarmRevenueAdminData.rds"))
con <- readRDS(paste0(pathsaveOriginal1600, "ConsumptionAdminData.rds"))
shk <- readRDS(paste0(pathsaveOriginal1600, "Shocks.rds"))
ar <- readRDS(paste0(pathsaveOriginal1600, "RosterAdminData.rds"))
ar[, teenum := 1:.N, by = .(hhid, survey)]
lab[, teenum := 1:.N, by = .(hhid, survey)]
con[, tee := (1:.N)+1, by = hhid]
datafiles <- c("s1", "s2", "ar", "ass", "lvo", "lab", "far", "con")
armtabs <- armtabs.o1600 <- NULL
for (i in 1:length(datafiles[-2])) {
  dx <- get(datafiles[-2][i])
  setorder(dx, hhid, survey, Year, Month)
  if (!any(grepl("^tee$", colnames(dx)))) dx[, tee := 1:.N, by = hhid]
  dx <- dx[tee < AttritIn, ]
  if (i != grep("con", datafiles[-2])) {
    for (j in 1:4) {
      armtabs <- rbind(armtabs,
         table0(dx[tee == j, RArm]))
      armtabs.o1600 <- rbind(armtabs.o1600,
         table0(dx[tee == j & o1600 == 1L, RArm]))
     }
  } else
    for (j in 2:4) {
      armtabs <- rbind(armtabs,
         table0(dx[tee == j & AttritIn != 2, RArm]))
      armtabs.o1600 <- rbind(armtabs.o1600,
         table0(dx[tee == j & AttritIn != 2 & o1600 == 1L, RArm]))
    }
}
armtabs <- data.table(armtabs)
armtabs[, total := rowSums(armtabs)]
armtabs <- data.table(
  files = 
    paste0("\\makebox[1cm]{\\scriptsize ", 
      c(rep(datafiles[-c(2, grep("con", datafiles))], each = 4), 
        rep("con", each = 3)),
      "}")
  , 
  rounds = 
    c(rep(1:4, length(datafiles)-2), 2:4)
  , armtabs)
armtabs[-seq(1, nrow(armtabs), 4), files := ""]
armtabs.o1600 <- data.table(armtabs.o1600)
armtabs.o1600[, total := rowSums(armtabs.o1600)]
armtabs.o1600 <- data.table(
  files = 
    paste0("\\makebox[1cm]{\\scriptsize ", 
      c(rep(datafiles[-c(2, grep("con", datafiles))], each = 4), 
        rep("con", each = 3)),
      "}")
  , 
  rounds = 
    c(rep(1:4, length(datafiles)-2), 2:4)
  , armtabs.o1600)
armtabs.o1600[-seq(1, nrow(armtabs.o1600), 4), files := ""]
amt <- latextab(as.matrix(armtabs), 
  hleft = "\\scriptsize\\hfil$", hcenter = c(1, rep(1.5, ncol(armtabs)-1)), hright = "$", 
  headercolor = "gray80", adjustlineskip = "-.4ex", delimiterline= NULL,
  alternatecolor = "gray90")
amt.o1600 <- latextab(as.matrix(armtabs.o1600), 
  hleft = "\\scriptsize\\hfil$", hcenter = c(1, rep(1.5, ncol(armtabs.o1600)-1)), hright = "$", 
  headercolor = "gray80", adjustlineskip = "-.4ex", delimiterline= NULL,
  alternatecolor = "gray90")
write.tablev(amt,  paste0(pathsaveHere, "NumObsOriginalHHs_all.tex"),
  colnamestrue = F)
write.tablev(amt.o1600,  paste0(pathsaveHere, "NumObsOriginalHHs_o1600.tex"),
  colnamestrue = F)
\end{Sinput}
\end{Schunk}

\hfil\begin{minipage}[t]{12cm}
\hfil\textsc{\normalsize Table \refstepcounter{table}\thetable: Number of observations  in each file at round 1 from HHs with single treatment\label{tab NObsOHall}}\\
\setlength{\tabcolsep}{.5pt}
\setlength{\baselineskip}{10pt}
\renewcommand{\arraystretch}{.7}
\hfil\begin{tikzpicture}
\node (tbl) {\input{c:/data/GUK/analysis/save/Original1600/NumObsOriginalHHs_all.tex}};
%\input{c:/dropbox/data/ramadan/save/tablecolortemplate.tex}
\end{tikzpicture}\\
\renewcommand{\arraystretch}{.8}
\setlength{\tabcolsep}{1pt}
\begin{tabular}{>{\hfill\scriptsize}p{1cm}<{}>{\hfill\scriptsize}p{.25cm}<{}>{\scriptsize}p{10cm}<{\hfill}}
Source:& \multicolumn{2}{l}{\scriptsize Estimated with GUK administrative and survey data.}\\
Notes: & 1. & Sample is all households: Original 1600 and added households through new groups and individuals replacing opt-out members. All households in traditional arm who received more than one loan are excluded.\\
& 2. &  
\end{tabular}
\end{minipage}

\hfil\begin{minipage}[t]{12cm}
\hfil\textsc{\normalsize Table \refstepcounter{table}\thetable: Number of observations in each file at round 1 from original 1600 HHs\label{tab NObsOHo1600}}\\
\setlength{\tabcolsep}{.5pt}
\setlength{\baselineskip}{10pt}
\renewcommand{\arraystretch}{.7}
\hfil\begin{tikzpicture}
\node (tbl) {\input{c:/data/GUK/analysis/save/Original1600/NumObsOriginalHHs_o1600.tex}};
%\input{c:/dropbox/data/ramadan/save/tablecolortemplate.tex}
\end{tikzpicture}\\
\renewcommand{\arraystretch}{.8}
\setlength{\tabcolsep}{1pt}
\begin{tabular}{>{\hfill\scriptsize}p{1cm}<{}>{\hfill\scriptsize}p{.25cm}<{}>{\scriptsize}p{10cm}<{\hfill}}
Source:& \multicolumn{2}{l}{\scriptsize Estimated with GUK administrative and survey data.}\\
Notes: & 1. & Sample is original 1600 households who agree to join the group. This includes households who later dropped out due to flood, group rejections, and individual rejections. All original 1600 households are tracked but some attrit from the sample.\\
& 2. &  
\end{tabular}
\end{minipage}


\newpage
\section{Descriptive statistics of original 1600 HHs}



\begin{Schunk}
\begin{Sinput}
# Following files are created in ImpactEstimatin_body1.rnw using paste0(path1234, "data_read_in_a_list_with_baseline_patched.rds") further trimmed in DataTrimmingMembership1or4.rnw
# All are in long format with time dummies.
s1 <- readRDS(paste0(pathsaveHere, "RosterSchoolingAdminDataUsedForEstimation.rds"))
ar <- readRDS(paste0(pathsaveHere, "RosterAdminDataUsedForEstimation.rds"))
ass <- readRDS(paste0(pathsaveHere, "AssetAdminDataUsedForEstimation.rds"))
lvo <- readRDS(paste0(pathsaveHere, "LivestockAdminDataUsedForEstimation.rds"))
lab <- readRDS(paste0(pathsaveHere, "LabourIncomeAdminDataUsedForEstimation.rds"))
far <- readRDS(paste0(pathsaveHere, "FarmRevenueAdminDataUsedForEstimation.rds"))
con <- readRDS(paste0(pathsaveHere, "ConsumptionAdminDataUsedForEstimation.rds"))
# what to do with errors like below?
#ass[hhid == 7043715, .(hhid, survey, tee)]
\end{Sinput}
\end{Schunk}
\begin{description}
\vspace{1.0ex}\setlength{\itemsep}{1.0ex}\setlength{\baselineskip}{12pt}
\item[c]	continuing members.
\item[d]	drop out members.
\item[a]	absence.
\item[n]	members of a new group.
\item[r]	replacing members.
\end{description}
\begin{Schunk}
\begin{Sinput}
table0(xid[survey == 1, .(ObPattern, Mpattern)])
\end{Sinput}
\begin{Soutput}
         Mpattern
ObPattern caaa caca cacc ccaa ccac ccca cccc daaa dada dadd ddaa ddda dddd naaa
     0111    0    0   14    0    0    0    0    0    0   13    0    0    0    0
     1000   25    0    0    0    0    0    0   68    0    0    0    0    0    5
     1010    0    4    0    0    0    0    0    0    1    0    0    0    0    0
     1011    0    0    0    0    1    0    0    0    0    0    0    0    0    0
     1100    0    0    0   11    0    0    0    0    0    0   14    0    0    0
     1110    0    0    0    0    0   13    0    0    0    0    0   54    0    0
     1111    0    0    0    0    0    0 1153    0    0    0    0    0  229    0
         Mpattern
ObPattern nann nnaa nnna nnnn raaa rara rarr rraa rrra rrrr
     0111    4    0    0    0    0    0    5    0    0    0
     1000    0    0    0    0    2    0    0    0    0    0
     1010    0    0    0    0    0    1    0    0    0    0
     1011    0    0    0    0    0    0    0    0    0    0
     1100    0    2    0    0    0    0    0    1    0    0
     1110    0    0    9    0    0    0    0    0    6    0
     1111    0    0    0  440    0    0    0    0    0  144
\end{Soutput}
\begin{Sinput}
xid[, AttritIn := 9L]
xid[grepl("^En|^2nd and 4", missing_followup), AttritIn := 4L]
xid[grepl("^3rd and 4", missing_followup), AttritIn := 3L]
xid[grepl("^2.*3.*4", missing_followup), AttritIn := 2L]
\end{Sinput}
\end{Schunk}
\textsf{AttritIn}: Attrition round. 9 is nonattriting members.
\begin{Schunk}
\begin{Sinput}
table0(xid[, AttritIn])
\end{Sinput}
\begin{Soutput}

   2    3    4    9 
 100   56  258 7975 
\end{Soutput}
\begin{Sinput}
table0(xid[survey == 1, .(AttritIn, ObPattern)])
\end{Sinput}
\begin{Soutput}
        ObPattern
AttritIn 0111 1000 1010 1011 1100 1110 1111
       2    0  100    0    0    0    0    0
       3    0    0    0    0   28    0    0
       4    0    0    6    0    0   82    0
       9   36    0    0    1    0    0 1966
\end{Soutput}
\begin{Sinput}
table0(xid[, .(AttritIn, survey)])
\end{Sinput}
\begin{Soutput}
        survey
AttritIn    1    2    3    4
       2  100    0    0    0
       3   28   28    0    0
       4   88   82   88    0
       9 2003 1967 2002 2003
\end{Soutput}
\begin{Sinput}
xid <- xid[, .(AssignOriginal, randomization, groupid, hhid, 
  survey, year, AttritIn, ObPattern, Mstatus, Mgroup, 
  creditstatus, IntDate, DistDate1)]
xid[, group.id := as.numeric(substr(hhid, 1, 5))]
xid[group.id == 81710, group.id := as.numeric(substr(hhid, 1, 6))]
xid <- merge(xid, probgp, by = "group.id", all.x = T)
xid[is.na(randomization), AssignOriginal := randomization0]
hhido <- unique(xid[hhid %in% hhid[!grepl("new|Rep", Mstatus) & survey == 1], 
   hhid])
hhidor <- unique(xid[hhid %in% hhid[!grepl("new", Mstatus) & survey == 1], 
   hhid])
\end{Sinput}
\end{Schunk}
\textsf{Mstatus} changes for some \textsf{groupid}s. Correct \textsf{Mstatus} by checking \textsf{comment} for dropping out (taken from CharRandomization2012.prn).
\begin{Schunk}
\begin{Sinput}
xid[, dM := length(unique(Mstatus)) > 1, by = hhid]
table(xid[(dM), .(Mstatus, survey)])
\end{Sinput}
\begin{Soutput}
              survey
Mstatus          1   2   3   4
  gErosion       0   0   0   0
  gRejection   114 114 114   0
  iRejection     1   1   1 114
  iReplacement   0   0   0   0
  newGroup       0   0   0   0
  oldMember      0   0   0   1
\end{Soutput}
\end{Schunk}
See how \textsf{Mstatus} changes at rd 4: This suggests \textsf{iRejection} needs to change to \textsf{gRejection}, and \textsf{iRejection} to \textsf{oldMember}.
\begin{Schunk}
\begin{Sinput}
table0(xid[groupid %in% groupid[(dM)], .(Mstatus, survey)])
\end{Sinput}
\begin{Soutput}
              survey
Mstatus          1   2   3   4
  gErosion      80  55  54   0
  gRejection   140 118 114   0
  iRejection     7   7   5 118
  iReplacement   6   6   6   6
  newGroup       0   0   0   0
  oldMember     13  13  13  14
\end{Soutput}
\begin{Sinput}
#table0(xid[(dM), .(group.id, survey)])
\end{Sinput}
\end{Schunk}
\textsf{group.id} (created from first characters of \textsf{hhid}) and their reasons for dropping out.
\begin{Schunk}
\begin{Sinput}
table0(xid[(dM) & survey == 1, .(group.id, comment)])
\end{Sinput}
\begin{Soutput}
        comment
group.id denial <NA>
   70317     19    0
   70319     20    0
   70539     16    0
   70858     20    0
   71372      0    1
   81483     20    0
   81697     19    0
\end{Soutput}
\end{Schunk}
Correct \textsf{Mstatus} in rd 4 from \textsf{iRejection} to \textsf{gRejection} if denial is the \textsf{comment}. \gobblepars
\begin{Schunk}
\begin{Sinput}
xid[(dM) & grepl("denial", comment) & survey == 4, Mstatus := "gRejection"]
table0(xid[(dM), .(Mstatus, survey)])
\end{Sinput}
\begin{Soutput}
              survey
Mstatus          1   2   3   4
  gErosion       0   0   0   0
  gRejection   114 114 114 114
  iRejection     1   1   1   0
  iReplacement   0   0   0   0
  newGroup       0   0   0   0
  oldMember      0   0   0   1
\end{Soutput}
\begin{Sinput}
xid[, dM2 := length(unique(Mstatus)) > 1, by = hhid]
\end{Sinput}
\end{Schunk}
Correct \textsf{Mstatus} in rd 1-3 from \textsf{iRejection} to \textsf{oldMember} if NA is the \textsf{comment}. \gobblepars
\begin{Schunk}
\begin{Sinput}
xid[(dM2), .(hhid, Mstatus, survey, creditstatus)]
\end{Sinput}
\begin{Soutput}
      hhid    Mstatus survey creditstatus
1: 7137220 iRejection      1          Yes
2: 7137220 iRejection      2          Yes
3: 7137220 iRejection      3          Yes
4: 7137220  oldMember      4          Yes
\end{Soutput}
\begin{Sinput}
table0(xid[(dM2), .(Mstatus, survey)])
\end{Sinput}
\begin{Soutput}
              survey
Mstatus        1 2 3 4
  gErosion     0 0 0 0
  gRejection   0 0 0 0
  iRejection   1 1 1 0
  iReplacement 0 0 0 0
  newGroup     0 0 0 0
  oldMember    0 0 0 1
\end{Soutput}
\begin{Sinput}
xid[(dM2) & is.na(comment) & survey < 4, Mstatus := "oldMember"]
table0(xid[(dM2), .(Mstatus, survey)])
\end{Sinput}
\begin{Soutput}
              survey
Mstatus        1 2 3 4
  gErosion     0 0 0 0
  gRejection   0 0 0 0
  iRejection   0 0 0 0
  iReplacement 0 0 0 0
  newGroup     0 0 0 0
  oldMember    1 1 1 1
\end{Soutput}
\begin{Sinput}
xid[, dM3 := length(unique(Mstatus)) > 1, by = hhid]
if (!any(xid[, dM3])) xid[, dM3 := NULL]
xid[, c("dM", "dM2") := NULL]
\end{Sinput}
\end{Schunk}
Original 1600 HHs (original sample) by arm and membership status.
\begin{Schunk}
\begin{Sinput}
table0(xid[survey==1 & hhid %in% hhido, .(Mstatus, AssignOriginal)])
\end{Sinput}
\begin{Soutput}
              AssignOriginal
Mstatus        traditional large large grace cow
  gErosion              40     0          20  20
  gRejection            80    40          20   0
  iRejection            53    12          22  72
  iReplacement           0     0           0   0
  newGroup               0     0           0   0
  oldMember            227   348         338 308
\end{Soutput}
\end{Schunk}
Including \textsf{r} or individually replacing HHs (replacing sample): 1759
\begin{Schunk}
\begin{Sinput}
table0(xid[survey==1 & hhid %in% hhidor, .(Mstatus, AssignOriginal)])
\end{Sinput}
\begin{Soutput}
              AssignOriginal
Mstatus        traditional large large grace cow
  gErosion              40     0          20  20
  gRejection            80    40          20   0
  iRejection            53    12          22  72
  iReplacement          53    12          22  72
  newGroup               0     0           0   0
  oldMember            227   348         338 308
\end{Soutput}
\end{Schunk}
First disbursement year of individual and replacing samples. We have about 100+ in 2013 for replacing sample.
\begin{Schunk}
\begin{Sinput}
rbind(
  original = table0(year(xid[survey==1 & hhid %in% hhido, DistDate1])),
  replacing = table0(year(xid[survey==1 & hhid %in% hhidor, DistDate1])))
\end{Sinput}
\begin{Soutput}
          2013 2014 2015 <NA>
original   679  313  203  405
replacing  771  348  232  408
\end{Soutput}
\end{Schunk}
Use original sample. \gobblepars
\begin{Schunk}
\begin{Sinput}
# By reshaping wide, I force all obs to have 4 round of data.
# Use only hhid as idvar, because AssignOriginal has NAs.
# If including AssignOriginal in idvar, rows with AssignOriginal = NA
# will be dropped in widened data. 
xidW <- reshape(xid, direction = "wide", 
  # idvar = c("Mstatus", "groupid", "AssignOriginal", "hhid"),
  idvar = "hhid",
  timevar = "survey", v.names = grepout("yea|Date", colnames(xid)))
# keep only original HHs
xidW <- xidW[hhid %in% hhido, ]
# force mechanical reshape long by stripping reshapeWide attributes
attributes(xidW)$reshapeWide <- NULL
xid <- reshape(xidW, direction = "long", 
  idvar = "hhid",
  varying = grepout("yea|Date", colnames(xidW)))
# rename survey to tee
setnames(xid, "time", "tee")
\end{Sinput}
\end{Schunk}
Attrition. % (\textsf{WillAttrit} is just a check if reshaping forced 4 obs per HH.)
\begin{Schunk}
\begin{Sinput}
xid[, WillAttrit := 1L]
xid[hhid %in% hhid[AttritIn>4L], WillAttrit := 0L]
table0(xid[, .(tee, WillAttrit)])
\end{Sinput}
\begin{Soutput}
   WillAttrit
tee    0    1
  1 1410  190
  2 1410  190
  3 1410  190
  4 1410  190
\end{Soutput}
\begin{Sinput}
xid[, Rejected := 0L]
xid[grepl("gR|iR", Mstatus), Rejected := 1L]
\end{Sinput}
\end{Schunk}
Merge \textsf{xid} with other files. Keep \textsf{all==T}.
\begin{Schunk}
\begin{Sinput}
xid[, Fromxid := T]
datafiles <- c("s1", "ar", "ass", "lvo", "lab", "far", "con")
Datafiles <- c("S1", "Ar", "Ass", "Lvo", "Lab", "Far", "Con")
DataFileNames <- c(
  "Schooling", "Repayment", "Asset", "Livestock", 
  "LabourIncome", "FarmIncome", "Consumption")
#lapply(datafiles, function(x) 
#  grepout("AssignO|^Arm$|groupi|hhid|tee", colnames(get(x))))
# use only rd 1 characteristics
xid[, c("year") := NULL]
setkey(xid, AssignOriginal, groupid, hhid, tee)
# tee numbering is not in line with survey. This causes multiple matches per hhid-tee below. Correct.
corrtee <- c("ar", "ass", "lvo")
for (i in corrtee) {
  this <- get(i)
  setkey(this, hhid, survey)
  this[, tee := NULL]
  this[, tee := 1:.N, by = hhid]
  assign(i, this)
}
for (i in 1:length(datafiles)) {
  X <- get(datafiles[i])
  X[, FromFile := 1L]
  # files up to livestock do not have AssignOriginal
  if (i >= 4)
    xx <- merge(xid, X, by = key(xid)[-1], all = T, 
      suffixes = c("", paste0("From", Datafiles[i]))) else
    xx <- merge(xid, X, by = key(xid), all = T, 
      suffixes = c("", paste0("From", Datafiles[i])))
  xx[is.na(FromFile), FromFile := 0L]
  assign(paste0(datafiles[i], "x"), xx)
  saveRDS(xx, paste0(pathsaveHere, "Roster", DataFileNames[i],
     "AdminOriginalHHsDataUsedForEstimation.rds"))
}
\end{Sinput}
\end{Schunk}
\begin{Schunk}
\begin{Sinput}
arx[, en := .N, by = .(hhid, tee)]
arx[hhid %in% hhid[en>1], ]
\end{Sinput}
\end{Schunk}
%Attririon by membership status in repayment-saving:
\begin{Schunk}
\begin{Sinput}
table0(assx[tee == 1, .(Mstatus, WillAttrit)])
\end{Sinput}
\end{Schunk}
Membership status in schooling: Schooling files have multiple observations per household.
\begin{Schunk}
\begin{Sinput}
table0(s1x[Fromxid & tee == 1, Mstatus])
\end{Sinput}
\begin{Soutput}

  gErosion gRejection iRejection  oldMember 
        80        140        234       1872 
\end{Soutput}
\begin{Sinput}
s1x[, teenum := 1:.N, by = .(hhid, tee)]
\end{Sinput}
\end{Schunk}
Number of obs per survey round in the schooling file:
\begin{Schunk}
\begin{Sinput}
table0(s1x[, .(teenum, tee)])
\end{Sinput}
\begin{Soutput}
      tee
teenum    1    2    3    4
     1 1600 1600 1600 1600
     2  682  511  446  322
     3  248  150  120   83
     4   50   26   17   11
     5   13    3    2    2
     6    2    0    0    0
\end{Soutput}
\end{Schunk}
Assets: Original arm assignment by membership status in rd 1: 1820 households.
\begin{Schunk}
\begin{Sinput}
table0(assx[tee == 1, .(Mstatus, AssignOriginal)])
\end{Sinput}
\begin{Soutput}
              AssignOriginal
Mstatus        traditional large large grace cow <NA>
  gErosion              40     0          20  20    0
  gRejection            80    40          20   0    0
  iRejection            53    12          22  72    0
  iReplacement           0     0           0   0    0
  newGroup               0     0           0   0    0
  oldMember            227   348         338 308    0
  <NA>                   0     0           0   0  220
\end{Soutput}
\end{Schunk}
\begin{Schunk}
\begin{Sinput}
arx[, teenum := 1:.N, by = .(hhid, tee)]
labx[, teenum := 1:.N, by = .(hhid, tee)]
datafiles <- c("s1", "ar", "ass", "lvo", "lab", "far", "con")
armtabs <- NULL
for (i in 1:length(datafiles[-2])) {
  dx <- get(paste0(datafiles[-2][i], "x"))
  dx <- dx[tee < AttritIn & FromFile == 1L, ]
  if (i != grep("con", datafiles[-2])) {
    for (j in 1:4)
      armtabs <- 
      #data.table(
      rbind(armtabs,
        table0(dx[Fromxid & tee == j, AssignOriginal]))
      #  )
  } else
    for (j in 2:4)
      armtabs <- 
      #data.table(
      rbind(armtabs,
        table0(conx[Fromxid & tee == j & AttritIn != 2, AssignOriginal]))
      #  )
}
# armtabs <- data.table(rbind(
#     table0(s1x[Fromxid & tee == 1, AssignOriginal]),
#     table0(arx[Fromxid & tee == 1, AssignOriginal]),
#     table0(assx[Fromxid & tee == 1, AssignOriginal]), 
#     table0(lvox[Fromxid & tee == 1, AssignOriginal]), 
#     table0(labx[Fromxid & tee == 1, AssignOriginal]),
#     table0(farx[Fromxid & tee == 1, AssignOriginal]),
#     table0(conx[Fromxid & tee == 2 & AttritIn != 2, AssignOriginal])))
    # consumption is not asked in rd 1
armtabs <- data.table(armtabs)
armtabs[, total := rowSums(armtabs)]
armtabs <- data.table(
  files = 
    paste0("\\makebox[1cm]{\\scriptsize ", 
      c(rep(datafiles[-c(2, grep("con", datafiles))], each = 4), 
        rep("con", each = 3)),
      "}")
  , 
  rounds = 
    c(rep(1:4, length(datafiles)-2), 2:4)
  , armtabs)
armtabs[-seq(1, nrow(armtabs), 4), files := ""]
amt <- latextab(as.matrix(armtabs), 
  hleft = "\\scriptsize\\hfil$", hcenter = c(1, rep(1.5, ncol(armtabs)-1)), hright = "$", 
  headercolor = "gray80", adjustlineskip = "-.2ex", delimiterline= NULL,
  alternatecolor = "gray90")
write.tablev(amt,  paste0(pathsaveHere, "NumObsOriginalHHs.tex"),
  colnamestrue = F)
\end{Sinput}
\end{Schunk}

\hfil\begin{minipage}[t]{12cm}
\hfil\textsc{\normalsize Table \refstepcounter{table}\thetable: Number of observations from original 1600 HHs in round 1\label{tab NObsOH}}\\
\setlength{\tabcolsep}{.5pt}
\setlength{\baselineskip}{10pt}
\renewcommand{\arraystretch}{.7}
\hfil\begin{tikzpicture}
\node (tbl) {\input{c:/data/GUK/analysis/save/Original1600/NumObsOriginalHHs.tex}};
%\input{c:/dropbox/data/ramadan/save/tablecolortemplate.tex}
\end{tikzpicture}\\
\renewcommand{\arraystretch}{.8}
\setlength{\tabcolsep}{1pt}
\begin{tabular}{>{\hfill\scriptsize}p{1cm}<{}>{\hfill\scriptsize}p{.25cm}<{}>{\scriptsize}p{10cm}<{\hfill}}
Source:& \multicolumn{2}{l}{\scriptsize Estimated with GUK administrative and survey data.}\\
Notes: & 1. & \\
& 2. &  
\end{tabular}
\end{minipage}







\section{Estimation using original 1600 HHs}


\subsection{Repayment and net saving}


\begin{Schunk}
\begin{Sinput}
ar <- readRDS(paste0(pathsaveHere, "RosterRepaymentAdminOriginalHHsDataUsedForEstimation.rds"))
ar[survey == 2, Time.2 := 1L]
ar[, Mid := 1:.N, by = .(hhid, survey)]
#ar <- ar[Mid == 1, ]
ar[, Mid := NULL]
ar[, CumSave := CumNetSaving - CumRepaid]
ar[, CumEffectiveRepayment := CumNetSaving + CumRepaid]
ar[, Arm := droplevels(Arm)]
ar[, HeadLiteracy := HeadLiteracy + 0]
source("c:/dropbox/settings/Rsetting/panel_estimator_functions.R")
setorder(ar, hhid, IntDate)
ar[, grepout("LoanY|^Time$", colnames(ar)) := NULL]
ar1 <- ar[, 
  #grepout("groupid|^hhid|tee|RArm|^dummy[A-Z]|^dummy.*[a-z]$|Time|CumRepaid$|CumE.*t$|CumNet|RMOther|RMvalue.[rN]|HeadA|HeadL|Floo|With|Size", 
  grepout("groupid|^hhid|tee|^dummy[A-Z]|^dummy.*[a-z]$|Time|CumRepaid$|CumE.*t$|CumNet|RMOther|RMvalue.[rN]|HeadA|HeadL|Floo|With|Size", 
  colnames(ar)), with = F]
ar1[, grepout("UD|[mM]issw|^Time$|Small|^Size", colnames(ar1)) := NULL]
#  hhid == 7096302, 3 have round 1 observation which corresponds to pre disbursement date. Drop their round 1 data.
# dar1 <- prepFDData(ar1[!((hhid == 7096302 & tee == 1) | (hhid == 7096303 & tee == 1)), ], 
#   Group = "^hhid$", TimeVar = "tee", Cluster = "groupid", 
#   LevelCovariates = "^dumm.*[a-z]$|RAr|Floo|^Time\\..$|HeadL|HeadA|LoanY", 
#   drop.if.NA.in.differencing = T, LevelPeriodToKeep = "last",
#   use.var.name.for.dummy.prefix = F, print.messages = F)
# dar2 <- prepFDData(ar1, Group = "^hhid$", TimeVar = "tee", Cluster = "groupid", 
#   LevelCovariates = "^dumm.*[a-z]$|RAr|Floo|^Time\\..$|HeadL|HeadA|LoanY", 
#   drop.if.NA.in.differencing = T, LevelPeriodToKeep = "last",
#   use.var.name.for.dummy.prefix = F, print.messages = F)
dl <- FirstDiffPanelData(X = ar1, 
  Group = "^hhid$", TimeVar = "tee", Cluster = "groupid",
  LevelCovariates = "^dummy|Head|^Time\\..$|Female$|Floo|Eldest|^Arm|^cred.*s$|xid$|Sch.*Pa")
\end{Sinput}
\begin{Soutput}
Dropped 10938 obs due to NA.
\end{Soutput}
\begin{Sinput}
dard <- dl$diff
dard[, grepout("^en$|Arm", colnames(dard)) := NULL]
datas <- "dard"
for (i in 1:length(datas)) {
  dat <- get(datas[i])
  dat[, grepout("Time.?2", colnames(dat)) := NULL]
  assign(datas[i], dat)
}
dard[, Tee := .N, by = hhid]
table(dard[, Tee])
\end{Sinput}
\begin{Soutput}

   1    2    3    4    5    6    7    8    9   10   11   12   13   14   15   16 
  13   82   42  284   80  108 1106  104  504  610 2574  348  871 3150  345  528 
  17   18   19   20   21   22   23   24   26   29 
1581  126  133  440  105  110  276   48   52   29 
\end{Soutput}
\begin{Sinput}
dard <- dard[Tee > 1, ]
\end{Sinput}
\end{Schunk}
Repayment started in round 2. So taking a first-difference leaves us with period 2-3 and period 3-4. After first-differencing, \textsf{ar} has 13630 rows with 1, 64, 17, 158, 13, 56, 61, 234, 29, 67, 225, 23, 33, 93, 7, 7, 22, 5, 5, 12, 2, 2, 1 individuals with repeatedly observed for 4, 5, 7, 8, 9, 10, 11, 12, 13, 14, 15, 16, 17, 18, 19, 20, 21, 22, 23, 24, 25, 27, 30 times, respectively. 
%Drop nrow(dar2d[tee <= 2, ]) observations in \textsf{ar} that have round 1 data (for unknown reasons). After first-differencing, \textsf{ar} has nrow(dar2d[tee>2, ]) rows with table(dar2d[tee > 2, Tee]) individuals with repeatedly observed for as.numeric(names(table(dar2d[tee>2, Tee])))+1 times, respectively. table(dar2d[tee > 2, Tee])["3"] individuals observed for 4 times started repayment even before official disbursement date, so its round 1 will be dropped.

Note all binary interaction terms are demeaned and then interacted.
\begin{Schunk}
\begin{Sinput}
ar <- readRDS(paste0(pathsaveHere, "RosterAdminDataUsedForEstimation.rds"))
ar[, tee := survey]
ar[, Mid := 1:.N, by = .(hhid, survey)]
ar <- ar[Mid == 1, ]
ar[, Mid := NULL]
ar[, CumSave := CumNetSaving - CumRepaid]
ar[, CumEffectiveRepayment := CumNetSaving + CumRepaid]
ar[, Arm := droplevels(Arm)]
ar[, HeadLiteracy := HeadLiteracy + 0]
source("c:/dropbox/settings/Rsetting/panel_estimator_functions.R")
setorder(ar, hhid, IntDate)
ar[, grepout("LoanY|^Time$", colnames(ar)) := NULL]
#ar[, c("dummyForcedDropOuts") := NULL]
table0(ar[, .(tee, RArm)])
\end{Sinput}
\begin{Soutput}
   RArm
tee traditional large large grace cow
  1         400   400         400 400
  2         280   384         342 366
  3         277   386         348 366
  4         240   382         343 342
\end{Soutput}
\end{Schunk}
NAs in \textsf{CumRepaid}.
\begin{Schunk}
\begin{Sinput}
table0(ar[is.na(CumRepaid), .(tee, Arm)])
\end{Sinput}
\begin{Soutput}
   Arm
tee traditional large large grace cow
  1         398   400         398 400
  2         113    41           0  20
  3         110    39           0  19
  4          75    39           0   0
\end{Soutput}
\end{Schunk}
Tabulation at rd 1:
\begin{Schunk}
\begin{Sinput}
table0(ar[survey == 1, .(Mstatus, RArm)])
\end{Sinput}
\begin{Soutput}
              RArm
Mstatus        traditional large large grace cow
  gErosion              40     0          20  20
  gRejection            80    40          20   0
  iRejection            54    12          22  72
  iReplacement           0     0           0   0
  newGroup               0     0           0   0
  oldMember            226   348         338 308
\end{Soutput}
\end{Schunk}
\begin{Schunk}
\begin{Sinput}
library(ggplot2)
ggplot(ar[!is.na(Date), .(Arm, hhid, Date, MonthsElapsed, CumNetSaving)], 
  aes(x = MonthsElapsed, y = CumNetSaving, colour = Arm, group = Arm)) +
  geom_point(aes(colour = Arm), size = .1, position = position_dodge(width = .5)) +
  geom_smooth(span = .5, aes(colour = Arm, group = Arm)) +
  theme(legend.position="none", legend.key = element_rect(fill = "white")) + 
  scale_y_continuous() +
  scale_x_continuous(limits = c(-12, 48), breaks = seq(-12, 48, 12)) +
  xlab("Months since 1st loan disbursement") + ylab("Normalised repayment amount") +
  facet_grid(. ~ Arm, scales = "free_y")
\end{Sinput}
\begin{figure}

{\centering \includegraphics[width=\maxwidth]{figure/ImpactEstimationOriginal1600/Cumulative_net_saving_original_HHs-1} 

}

\caption[Cumulative weekly net saving]{Cumulative weekly net saving}\label{Figure Cumulative net saving original HHs}
\end{figure}
\end{Schunk}

\begin{Schunk}
\begin{Sinput}
ar <- readRDS(paste0(pathsaveHere, "RosterRepaymentAdminOriginalHHsDataUsedForEstimation.rds"))
ar[survey == 2, Time.2 := 1L]
ar[, Mid := 1:.N, by = .(hhid, survey)]
#ar <- ar[Mid == 1, ]
ar[, Mid := NULL]
ar[, CumSave := CumNetSaving - CumRepaid]
ar[, CumEffectiveRepayment := CumNetSaving + CumRepaid]
ar[, Arm := droplevels(Arm)]
ar[, HeadLiteracy := HeadLiteracy + 0]
source("c:/dropbox/settings/Rsetting/panel_estimator_functions.R")
setorder(ar, hhid, IntDate)
ar[, grepout("LoanY|^Time$", colnames(ar)) := NULL]
ar1 <- ar[, 
  #grepout("groupid|^hhid|tee|RArm|^dummy[A-Z]|^dummy.*[a-z]$|Time|CumRepaid$|CumE.*t$|CumNet|RMOther|RMvalue.[rN]|HeadA|HeadL|Floo|With|Size", 
  grepout("groupid|^hhid|tee|^dummy[A-Z]|^dummy.*[a-z]$|Time|CumRepaid$|CumE.*t$|CumNet|RMOther|RMvalue.[rN]|HeadA|HeadL|Floo|With|Size", 
  colnames(ar)), with = F]
ar1[, grepout("UD|[mM]issw|^Time$|Small|^Size", colnames(ar1)) := NULL]
#  hhid == 7096302, 3 have round 1 observation which corresponds to pre disbursement date. Drop their round 1 data.
# dar1 <- prepFDData(ar1[!((hhid == 7096302 & tee == 1) | (hhid == 7096303 & tee == 1)), ], 
#   Group = "^hhid$", TimeVar = "tee", Cluster = "groupid", 
#   LevelCovariates = "^dumm.*[a-z]$|RAr|Floo|^Time\\..$|HeadL|HeadA|LoanY", 
#   drop.if.NA.in.differencing = T, LevelPeriodToKeep = "last",
#   use.var.name.for.dummy.prefix = F, print.messages = F)
# dar2 <- prepFDData(ar1, Group = "^hhid$", TimeVar = "tee", Cluster = "groupid", 
#   LevelCovariates = "^dumm.*[a-z]$|RAr|Floo|^Time\\..$|HeadL|HeadA|LoanY", 
#   drop.if.NA.in.differencing = T, LevelPeriodToKeep = "last",
#   use.var.name.for.dummy.prefix = F, print.messages = F)
dl <- FirstDiffPanelData(X = ar1, 
  Group = "^hhid$", TimeVar = "tee", Cluster = "groupid",
  LevelCovariates = "^dummy|Head|^Time\\..$|Female$|Floo|Eldest|^Arm|^cred.*s$|xid$|Sch.*Pa")
\end{Sinput}
\begin{Soutput}
Dropped 10938 obs due to NA.
\end{Soutput}
\begin{Sinput}
dard <- dl$diff
dard[, grepout("^en$|Arm", colnames(dard)) := NULL]
datas <- "dard"
for (i in 1:length(datas)) {
  dat <- get(datas[i])
  dat[, grepout("Time.?2", colnames(dat)) := NULL]
  assign(datas[i], dat)
}
dard[, Tee := .N, by = hhid]
table(dard[, Tee])
\end{Sinput}
\begin{Soutput}

   1    2    3    4    5    6    7    8    9   10   11   12   13   14   15   16 
  13   82   42  284   80  108 1106  104  504  610 2574  348  871 3150  345  528 
  17   18   19   20   21   22   23   24   26   29 
1581  126  133  440  105  110  276   48   52   29 
\end{Soutput}
\begin{Sinput}
dard <- dard[Tee > 1, ]
\end{Sinput}
\end{Schunk}
\begin{Schunk}
\begin{Sinput}
FileName <- "Saving"
FileNameHeader <- paste0(c("", "Grace", "PovertyStatus", "Size", "Attributes"),
  "OriginalHHs")
arsuffixes <- c("", "g", "p", "s", "a")
listheader <- paste0("sv", arsuffixes)
Regressands <-  c(rep("CumNetSaving", 2), rep("CumRepaid", 3), 
  rep("CumEffectiveRepayment", 3))
DataToUse1 <- DataToUse2 <- rep("dard", 8)
Addseparatingcols = c(2,5); Separatingcolwidth = rep(.2, 2)
Separatingcoltitle = c("Cumulative net saving", "Cumulative repayment", 
   "Cumulative net saving + cumulative repayment")
\end{Sinput}
\end{Schunk}
\begin{Schunk}
\begin{Sinput}
source(paste0(pathprogram, "RepaymentCovariateSelection.R"))
\end{Sinput}
\end{Schunk}
\begin{Schunk}
\begin{Sinput}
exclheader <- paste0("excl", arsuffixes)
source(paste0(pathprogram, "FDEstimationFile.R"))
\end{Sinput}
\begin{Soutput}
Loading required package: sandwich
\end{Soutput}
\begin{Soutput}
Warning: package 'sandwich' was built under R version 3.5.2
\end{Soutput}
\begin{Soutput}
Loading required package: lmtest
\end{Soutput}
\end{Schunk}
\begin{Schunk}
\begin{Sinput}
# svX <- sv12$data[, .(tee,
#   T2 = dummyTraditional.Time2 > 0, L2 = dummyLarge.Time2 > 0,
#   G2 = dummyLargeGrace.Time2 > 0, C2 = dummyCow.Time2 > 0, 
#   T3 = dummyTraditional.Time3 > 0, L3 = dummyLarge.Time3 > 0,
#   G3 = dummyLargeGrace.Time3 > 0, C3 = dummyCow.Time3 > 0, 
#   T4 = dummyTraditional.Time4 > 0, L4 = dummyLarge.Time4 > 0,
#   G4 = dummyLargeGrace.Time4 > 0, C4 = dummyCow.Time4 > 0 )]
# svX <- sv12$data[, .(
#   dummyTraditional.Time2 , dummyLarge.Time2 ,
#   dummyLargeGrace.Time2 , dummyCow.Time2 , 
#   dummyTraditional.Time3 , dummyLarge.Time3 ,
#   dummyLargeGrace.Time3 , dummyCow.Time3 , 
#   dummyTraditional.Time4 , dummyLarge.Time4 ,
#   dummyLargeGrace.Time4 , dummyCow.Time4 )]
LinDependent <- function(z, ShowMostDependent = F, ReturnColNames = F) 
#  From CrossVal: https://stats.stackexchange.com/questions/16327/testing-for-linear-dependence-among-the-columns-of-a-matrix#39321
#  The weakness of this function is that it does not specify which columns are jointly linearly dependent.
#  ShowMostDependent: if T, returns column that is least linearly independent, if F, returns columns that are linearly dependent.
{
  if (!is.matrix(z)) z <- as.matrix(z)
  rankofz <- qr(z)$rank
  if (rankofz == ncol(z)) message("Full rank.") else
  {
    rankifremoved <- sapply(1:ncol(z), function (x) qr(z[, -x])$rank)
    if (ReturnColNames) {
      if (ShowMostDependent)
        this <- colnames(z)[rankifremoved == max(rankifremoved)] else
        this <- colnames(z)[rankifremoved == ncol(z) - 1] 
    } else {
      if (!ShowMostDependent)
        this <- which(rankifremoved == max(rankifremoved)) else
        this <- which(rankifremoved == ncol(z) - 1)
    }
    return(this)
  }
}
# svX <- as.matrix(sv12$data[, .(
#   Time.2, dummyLarge.Time2,
#   dummyLargeGrace.Time2, dummyCow.Time2, 
#   Time.3 , dummyLarge.Time3 ,
#   dummyLargeGrace.Time3 , dummyCow.Time3 , 
#   Time.4 , dummyLarge.Time4 ,
#   dummyLargeGrace.Time4, dummyCow.Time4 )])
#LinDependent(svX, F, T) 
\end{Sinput}
\end{Schunk}
\begin{Schunk}
\begin{Sinput}
arsv <- ar[, .(Arm, groupid, hhid, tee = as.factor(tee))]
svDatalist <- list(arsv, arsv, arsv, arsv, arsv, arsv, arsv, arsv)
\end{Sinput}
\end{Schunk}
\begin{Schunk}
\begin{Sinput}
InTermsSV <- lapply(svDatalist, function(x) 
  interactXY(
    makeDummyFromFactor(x[, Arm], NULL), 
    makeDummyFromFactor(x[, tee], NULL)
    ))
InTermsSV <- rbindlist(lapply(InTermsSV, function(x) {
  z <- data.table(t(c(nrow(x), unlist(lapply(1:ncol(x), function(i) sum(x[, i, with = F]))))))
  setnames(z, gsub(" ", "", gsub("dummy", "", c("total", colnames(x)))))
  z
}))
InTermsSV <- InTermsSV[, which(unlist(lapply(InTermsSV, function(x) !all(is.na(x) | x == 0)))), with=F]
InTermsSV <- t(InTermsSV)
colnames(InTermsSV) <- paste0("(", 1:ncol(InTermsSV), ")") 
InTermsSV <- InTermsSV[c(grep("Tra", rownames(InTermsSV)), 
  grep("Large[^g]", rownames(InTermsSV)),
  grep("Largeg", rownames(InTermsSV)),
  grep("Cow", rownames(InTermsSV)),
  grep("total", rownames(InTermsSV))
  ), ]
# reorder within a group
rn.j <- rownames(InTermsSV)
newroworder <- NULL
for (j in c("Tra", "Large[^g]", "Largeg", "Cow"))
  newroworder <- c(newroworder, 
    c(grep(paste0(j, ".*ale$"), rn.j), grep(paste0(j, ".*P"), rn.j), 
      grep(paste0(j, ".*J"), rn.j), grep(paste0(j, ".*H"), rn.j)))
InTermsSV <- InTermsSV[c(newroworder, nrow(InTermsSV)), ]
\end{Sinput}
\end{Schunk}
\begin{Schunk}
\begin{Sinput}
#dummy chunk
\end{Sinput}
\end{Schunk}


\hspace{-1cm}\begin{minipage}[t]{14cm}
\hfil\textsc{\normalsize Table \refstepcounter{table}\thetable: FD estimation of cumulative net saving and repayment\label{tab FD saving original HH}}\\
\setlength{\tabcolsep}{1pt}
\setlength{\baselineskip}{8pt}
\renewcommand{\arraystretch}{.55}
\hfil\begin{tikzpicture}
\node (tbl) {\input{c:/data/GUK/analysis/save/Original1600/SavingOriginalHHsFDEstimationResults.tex}};
%\input{c:/dropbox/data/ramadan/save/tablecolortemplate.tex}
\end{tikzpicture}\\
\renewcommand{\arraystretch}{.8}
\setlength{\tabcolsep}{1pt}
\begin{tabular}{>{\hfill\scriptsize}p{1cm}<{}>{\hfill\scriptsize}p{.25cm}<{}>{\scriptsize}p{12cm}<{\hfill}}
Source:& \multicolumn{2}{l}{\scriptsize Estimated with GUK administrative and survey data.}\\
Notes: & 1. & First-difference estimates using rd 2 - rd 4 data. First-differenced ($\Delta x_{t+1}\equiv x_{t+1} - x_{t}$) regressands are regressed on categorical and time-variant covariates. Net saving is taken from administrative data and merged with survey data at Year-Month of survey interviews. Head age and literacy are from baseline data. Intercept terms are omitted in estimating equations. Net saving is saving - withdrawal. \\
& 2. & ${}^{***}$, ${}^{**}$, ${}^{*}$ indicate statistical significance at 1\%, 5\%, 10\%, respetively. Standard errors are clustered at group (village) level.
\end{tabular}
\end{minipage}


\hspace{-1cm}\begin{minipage}[t]{14cm}
\hfil\textsc{\normalsize Table \refstepcounter{table}\thetable: FD estimation of cumulative net saving and repayment by attributes\label{tab FD saving attributes original HH}}\\
\setlength{\tabcolsep}{1pt}
\setlength{\baselineskip}{8pt}
\renewcommand{\arraystretch}{.55}
\hfil\begin{tikzpicture}
\node (tbl) {\input{c:/data/GUK/analysis/save/Original1600/SavingAttributesOriginalHHsFDEstimationResults.tex}};
%\input{c:/dropbox/data/ramadan/save/tablecolortemplate.tex}
\end{tikzpicture}\\
\renewcommand{\arraystretch}{.8}
\setlength{\tabcolsep}{1pt}
\begin{tabular}{>{\hfill\scriptsize}p{1cm}<{}>{\hfill\scriptsize}p{.25cm}<{}>{\scriptsize}p{12cm}<{\hfill}}
Source:& \multicolumn{2}{l}{\scriptsize Estimated with GUK administrative and survey data.}\\
Notes: & 1. & First-difference estimates using rd 2 - rd 4 data. First-differenced ($\Delta x_{t+1}\equiv x_{t+1} - x_{t}$) regressands are regressed on categorical and time-variant covariates. Net saving is taken from administrative data and merged with survey data at Year-Month of survey interviews. Head age and literacy are from baseline data. Intercept terms are omitted in estimating equations. Net saving is saving - withdrawal. \\
& 2. & ${}^{***}$, ${}^{**}$, ${}^{*}$ indicate statistical significance at 1\%, 5\%, 10\%, respetively. Standard errors are clustered at group (village) level.
\end{tabular}
\end{minipage}

\hspace{-1cm}\begin{minipage}[t]{14cm}
\hfil\textsc{\normalsize Table \refstepcounter{table}\thetable: FD estimation of net cumulative saving and repayment, ultra poor vs. moderately poor\label{tab FD saving2 original HH}}\\
\setlength{\tabcolsep}{1pt}
\setlength{\baselineskip}{8pt}
\renewcommand{\arraystretch}{.55}
\hfil\begin{tikzpicture}
\node (tbl) {\input{c:/data/GUK/analysis/save/Original1600/SavingPovertystatusOriginalHHsFDEstimationResults.tex}};
%\input{c:/dropbox/data/ramadan/save/tablecolortemplate.tex}
\end{tikzpicture}\\
\renewcommand{\arraystretch}{.8}
\setlength{\tabcolsep}{1pt}
\begin{tabular}{>{\hfill\scriptsize}p{1cm}<{}>{\hfill\scriptsize}p{.25cm}<{}>{\scriptsize}p{12cm}<{\hfill}}
Source:& \multicolumn{2}{l}{\scriptsize Estimated with GUK administrative and survey data.}\\
Notes: & 1. & First-difference estimates using rd 2 - rd 4 data. First-differenced ($\Delta x_{t+1}\equiv x_{t+1} - x_{t}$) regressands are regressed on categorical and time-variant covariates. Net saving is taken from administrative data and merged with survey data at Year-Month of survey interviews. Head age and literacy are from baseline data. Intercept terms are omitted in estimating equations. Net saving is saving - withdrawal. \\
& 2. & ${}^{***}$, ${}^{**}$, ${}^{*}$ indicate statistical significance at 1\%, 5\%, 10\%, respetively. Standard errors are clustered at group (village) level.
\end{tabular}
\end{minipage}

\hspace{-1cm}\begin{minipage}[t]{14cm}
\hfil\textsc{\normalsize Table \refstepcounter{table}\thetable: FD estimation of net cumulative saving and repayment, with vs. without a grace period\label{tab FD saving3 original HH}}\\
\setlength{\tabcolsep}{1pt}
\setlength{\baselineskip}{8pt}
\renewcommand{\arraystretch}{.55}
\hfil\begin{tikzpicture}
\node (tbl) {\input{c:/data/GUK/analysis/save/Original1600/SavingGraceOriginalHHsFDEstimationResults.tex}};
%\input{c:/dropbox/data/ramadan/save/tablecolortemplate.tex}
\end{tikzpicture}\\
\renewcommand{\arraystretch}{.8}
\setlength{\tabcolsep}{1pt}
\begin{tabular}{>{\hfill\scriptsize}p{1cm}<{}>{\hfill\scriptsize}p{.25cm}<{}>{\scriptsize}p{12cm}<{\hfill}}
Source:& \multicolumn{2}{l}{\scriptsize Estimated with GUK administrative and survey data.}\\
Notes: & 1. & First-difference estimates using rd 2 - rd 4 data. First-differenced ($\Delta x_{t+1}\equiv x_{t+1} - x_{t}$) regressands are regressed on categorical and time-variant covariates. Net saving is taken from administrative data and merged with survey data at Year-Month of survey interviews. Head age and literacy are from baseline data. Intercept terms are omitted in estimating equations. Net saving is saving - withdrawal. All dummy interaction terms are first demeaned and then interacted.\\
& 2. & ${}^{***}$, ${}^{**}$, ${}^{*}$ indicate statistical significance at 1\%, 5\%, 10\%, respetively. Standard errors are clustered at group (village) level.
\end{tabular}
\end{minipage}

\hspace{-1cm}\begin{minipage}[t]{14cm}
\hfil\textsc{\normalsize Table \refstepcounter{table}\thetable: FD estimation of net cumulative saving and repayment, small size vs. large size\label{tab FD saving4 original HH}}\\
\setlength{\tabcolsep}{1pt}
\setlength{\baselineskip}{8pt}
\renewcommand{\arraystretch}{.55}
\hfil\begin{tikzpicture}
\node (tbl) {\input{c:/data/GUK/analysis/save/Original1600/SavingSizeOriginalHHsFDEstimationResults.tex}};
%\input{c:/dropbox/data/ramadan/save/tablecolortemplate.tex}
\end{tikzpicture}\\
\renewcommand{\arraystretch}{.8}
\setlength{\tabcolsep}{1pt}
\begin{tabular}{>{\hfill\scriptsize}p{1cm}<{}>{\hfill\scriptsize}p{.25cm}<{}>{\scriptsize}p{12cm}<{\hfill}}
Source:& \multicolumn{2}{l}{\scriptsize Estimated with GUK administrative and survey data.}\\
Notes: & 1. & First-difference estimates using rd 2 - rd 4 data. First-differenced ($\Delta x_{t+1}\equiv x_{t+1} - x_{t}$) regressands are regressed on categorical and time-variant covariates. Net saving is taken from administrative data and merged with survey data at Year-Month of survey interviews. Head age and literacy are from baseline data. Intercept terms are omitted in estimating equations. Net saving is saving - withdrawal. All dummy interaction terms are first demeaned and then interacted.\\
& 2. & ${}^{***}$, ${}^{**}$, ${}^{*}$ indicate statistical significance at 1\%, 5\%, 10\%, respetively. Standard errors are clustered at group (village) level.
\end{tabular}
\end{minipage}


\begin{palepinkleftbar}
\begin{finding}
\textsc{\small Table \ref{tab FD saving original HH}} (1) shows net saving increases, (2) shows that initially a larger then a smaller extent in the later rounds. This reduction may reflect the use of saving for repayment. \textsf{traditional} arm has the lowest repayment rates. Ultra poor and moderately poor have similar repayment rates as indicated in \textsc{\small Table \ref{tab FD saving2 original HH}}. \textsc{Table \ref{tab FD saving3 original HH}} (2) shows having a grace period increases the repayment amount while reduces net saving in later rounds. (4) and (5) show cumulative repayment is greater for with grace because each installment is larger. These are all by design that they do not repay in rd 1 so saving increases then they tap in these saving for repayment. 
\end{finding}
\end{palepinkleftbar}


\subsection{Schooling}



\begin{Schunk}
\begin{Sinput}
source(paste0(pathprogram, "ReadTrimSchoolingOriginalHHsFDData2.R"))
\end{Sinput}
\begin{Soutput}
Warning in `[.data.table`(s1xR, , `:=`(c("Age_1", grepout("Primary", colnames(s1xR))), : Adding new column 'Age_1' then assigning NULL (deleting it).
\end{Soutput}
\begin{Soutput}
Dropped 1721 obs due to NA.
Dropped 1721 obs due to NA.
Dropped 399 obs due to T<2.
Dropped 1136 obs due to NA.
\end{Soutput}
\end{Schunk}
Enrollment pattern in original schooling panel. `n' indicates NA (either attrition or not reported).
\begin{Schunk}
\begin{Sinput}
table0(s.1x[tee == 1, .(ObPattern, SchPattern)])
\end{Sinput}
\begin{Soutput}
         SchPattern
ObPattern 0000 0001 000n 0011 001n 00nn 0100 010n 0111 011n 01nn 0nnn 1000 1001
     0111    0    0    0    0    0    0    0    0    0    2    2    6    0    0
     1000    0    0    0    0    0    0    0    0    0    0    0   63    0    0
     1010    0    0    0    0    0    1    0    0    0    0    0    4    0    0
     1011    0    0    0    0    0    0    0    0    0    0    0    0    0    0
     1100    0    0    0    0    0    2    0    0    0    0    5    2    0    0
     1110    0    0    7    0    2    2    0    0    0    8    0    3    0    0
     1111   40    7   41   25    4   50    2    2  173   15   11  182   13    2
         SchPattern
ObPattern 100n 1011 101n 10nn 1100 1101 110n 1110 1111 111n 11n1 11nn 1nnn
     0111    0    0    0    1    0    0    0    0    0   12    0    0    5
     1000    0    0    0    0    0    0    0    0    0    0    0    0   56
     1010    0    0    0    0    0    0    0    0    0    0    0    1    4
     1011    0    0    0    0    0    0    0    0    0    0    0    1    0
     1100    0    0    0    0    0    0    0    0    0    0    0   12    3
     1110    2    0    1    0    0    0    1    0    0   42    0    5    0
     1111    9    9    4   17   11    1   16    4  781   77    1   44  135
\end{Soutput}
\end{Schunk}
Left panel is before dropping \textsf{nnn}, right panel is after: Original panel.
\begin{Schunk}
\begin{Sinput}
cbind(table0(s.1x[, .(tee, RArm)]), 
  table0(s1x[, .(tee, RArm)]))
\end{Sinput}
\begin{Soutput}
  traditional large large grace cow traditional large large grace cow
1         460   479         505 487         300   396         369 403
2         300   396         369 403         300   396         369 403
3         266   356         340 351         266   356         340 351
4         204   306         282 277         204   306         282 277
\end{Soutput}
\end{Schunk}
If using \textsf{s1x}, retain only the complete portion of panel. \textsf{sch1} has 5781 rows. Drop 463 observations in \textsf{sch1} with nnn in \textsf{SchPattern}. 
%and nrow(s.1x[!grepl("nnn", Spattern) & grepl("1001", EnrollPattern), ]) observations with 1001 in \textsf{EnrollPattern} because they are likely to be errors. This leaves us with nrow(s1x) rows. 
\begin{Schunk}
\begin{Sinput}
#s.1 <- s.1[!grepl("1001", EnrollPattern), ]
s1x[, Enrolled := as.numeric(Enrolled)]
s1x[, Fromxid := NULL]
s1x[, Tee := .N, by = HHMid]
ds1xd[, Tee := .N, by = HHMid]
\end{Sinput}
\end{Schunk}
With OLS,  154, 246, 1068 individuals are repeatedly observed for 2, 3, 4 times, respectively. With FD, \textsf{s1x} is reduced to 3597 rows after first-differencing with 140, 231, 993 individuals with repeatedly observed for 2, 3, 4 times, respectively.
Individuals with NAs in \textsf{Enrolled}. 0 obs for \textsf{s1x}. 
%Mostly older children (round(mean(s.1[is.na(Enrolled), Age_1], na.rm = T), 1) in \textsf{s1x}, round(mean(s.2[is.na(Enrolled), Age_1], na.rm = T), 1) in \textsf{s.2}) but with a high reported enrollment rate (round(mean(s.1[is.na(Enrolled) & tee == 4, Enrolled]), 1) for \textsf{s1x}, round(mean(s.2[is.na(Enrolled) & tee == 4, Enrolled]), 1) for \textsf{s.2}) at rd 4. We will substitute relevant schooling levels to \textsf{Enrolled}.
Check missingness in schooling level information.
\begin{Schunk}
\begin{Sinput}
table0(apply(s1x[, .(dummyJunior, dummyHigh)], 1, sum))
\end{Sinput}
\begin{Soutput}

   0    1 
3065 2253 
\end{Soutput}
\end{Schunk}
Check missingness in arm information.
\begin{Schunk}
\begin{Sinput}
table0(apply(s1x[, .(dummyTraditional, dummyLarge, dummyLargeGrace, dummyCow)], 1, sum))
\end{Sinput}
\begin{Soutput}
   1 
5318 
\end{Soutput}
\end{Schunk}
Drop 3065 obs without school level information.
\begin{Schunk}
\begin{Sinput}
s1x <- s1x[apply(s1x[, .(dummyJunior, dummyHigh)], 1, sum) == 1, ]
ds1xd[, grepout("^Tee$", colnames(ds1xd)) := NULL]
\end{Sinput}
\end{Schunk}
An example of dummy interactions: \textsf{\footnotesize dummyNonCash.dummyPrimary.Time.2, dummyCash.dummyPrimary.Time.2, dummyNonCash.dummyJunior.Time.2, dummyCash.dummyJunior.Time.2, dummyNonCash.dummyHigh.Time.2, dummyCash.dummyHigh.Time.2, dummyNonCash.dummyPrimary.Time.3, dummyCash.dummyPrimary.Time.3, dummyNonCash.dummyJunior.Time.3, dummyCash.dummyJunior.Time.3, dummyNonCash.dummyHigh.Time.3, dummyCash.dummyHigh.Time.3, dummyNonCash.dummyPrimary.Time.4, dummyCash.dummyPrimary.Time.4, dummyNonCash.dummyJunior.Time.4, dummyCash.dummyJunior.Time.4, dummyNonCash.dummyHigh.Time.4, dummyCash.dummyHigh.Time.4}.
Obs for \textsf{s1x}.
\begin{Schunk}
\begin{Sinput}
table(ds1xd[, tee])
\end{Sinput}
\begin{Soutput}

   2    3    4 
1364 1228 1005 
\end{Soutput}
\end{Schunk}
Obs for \textsf{s1x} and admin repayment data.
\begin{Schunk}
\begin{Sinput}
table(ds1xRd[, tee])
\end{Sinput}
\begin{Soutput}

   2    3    4 
1364 1228 1005 
\end{Soutput}
\end{Schunk}
\begin{Schunk}
\begin{Sinput}
source(paste0(pathprogram, "ReadTrimSchoolingOriginalHHsFDData2.R"))
\end{Sinput}
\begin{Soutput}
Warning in `[.data.table`(s1xR, , `:=`(c("Age_1", grepout("Primary", colnames(s1xR))), : Adding new column 'Age_1' then assigning NULL (deleting it).
\end{Soutput}
\begin{Soutput}
Dropped 1721 obs due to NA.
Dropped 1721 obs due to NA.
Dropped 399 obs due to T<2.
Dropped 1136 obs due to NA.
\end{Soutput}
\end{Schunk}
\begin{Schunk}
\begin{Sinput}
FileName <- "Schooling"
Regressands <- rep("Enrolled", 4)
Addseparatingcols = NULL; Separatingcolwidth = NULL
Separatingcoltitle = NULL
\end{Sinput}
\end{Schunk}
\begin{Schunk}
\begin{Sinput}
Scsuffixes <- c("", "g", "p", "s", "a", "T", "Tg", "Ts", "D", "Dg", "Da")
exclheader <- paste0("excl", Scsuffixes)
source(paste0(pathprogram, "SchoolingCovariateSelection.R"))
\end{Sinput}
\end{Schunk}
\begin{Schunk}
\begin{Sinput}
# Need to place ED14Diff after k > 5.
FileNameHeaderSchooling <- c("", "Grace", "PovertyStatus", "Size", "Attributes",
  "Rd14Diff", "Rd14DiffGrace", "Rd14DiffAttributes")
FileNameHeader <- paste0(FileNameHeaderSchooling, "OriginalHHs")
Scsuffixes <- c("", "g", "p", "s", "a", "D", "Dg", "Da")
listheader <- paste0("sc", Scsuffixes)
exclheader <- paste0("excl", Scsuffixes)
DataToUse1 <- rep("ds1xd", 4)
DataToUse2 <- rep("ds1x34d", 4)
source(paste0(pathprogram, "FDEstimationFile.R"))
\end{Sinput}
\end{Schunk}
\begin{Schunk}
\begin{Sinput}
FileNameHeaderSchooling <- c("TInt", "TIntGrace", "TIntSize")
FileNameHeader <- paste0(FileNameHeaderSchooling, "OriginalHHs")
Scsuffixes <- c("T", "Tg", "Ts")
exclheader <- paste0("excl", Scsuffixes)
listheader <- paste0("sc", Scsuffixes)
source(paste0(pathprogram, "FDEstimationFileSchooling.R"))
\end{Sinput}
\end{Schunk}
\begin{Schunk}
\begin{Sinput}
#dummy chunk
\end{Sinput}
\end{Schunk}

\hspace{-1cm}\begin{minipage}[t]{14cm}
\hfil\textsc{\normalsize Table \refstepcounter{table}\thetable: FD estimation of school enrollment\label{tab FD enroll original HH}}\\
\setlength{\tabcolsep}{1pt}
\setlength{\baselineskip}{8pt}
\renewcommand{\arraystretch}{.48}
\hfil\begin{tikzpicture}
\node (tbl) {\input{c:/data/GUK/analysis/save/Original1600/SchoolingOriginalHHsFDEstimationResults.tex}};
%\input{c:/dropbox/data/ramadan/save/tablecolortemplate.tex}
\end{tikzpicture}\\
\renewcommand{\arraystretch}{.8}
\setlength{\tabcolsep}{1pt}
\begin{tabular}{>{\hfill\scriptsize}p{1cm}<{}>{\hfill\scriptsize}p{.25cm}<{}>{\scriptsize}p{12cm}<{\hfill}}
Source:& \multicolumn{2}{l}{\scriptsize Estimated with GUK administrative and survey data.}\\
Notes: & 1. & First-difference estimates. A first-difference is defined as $\Delta x_{t+1}\equiv x_{t+1} - x_{t}$. First-differenced regressands are regressed on categorical and time-variant covariates. Net saving is taken from administrative data and merged with survey data at Year-Month of survey interviews. Head age and literacy are from baseline data. Intercept terms are omitted in estimating equations. Net saving is saving - withdrawal. \\
& 2. & ${}^{***}$, ${}^{**}$, ${}^{*}$ indicate statistical significance at 1\%, 5\%, 10\%, respetively. Standard errors are clustered at group (village) level.
\end{tabular}
\end{minipage}

\hspace{-1cm}\begin{minipage}[t]{14cm}
\hfil\textsc{\normalsize Table \refstepcounter{table}\thetable: FD estimation of school enrollment by attributes\label{tab FD enroll attributes original HH}}\\
\setlength{\tabcolsep}{1pt}
\setlength{\baselineskip}{8pt}
\renewcommand{\arraystretch}{.48}
\hfil\begin{tikzpicture}
\node (tbl) {\input{c:/data/GUK/analysis/save/Original1600/SchoolingAttributesOriginalHHsFDEstimationResults.tex}};
%\input{c:/dropbox/data/ramadan/save/tablecolortemplate.tex}
\end{tikzpicture}\\
\renewcommand{\arraystretch}{.8}
\setlength{\tabcolsep}{1pt}
\begin{tabular}{>{\hfill\scriptsize}p{1cm}<{}>{\hfill\scriptsize}p{.25cm}<{}>{\scriptsize}p{12cm}<{\hfill}}
Source:& \multicolumn{2}{l}{\scriptsize Estimated with GUK administrative and survey data.}\\
Notes: & 1. & First-difference estimates. A first-difference is defined as $\Delta x_{t+1}\equiv x_{t+1} - x_{t}$. First-differenced regressands are regressed on categorical and time-variant covariates. Net saving is taken from administrative data and merged with survey data at Year-Month of survey interviews. Head age and literacy are from baseline data. Intercept terms are omitted in estimating equations. Net saving is saving - withdrawal. \\
& 2. & ${}^{***}$, ${}^{**}$, ${}^{*}$ indicate statistical significance at 1\%, 5\%, 10\%, respetively. Standard errors are clustered at group (village) level.
\end{tabular}
\end{minipage}

\hspace{-1cm}\begin{minipage}[t]{14cm}
\hfil\textsc{\normalsize Table \refstepcounter{table}\thetable: FD estimation of net school enrollment, ultra poor vs. moderately poor\label{tab FD enroll2 original HH}}\\
\setlength{\tabcolsep}{1pt}
\setlength{\baselineskip}{8pt}
\renewcommand{\arraystretch}{.55}
\hfil\begin{tikzpicture}
\node (tbl) {\input{c:/data/GUK/analysis/save/Original1600/SchoolingPovertystatusOriginalHHsFDEstimationResults.tex}};
%\input{c:/dropbox/data/ramadan/save/tablecolortemplate.tex}
\end{tikzpicture}\\
\renewcommand{\arraystretch}{.8}
\setlength{\tabcolsep}{1pt}
\begin{tabular}{>{\hfill\scriptsize}p{1cm}<{}>{\hfill\scriptsize}p{.25cm}<{}>{\scriptsize}p{12cm}<{\hfill}}
Source:& \multicolumn{2}{l}{\scriptsize Estimated with GUK administrative and survey data.}\\
Notes: & 1. & First-difference estimates. A first-difference is defined as $\Delta x_{t+1}\equiv x_{t+1} - x_{t}$. First-differenced regressands are regressed on categorical and time-variant covariates. Net saving is taken from administrative data and merged with survey data at Year-Month of survey interviews. Head age and literacy are from baseline data. Intercept terms are omitted in estimating equations. Net saving is saving - withdrawal. \\
& 2. & ${}^{***}$, ${}^{**}$, ${}^{*}$ indicate statistical significance at 1\%, 5\%, 10\%, respetively. Standard errors are clustered at group (village) level.
\end{tabular}
\end{minipage}

\hspace{-1cm}\begin{minipage}[t]{14cm}
\hfil\textsc{\normalsize Table \refstepcounter{table}\thetable: FD estimation of school enrollment, with vs. without a grace period\label{tab FD enroll3 original HH}}\\
\setlength{\tabcolsep}{1pt}
\setlength{\baselineskip}{8pt}
\renewcommand{\arraystretch}{.55}
\hfil\begin{tikzpicture}
\node (tbl) {\input{c:/data/GUK/analysis/save/Original1600/SchoolingGraceOriginalHHsFDEstimationResults.tex}};
%\input{c:/dropbox/data/ramadan/save/tablecolortemplate.tex}
\end{tikzpicture}\\
\renewcommand{\arraystretch}{.8}
\setlength{\tabcolsep}{1pt}
\begin{tabular}{>{\hfill\scriptsize}p{1cm}<{}>{\hfill\scriptsize}p{.25cm}<{}>{\scriptsize}p{12cm}<{\hfill}}
Source:& \multicolumn{2}{l}{\scriptsize Estimated with GUK administrative and survey data.}\\
Notes: & 1. & First-difference estimates. A first-difference is defined as $\Delta x_{t+1}\equiv x_{t+1} - x_{t}$. First-differenced regressands are regressed on categorical and time-variant covariates. Net saving is taken from administrative data and merged with survey data at Year-Month of survey interviews. Head age and literacy are from baseline data. Intercept terms are omitted in estimating equations. Net saving is saving - withdrawal. All dummy interaction terms are first demeaned and then interacted.\\
& 2. & ${}^{***}$, ${}^{**}$, ${}^{*}$ indicate statistical significance at 1\%, 5\%, 10\%, respetively. Standard errors are clustered at group (village) level.
\end{tabular}
\end{minipage}

\hspace{-1cm}\begin{minipage}[t]{14cm}
\hfil\textsc{\normalsize Table \refstepcounter{table}\thetable: FD estimation of school enrollment, small size vs. large size\label{tab FD enroll4 original HH}}\\
\setlength{\tabcolsep}{1pt}
\setlength{\baselineskip}{8pt}
\renewcommand{\arraystretch}{.55}
\hfil\begin{tikzpicture}
\node (tbl) {\input{c:/data/GUK/analysis/save/Original1600/SchoolingSizeOriginalHHsFDEstimationResults.tex}};
%\input{c:/dropbox/data/ramadan/save/tablecolortemplate.tex}
\end{tikzpicture}\\
\renewcommand{\arraystretch}{.8}
\setlength{\tabcolsep}{1pt}
\begin{tabular}{>{\hfill\scriptsize}p{1cm}<{}>{\hfill\scriptsize}p{.25cm}<{}>{\scriptsize}p{12cm}<{\hfill}}
Source:& \multicolumn{2}{l}{\scriptsize Estimated with GUK administrative and survey data.}\\
Notes: & 1. & First-difference estimates. A first-difference is defined as $\Delta x_{t+1}\equiv x_{t+1} - x_{t}$. First-differenced regressands are regressed on categorical and time-variant covariates. Net saving is taken from administrative data and merged with survey data at Year-Month of survey interviews. Head age and literacy are from baseline data. Intercept terms are omitted in estimating equations. Net saving is saving - withdrawal. All dummy interaction terms are first demeaned and then interacted.\\
& 2. & ${}^{***}$, ${}^{**}$, ${}^{*}$ indicate statistical significance at 1\%, 5\%, 10\%, respetively. Standard errors are clustered at group (village) level.
\end{tabular}
\end{minipage}


\hspace{-1cm}\begin{minipage}[t]{14cm}
\hfil\textsc{\normalsize Table \refstepcounter{table}\thetable: FD estimation of school enrollment, round 1 vs. round 4 differences\label{tab FD enroll5 original HH}}\\
\setlength{\tabcolsep}{1pt}
\setlength{\baselineskip}{8pt}
\renewcommand{\arraystretch}{.55}
\hfil\begin{tikzpicture}
\node (tbl) {\input{c:/data/GUK/analysis/save/Original1600/SchoolingRd14DiffOriginalHHsFDEstimationResults.tex}};
%\input{c:/dropbox/data/ramadan/save/tablecolortemplate.tex}
\end{tikzpicture}\\
\renewcommand{\arraystretch}{.8}
\setlength{\tabcolsep}{1pt}
\begin{tabular}{>{\hfill\scriptsize}p{1cm}<{}>{\hfill\scriptsize}p{.25cm}<{}>{\scriptsize}p{12cm}<{\hfill}}
Source:& \multicolumn{2}{l}{\scriptsize Estimated with GUK administrative and survey data.}\\
Notes: & 1. & First-difference estimates. A first-difference is defined as $\Delta x_{t+1}\equiv x_{t+1} - x_{t}$. First-differenced regressands are regressed on categorical and time-variant covariates. Net saving is taken from administrative data and merged with survey data at Year-Month of survey interviews. Head age and literacy are from baseline data. Intercept terms are omitted in estimating equations. Net saving is saving - withdrawal. All dummy interaction terms are first demeaned and then interacted.\\
& 2. & ${}^{***}$, ${}^{**}$, ${}^{*}$ indicate statistical significance at 1\%, 5\%, 10\%, respetively. Standard errors are clustered at group (village) level.
\end{tabular}
\end{minipage}

\hspace{-1cm}\begin{minipage}[t]{14cm}
\hfil\textsc{\normalsize Table \refstepcounter{table}\thetable: FD estimation of school enrollment, round 1 vs. round 4 differences by attributes\label{tab FD enroll5 attributes original HH}}\\
\setlength{\tabcolsep}{1pt}
\setlength{\baselineskip}{8pt}
\renewcommand{\arraystretch}{.55}
\hfil\begin{tikzpicture}
\node (tbl) {\input{c:/data/GUK/analysis/save/Original1600/SchoolingRd14DiffAttributesOriginalHHsFDEstimationResults.tex}};
%\input{c:/dropbox/data/ramadan/save/tablecolortemplate.tex}
\end{tikzpicture}\\
\renewcommand{\arraystretch}{.8}
\setlength{\tabcolsep}{1pt}
\begin{tabular}{>{\hfill\scriptsize}p{1cm}<{}>{\hfill\scriptsize}p{.25cm}<{}>{\scriptsize}p{12cm}<{\hfill}}
Source:& \multicolumn{2}{l}{\scriptsize Estimated with GUK administrative and survey data.}\\
Notes: & 1. & First-difference estimates. A first-difference is defined as $\Delta x_{t+1}\equiv x_{t+1} - x_{t}$. First-differenced regressands are regressed on categorical and time-variant covariates. Net saving is taken from administrative data and merged with survey data at Year-Month of survey interviews. Head age and literacy are from baseline data. Intercept terms are omitted in estimating equations. Net saving is saving - withdrawal. All dummy interaction terms are first demeaned and then interacted.\\
& 2. & ${}^{***}$, ${}^{**}$, ${}^{*}$ indicate statistical significance at 1\%, 5\%, 10\%, respetively. Standard errors are clustered at group (village) level.
\end{tabular}
\end{minipage}

\hspace{-1cm}\begin{minipage}[t]{14cm}
\hfil\textsc{\normalsize Table \refstepcounter{table}\thetable: FD estimation of school enrollment, round 1 vs. round 4 differences, grace period\label{tab FD enroll6 14  original HH}}\\
\setlength{\tabcolsep}{1pt}
\setlength{\baselineskip}{8pt}
\renewcommand{\arraystretch}{.55}
\hfil\begin{tikzpicture}
\node (tbl) {\input{c:/data/GUK/analysis/save/Original1600/SchoolingRd14DiffGraceOriginalHHsFDEstimationResults.tex}};
%\input{c:/dropbox/data/ramadan/save/tablecolortemplate.tex}
\end{tikzpicture}\\
\renewcommand{\arraystretch}{.8}
\setlength{\tabcolsep}{1pt}
\begin{tabular}{>{\hfill\scriptsize}p{1cm}<{}>{\hfill\scriptsize}p{.25cm}<{}>{\scriptsize}p{12cm}<{\hfill}}
Source:& \multicolumn{2}{l}{\scriptsize Estimated with GUK administrative and survey data.}\\
Notes: & 1. & First-difference estimates. A first-difference is defined as $\Delta x_{t+1}\equiv x_{t+1} - x_{t}$. First-differenced regressands are regressed on categorical and time-variant covariates. Net saving is taken from administrative data and merged with survey data at Year-Month of survey interviews. Head age and literacy are from baseline data. Intercept terms are omitted in estimating equations. Net saving is saving - withdrawal. All dummy interaction terms are first demeaned and then interacted.\\
& 2. & ${}^{***}$, ${}^{**}$, ${}^{*}$ indicate statistical significance at 1\%, 5\%, 10\%, respetively. Standard errors are clustered at group (village) level.
\end{tabular}
\end{minipage}


\hspace{-1cm}\begin{minipage}[t]{14cm}
\hfil\textsc{\normalsize Table \refstepcounter{table}\thetable: FD estimation of school enrollment, period interactions\label{tab FD enroll6 original HH}}\\
\setlength{\tabcolsep}{1pt}
\setlength{\baselineskip}{8pt}
\renewcommand{\arraystretch}{.45}
\hfil\begin{tikzpicture}
\node (tbl) {\input{c:/data/GUK/analysis/save/Original1600/SchoolingTIntOriginalHHsFDEstimationResults1.tex}};
\end{tikzpicture}\\
\renewcommand{\arraystretch}{.8}
\setlength{\tabcolsep}{1pt}
\begin{tabular}{>{\hfill\scriptsize}p{1cm}<{}>{\hfill\scriptsize}p{.25cm}<{}>{\scriptsize}p{12cm}<{\hfill}}
Source:& \multicolumn{2}{l}{\scriptsize Estimated with GUK administrative and survey data.}\\
Notes: & 1. & First-difference estimates. A first-difference is defined as $\Delta x_{t+1}\equiv x_{t+1} - x_{t}$. First-differenced regressands are regressed on categorical and time-variant covariates. Net saving is taken from administrative data and merged with survey data at Year-Month of survey interviews. Head age and literacy are from baseline data. Intercept terms are omitted in estimating equations. Net saving is saving - withdrawal. All dummy interaction terms are first demeaned and then interacted.\\
& 2. & ${}^{***}$, ${}^{**}$, ${}^{*}$ indicate statistical significance at 1\%, 5\%, 10\%, respetively. Standard errors are clustered at group (village) level.
\end{tabular}
\end{minipage}

\hspace{-1cm}\begin{minipage}[t]{14cm}
\hfil\textsc{\normalsize Table \ref{tab FD enroll6}: FD estimation of school enrollment, period interactions, continued \label{tab FD enroll6c original HH}}\\
\setlength{\tabcolsep}{1pt}
\setlength{\baselineskip}{8pt}
\renewcommand{\arraystretch}{.45}
\hfil\begin{tikzpicture}
\node (tbl) {\input{c:/data/GUK/analysis/save/Original1600/SchoolingTIntOriginalHHsFDEstimationResults2.tex}};
\end{tikzpicture}\\
\renewcommand{\arraystretch}{.8}
\setlength{\tabcolsep}{1pt}
\begin{tabular}{>{\hfill\scriptsize}p{1cm}<{}>{\hfill\scriptsize}p{.25cm}<{}>{\scriptsize}p{12cm}<{\hfill}}
Source:& \multicolumn{2}{l}{\scriptsize Estimated with GUK administrative and survey data.}\\
Notes: & 1. & First-difference estimates. A first-difference is defined as $\Delta x_{t+1}\equiv x_{t+1} - x_{t}$. First-differenced regressands are regressed on categorical and time-variant covariates. Net saving is taken from administrative data and merged with survey data at Year-Month of survey interviews. Head age and literacy are from baseline data. Intercept terms are omitted in estimating equations. Net saving is saving - withdrawal. All dummy interaction terms are first demeaned and then interacted.\\
& 2. & ${}^{***}$, ${}^{**}$, ${}^{*}$ indicate statistical significance at 1\%, 5\%, 10\%, respetively. Standard errors are clustered at group (village) level.
\end{tabular}
\end{minipage}

\hspace{-1cm}\begin{minipage}[t]{14cm}
\hfil\textsc{\normalsize Table \refstepcounter{table}\thetable: FD estimation of school enrollment, period interactions, grace period\label{tab FD enroll7 original HH}}\\
\setlength{\tabcolsep}{1pt}
\setlength{\baselineskip}{8pt}
\renewcommand{\arraystretch}{.5}
\hfil\begin{tikzpicture}
\node (tbl) {\input{c:/data/GUK/analysis/save/Original1600/SchoolingTIntGraceOriginalHHsFDEstimationResults.tex}};
%\input{c:/dropbox/data/ramadan/save/tablecolortemplate.tex}
\end{tikzpicture}\\
\renewcommand{\arraystretch}{.8}
\setlength{\tabcolsep}{1pt}
\begin{tabular}{>{\hfill\scriptsize}p{1cm}<{}>{\hfill\scriptsize}p{.25cm}<{}>{\scriptsize}p{12cm}<{\hfill}}
Source:& \multicolumn{2}{l}{\scriptsize Estimated with GUK administrative and survey data.}\\
Notes: & 1. & First-difference estimates. A first-difference is defined as $\Delta x_{t+1}\equiv x_{t+1} - x_{t}$. First-differenced regressands are regressed on categorical and time-variant covariates. Net saving is taken from administrative data and merged with survey data at Year-Month of survey interviews. Head age and literacy are from baseline data. Intercept terms are omitted in estimating equations. Net saving is saving - withdrawal. All dummy interaction terms are first demeaned and then interacted.\\
& 2. & ${}^{***}$, ${}^{**}$, ${}^{*}$ indicate statistical significance at 1\%, 5\%, 10\%, respetively. Standard errors are clustered at group (village) level.
\end{tabular}
\end{minipage}


\hspace{-1cm}\begin{minipage}[t]{14cm}
\hfil\textsc{\normalsize Table \refstepcounter{table}\thetable: FD estimation of school enrollment, period interactions, small vs. large sized loans\label{tab FD enroll8 original HH}}\\
\setlength{\tabcolsep}{1pt}
\setlength{\baselineskip}{8pt}
\renewcommand{\arraystretch}{.5}
\hfil\begin{tikzpicture}
\node (tbl) {\input{c:/data/GUK/analysis/save/Original1600/SchoolingTIntSizeOriginalHHsFDEstimationResults.tex}};
%\input{c:/dropbox/data/ramadan/save/tablecolortemplate.tex}
\end{tikzpicture}\\
\renewcommand{\arraystretch}{.8}
\setlength{\tabcolsep}{1pt}
\begin{tabular}{>{\hfill\scriptsize}p{1cm}<{}>{\hfill\scriptsize}p{.25cm}<{}>{\scriptsize}p{12cm}<{\hfill}}
Source:& \multicolumn{2}{l}{\scriptsize Estimated with GUK administrative and survey data.}\\
Notes: & 1. & First-difference estimates. A first-difference is defined as $\Delta x_{t+1}\equiv x_{t+1} - x_{t}$. First-differenced regressands are regressed on categorical and time-variant covariates. Net saving is taken from administrative data and merged with survey data at Year-Month of survey interviews. Head age and literacy are from baseline data. Intercept terms are omitted in estimating equations. Net saving is saving - withdrawal. All dummy interaction terms are first demeaned and then interacted.\\
& 2. & ${}^{***}$, ${}^{**}$, ${}^{*}$ indicate statistical significance at 1\%, 5\%, 10\%, respetively. Standard errors are clustered at group (village) level.
\end{tabular}
\end{minipage}


\subsection{Assets}

Assets reportd in rd 1 is too small, indicating possible errors or different way of reporting only in rd 1. So we also examine rd 2 vs. rd 4 differences (\textsf{as3, as4}).



\begin{Schunk}
\begin{Sinput}
source(paste0(pathprogram, "ReadTrimAssetOriginalHHsFDData.R"))
\end{Sinput}
\begin{Soutput}
Dropped 2804 obs due to NA.
Dropped 4027 obs due to NA.
Dropped 2804 obs due to NA.
Dropped 4027 obs due to NA.
Dropped 2039 obs due to NA.
Dropped 2040 obs due to NA.
Dropped 2039 obs due to NA.
Dropped 2040 obs due to NA.
\end{Soutput}
\end{Schunk}

Main assets are household assets (\textsf{HAssetAmount}) and production assets (\textsf{PAssetAmount}) both with 4973 observations. After first-differencing, they become 3595 observations, with 21, 94, 3480 households observed for 2, 3, 4 times. We also examine rd 2 vs. rd 4 differences, which has 2389 observations. After first-differencing, they become 1161 observations.


\begin{Schunk}
\begin{Sinput}
FileName <- "Asset"
FileNameHeader <- paste0(c("", "Grace", "PovertyStatus", "Size", "Attributes",
  "Rd24Diff", "Rd24Grace", "Rd24DiffAttributes"), "OriginalHHs")
Assuffixes <- c("", "G", "P", "S", "a", "D", "DG", "Da")
listheader <- paste0("as", Assuffixes)
DataToUse1 <- c(rep("das1d", 3), "das1Rd", rep("das2d", 3), "das2Rd")
DataToUse2 <- c(rep("das3d", 3), "das3Rd", rep("das4d", 3), "das4Rd")
Regressands <- c(rep("HAssetAmount", 4), rep("PAssetAmount", 4))
Addseparatingcols = 4; Separatingcolwidth = .2
Separatingcoltitle = c("Household asset amount (Tk)", "Productive asset amount (Tk)")
\end{Sinput}
\end{Schunk}
\begin{Schunk}
\begin{Sinput}
source(paste0(pathprogram, "AssetCovariateSelection.R"))
\end{Sinput}
\end{Schunk}
\begin{Schunk}
\begin{Sinput}
exclheader <- paste0("excl", Assuffixes)
source(paste0(pathprogram, "FDEstimationFile.R"))
\end{Sinput}
\end{Schunk}
\begin{Schunk}
\begin{Sinput}
#dummy chunk
\end{Sinput}
\end{Schunk}

\hspace{-1cm}\begin{minipage}[t]{14cm}
\hfil\textsc{\normalsize Table \refstepcounter{table}\thetable: FD estimation of assets\label{tab FD assets original HH}}\\
\setlength{\tabcolsep}{1pt}
\setlength{\baselineskip}{8pt}
\renewcommand{\arraystretch}{.55}
\hfil\begin{tikzpicture}
\node (tbl) {\input{c:/data/GUK/analysis/save/Original1600/AssetOriginalHHsFDEstimationResults.tex}};
%\input{c:/dropbox/data/ramadan/save/tablecolortemplate.tex}
\end{tikzpicture}\\
\renewcommand{\arraystretch}{.8}
\setlength{\tabcolsep}{1pt}
\begin{tabular}{>{\hfill\scriptsize}p{1cm}<{}>{\hfill\scriptsize}p{.25cm}<{}>{\scriptsize}p{12cm}<{\hfill}}
Source:& \multicolumn{2}{l}{\scriptsize Estimated with GUK administrative and survey data.}\\
Notes: & 1. & First-difference estimates. A first-difference is defined as $\Delta x_{t+k}\equiv x_{t+k} - x_{t}$  for $k=1, 2, \dots$. Saving and repayment misses are taken from administrative data and merged with survey data at Year-Month of survey interviews. Intercept terms are omitted in estimating equations. Sample is continuing members and replacing members of early rejecters and received loans prior to 2015 Janunary. Household assets do not include livestock. Regressions (1)-(3), (5)-(6) use only arm and calendar information. (4) and (7) use previous six month repayment and saving information which is lacking in rd 1, hence starts from rd 2.\\
& 2. & ${}^{***}$, ${}^{**}$, ${}^{*}$ indicate statistical significance at 1\%, 5\%, 10\%, respetively. Standard errors are clustered at group (village) level.
\end{tabular}
\end{minipage}

\hspace{-1cm}\begin{minipage}[t]{14cm}
\hfil\textsc{\normalsize Table \refstepcounter{table}\thetable: FD estimation of assets by attributes\label{tab FD assets attributes original HH}}\\
\setlength{\tabcolsep}{1pt}
\setlength{\baselineskip}{8pt}
\renewcommand{\arraystretch}{.55}
\hfil\begin{tikzpicture}
\node (tbl) {\input{c:/data/GUK/analysis/save/Original1600/AssetAttributesOriginalHHsFDEstimationResults.tex}};
%\input{c:/dropbox/data/ramadan/save/tablecolortemplate.tex}
\end{tikzpicture}\\
\renewcommand{\arraystretch}{.8}
\setlength{\tabcolsep}{1pt}
\begin{tabular}{>{\hfill\scriptsize}p{1cm}<{}>{\hfill\scriptsize}p{.25cm}<{}>{\scriptsize}p{12cm}<{\hfill}}
Source:& \multicolumn{2}{l}{\scriptsize Estimated with GUK administrative and survey data.}\\
Notes: & 1. & First-difference estimates. A first-difference is defined as $\Delta x_{t+k}\equiv x_{t+k} - x_{t}$  for $k=1, 2, \dots$. Saving and repayment misses are taken from administrative data and merged with survey data at Year-Month of survey interviews. Intercept terms are omitted in estimating equations. Sample is continuing members and replacing members of early rejecters and received loans prior to 2015 Janunary. Household assets do not include livestock. Regressions (1)-(3), (5)-(6) use only arm and calendar information. (4) and (7) use previous six month repayment and saving information which is lacking in rd 1, hence starts from rd 2.\\
& 2. & ${}^{***}$, ${}^{**}$, ${}^{*}$ indicate statistical significance at 1\%, 5\%, 10\%, respetively. Standard errors are clustered at group (village) level.
\end{tabular}
\end{minipage}

\hspace{-1cm}\begin{minipage}[t]{14cm}
\hfil\textsc{\normalsize Table \refstepcounter{table}\thetable: FD estimation of assets, moderately poor vs. ultra poor\label{tab FD assets2 original HH}}\\
\setlength{\tabcolsep}{1pt}
\setlength{\baselineskip}{8pt}
\renewcommand{\arraystretch}{.55}
\hfil\begin{tikzpicture}
\node (tbl) {\input{c:/data/GUK/analysis/save/Original1600/AssetPovertyStatusOriginalHHsFDEstimationResults.tex}};
%\input{c:/dropbox/data/ramadan/save/tablecolortemplate.tex}
\end{tikzpicture}\\
\renewcommand{\arraystretch}{.8}
\setlength{\tabcolsep}{1pt}
\begin{tabular}{>{\hfill\scriptsize}p{1cm}<{}>{\hfill\scriptsize}p{.25cm}<{}>{\scriptsize}p{12cm}<{\hfill}}
Source:& \multicolumn{2}{l}{\scriptsize Estimated with GUK administrative and survey data.}\\
Notes: & 1. & First-difference estimates. A first-difference is defined as $\Delta x_{t+k}\equiv x_{t+k} - x_{t}$  for $k=1, 2, \dots$. Saving and repayment misses are taken from administrative data and merged with survey data at Year-Month of survey interviews. Intercept terms are omitted in estimating equations. Sample is continuing members and replacing members of early rejecters and received loans prior to 2015 Janunary. Household assets do not include livestock. \\
& 2. & ${}^{***}$, ${}^{**}$, ${}^{*}$ indicate statistical significance at 1\%, 5\%, 10\%, respetively. Standard errors are clustered at group (village) level.
\end{tabular}
\end{minipage}

\hspace{-1cm}\begin{minipage}[t]{14cm}
\hfil\textsc{\normalsize Table \refstepcounter{table}\thetable: FD estimation of assets, with vs. without a grace period\label{tab FD assets3 original HH}}\\
\setlength{\tabcolsep}{1pt}
\setlength{\baselineskip}{8pt}
\renewcommand{\arraystretch}{.55}
\hfil\begin{tikzpicture}
\node (tbl) {\input{c:/data/GUK/analysis/save/Original1600/AssetGraceOriginalHHsFDEstimationResults.tex}};
%\input{c:/dropbox/data/ramadan/save/tablecolortemplate.tex}
\end{tikzpicture}\\
\renewcommand{\arraystretch}{.8}
\setlength{\tabcolsep}{1pt}
\begin{tabular}{>{\hfill\scriptsize}p{1cm}<{}>{\hfill\scriptsize}p{.25cm}<{}>{\scriptsize}p{12cm}<{\hfill}}
Source:& \multicolumn{2}{l}{\scriptsize Estimated with GUK administrative and survey data.}\\
Notes: & 1. & First-difference estimates. A first-difference is defined as $\Delta x_{t+k}\equiv x_{t+k} - x_{t}$  for $k=1, 2, \dots$. Saving and repayment misses are taken from administrative data and merged with survey data at Year-Month of survey interviews. Intercept terms are omitted in estimating equations. Sample is continuing members and replacing members of early rejecters and received loans prior to 2015 Janunary. Household assets do not include livestock. \\
& 2. & ${}^{***}$, ${}^{**}$, ${}^{*}$ indicate statistical significance at 1\%, 5\%, 10\%, respetively. Standard errors are clustered at group (village) level.
\end{tabular}
\end{minipage}

\hspace{-1cm}\begin{minipage}[t]{14cm}
\hfil\textsc{\normalsize Table \refstepcounter{table}\thetable: FD estimation of assets, small vs. large size loans\label{tab FD assets4 original HH}}\\
\setlength{\tabcolsep}{1pt}
\setlength{\baselineskip}{8pt}
\renewcommand{\arraystretch}{.55}
\hfil\begin{tikzpicture}
\node (tbl) {\input{c:/data/GUK/analysis/save/Original1600/AssetSizeOriginalHHsFDEstimationResults.tex}};
%\input{c:/dropbox/data/ramadan/save/tablecolortemplate.tex}
\end{tikzpicture}\\
\renewcommand{\arraystretch}{.8}
\setlength{\tabcolsep}{1pt}
\begin{tabular}{>{\hfill\scriptsize}p{1cm}<{}>{\hfill\scriptsize}p{.25cm}<{}>{\scriptsize}p{12cm}<{\hfill}}
Source:& \multicolumn{2}{l}{\scriptsize Estimated with GUK administrative and survey data.}\\
Notes: & 1. & First-difference estimates. A first-difference is defined as $\Delta x_{t+k}\equiv x_{t+k} - x_{t}$  for $k=1, 2, \dots$. Saving and repayment misses are taken from administrative data and merged with survey data at Year-Month of survey interviews. Intercept terms are omitted in estimating equations. Sample is continuing members and replacing members of early rejecters and received loans prior to 2015 Janunary. Household assets do not include livestock. \\
& 2. & ${}^{***}$, ${}^{**}$, ${}^{*}$ indicate statistical significance at 1\%, 5\%, 10\%, respetively. Standard errors are clustered at group (village) level.
\end{tabular}
\end{minipage}

\hspace{-1cm}\begin{minipage}[t]{14cm}
\hfil\textsc{\normalsize Table \refstepcounter{table}\thetable: FD estimation of assets, round 2 and 4 comparison\label{tab FD assets rd24 original HH}}\\
\setlength{\tabcolsep}{1pt}
\setlength{\baselineskip}{8pt}
\renewcommand{\arraystretch}{.55}
\hfil\begin{tikzpicture}
\node (tbl) {\input{c:/data/GUK/analysis/save/Original1600/AssetRd24DiffOriginalHHsFDEstimationResults.tex}};
%\input{c:/dropbox/data/ramadan/save/tablecolortemplate.tex}
\end{tikzpicture}\\
\renewcommand{\arraystretch}{.8}
\setlength{\tabcolsep}{1pt}
\begin{tabular}{>{\hfill\scriptsize}p{1cm}<{}>{\hfill\scriptsize}p{.25cm}<{}>{\scriptsize}p{12cm}<{\hfill}}
Source:& \multicolumn{2}{l}{\scriptsize Estimated with GUK administrative and survey data.}\\
Notes: & 1. & First-difference estimates between round 2 and 4. A first-difference is defined as $\Delta x_{t+k}\equiv x_{t+k} - x_{t}$ for $k=1, 2, \dots$. Saving and repayment misses are taken from administrative data and merged with survey data at Year-Month of survey interviews. Intercept terms are omitted in estimating equations. Sample is continuing members and replacing members of early rejecters and received loans prior to 2015 Janunary. Household assets do not include livestock. Regressions (1)-(3), (5)-(6) use only arm and calendar information. (4) and (7) use previous six month repayment and saving information which is lacking in rd 1, hence starts from rd 2.\\
& 2. & ${}^{***}$, ${}^{**}$, ${}^{*}$ indicate statistical significance at 1\%, 5\%, 10\%, respetively. Standard errors are clustered at group (village) level.
\end{tabular}
\end{minipage}

\hspace{-1cm}\begin{minipage}[t]{14cm}
\hfil\textsc{\normalsize Table \refstepcounter{table}\thetable: FD estimation of assets, round 2 and 4 comparison, grace period\label{tab FD assets rd24 grace original HH}}\\
\setlength{\tabcolsep}{1pt}
\setlength{\baselineskip}{8pt}
\renewcommand{\arraystretch}{.55}
\hfil\begin{tikzpicture}
\node (tbl) {\input{c:/data/GUK/analysis/save/Original1600/AssetRd24GraceOriginalHHsFDEstimationResults.tex}};
%\input{c:/dropbox/data/ramadan/save/tablecolortemplate.tex}
\end{tikzpicture}\\
\renewcommand{\arraystretch}{.8}
\setlength{\tabcolsep}{1pt}
\begin{tabular}{>{\hfill\scriptsize}p{1cm}<{}>{\hfill\scriptsize}p{.25cm}<{}>{\scriptsize}p{12cm}<{\hfill}}
Source:& \multicolumn{2}{l}{\scriptsize Estimated with GUK administrative and survey data.}\\
Notes: & 1. & First-difference estimates between round 2 and 4. A first-difference is defined as $\Delta x_{t+k}\equiv x_{t+k} - x_{t}$ for $k=1, 2, \dots$. Saving and repayment misses are taken from administrative data and merged with survey data at Year-Month of survey interviews. Intercept terms are omitted in estimating equations. Sample is continuing members and replacing members of early rejecters and received loans prior to 2015 Janunary. Household assets do not include livestock. Regressions (1)-(3), (5)-(6) use only arm and calendar information. (4) and (7) use previous six month repayment and saving information which is lacking in rd 1, hence starts from rd 2.\\
& 2. & ${}^{***}$, ${}^{**}$, ${}^{*}$ indicate statistical significance at 1\%, 5\%, 10\%, respetively. Standard errors are clustered at group (village) level.
\end{tabular}
\end{minipage}



\begin{Schunk}
\begin{Sinput}
FileNameHeader <- paste0(FileNameHeader, "Robustness")
listheader <- paste0("as", Assuffixes)
exclheader <- paste0("excl", Assuffixes)
DataToUse1 <- DataToUse2 <- c(rep("das1d", 3), rep("das2d", 3))
Regressands <- c(rep("HAssetAmount", 3), rep("PAssetAmount", 3))
Addseparatingcols = 3; Separatingcolwidth = .2
Separatingcoltitle = c("Household asset amount (Tk)", "Productive asset amount (Tk)")
source(paste0(pathprogram, "FDEstimationFile.R"))
\end{Sinput}
\end{Schunk}

Robustness: To understand underlying pattern of asset accumulation, we compare the loan recipients and loan rejecters. This distinction is made by households by choice, so the indicator variable is considered to be endogenous to asset level. This is a limitation, however, it has its own merit in giving an idea how loan recipients faired during the study period relative to loan nonrecipients. \textsc{\small Table \ref{tab FD assets pure control original HHs}} shows that the pure controls also experience similar increase-increase-decrease pattern for household assets. This suggests the pattern observed among the loan recipients may be a systemic pattern of the area, not necessarily reflecting the repayment burdern. This partially relieves a concern that repayment burden was excessive for loan recipients.

\hspace{-1cm}\begin{minipage}[t]{14cm}
\hfil\textsc{\normalsize Table \refstepcounter{table}\thetable: FD estimation of assets, loan recipients vs. pure control\label{tab FD assets pure control original HHs}}\\
\setlength{\tabcolsep}{1pt}
\setlength{\baselineskip}{8pt}
\renewcommand{\arraystretch}{.55}
\hfil\begin{tikzpicture}
\node (tbl) {\input{c:/data/GUK/analysis/save/Original1600/AssetOriginalHHsRobustnessFDEstimationResults.tex}};
%\input{c:/dropbox/data/ramadan/save/tablecolortemplate.tex}
\end{tikzpicture}\\
\renewcommand{\arraystretch}{.8}
\setlength{\tabcolsep}{1pt}
\begin{tabular}{>{\hfill\scriptsize}p{1cm}<{}>{\hfill\scriptsize}p{.25cm}<{}>{\scriptsize}p{12cm}<{\hfill}}
Source:& \multicolumn{2}{l}{\scriptsize Estimated with GUK administrative and survey data.}\\
Notes: & 1. & First-difference estimates between round 2 and 4. A first-difference is defined as $\Delta x_{t+k}\equiv x_{t+k} - x_{t}$ for $k=1, 2, \dots$. Saving and repayment misses are taken from administrative data and merged with survey data at Year-Month of survey interviews. Pure control is members not receiving loans while they were put on a wait list. 
Sample is continuing members and replacing members of early rejecters. Household assets do not include livestock. Regressions (1)-(2), (4)-(5) use only arm and calendar information. (3) and (6) information if the household was exposed to the flood in round 1. Pure controls are households who rejected to receive a loan.\\
& 2. & ${}^{***}$, ${}^{**}$, ${}^{*}$ indicate statistical significance at 1\%, 5\%, 10\%, respetively. Standard errors are clustered at group (village) level.
\end{tabular}
\end{minipage}

\begin{palepinkleftbar}
\begin{finding}
\textsc{\small Table \ref{tab FD assets original HH}} (1) shows household assets increase after receiving the loans in all arms. Total incremant is largest among the \textsf{large grace} arm as indicated in (2). In (3), increments are positive in rd 2 - 3, suggesting substantial purchase after receiving a loan. Significant decreases in rd 3 - 4 for all arms indicate liquidation of assets for repayment. Productive assets of large size loan arms decrease in rd 3 - 4 in \textsc{\small Table \ref{tab FD assets3 original HH}} (6). These may indicate forced liquidation for repayment, which can entail efficiency losses.
\end{finding}
\end{palepinkleftbar}


\subsection{Livestock}


\begin{Schunk}
\begin{Sinput}
lvo <-  readRDS(paste0(pathsaveHere, "RosterLivestockAdminOriginalHHsDataUsedForEstimation.rds"))
lvo[, grepout("Loan|UD|Forced", colnames(lvo)) := NULL]
lvostrings <- "^groupid$|hhid|^Arm$|tee|^dummy[TLCMUWSN]|^TotalIm|Floo|Time\\.|Head"
lvoR <- lvo[(Fromxid), grepout(paste0(lvostrings, "|RM"), colnames(lvo)), with = F]
lvo <- lvo[(Fromxid), grepout(lvostrings, colnames(lvo)), with = F]
lvo3 <- lvo[tee == 2 | tee == 4, ]
lvoR3 <- lvoR[tee == 2 | tee == 4, ]
datas <- c("lvo", "lvoR", "lvo3", "lvoR3")
ddatas <- paste0("d", datas)
ddatasd <- paste0(ddatas, "d")
for (i in 1:length(datas)) {
#   dl <- prepFDData(get(datas[i]), Group = "^hhid$", TimeVar = "tee", Cluster = "groupid", 
#     LevelCovariates = "^dummy[A-Z].*[a-z]$|^Arm|Floo|^Time\\..$", 
#     drop.if.NA.in.differencing = T, LevelPeriodToKeep = "last",
#     use.var.name.for.dummy.prefix = F, print.messages = F)
dl <- FirstDiffPanelData(get(datas[i]), 
  Group = "^hhid$", TimeVar = "tee", Cluster = "groupid",
  LevelCovariates = "^dummy|Head|^Time\\..$|Female$|Floo|Eldest|^Arm|^cred.*s$|xid$|Sch.*Pa")
  dat <- dl$diff
  dat[, grepout("^en$", colnames(dat)) := NULL]
  # Recreate Time.4 which is dropped when kept only 1:(T-1) obs.
  dat[, grepout("Time.?2", colnames(dat)) := NULL]
  assign(ddatas[i], dl)
  assign(ddatasd[i], dat)
}
\end{Sinput}
\begin{Soutput}
Dropped 2807 obs due to NA.
Dropped 4031 obs due to NA.
Dropped 2041 obs due to NA.
Dropped 2042 obs due to NA.
\end{Soutput}
\begin{Sinput}
dlvoRd <- dlvoRd[tee > 2, ]
\end{Sinput}
\end{Schunk}
\begin{Schunk}
\begin{Sinput}
source(paste0(pathprogram, "ReadTrimLivestockFDData.R"))
\end{Sinput}
\begin{Soutput}
Dropped 196 obs due to T<2.
Dropped 1402 obs due to NA.
Dropped 196 obs due to T<2.
Dropped 3080 obs due to NA.
Dropped 154 obs due to T<2.
Dropped 1272 obs due to NA.
Dropped 154 obs due to T<2.
Dropped 1386 obs due to NA.
\end{Soutput}
\end{Schunk}
\begin{Schunk}
\begin{Sinput}
FileName <- "Livestock"
FileNameHeader <- paste0(c("", "Grace", "PovertyStatus", "Size", "Attributes",
    "TInt", "TIntGrace", "TIntSize", "Rd14Diff", "Rd14DiffGrace", "Rd14DiffAttributes"),
     "OriginalHHs")
Lvsuffixes <- c("", "G", "P", "S", "a", "T", "TG", "TS", "D", "DG", "Da")
listheader <- paste0("lv", Lvsuffixes)
DataToUse1 <- rep("dlvod", 6)
DataToUse2 <- rep("dlvo3d", 6)
tableboxwidth <- 4.5
Regressands <- rep("TotalImputedValue", 6)
Addseparatingcols <- NULL; Separatingcolwidth <- NULL
Separatingcoltitle <- NULL
\end{Sinput}
\end{Schunk}
\begin{Schunk}
\begin{Sinput}
source(paste0(pathprogram, "LivestockCovariateSelection.R"))
\end{Sinput}
\end{Schunk}
\begin{Schunk}
\begin{Sinput}
exclheader <- paste0("excl", Lvsuffixes)
source(paste0(pathprogram, "FDEstimationFile.R"))
\end{Sinput}
\end{Schunk}



\hspace{-1cm}\begin{minipage}[t]{14cm}
\hfil\textsc{\normalsize Table \refstepcounter{table}\thetable: FD estimation of livestock holding values\label{tab FD livestock original HH}}\\
\setlength{\tabcolsep}{1pt}
\setlength{\baselineskip}{8pt}
\renewcommand{\arraystretch}{.55}
\hfil\begin{tikzpicture}
\node (tbl) {\input{c:/data/GUK/analysis/save/Original1600/LivestockOriginalHHsFDEstimationResults.tex}};
%\input{c:/dropbox/data/ramadan/save/tablecolortemplate.tex}
\end{tikzpicture}\\
\renewcommand{\arraystretch}{.8}
\setlength{\tabcolsep}{1pt}
\begin{tabular}{>{\hfill\scriptsize}p{1cm}<{}>{\hfill\scriptsize}p{.25cm}<{}>{\scriptsize}p{12cm}<{\hfill}}
Source:& \multicolumn{2}{l}{\scriptsize Estimated with GUK administrative and survey data.}\\
Notes: & 1. & First-difference estimates. A first-difference is defined as $\Delta x_{t+1}\equiv x_{t+1} - x_{t}$. Saving and repayment misses are taken from administrative data and merged with survey data at Year-Month of survey interviews. Intercept terms are omitted in estimating equations. Sample is continuing members and replacing members of early rejecters and received loans prior to 2015 Janunary. Regressand is \textsf{TotalImputedValue}, a sum of all livestock holding values evaluated at respective median market prices in the same year. \\
& 2. & ${}^{***}$, ${}^{**}$, ${}^{*}$ indicate statistical significance at 1\%, 5\%, 10\%, respetively. Standard errors are clustered at group (village) level.
\end{tabular}
\end{minipage}

\hspace{-1cm}\begin{minipage}[t]{14cm}
\hfil\textsc{\normalsize Table \refstepcounter{table}\thetable: FD estimation of livestock holding values by attributes\label{tab FD livestock attributes original HH}}\\
\setlength{\tabcolsep}{1pt}
\setlength{\baselineskip}{8pt}
\renewcommand{\arraystretch}{.55}
\hfil\begin{tikzpicture}
\node (tbl) {\input{c:/data/GUK/analysis/save/Original1600/LivestockAttributesOriginalHHsFDEstimationResults.tex}};
%\input{c:/dropbox/data/ramadan/save/tablecolortemplate.tex}
\end{tikzpicture}\\
\renewcommand{\arraystretch}{.8}
\setlength{\tabcolsep}{1pt}
\begin{tabular}{>{\hfill\scriptsize}p{1cm}<{}>{\hfill\scriptsize}p{.25cm}<{}>{\scriptsize}p{12cm}<{\hfill}}
Source:& \multicolumn{2}{l}{\scriptsize Estimated with GUK administrative and survey data.}\\
Notes: & 1. & First-difference estimates. A first-difference is defined as $\Delta x_{t+1}\equiv x_{t+1} - x_{t}$. Saving and repayment misses are taken from administrative data and merged with survey data at Year-Month of survey interviews. Intercept terms are omitted in estimating equations. Sample is continuing members and replacing members of early rejecters and received loans prior to 2015 Janunary. Regressand is \textsf{TotalImputedValue}, a sum of all livestock holding values evaluated at respective median market prices in the same year. \\
& 2. & ${}^{***}$, ${}^{**}$, ${}^{*}$ indicate statistical significance at 1\%, 5\%, 10\%, respetively. Standard errors are clustered at group (village) level.
\end{tabular}
\end{minipage}



\hspace{-1cm}\begin{minipage}[t]{14cm}
\hfil\textsc{\normalsize Table \refstepcounter{table}\thetable: FD estimation of livestock holding values, ultra vs. moderately poor\label{tab FD livestock poor original HH}}\\
\setlength{\tabcolsep}{1pt}
\setlength{\baselineskip}{8pt}
\renewcommand{\arraystretch}{.55}
\hfil\begin{tikzpicture}
\node (tbl) {\input{c:/data/GUK/analysis/save/Original1600/LivestockPovertyStatusOriginalHHsFDEstimationResults.tex}};
%\input{c:/dropbox/data/ramadan/save/tablecolortemplate.tex}
\end{tikzpicture}\\
\renewcommand{\arraystretch}{.8}
\setlength{\tabcolsep}{1pt}
\begin{tabular}{>{\hfill\scriptsize}p{1cm}<{}>{\hfill\scriptsize}p{.25cm}<{}>{\scriptsize}p{12cm}<{\hfill}}
Source:& \multicolumn{2}{l}{\scriptsize Estimated with GUK administrative and survey data.}\\
Notes: & 1. & First-difference estimates. A first-difference is defined as $\Delta x_{t+1}\equiv x_{t+1} - x_{t}$. Saving and repayment misses are taken from administrative data and merged with survey data at Year-Month of survey interviews. Intercept terms are omitted in estimating equations. Sample is continuing members and replacing members of early rejecters and received loans prior to 2015 Janunary. Regressand is \textsf{TotalImputedValue}, a sum of all livestock holding values evaluated at respective median market prices in the same year. \\
& 2. & ${}^{***}$, ${}^{**}$, ${}^{*}$ indicate statistical significance at 1\%, 5\%, 10\%, respetively. Standard errors are clustered at group (village) level.
\end{tabular}
\end{minipage}

\hspace{-1cm}\begin{minipage}[t]{14cm}
\hfil\textsc{\normalsize Table \refstepcounter{table}\thetable: FD estimation of livestock holding values, rd 1 vs. rd 4 comparison\label{tab FD livestock3 original HH}}\\
\setlength{\tabcolsep}{1pt}
\setlength{\baselineskip}{8pt}
\renewcommand{\arraystretch}{.55}
\hfil\begin{tikzpicture}
\node (tbl) {\input{c:/data/GUK/analysis/save/Original1600/LivestockTIntSizeOriginalHHsFDEstimationResults.tex}};
%\input{c:/dropbox/data/ramadan/save/tablecolortemplate.tex}
\end{tikzpicture}\\
\renewcommand{\arraystretch}{.8}
\setlength{\tabcolsep}{1pt}
\begin{tabular}{>{\hfill\scriptsize}p{1cm}<{}>{\hfill\scriptsize}p{.25cm}<{}>{\scriptsize}p{12cm}<{\hfill}}
Source:& \multicolumn{2}{l}{\scriptsize Estimated with GUK administrative and survey data.}\\
Notes: & 1. & First-difference estimates. A first-difference is defined as $\Delta x_{t+1}\equiv x_{t+1} - x_{t}$. Saving and repayment misses are taken from administrative data and merged with survey data at Year-Month of survey interviews. Intercept terms are omitted in estimating equations. Sample is continuing members and replacing members of early rejecters and received loans prior to 2015 Janunary. Regressand is \textsf{TotalImputedValue}, a sum of all livestock holding values evaluated at respective median market prices in the same year. \\
& 2. & ${}^{***}$, ${}^{**}$, ${}^{*}$ indicate statistical significance at 1\%, 5\%, 10\%, respetively. Standard errors are clustered at group (village) level.
\end{tabular}
\end{minipage}

Check quickly if the estimated results make sense.
\begin{Schunk}
\begin{Sinput}
lvo <-  readRDS(paste0(pathsaveHere, "LivestockAdminDataUsedForEstimation.rds"))
setkey(lvo, Arm, tee)
lvostat <- lvo[grepl("es", creditstatus),.(MeanIV = mean(TotalImputedValue, na.rm = T), 
  StdIV = var(TotalImputedValue, na.rm = T)^(.5), 
  N = sum(!is.na(TotalImputedValue))), by = .(Arm, tee)]
lvostat[, c("ciLB", "ciUB") := list(MeanIV - StdIV * qt(.975, N- 1), MeanIV + StdIV * qt(.975, N- 1))]
setkey(lvo, hhid, survey)
lvo[hhid %in% hhid[TotalImputedValue > 100000],
  .(hhid, Arm, Year, LivestockCode, number_owned, 
  mrkt_value, TotalImputedValue)]
\end{Sinput}
\begin{Soutput}
       hhid         Arm Year LivestockCode number_owned mrkt_value
 1: 7020319       large 2012        cow/ox            2          0
 2: 7020319       large 2014        cow/ox            5      18000
 3: 7020319       large 2015        cow/ox            6      19000
 4: 7020319       large 2017        cow/ox            5      25000
 5: 7020614 large grace 2012                          0          0
 6: 7020614 large grace 2014        cow/ox            2      16000
 7: 7020614 large grace 2015        cow/ox            5      16000
 8: 7020614 large grace 2017        cow/ox            6      24000
 9: 7021003 large grace 2012        cow/ox            1          0
10: 7021003 large grace 2014        cow/ox            8      18000
11: 7021003 large grace 2015        cow/ox            4      20000
12: 7021003 large grace 2017        cow/ox            4      23000
13: 7021012 large grace 2012  Chicken/duck            4          0
14: 7021012 large grace 2014        cow/ox            2      24000
15: 7021012 large grace 2015        cow/ox            3      19000
16: 7021012 large grace 2017        cow/ox            8      25000
17: 7021216         cow 2012        cow/ox            6          0
18: 7021216         cow 2014        cow/ox            5      20000
19: 7021216         cow 2015        cow/ox            3      18000
20: 7021216         cow 2017        cow/ox            3      30000
21: 7031706       large 2012        cow/ox            1          0
22: 7031706       large 2014        cow/ox            7      12000
23: 7031706       large 2015        cow/ox            3      15000
24: 7031706       large 2017        cow/ox            3      38000
25: 7031715       large 2012        cow/ox            2          0
26: 7031715       large 2014        cow/ox            9      15000
27: 7031715       large 2015        cow/ox            8      16000
28: 7031715       large 2017        cow/ox            1      30000
29: 7031716       large 2012        cow/ox            1          0
30: 7031716       large 2014        cow/ox            6      16000
31: 7031716       large 2015        cow/ox            5      17000
32: 7031716       large 2017        cow/ox            2      42000
33: 7031905       large 2012        cow/ox            4          0
34: 7031905       large 2014        cow/ox            5      16000
35: 7031905       large 2015        cow/ox            7      20000
36: 7031905       large 2017        cow/ox            7      20000
37: 7042017       large 2012                          0          0
38: 7042017       large 2014        cow/ox            3      18000
39: 7042017       large 2015        cow/ox            6      20000
40: 7042017       large 2017        cow/ox            4      20000
41: 7054005 large grace 2012                          0          0
42: 7054005 large grace 2014        cow/ox            4      18000
43: 7054005 large grace 2015        cow/ox            2      16000
44: 7054005 large grace 2017        cow/ox           10      15000
45: 7054012 large grace 2012        cow/ox            4          0
46: 7054012 large grace 2014        cow/ox           15      20000
47: 7054012 large grace 2015        cow/ox           12      16000
48: 7054012 large grace 2017        cow/ox           10      22000
49: 7085916         cow 2012                          0          0
50: 7085916         cow 2014                         NA         NA
51: 7085916         cow 2015        cow/ox            2      18000
52: 7085916         cow 2017        cow/ox            6      20000
53: 7096202       large 2012        cow/ox            4          0
54: 7096202       large 2014        cow/ox            2      10000
55: 7096202       large 2015        cow/ox            8      14000
56: 7096202       large 2017        cow/ox            9      20000
57: 7096207       large 2012        cow/ox            1          0
58: 7096207       large 2014        cow/ox            6      12000
59: 7096207       large 2015        cow/ox            7      22000
60: 7096207       large 2017        cow/ox            6      16000
61: 7096218       large 2012        cow/ox            1          0
62: 7096218       large 2014        cow/ox            9      16000
63: 7096218       large 2015        cow/ox            7      16000
64: 7096218       large 2017        cow/ox            6      20000
65: 7106408         cow 2012        cow/ox            2          0
66: 7106408         cow 2014        cow/ox            3      15000
67: 7106408         cow 2016        cow/ox            7      14500
68: 7137207 traditional 2012                          0          0
69: 7137207 traditional 2014        cow/ox            1      16000
70: 7137207 traditional 2015        cow/ox            6      14000
71: 7137207 traditional 2017        cow/ox            6      16000
72: 8169519 large grace 2012  Chicken/duck            4          0
73: 8169519 large grace 2014        cow/ox            1      20000
74: 8169519 large grace 2015        cow/ox            6      25000
75: 8169519 large grace 2017        cow/ox            3      40000
76: 8169619       large 2012  Chicken/duck            4          0
77: 8169619       large 2014        cow/ox            3      16000
78: 8169619       large 2016        cow/ox            6      18000
79: 8169619       large 2017        cow/ox            6      38000
       hhid         Arm Year LivestockCode number_owned mrkt_value
    TotalImputedValue
 1:             40000
 2:            100000
 3:            120000
 4:            100000
 5:                 0
 6:             40000
 7:            100000
 8:            120000
 9:             20000
10:            160000
11:             80000
12:             80000
13:                 0
14:             40000
15:             60000
16:            160000
17:            120000
18:            100000
19:             60000
20:             60000
21:             20000
22:            140000
23:             60000
24:             60000
25:             40000
26:            180000
27:            160000
28:             20000
29:             20000
30:            120000
31:            100000
32:             40000
33:             80000
34:            100000
35:            140000
36:            140000
37:                 0
38:             60000
39:            120000
40:             80000
41:                 0
42:             80000
43:             40000
44:            200000
45:             80000
46:            300000
47:            240000
48:            200000
49:                 0
50:                 0
51:             40000
52:            120000
53:             80000
54:             40000
55:            160000
56:            180000
57:             20000
58:            120000
59:            140000
60:            120000
61:             20000
62:            180000
63:            140000
64:            120000
65:             40000
66:             60000
67:            140000
68:                 0
69:             20000
70:            120000
71:            120000
72:                 0
73:             20000
74:            120000
75:             60000
76:                 0
77:             60000
78:            120000
79:            120000
    TotalImputedValue
\end{Soutput}
\begin{Sinput}
lvo[, HoldingClass := "below 1000"]
lvo[TotalImputedValue >= 1000 & TotalImputedValue < 30000, 
  HoldingClass := "1000-29999"]
lvo[TotalImputedValue >= 30000 & TotalImputedValue < 50000, 
  HoldingClass := "30000-49999"]
lvo[TotalImputedValue >= 50000, 
  HoldingClass := "above 50000"]
lvo[, HoldingClass := factor(HoldingClass, 
  levels = c("below 1000", "1000-29999", "30000-49999", "above 50000"))]
setkey(lvo, Arm, HoldingClass, tee)
lvostat2 <- lvo[grepl("es", creditstatus),.(MeanIV = mean(TotalImputedValue, na.rm = T), 
  StdIV = var(TotalImputedValue, na.rm = T)^(.5), 
  N = sum(!is.na(TotalImputedValue))), by = .(Arm, HoldingClass, tee)]
lvostat2[, c("ciLB", "ciUB") := list(MeanIV - StdIV * qt(.975, N- 1), MeanIV + StdIV * qt(.975, N- 1))]
lvostat3 <- lvo[grepl("es", creditstatus),.(MeanIV = mean(TotalImputedValue, na.rm = T), 
  StdIV = var(TotalImputedValue, na.rm = T)^(.5), 
  N = sum(!is.na(TotalImputedValue))), by = .(Arm, HoldingClass, Year)]
lvostat3[, c("ciLB", "ciUB") := list(MeanIV - StdIV * qt(.975, N- 1), MeanIV + StdIV * qt(.975, N- 1))]
\end{Sinput}
\end{Schunk}
\begin{Schunk}
\begin{Sinput}
library(ggplot2)
ggplot(data = lvo[TotalImputedValue > 0], aes(TotalImputedValue)) + 
  geom_histogram(breaks = c(0, seq(10000, 200000, 10000))) + 
  #scale_x_log10(breaks = c(1, 100, 1000, 10000, 20000, 30000, 50000)) +
  scale_x_continuous(breaks = seq(0, 200000, 20000)) +
  theme(axis.text.x = element_text(angle = 90, vjust = 1, hjust = 1), 
   strip.text.y = element_text(colour = "blue"))+
  facet_grid(tee ~ Arm)
\end{Sinput}
\begin{figure}

{\centering \includegraphics[width=\maxwidth]{figure/ImpactEstimationOriginal1600/Total_imputed_value_histogram_original_HHs-1} 

}

\caption{Total imputed value of livestock holding\\ {\footnotesize Livestock holding values are computed by using respective median prices of each year.\setlength{\baselineskip}{8pt}}}\label{Figure Total imputed value histogram original HHs}
\end{figure}
\end{Schunk}
\begin{Schunk}
\begin{Sinput}
library(ggplot2)
ggplot(data = lvostat2, aes(HoldingClass, N)) + 
  geom_col() +
  xlab("Livestock holding classes") +
  theme(axis.text.x = element_text(angle = 90, vjust = 1, hjust = 1), 
    strip.text.y = element_text(colour = "blue"))+
  facet_grid(tee ~ Arm)
\end{Sinput}
\begin{figure}

{\centering \includegraphics[width=\maxwidth]{figure/ImpactEstimationOriginal1600/Histogram_of_livestock_holding_classes_original_HHs-1} 

}

\caption{Histogram of livestock holding classes\\ {\footnotesize Livestock holding values are computed by using respective median prices of each year.\setlength{\baselineskip}{8pt}}}\label{Figure Histogram of livestock holding classes original HHs}
\end{figure}
\end{Schunk}
\begin{Schunk}
\begin{Sinput}
library(ggplot2)
ggplot(data = lvostat3, aes(HoldingClass, N)) + 
  geom_col() +
  xlab("Livestock holding classes") +
  theme(axis.text.x = element_text(angle = 90, vjust = 1, hjust = 1), 
    strip.text.y = element_text(colour = "blue"))+
  facet_grid(Year ~ Arm)
\end{Sinput}
\begin{figure}

{\centering \includegraphics[width=\maxwidth]{figure/ImpactEstimationOriginal1600/Histogram_of_livestock_holding_classes_by_year_original_HHs-1} 

}

\caption{Histogram of livestock holding classes by year\\ {\footnotesize Livestock holding values are computed by using respective median prices of each year.\setlength{\baselineskip}{8pt}}}\label{Figure Histogram of livestock holding classes by year original HHs}
\end{figure}
\end{Schunk}
\begin{itemize}
\vspace{1.0ex}\setlength{\itemsep}{1.0ex}\setlength{\baselineskip}{12pt}
\item	Why does \textsf{cow} report below 1000 holding in rds 2-4?
\end{itemize}
\begin{Schunk}
\begin{Sinput}
lvo[, NumberOfCows := 0L]
lvo[grepl("ow", LivestockCode), NumberOfCows := as.integer(number_owned)]
lvo[,
  .(MeanImputedVal = mean(TotalImputedValue, na.rm = T), 
  MeanNumCows = mean(NumberOfCows, na.rm = T), 
  N = sum(!is.na(TotalImputedValue))), by = .(Arm, survey)]
\end{Sinput}
\begin{Soutput}
            Arm survey MeanImputedVal MeanNumCows   N
 1: traditional      1        5065.33    0.233668 398
 2: traditional      2       15854.00    0.817844 280
 3: traditional      3       20179.62    1.022059 277
 4: traditional      4       21233.75    1.050000 240
 5:       large      1        6092.42    0.275689 399
 6:       large      3       31056.41    1.625000 386
 7:       large      2       24992.86    1.278820 383
 8:       large      4       32686.07    1.630890 382
 9: large grace      1        7392.54    0.333333 399
10: large grace      2       21510.32    1.150943 341
11: large grace      3       27565.65    1.422619 347
12: large grace      4       30276.97    1.528024 343
13:         cow      1        4997.68    0.218045 399
14:         cow      2       20550.29    1.078035 364
15:         cow      3       25399.62    1.300562 365
16:         cow      4       28700.23    1.436950 342
\end{Soutput}
\begin{Sinput}
#lvo[,.(N = sum(!is.na(TotalImputedValue))), by = .(Arm, survey)]
\end{Sinput}
\end{Schunk}
\begin{Schunk}
\begin{Sinput}
library(ggplot2)
lvo[, LivestockType := LivestockCode]
lvo[grepl("Ox|Cow", LivestockCode), LivestockType := "Cow/Ox"]
lvo[grepl("Goat|She", LivestockCode), LivestockType := "Goat/Sheep"]
lvo[grepl("Duc|Hen", LivestockCode), LivestockType := "Poultry"]
lvo[, LivestockType := factor(LivestockType)]
lvotype <- lvo[grepl("es", creditstatus), 
  .(Std = var(number_owned, na.rm = T)^(.5), 
    Total = sum(number_owned, na.rm = T), 
    N = sum(!is.na(number_owned))), 
    by = .(Arm,  LivestockType, Year)]
lvotype <- lvotype[!is.na(Arm), ]
lvotype[, Mean := round(Total/N, 1)]
setnames(lvotype, grepout("^T|N|^S|^M", colnames(lvotype)), 
  paste0("value.", grepout("^T|N|^S|^M", colnames(lvotype))))
lvotype[is.na(LivestockType)|LivestockType == "", LivestockType := "Other"]
lvotype[grepl("cow", LivestockType), LivestockType := "Cow/Ox"]
lvotypel <- reshape(lvotype, direction = "long", 
  idvar = c("Arm", "LivestockType", "Year"),
  varying = grepout("val", colnames(lvotype)))
lvotypel <- lvotypel[grepl("Cow", LivestockType) & grepl("Mean|Tot|^N", time), ]
lvotypel <- lvotypel[!is.na(Year), ]
setkey(lvotypel, Arm, Year, LivestockType)
lvotypel[, Variable := time]
lvotypel[grepl("N", time), Variable := "Number of owners"]
lvotypel[, Variable := factor(Variable, levels = c("Mean", "Total", "Number of owners"))]
ggplot(data = lvotypel, aes(Year, value)) + 
  geom_col(data = lvotypel[grepl("Total", Variable), ]) +
  geom_col(data = lvotypel[grepl("Mean", Variable), ]) +
  geom_col(data = lvotypel[grepl("N", Variable), ]) +
  xlab("Year") +
  theme(axis.text.x = element_text(angle = 90, vjust = 1, hjust = 1), 
    strip.text.y = element_text(colour = "blue"))+
  facet_grid(Variable ~ Arm, scale = "free_y")
\end{Sinput}
\begin{figure}

{\centering \includegraphics[width=\maxwidth]{figure/ImpactEstimationOriginal1600/Number_of_cows_by_year_original_HHs-1} 

}

\caption{Number of cows/oxen by year\\ {\footnotesize Means are mean holding among the owners. Totals are total number of cows/oxen owned. Mean and total number of cows/oxen may diverge because the number of owners differ across round.\setlength{\baselineskip}{8pt}}}\label{Figure Number of cows by year original HHs}
\end{figure}
\end{Schunk}
\begin{palepinkleftbar}
\begin{finding}
\textsc{\small Figure \ref{Figure Total imputed value histogram}} shows general increase in upper holding classes round 3 and further upper holding classes in round 4. \textsc{\small Figure \ref{Figure Number of cows by year}} shows livestock type is not entered (yet collected) in rd3. At this moment, one needs to omit rd 3. All estimation results by far are subject to this omission.
\end{finding}
\end{palepinkleftbar}


\subsection{Assets+Livestock}




\begin{Schunk}
\begin{Sinput}
ass <-  readRDS(paste0(pathsaveHere, "AssetAdminDataUsedForEstimation.rds"))
# creaditstatus != yes are pure controls
table0(ass[survey == 1,.(BorrowerStatus, creditstatus)])
\end{Sinput}
\begin{Soutput}
                 creditstatus
BorrowerStatus     Yes   No
  borrower        1192  157
  pure saver         0   26
  quit membership    0  220
\end{Soutput}
\begin{Sinput}
table0(ass[survey == 1,.(Mstatus, creditstatus)])
\end{Sinput}
\begin{Soutput}
              creditstatus
Mstatus         Yes   No
  gErosion        0   80
  gRejection      0  140
  iRejection      1  157
  iReplacement    0    0
  newGroup        0    0
  oldMember    1191   26
\end{Soutput}
\begin{Sinput}
ass[, grepout("Loan|UD|Forced", colnames(ass)) := NULL]
CovStrings <- "^groupid$|hhid|tee|^dummy.*[a-z]$|Floo|Time\\.?.|With|.Size|Head|^creditstatus$|"
ass <- ass[!(hhid == 7043715 & HAssetAmount == 0), ]
ass1 <- ass[, grepout(paste0(CovStrings, "^HAsse"), colnames(ass)), with = F]
ass1R <- ass[, grepout(paste0(CovStrings, "^HAsse|RM"), colnames(ass)), with = F]
ass2 <- ass[, grepout(paste0(CovStrings, "^PAsse"), colnames(ass)), with = F]
ass2R <- ass[, grepout(paste0(CovStrings, "^PAsse|RM"), colnames(ass)), with = F]
# before-after style 2 time point data. Choose tee == 2 as baseline because there are many zeros in tee == 1.
ass <-  readRDS(paste0(pathsaveHere, "AssetAdminDataUsedForEstimation.rds"))
ass <- ass[!(hhid == 7043715 & HAssetAmount == 0), ]
ass[, grepout("Time|Loan", colnames(ass)) := NULL]
ass3 <- ass[tee == 2 | tee == 4, grepout(paste0(CovStrings, "^HAsse"), colnames(ass)), with = F]
ass3R <- ass[tee == 2 | tee == 4, grepout(paste0(CovStrings, "^HAsse|RM"), colnames(ass)), with = F]
ass4 <- ass[tee == 2 | tee == 4, grepout(paste0(CovStrings, "^PAsse"), colnames(ass)), with = F]
ass4R <- ass[tee == 2 | tee == 4, grepout(paste0(CovStrings, "^PAsse|RM"), colnames(ass)), with = F]
datas0 <- paste0("ass", rep(1:4, each = 2), c("", "R"))
datas <- paste0("as", rep(1:4, each = 2), c("", "R"))
ddatas <- paste0("d", datas)
ddatasd <- paste0(ddatas, "d")
for (i in 1:length(datas)) {
#   dl <- prepFDData(get(datas0[i]), Group = "^hhid$", TimeVar = "tee", Cluster = "groupid", 
#     # before considering pure control contrast
#     #LevelCovariates = "^dummy|Floo|^Time\\..$|Head", 
#     # after considering pure control contrast
#     LevelCovariates = "^dummy|Floo|^Time\\..$|Head|^cred.*s$", 
#     drop.if.NA.in.differencing = T, LevelPeriodToKeep = "last",
#     use.var.name.for.dummy.prefix = F, print.messages = F)
   dl <- FirstDiffPanelData(X = get(datas0[i]), 
     Group = "^hhid$", TimeVar = "tee", Cluster = "groupid",
     LevelCovariates = "^dummy|Head|^Time\\..$|Female$|Floo|Eldest|^cred.*s$|xid$|SchPa")
  dat <- dl$diff
  dat[, grepout("^en$", colnames(dat)) := NULL]
  # create PureControl*Time2, Time3 interactions and drop creditstatus
  if (grepl("ass[12]", datas0[i]) & any(grepl("cred.*s$", colnames(dat)))) {
    dat[, PureControl := 0L]
    dat[!grepl("es$", creditstatus), PureControl := 1L]
    dat[, creditstatus := NULL]
    dat[, c("PureControl.Time3", "PureControl.Time4") := 
      .(PureControl * Time.3, PureControl * Time.4)]
  }
  assign(ddatas[i], dl)
  assign(ddatasd[i], dat)
}
\end{Sinput}
\begin{Soutput}
Dropped 196 obs due to T<2.
Dropped 1402 obs due to NA.
Dropped 196 obs due to T<2.
Dropped 3080 obs due to NA.
Dropped 196 obs due to T<2.
Dropped 1402 obs due to NA.
Dropped 196 obs due to T<2.
Dropped 3080 obs due to NA.
Dropped 130 obs due to T<2.
Dropped 1274 obs due to NA.
Dropped 130 obs due to T<2.
Dropped 1388 obs due to NA.
Dropped 130 obs due to T<2.
Dropped 1274 obs due to NA.
Dropped 130 obs due to T<2.
Dropped 1388 obs due to NA.
\end{Soutput}
\begin{Sinput}
das1Rd <- das1Rd[tee > 2, ]
das2Rd <- das2Rd[tee > 2, ]
das1d[, Tee := .N, by = hhid]
das2d[, Tee := .N, by = hhid]
\end{Sinput}
\end{Schunk}

\begin{Schunk}
\begin{Sinput}
lvo <-  readRDS(paste0(pathsaveHere, "LivestockAdminDataUsedForEstimation.rds"))
table0(lvo[, .(tee, Arm)])
\end{Sinput}
\begin{Soutput}
   Arm
tee traditional large large grace cow
  1         398   399         379 398
  2         283   390         373 379
  3         276   384         348 365
  4         238   377         330 328
\end{Soutput}
\begin{Sinput}
table0(lvo[grepl("ow", LivestockCode), .(tee, Arm)])
\end{Sinput}
\begin{Soutput}
   Arm
tee traditional large large grace cow
  1          66    78          81  63
  2         151   254         258 283
  3         189   348         323 324
  4         156   328         291 287
\end{Soutput}
\begin{Sinput}
# xid <- readRDS(paste0(path1234, "ID.rds"))
# xidlv <- xid[,.(Mstatus, AssignOriginal, groupid, hhid, survey, year)]
# setnames(xidlv, "AssignOriginal", "Arm")
# setkey(lvo, Arm, groupid, hhid, survey, Mstatus)
# setkey(xidlv, Arm, groupid, hhid, survey, Mstatus)
# lvo <- merge(lvo, xidlv, by = key(xidlv), all = T)
lvo[, grepout("Loan|UD|Forced", colnames(lvo)) := NULL]
lvostrings <- "^groupid$|hhid|^Arm$|tee|^dummy[TLCMUWS]|creditst|^TotalIm|Floo|Time\\.|live.*de$|Head|Cows"
lvoR <- lvo[, grepout(paste0(lvostrings, "|RM"), colnames(lvo)), with = F]
lvo <- lvo[, grepout(lvostrings, colnames(lvo)), with = F]
lvo3 <- lvo[tee == 2 | tee == 4, ]
lvoR3 <- lvoR[tee == 2 | tee == 4, ]
datas <- c("lvo", "lvoR", "lvo3", "lvoR3")
ddatas <- paste0("d", datas)
ddatasd <- paste0(ddatas, "d")
for (i in 1:length(datas)) {
#   dl <- prepFDData(get(datas[i]), Group = "^hhid$", TimeVar = "tee", Cluster = "groupid", 
#     LevelCovariates = "^dummy|^Arm$|Floo|^Time\\..$|Head|Cows|liv.*de$|credits", 
#     drop.if.NA.in.differencing = T, LevelPeriodToKeep = "last",
#     use.var.name.for.dummy.prefix = F, print.messages = F)
   dl <- FirstDiffPanelData(X = get(datas[i]), 
     Group = "^hhid$", TimeVar = "tee", Cluster = "groupid",
     LevelCovariates = "^dummy|^Arm$|Floo|^Time\\..$|Head|Cows|liv.*de$|credits|xid$|SchPa")
  dat <- dl$diff
  dat[, grepout("^en$", colnames(dat)) := NULL]
  assign(ddatas[i], dl)
  assign(ddatasd[i], dat)
}
\end{Sinput}
\begin{Soutput}
Dropped 196 obs due to T<2.
Dropped 1402 obs due to NA.
Dropped 196 obs due to T<2.
Dropped 3080 obs due to NA.
Dropped 154 obs due to T<2.
Dropped 1272 obs due to NA.
Dropped 154 obs due to T<2.
Dropped 1386 obs due to NA.
\end{Soutput}
\begin{Sinput}
dlvoRd <- dlvoRd[tee > 1, ]
\end{Sinput}
\end{Schunk}

\begin{Schunk}
\begin{Sinput}
ass <-  readRDS(paste0(pathsaveHere, "RosterAssetAdminOriginalHHsDataUsedForEstimation.rds"))
assstrings <- "^Arm$|^groupid$|hhid|tee|^.Asse|^dummy.*[a-z]$|Floo|Time\\.?.|Head|With|.Size"
lvostrings <- "^groupid$|hhid|tee|^TotalIm|Cows"
ass[, grepout("Loan|UD|Forced", colnames(ass)) := NULL]
ass1 <- ass[(Fromxid), grepout(assstrings, colnames(ass)), with = F]
ass1R <- ass[(Fromxid), grepout(paste0(assstrings, "|RM"), colnames(ass)), with = F]
# before-after style 2 time point data. Choose tee == 2 as baseline because there are many zeros in tee == 1.
ass <-  readRDS(paste0(pathsaveHere, "RosterAssetAdminOriginalHHsDataUsedForEstimation.rds"))
ass[, grepout("Time|Loan", colnames(ass)) := NULL]
lvo <-  readRDS(paste0(pathsaveHere, "RosterLivestockAdminOriginalHHsDataUsedForEstimation.rds"))
lvo[, grepout("Loan|UD|Forced", colnames(lvo)) := NULL]
lvo1 <- lvo[(Fromxid), grepout(lvostrings, colnames(lvo)), with = F]
# merge
#commonstrings <- "^groupid$|hhid|^Arm|tee|Floo|Time\\.?.|Head"
commoncols <- intersect(colnames(ass1), colnames(lvo1))
AL1 <- merge(ass1, lvo1, by = commoncols, ALl = T)
AL1[is.na(TotalImputedValue), TotalImputedValue := 0]
AL1[, TotalValue := TotalImputedValue + HAssetAmount + PAssetAmount]
ALfig <- AL1[, .(Arm, groupid, hhid, tee, TotalValue)]
AL1[, c("TotalImputedValue", "HAssetAmount", "PAssetAmount", "Arm") := NULL]
AL1 <- unique(AL1)
AL2 <- AL1[tee == 2 | tee == 4, ]
AL2[, grepout("Time", colnames(AL2)) := NULL]

commoncols <- intersect(colnames(ass1R), colnames(lvo1))
AL1R <- merge(ass1R, lvo1, by = commoncols, ALl = T)
AL1R[is.na(TotalImputedValue), TotalImputedValue := 0]
AL1R[, TotalValue := TotalImputedValue + HAssetAmount + PAssetAmount]
ALfig <- AL1R[, .(Arm, groupid, hhid, tee, TotalValue)]
AL1R[, c("TotalImputedValue", "HAssetAmount", "PAssetAmount", "Arm") := NULL]
AL1R <- unique(AL1R)
AL2R <- AL1R[tee == 2 | tee == 4, ]
AL2R[, grepout("Time", colnames(AL2)) := NULL]
\end{Sinput}
\begin{Soutput}
Warning in `[.data.table`(AL2R, , `:=`(grepout("Time", colnames(AL2)), NULL)): length(LHS)==0; no columns to delete or assign RHS to.
\end{Soutput}
\begin{Sinput}
datas <- c(paste0("AL", 1:2), paste0("AL", 1:2, "R"))
ddatas <- paste0("d", datas)
ddatasd <- paste0(ddatas, "d")
for (i in 1:length(datas)) {
  dl <- prepFDData(get(datas[i]), Group = "^hhid$", TimeVar = "tee", Cluster = "groupid", 
    LevelCovariates = "^dummy|^Arm|Floo|^Time\\..$|Head|Cows", 
    drop.if.NA.in.differencing = T, LevelPeriodToKeep = "last",
    use.var.name.for.dummy.prefix = F, print.messages = F)
  dat <- dl$diff
  if (i == 1) {
    # Recreate Time.4 which is dropped when kept only 1:(T-1) obs.
    #dat[, c("Time.2", "Time.3", "Time.4") := 0L]
    #dat[tee == 1, Time.2 := 1L]
    #dat[tee == 2, Time.3 := 1L]
    #dat[tee == 3, Time.4 := 1L]
    dat[, grepout("Time.?2", colnames(dat)) := NULL]
  }
  assign(ddatas[i], dl)
  assign(ddatasd[i], dat)
}
dAL1Rd <- dAL1Rd[tee > 2, ]
\end{Sinput}
\end{Schunk}


\begin{Schunk}
\begin{Sinput}
FileName <- "AssetLivestock"
FileNameHeader <- 
  paste0(c("", "Grace", "PovertyStatus", "Size", "Attributes",
    "TInt", "TIntGrace", "TIntSize", "Rd24Diff", "Rd24DiffGrace", 
    "Rd24DiffPovertyStatus", "Rd24DiffSize", "Rd24DiffAttributes"), "OriginalHHs")
alsuffixes <- c("", "G", "P", "S", "a", "T", "TG", "TS", "D", "DG", "DP", "DS", "Da")
listheader <- paste0("al", alsuffixes)
DataToUse1 <- rep("dAL1d", 6)
DataToUse2 <- rep("dAL2d", 6)
Addseparatingcols <- NULL; Separatingcolwidth <- NULL
Separatingcoltitle <- NULL
Regressands <- rep("TotalValue", 6)
tableboxwidth <- 4.5
\end{Sinput}
\end{Schunk}
\begin{Schunk}
\begin{Sinput}
source(paste0(pathprogram, "AssetLivestockCovariateSelection.R"))
\end{Sinput}
\end{Schunk}
\begin{Schunk}
\begin{Sinput}
exclheader <- paste0("excl", alsuffixes)
source(paste0(pathprogram, "FDEstimationFile.R"))
\end{Sinput}
\end{Schunk}
\begin{Schunk}
\begin{Sinput}
library(ggplot2)
ggplot(data = ALfig, aes(group = tee)) + 
#  geom_point(size = .1, position = position_dodge(width = .5)) +
#  geom_smooth(span = .5, aes(colour = Arm, group = Arm)) +
  #scale_x_log10(breaks = c(1, 100, 1000, 10000, 20000, 30000, 50000)) +
  geom_boxplot(aes(x= tee, y = TotalValue, colour = Arm))+
  #scale_y_log10(breaks = c(1, 1000, 5000, 10000, 20000, 50000, 100000, 500000)) +
  scale_y_continuous(breaks = seq(0, 100000, 10000), limits = c(0, 100000)) +
  theme(axis.text.x = element_text(angle = 90, vjust = 1, hjust = 1), 
   strip.text.y = element_text(colour = "blue"), legend.position = "none") +
  facet_grid(. ~ Arm)
\end{Sinput}
\begin{figure}

{\centering \includegraphics[width=\maxwidth]{figure/ImpactEstimationOriginal1600/Total_values_original_HHs-1} 

}

\caption{Total asset values\\ {\footnotesize Sum of assets and livestock holding values. Original 1600 HHs.\setlength{\baselineskip}{8pt}}}\label{Figure Total values original HHs}
\end{figure}
\end{Schunk}
\begin{Schunk}
\begin{Sinput}
# dummy chunk
\end{Sinput}
\end{Schunk}

\hspace{-1cm}\begin{minipage}[t]{14cm}
\hfil\textsc{\normalsize Table \refstepcounter{table}\thetable: FD estimation of total assets, original HHs\label{tab FD total assets original HHs}}\\
\setlength{\tabcolsep}{1pt}
\setlength{\baselineskip}{8pt}
\renewcommand{\arraystretch}{.55}
\hfil\begin{tikzpicture}
\node (tbl) {\input{c:/data/GUK/analysis/save/Original1600/AssetLivestockOriginalHHsFDEstimationResults.tex}};
%\input{c:/dropbox/data/ramadan/save/tablecolortemplate.tex}
\end{tikzpicture}\\
\renewcommand{\arraystretch}{.8}
\setlength{\tabcolsep}{1pt}
\begin{tabular}{>{\hfill\scriptsize}p{1cm}<{}>{\hfill\scriptsize}p{.25cm}<{}>{\scriptsize}p{12cm}<{\hfill}}
Source:& \multicolumn{2}{l}{\scriptsize Estimated with GUK administrative and survey data.}\\
Notes: & 1. & First-difference estimates. A first-difference is defined as $\Delta x_{t+k}\equiv x_{t+k} - x_{t}$  for $k=1, 2, \dots$. Saving and repayment misses are taken from administrative data and merged with survey data at Year-Month of survey interviews. Intercept terms are omitted in estimating equations. Sample is continuing members and replacing members of early rejecters and received loans prior to 2015 Janunary. Household assets do not include livestock. Regressions (1)-(3), (5)-(6) use only arm and calendar information. (4) and (7) use previous six month repayment and saving information which is lacking in rd 1, hence starts from rd 2.\\
& 2. & ${}^{***}$, ${}^{**}$, ${}^{*}$ indicate statistical significance at 1\%, 5\%, 10\%, respetively. Standard errors are clustered at group (village) level.
\end{tabular}
\end{minipage}

\hspace{-1cm}\begin{minipage}[t]{14cm}
\hfil\textsc{\normalsize Table \refstepcounter{table}\thetable: FD estimation of total assets by attributes\label{tab FD total assets attributes original HHs}}\\
\setlength{\tabcolsep}{1pt}
\setlength{\baselineskip}{8pt}
\renewcommand{\arraystretch}{.55}
\hfil\begin{tikzpicture}
\node (tbl) {\input{c:/data/GUK/analysis/save/Original1600/AssetLivestockAttributesOriginalHHsFDEstimationResults.tex}};
%\input{c:/dropbox/data/ramadan/save/tablecolortemplate.tex}
\end{tikzpicture}\\
\renewcommand{\arraystretch}{.8}
\setlength{\tabcolsep}{1pt}
\begin{tabular}{>{\hfill\scriptsize}p{1cm}<{}>{\hfill\scriptsize}p{.25cm}<{}>{\scriptsize}p{12cm}<{\hfill}}
Source:& \multicolumn{2}{l}{\scriptsize Estimated with GUK administrative and survey data.}\\
Notes: & 1. & First-difference estimates. A first-difference is defined as $\Delta x_{t+k}\equiv x_{t+k} - x_{t}$  for $k=1, 2, \dots$. Saving and repayment misses are taken from administrative data and merged with survey data at Year-Month of survey interviews. Intercept terms are omitted in estimating equations. Sample is continuing members and replacing members of early rejecters and received loans prior to 2015 Janunary. Household assets do not include livestock. Regressions (1)-(3), (5)-(6) use only arm and calendar information. (4) and (7) use previous six month repayment and saving information which is lacking in rd 1, hence starts from rd 2.\\
& 2. & ${}^{***}$, ${}^{**}$, ${}^{*}$ indicate statistical significance at 1\%, 5\%, 10\%, respetively. Standard errors are clustered at group (village) level.
\end{tabular}
\end{minipage}


\hspace{-1cm}\begin{minipage}[t]{14cm}
\hfil\textsc{\normalsize Table \refstepcounter{table}\thetable: FD estimation of total assets, moderately poor vs. ultra poor, original HHs \label{tab FD total assets2 original HH}}\\
\setlength{\tabcolsep}{1pt}
\setlength{\baselineskip}{8pt}
\renewcommand{\arraystretch}{.55}
\hfil\begin{tikzpicture}
\node (tbl) {\input{c:/data/GUK/analysis/save/Original1600/AssetLivestockPovertyStatusOriginalHHsFDEstimationResults.tex}};
%\input{c:/dropbox/data/ramadan/save/tablecolortemplate.tex}
\end{tikzpicture}\\
\renewcommand{\arraystretch}{.8}
\setlength{\tabcolsep}{1pt}
\begin{tabular}{>{\hfill\scriptsize}p{1cm}<{}>{\hfill\scriptsize}p{.25cm}<{}>{\scriptsize}p{12cm}<{\hfill}}
Source:& \multicolumn{2}{l}{\scriptsize Estimated with GUK administrative and survey data.}\\
Notes: & 1. & First-difference estimates. A first-difference is defined as $\Delta x_{t+k}\equiv x_{t+k} - x_{t}$  for $k=1, 2, \dots$. Saving and repayment misses are taken from administrative data and merged with survey data at Year-Month of survey interviews. Intercept terms are omitted in estimating equations. Sample is continuing members and replacing members of early rejecters and received loans prior to 2015 Janunary. Household assets do not include livestock. \\
& 2. & ${}^{***}$, ${}^{**}$, ${}^{*}$ indicate statistical significance at 1\%, 5\%, 10\%, respetively. Standard errors are clustered at group (village) level.
\end{tabular}
\end{minipage}

\hspace{-1cm}\begin{minipage}[t]{14cm}
\hfil\textsc{\normalsize Table \refstepcounter{table}\thetable: FD estimation of total assets, small vs. large size loans, original HHs \label{tab FD total assets3 original HHs}}\\
\setlength{\tabcolsep}{1pt}
\setlength{\baselineskip}{8pt}
\renewcommand{\arraystretch}{.55}
\hfil\begin{tikzpicture}
\node (tbl) {\input{c:/data/GUK/analysis/save/Original1600/AssetLivestockSizeOriginalHHsFDEstimationResults.tex}};
%\input{c:/dropbox/data/ramadan/save/tablecolortemplate.tex}
\end{tikzpicture}\\
\renewcommand{\arraystretch}{.8}
\setlength{\tabcolsep}{1pt}
\begin{tabular}{>{\hfill\scriptsize}p{1cm}<{}>{\hfill\scriptsize}p{.25cm}<{}>{\scriptsize}p{12cm}<{\hfill}}
Source:& \multicolumn{2}{l}{\scriptsize Estimated with GUK administrative and survey data.}\\
Notes: & 1. & First-difference estimates. A first-difference is defined as $\Delta x_{t+k}\equiv x_{t+k} - x_{t}$  for $k=1, 2, \dots$. Saving and repayment misses are taken from administrative data and merged with survey data at Year-Month of survey interviews. Intercept terms are omitted in estimating equations. Sample is continuing members and replacing members of early rejecters and received loans prior to 2015 Janunary. Household assets do not include livestock. \\
& 2. & ${}^{***}$, ${}^{**}$, ${}^{*}$ indicate statistical significance at 1\%, 5\%, 10\%, respetively. Standard errors are clustered at group (village) level.
\end{tabular}
\end{minipage}

\hspace{-1cm}\begin{minipage}[t]{14cm}
\hfil\textsc{\normalsize Table \refstepcounter{table}\thetable: FD estimation of total assets, with vs. without a grace period, original HHs \label{tab FD total assets4 original HHs}}\\
\setlength{\tabcolsep}{1pt}
\setlength{\baselineskip}{8pt}
\renewcommand{\arraystretch}{.55}
\hfil\begin{tikzpicture}
\node (tbl) {\input{c:/data/GUK/analysis/save/Original1600/AssetLivestockGraceOriginalHHsFDEstimationResults.tex}};
%\input{c:/dropbox/data/ramadan/save/tablecolortemplate.tex}
\end{tikzpicture}\\
\renewcommand{\arraystretch}{.8}
\setlength{\tabcolsep}{1pt}
\begin{tabular}{>{\hfill\scriptsize}p{1cm}<{}>{\hfill\scriptsize}p{.25cm}<{}>{\scriptsize}p{12cm}<{\hfill}}
Source:& \multicolumn{2}{l}{\scriptsize Estimated with GUK administrative and survey data.}\\
Notes: & 1. & First-difference estimates. A first-difference is defined as $\Delta x_{t+k}\equiv x_{t+k} - x_{t}$  for $k=1, 2, \dots$. Saving and repayment misses are taken from administrative data and merged with survey data at Year-Month of survey interviews. Intercept terms are omitted in estimating equations. Sample is continuing members and replacing members of early rejecters and received loans prior to 2015 Janunary. Household assets do not include livestock. \\
& 2. & ${}^{***}$, ${}^{**}$, ${}^{*}$ indicate statistical significance at 1\%, 5\%, 10\%, respetively. Standard errors are clustered at group (village) level.
\end{tabular}
\end{minipage}

\hspace{-1cm}\begin{minipage}[t]{14cm}
\hfil\textsc{\normalsize Table \refstepcounter{table}\thetable: FD estimation of total assets, round 2 and 4 comparison, original HHs \label{tab FD total assets rd24, original HHs }}\\
\setlength{\tabcolsep}{1pt}
\setlength{\baselineskip}{8pt}
\renewcommand{\arraystretch}{.55}
\hfil\begin{tikzpicture}
\node (tbl) {\input{c:/data/GUK/analysis/save/Original1600/AssetLivestockRd24DiffOriginalHHsFDEstimationResults.tex}};
%\input{c:/dropbox/data/ramadan/save/tablecolortemplate.tex}
\end{tikzpicture}\\
\renewcommand{\arraystretch}{.8}
\setlength{\tabcolsep}{1pt}
\begin{tabular}{>{\hfill\scriptsize}p{1cm}<{}>{\hfill\scriptsize}p{.25cm}<{}>{\scriptsize}p{12cm}<{\hfill}}
Source:& \multicolumn{2}{l}{\scriptsize Estimated with GUK administrative and survey data.}\\
Notes: & 1. & First-difference estimates between round 2 and 4. A first-difference is defined as $\Delta x_{t+k}\equiv x_{t+k} - x_{t}$ for $k=1, 2, \dots$. Saving and repayment misses are taken from administrative data and merged with survey data at Year-Month of survey interviews. Intercept terms are omitted in estimating equations. Sample is continuing members and replacing members of early rejecters and received loans prior to 2015 Janunary. Household assets do not include livestock. Regressions (1)-(3), (5)-(6) use only arm and calendar information. (4) and (7) use previous six month repayment and saving information which is lacking in rd 1, hence starts from rd 2.\\
& 2. & ${}^{***}$, ${}^{**}$, ${}^{*}$ indicate statistical significance at 1\%, 5\%, 10\%, respetively. Standard errors are clustered at group (village) level.
\end{tabular}
\end{minipage}

\hspace{-1cm}\begin{minipage}[t]{14cm}
\hfil\textsc{\normalsize Table \refstepcounter{table}\thetable: FD estimation of total assets, round 2 and 4 comparison, grace period, original HHs \label{tab FD total assets rd24 grace original HHs}}\\
\setlength{\tabcolsep}{1pt}
\setlength{\baselineskip}{8pt}
\renewcommand{\arraystretch}{.55}
\hfil\begin{tikzpicture}
\node (tbl) {\input{c:/data/GUK/analysis/save/Original1600/AssetLivestockRd24DiffGraceOriginalHHsFDEstimationResults.tex}};
%\input{c:/dropbox/data/ramadan/save/tablecolortemplate.tex}
\end{tikzpicture}\\
\renewcommand{\arraystretch}{.8}
\setlength{\tabcolsep}{1pt}
\begin{tabular}{>{\hfill\scriptsize}p{1cm}<{}>{\hfill\scriptsize}p{.25cm}<{}>{\scriptsize}p{12cm}<{\hfill}}
Source:& \multicolumn{2}{l}{\scriptsize Estimated with GUK administrative and survey data.}\\
Notes: & 1. & First-difference estimates between round 2 and 4. A first-difference is defined as $\Delta x_{t+k}\equiv x_{t+k} - x_{t}$ for $k=1, 2, \dots$. Saving and repayment misses are taken from administrative data and merged with survey data at Year-Month of survey interviews. Intercept terms are omitted in estimating equations. Sample is continuing members and replacing members of early rejecters and received loans prior to 2015 Janunary. Household assets do not include livestock. Regressions (1)-(3), (5)-(6) use only arm and calendar information. (4) and (7) use previous six month repayment and saving information which is lacking in rd 1, hence starts from rd 2.\\
& 2. & ${}^{***}$, ${}^{**}$, ${}^{*}$ indicate statistical significance at 1\%, 5\%, 10\%, respetively. Standard errors are clustered at group (village) level.
\end{tabular}
\end{minipage}

\hspace{-1cm}\begin{minipage}[t]{14cm}
\hfil\textsc{\normalsize Table \refstepcounter{table}\thetable: FD estimation of total assets, round 2 and 4 comparison, ultra poor vs. moderately poor, original HHs \label{tab FD total assets rd24 poor original HHs}}\\
\setlength{\tabcolsep}{1pt}
\setlength{\baselineskip}{8pt}
\renewcommand{\arraystretch}{.55}
\hfil\begin{tikzpicture}
\node (tbl) {\input{c:/data/GUK/analysis/save/Original1600/AssetLivestockRd24DiffPovertyStatusOriginalHHsFDEstimationResults.tex}};
%\input{c:/dropbox/data/ramadan/save/tablecolortemplate.tex}
\end{tikzpicture}\\
\renewcommand{\arraystretch}{.8}
\setlength{\tabcolsep}{1pt}
\begin{tabular}{>{\hfill\scriptsize}p{1cm}<{}>{\hfill\scriptsize}p{.25cm}<{}>{\scriptsize}p{12cm}<{\hfill}}
Source:& \multicolumn{2}{l}{\scriptsize Estimated with GUK administrative and survey data.}\\
Notes: & 1. & First-difference estimates between round 2 and 4. A first-difference is defined as $\Delta x_{t+k}\equiv x_{t+k} - x_{t}$ for $k=1, 2, \dots$. Saving and repayment misses are taken from administrative data and merged with survey data at Year-Month of survey interviews. Intercept terms are omitted in estimating equations. Sample is continuing members and replacing members of early rejecters and received loans prior to 2015 Janunary. Household assets do not include livestock. Regressions (1)-(3), (5)-(6) use only arm and calendar information. (4) and (7) use previous six month repayment and saving information which is lacking in rd 1, hence starts from rd 2.\\
& 2. & ${}^{***}$, ${}^{**}$, ${}^{*}$ indicate statistical significance at 1\%, 5\%, 10\%, respetively. Standard errors are clustered at group (village) level.
\end{tabular}
\end{minipage}

\clearpage
\subsection{Incomes}



\begin{Schunk}
\begin{Sinput}
source(paste0(pathprogram, "ReadTrimIncomeOriginalHHsFDData.R"))
\end{Sinput}
\begin{Soutput}
Dropped 4546 obs due to T<2.
Dropped 1133 obs due to NA.
Dropped 4546 obs due to T<2.
Dropped 1469 obs due to NA.
Dropped 6242 obs due to NA.
Dropped 6250 obs due to NA.
\end{Soutput}
\end{Schunk}
\begin{Schunk}
\begin{Sinput}
source(paste0(pathprogram, "ReadTrimIncomeOriginalHHsFDData.R"))
\end{Sinput}
\begin{Soutput}
Dropped 4546 obs due to T<2.
Dropped 1133 obs due to NA.
Dropped 4546 obs due to T<2.
Dropped 1469 obs due to NA.
Dropped 6242 obs due to NA.
Dropped 6250 obs due to NA.
\end{Soutput}
\end{Schunk}

Income sources are mainly labour incomes (\textsf{lab}) and farm revenues (\textsf{far}) with 6165 and 6400 observations, respectively. After first-differencing, they become 486 and 150 observations, with 486 households observed for 487 times. 


Obs for survey labour income.
\begin{Schunk}
\begin{Sinput}
table(dlabd[, tee])
\end{Sinput}
\begin{Soutput}

  1   2   3   4 
  1 311 128  46 
\end{Soutput}
\end{Schunk}
Obs for survey labour income and admin repayment data.
\begin{Schunk}
\begin{Sinput}
table(dlabRd[, tee])
\end{Sinput}
\begin{Soutput}

  3   4 
106  43 
\end{Soutput}
\begin{Sinput}
table(dfarRd[, tee])
\end{Sinput}
\begin{Soutput}

 3  4 
79 71 
\end{Soutput}
\end{Schunk}
Obs for survey farm revenue.
\begin{Schunk}
\begin{Sinput}
table(dfard[, tee])
\end{Sinput}
\begin{Soutput}

 3  4 
79 71 
\end{Soutput}
\end{Schunk}
Obs for survey farm revenue and admin repayment data.
\begin{Schunk}
\begin{Sinput}
table(dfarRd[, tee])
\end{Sinput}
\begin{Soutput}

 3  4 
79 71 
\end{Soutput}
\begin{Sinput}
dlabRd <- dlabRd[tee > 2, ]
dfard <- dfard[tee > 2, ]
dfarRd <- dfarRd[tee > 2, ]
\end{Sinput}
\end{Schunk}
\begin{Schunk}
\begin{Sinput}
FileName <- "Incomes"
FileNameHeader <- paste0(c("", "Grace", "PovertyStatus", "Size", "Attributes"),
   "OriginalHHs")
lbsuffixes <- c("", "g", "p", "s", "a")
listheader <- paste0("lb", lbsuffixes)
Regressands <- c(rep("TotalHHLabourIncome", 4), rep("TotalRevenue", 3))
DataToUse1 <- DataToUse2 <- c(rep("dlabd", 3), "dlabRd", rep("dfard", 2), "dfarRd")
Addseparatingcols = 4; Separatingcolwidth = .2
Separatingcoltitle = c("Labour income (Tk)", "Farm income (Tk)")
\end{Sinput}
\end{Schunk}
\begin{Schunk}
\begin{Sinput}
source(paste0(pathprogram, "IncomeCovariateSelection.R"))
\end{Sinput}
\end{Schunk}
\begin{Schunk}
\begin{Sinput}
exclheader <- paste0("excl", lbsuffixes)
source(paste0(pathprogram, "FDEstimationFile.R"))
\end{Sinput}
\end{Schunk}


\begin{Schunk}
\begin{Sinput}
#dummy chunk
\end{Sinput}
\end{Schunk}

\hspace{-1cm}\begin{minipage}[t]{14cm}
\hfil\textsc{\normalsize Table \refstepcounter{table}\thetable: FD estimation of incomes\label{tab FD incomes original HH}}\\
\setlength{\tabcolsep}{1pt}
\setlength{\baselineskip}{8pt}
\renewcommand{\arraystretch}{.55}
\hfil\begin{tikzpicture}
\node (tbl) {\input{c:/data/GUK/analysis/save/Original1600/IncomesOriginalHHsFDEstimationResults.tex}};
%\input{c:/dropbox/data/ramadan/save/tablecolortemplate.tex}
\end{tikzpicture}\\
\renewcommand{\arraystretch}{.8}
\setlength{\tabcolsep}{1pt}
\begin{tabular}{>{\hfill\scriptsize}p{1cm}<{}>{\hfill\scriptsize}p{.25cm}<{}>{\scriptsize}p{12cm}<{\hfill}}
Source:& \multicolumn{2}{l}{\scriptsize Estimated with GUK administrative and survey data.}\\
Notes: & 1. & First-difference estimates. A first-difference is defined as $\Delta x_{t+1}\equiv x_{t+1} - x_{t}$. Saving and repayment misses are taken from administrative data and merged with survey data at Year-Month of survey interviews. Intercept terms are omitted in estimating equations. Sample is continuing members and replacing members of early rejecters and received loans prior to 2015 Janunary. Labour income is in 1000 Tk unit andis sum of all earned labour incomes. Farm revenue is total of agricultural produce sales. \\
& 2. & ${}^{***}$, ${}^{**}$, ${}^{*}$ indicate statistical significance at 1\%, 5\%, 10\%, respetively. Standard errors are clustered at group (village) level.
\end{tabular}
\end{minipage}

\hspace{-1cm}\begin{minipage}[t]{14cm}
\hfil\textsc{\normalsize Table \refstepcounter{table}\thetable: FD estimation of incomesby attributes \label{tab FD incomes attributes original HH}}\\
\setlength{\tabcolsep}{1pt}
\setlength{\baselineskip}{8pt}
\renewcommand{\arraystretch}{.55}
\hfil\begin{tikzpicture}
\node (tbl) {\input{c:/data/GUK/analysis/save/Original1600/IncomesAttributesOriginalHHsFDEstimationResults.tex}};
%\input{c:/dropbox/data/ramadan/save/tablecolortemplate.tex}
\end{tikzpicture}\\
\renewcommand{\arraystretch}{.8}
\setlength{\tabcolsep}{1pt}
\begin{tabular}{>{\hfill\scriptsize}p{1cm}<{}>{\hfill\scriptsize}p{.25cm}<{}>{\scriptsize}p{12cm}<{\hfill}}
Source:& \multicolumn{2}{l}{\scriptsize Estimated with GUK administrative and survey data.}\\
Notes: & 1. & First-difference estimates. A first-difference is defined as $\Delta x_{t+1}\equiv x_{t+1} - x_{t}$. Saving and repayment misses are taken from administrative data and merged with survey data at Year-Month of survey interviews. Intercept terms are omitted in estimating equations. Sample is continuing members and replacing members of early rejecters and received loans prior to 2015 Janunary. Labour income is in 1000 Tk unit andis sum of all earned labour incomes. Farm revenue is total of agricultural produce sales. \\
& 2. & ${}^{***}$, ${}^{**}$, ${}^{*}$ indicate statistical significance at 1\%, 5\%, 10\%, respetively. Standard errors are clustered at group (village) level.
\end{tabular}
\end{minipage}

\hspace{-1cm}\begin{minipage}[t]{14cm}
\hfil\textsc{\normalsize Table \refstepcounter{table}\thetable: FD estimation of incomes, moderately poor vs. ultra poor\label{tab FD incomes2 original HH}}\\
\setlength{\tabcolsep}{1pt}
\setlength{\baselineskip}{8pt}
\renewcommand{\arraystretch}{.55}
\hfil\begin{tikzpicture}
\node (tbl) {\input{c:/data/GUK/analysis/save/Original1600/IncomesPovertyStatusOriginalHHsFDEstimationResults.tex}};
%\input{c:/dropbox/data/ramadan/save/tablecolortemplate.tex}
\end{tikzpicture}\\
\renewcommand{\arraystretch}{.8}
\setlength{\tabcolsep}{1pt}
\begin{tabular}{>{\hfill\scriptsize}p{1cm}<{}>{\hfill\scriptsize}p{.25cm}<{}>{\scriptsize}p{12cm}<{\hfill}}
Source:& \multicolumn{2}{l}{\scriptsize Estimated with GUK administrative and survey data.}\\
Notes: & 1. & First-difference estimates. A first-difference is defined as $\Delta x_{t+1}\equiv x_{t+1} - x_{t}$. Saving and repayment misses are taken from administrative data and merged with survey data at Year-Month of survey interviews. Intercept terms are omitted in estimating equations. Sample is continuing members and replacing members of early rejecters and received loans prior to 2015 Janunary. Labour income is in 1000 Tk unit andis sum of all earned labour incomes. Farm revenue is total of agricultural produce sales. \\
& 2. & ${}^{***}$, ${}^{**}$, ${}^{*}$ indicate statistical significance at 1\%, 5\%, 10\%, respetively. Standard errors are clustered at group (village) level.
\end{tabular}
\end{minipage}

\hspace{-1cm}\begin{minipage}[t]{14cm}
\hfil\textsc{\normalsize Table \refstepcounter{table}\thetable: FD estimation of incomes, loan size\label{tab FD incomes3 original HH}}\\
\setlength{\tabcolsep}{1pt}
\setlength{\baselineskip}{8pt}
\renewcommand{\arraystretch}{.55}
\hfil\begin{tikzpicture}
\node (tbl) {\input{c:/data/GUK/analysis/save/Original1600/IncomesSizeOriginalHHsFDEstimationResults.tex}};
%\input{c:/dropbox/data/ramadan/save/tablecolortemplate.tex}
\end{tikzpicture}\\
\renewcommand{\arraystretch}{.8}
\setlength{\tabcolsep}{1pt}
\begin{tabular}{>{\hfill\scriptsize}p{1cm}<{}>{\hfill\scriptsize}p{.25cm}<{}>{\scriptsize}p{12cm}<{\hfill}}
Source:& \multicolumn{2}{l}{\scriptsize Estimated with GUK administrative and survey data.}\\
Notes: & 1. & First-difference estimates. A first-difference is defined as $\Delta x_{t+1}\equiv x_{t+1} - x_{t}$. Saving and repayment misses are taken from administrative data and merged with survey data at Year-Month of survey interviews. Intercept terms are omitted in estimating equations. Sample is continuing members and replacing members of early rejecters and received loans prior to 2015 Janunary. Labour income is in 1000 Tk unit andis sum of all earned labour incomes. Farm revenue is total of agricultural produce sales. \\
& 2. & ${}^{***}$, ${}^{**}$, ${}^{*}$ indicate statistical significance at 1\%, 5\%, 10\%, respetively. Standard errors are clustered at group (village) level.
\end{tabular}
\end{minipage}

\hspace{-1cm}\begin{minipage}[t]{14cm}
\hfil\textsc{\normalsize Table \refstepcounter{table}\thetable: FD estimation of incomes, with vs. without a grace period\label{tab FD incomes4 original HH}}\\
\setlength{\tabcolsep}{1pt}
\setlength{\baselineskip}{8pt}
\renewcommand{\arraystretch}{.55}
\hfil\begin{tikzpicture}
\node (tbl) {\input{c:/data/GUK/analysis/save/Original1600/IncomesGraceOriginalHHsFDEstimationResults.tex}};
%\input{c:/dropbox/data/ramadan/save/tablecolortemplate.tex}
\end{tikzpicture}\\
\renewcommand{\arraystretch}{.8}
\setlength{\tabcolsep}{1pt}
\begin{tabular}{>{\hfill\scriptsize}p{1cm}<{}>{\hfill\scriptsize}p{.25cm}<{}>{\scriptsize}p{12cm}<{\hfill}}
Source:& \multicolumn{2}{l}{\scriptsize Estimated with GUK administrative and survey data.}\\
Notes: & 1. & First-difference estimates. A first-difference is defined as $\Delta x_{t+1}\equiv x_{t+1} - x_{t}$. Saving and repayment misses are taken from administrative data and merged with survey data at Year-Month of survey interviews. Intercept terms are omitted in estimating equations. Sample is continuing members and replacing members of early rejecters and received loans prior to 2015 Janunary. Labour income is in 1000 Tk unit andis sum of all earned labour incomes. Farm revenue is total of agricultural produce sales. \\
& 2. & ${}^{***}$, ${}^{**}$, ${}^{*}$ indicate statistical significance at 1\%, 5\%, 10\%, respetively. Standard errors are clustered at group (village) level.
\end{tabular}
\end{minipage}


\begin{palepinkleftbar}
\begin{finding}
\textsc{\small Table \ref{tab FD incomes original HH}} (1) and (3) show a general decrease in rd 1 - 2 period and a general increase in rd 2 - 4 periods for labour incomes. (2) and (4) suggest \textsf{Large grace} arm saw a greater swing (decrease and increases) which resulted in overall significant mean increase of -5.55 (at $p$ value of 21.66\%), yet not statistically different from \textsf{traditional}, while other arms have estimates closer to \textsf{traditional}. This labour income response can be due to the flood in rd 1 which reduced the labour incomes while repayment burden in later rounds prompted households to earn more labour incomes. Strong positive correlation with other members' previous 6 month repayment in (4) may be due to concerted peer efforts in repayment. Farm revenues do not show any systematic trend.
\end{finding}
\end{palepinkleftbar}


\subsection{Consumption}


\begin{Schunk}
\begin{Sinput}
con <-  readRDS(paste0(pathsaveHere, "RosterConsumptionAdminOriginalHHsDataUsedForEstimation.rds"))
con[, ConsumptionBaseline := 0L]
con[as.Date(IntDate) < as.Date(DisDate1), ConsumptionBaseline := 1L]
con[, ConsumptionBaseline := as.integer(any(ConsumptionBaseline == 1L)), 
  by = hhid]
\end{Sinput}
\end{Schunk}
%Number of HHs with consumption before the loan is disbursed (\textsf{ConsumptionBaseline} == 1) is small.
\begin{Schunk}
\begin{Sinput}
table(con[, .(Arm, ConsumptionBaseline)])
\end{Sinput}
\begin{Soutput}
             ConsumptionBaseline
Arm              0    1
  traditional  513  284
  large        146 1002
  large grace   51  981
  cow          200  874
\end{Soutput}
\begin{Sinput}
con <- con[(Fromxid),
  grepout("groupid|hhid|tee|^dummy[A-Z]|Floo|Tim|Size|With|Poo|RM|Expen|Head|HH", 
  colnames(con)), with = F]
expcol <- grepout("Exp", colnames(con))
con[, paste0("PC", expcol) := .SD/HHsize, .SDcols = expcol]
pcexpcol <- grepout("PC", colnames(con))
con[, c("PCExpenditure", "TotalExpenditure") := 
  .(eval(parse(text=paste(pcexpcol, collapse = "+"))), 
    eval(parse(text=paste(expcol, collapse = "+"))))]
con[, grepout("Loan|UD|^Tota|Food|Ener|Soc|^Hygi|^Time$", colnames(con)) := NULL]
# drop Time 2 (period 1-2) and its iteractions, because data starts from t=2
#conR[, grepout("Time.?2|Time.?3|^Time$", colnames(con)) := NULL]
conR = copy(con)
conR[, grepout("Time.?2|^Time$", colnames(con)) := NULL]
con[, grepout("RM", colnames(con)) := NULL]
datas <- c("con", "conR")
ddatas <- paste0("d", datas)
ddatasd <- paste0(ddatas, "d")
for (i in 1:length(datas)) {
# a  dl <- prepFDData(get(datas[i]), Group = "^hhid$", TimeVar = "tee", Cluster = "groupid", 
# a    LevelCovariates = "^dummy[A-Z].*[a-z]$|Floo|^Time\\..$|Head|HH", 
# a    drop.if.NA.in.differencing = T, LevelPeriodToKeep = "last",
# a    use.var.name.for.dummy.prefix = F, print.messages = F)
# a  dat <- dl$diff
  dl <- FirstDiffPanelData(get(datas[i]), 
    Group = "^hhid$", TimeVar = "tee", Cluster = "groupid",
     LevelCovariates = "^dummy|Head|^Time\\..$|Female$|Floo|Eldest|HH|credits|xid$|SchPa|^Size$")
  dat <- dl$diff
  dat[, grepout("^en$", colnames(dat)) := NULL]
  # Recreate Time.4 which is dropped when kept only 1:(T-1) obs.
  dat[, grepout("Time.?2", colnames(dat)) := NULL]
  assign(ddatas[i], dl)
  assign(ddatasd[i], dat)
}
\end{Sinput}
\begin{Soutput}
Dropped 4028 obs due to NA.
Dropped 4029 obs due to NA.
\end{Soutput}
\begin{Soutput}
Warning in `[.data.table`(dat, , `:=`(grepout("Time.?2", colnames(dat)), : length(LHS)==0; no columns to delete or assign RHS to.
\end{Soutput}
\begin{Sinput}
dcond[, Tee := .N, by = hhid]
\end{Sinput}
\end{Schunk}

Consumption is observed in rd 2-4. There are 6400 observations, with first-differencing, it becomes 2372 observations with 42, 2330 households observed for 2, 3 times. 

\begin{Schunk}
\begin{Sinput}
source(paste0(pathprogram, "ReadTrimConsumptionOriginalHHsFDData.R"))
\end{Sinput}
\begin{Soutput}
Dropped 4028 obs due to NA.
Dropped 4029 obs due to NA.
\end{Soutput}
\begin{Soutput}
Warning in `[.data.table`(dat, , `:=`(grepout("Time.?2|Arm", colnames(dat)), : length(LHS)==0; no columns to delete or assign RHS to.
\end{Soutput}
\end{Schunk}

\begin{Schunk}
\begin{Sinput}
FileName <- "Consumption"
cnsuffixes <- c("", "g", "p", "s", "a")
listheader <- paste0("cn", cnsuffixes)
Regressands <- c(rep("PCExpenditure", 4), rep("PCHygieneExpenditure", 3))
DataToUse1 <- DataToUse2 <- c(rep("dcond", 3), "dconRd", rep("dcond", 2), "dconRd")
Addseparatingcols = 4; Separatingcolwidth = .2
Separatingcoltitle = c("Per capita consumption (Tk)", "Per capita hygiene consumption (Tk)")
\end{Sinput}
\end{Schunk}
\begin{Schunk}
\begin{Sinput}
source(paste0(pathprogram, "ConsumptionCovariateSelection.R"))
\end{Sinput}
\end{Schunk}
\begin{Schunk}
\begin{Sinput}
FileNameHeader <- paste0(c("", "Grace", "PovertyStatus", "Size", "Attributes"),
  "OriginalHHs")
exclheader <- paste0("excl", cnsuffixes)
source(paste0(pathprogram, "FDEstimationFile.R"))
\end{Sinput}
\end{Schunk}

\begin{Schunk}
\begin{Sinput}
FileNameHeader <- paste0(FileNameHeader, "Robustness")
exclheader <- paste0("excl", cnsuffixes)
source(paste0(pathprogram, "FDEstimationFile.R"))
\end{Sinput}
\end{Schunk}
\begin{Schunk}
\begin{Sinput}
#dummy chunk
\end{Sinput}
\end{Schunk}

\hspace{-1cm}\begin{minipage}[t]{14cm}
\hfil\textsc{\normalsize Table \refstepcounter{table}\thetable: FD estimation of consumption\label{tab FD consumption original HH}}\\
\setlength{\tabcolsep}{1pt}
\setlength{\baselineskip}{8pt}
\renewcommand{\arraystretch}{.55}
\hfil\begin{tikzpicture}
\node (tbl) {\input{c:/data/GUK/analysis/save/Original1600/ConsumptionOriginalHHsFDEstimationResults.tex}};
%\input{c:/dropbox/data/ramadan/save/tablecolortemplate.tex}
\end{tikzpicture}\\
\renewcommand{\arraystretch}{.8}
\setlength{\tabcolsep}{1pt}
\begin{tabular}{>{\hfill\scriptsize}p{1cm}<{}>{\hfill\scriptsize}p{.25cm}<{}>{\scriptsize}p{12cm}<{\hfill}}
Source:& \multicolumn{2}{l}{\scriptsize Estimated with GUK administrative and survey data.}\\
Notes: & 1. & First-difference estimates. A first-difference is defined as $\Delta x_{t+1}\equiv x_{t+1} - x_{t}$. Saving and repayment misses are taken from administrative data and merged with survey data at Year-Month of survey interviews. Intercept terms are omitted in estimating equations. Sample is continuing members and replacing members of early rejecters and received loans prior to 2015 Janunary. Consumption is annualised values. \\
& 2. & ${}^{***}$, ${}^{**}$, ${}^{*}$ indicate statistical significance at 1\%, 5\%, 10\%, respetively. Standard errors are clustered at group (village) level.
\end{tabular}
\end{minipage}

\hspace{-1cm}\begin{minipage}[t]{14cm}
\hfil\textsc{\normalsize Table \refstepcounter{table}\thetable: FD estimation of consumption by attributes \label{tab FD consumption attributes original HH}}\\
\setlength{\tabcolsep}{1pt}
\setlength{\baselineskip}{8pt}
\renewcommand{\arraystretch}{.55}
\hfil\begin{tikzpicture}
\node (tbl) {\input{c:/data/GUK/analysis/save/Original1600/ConsumptionAttributesOriginalHHsFDEstimationResults.tex}};
%\input{c:/dropbox/data/ramadan/save/tablecolortemplate.tex}
\end{tikzpicture}\\
\renewcommand{\arraystretch}{.8}
\setlength{\tabcolsep}{1pt}
\begin{tabular}{>{\hfill\scriptsize}p{1cm}<{}>{\hfill\scriptsize}p{.25cm}<{}>{\scriptsize}p{12cm}<{\hfill}}
Source:& \multicolumn{2}{l}{\scriptsize Estimated with GUK administrative and survey data.}\\
Notes: & 1. & First-difference estimates. A first-difference is defined as $\Delta x_{t+1}\equiv x_{t+1} - x_{t}$. Saving and repayment misses are taken from administrative data and merged with survey data at Year-Month of survey interviews. Intercept terms are omitted in estimating equations. Sample is continuing members and replacing members of early rejecters and received loans prior to 2015 Janunary. Consumption is annualised values. \\
& 2. & ${}^{***}$, ${}^{**}$, ${}^{*}$ indicate statistical significance at 1\%, 5\%, 10\%, respetively. Standard errors are clustered at group (village) level.
\end{tabular}
\end{minipage}


\hspace{-1cm}\begin{minipage}[t]{14cm}
\hfil\textsc{\normalsize Table \refstepcounter{table}\thetable: FD estimation of consumption, moderately poor vs. ultra poor\label{tab FD consumption2 original HH}}\\
\setlength{\tabcolsep}{1pt}
\setlength{\baselineskip}{8pt}
\renewcommand{\arraystretch}{.55}
\hfil\begin{tikzpicture}
\node (tbl) {\input{c:/data/GUK/analysis/save/Original1600/ConsumptionPovertyStatusOriginalHHsFDEstimationResults.tex}};
%\input{c:/dropbox/data/ramadan/save/tablecolortemplate.tex}
\end{tikzpicture}\\
\renewcommand{\arraystretch}{.8}
\setlength{\tabcolsep}{1pt}
\begin{tabular}{>{\hfill\scriptsize}p{1cm}<{}>{\hfill\scriptsize}p{.25cm}<{}>{\scriptsize}p{12cm}<{\hfill}}
Source:& \multicolumn{2}{l}{\scriptsize Estimated with GUK administrative and survey data.}\\
Notes: & 1. & First-difference estimates. A first-difference is defined as $\Delta x_{t+1}\equiv x_{t+1} - x_{t}$. Saving and repayment misses are taken from administrative data and merged with survey data at Year-Month of survey interviews. Intercept terms are omitted in estimating equations. Sample is continuing members and replacing members of early rejecters and received loans prior to 2015 Janunary. Consumption is annualised values. \\
& 2. & ${}^{***}$, ${}^{**}$, ${}^{*}$ indicate statistical significance at 1\%, 5\%, 10\%, respetively. Standard errors are clustered at group (village) level.
\end{tabular}
\end{minipage}


\hspace{-1cm}\begin{minipage}[t]{14cm}
\hfil\textsc{\normalsize Table \refstepcounter{table}\thetable: FD estimation of consumption, large vs. small size loans\label{tab FD consumption3 original HH}}\\
\setlength{\tabcolsep}{1pt}
\setlength{\baselineskip}{8pt}
\renewcommand{\arraystretch}{.55}
\hfil\begin{tikzpicture}
\node (tbl) {\input{c:/data/GUK/analysis/save/Original1600/ConsumptionSizeOriginalHHsFDEstimationResults.tex}};
%\input{c:/dropbox/data/ramadan/save/tablecolortemplate.tex}
\end{tikzpicture}\\
\renewcommand{\arraystretch}{.8}
\setlength{\tabcolsep}{1pt}
\begin{tabular}{>{\hfill\scriptsize}p{1cm}<{}>{\hfill\scriptsize}p{.25cm}<{}>{\scriptsize}p{12cm}<{\hfill}}
Source:& \multicolumn{2}{l}{\scriptsize Estimated with GUK administrative and survey data.}\\
Notes: & 1. & First-difference estimates. A first-difference is defined as $\Delta x_{t+1}\equiv x_{t+1} - x_{t}$. Saving and repayment misses are taken from administrative data and merged with survey data at Year-Month of survey interviews. Intercept terms are omitted in estimating equations. Sample is continuing members and replacing members of early rejecters and received loans prior to 2015 Janunary. Consumption is annualised values. \\
& 2. & ${}^{***}$, ${}^{**}$, ${}^{*}$ indicate statistical significance at 1\%, 5\%, 10\%, respetively. Standard errors are clustered at group (village) level.
\end{tabular}
\end{minipage}

\hspace{-1cm}\begin{minipage}[t]{14cm}
\hfil\textsc{\normalsize Table \refstepcounter{table}\thetable: FD estimation of consumption, with vs. without a grace period\label{tab FD consumption4 original HH}}\\
\setlength{\tabcolsep}{1pt}
\setlength{\baselineskip}{8pt}
\renewcommand{\arraystretch}{.55}
\hfil\begin{tikzpicture}
\node (tbl) {\input{c:/data/GUK/analysis/save/Original1600/ConsumptionGraceOriginalHHsFDEstimationResults.tex}};
%\input{c:/dropbox/data/ramadan/save/tablecolortemplate.tex}
\end{tikzpicture}\\
\renewcommand{\arraystretch}{.8}
\setlength{\tabcolsep}{1pt}
\begin{tabular}{>{\hfill\scriptsize}p{1cm}<{}>{\hfill\scriptsize}p{.25cm}<{}>{\scriptsize}p{12cm}<{\hfill}}
Source:& \multicolumn{2}{l}{\scriptsize Estimated with GUK administrative and survey data.}\\
Notes: & 1. & First-difference estimates. A first-difference is defined as $\Delta x_{t+1}\equiv x_{t+1} - x_{t}$. Saving and repayment misses are taken from administrative data and merged with survey data at Year-Month of survey interviews. Intercept terms are omitted in estimating equations. Sample is continuing members and replacing members of early rejecters and received loans prior to 2015 Janunary. Consumption is annualised values. \\
& 2. & ${}^{***}$, ${}^{**}$, ${}^{*}$ indicate statistical significance at 1\%, 5\%, 10\%, respetively. Standard errors are clustered at group (village) level.
\end{tabular}
\end{minipage}


\begin{palepinkleftbar}
\begin{finding}
\textsc{\small Table \ref{tab FD consumption  original HH}} uses rd 2 - 4 data and shows an increase in per member consumption in rd 2 - 3 period. The estimates are imprecise for all interaction terms. Continued increases in consumption hints welfare gains, but do not differ by arms. Per member food consumption increases in rd 2- 3 period but decreases in rd 3 - 4 period.
\end{finding}
\end{palepinkleftbar}


\subsection{IGA}

\begin{Schunk}
\begin{Sinput}
adw2 <- readRDS(paste0(path1234, "admin_data_wide2.rds"))
iga <- adw2[, .(hhid, Arm, Date, iga11st, iga12nd, iga13rd)]
setnames(iga, c("hhid", "Arm", "Date", paste0("iga", 1:3)))
#table0(iga[, iga1])
#table0(iga[, iga2])
#table0(iga[, iga3])
setkey(iga, hhid, Date)
iga[, NumIGA := sum(!is.na(iga1)) + sum(!is.na(iga2)) + sum(!is.na(iga3)), by = .(hhid, Date)]
#iga[NumIGA == 0 & !is.na(iga1), ]
setkey(iga, NumIGA, iga1, iga2, iga3)
iga.unique <- unique(iga[, .(NumIGA, iga1, iga2, iga3)])
iga.unique <- iga[iga.unique, .N/48, by = .EACHI]
setnames(iga.unique, "V1", "N")
setorder(iga.unique, -NumIGA, -N, iga1, iga2, iga3)
setkey(iga, NumIGA, iga1, Arm)
igaArm.unique <- unique(iga[, .(NumIGA, iga1, Arm)])
igaArm.unique <- iga[igaArm.unique, .N/48, by = .EACHI]
setnames(igaArm.unique, "V1", "N")
setorder(igaArm.unique, -NumIGA, -N, iga1)
for (i in 1:3) {
  iga[, paste0("IGA.", i) := as.character(NA)]
  iga[grepl("Cow|oxen", eval(parse(text = paste0("iga", i)))), 
    paste0("IGA.", i) := "cow"]
  iga[grepl("Goa|heep", eval(parse(text = paste0("iga", i)))), 
    paste0("IGA.", i) := "goat sheep"]
  iga[grepl("small", eval(parse(text = paste0("iga", i)))), 
    paste0("IGA.", i) := "small trade"]
  iga[grepl("house|land", eval(parse(text = paste0("iga", i)))), 
    paste0("IGA.", i) := "house land"]
  iga[grepl("machi", eval(parse(text = paste0("iga", i)))), 
    paste0("IGA.", i) := "machinery"]
  iga[grepl("addy|nut", eval(parse(text = paste0("iga", i)))), 
    paste0("IGA.", i) := "paddy nuts corn"]
  iga[, paste0("IGA.", i) := factor(eval(parse(text = paste0("IGA.", i))), 
    levels = c("cow", "goat sheep", "machinery", "small trade", "house land", "paddy nuts corn", NA))]
}
setkey(iga, NumIGA, IGA.1, IGA.2, IGA.3, Arm)
iga.unique3 <- unique(iga[, .(NumIGA, IGA.1, IGA.2, IGA.3, Arm)])
iga.unique3 <- iga[iga.unique3, .N/48, by = .EACHI]
setnames(iga.unique3, "V1", "N")
setorder(iga.unique3, -NumIGA, -N, Arm, IGA.1, IGA.2, IGA.3)
iga.unique3[, NumIGA := factor(NumIGA, levels = 3:0)]
\end{Sinput}
\end{Schunk}
\begin{Schunk}
\begin{Sinput}
library(ggplot2)
ggplot(data = iga.unique3[NumIGA != 0 & !is.na(IGA.1), ], aes(IGA.1, N)) + 
  geom_col() +
  xlab("First IGA choices") +
  theme(axis.text.x = element_text(angle = 30, vjust = .5, hjust = 1), 
    strip.text.y = element_text(colour = "blue"))+
  facet_grid(NumIGA ~ Arm, switch = "y")
\end{Sinput}
\begin{figure}

{\centering \includegraphics[width=\maxwidth]{figure/ImpactEstimationOriginal1600/IGA_choices_original_HHs-1} 

}

\caption{Income generatng activity choices\\ {\footnotesize The first income generating activity choices are plotted.\setlength{\baselineskip}{8pt}}}\label{Figure IGA choices original HHs}
\end{figure}
\end{Schunk}
\begin{Schunk}
\begin{Sinput}
iga.unique3[, num := 1:.N]
igaUL <- reshape(iga.unique3, direction = "long", idvar = c("num", "NumIGA", "Arm", "N"),
  varying = paste0("IGA.", 1:3))
setnames(igaUL, "time", "rank")
setkey(igaUL, num, rank)
library(ggplot2)
ggplot(data = igaUL[NumIGA != 0 & !is.na(IGA), ], aes(IGA, N)) + 
  geom_col() +
  xlab("First IGA choices") +
  theme(axis.text.x = element_text(angle = 30, vjust = .5, hjust = 1), 
    strip.text.y = element_text(colour = "blue"))+
  facet_grid(NumIGA ~ Arm, switch = "y")
\end{Sinput}
\begin{figure}

{\centering \includegraphics[width=\maxwidth]{figure/ImpactEstimationOriginal1600/All_IGA_choices_original_HHs-1} 

}

\caption{All income generatng activity choices\\ {\footnotesize All of multiple investment choices are summed by arms and the number of IGAs and plotted as bars. \setlength{\baselineskip}{8pt}}}\label{Figure All IGA choices original HHs}
\end{figure}
\end{Schunk}
\begin{Schunk}
\begin{Sinput}
iga.unique3[, num := 1:.N]
igaUL <- reshape(iga.unique3, direction = "long", idvar = c("num", "NumIGA", "Arm", "N"),
  varying = paste0("IGA.", 1:3))
setnames(igaUL, "time", "rank")
setkey(igaUL, num, rank)
library(ggplot2)
ggplot(data = igaUL[NumIGA != 0 & !is.na(IGA), ], aes(IGA, N)) + 
  geom_col() +
  xlab("First IGA choices") +
  theme(axis.text.x = element_text(angle = 90, vjust = .5, hjust = 1), 
    strip.text.y = element_text(colour = "blue"))+
  facet_grid(. ~ Arm, switch = "y")
\end{Sinput}
\begin{figure}

{\centering \includegraphics[width=\maxwidth]{figure/ImpactEstimationOriginal1600/All_IGA_choices_collapsed_original_HHs-1} 

}

\caption{All income generatng activity choices collapsed over different number of IGAs\\ {\footnotesize All of multiple investment choices are summed by arms and plotted as bars. \setlength{\baselineskip}{8pt}}}\label{Figure All IGA choices collapsed original HHs}
\end{figure}
\end{Schunk}

\begin{palepinkleftbar}
\begin{finding}
\textsc{\small Figure \ref{Figure IGA choices}, \ref{Figure All IGA choices}} show that there are very few members who chose to invest in more than one project for the ``large'' arms, while in the \textsf{traditional} arm, almost no one invested only in one project. Goat/sheep and small trades are the top choices for the first IGA in \textsf{traditional}. This indicates the exitence of both a liquidity constraint and convexity in the production technology of large domestic animals. This also validates our supposition that dairy livestock production is the most preferred and probably the only economically viable investment choice. It reduces a concern that the \textsf{cow} arm may have imposed an unnecessary restriction in an investment choice by forcing to receive a cow. \textsc{\small Figure \ref{Figure All IGA choices collapsed}} shows there are a significant number of cases in the \textsf{traditional} arm that members reportedly raise cows, yet they are also accompanied by pararell projects in smaller livestock production and small trades. Contrasting \textsf{large}, \textsf{large grace} with \textsf{cow} arms, it suggests that entrepreneurship (to the extent that is necessary for dairy livestock production) may not be an impediment for a microfinance loan uptake among members.
\end{finding}
\end{palepinkleftbar}

Together with \textsc{\small Table \ref{tab FD saving}} showing smaller net saving and repayment among \textsf{traditional}, the restriction on a project choice induced by a smaller loaned sum resulted in smaller returns. Between with or no grace period loans, cumulative net saving and repayment are both larger with loans with a grace period. No such difference is found between \textsf{cow} and other arms.


\end{document}
