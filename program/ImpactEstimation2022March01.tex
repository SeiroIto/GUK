%  path0 <- "c:/data/GUK/"; path <- paste0(path0, "analysis/"); setwd(pathprogram <- paste0(path, "program/")); system("recycle c:/data/GUK/analysis/program/cache/ImpactEstimation2022March01/"); library(knitr); knit("ImpactEstimation2022March01.rnw", "ImpactEstimation2022March01.tex"); system("platex ImpactEstimation2022March01"); system("pbibtex ImpactEstimation2022March01"); system("dvipdfmx ImpactEstimation2022March01")




\input{c:/seiro/settings/Rsetting/knitrPreamble/knitr_preamble.rnw}
\newcounter{armindex}
\renewcommand\Routcolor{\color{gray30}}
\newcommand{\SetLengthSkip}[1]{\setlength{\baselineskip}{#1}}
\newtheorem{finding}{Finding}[section]
\makeatletter
\g@addto@macro{\UrlBreaks}{\UrlOrds}
\newcommand\gobblepars{%
    \@ifnextchar\par%
        {\expandafter\gobblepars\@gobble}%
        {}}
\newenvironment{lightgrayleftbar}{%
  \def\FrameCommand{\textcolor{lightgray}{\vrule width 1zw} \hspace{10pt}}% 
  \MakeFramed {\advance\hsize-\width \FrameRestore}}%
{\endMakeFramed}
\newenvironment{palepinkleftbar}{%
  \def\FrameCommand{\textcolor{palepink}{\vrule width 1zw} \hspace{10pt}}% 
  \MakeFramed {\advance\hsize-\width \FrameRestore}}%
{\endMakeFramed}
\newcommand{\gettikzxy}[3]{%
  \tikz@scan@one@point\pgfutil@firstofone#1\relax
  \edef#2{\the\pgf@x}%
  \edef#3{\the\pgf@y}%
}
\def\BibTeX{{\textrm{B}\kern-.05em{\textsc{i}\kern-.025em \textsc{b}}\kern-.08em
    \textrm{T}\kern-.1667em\lower.7ex\hbox{\textrm{E}}\kern-.125em\textrm{X} }}
\def\BGColour{green!10}
\makeatother
\AtBeginDvi{\special{pdf:tounicode 90ms-RKSJ-UCS2}}
\special{papersize= 209.9mm, 297.04mm}
\usepackage{caption}
\usepackage{setspace}
\usepackage{tocloft}% http://ctan.org/pkg/tocloft
\setlength{\cftsecnumwidth}{2.5em} % Set length of number width in ToC for \section
\setlength{\cftsubsecnumwidth}{3em} % Set length of number width in ToC for \subsection
\setlength{\cftsubsubsecnumwidth}{3.75em} % Set length of number width in ToC for \subsubsection

\usepackage{framed}
\usepackage[framemethod=TikZ]{mdframed}
\captionsetup[figure]{font={stretch=.6}} 
\def\pgfsysdriver{pgfsys-dvipdfm.def}
\usepackage{tikz}
\usetikzlibrary{intersections, calc, arrows, decorations, decorations.pathreplacing, backgrounds, positioning, fit, shadows}
\usepackage{pgfplots, pgfplotstable}
\usepgfplotslibrary{fillbetween}
\pgfplotsset{compat=1.3}
\usepackage{adjustbox}
\tikzstyle{toprow} =
[
top color = gray!20, bottom color = gray!50, thick
]
\tikzstyle{maintable} =
[
top color = blue!1, bottom color = blue!20, draw = white
%top color = green!1, bottom color = green!20, draw = white
]
\tikzset{
%Define standard arrow tip
>=stealth',
%Define style for different line styles
help lines/.style={dashed, thick},
axis/.style={<->},
important line/.style={thick},
connection/.style={thick, dotted},
}
\mdfsetup{
linecolor=black!40,
outerlinewidth=1pt,
roundcorner=.5em,
innertopmargin=1ex,
innerbottommargin=.5\baselineskip,
innerrightmargin=1em,
innerleftmargin=1em,
backgroundcolor=blue!10,
%userdefinedwidth=1\textwidth,
shadow=true,
shadowsize=6,
shadowcolor=black!20,
frametitlebackgroundcolor=cyan!40,
frametitlerulewidth=10pt,
splittopskip=2\topsep
}
\global\mdfdefinestyle{SecItemize}{%
linecolor=black!40,
outerlinewidth=1pt,
roundcorner=1em,
innertopmargin=1ex,
innerbottommargin=.5\baselineskip,
innerrightmargin=1em,
innerleftmargin=1em,
backgroundcolor=blue!10,
%userdefinedwidth=1\textwidth,
shadow=true,
shadowsize=2,
shadowcolor=black!20,
frametitlebackgroundcolor=cyan!10,
frametitlerulewidth=10pt
}

\newcommand{\myquote}[1]{\begin{quotation}#1\end{quotation}}
\newcommand{\BGbox}[1]{\par\noindent\colorbox{lightblue}
{\parbox{\dimexpr\textwidth-2\fboxsep\relax}{#1}}}

\begin{document}
\setlength{\baselineskip}{12pt}

\hfil An escape from a poverty trap and the role of entrepreneurship:\\
\hfil Microfinance lending to the ultra poor in Bangladesh\\

\hfil\MonthDY\\
\hfil{\footnotesize\currenttime}\\

\hfil Seiro Ito\footnote{Corresponding author. IDE-JETRO. seiroi@gmail.com}, Takashi Kurosaki\footnote{Hitotsubashi University.}, Abu Shonchoy\footnote{Florida International University.}, Kazushi Takahashi\footnote{National Graduate Institute for Policy Studies.}\\

\hfil\mpage{10cm}{\footnotesize
\textsc{\textbf{Abstract}} \hspace{1em} The existing microcredit programs rarely lend to the ultra poor. With a randomised controlled trial in a rural, low income setting of northern Bangladesh, we assess the creditworthiness of the ultra poor and suitability of various debt contract designs to help them escape from poverty through productive investments. We use a stepped-wedge design over the key features of loans, i.e., small-scale sequential disbursement vs. lumpy upfront disbursement, with vs. without a grace period, and cash vs. in-kind loan with a managerial support program. Compared with the traditional microcredit, provision of large, upfront liquidity increases both repayment rates and net asset levels. This is consistent with the existence of an asset-based poverty trap which can be overcome by increasing the loan size. Provision of a grace period does not change the repayment rates or asset levels. We found that managerial supports induce participation of less experienced and poorer households to microfinance, yet resulted in similar repayment rates and asset accumulation as with other participants, indicating a further outreach to the ultra poor. For all households, labour incomes become larger towards the end of loan cycle while consumption stays the same, which we interpret as evidence of repayment discipline. %Given the lack of alternative lenders in the area, we argue that the high repayment rates need not generalise to other contexts. 
Our main findings, a large, upfront disbursement results in faster asset accumulation that is suggestive of an escape from a poverty trap and managerial support programs induce the participation of the ultra poor, are generalisable to other rural areas with liquidity constraints.
}

\newpage
\pagenumbering{roman}






\textbf{\textbf{Revisions}}

\vspace{2ex}
Overall changes:
\begin{enumerate}
\vspace{1.0ex}\setlength{\itemsep}{1.0ex}\setlength{\baselineskip}{12pt}
\item	For simplicity, I changed the names: net assets, narrow net assets $\Rightarrow$ broad net assets, net assets. Only the names are changed.
\item	Correspondingly, I keep only net assets in main results, and moved other asset concepts to robustness checks and appendices, except where we digress livestock holding and investments.
\end{enumerate}
Title and abstract:
\begin{enumerate}
\vspace{1.0ex}\setlength{\itemsep}{1.0ex}\setlength{\baselineskip}{12pt}
\item	No changes.
\end{enumerate}
Introduction:
\begin{enumerate}
\vspace{1.0ex}\setlength{\itemsep}{1.0ex}\setlength{\baselineskip}{12pt}
\item	No changes.
\end{enumerate}
Existing studies:
\begin{enumerate}
\vspace{1.0ex}\setlength{\itemsep}{1.0ex}\setlength{\baselineskip}{12pt}
\item	No changes.
\end{enumerate}
Theory:
\begin{enumerate}
\vspace{1.0ex}\setlength{\itemsep}{1.0ex}\setlength{\baselineskip}{12pt}
\item	No changes.
\end{enumerate}
Experimental design:
\begin{enumerate}
\vspace{1.0ex}\setlength{\itemsep}{1.0ex}\setlength{\baselineskip}{12pt}
\item	No changes
\end{enumerate}
Study sample:
\begin{enumerate}
\vspace{1.0ex}\setlength{\itemsep}{1.0ex}\setlength{\baselineskip}{12pt}
\item	Added preference related variables to descriptive statistics (p.\pageref{StudySamplePrefDescription}). This is necessary as we use these variables in permutation tests.
\myquote{\textsf{RiskPrefVal, TimePref1Val, TimePref2Val} are the minimum acceptable excess of risky options (vs. certainty), 3 month future options (vs. present), and 15 month future options (vs. 12 months in future), respectively, in monetary values. Smaller values indicate greater risk torelance and patience. \textsf{PresentBias} is the ratio of respondents who indicate present biasedness, \textsf{TimePref1Val} $>$ \textsf{TimePref2Val}. All these measures are similar across arms.}
\item	
\end{enumerate}
Results: 
\begin{enumerate}
\vspace{1.0ex}\setlength{\itemsep}{1.0ex}\setlength{\baselineskip}{12pt}
\item	In all tables, $p$ values are presented in percentage units ($100p\%$).
\item	Added time and risk preference estimates in permutation tests (\textsc{\normalsize Table \ref{tab MainTextIRjecters}}, \textsc{\normalsize Table \ref{tab main cownoncow}}). In the main text (p.\pageref{ParticipationPrefTexts}), added below:
\myquote{\textsf{RiskPrefVal} indicates that individual rejecters tend to demand higher compensation for risks, and the $p$ value becomes small enough only with entire sample of \textsf{All arms}. This suggests some individual rejecters are more risk averse than non rejecters. \textsf{TimePrefVal1, TimePrefVal2, PresentBias} all do not show statistically recognisable differences. }
\item	Added that the \textsf{Cattle} arm borrowers are more risk averse but participated, implying a role of managerial supports (p.\pageref{ParticipationPrefTexts}):
\myquote{In addition, \textsf{cattle} arm borrowers show stronger risk intorelance of as indicated in \textsf{TimePrefVal1, TimePrefVal2}. These features notwithstanding, the \textsf{cattle} arm, which provides managerial supports and in-kind lending, induced partcipation. As we will see in Section \ref{Sec Impacts}, the choice of lending instrument (cash or in-kind) does not matter in investments. So it is natural to infer that the managerial support component has induced the members with less experiences and fewer assets to take up loans.}
\item	Added \textsc{\footnotesize Figure \ref{fig NetAssets}: Net assets by period} as a part of descriptive statistics.
\item	For simplicity, keep only impacts on net assets in \textsc{\footnotesize Figure \ref{fig NetAssetEffects}}. Impacts on cattle holding and broad net assets are shown in the appendix.
\item	A subsection called Summary of impacts are added.
\myquote{In summary, we found that our managerial support programs induce the members of disadvantaged background to participate in microfinance, achieving the further outreach, and achieve the results that are no different with other borrowers. This is consistent with the finding of the previous studies that a certain level of skills is necessary for participation, and our managerial support progams supplemented the lack thereof. We found that the large upfront disbursements allows borrowers to invest in cattle while members with sequential disbursements mostly opted for smaller livestock and small trades. In combination with a greater return to cattle on net asset accumulation and a greater rate of loan repayment, we consider it as evidence of a poverty trap and an effective measure to break it. We also found the impacts and repayment rates are indistinguishable between the moderately poor and the ultra poor.}
\item	A subsection on Robustness check is added:
	\begin{itemize}
	\vspace{1.0ex}\setlength{\itemsep}{1.0ex}\setlength{\baselineskip}{12pt}
	\item	To examine impacts by previous cattle rearing experiences, \hfil\textsc{\footnotesize Figure \ref{fig AssetRelativeToCumulativeConcurrentTradEffectsByExperience}: Cumulative impacts on net assets relative to traditional arm by experience}, \textsc{\footnotesize Figure \ref{fig NumCowsRelativeToCumulativeConcurrentTradEffectsByExperience}: Cumulative impacts on cattle holding relative to traditional arm by experience} are added.
	
	\myquote{The previous literature has shown that returns to lending are higher for the borrowers with business experiences \citep{BanerjeeKarlanZinman2015}. To check if the same can be found in our experiment, we divide the subjects into three groups of different cattle rearing experiences  at the baseline: \textsf{Own} group, defined by the cattle ownership, \textsf{Adi} group, defined by no cattle ownership but having an experience with cattle lease contracts (called \textit{Adi}) up to 3 years prior to the baseline, and \textsf{None} group who has neither of the two. In \textsc{\footnotesize Figure \ref{fig AssetRelativeToCumulativeConcurrentTradEffectsByExperience}} and \textsc{\footnotesize Figure \ref{fig NumCowsRelativeToCumulativeConcurrentTradEffectsByExperience}}, we plot the group wise impacts by arm on the net assets and cattle holding, respectively. These show that the \textsf{Own} group has the highest returns in both outcomes under the \textsf{Large/Upfront} treatment, followed by the \textsf{None} group, and virtually no impact among the \textsf{Adi} group.
		
		Consistent with the previous litereture, we thus find the returns to microfinance are higher among the members with previous cattle rearing experience through ownership. We also find the returns among the members with no previous experience are small yet not statiscally zero. In particular, when we choose cattle holding as an outcome, \textsf{Cattle} arm has a statistically meaningful impact even among the \textsf{None} group, which is consistent with our main finding that the managerial support program may have helped them in participating and sustaining the level of returns. 

		In contrast, the \textsf{Adi} group, who has cattle rearing experiences, does not attain statistically positive returns. As one examines the estimated results in \textsc{\normalsize Table \ref{tab ANCOVA narrow net assets Experience timevarying 1}} and \textsc{\normalsize Table \ref{tab ANCOVA narrow net assets Experience timevarying attributes 1}}, we see that the returns of \textsf{Adi} group becomes negligible once we add baseline household size \textsf{HHsize0} as a covariate, which has large positive estimates across specifications. This is indicative of \textsf{Adi} group is constrained by the household size, which is consistent with our main finding that a domestic capacity constraint, be it domestic labor or house building size or both, may bind some households to attain positive returns	, possibly because they are already engaging in cattle rearing.
	}
	\item	Referred to robustness of results regardless of asset concepts.
	\myquote{
		We also ran a robustness check over the choice of asset concepts by using various measures of net assets: \textsf{Broad net assets} which we include all other household assets that are observed in certain rounds of surveys, \textsf{Broad net assets, annual price} which we use annual median price of cattle in computing the livestock values in broad net assets, \textsf{Net non livestock assets} which we drop livestock values from net assets, \textsf{Land} which is the total of land asset values, and \textsf{Cattle} which is the number of cattle holding.  In the Appendix \textsc{\footnotesize Figure \ref{fig AssetCumRelativeToConcurrentTradEffects}}, we show the time paths of various assets by arm. The dynamic patterns of asset accumulation is similar to \textsc{\footnotesize Figure \ref{fig NetAssets}}.\footnote{\textsf{Net assets}=total assets - debts. Debts include outstanding loaned amount of the experiment. Total assets use items observed in all 4 rounds of household surveys. \textsf{Broad net assets}=total broad assets - debts, where total broad assets use all assets observed in the household surveys. \textsf{Broad net assets annual price} use annual median price for computing livestock values. \textsf{Net NL assets}=\textsf{net assets}-livestock asset values, \textsf{Broad net NL assets}=\textsf{Broad net assets}-livestock asset values. \textsf{Net assets, broad net assets} uses median baseline price for livestock.  } 

		All asset measures show similar patterns (see \textsc{\footnotesize Figure \ref{fig ImpactsOnAllAssetsCumRelativeToConcurrentTradEffects}}). \textsf{Broad net assets} show a similar pattern yet the standard error bars cross zero in round 4 for some specifications, possibly because of larger noises in computing the values as some asset items are observed only in certain rounds. \textsf{Net non livestock assets} show smaller impacts that are not statistically distinguishable from zero, which is consistent with our findings that borrowers used the funds in productive investments and kept repayment efforts. 
	}
	\end{itemize}
\end{enumerate}
Conclusion:
\begin{enumerate}
\vspace{1.0ex}\setlength{\itemsep}{1.0ex}\setlength{\baselineskip}{12pt}
\item	No changes.
\end{enumerate}
Appendix:
\begin{enumerate}
\vspace{1.0ex}\setlength{\itemsep}{1.0ex}\setlength{\baselineskip}{12pt}
\item	Dropped the word `survival' from table headers of 
\textsc{\normalsize Table \ref{}}, 
\textsc{\normalsize Table \ref{tab1 Permutation test results of non-attriting borrowers of cattle and large grace arms}}, 
\textsc{\normalsize Table \ref{tab1 Permutation test results of non-attriting borrowers of cattle and all other arms}}. 
\item	Revised the description of these tables. I made clear that:
	\begin{itemize}
	\vspace{1.0ex}\setlength{\itemsep}{1.0ex}\setlength{\baselineskip}{12pt}
	\item	\textsc{\normalsize Table \ref{}} compares potential MFI targets (nonattriting borrowers, noted as \textsf{Active}) vs. non-targets (attriting borrowers or loan rejecters, noted as \textsf{NonActive}), and it is about if screening variables have any predictive power in terms of loan rejection or attrition under our lending. 
	\item	For the other two tables, it is about the managerial support assisted non-attrition of less advantaged borrowers. \\~\\
	\end{itemize} 
\end{enumerate}
\textbf{Requests for Abu-san}
\begin{enumerate}
\vspace{1.0ex}\setlength{\itemsep}{1.0ex}\setlength{\baselineskip}{12pt}
\item		Our sample is drawn from the population of river island villages in Northern Bangladesh. Abu-san, please provide the regional characteristics of the area, esp. poverty, using CLP/TUP program data and reports. 
	\begin{itemize}
	\vspace{1.0ex}\setlength{\itemsep}{1.0ex}\setlength{\baselineskip}{12pt}
	\item	Please provode a succinct description.
	\end{itemize}
\item	A leading proponent is the nobel laureate Professor Mohammad Yunus who claims that ``we are all entrepreneurs.'' \citet{Yunus2003}, \citet{Cosic2017} \textcolor{red}{[Abu-san: Do you have access to library to get the exact page number(s) in his book?}
	\begin{itemize}
	\vspace{1.0ex}\setlength{\itemsep}{1.0ex}\setlength{\baselineskip}{12pt}
	\item	Please get the information when everything is settled down.
	\end{itemize}
\end{enumerate}


\textbf{\textbf{Contrasts with \citet{Balboni2020}}}
\setlength{\parindent}{1em}

	\citet{Balboni2020} collect data from transfer recipients and control group of BRAC's TUP. Using the recipient data, they estimate the equation of motion $K_{t+1}=\phi(K_{t})$, show the S shape, and compute the threshold asset level $\hat{k}$ that separates the low and high equilibria. They then show that individuals who are above $\hat{k}$ increase the assets while individuals below it decrease them. The variations of initial asset level allows the identification of bifurcation as these variations effectively allocate inidividuals to below and above the threshold. Anticipating the endogeneity of initial asset levels to asset growths, they show that initial asset levels have no correlation with post intervention asset growths after conditioning on the above-threshold dummy. This is suggestive evidence that the unobservables that correlate with initial asset levels are exogenous to post-transfer asset increases.

	In the poverty trap dynamics, the key is the low returns among the low $k$ holders. The returns to high $k$ holders are qualitatively similar in the covex and concave production functions so long as they are above the 45 degree line. Why are they low? Authors show the baseline vehicle ownership is statistically smaller by 4\% (but not for other assets, total assets are not tested) for the below threshold households, and the differences relative to the above threshold households grew after the transfer receipt. They conclude that the complimentary assets serve as the fixed inputs of production, and the lack thereof withholds households from escaping the poverty trap.

\begin{description}
\vspace{1.0ex}\setlength{\itemsep}{1.0ex}\setlength{\baselineskip}{12pt}
\item[their strength]	Large sample size, precise asset and labour data, direct estimation of equation of motion, and associated tests of multiple equilibria.
\item[our strength]	Experimental variations in contract design (\textsf{Upfront}, support programs), use of IGA information that allows the (poverty trap) interpretations without structural estimation, reference to market costs/prices. 
\end{description}

\noindent
\textbf{\textbf{Contrasts with \citet{Banerjee2019MFPovertyTrap}}}

	\citet{Banerjee2019MFPovertyTrap} use regionally matched-pair data under staggered branch opening of an urban MFI. They divide the sample into borrowers with a prior business experience (GE) and others (non-GE), and show that impacts are persistently positive for GE borrowers but not for non-GE borrowers. With structural estimation, given a talent distribution, they interpret this as evidence of a poverty trap through a liquidity constraint. They also note the impact heterogeneity is due to MFI selection but not talent heterogeneity, as pre-MFI entry businesses are more profitable than post-MFI entry businesses of the same firm age. 

\begin{description}
\vspace{1.0ex}\setlength{\itemsep}{1.0ex}\setlength{\baselineskip}{12pt}
\item[their strength]	Urban setting, contrast of long-run versus short-run impacts, data on business outcomes, gross substitute/compliment with demand for informal loans, explaining the lack of average impacts by finding the subpopulation with superior talents and contrasting with other subpopulation.
\item[our strength]	Ultra poor population, rural and fragile setting, selection on entrepreneurship without affecting outcomes, \textsf{Upfront} leads to faster asset accumulation and higher repayment rates.
\end{description}

\noindent
\textbf{What the three papers agree}

\begin{itemize}
\vspace{1.0ex}\setlength{\itemsep}{1.0ex}\setlength{\baselineskip}{12pt}
\item	A need for larger lending than regular MF.
\item	Existence of a poverty trap.
\item	Evidence of a nonconvex production set as a source of poverty trap.
\end{itemize}

\newpage
\pagenumbering{arabic}
\setcounter{tocdepth}{3}
\tableofcontents
\thispagestyle{empty}\setcounter{page}{0}
\newpage

\setlength{\parindent}{1em}
\vspace{2ex}




\renewcommand{\thefootnote}{*\arabic{footnote}}
\setcounter{footnote}{0}



\section{Introduction}
\label{SecIntroduction}


	Since the microcredit became popular in Bangladesh in the late 1980's, the number of borrowers increased rapidly throughout the world. According to over 3700 microfinance institutions (MFIs), there are estimated 204 million borrowers around the world in 2013, of which 110 million are the poor borrowers whose incomes are below the national poverty line \citep{MFGateway2015}. The outreach to the extremely poor population or the \textit{ultra poor}, however, is arguably slow in comparison.\footnote{MF is not successful in reaching out to the poorest of the poor, or the ultra poor \citep[][]{Scully2004}. Empirical evidence in \citet{Yaron1994, Navajas2000, RahmanRazzaque2000, AghionMorduch2007} supports this claim. Some authors discuss the tradeoff between sustainability and outreach for microfinance institutions (MFIs) \citet{HermesLensink2011, HermesLensinkMeesters2011, Cull2011}. } 

	There are demand and supply side reasons behind the slow outreach to the ultra poor. On the demand side, the ultra poor borrowers may not be entrepreneurial enough to demand credits for production, or may face an inferior production possibility than the wealthier borrowers. On the supply side, MFIs may perceive the ultra poor as riskier than the moderately poor, or their loan size may be too small to justify the fixed transaction costs while the lender is constrained to keep the interest rate low to avoid adverse selection and moral hazard. 

	As the rigorous evaluations of microfinance progress, it has become clear that the impacts are not uniformly positive. A group of influential research has shown that only a subgroup of borrowers, those with prior experiences or high ability, have positive returns from borrowing \citep{Banerjee2015Miracle, Mckenzie2017Spurring, Buera2017, Banerjee2019MFPovertyTrap}. This is in a stark contrast to the popular belief in microfinance that anyone can become a successful borrower.\footnote{A leading proponent is the nobel laureate Professor Mohammad Yunus who claims that ``we are all entrepreneurs.'' \citep{Yunus2003}, \citep{Cosic2017} \textcolor{red}{[Abu-san: Can you get the exact page number(s) in his book?} } Logically, there must be some minimal level of entrepreneurship to participate and continue as a borrower in any form of finance. Then, the question is, what sort or how much of entrepreneurship is required in microfinance? 
	
	To shed light on the required level of entrepreneurship, we took this question to the Northern Bangladesh where a flood threat limits the leading production process to be least complex: livestock rearing. Majority of residents own livestock so its know-how is semi-public knowledge. The required entrepreneurship, then, is to gather all the pieces of relevant information, decide to raise livestock, form a production and a sales plan, and implement. This is the definition of entrepreneurship we use in our paper. In our intervention, we provided a heifer to a selected subgroup of participants as an in-kind loan and bundled it with training and consultation services to make sure the borrower has the right cookbook to follow. Under this treatment, the entrepreneurship to decide what to invest and how to come up with a solid plan is no longer a necessity.

	In our study, we compare the borrowers who were provided with such knowledge and the managerial supports with the borrowers who were not. By randomising the offers, we identify the causal impacts of not requiring the entrepreneurship on both the participation and the outcomes. We find that our managerial support program, that helps fill the gap in entrepreneurial skills, induces more residents with fewer experiences in livestock production and a lower asset level to participate while keeping the mean outcomes the same as in the comparison group.%, despite the lower qualification. 

	There is yet another motivation for our study. In bridging these two experimental arms that are different in multiple aspects, we introduced intermediate arms. At the end, we were left with an arm of conventional microcredit that disburses small upfront liquidity for three times, and several arms with large upfront liquidity that disburse the equivalent total amount once under three period maturity. This gives an opportunity to test if the upfront liquidity provision, while keeping the total loan size and maturity equivalent, matters in the future asset levels. If the production technology is nonconvex and if there is a credit constraint, it gives rise to a poverty trap which can result in larger asset accumulation when provided with large upfront liquidity. While we do not directly test for a poverty trap, the investment choices strongly indicate its existence: Only borrowers without a large upfront loan opted for smaller, multiple investments. Our experimental design tests if the upfront liquidity provision breaks a poverty trap, under the assumption that there is one, and found that it results in larger asset accumulation by 48.9 to 52.3 percentage points without affecting the repayment rates.

	Our study follows the literature of microfinance debt contract design as hallmarked in \citet{Field2013} who found a grace period induces more risk taking and subsequent loan delinquency. Under our setting of limited production choices, it is irrational to invest in riskier assets, such as goats, when the designed grace period suits the heifer cash flow and a heifer's risk-return profile is considered to be Pareto-dominating. A strategic default is also more difficult in our setting because the number of formal credit suppliers is limited, which is probably zero,\footnote{As we surveyed the area before the study, we note several NGOs provide a relief credit to flood victims, but not regular finance. In selecting the study site, we purposefully chose the population without access to any financial institution.\textcolor{red}{[Abu-san: A better description for this?]} } and relocation is costly, because it requires not just boat rentals with a certain loading capacity during the daytime but also financially reliable mainland contacts to settle in. The repayment rates in our study turned out to be no lower than the comparable microfinance schemes \citep{BanerjeeKarlanZinman2015}.

	Our study is closely related to a large scale cattle transfer study conducted in the neighbouring area \citep{BandieraBRAC2017, Balboni2020}. The targeted population of their study is similar to ours, yet our study population resides on less stable terrain, are more exposed to flood and water logging, are considered to be less well connected to the market, are equally less trained, and are probably poorer. The chance of survival for each investment project is expected to be no higher. More prominently, our study is commercially oriented: It uses a loan than a transfer, and charge market level fees to all the services provided. 
	



	We find that borrowers of the arm with managerial supports have lower cattle holding 0.22 per household (while borrowers from other arms have .308, $p$ value = .156), and smaller net asset values BDT 5603 (in contrast to BDT 8204 in other arms, $p$ value = .058). The outcomes and repayment rates are no lower than the other arms, implying the managerial supports had a further outreach without compromising the outcomes. We also find that, relative to the traditional microfinance lending, the upfront liquidity provision increases the net assets by 
	1.06 times the baseline standard deviation (denoted hereafter with $\sigma$) in the second year, 
	1.20$\sigma$ by the end of fourth year, 
	and the number of cattle holding by 0.59$\sigma$ in the second year, and 
	0.58$\sigma$ by the end of fourth year. 
	These results hold broadly regardless of variable cattle rearing experiences prior to the intervention.

%	round(confi[grepl("Nar", FileName) & grepl("cum", ImpactType) & grepl("ge$", attributes) & grepl("T$", regtype) & num == 2 & period == 2,  estimate], 0) 
%	(CI round(confi[grepl("Nar", FileName) & grepl("cum", ImpactType) & grepl("ge$", attributes) & grepl("T$", regtype) & num == 2 & period == 2,  .(lb, ub)], 0)) in the second year, 
%	BDT round(confi[grepl("Nar", FileName) & grepl("cum", ImpactType) & grepl("ge$", attributes) & grepl("T$", regtype) & num == 2 & period == 4,  estimate], 0) 
%	(CI round(confi[grepl("Nar", FileName) & grepl("cum", ImpactType) & grepl("ge$", attributes) & grepl("T$", regtype) & num == 2 & period == 4,  .(lb, ub)], 0)) by the end of fourth year, 
%	and the number of cattle holding by round(confi[grepl("NumCow", FileName) & grepl("cum", ImpactType) & grepl("ge$", attributes) & grepl("T$", regtype) & num == 2 & period == 2,  estimate], 2) 
%	(CI round(confi[grepl("NumCow", FileName) & grepl("cum", ImpactType) & grepl("ge$", attributes) & grepl("T$", regtype) & num == 2 & period == 2,  .(lb, ub)], 2)) in the second year, and 
%	round(confi[grepl("NumCow", FileName) & grepl("cum", ImpactType) & grepl("ge$", attributes) & grepl("T$", regtype) & num == 2 & period == 4,  estimate], 2) 
%	(CI round(confi[grepl("NumCow", FileName) & grepl("cum", ImpactType) & grepl("ge$", attributes) & grepl("T$", regtype) & num == 2 & period == 4,  .(lb, ub)], 2)) by the end of fourth year. 
	
	We consider our finding is generalisable to rural areas where small scale livestock production is prevalent. While there is a caveat that the domain of our results is a low level herd size and the entrepreneurial capacity to hold a larger herd size can be different from what our study suggests,\footnote{Our study matches the scale of lower equilibrium of \citet{Lybbertetal2004} which is much smaller than the scale of the high equilibrium of around 50 herd size. } the successful livestock transfer program in the neighbouring areas \citep{BandieraBRAC2017, Balboni2020} and our results indicate that supporting asset accumulation through large livestock has wide applicability in assisting the rural ultra poor to escape from poverty.


	In the following section, we summarise the existing literature. Section \ref{SecExistingStudies} gives the link to the previous literature. Section \ref{SecBackground} gives the brief account of background of study site. Section \ref{SecExperimentalDesign} lays out the details of experimental design. Section \ref{SecEmpiricalStrategy} explains the estimation strategy. In section \ref{SecResults}, we provide the experimental results and contents of income generating activities (IGAs). Section \ref{SecTheory} shows a possible mechanism of poverty trap that our target population is under. Section \ref{SecConclusion} discusses the interpretation of results.

\section{A brief review of existing studies}
\label{SecExistingStudies}

	There are four aspects in our study that relate to the existing literature: The role of entrepreneurship in microfinance impacts, variations in debt contract design, empirical assessment of a poverty trap, and targeting the ultra poor. We will discuss these in turn.

	Much has been discussed about the poverty reduction impacts of microfinance in the early days of microfinance studies \citep{PittKhandker1998, Morduch1999}. Recently, doubts are cast on the magnitude of microfinance impacts \citep{BanerjeeKarlanZinman2015, DuvendackMader2019, Meager2019} while asset grants (capital injection) remain to show high returns \citep{deMel2008, DeMel2014, FafchampsFlypaper2014, BandieraBRAC2017}. \footnote{This is due partly to insufficient statistical power \citep{MckenzieWoodruff2013}. \citet{BanerjeeKarlanZinman2015} collects six studies of microfinance lending impacts. They also point the lack of statistical power due to low take up while noting more able and experienced borrowers saw larger ``transformative effects.'' %However, one must be careful with a low statistical power study as it naturally gives a way to erroneously large impacts. 
	In the current study, in contrast, the take up rate is relatively high at 74.32\%, of which 5.16\% is lost to the flood.  } Lack of mean impacts in microcredit led researchers to look for a particular subgroup which shows impacts, or impact heterogeneity \citep{Banerjee2017HyderabadFollowup}: Borrowers with prior experiences or high ability are shown to have higher returns \citep{Banerjee2015Miracle, Mckenzie2017Spurring, Buera2017, Banerjee2019MFPovertyTrap}. The studies with a focus on experienced members or existing firms can be considered as looking at impacts on the intensive margins. In contrast, our study is focused on an isolated greenfield population, or poverty impacts on the extensive margins, which are relatively less studied.

	The fact that experienced members gain larger benefits from microcredit is consistent with the positive impacts of capital grant programs on existing firm owners. Whether such experience is trainable for novice entrepreneurs remains unsettled. A recent microfinance study indicates that there is an advantageous selection through talents in the existing firm owners, so trainability is called into a question \citep{Banerjee2019MFPovertyTrap}. A growing body of management capital literature in developing countries is insightful yet most of the research is necessarily geared to existing firms, so it does not inform much on how one can assist novice entrepreneurs.\footnote{\citet{BruhnKarlanSchoar2018} shows intensive management consulting services to the small scale firms in Mexico resulted in sustained improvements in management practices which led to higher TFP and larger employment. Others also show effectiveness \citep{Calderon2011, Berge2012, Bloometal2013} while others do not \citep{Bruhn2012, KarlanKnightUdry2015}. \citet{MckenzieWoodruff2013} put them as: These managerial impacts studies are too different to compare, in terms of population, interventions, measurement (variables, timing), and most importantly, implied statistical power in the design. } \citet{KarlanValvidia2011, BruhnZia2011, Argent2014} are the exceptions, but results and quality of evidence are mixed and inconclusive. The current study explicitly tests if the entrepreneurship matters in microfinance by using a heifer lending with a manegerial support program. We also examine the self-selection on entrepreneurship into microcredit, which we find to exist.%Entrepreneurship and training components in the current study are to provide basic knowledge of dairy cattle production which can easily be written down. They, the cristalised intelligence, are outsourceable in nature. We consider it is the skills to deploy services in a timely manner, rather than the knowledge contents \textit{per se}, that we provide to help borrowers in increasing efficiency.

	The corporate finance devotes a substantial part of its field in understanding the consequences of contract designs on entrepreneur's incentives. \citet{Field2013} was the first to examine if the traditional lending style of microfinance inhibits the spawning of entrepreneurship by experimentally allocating different types of debt contracts.  %shows that a two-month grace period increases the investment size, raises profitability, but also increases the default rates. They discuss how it influences the investment riskiness that varies along risk preference heterogeneity. 	Compared to them, the experimental setting of the current study has a smaller investment opportunities that limits the scope of risk taking. 
	As we will disscuss in the Section \ref{SecExperimentalDesign}, our study follows the similar strategy. In an attempt to tease out the impacts of entrepreneurship, we introduced longer maturity and a grace period in other arms. While there was a strong concern among practitioners that a grace period induces untruthful borrowing, there was no alternative in borrowing other than relatives and money lenders due to ruralness and isolation. This gave us flexibility in designing the debt contracts. Similar to \citet{Beaman2015} who redesigned the repayment schedule to adapt the borrower's cash flow profile (repay after harvest), we designed the debt contract to best suit the cash flow profile of the most popular investment project in the area, rearing a heifer. %Fortunately, our program ended with relatively low delinquency rates compared to other programs, and there is even suggestive evidence of repayment discipline among the borrowers. 
	Our study exemplifies the economic gains from designing the debt contract to match the presumed investment choices in microfinance.

	Another strand of the literature related to our study links capital grant effectiveness with the production set nonconvexity. Theories base lumpiness and credit market imperfection as keys to a povety trap \citep[e.g., ][]{GalorZeira1993}. When the production set is nonconvex, a small scale transfer may not lead to a sustained increase in income, as it can be either consumed or invested to a technology with decreasing marginal returns that brings back to the original income level (i.e., the lower equilibrium of a poverty trap). 
	
	Despite its popularity as a theory, the empirical evidence of a poverty trap is mixed. \citet{KraayMcKenzie2014} note that a poverty trap finding is rare, while \citet{BarrettGargMcBride2016} state the otherwise and there is overwhelming evidence.\footnote{\citet{KraayMcKenzie2014} also note that upward transition from one poverty trap to another may negate the notion of a trap, while \citet{BarrettGargMcBride2016} base their affirmation by counting both the direct asset dynamics and the indirect inference that tests the behavioral responses that are consistent with poverty traps. See also \citet{CarterBarrett2006, BarrettCarter2013} for earlier evidence and discussions. } Interestingly, however, they both agree that, when there is a range of assets and production opportunities, it is inherently difficult to emipirically single out a particular poverty trap. The latter authors note that existing evidence comes mostly from remote and isolated areas with a single primary production opportunity and an associated asset. Our study is no exception. It comes from a remote and isolated area of northern Bangladesh where the single most important production opportunity to increase income in otherwise subsistence-oriented paddy producing villages is livestock production.  
	
	An earlier finding of a poverty trap includes the cattle herd size dynamics of Southern Ethiopian pastoralists that indicates existence of a poverty trap over a 17 year recall period \citep{Lybbertetal2004}. More recently, \citet{Balboni2020} estimate the equation of motion for assets and show the direct evidence of a poverty trap among the recipients of a large scale transfer program targeted in the neighbouring areas of our study site. The source of nonconvexity is cattle and the complimentary assets (vehicles) at the baseline which serve as a fixed input that the ultra poor cannot afford. Similar to these studies, our study examines the nonconvexity of a higher-return production set. Our study regresses the future asset values on the current asset values and intervention dummies, thereby adding evidence, in the \citet{BarrettGargMcBride2016}'s terminology, using the \textit{direct method}. %The estimation shows elevated returns to frontloading, which is consistent with a poverty trap but does not constitute as direct evidence. So we turn to investment choices and find that the large upfront liquidity is mostly associated with a heifer purchase while the small upfront liquidity is associated with smaller livestock purchase and other small business trading which are known to have lower returns and higher risks, and only at a later point in time they acquire a heifer. 
	By complementing this estimated result with the fact that borrowers purchase cattle only when large upfront liquidity is provided, we conclude that there is a poverty trap. In our study, the source of nonconvexity is the price of a heifer that is about three times the price of a goat. We also show that frontloading the liquidity in lending is effective in escaping the poverty trap.

	Lastly, selecting the ultra poor as the population to provide supports have often involved free consultation/training and transfers in the past. A handful of studies on ultra poor transfer programs report sustained increase in assets and incomes \citep{Blattman2014, BanerjeeetalScience2015, Blattmanetal2016, HaushoferShapiro2016}. A transfer program in the Northern Bangladesh shows an occupational change and an income increase \citep{BandieraBRAC2017} and long-run asset accumulation \citep{Balboni2020}. %\citet{Kaboski2018Indivisibility} uses a lab-in-the-field experiment to show the link between investment indivisibility, saving, and patience. 
	In an attempt to test commecial viability, our experimental design makes a reference to markets. It uses loans rather than transfers, and any training and consulting components charge a fee for services. The resulting repayment rates are not lower than the majority of representative microfinance programs, and we also find significant accumulation of assets. These results show that the costs of microfinance programs targeted to the ultra poor can be reduced by using loans and charging fees. Use of loans and fees can increase the likelihood of long run viability often overlooked in transfer programs.\label{PageOfCostsMFReduced} Our study can be considered as an example of market based interventions that can play a role in ultra poor graduation programs.


\section{Background}
\label{SecBackground}


	The study area is in the river island, known as \textit{chars} in Bengali, of northern Bangladesh in Gaibandha and Kurigram districts. Chars are formed by sediments and silt depositions and are prone to cyclical river erosions and floods. Chars are not stable in size and even in existence, and episodes of their partial or complete erosion or submerging are common. Chars accommodate ultra-poor inhabitants who are forced, as a desperate attempt for survival, to relocate across islands due to river erosion and floods.

	In the study area, heifers are the prime investment choice. Female goats are considered to be a secondary choice by residents. A heifer needs to be at least two years old to start lactation.\footnote{They typically need to be about 15 months old to be ready for insemination and takes about 9.5 months to deliver a calf as it starts lactation, or the total of about 2 years. \label{heifer2years}} Rearing costs are higher for cattle as it requires fodder while a goat will eat the bushes. Cattle requires vaccination shots when a goat is usually left unvaccinated. Reproductive capacity of goats are high.\footnote{Parity size approaches to 2 at the third birth, and the birth interval is about 200 days \citep{Hasan2014goat}. An indigenous cow has a birth interval of 375 to 458 days \citep{Hasan2018}, resulting in about 2 years for gestation and calving interval \citep{Habib2012} with the mean lifetime births of 4 \citep[][Table 1]{Hasan2018}. } However, in comparison with cattle, their higher reproductive capacity and lower rearing costs are more than offset by the elevated morbidity and mortality risks,\footnote{Indeed, morbidity of goat kids ranges from 12\% \citep{Mahmud2015} to more than 50\% in some diseases \citep[][Table 5]{Nandi2011}, while cattle morbidity is around 22\% \citep{Bangar2013}. Goat kid mortality ranges from 6\% \citep{Mahmud2015} to 30\% \citep[][Table 5]{Paul2014} \citep{Ershaduzzaman2007}. Heifer mortality is between 5\% \citep[][p.332R]{Hossain2014} to 10\% \citep{Alauddin2018}. Higher morbidity of goat kids partly reflects their eating style that uses lips rather than tongues (as cattles do) and vulnerability to logging water. } and a less frequent cash flow.\footnote{The produce of goats is mostly meat and their milk is seldom marketed. A meat market requires a cluster of relatively high income earners, usually located far from river islands, and the demand is highly seasonal. In contrast, cow milk can be marketed locally with stable demand, the lactation length is 227 days, and milk yield is 2.2 kg per day \citep{Rokonuzzaman2009}. } Residents also report that a goat herd is less mobile than single cattle when they are forced to evacuate during the flood. All of these considerations prompt residents to opt for cattle when they can afford it, and do not expand the herd size of goats, which are both confirmed in our data.

\section{Theory}
\label{SecTheory}

\begin{figure}
\hfil\mpage{\linewidth}{
\hfil\textsc{\normalsize Figure \refstepcounter{figure}\thefigure: A poverty trap with goats and cattle\label{fig poverty trap}}\\

\hfil\begin{adjustbox}{width=.5\linewidth}
\input{GoatCowProdFunctions1.tkz}
\end{adjustbox}\\
\hfil\begin{adjustbox}{width=\linewidth}
\input{GoatCowProdFunctions3.tkz}\hspace{-2em}
\input{poverty_trapCompoundScurve3.tkz}
\end{adjustbox}

\noindent\hfil\mpage{12cm}{\footnotesize Note: In the top panel, the current period per capita asset size $k_{t}$ is on the horizontal axis, the current period production $y_{t}$ is on the vertical axis. For production set $j=\{\mbox{goats, cattle}\}$, the production becomes positive only after $k_{t}$ becomes greater than its fixed input portion $\underline{k}_{j}\in\mathbb R_{++}$. The positive production portion follows a decreasing return to scale production function for each $j$. In two bottom panels, period $t$ per capita asset size in monetary units $k_{t}$ is given on the horizontal axis and the period $t+1$ per capita asset size in monetary units $k_{t+1}$ on the vertical axis. In the bottom left panel, the production function for cattle $f(k)$ is multiplied with a fixed saving rate $s$ and is added current herd size net of mortality $(1-\delta)k_{t}$ that is passed on to the next period. The depreciation rate $\delta$ is applied in the fixed cost segment. Similar description applies to the goat production function $g(k_{t})$. Saving rate and depreciation rate are assumed to be common with the cattle production. The bottom left panel shows each production sets, the bottom right panel shows the contour of two production sets. The point $C$ exists as long as the Inada condition $\lim\limits_{k_{t}\rightarrow \infty} f'(k_{t})=0$ is met. Red points are stable equilibria, blue points are unstable equilibria. \setlength{\baselineskip}{8pt} }
}
\end{figure}

\vspace{2ex}
	In this section, we use a simplified version of \citet{GalorZeira1993} to illustrate a theoretical framework to aid the interpretation of the empirical finding that asset accumulation is faster while the repayment rate is higher for upfront lending. Let us consider that there are two production sets called `goats' and `cattle.' Both sets are nonconvex with fixed inputs as shown in \textsc{\small Figure \ref{fig poverty trap} (top panel)}. 
	In the top panel, the current period per capita asset size in monetary units $k_{t}$ is shown on the horizontal axis, the current period production in monetary units $y_{t}$ is shown on the vertical axis. For production set $j=\{\mbox{goats, cattle}\}$, the production becomes positive only after $k_{t}$ becomes greater than its fixed input portion $\underline{k}_{j}\in\mathbb R_{++}$.  The production after $\underline{k}_{j}$ follows a decreasing return to scale technology. 
	
	In the two bottom panels, period $t$ per capita asset size in monetary units $k_{t}$ is given on the horizontal axis and the period $t+1$ per capita asset size in monetary units $k_{t+1}$ is given on the vertical axis. We take cattle production as an example. In the bottom left panel, saving out of production is given by the dotted line $sf(k_{t})$ with a fixed saving rate $s\in(0, 1)$. Saving is zero for the flat segment, and becomes positive once the production becomes positive. For $k_{t}>\underline{k}_{\mbox{\scriptsize cattle}}$, the saving traces the cattle production set after rescaling with the saving rate $s$, or $sf(k)$. 
	
	The next period net per capita asset size is given by the sum of saving and carry over asset net of depreciation (including mortality) $(1-\delta)k_{t}$.  The depreciation rate $\delta\in(0, 1)$ is assumed to be constant. To keep the figure being overly complicated, the depreciation rate is assumed to be common between the cattle and goat production. Carry over asset net of depreciation is given as the linear slope segment next to the origin. Once the production becomes positive, saving out of production is added to the linear carry over asset line, which forms an S-shaped line as depicted with a thick blue line. When cattle production is feasible in the long run, which we assume to be so, or when the thick blue line gets to pass the steady state line $k_{t+1}=k_{t}$, it will have two intersections, $C, D$. As shown in the bottom left figure, when the current asset level is greater than $k_{D}$, the asset level corresponding to the intersection $D$, the production eventually reaches $C$, a steady state where the per capita asset size is constant, or $k_{t+1}=k_{t}$. If the current asset level is smaller than $k_{D}$, the producer will not choose to invest in cattle.
	
	Similarly for the goat production, there is much smaller fixed inputs and production, hence smaller saving $sg(k_{t})$. The shape of next period net per capita asset size is similar with the cattle, only smaller. We note from the previous section that the returns to goats net of mortality and the steady state goat asset size are smaller than the cattle in the region depicted in \textsc{\small Figure \ref{fig poverty trap}}. We also note that a goat investment, when compared to a cattle investment, requires smaller upfront costs but has an infrequent income stream, faces a more limited local demand, shows vulnerability to logging water, all pointing to smaller investments and their returns. We will use these points to assume that the fixed costs and steady state production level are smaller for goats than cattle.  

	For simplicity, we assume that all individual has an asset no smaller than $k_{B}$. Then, when there is only a goat production technology, individuals eventually reaches the point $G$. When the cattle production technology is added to the picture, there is no change in the equlibrium for individuals whose initial assets are in $[k_{B}, k_{D})$. For individuals with initial assets in $[k_{D}, \infty)$, one chooses cattle, because the resulting income level is higher, and eventually arrive at the steady state $C$.\footnote{$k_{D}$ is an unstable equilibrium that all individuals would deviate from, but we include this point to the region of attraction of $C$ for the sake of simplicity. } 

	Over the domain of $k_{t}\in[0, \infty)$, the production possibility frontier, or the contour of the union of two production sets, becomes M-shaped (\textsc{bottom right panel}). Under the configuration depicted in the figure, there will be five equilibria of which three are stable. Ruling out the zero equilbrium as irrelevant, one is left with two stable equilibria, named as goats and cattle in the figure.\footnote{A similar diagram is found in \citet[][Figure 3, with $k-y$ space]{KraayMcKenzie2014}. }

	Formally, one requires the production set $j=\{\mbox{goat, cattle}\}$ to satisfy: there exists $\underline{k}_{j}>0$ that the production is zero for input $k<\underline{k}_{j}$ and is strictly positive for $k\geqslant\underline{k}_{j}$. We assume the production set exhibits decreasing returns to scale for $k\geqslant\underline{k}_{j}$. Let the contour of the production set be $f_{j}(k)$. Assume for expositional simplcity that the saving rate $s$ and depreciation rate $\delta$ are fixed. Further assume that there exists $k_{D}>\underline{k}_{j}$ such that $sf_{j}(k)+(1-\delta)k>k$ for $k\in(k_{D}, k^{*})$, with $k^{*}>k_{D}$ is a fixed point $k^{*}=sf_{j}(k^{*})+(1-\delta)k^{*}$. For $k^{*}$ to exist, under the assumption that cattle rearing is feasible so the intersection $D$ exists, we need $sf'_{j}(k^{*})+(1-\delta)<1$ which holds for any $s>0$ as long as the Inada condition $\lim\limits_{k_{t}\rightarrow \infty} f'(k_{t})=0$ is met. Under these assumptions, for $j$, there exists two intersections between the steady state line, one unstable and the other stable equilibria.\footnote{In \textsc{Figure \ref{fig poverty trap}}, depreciation below $\underline{k}$ is not accounted, because capital cannot be negative. Once the production starts for $k>\underline{k}$, the contour shows net of depreciation so $sf(k)+(1-\delta)k$. } 

	In light of this argument, a loan that is larger than $k_{D}$ allows individuals in the goat equilibrium to transition to cattle production and arrive at the cattle equilibrium. The entire region depicted in the diagram is considered as in the realm of poverty, so it shows a poverty trap within poverty (i.e., goat as ultra poor and cattle as moderately poor). 

	A government or not-for-profit lender can support the productive investments of borrowers without incurring an efficiencly loss through lending. If the lender charges according to marginal costs, the interest rate is the same as the marginal return on capital. The slope of such interest rate is smaller than the 45 degree line, and a line with this slope passing through $D$ will stay below $C$, so the return on investment is strictly greater than the loan interest rate. Therefore, a lending, not a transfer, suffices for the transition, so long as the upperbound of the loan size is no smaller than $k_{D}$ and if there is a way to reduce the costs of information asymmetry and transactions, for example, by group lending and an overhead cost subsidy.

	In the empirical section, we follow \citet{BandieraBRAC2017} in interpreting the lower repayment rates and smaller cattle holding for a smaller upfront loan size as evidence consistent with a poverty trap with a nonconvex production technology. 

\section{Study sample}
\label{SecStudySample}

	Our sample is drawn from the population of river island villages in Northern Bangladesh. 
	\textcolor{red}{[Abu-san will provide the regional characteristics of the area, esp. poverty, using CLP/TUP program data and reports.]}
	
	In the \textit{char} region, the majority of \textit{char}s have only one village. The majority of \textit{char}s have no MFI activity, and we delisted the \textit{char}s if an MFI or an NGO is engaging in microfinance activies, or if \textit{Char Livelihood Program (CLP)} is active.\footnote{The \textit{Char Livelihood Program (CLP)} is run by DFID of the United Kingdom and transfers assets to the poor. } Using Landsat imagery, we identified 128 \textit{char}s within a day's boat ride from the Gaibandha peer and collected information by field visits. From this list of \textit{char}s, we randomly selected 80 \textit{char}s. In each village, we conducted a census of households with their wealth ranking made through a participatory ranking process. Following a process similar to the paired ranking as in \citet[][p.1212]{Alatas2012} and the Peruvian ultra poor case of \citet[][p.66]{KarlanThuysbaert2019}, we asked the least wealthy households in terms of asset ownership. We then asked to form a member committee of 20 households, of which 14 are ultra poor and six are moderately poor. The ultra poor are the poorest in the ranking, while the moderately poor are all other households. As we admitted households on a first come, first served basis, these 20 households are the first to join the membership of microfinance in respective poverty classes. After receiving acceptance for study participation (`pre-acceptance' in \textsc{\normalsize Figure \ref{fig experimantal design}}) from 80 groups comprising 1,600 members, baseline data was collected in 2012 prior to the debt contract type randomization. In each group, 10 out of 20 members were randomly offered the credit and the remaining members were kept as pure control groups who did not receive a loan until 1 or 2 years into the program. Due to a concern for within group spill overs, we do not use the subsample of these control members in this paper. We thus have 800 members for the impact evaluation of this paper whom we surveyed in the baseline and offered one of the four credit products. From these 800 members, we exclude 24 members whose intervention did not strictly follow the experimental design explained below.
	
	After the baseline data was collected in 2012, we offered the debt contracts to each group. There are four debt contract types that are randomised at the group level. After offering the each type of debt contract, three groups opted out as a group, resulting in 77 groups participating the intervention. In addition to the group level rejection, we had 89 individual loan rejectors before loan disbursements. This happened despite we had explained about the debt contract types, random assignment process, various other group based obligations, and had obtained everyone's consent to participate before randomisation. Although both types of rejecters refused to receive a loan, they gave a consent to be surveyed so we tracked them in subsequent survey rounds.



	While loan rejecters remained in our sample, we lost four groups to floods in 2013. As they relocated, we had no choice but to drop them from the study. This resulted in 76 groups including 4 groups who group-rejected the loans remaining in our data. In our study, attrition refers to a drop out from our household survey. Rejection refers to a loan rejection in our intervention, and majority of rejecters (81.25\%) did not attrit from our household survey. Counting all individual attriters, we have a total of 92 subjects (11.9\%) out of 776 subjects who attrited by the final round of the household survey. 

	As a result, among the baseline survey sample, there are flood victims whom we do not track, as well as group rejectors, individual rejectors and borrowers that we track. See \citet{GUK2016} for more details on the randomisation and acceptance process. We track all --- barring the flood victims whose villages were washed away and other attriters --- the potential borrowers including who eventually opted out the borrowing in the data. This enables us to estimate the intention-to-treat effects of offering vrious debt contracts on the population who showed interests in joining microfinance membership. 

\begin{table}
\hfil\begin{minipage}[t]{14cm}
\hfil\textsc{\normalsize Table \refstepcounter{table}\thetable: Descriptive statistics by RCT arm for all households including nonparticipants\label{tab DestatMainByArmNarrow}}\\
% Created in 
\setlength{\tabcolsep}{1pt}
\setlength{\baselineskip}{8pt}
\renewcommand{\arraystretch}{.55}
\hfil\begin{tikzpicture}
\node (tbl) {\input{c:/data/GUK/analysis/save/EstimationMemo/DestatMainByArmNarrow.tex}};
\end{tikzpicture}\\
\renewcommand{\arraystretch}{.8}
\setlength{\tabcolsep}{1pt}
\begin{tabular}{>{\hfill\scriptsize}p{1cm}<{}>{\hfill\scriptsize}p{.25cm}<{}>{\scriptsize}p{12cm}<{\hfill}}
Source:& \multicolumn{2}{l}{\mpage{12.25cm}{\scriptsize Information of 776 households in GUK administrative data and household survey data at the baseline. Survey respondents include nonparticipants to the experiments.}}\\
Notes: & 1. & Mean values at the baseline. Values in brackets are standard deviations. \\
& 2. & \textsf{HeadLiteracy} is an indicator variable of household head literacy. \textsf{HeadAge} is age of household head. \textsf{HHsize} is total number of household members. \textsf{FloodInRd1} is an indicator variable of flood exposure. \textsf{HAssetAmount} and \textsf{PAssetAmount} are amount of household and productive assets, respectively, in BDT, \textsf{NumCows} is cattle holding per household. \textsf{NetValue} is net asset values in BDT per housheold using asset items observed in all 4 rounds. \textsf{BroadNetValue} is net asset values in BDT per housheold for all asset items. \textsf{Attrited} indicates attrition rates in the household survey, and \textsf{GRejected} and \textsf{IRejected} show group rejection rates and individual rejection rates to the lending program. \textsf{Non-attriting borrowers} indicates the ratio of non-attriting borrowers to all borrowers. Because attrition and rejection are separate events, a household can reject and attrit, so non-attrited borrowers $\geqslant$ total - (rejected members + attrited members). USD 1 is about BDT 80. \textsf{RiskPrefVal} is the respondent's choice of the acceptable minimum excess monetary value of the risky option over a certainty option. Lower values indicate a greater risk tolerance. \textsf{TimePref1val} is the respondent's choice of the acceptable minimum excess monetary value in 3 months that is no smaller than present monetary benefit, and \textsf{TimePref2Val} is the the minimum excess value in 1 year and 3 months that is no smaller than monetary benefits of 1 year from now. Lower values indicate a greater patience. If a respondent's \textsf{TimePref1val} is greater than \textsf{TimePref2val}, the respondent is considered to be present-biased. \textsf{PresentBias} is an indicator function that takes the value of 1 if the respondent is considered to be present-biased, 0 otherwise.%The last two columns show $p$ values in percentages ($100*p$) of the null of same location (mean) between all groups, \textsf{ANOVA} using one-way ANOVA under normality and equal variances assumptions and \textsf{K-W} using Kruskal-Wallis under nonnormality and unequal variances assumptions. Columns under \textsf{S-W} gives Shapiro-Wilks tests of normality, and \textsf{F-K} gives Flinger-Kileen tests of equal variance.
\end{tabular}
\end{minipage}
\end{table}

	\textsc{\small Table \ref{tab DestatMainByArmNarrow}} shows descriptive statistics of sample households. As we randomly allocate them into four different arms named as \textsf{traditional}, \textsf{large}, \textsf{large grace}, and \textsf{cattle}, summary is shown by the arms and the overall. As shown in the Appendix \ref{AppSecRandomisation} \textsc{\normalsize Table \ref{tab perm}}, these baseline household characteristics do not differ statistically between the arms. Our sample is characterised by relatively low literacy rate (\textsf{HeadLiteracy}) and relatively young age (\textsf{HeadAge}) of the household heads. Literacy rate is lower than the national average of adult males at 61.54\% in 2012 (UNESCO). Household size (\textsf{HHsize}) is not large, 4.189 members overall, due probably to the constant flood threats, as indicated by above 40\% exposure at the baseline (\textsf{FloodInRd1}), that do not easily allow a large household formation. Cattle holding per household (\textsf{NumCows}) shows cattle rearing is not common and the mean herd size is between .2 to .4.\footnote{ \textsc{Table \ref{table anova CattleHoldingArm}} in Appendix \ref{AppSecRandomisation} shows the test results that \textsf{NumCows} do not differ across arms at the baseline. } Mean net asset values per household (\textsf{NetValue}) and its components, household asset values per household (\textsf{HAssetAmount}), productive asset values per household (\textsf{PAssetAmount}), differ to some extent by arms, but they mostly reflect sampling errors as indicated in the large standard deviations.\footnote{There is an alternative measure for net assets, which we call broad net assets: Broad net assets = Broad assets + net saving - debt to GUK - debts to relatives and money lenders. While regular assets use only items observed for all 4 rounds for household assets, broad assets use all asset items. All estimation results hold with broad net assets with wider confidence intervals due to greater noises across time. See \textsc{\footnotesize Figure \ref{fig NarrowNetAssetsLivestockEffects}} for details. } \textsf{Attrited} indicates attrition rates in the household survey, and \textsf{GRejected} and \textsf{IRejected} show group rejection rates and individual rejection rates to the lending program. We will analyse attrition and rejection later in Section \ref{ResultsSectionParticipation}, \ref{SecAttrition}, but at this point, we just note that the attrition rates are not statistically different between the arms at the group level. \textsf{Non-attriting borrowers} indicates the ratio of non-attriting borrowers to all borrowers. Because there are more rejecters in the \textsf{traditional} arm, this ratio is smaller than in other arms. \textsf{RiskPrefVal, TimePref1Val, TimePref2Val} are the minimum acceptable excess of risky options (vs. certainty), 3 month future options (vs. present), and 15 month future options (vs. 12 months in future), respectively, in monetary values. \label{StudySamplePrefDescription} Smaller values indicate greater risk torelance and patience. \textsf{PresentBias} is the ratio of respondents who indicate present biasedness, \textsf{TimePref1Val} $>$ \textsf{TimePref2Val}. All these measures are statistically similar across arms.



\section{Experimental design}
\label{SecExperimentalDesign}


	To investigate the detailed demand-side constraints and suitable credit scheme for the ultra poor, we implemented the village-level clustered randomization across the four treatment arms as follows (see \textsc{\normalsize Figure \ref{fig DestatMainByArm}}):

\begin{figure}
\hfil\mpage{12cm}{\footnotesize
\hfil\textsc{\normalsize Figure \refstepcounter{figure}\thefigure: Description of experimental arms\label{fig DestatMainByArm}}\\
\hfil\BGbox{
	\begin{description}
	\vspace{1.0ex}\setlength{\itemsep}{1.0ex}\setlength{\baselineskip}{12pt}
	\item[T1]	Traditional microcredit. 
		\begin{description}
		\vspace{1ex}\setlength{\itemsep}{.5ex}\setlength{\baselineskip}{8pt}
		\item[Credit]	5600 BDT (approximately USD 50).
		\item[Repayment start]	Two weeks after the disbursement.
		\item[Installments]	Repay with weekly installments of 125 BDT (approximately USD 1.1) which amounts to a simple interest rate of 11.61\%.
		\item[Maturity]	Total installments of 50 or a loan maturity of one year. Take another two loan contracts of equivalent amounts over the next consecutive years.
		\item[Weekly obligations]	Attend a meeting and deposit an amount decided jointly with group members.
		\end{description}
	\item[T2]	Upfront lumpy credit. Following conditions in black colours differ from \textsf{T1}:
		\begin{description}
		\vspace{1ex}\setlength{\itemsep}{.5ex}\setlength{\baselineskip}{8pt}
		\item[Credit]	16,800 BDT (approximately USD 145).
		\textcolor{gray}{\item[Repayment start]	Two weeks after the disbursement.
		\textcolor{gray}{\item[Installments]	Repay with weekly installments of 125 BDT (approximately USD 1.1)which amounts to a simple interest rate of 11.61\%.}
		\item[Maturity]	Total installments of 150 or a loan maturity of three years.
		\item[Weekly obligations]	Attend a meeting and deposit an amount decided jointly with group members.}
		\end{description}
	\item[T3]	Upfront lumpy credit with a grace period. Following conditions in black colour differ from \textsf{T2}:
		\begin{description}
		\vspace{1ex}\setlength{\itemsep}{.5ex}\setlength{\baselineskip}{8pt}
		\item[\textcolor{gray}{Credit}]	\textcolor{gray}{16,800 BDT (approximately USD 145).}
		\item[Repayment start]	One year after the disbursement.
		\item[Installments]	Repay with weekly installments of 190 BDT (approximately USD 1.7) which amounts to a simple interest rate of 13.1\% when repaying.
		\item[Maturity]	Total installments of 100 or two years. 
		\item[\textcolor{gray}{Weekly obligations}]	\textcolor{gray}{Attend a meeting and deposit an amount decided jointly with group members.}
		\end{description}
	\item[T4]	In-kind credit with a one-year grace period and managerial support programs. Following conditions in black colours differ from \textsf{T3}:
		\begin{description}
		\vspace{1ex}\setlength{\itemsep}{.5ex}\setlength{\baselineskip}{8pt}
		\item[Credit]	Receive a credit in the form of a one-year old heifer with the price of 16,000 BDT (approximately USD 145).
		\item[\textcolor{gray}{Repayment start}]	\textcolor{gray}{One year after the disbursement.}
		\item[\textcolor{gray}{Installments}]	\textcolor{gray}{Repay with weekly installments of 190 BDT (approximately USD 1.7)} which amounts to a simple interest rate of 18.75\% when repaying. After adding the support program costs to the principal, the interest rate will be the same as \textsf{T3}.
		\item[\textcolor{gray}{Maturity}]	\textcolor{gray}{Total installments of 100 or two years. }
		\item[\textcolor{gray}{Weekly obligations}]	\textcolor{gray}{Attend a meeting and deposit an amount decided jointly with group members.}
		\item[Support program] 	Provided input support (fodder, veterinary and vaccination services), marketing consultancy (milk sales), and basic training on cattle rearing with the local NGO, at the total fee of 800 BDT (approximately USD 7.2) charged for the three years. With 800 BDT for the support program, the total cost sums to BDT 16,800 which is the same as in all other arms.
		\end{description}
	\end{description}
	}
}
\end{figure}

	\begin{description}
	\vspace{1.0ex}\setlength{\itemsep}{1.0ex}\setlength{\baselineskip}{12pt}
	\item[T1]	Traditional microcredit. Members of the group receive 5600 BDT (approximately USD 50) credit, and the loan repayment begins two weeks after the disbursement. Members repay with weekly installments and are required to attend weekly meetings as well as to regularly save an amount decided jointly by the group members. The loan maturity is one year, and borrowers are allowed to take another two loan contracts of equivalent amounts over the next consecutive years. The weekly repayment is 125 BDT (approximately USD 1.1) payable in 50 installments.
	\item[T2]	Upfront lumpy credit. Members receive 16,800 BDT credit with a longer loan maturity, and the loan repayments begin two weeks after the disbursement. The weekly repayment and the design of compulsory saving are exactly the same as in \textsf{T1} arm. The loan maturity is three years. The required weekly repayment is 125 BDT payable in 150 weekly instalments (for three years). 
	\item[T3]	Upfront lumpy credit with a grace period. Members receive 16,800 BDT credit with loan repayments begin one year after the disbursement. During the first year grace period, members are required to meet weekly and follow group activities such as compulsory savings just as in other arms. The design of compulsory saving is the same as in the \textsf{T1}, \textsf{T2} arms. The loan maturity is three years. The required weekly repayment is 190 BDT (approximately USD 1.7) payable in 100 weekly installments, starting after one year.
	\item[T4]	In-kind credit with a one-year grace period and managerial support programs.\footnote{Heifer ownership was never explicitly agreed upon, but it was generally understood by the borrowers that they owned the heifer. T4 is thus more similar to a debt contract with the purchased asset as collateral than to a finance lease under which the asset ownership belongs to the lessor. } Members receive in-kind credit in the form of a one-year old heifer with the price of 16,000 BDT (approximately USD 145), and the loan repayment begin one year after the disbursement. The grace period length is equal to the one provided under \textsf{T3} and \textsf{T4} arms. In addition, the members receive input (fodder, veterinary and vaccination services) procurement supports, marketing consultancy (milk sales), and basic training on cattle rearing with the local NGO, at the total fee of 800 BDT (approximately USD 7.2) charged for the period of three years. With 800 BDT for the support program, the total cost sums to BDT 16,800 which is the same as in all other arms.
	\end{description}

	One of the aims of the study is to assess if the entrepreneurship matters in microfinance lending outcomes. Assuming that, below 17000 Taka, the productive asset with the highest return is a heifer, we bundle training and consultation with a heifer lending. At the start of a loan, the NGO's procurement officer buys a heifer from the local market, so the borrower does not have to have the knowledge required for the quality purchase. By providing the knowledge to a group of borrowers through training and disallowing an investment choice with a in-kind, heifer lending, some aspects of entrepreneurship will no longer be a prerequisite. It can be seen that we are offering a capacity to use the best practice or the \textit{cristalised intelligence} related to cattle production \citep{Cattell1963}. This is only a part of entrepreneurial skills. The remainder, a capacity to apply a suitable action to unforeseen events or the \textit{fluid intelligence} related to cattle production, and other inter-personal skills, are left unchanged. If the entrepreneurship raises productivity, borrowers of other arms who are not provided the knowledge are expected to opt out the loan more frequently or perform worse. One can measure effects of the entrepreneurship on participation and outcomes by comparing these two groups, in-kind credit with training vs. cash credit.

	As a natural reference, we want to compare the training cum in-kind loan \textsf{T4} with the traditional regular microcredit \textsf{T1}, a classic Grameen style loan that is about a third in loan size and maturity with no grace period. In order to make comparison feasible, we added two intermediate treatment arms to bridge them: Two arms with upfront lumpy lending that is equivalent of a heifer price, one with a grace period \textsf{T3} and another without a grace period \textsf{T2}. With the loan sizes that are three times the traditional microfinance loans, we extended the maturity to three years to even out the repayment burden. The comparison arm, the traditional regular microcredit, has only one year maturity. We therefore provided the total of three loans in three loan cycles in \textsf{T1} which are unconditionally disbursed annually, so the total loaned amount will be aligned and there is no exit-selection due to delinquency before three cycles are complete. 
	
	Under this setting, frontloading liquidity without changing the total loan size eases a liquidity constraint, attaching a grace period under the same loan size and disbursement timing eases a saving constraint prior to a loan receipt, and offering an in-kind loan with a managerial support without changing other features eases an entrepreneurship constraint. In effect, we constructed a stepped-wedge design over these key features of loans, namely, upfront liquidity (\textsf{Upfront}), a grace period (\textsf{WithGrace}), and in-kind with managerial supports (\textsf{InKind}), to assess the impacts of respective constraints on participation and outcomes as indicated in \textsc{\normalsize Table \ref{tab factorial design}}. 

	An in-kind offer in treatment \textsf{T4} is generally thought to be less efficient than a cash offer as it takes away an investment choice from the borrower. However, the local microfinance practitioners widely agree that other production opportunities are limited, so not much is lost in terms of the choice set, under our setting of island location and occasional floods.\footnote{A closely related project in the neighbouring areas transfers an asset in the form of a cow \citep{BandieraBRAC2017}. }  Given the small set of the productive investment choices, our experiment gives a unique chance to compare cash lending against in-kind lending, even without controlling for a potentially wider choice set of cash lending. 
	%It is generally thought in practice that an in-kind offer, with only a single asset to lease out, is less efficient than a cash offer as it takes away a choice from the borrower. However, the local microfinance practitioners widely agree that little is lost in a production opportunity even when the loan takes an in-kind form in a heifer, because a heifer is almost the only investment choice in our study area.\footnote{I is also notable that a closely related project in the neighbouring areas transfers an asset in the form of a cow\citep{BandieraBRAC2017}. } If this presumption is correct, it gives a unique chance to compare cash lending with in-kind lending, even without controlling for the different choice set of projects. In the later section examining the income generating activities, we show that this is actually the case.
	Indeed, we found in our data that most of \textsf{T2} and \textsf{T3} cash borrowers started to invest in cattle after receiving a loan. Consequently, in our study, the cash-grace-period and in-kind-grace-period lending differ effectively only in the managerial support services bundled in the latter. 

	All loan products are of individual liability and the committee was intended to serve as an activity platform for microfinance operations. Among the \textsf{traditional} members, there were 24 members who received disbursements twice, not three times, due to logistical limitations. We drop them from the analysis and use 776 members in the below. %\textcolor{red}{[Abu-san: Do you know why these 24 households received the loans twice, not three times? The answer is ``unknown.'' So we will just drop them from the analysis.]}

	Lastly, because of the severe flood damages caused on borrowers and the associated administrative delays in 2013, the repayment was halted in 2013 and resumed after one year in 2014. This resulted in an extension of loan maturity from 36 months to 48 months for all arms. This gave substantial leniency to the borrowers in terms of loan repayment burden. 

\begin{table}
\hspace{-1cm}\begin{minipage}[t]{14cm}
\hfil\textsc{\normalsize Table \refstepcounter{table}\thetable: A 4$\times$4 factorial, stepped wedge design\label{tab factorial design}}\\
\setlength{\tabcolsep}{1pt}
\setlength{\baselineskip}{8pt}
\renewcommand{\arraystretch}{.55}
\vspace{2ex}
\hfil\begin{tabular}{>{\footnotesize\hfill}p{2cm}<{}
>{\footnotesize\hfil}p{2.5cm}<{}
>{\footnotesize\hfil}p{2.5cm}<{}
>{\footnotesize\hfil}p{2.5cm}<{}}
					& \cellcolor{paleblue}\textcolor{black}{Large, grace} 			& \cellcolor{paleblue}\textcolor{black}{Large} & \cellcolor{paleblue}\textcolor{black}{Traditional} \\\cellcolor{paleblue}
\textcolor{black}{Cattle} 				& \mpage{2.5cm}{\hfil entrepreneurship\\\hfil constraint\\\hfil (\textsf{InKind})} &\cellcolor{gray80}\mpage{2.5cm}{\textcolor{gray}{\hfil saving\\\hfil constraint\\\hfil (\textsf{WithGrace})}} &\cellcolor{gray80}\mpage{2.5cm}{\textcolor{gray}{\hfil liquidity\\\hfil constraint\\\hfil (\textsf{Upfront})}}\\\cellcolor{paleblue}
\textcolor{black}{Large, grace} &\cellcolor{gray20} 	&  \mpage{2.5cm}{\hfil saving\\\hfil constraint\\\hfil (\textsf{WithGrace})} & \cellcolor{gray80}\mpage{2.5cm}{\textcolor{gray}{\hfil liquidity\\\hfil constraint\\\hfil (\textsf{Upfront})}}\\\cellcolor{paleblue}
\textcolor{black}{Large} 			&\cellcolor{gray20} 	&\cellcolor{gray20}& \mpage{2.5cm}{\hfil liquidity\\\hfil constraint\\\hfil (\textsf{Upfront})}%\\\cellcolor{pink}
%\textcolor{black}{control} & \multicolumn{3}{c}{\cellcolor{green}\textcolor{black}{level \hspace{1em} impacts}}
\end{tabular}
\end{minipage}

\footnotesize Note: \mpage{12cm}{\footnotesize Cell contents are hypothesised constraints on investments that exists in the column arm but are eased in the row arm. Contents in brackets are variable names of respective attributes.}
\end{table}


\begin{figure}
\mpage{\linewidth}{
\vspace{2ex}

\hfil\textsc{\normalsize Figure \refstepcounter{figure}\thefigure: Sampling framework, rejection, and attrition\label{fig experimantal design}}\\[2ex]
\hfil\begin{adjustbox}{max size={.9\textwidth}{.8\textheight}}
\hfil\tikzstyle{attritionbox} = 
[rectangle, rounded corners, text width = 2.5cm, minimum height=1cm, text centered, fill=blue!10,
 drop shadow={opacity=.5, shadow scale=1.05}]
\tikzstyle{dummybox} = 
[draw = none, fill = {yellow!10}, text width = 0cm, drop shadow={opacity=1, fill = {yellow!10}, shadow scale=1.01}]
\begin{tikzpicture}[
    every node/.style={
        font=\sffamily,
        drop shadow,
        fill=red!10,
        text width=5cm,
        align=center},
        >=latex, %Make the arrow tips latex
        myline/.style={ultra thick, black!50},
        shorter/.style={shorten <=1mm, shorten >=0.5mm}]
% start node [for some reasons, I cannot place a node by writing (5, 0)]
%\node[fill = white, text width = 0cm, drop shadow={fill=white, opacity=1, shadow scale=1.01}] (L0) at (0, 0) {};
\node (startnode) {\mpage{5cm}{\hfil Pre-acceptance ($g=80$, $n=800$)\\\\[-1ex]\mpage{5cm}{\footnotesize\hfill moderately poor (240)\\\hfill utra poor (560)\SetLengthSkip{10pt}}\SetLengthSkip{10pt}}};
\node[below = .75cm of startnode] (baseline) {Baseline survey};
\node[below = .75cm of baseline] (random) {Cluster randomisation};

% randomisaion
\node[below = 2.5cm of random, dummybox] (dummy) {};
\node[left = 4.75cm of dummy, text width = 3cm] 
  (trad) {\mpage{3cm}{\hfil T1 Traditional\\\hfil ($n=176$)\\
  \mpage{3cm}{\footnotesize\hfill accepted (105)\\
  \hfill group rejection ($40$)\\
  \hfill individual rejection ($31$)%\\
  %\hfill two disbursements ($24$)
  \SetLengthSkip{10pt}}\SetLengthSkip{10pt}}};
\node[right = 1cm of trad, text width = 3cm] 
  (large) {\mpage{3cm}{\hfil T2 Large\\\hfil ($n=200$)\\
  \mpage{3cm}{\footnotesize\hfill accepted (171)\\
  \hfill group rejection ($20$)\\
  \hfill individual rejection ($9$)%\\
  \SetLengthSkip{10pt}}\SetLengthSkip{10pt}}};
\node[right = 1cm of large, text width = 3cm] 
  (grace) {\mpage{3cm}{\hfil T3 Large grace\\\hfil ($n=200$)\\
  \mpage{3cm}{\footnotesize\hfill accepted (177)\\
  \hfill group rejection ($10$)\\
  \hfill individual rejection ($13$)%\\
  \SetLengthSkip{10pt}}\SetLengthSkip{10pt}}};
\node[right = 1cm of grace, text width = 3cm] 
  (cattle) {\mpage{3cm}{\hfil T4 Cattle\\\hfil ($n=200$)\\
  \mpage{3cm}{\footnotesize\hfill accepted (163)\\
  \hfill group rejection ($0$)\\
  \hfill individual rejection ($37$)%\\
  \SetLengthSkip{10pt}}\SetLengthSkip{10pt}}};
% add arrows from start to each arms
\draw[myline, ->, shorter] (startnode) -- (baseline);
\draw[myline, ->, shorter] (baseline) -- (random);
\draw[myline, ->, shorter, draw = none] 
  (random) -- (trad.north) node[midway, dummybox, yshift = .45cm] (rthalf) {};
\draw[myline, ->, shorter] (random) -- (large.north);
\draw[myline, ->, shorter] (random) -- (grace.north);
\draw[myline, ->, shorter] (random) -- (cattle.north);
% draw an invisible horizontal line at rthalf 
\draw[name path=tradArrow, myline, ->, shorter, draw = none] (random) -- (trad.north);
\draw[name path=invisH, draw = none] (rthalf -| trad.west) -- (rthalf -| cattle.west);
% find an intersection of invisH and tradArrow
\path [name intersections={of=invisH and tradArrow, by={tradMidArrow}}];
\draw[myline, ->, shorter] ([xshift = .1cm] tradMidArrow) -- ($(tradMidArrow) + (-2.75, 0)$) 
  node[attritionbox, anchor = east] (DoubleTrad) 
  {\mpage{2.5cm}{\hfil Not receiving\\\hfil 3 loans\\\hfil ($n=24$)\SetLengthSkip{10pt}}};
\draw[myline, ->, shorter] (random) -- (trad.north);

% trad to attriters
\node[below = 1cm of trad.south west, xshift = 1.0cm, anchor = west, attritionbox]  
  (tradFlood) {\mpage{2.5cm}{\hfil flood victims\\\hfil ($n=20$)\SetLengthSkip{10pt}}};
\node[below = .25cm of tradFlood, attritionbox]  
  (tradAttrit2) {\mpage{2.5cm}{\hfil attrited in rd 2\\\hfil ($n=6$)\SetLengthSkip{10pt}}};
\node[below = .25cm of tradAttrit2, attritionbox]  
  (tradAttrit3) {\mpage{2.5cm}{\hfil attrited in rd 3\\\hfil ($n=4$)\SetLengthSkip{10pt}}};
\node[below = .25cm of tradAttrit3, attritionbox]  
  (tradAttrit4) {\mpage{2.5cm}{\hfil attrited in rd 4\\\hfil ($n=2$)\SetLengthSkip{10pt}}};
\node[below = 6cm of trad, text width = 3cm]
  (tradLast) {\mpage{3cm}{\hfil Traditional\\\hfil ($n=144$)\\
  \mpage{3cm}{\footnotesize\hfill accepted (83)\\
  \hfill group rejection ($36$)\\
  \hfill individual rejection ($25$)
  \SetLengthSkip{10pt}}\SetLengthSkip{10pt}}};
  % draw kinked arrows
	\draw[myline, shorter, ->] ([xshift = .5cm] trad.south west) |- (tradFlood.west);
	\draw[myline, shorter, ->] ([xshift = .5cm] trad.south west) |- (tradAttrit2.west);
	\draw[myline, shorter, ->] ([xshift = .5cm] trad.south west) |- (tradAttrit3.west);
	\draw[myline, shorter, ->] ([xshift = .5cm] trad.south west) |- (tradAttrit4.west);
	%\draw[myline, shorter, ->] ([xshift = 1cm] trad.south west) --++ (0, -5cm);
	\draw[myline, shorter, ->] ([xshift = .5cm] trad.south west) -- ([xshift = .5cm] tradLast.north west);
% large to attriters
\node[right = 1.5cm of tradAttrit2, attritionbox]  
  (largeAttrit2) {\mpage{2.5cm}{\hfil attrited in rd 2\\\hfil ($n=5$)\SetLengthSkip{10pt}}};
\node[below = .25cm of largeAttrit2, attritionbox]  
  (largeAttrit3) {\mpage{2.5cm}{\hfil attrited in rd 3\\\hfil ($n=2$)\SetLengthSkip{10pt}}};
\node[below = .25cm of largeAttrit3, attritionbox]  
  (largeAttrit4) {\mpage{2.5cm}{\hfil attrited in rd 4\\\hfil ($n=1$)\SetLengthSkip{10pt}}};
\node[below = 6cm of large, text width = 3cm]  
  (largeLast) {\mpage{3cm}{\hfil Large\\\hfil ($n=192$)\\
  \mpage{3cm}{\footnotesize\hfill accepted (164)\\
  \hfill group rejection ($19$)\\
  \hfill individual rejection ($9$)
  \SetLengthSkip{10pt}}\SetLengthSkip{10pt}}};
  % draw kinked arrows
%	\draw[myline, shorter, ->] ([xshift = 1cm] large.south west) |- (largeFlood.west);
	\draw[myline, shorter, ->] ([xshift = .5cm] large.south west) |- (largeAttrit2.west);
	\draw[myline, shorter, ->] ([xshift = .5cm] large.south west) |- (largeAttrit3.west);
	\draw[myline, shorter, ->] ([xshift = .5cm] large.south west) |- (largeAttrit4.west);
	\draw[myline, shorter, ->] ([xshift = .5cm] large.south west) -- ([xshift = .5cm] largeLast.north west);
% grace to attriters
\node[below = 1cm of grace.south west, xshift = 1.0cm, anchor = west, attritionbox]  
  (graceFlood) {\mpage{2.5cm}{\hfil flood victims\\\hfil ($n=10$)\SetLengthSkip{10pt}}};
\node[below = .25cm of graceFlood, attritionbox]  
  (graceAttrit2) {\mpage{2.5cm}{\hfil attrited in rd 2\\\hfil ($n=13$)\SetLengthSkip{10pt}}};
\node[below = .25cm of graceAttrit2, attritionbox]  
  (graceAttrit3) {\mpage{2.5cm}{\hfil attrited in rd 3\\\hfil ($n=3$)\SetLengthSkip{10pt}}};
\node[below = .25cm of graceAttrit3, attritionbox]  
  (graceAttrit4) {\mpage{2.5cm}{\hfil attrited in rd 4\\\hfil ($n=3$)\SetLengthSkip{10pt}}};
\node[below = 6cm of grace, text width = 3cm]  
  (graceLast) {\mpage{3cm}{\hfil Large grace\\\hfil ($n=171$)\\
  \mpage{3cm}{\footnotesize\hfill accepted (160)\\
  \hfill group rejection ($0$)\\
  \hfill individual rejection ($11$)
  \SetLengthSkip{10pt}}\SetLengthSkip{10pt}}};
  % draw kinked arrows
	\draw[myline, shorter, ->] ([xshift = .5cm] grace.south west) |- (graceFlood.west);
	\draw[myline, shorter, ->] ([xshift = .5cm] grace.south west) |- (graceAttrit2.west);
	\draw[myline, shorter, ->] ([xshift = .5cm] grace.south west) |- (graceAttrit3.west);
	\draw[myline, shorter, ->] ([xshift = .5cm] grace.south west) |- (graceAttrit4.west);
	\draw[myline, shorter, ->] ([xshift = .5cm] grace.south west) -- ([xshift = .5cm] graceLast.north west);
% cattle to attriters
\node[below = 1cm of cattle.south west, xshift = 1.0cm, anchor = west, attritionbox]  
  (cattleFlood) {\mpage{2.5cm}{\hfil flood victims\\\hfil ($n=10$)\SetLengthSkip{10pt}}};
\node[below = .25cm of cattleFlood, attritionbox]  
  (cattleAttrit2) {\mpage{2.5cm}{\hfil attrited in rd 2\\\hfil ($n=5$)\SetLengthSkip{10pt}}};
\node[below = .25cm of cattleAttrit2, attritionbox]  
  (cattleAttrit3) {\mpage{2.5cm}{\hfil attrited in rd 3\\\hfil ($n=5$)\SetLengthSkip{10pt}}};
\node[below = .25cm of cattleAttrit3, attritionbox]  
  (cattleAttrit4) {\mpage{2.5cm}{\hfil attrited in rd 4\\\hfil ($n=3$)\SetLengthSkip{10pt}}};
\node[below = 6cm of cattle, text width = 3cm]  
  (cattleLast) {\mpage{3cm}{\hfil Cattle\\\hfil ($n=177$)\\
  \mpage{3cm}{\footnotesize\hfill accepted (147)\\
  \hfill group rejection ($0$)\\
  \hfill individual rejection ($30$)
  \SetLengthSkip{10pt}}\SetLengthSkip{10pt}}};
  % draw kinked arrows
	\draw[myline, shorter, ->] ([xshift = .5cm] cattle.south west) |- (cattleFlood.west);
	\draw[myline, shorter, ->] ([xshift = .5cm] cattle.south west) |- (cattleAttrit2.west);
	\draw[myline, shorter, ->] ([xshift = .5cm] cattle.south west) |- (cattleAttrit3.west);
	\draw[myline, shorter, ->] ([xshift = .5cm] cattle.south west) |- (cattleAttrit4.west);
	\draw[myline, shorter, ->] ([xshift = .5cm] cattle.south west) -- ([xshift = .5cm] cattleLast.north west);

\begin{pgfonlayer}{background}
\node [fill={yellow!10}, fit=(trad) (startnode) (cattleAttrit4) (cattleLast)] {};
\end{pgfonlayer}

\end{tikzpicture}

\end{adjustbox}

\vspace{2ex}
\hfil{\footnotesize Note:} \mpage{12cm}{\footnotesize Each 20 subjects (14 ultra poor, 6 moderately poor) in 80 groups agreed to participate in the lending program. Each 10 subjects (7 ultra poor, 3 moderately poor) in 80 groups were randomly assigned to the experiment. 80 groups were randomly assigned to 4 arms after the baseline household survey. After the arm assignment is revealed, 7 groups (70 subjects) group-rejected and 90 subjects individually-rejected to participate in the lending program. 24 subjects in the \textsf{traditional} arm were given the same loan amount but in 2 disbursements for logistical errors, and they were dropped from the analysis sample. Total of 706 subjects participated in the lending program while all 776 subjects were tracked in the subsequent household surveys. The household survey sample size was reduced to 684 by attrition at the round 4 survey (attrition rate 0.119). See \textsc{Figure \ref{fig DestatMainByArm}} for description of each arms. 
\setlength{\baselineskip}{7pt}}
}
\end{figure}

\section{Empirical strategy}
\label{SecEmpiricalStrategy}


	We collected data at one baseline survey and three annual follow up surveys. With successful randomisation (see Section \ref{ResultsSectionParticipation} and Appendix \ref{AppSecRandomisation}), we use ANCOVA estimators to measure impacts of each experimental arms and loan attributes. ANCOVA estimators are more efficient than DID estimators \citep{FrisonPocock1992, McKenzie2012}. As we include loan rejecters, what we are estimating is intention-to-treat effects. For an ease of interpretation, we sometimes use indicator variables of each attributes, \textsf{Upfront, WithGrace, InKind} in place of arms in several estimating equations. Numerically, both are equivalent. In what follows, we will refer to these attributes as \textit{functional attributes}.
	
	The estimating equation for our intention-to-treat effects is:
	\begin{equation}
	y_{it}=b_{a_{0}}+b_{1}y_{i1}+\bfb'\bfdee_{i}+e_{it}, \quad t=2, 3, 4,
	\label{1steq}
	\end{equation}
	where, for member $i$ in survey round $t$ ($t=1$ is the baseline), $y_{it}$ is an outcome measure, $\bfdee_{i}$ is a vector of three indicator variables in nontraditional arms or functional attributes that $i$ receives, $\bfb'=\left(
	\begin{array}{ccc}
	b_{a_{1}} & b_{a_{2}} & b_{a_{3}}
	\end{array}
	\right)$ is associated impacts relative to \textsf{traditional} arm, $e_{it}$ is an error term.  For the \textsf{traditional} arm, the conditional mean of outcome given covariates and baseline outcome variable is given by $b_{a_{0}}$. For an arm or a functional attribute $a_{k}$, the impact relative to the traditional arm is measured with $b_{a_{k}}$. As we are interested in the time course of relative impacts, we extend equation \eqref{1steq} as:
	\begin{equation}
	y_{it}=b_{1}y_{i1}+b_{2a_{0}}+\bfb'_{2}\bfdee_{i}+b_{3a_{0}}c_{3}+\bfb'_{3}\bfdee_{i}c_{3}+b_{4a_{0}}c_{4}+\bfb'_{4}\bfdee_{i}c_{4}+e_{it}, \quad t=2, 3, 4,
	\label{EstimatingEqTimeVarying}
	\end{equation}
	where $\bfb'_{t}=\left(
	\begin{array}{ccc}
	b_{ta_{1}} & b_{ta_{2}} & b_{ta_{3}}
	\end{array}
	\right)$ is a vector of time-varying impacts relative to concurrent \textsf{traditional} arm, $c_{3}$ is a dummy variable for $t=3$ and $c_{4}$ is a dummy variable for $t=4$. Our main interest is on the cumulative deviation of impacts of a non-traditional arm from impacts of the traditional arm. In equation (2), this is captured by $b_{2a_{k}}$ for period 2, $b_{2a_{k}} + b_{3a_{k}}$ for period 3, and $b_{2a_{k}} + b_{4a_{k}}$ for period 4. We thus plot these estimates for cumulative impacts in main figures in the next section. In some specifications, equation (2) is further extended to include controls of other baseline characteristics and their interactions with treatment dummies to allow heterogeneous impacts. All the standard errors are clustered at the group (char) level as suggested by \citet{AbadieAtheyImbensWooldridge2017}.%\footnote{To aid the understanding if the data is more suited to the assumption of first-difference (FD) rather than fixed-effects (FE), we use a check suggested by \citet[][10.71]{Wooldridge2010}. It is an AR(1) regression using FD residuals. Most of results show low autocorrelations in FD residuals which is consistent with the assumption of FD estimator. The issue of choice between FD or FE is not of primary importance, as the use of cluster-robust standard errors gives consistent estimates of SEs in both estimators, and it boils down to efficiency.  }


\section{Results}
\label{SecResults}








	The reasons behind nonparticipation are fundamental in understanding the outreach. We analyse nonparticipation in relation to the debt contract design that they were randomly allocated to. Given our interests on the \textsf{cattle} arm, we further compare the participant characteristics of \textsf{cattle} arm members and of all other arms. In addition, selective attrition from the household surveys, if any, biases the estimates so we need to compare the attriter's characteristics with the nonattriters. In this section, we examine the difference in characteristics of participants and rejecters, attriters and non-attriters. After these exercises, we assess the impacts of debt contract design on assets, labour incomes, consumption, and schooling. 


\subsection{Participation}
\label{ResultsSectionParticipation}


	As noted in Section \ref{SecExperimentalDesign}, there are two kinds of rejecters in participation. One is group rejecters who turned down the offer jointly as a group, and another is individual rejecters who decided not to participate while fellow members of the group participated. To see if the differences are statistically meaningful, we use permutation tests of \textsf{R}'s \textsf{coin} package with 100000 random draws from all admissible permutations. 

\begin{table}
\hfil\begin{minipage}[t]{14cm}
\hfil\textsc{\normalsize Table \refstepcounter{table}\thetable: Individual rejecters vs. non-rejecters\label{tab MainTextIRjecters}}\\
\setlength{\tabcolsep}{1pt}
\setlength{\baselineskip}{8pt}
\renewcommand{\arraystretch}{.55}
\hfil\input{c:/data/GUK/analysis/save/PermutationTests/IndividualRejectionTestResults.tex}
\vspace{2ex}
\end{minipage}

\hfil{\footnotesize Note:} \mpage{12cm}{\footnotesize Individual rejecters are the members who did not accept a loan based on an individual decision after the period when group rejection was decided. After 70 people group-rejected, the total number of individuals who was in a position to individually reject the loan was 706 people, of which 90 individually rejected. \textsf{Traditional arm} panel compares individual rejecters against non rejecters in the \textsf{traditional} arm, \textsf{non-Traditional arm} panel shows the comparison in the non-\textsf{traditional} arms, \textsf{All arms} panel shows the comparison in the all arms. Non-\textsf{traditional} arms are \textsf{large}, \textsf{large grace} and \textsf{cattle} arms. The variable \textsf{Arm} is the ratio of \textsf{traditional} arm members in individual nonrejecters and individual rejecters. Respective rejection rates are given in the brackets in the row \textsf{n}. \textsf{HeadLiteracy} is an indicator variable of household head literacy. \textsf{HeadAge} is age of household head. \textsf{HHsize} is total number of household members. \textsf{FloodInRd1} is an indicator variable of flood exposure. \textsf{HAssetAmount} and \textsf{PAssetAmount} are amount of household and productive assets, respectively, in BDT, \textsf{NumCows} is cattle holding per household. \textsf{NetValue} is net asset values in BDT per housheold. \textsf{Attrited} indicates attrition rates in the household survey, and \textsf{GRejected} and \textsf{IRejected} show group rejection rates and individual rejection rates to the lending program. USD 1 is about BDT 80.\setlength{\baselineskip}{7pt}}

\vspace{3ex}
\hfil\begin{minipage}[t]{14cm}
\hfil\textsc{\normalsize Table \refstepcounter{table}\thetable: Contrasting \textsf{cattle} arm and other arms, borrowers and non-attriting borrowers\label{tab main cownoncow}}\\
\setlength{\tabcolsep}{1pt}
\setlength{\baselineskip}{8pt}
\renewcommand{\arraystretch}{.55}
\hfil\input{c:/data/GUK/analysis/save/PermutationTests/CowVsNonCowTestResults.tex}
\vspace{2ex}
\end{minipage}

\hfil{\footnotesize Note:} \mpage{12cm}{\footnotesize Borrowers are members who accepted a loan, non-attriting borrowers are borrowers who stayed in the household survey until the final round. Both \textsf{Borrowers} panel compares the difference in participant characteristics between \textsf{cattle} and other arms.
\textsf{Non-attriting borrowers} panel compares the difference in non-attriting participant characteristics between \textsf{cattle} and other arms. Both show \textsf{cattle} arm induced participation of asset-poor households at the beginning and until the end of the project. Ratios of \textsf{cattle} arm members in respective groups are given in the brackets in the row \textsf{n}. See \textsc{Table \ref{tab MainTextIRjecters}} for variable descriptions. \setlength{\baselineskip}{7pt}}
\end{table}


	Group rejecters of \textsf{traditional} and non-\textsf{traditional} arms differ in household characteristics. In the Appendix \ref{AppSecRejection}, it is shown that the asset-poor households group-rejected in the \textsf{traditional} arm (\textsc{\normalsize Table \ref{tab1 Permutation test results of group rejection among traditional arm}}), while it is younger, recent flood victims who group-rejected in the non-\textsf{traditional} arms (\textsc{\normalsize Table \ref{tab1 Permutation test results of group rejection among non-traditional arm}}). We conjecture that it is lack of \textsf{Upfront} liquidity that prevented asset-poor households of \textsf{traditional} arm from participating because they cannot purchase cattle due to insufficient net asset values or an insufficient resale value of owned livestock, when members of similar characteristics partcipated in non-\textsf{traditional} arms. Group rejecters of non-\textsf{traditional} arms did not participate mostly because of negative asset shocks. %This is a real resource constraint that binds the households. This is different from a psychological constraint that, so long as there is a cost or a payment involved, albeit at a minimal level, there remains a group of households who would not take up the investment \citep{Ashraf2010, CohenDupas2010}. 

	\textsc{\normalsize Table \ref{tab MainTextIRjecters}} contrasts individual rejecters and individual non-rejecters for \textsf{traditional} arm, non-\textsf{traditional} arms, and all arms combined.\footnote{As shown in \textsc{Table \ref{tab1 Permutation test results of individual rejecters, traditional vs. non-traditional arm}}, characteristics of individual rejecters are similar between \textsf{traditional} and non-\textsf{traditional} arms. } As seen in \textsf{livestock values, cattle holding, net assets}, individual rejecters in both subsamples tend to have less assets. The differences between individual rejecters and non individual rejecters are not statistically meaningful due to small sample sizes in both subsamples, but they become statistically unignorable when both subsamples are combined\footnote{For example, net asset values have $p$ values of 29.9\% and 13.1\% for both subsamples, which is reduced to 6.8\% in the all arms sample. }: In the \textsf{All arms} panel comparing individual rejecters and non individual rejecters in all arms, the common factors associated with nonparticipation are a smaller household size ($p=1.0\%$), smaller livestock holding ($p=8.3\%$), and smaller net asset values ($p=6.9\%$).%(\textsc{\normalsize Table \ref{tab1 Permutation test results of individual rejection among traditional arm}} and \textsc{\normalsize Table \ref{tab1 Permutation test results of individual rejection among non-traditional arm}}) 
	\footnote{\textsf{NetValue} also shows a difference but this is due mostly to a difference in livestock holding. } 
		
	% (\textsc{\normalsize Table \ref{TabLabel1[grep("l rejection$", TabLabel1)]}})
	%In non-\textsf{traditional} arms, the individual rejecters have only marginally different mean values relative to individual nonrejecters (\textsc{\normalsize Table \ref{tab Ireject nontrad perm}}). 

	Smaller household size of rejecters hints that cattle rearing may require a certain household size. %, and the households who have more livestock may have the capacity to raise more. To interpret this, 
	It is possible that smaller households may be facing a domestic labour constraint or a space limitation to accommodate cattle under the roof. These constraints are expected to be absent in asset transfer programs where targeted residents can sell the asset if either of constraints binds. We conjecture that the households under a binding domestic capacity constraint did not meet the conditions to raise cattle, and have withheld themselves from the program with an individual rejection. This self-selection may have caused the repayment rates to be higher than when everyone participated. 

	One of the few differences between subsamples is flood exposure: It is related to individual rejection only among the non-\textsf{traditional} arm members. %This is consistent with the conjecture that, had the \textsf{traditional} arm group rejecters been offered any of the non-\textsf{traditional} arms, they, as a group, may have accepted it.  %It shows the latter is more exposed to flood in baseline and has larger livestock values. This implies that, once large enough sum of loan is disbursed, %there is no minimum livestock and asset holding level to partake in the larger loans, and despite a negative asset shock in flood and a poverty trap at this level may be overcome once household size and negative asset shocks are accounted for.
	A strong correlation between baseline flood exposure and individual rejection among the non-\textsf{traditional} arm members suggests that a population prone to natural calamity and associated asset shocks have voluntarily opted out the borrowing. This partly explains the lack of commercial and even noncommercial/NGO lenders in the flood prone areas. 
	
	\textsf{RiskPrefVal} indicates that individual rejecters tend to demand higher compensation for risks, and the $p$ value becomes small enough only with entire sample of \textsf{All arms}. This suggests some individual rejecters are more risk averse than non rejecters. \textsf{TimePrefVal1, TimePrefVal2, PresentBias} all do not show statistically recognisable differences. 

	In \textsc{\small Table \ref{tab main cownoncow}}, we compare if the \textsf{cattle} arm participants (borrowers) differ from participants in other arms at the baseline. In the left panel, we compare all participants including atteriters. Interestingly, partcipants of \textsf{cattle} arm differ from other arms in having less cattle rearing experience as observed in smaller initial cattle holding ($p$ value = 15.6\%) and in having lower net asset values ($p$ value = 6.2\%). 
	%\textsc{\normalsize Table \ref{TabLabel1[grep("bo.*non-ca", TabLabel1)]}}
	Despite these disadvantageous features in rearing a heifer, the \textsf{cattle} arm, which provides managerial supports and in-kind lending, induced partcipation.\footnote{In addition, \textsf{cattle} arm borrowers show relative impatience as indicated in \textsf{TimePrefVal1} ($p$ value = 0.6\%), \textsf{TimePrefVal2} ($p$ value = 4.4\%). } As we will see in Section \ref{Sec Impacts}, the choice of lending instrument (cash or in-kind) does not matter in the investment choice. So it is natural to infer that the managerial support component has induced the members with less experiences and fewer assets to take up loans. 
	
	In the right panel, we compare the borrowers who did not attrit by the end of final survey round between \textsf{cattle} arm with other arms. These borrowers can be seen as successful borrowers. At the baseline, these successful borrowers of \textsf{cattle} arm have smaller baseline livestock holding ($p$ value = 1.6\%) and smaller baseline net asset holding ($p$ value = 0.7\%) than other arms' successful borrowers. %(\textsc{\normalsize Table \ref{}})
	This hints that asset poor borrowers participated and managed to stay on the survey until the end of the study in the \textsf{cattle} arm with a help of managerial supports.
	
	%	Group level rejection to participate is negatively correlated with literacy of household head (\textsc{\normalsize Table \ref{tab Greject perm MainText}}). Acknowledging the reasons for rejection can be different for individuals, we also tested the independence of each characteristics for individual rejecters (vs. non-individual rejecters) in \textsc{\normalsize Table \ref{tab Ireject perm MainText}}. One sees that smaller \textsf{HHsize}, being affected with \textsf{FloodInRd1}, and smaller \textsf{LivestockValues} and \textsf{NumCows} are associated with individual rejecters. We conjecture that individual decisions not to participate may be understood as: Smaller household size leaves a smaller capacity for cattle production labour in a household, and being hit with a flood may have resulted in lower livestock levels that would prompt them to reconsider partaking in another livestock project. 
%
%	A closer look at the nonparticipation correlates among \textsf{traditional} arm members in \textsc{\normalsize Table \ref{tab reject trad perm MainText}} and non-\textsf{traditional} arm members in \textsc{\normalsize Table \ref{tab reject nontrad perm MainText}} reveals possible differences in the causes. Rejection among \textsf{traditional} members tend to be associated with lower livestock holding but not with higher flood exposure nor smaller household size, while rejecters among non-\textsf{traditional} members are more likely to have suffered from flood at the baseline and have smaller household size. \textsc{\normalsize Table \ref{tab reject trad nontrad perm MainText}} shows rejecters of \textsf{traditional} have less flood exposure, smaller livestock and cattle holding, but not necessarily poorer as indicated by head literacy and asset holding than non-\textsf{traditional} counterpart. Given \textsf{traditional} rejecters at the mean have smaller livestock while household size is similar, it hints some capacity to supply labour for cattle production if an opportunity arises.

%	Since the offered arms were randomised, individual rejecters of \textsf{traditional} arm, who are similar in characteristcs to individual rejecters of non-\textsf{traditional} arm whose impediments are baseline flood and small household size, may have accepted the offer had their household size is larger and had they been offered non-\textsf{traditional} lending. Henceforth, we conjecture that flood exposure and household size are the potential impediments for uptake in larger size loans. 


	%Even we are targeting the ultra poor and designed the loan products to help them rise above the poverty trap, we still find lacking minimum level of assets, despite at a very low level, had kept the ultra poor from participating in microfinance. In the results of lending we consider in the below, the bottom class of the ultra poor had not lept its benefits.

\subsection{Attrition}
\label{SecAttrition}

	The survey resulted in the attrition (including the flood victims) of a moderate rate, 11.9\%. We checked for systematic differences between attriters and nonattriters 
	%\textsc{\normalsize Table \ref{TabLabel1[1]}}.
	and found the attrition is not correlated with any household level characteristics (see more detailed attrition examination in Appendix \ref{AppSecAttrition}). %As attrition 	rates differ between \textsf{traditional} and non-\textsf{traditional} arms, we %compare them %in \textsc{\normalsize Table \ref{tab1 Permutation test results of attriters of traditional and non-traditional arms}}. It shows 
	We also found that \textsf{traditional} arm attriters have a lower rate of head literacy while non-\textsf{traditional} arm attriters are more exposed to the flood and have a larger household size. One can argue that, with attrition, the estimated impacts of borrowing could have increased for the \textsf{traditional} arm while not for the non-\textsf{traditional} arms. Such a conjecture hints there may be underestimation, if any, but it is unlikely to inflate the impact estimates.

\subsection{Impacts}
\label{Sec Impacts}




\subsubsection{Assets}

\begin{figure}
\renewcommand{\arraystretch}{.6}
\mpage{\linewidth}{
\hfil\textsc{\footnotesize Figure \refstepcounter{figure}\thefigure: Net assets by period\label{fig NetAssets}}\\
\hfil\includegraphics[width = 12cm]{c:/data/GUK/analysis/program/figure/EstimationMemo/NetAssetsByPeriod.png}\\
\renewcommand{\arraystretch}{1}
\hfil\begin{tabular}{>{\hfill\scriptsize}p{1cm}<{}>{\scriptsize}p{12cm}<{\hfill}}
Source: & Tabulated with survey data.\\
Note:&  Red squares are means of respective data. Vertical axis is logarithmic scaled. Horizontal green lines indicate zero. \\[1ex]
\end{tabular}
}
\end{figure}

\begin{figure}
\mpage{\linewidth}{
\hfil\textsc{\footnotesize Figure \refstepcounter{figure}\thefigure: Cumulative effects on net assets\label{fig NetAssetEffects}}\\

\hfil\includegraphics[width = 12cm]{c:/data/GUK/analysis/program/figure/EstimationMemo/NetAssetsCumRelativeToConcurrentTradEffects.pdf}\\
\renewcommand{\arraystretch}{1}
\hfil\begin{tabular}{>{\hfill\scriptsize}p{1cm}<{}>{\scriptsize}p{12cm}<{\hfill}}
Source: & Estimated with survey data. Constructed from ANCOVA estimation results \textsc{Table \ref{tab ANCOVA narrow net assets timevarying}}, \textsc{Table \ref{tab ANCOVA narrow net assets timevarying attributes}}. 
\\
Note:&  Cumulative impacts on net assets. \textsf{Large/Upfront}, \textsf{Large grace}, \textsf{Cattle} are impacts relative to \textsf{Traditional} arm. \textsf{WithGrace} and \textsf{InKind} are the impacts of respective marginal functional attributes. Panels show cumulative impacts of respective arm or attributes \textsf{k} relative to \textsf{tradiotional} arm which are obtained by $\mbox{2nd period}=b_{2k}, \mbox{3rd period}=b_{2k}+b_{3k}$, $\mbox{4th period}=b_{2k}+b_{4k}$ in the estimating equation $y_{it}=b_{1}y_{i1}+b_{2}+\bfb'_{2}\bfdee_{i}+b_{3}c_{3t}+\bfb'_{3}\bfdee_{i}c_{3t}+b_{4}c_{4t}+\bfb'_{4}\bfdee_{i}c_{4t}+e_{it}, \ t=2, 3, 4$, where $y_{it}$ is the outcome measure of member $i$ in period $t$, $\bfdee_{i}$ is a vector of arms or functional attributes, $c_{3t}, c_{4t}$ are indicator variables of period 3 and 4. Bars show 95\% confidence intervals using cluster robust standard errors.\\[1ex]
\end{tabular}
}
\end{figure}

	In \textsc{\footnotesize Figure \ref{fig NetAssets}}, we show the time paths of net assets by arm in logarithms. Overall, we see increased levels of asset holding in all arms. The difference between \textsf{Traditional} arm and non-\textsf{Traditional} arms are subtle that one must use statistical tests. \footnote{In all regressions, specification (1) is an OLS on arms/functional attributes, (2) is ANCOVA only with arms/functional attributes, (3) adds flood exposure, household head characteristics, household size to (2), (4) adds baseline cattle ownership and interactions to (3), (5) addes baseline cattle holding to (3), (6) adds adds baseline cattle ownership and interactions and baseline cattle holding to (3). } 

	\textsc{\footnotesize Figure \ref{fig NetAssetEffects}} summarises the cumulative impacts on net assets in time-varying specification of \eqref{EstimatingEqTimeVarying}.  See Appendix \ref{AppendixEstimationTables} for full estimation results. There are five panels of arms and functional attributes. Since \textsf{large} arm and \textsf{Upfront} functional attribute are numerically same in \eqref{EstimatingEqTimeVarying}, they are put in one panel column. In all panels, points show the estimates of cumulative deviation from concurrent \textsf{traditional} arm values, or how much the impacts relative to \textsf{traditional} arm have evolved. Vertical bars indicate 95\% confidence intervals. 
	For all panels, in each period, there are several estimation specifications bunched side-by-side. \footnote{Specification 1 is omitted from the plot, because it is an OLS regression without the baseline outcome that is intended to provide a reference for ANCOVA estimates. } This is intended to show robustness to specification changes at a glance.\footnote{As multiple specifications are estimated to show uniformity of results, not to pick one specific estimate, inference corrections for multiple testing are unncessary. } One sees that there is little variation across specifications. %Cattle and net assets have more regression specifications due to their possible dependence on previous cattle ownership and its inclusion as a covariate.
	
	There are notable tendencies in the figure. First, point estimates show there is a one time increase at period 2 in the \textsf{Large/Upfront} arm/attribute. %while the conditional mean values are relatively unchanged for the \textsf{traditional} arm. 
	All the non-\textsf{traditional} arms, which have the \textsf{Upfront} functional attribute in common, have increased net assets once and stayed increased relative to the \textsf{traditional} arm. We interpret the impacts as once-off, given that the point estimates change only marginally and standard error bars grow longer in time. As time passes, standard errors get necessarily magnified because borrowers get exposed to more random variations, and in the absence of further impacts that lift the point estimates away from zero, the error bars eventually cross the zero line at round 4 in two regresion specifications. %Estimates for cattle holding of \textsf{traditional} arm remain relatively unchanged in all periods, so a one time increase implies a gap in cattle holding was created in period 2 and the gap stayed unchanged. 

	Secondly, it is the \textsf{Upfront} functinal attribute that shows positive impacts on net assets. %Estimates for net assets of \textsf{traditional} arm show an upward trend. On top of this underlying trend, all non-\textsf{traditional} arms show a one-time increase, or a gap relative to \textsf{traditional} due to the \textsf{Upfront} aspect of lending. 
	This is consistent with the nonconvex production technology of a larger investment under a liquidity constraint, coupled with an inferior, smaller investment technology. \textsc{\normalsize Table \ref{tab ANCOVA narrow net assets timevarying}} [specification (2)] in the Appendix shows that, relative to the traditional microfinance lending, the upfront liquidity provision increases the net assets by BDT 
	14478 
	(CI 6868, 22088) or 1.06$\sigma$ (of the baseline standard deviation) in the second year, 
	BDT 16417 
	(CI 1700, 31135) or 1.2$\sigma$ by the end of fourth year. 
	%and the number of cattle holding by round(confi[grepl("NumCows$", FileName) & grepl("cum", ImpactType) & grepl("ge$", attributes) & grepl("T$", regtype) & num == 2 & period == 2,  estimate], 2) (CI round(confi[grepl("NumCows$", FileName) & grepl("cum", ImpactType) & grepl("ge$", attributes) & grepl("T$", regtype) & num == 2 & period == 2,  .(lb, ub)], 2)) or round((confi[grepl("NumCows$", FileName) & grepl("cum", ImpactType) & grepl("ge$", attributes) & grepl("T$", regtype) & num == 2 & period == 2,  estimate]/NumCowsSummary["stdNumCow"]), 2)$\sigma$ in the second year, and round(confi[grepl("NumCows$", FileName) & grepl("cum", ImpactType) & grepl("ge$", attributes) & grepl("T$", regtype) & num == 2 & period == 4,  estimate], 2) 	(CI round(confi[grepl("NumCows$", FileName) & grepl("cum", ImpactType) & grepl("ge$", attributes) & grepl("T$", regtype) & num == 2 & period == 4,  .(lb, ub)], 2)) or round((confi[grepl("NumCows$", FileName) & grepl("cum", ImpactType) & grepl("ge$", attributes) & grepl("T$", regtype) & num == 2 & period == 4,  estimate]/NumCowsSummary["stdNumCow"]), 2)$\sigma$ by the end of fourth year. 
	As we discuss in the robustness checks, these results hold when other various definitions of assets are adopted or other covariates, including cattle rearing experiences, are controlled.

	Thirdly, comparing the impacts of the \textsf{InKind} and \textsf{WithGrace} functional attributes, we see statistically zero differences while \textsf{Large/Upfront} shows statistically positive impacts. The marginal contributions of the former functional attributes are zero. Accordingly, the \textsf{Cattle} arm, a combination of \textsf{InKind} and \textsf{WithGrace} functional attributes in addition to \textsf{Upfront}, has the impacts that are statistically zero beyond the \textsf{Large} arm. The finding that the \textsf{Cattle} arm outcomes are statistically indistinguishable from other non-\textsf{traditional} arms implies that it facilitated the returns to cattle rearing at a no lower level. In light of the fact that individuals with less cattle rearing experiences and lower asset values participated and continued in the \textsf{Cattle} arm, the returns at a no lower level by themselves are an achievement. %The \textsf{Cattle} arm have induced participation to lending and have achieved the same level of impacts as in other arms, on the members who would otherwise not take a loan because of their relatively disadvantaged background characteristics. 
	The reason can either be the managerial support program complemented the necessary codifiable knowledge, or these participants had the same level of knowledge as other participants but noticed the managerial support program as useful. Either possibility is consisitent with the finding by previous studies that only the experienced or skilled members could reap the benefits of traditional microfinance. Previous studies cited in the Section \ref{SecIntroduction} have targeted the population with a richer set of investment possibilities in a more urbanised setting under which the experience may have a positive return. In the current study, the population resides in a remote, rural area. Even the simpler production process of cattle rearing that consists of procuring feeding, grazing, insemination and calving turns out to demand unignorable codifiable skills, or the crystalised intelligence, to participate and sustain in traditional microfinance. We examine how the returns to experience may have affected the outcomes in more details in \ref{subsec RobustnessCheck}.
	
	One of the reasons behind the difference in net asset impacts is number of cattle holding. 	In the appendix, we examine the impacts on the subcomponents of net assets. The \textsf{Cattle holding} row in \textsc{\footnotesize Figure \ref{fig ImpactsOnAllAssetsCumRelativeToConcurrentTradEffects}} shows the impacts on number of cattle owned and it also serves as a check that non-\textsf{traditional} members actually own cattle once the loan is made. The ANCOVA estimates plotted in the figure are net of baseline cattle holding, so even the non-traditional holding estimates sometimes add up to less than 1. %As shown in column (1) of \textsc{\normalsize Table \ref{tab ANCOVA cow time varying}}, both \textsf{traditional} arm and the non-\textsf{traditional} arms increase the cattle ownership. 
	The figure shows that, on average, the non-\textsf{traditional} arms continue to own about .4 more cattle than the \textsf{traditional} arm members conditional on the initial cattle holding, although estimates are indistinguishable from zero for some arms and regression specifications. 

	To gain insights on larger cattle holding among the non-\textsf{traditional} members, we decompose the cattle ownership of each arms in \textsc{\footnotesize Figure \ref{fig CattleHoldingArm}}. Holder rates (\textsf{HolderRates}) are the number of cattle owners per arm size, holding size (\textsf{HoldingSize}) is average holding per owner, initial owners' holding (\textsf{InitalOwnerHolding}) are herd size for owners who held cattle at baseline, and per capita holding (\textsf{PerCapitaHolding}) is mean cattle holding in each arm. Initial owner holding and holder rates reflect impacts on the intensive and extensive margins, respectively. Per capita holding tracks impacts on both the intensive margins (growth of initial owners) and the extensive margins (growth of new owners). All the indicators are similar across arms at the baseline. 
	
	We see that the holder rates increased in all arms, although the increase was smallest for the \textsf{traditional}. This shows that, even the small upfront lending of \textsf{traditional} arm helped increase catte ownership but to a lesser degree. With no large upfront liquidity provision and the repayment pressure that begins immediately after the disbursement, a smaller fraction of borrowers could purchase their first cattle. \textsf{HoldingSize} increased in all non-\textsf{traditional} arms, while the \textsf{traditional} arm remained stagnant. In \textsf{InitialOwnerHolding}, it is also the \textsf{traditional} arm that has the smallest, or negligible, changes between round 1 and 4. For the non-\textsf{traditional} arm, \textsf{InitialOwnerHolding} size is larger than the average holding size per owner, hinting the higher returns to members with experiences, or on the intensive margins. The per capita holding growth was smallest in the \textsf{traditional} arm. This is due to smaller changes on the extensive margins (fewer new ownership, smaller growth by new owners) and little change on the intensive margins (negligible growth by initial owners). 

	%One also sees that about 79\% of \textsf{traditional} arm members own cattle in period 2. This indicates that even a small loan helped some borrower to increase cattle ownership, but to a smaller magnitude than in the non-\textsf{traditional} arm.

\begin{figure}
\mpage{\linewidth}{
\hfil\textsc{\footnotesize Figure \refstepcounter{figure}\thefigure: Cattle holding by arm\label{fig CattleHoldingArm}}\\
\hfil\includegraphics[height = 8cm]{c:/data/GUK/analysis/program/figure/EstimationMemo/CattleHoldingByArm.pdf}\\
\renewcommand{\arraystretch}{1}
\hfil\begin{tabular}{>{\hfill\scriptsize}p{1cm}<{}>{\scriptsize}p{12cm}<{\hfill}}
Source: & Household survey data.\\
Note:& \textsf{HolderRate} is the ratio of cattle owners in each arm, \textsf{HoldingSize} is average holding per owner, \textsf{InitialOownerHolding} are average holding per owner who held cattle at baseline, and \textsf{PerCapitaHolding} is cattle owned per arm member. \textsf{InitialOownerHolding} and \textsf{HolderRates} show impacts on the intensive and extensive margins, respectively. \textsf{PerCapitaHolding} shows the time trend in mean cattle holding.\\[1ex]
\end{tabular}
}
\end{figure}


% \begin{figure}
% \mpage{\linewidth}{
% \hfil\textsc{\footnotesize Figure \refstepcounter{figure}\thefigure: All IGAs\label{fig AllIGAChoices}}\\
% 
% \vspace{1ex}
% \hfil\includegraphics[height = 6cm, width = 12cm]{      paste0(pathprogram, "figure/EstimationMemo/AllIGAChoicesByNumIGA.pdf")}\\
% \renewcommand{\arraystretch}{1}
% \hfil\begin{tabular}{>{\hfill\scriptsize}p{1cm}<{}>{\scriptsize}p{12cm}<{\hfill}}
% Source: & Administrative data, based on the information reported at the weekly meeting. Only borrowing member data are shown.\\
% Note:& Contents of IGAs are cattle, goat/sheep, growing cereals (paddy, corn) and nuts, small trades, and house and land leasing. Row panels indicate the total number of IGAs that borrowers own. For example, the row panel under the number `1' indicates the distribution of projects owned by single project members. There is no borrower with only one project in the \textsf{traditional} arm. 
% \end{tabular}
% }
% \end{figure}

\begin{figure}
\mpage{\linewidth}{
\hfil\textsc{\footnotesize Figure \refstepcounter{figure}\thefigure: Contents of first IGA \label{fig FirstIGAChoicesCattleVsOther}}\\
\hfil\includegraphics[height = 4cm, width = 8cm]{c:/data/GUK/analysis/program/figure/EstimationMemo/FirstIGAChoicesCattleVsOther.pdf}\\
\renewcommand{\arraystretch}{1}
\hfil\begin{tabular}{>{\hfill\scriptsize}p{1cm}<{}>{\scriptsize}p{12cm}<{\hfill}}
Source: & Administrative data, based on the information reported at the weekly meeting. Only borrowing member data are shown.\\
Note:&  Contents of IGAs are cattle, goat/sheep, growing cereals (paddy, corn) and nuts, small trades, and house and land leasing. The first IGA is defined as the oldest IGA for the household. Colour-filled bars are the cattle rearing, white bars are the sum of all other projects. \\[1ex]
\end{tabular}
}
\end{figure}


% \begin{figure}
% \mpage{\linewidth}{
% \hfil\textsc{\footnotesize Figure \refstepcounter{figure}\thefigure: All IGA choices\label{fig AllIGAChoicesCollapsed}}\\
% 
% \vspace{1ex}
% \hfil\includegraphics[height = 4cm, width = 12cm]{    paste0(pathprogram, "figure/ImpactEstimationOriginal1600Memo3/AllIGAChoicesCollapsed.eps")}\\
% \renewcommand{\arraystretch}{1}
% \hfil\begin{tabular}{>{\hfill\scriptsize}p{1cm}<{}>{\scriptsize}p{12cm}<{\hfill}}
% Source: & Administrative data. Only borrowing member data are shown.\\
% Note:& Based on information reported at the weekly meeting. The figure shows the sum collapsed over the total number of projects in each arms of \textsc{\footnotesize Figure \ref{fig AllIGAChoices}}. 
% \end{tabular}
% }
% \end{figure}
% 




	To understand the reasons behind the slower pace of asset accumulation of \textsf{traditional} arm, %in \textsc{\small Figure \ref{fig AllIGAChoices}}, we plot borrower's reported income generating activities (IGAs) separately by the total number of projects that the borrowers report. Contents of IGAs are cattle, goat/sheep, growing cereals (paddy, corn) and nuts, small trades, and house and land leasing. The row panel headed by the number `1' indicates the distribution of projects among single-project owners, `2' indicates the distribution among double-project owners, and so on. This shows that almost no one of the \textsf{traditional} arm invested only in one project while only few members did so in the non-\textsf{traditional} arms. We also note that there are a significant number of cases in the \textsf{traditional} arm that members reportedly raise cattle, yet they are also accompanied by pararell projects in smaller livestock production and small trades. Popularity of small trades and smaller livestock for the \textsf{traditional} arm members is consistent with convexity in the production technology of large domestic animals under a liquidity constraint. This also validates our supposition in experimental design that cattle production is the most preferred and probably the only economically viable investment choice. It eases a concern that the \textsf{cattle} arm may have imposed an unnecessary restriction in an investment choice by forcing to receive a heifer. 
	in \textsc{\footnotesize Figure \ref{fig FirstIGAChoicesCattleVsOther}}, we plot the contents of first IGAs of members. The first IGA is defined as the oldest IGA for the household. For most of the households, the oldest IGA had started after the baseline, and it is the IGA with the largest cash flow. Of course, there are a small percentage of households with an existing IGA before the baseline, but, with randomisation, the fraction of such households are similar across arms. Therefore, the between arm comparison of the first IGA gives us an idea about how the households had chosen the initial investments. In the \textsf{traditional} arm, there are 33 borrowing members who report cattle as their first IGA, and 76 borrowing members (69.72\%) who report other than cattle as their first IGA. This contrasts with the non-\textsf{traditional} arms that 466 borrowing members who report cattle as their first IGA and 25 borrowing members (5.09\%) other than cattle as their first IGA. Correspondingly, the data confirms that the \textsf{traditional} arm borrowers hold a diversified IGA portfolio while only a small minority of non-\textsf{traditional} arm borrowers have a diversified portfolio.\footnote{Results are available from the authors upon a request. }$^{,}$ \footnote{As observed earlier, a stagnant growth of \textsf{InitialOownerHolding} indicates the \textsf{traditional} arm initial owners diversified their portfolio rather than increasing the cattle investments. }
	%As can be seen from \textsc{\small Figure \ref{fig AllIGAChoices}}, for the 2nd and 3rd IGAs, a diversified IGA portfolio is continued to be held by all the \textsf{traditional} arm borrowers, and only the minority of non-\textsf{traditional} arm borrowers has a diversified portfolio.
	%\textsc{\small Figure \ref{fig AllIGAChoicesCollapsed}} collapses the all reported projects over borrowers and shows the total number of IGAs in each arms. 

\subsubsection{Labour incomes and consumption}

\begin{figure}
\mpage{12cm}{
\hfil\textsc{\footnotesize Figure \refstepcounter{figure}\thefigure: Cumulative effects on labour income and per capita consumption\label{fig IncomeConsumptionEffects}}\\

\vspace{2ex}
\hspace{-2em}\includegraphics[height = 8cm, width = 14cm]{c:/data/GUK/analysis/program/figure/EstimationMemo/ConsumptionIncomeCumRelativeToConcurrentTradEffects.pdf}\\
\renewcommand{\arraystretch}{1}
\setlength{\tabcolsep}{1pt}
\hfil\begin{tabular}{>{\hfill\scriptsize}p{1cm}<{}>{\scriptsize}p{12.5cm}<{\hfill}}
Source: & Constructed from ANCOVA estimation results 
\textsc{Table \ref{tab ANCOVA consumption timevarying}}, \textsc{Table \ref{tab ANCOVA consumption timevarying attributes original HH}}, \textsc{Table \ref{tab ANCOVA labour incomes timevarying}}, \textsc{Table \ref{tab ANCOVA labour incomes timevarying attributes}}.\\
Note:& Style and placement of panels follow the \textsc{\footnotesize Figure \ref{fig NetAssetEffects}}. \textsf{Large/Upfront}, \textsf{Large grace}, \textsf{Cattle} are impacts relative to \textsf{Traditional} arm. \textsf{WithGrace} and \textsf{InKind} are the impacts of respective marginal functional attributes. Panels show cumulative impacts of respective arm or attributes \textsf{k} relative to \textsf{tradiotional} arm which are obtained by $\mbox{2nd period}=b_{2k}, \mbox{3rd period}=b_{2k}+b_{3k}$, $\mbox{4th period}=b_{2k}+b_{4k}$ in the estimating equation $y_{it}=b_{1}y_{i1}+b_{2}+\bfb'_{2}\bfdee_{i}+b_{3}c_{3t}+\bfb'_{3}\bfdee_{i}c_{3t}+b_{4}c_{4t}+\bfb'_{4}\bfdee_{i}c_{4t}+e_{it}, \ t=2, 3, 4$, where $y_{it}$ is the outcome measure of member $i$ in period $t$, $\bfdee_{i}$ is a vector of arms or functional attributes, $c_{3t}, c_{4t}$ are indicator variables of period 3 and 4. Bars show 95\% confidence intervals using cluster robust standard errors. \textsf{Per capita consumption} is a total of food, hygiene, social, and energy expenditure divided by the number of household members, expressed as the annuaslied values in BDT. In-kind consumption of home made products is imputed at median prices. \textsf{Labour income} is labour incomes of household in 1000 BDT units. \\[1ex]
\end{tabular}
}
\end{figure}

	\textsc{\small Figure \ref{fig IncomeConsumptionEffects}} shows impacts on consumption and labour incomes. Style and placement of panels follow the \textsc{\footnotesize Figure \ref{fig NetAssetEffects}}. Consumption is not measured at the baseline, so we do not use it to understand the welfare impacts. Instead, using period 2 consumption as a reference point, we can understand how the members have dealt with the loan repayment through consumption choices. Given randomisation, one can still identify impacts on repayment efforts in terms of consumption suppression relative to the \textsf{traditional} arm. The upper row of \textsc{\small Figure \ref{fig IncomeConsumptionEffects}} plots ANCOVA estimates, conditional on period 2 consumption. In theory, this can be problematic as period 2 consumption can be correlated with arm assignment. But the results do not change if we estimate without period 2 consumption as a covariate in specification 1. Consumption is per capita expenditure of the household. Labour income is a household level variable and measures earnings from casual jobs. Both consumption and labour incomes do not show any impact by the arms or functional attributes.

\begin{figure}
\renewcommand{\arraystretch}{.6}
\mpage{\linewidth}{
\hfil\textsc{\footnotesize Figure \refstepcounter{figure}\thefigure: Mean per capita consumption and labour incomes by arm and period\label{fig MeanOutcomes}}\\
\hfil\includegraphics[width = 12cm]{c:/data/GUK/analysis/program/figure/EstimationMemo/MeanConsumptionIncomeByArmAndPeriod.pdf}\\
\renewcommand{\arraystretch}{1}
\hfil\begin{tabular}{>{\hfill\scriptsize}p{1cm}<{}>{\scriptsize}p{12cm}<{\hfill}}
Source: & Survey data.\\
Note:& Points indicate means, vertical bars indicate 95\% confidence intervals. \textsf{Per capita consumption} is an annualised total of food, hygiene, social, and energy expenditure divided by the number of household members in BDT. In-kind consumption of home made products is imputed at median prices. \textsf{Household labour income} is annual labour income of household in BDT. 
\end{tabular}
}
\end{figure}



	In \textsc{\footnotesize Figure \ref{fig MeanOutcomes}}, we see that, in all arms, the labour income is increasing from period 3, and per capita consumption did not change between periods 3 and 4 despite the growths in labour incomes.\footnote{One notes that the labour income is lowest in period 2 for all non-\textsf{traditional} arms, second lowest for the \textsf{traditional} arm, and start increasing from period 3. The fall in period 2 is due to the floods. Period 2 consumption is reportedly lower than period 3 and 4 because of flood damages. }$^{,}$ \footnote{Consumption is based on the annualised expenditure on following items: rice, wheat, maize, potato, lentil, other pulses, other staples, chicken, other meat, fish, milk, egg, chili, stem, carrot, leafy vegetable, other vegetable, banana, seasonal fruits, other fruit, puffed rice, onion, pgarlic, ginger, oil, sugar, salt, hard spices, soft spices, tea, bettle, other drinks, biscuit, cigarette, chew tobacco, transport, fuel wood, cloth, soap, haircut, cosmetic, communication, festivities, mosque related, contraceptive, wedding/funeral, other. It focuses on daily consumption, while education, health, housing, maintainance and other productive (livestock, farming) expenditures are not included.  } %As labour income growths and steady consumption are uniformly observed, it is accrued to the loan and its repayment, not to the functional attributes. 
	The households seem to have put asset accumulation and repayment a priority before consumption growths. It indicates that the borrowers did not choose to strategically default but tried to repay. 

\subsubsection{Repayments}

\begin{table}
\hfil\textsc{\footnotesize Figure \refstepcounter{figure}\thefigure: Cumulative weekly net repayment rates\label{fig weeklysavingrepayrate}}\\
\hfil\includegraphics{c:/data/GUK/analysis/program/figure/ImpactEstimationOriginal1600Memo2/CumulativeWeeklyRepaymentRateByPovertystatus.png}\\
\renewcommand{\arraystretch}{1}
\hfil\begin{tabular}{>{\hfill\scriptsize}p{1cm}<{}>{\scriptsize}p{12cm}<{\hfill}}
Note:& Each dot represents weekly observations. Only members who received loans are shown. Each panel shows ratios of cumulative repayment against cumulative due amount, sum of cumulative repayment and cumulative net saving (saving - withdrawal) against cumulative due amount, against weeks after first disbursement. Lines are smoothed lines with a penalized cubic regression spline in \textsf{ggplot2::geom\_smooth} function, originally from \textsf{mgcv::gam} with \textsf{bs=`cs'}. \\[-1ex]
\end{tabular}
\end{table}



	\textsc{\footnotesize Figure \ref{fig weeklysavingrepayrate}} shows the repayment results. Top panel shows the ratios of cumulative repayment to cumulative planned installment, the bottom panel shows the ratios of sum of cumulative repayment and cumulative net saving (saving - withdrawal) to cumulative planned installment. Both are plotted against weeks after first disbursement. Each dot represents a member at each time point. Value of 1, which is given by a horizontal line, indicates the member is at per with repayment schedule. Some members saved more than the required repayment at each time points that go beyond 1 in the figure. One sees that repayment rates are above 1 at the beginning but stay below 1 for most of the time. The majority of borrowing members did not repay the loan by the 48th month with prespecified installments. One notes the \textsf{traditional} arm has more of lower repayment rates among all arms. When a member does not reach the due amount with installments, they had to repay from the (net) saving, an arrangement to which the lender and the borrowers made at the loan contract signment. Repayment rates after using net saving are 44.71, 93.57, 97.01, 95.42\%, respectively, for \textsf{traditional, large, large grace, cow} arms, 87.85\% for overall, and 95.32\% for the average of non-\textsf{traditional} lending arms. %(from \textsf{\footnotesize AllMeetingsRepaymentInitialSample.rds}). 
	The overall repayment rate is comparable to the two microfinance programs with repayment rate information 74\% and 99\% examined in \citet{BanerjeeKarlanZinman2015}, and the non-\textsf{traditional} lending has exceptionally high repayment rates. The low repayment rates among \textsf{traditional} arm borrowers may be due to our experimental design that a new loan is disbursed unconditionally up to three cycles, lacking the dynamic incentives to repay, or due to the fact that they had lower returns on their investments. Our finding of growing labour incomes and the steady consumption levels indicates the latter possibility is more likely. 

	There is little difference in repayment rates by poverty classes. \textsc{\footnotesize Figure \ref{fig weeklysavingrepayrate}} depicts both moderately poor and ultra poor in different colours. It is impossible to distinguish between them with eyeballs, and ANCOVA estimates also confirms this  (see Appendix \ref{Sec App Repay}, \textsc{\small Table \ref{tab shortfall indiv o800}} for details). We also observe that impacts on all outcome measures are not statistically different between the poverty classes (see Appendix \ref{AppendixEstimationTables}). All of these are in contrast to a popular belief that the ultra poor are the riskiest among all income classes. %Poverty gradation through a participatory process, however, does not distinguish the moderately poor and the ultra poor on the observables. \textsc{\footnotesize Figure \ref{fig NetAssetValuesAtRd1}} shows net asset values at baseline by poverty class, and \textsc{\footnotesize Figure \ref{fig LivestockValuesAtRd1}} shows initial livestock values at baseline by poverty class. Both show little difference in these observable characteristics. \textcolor{red}{[According to Abu-san, participatory poverty gradation may have been imprecise.]}

	Smaller cumulative impacts and lower repayment rates of \textsf{traditional} arm members stand out once we acknowledge that they are receiving an equivalent amount and their contract differs with other arms only in the attributes we focus. These differences arose partly from the different investment choices observed in \textsc{\small Figure \ref{fig FirstIGAChoicesCattleVsOther}}, which were induced by the lack of \textsf{Upfront} functional attribute in lending.
	
% \mpage{12cm}{
% \hfil\textsc{\footnotesize Figure \refstepcounter{figure}\thefigure: Repayments\label{fig Repayments}}\\
% 
% \vspace{2ex}
% \hspace{-2em}\includegraphics[height = 10cm, width = 12cm]{       paste0(pathprogram, "figure/ImpactEstimationOriginal1600Memo3/Repayments.eps") }\\
% \renewcommand{\arraystretch}{1}
% \hfil\begin{tabular}{>{\hfill\scriptsize}p{1cm}<{}>{\scriptsize}p{11cm}<{\hfill}}
% Source: & Constructed from FD estimation results.\\
% Note:& CumImpactText3 \\[1ex]
% \end{tabular}
% }
%
%	Annualised repayment is depicted in \hfil\textsc{\footnotesize Figure \ref{fig Repayments}}. The top three panels show net saving. As in \textsc{\footnotesize Figure \ref{fig LivestockCumulativeEffects}}, each subpanel shows cumulative changes, per period changes, and differences in changes relative to concurrent changes of \textsf{traditional} arm. 
%
% \textsf{InKind} attribute does not increase repayment by itself. Repayment is highest with the \textsf{Upfront} attribute. It is driven by the initial year repayment and subsequent repayment is smaller than \textsf{traditional}. With \textsf{WithGrace} attribute, repayment is larger by more than Tk. 15000 in total during period 2 and 4 due to the accumulated saving in period 1 or the grace period. For net saving, there is a steady increase in all arms. \textsf{Upfront} and \textsf{WithGrace} attributes see a large boost in period 1 and the growth relative to \textsf{traditional} becomes negative subsequently.

\subsubsection{Schooling}


\begin{figure}
\hfil\mpage{12cm}{
\hfil\textsc{\footnotesize Figure \refstepcounter{figure}\thefigure: Period wise effects on schooling\label{fig SchoolingEffects}}\\

\vspace{2ex}
\hfil\includegraphics[height = 12cm, width = 12cm]{c:/data/GUK/analysis/program/figure/EstimationMemo/SchoolingEffects.png}\\
\renewcommand{\arraystretch}{1}
\hfil\begin{tabular}{>{\hfill\scriptsize}p{1cm}<{}>{\scriptsize}p{11cm}<{\hfill}}
Source: & Constructed from ANCOVA estimation results
\textsc{Table \ref{tab ANCOVA enroll time varying1}}, \textsc{Table \ref{tab ANCOVA enroll time varying attributes}}.\\
Note:&  Left most column panel shows the conditional means of \textsf{traditional} arm which serves as a benchmark in estimating impacts. In other column panels, all points show the relative difference from concurrent \textsf{traditional} levels depicted in the left most column. \textsf{Large} and \textsf{Upfront} are the same values. Other column panels are grouped either by arm or by attribute. Row panels show different outcomes. Bars show 95\% confidence intervals using cluster robust standard errors.  \\[1ex]
\end{tabular}
}
\end{figure}

	In Section \ref{ResultsSectionParticipation}, we observed that nonparticipation is correlated with smaller household size. If the household size limits the participation to microfinance, we may observe adverse impacts of borrowing on the children's school enrollment. In \textsc{\footnotesize Figure \ref{fig SchoolingEffects}}, the effects on child school enrollment are displayed. Unlike the previous figures, we show per period impacts relative to the concurrent \textsf{traditional} arm values. Using estimated parameters of \eqref{EstimatingEqTimeVarying}, what we display in \textsc{\footnotesize Figure \ref{fig SchoolingEffects}} are the per period impacts ($\bfb_{3k}$ for period 3 and $\bfb_{4k}$ for period 4), not the cumulative impacts ($\bfb_{2k}+\bfb_{3k}$ for period 3 and $\bfb_{2k}+\bfb_{4k}$ for period 4). We chose to show per period impacts because annual enrollment status matters in schooling. 
	
	In general, there is no detectable impact of the intervention, except for a negative impact for women at the college level for \textsf{Upfront} in period 4 and a positive impact for women at the college level for \textsf{WithGrace} in period 4. Women at the college level were found from about 5.9\% of the whole sample, so the effective sample size of each cell is about 11-12 (=776*.059/4), and it is difficult to interpret the results on these small samples. If anything, negative impacts of elder girl's schooling may be due to a stronger demand for cattle production in a household. This is in line with the finding in rejection that the limited household size can be a constraint on participation, especially when there is no grace period. Cattle ownership naturally shifts the relative shadow prices in a household against child schooling, especially for the elder girls as their returns on human capital are considered to be lower than younger girls, and the task contents of cattle rearing labour are less brawn intensive yet requires to be above the primary school ages. This may be a potential downside of having greater cattle production in a household.

\subsubsection{Summary of impacts}

	In summary, we found that our managerial support programs induce the members of disadvantaged background to participate in microfinance, achieving the further outreach, and achieve the results that are no different with other borrowers. This is consistent with the finding of the previous studies that a certain level of skills is necessary for participation, and our managerial support progams supplemented the lack thereof. We found that the large upfront disbursements allows borrowers to invest in cattle while members with sequential disbursements mostly opted for smaller livestock and small trades. In combination with a greater return to cattle on net asset accumulation and a greater rate of loan repayment, we consider it as evidence of a poverty trap and an effective measure to break it. We also found the impacts and repayment rates are indistinguishable between the moderately poor and the ultra poor.


\subsection{Robustness checks}
\label{subsec RobustnessCheck}


\begin{figure}
\renewcommand{\arraystretch}{.6}
\mpage{\linewidth}{
\hfil\textsc{\footnotesize Figure \refstepcounter{figure}\thefigure: Cumulative impacts on net assets relative to traditional arm by experience\label{fig AssetRelativeToCumulativeConcurrentTradEffectsByExperience}}\\
\hfil\includegraphics[width = 14cm]{c:/data/GUK/analysis/program/figure/EstimationMemo/NetAssetsByExperienceCumulativeRelativeToConcurrentTradEffects.pdf}\\
\renewcommand{\arraystretch}{.6}
\hfil\begin{tabular}{>{\hfill\scriptsize}p{1cm}<{}>{\scriptsize}p{12cm}<{\hfill}}
Source: & Estimated with survey data.\\
Note:&  \textsf{Adi} is a group who has an experience of lease-in cattle contract at the baseline, \textsf{Own} is a group who holds cattle at the baseline, and \textsf{None} are all other individuals. There are 141 members who owned cattle at the baseline, 112 members who ever practiced \textsf{Adi} at the baseline, and  523 members who have no experience in cattle rearing.\setlength{\baselineskip}{6pt}\\[2ex]
\end{tabular}
}

\renewcommand{\arraystretch}{.6}
\mpage{\linewidth}{
\hfil\textsc{\footnotesize Figure \refstepcounter{figure}\thefigure: Cumulative impacts on cattle holding relative to traditional arm by experience\label{fig NumCowsRelativeToCumulativeConcurrentTradEffectsByExperience}}\\
\hfil\includegraphics[width = 14cm]{c:/data/GUK/analysis/program/figure/EstimationMemo/NumCowsByExperienceCumulativeRelativeToConcurrentTradEffects.pdf}\\
\renewcommand{\arraystretch}{.6}
\hfil\begin{tabular}{>{\hfill\scriptsize}p{1cm}<{}>{\scriptsize}p{12cm}<{\hfill}}
Source: & Estimated with survey data.\\
Note:&  See the footnote of \textsc{\footnotesize Figure \ref{fig AssetRelativeToCumulativeConcurrentTradEffectsByExperience}}. \\[1ex]
\end{tabular}
}
\end{figure}

	The previous literature has shown that returns to lending are higher for the borrowers with business experiences \citep{BanerjeeKarlanZinman2015}. To check if the same can be found in our experiment, we divide the subjects into three groups of different cattle rearing experiences  at the baseline: \textsf{Own} group, defined by the cattle ownership, \textsf{Adi} group, defined by no cattle ownership but having an experience with cattle lease contracts (called \textit{Adi}) up to 3 years prior to the baseline, and \textsf{None} group who has neither of the two. In \textsc{\footnotesize Figure \ref{fig AssetRelativeToCumulativeConcurrentTradEffectsByExperience}} and \textsc{\footnotesize Figure \ref{fig NumCowsRelativeToCumulativeConcurrentTradEffectsByExperience}}, we plot the group wise impacts by arm on the net assets and cattle holding, respectively. These show that the \textsf{Own} group has the highest returns in both outcomes under the \textsf{Large/Upfront} treatment, followed by the \textsf{None} group, and virtually no impact among the \textsf{Adi} group.
	
	Consistent with the previous litereture, we thus find the returns to microfinance are higher among the members with previous cattle rearing experience through ownership. We also find the returns among the members with no previous experience are small yet not statiscally zero. In particular, when we choose cattle holding as an outcome, \textsf{Cattle} arm has a statistically meaningful impact even among the \textsf{None} group, which is consistent with our main finding that the managerial support program may have helped them in participating and sustaining the level of returns. 

	In contrast, the \textsf{Adi} group, who has cattle rearing experiences, does not attain statistically positive returns. As one examines the estimated results in \textsc{\normalsize Table \ref{tab ANCOVA narrow net assets Experience timevarying 1}} and \textsc{\normalsize Table \ref{tab ANCOVA narrow net assets Experience timevarying attributes 1}}, we see that the returns of \textsf{Adi} group becomes negligible once we add baseline household size \textsf{HHsize0} as a covariate, which has large positive estimates across specifications. This is indicative of \textsf{Adi} group is constrained by the household size, which is consistent with our main finding that a domestic capacity constraint, be it domestic labor or house building size or both, may bind some households to attain positive returns	, possibly because they are already engaging in cattle rearing.

	We also ran a robustness check over the choice of asset concepts by using various measures of net assets: \textsf{Broad net assets} which we include all other household assets that are observed in certain rounds of surveys, \textsf{Broad net assets, annual price} which we use annual median price of cattle in computing the livestock values in broad net assets, \textsf{Net non livestock assets} which we drop livestock values from net assets, \textsf{Land} which is the total of land asset values, and \textsf{Cattle} which is the number of cattle holding.  In the Appendix \textsc{\footnotesize Figure \ref{fig AssetCumRelativeToConcurrentTradEffects}}, we show the time paths of various assets by arm. The dynamic patterns of asset accumulation is similar to \textsc{\footnotesize Figure \ref{fig NetAssets}}.\footnote{\textsf{Net assets}=total assets - debts. Debts include outstanding loaned amount of the experiment. Total assets use items observed in all 4 rounds of household surveys. \textsf{Broad net assets}=total broad assets - debts, where total broad assets use all assets observed in the household surveys. \textsf{Broad net assets annual price} use annual median price for computing livestock values. \textsf{Net NL assets}=\textsf{net assets}-livestock asset values, \textsf{Broad net NL assets}=\textsf{Broad net assets}-livestock asset values. \textsf{Net assets, broad net assets} uses median baseline price for livestock.  } 

	All asset measures show similar patterns (see \textsc{\footnotesize Figure \ref{fig ImpactsOnAllAssetsCumRelativeToConcurrentTradEffects}}). \textsf{Broad net assets} show a similar pattern yet the standard error bars cross zero in round 4 for some specifications, possibly because of larger noises in computing the values as some asset items are observed only in certain rounds. \textsf{Net non livestock assets} show smaller impacts that are not statistically distinguishable from zero, which is consistent with our findings that borrowers used the funds in productive investments and kept repayment efforts. 




\section{Conclusion}
\label{SecConclusion}

\begin{mdframed}[style={SecItemize}, frametitle={Conclusion}]
\begin{itemize}
\vspace{1.0ex}\setlength{\itemsep}{1.0ex}\setlength{\baselineskip}{12pt}
\item	Entrepreneurship is necessary for project success, even with a simpler production process.
\item	Upfront liquidity increases asset holding and repayment rates.
\item	Cattle has higher returns and lower risks, resulting in higher repayment rates, but also has larger initial fixed costs, possibly generating a poverty trap.
\item	Lending uptake is impeded by small household size, asset shocks, and a lack of supports for managerial capacity.
\item	If these are relaxed, a poverty trap may be overcome.
\item	In the remote rural setting, larger upfront loan suited to the project cash flow is shown to be Pareto improving, despite widely believed fears of inefficiency due to information asymmetry.
\item	Consumption and labour incomes were not affected in non-\textsf{traditional} arms. Labour incomes increased toward the end of repayment for all arms which can be a repayment effort.
\item	Schooling was not affected in general. It finds a sign of a loss to higher secondary level women, hinting a domestic labour constraint in cattle production. But there was also a positive impact for women at the higher secondary level in \textsf{WithGrace} arm. While these are possibilities, cell sample sizes are too small to draw anything conclusive.
\end{itemize}
\end{mdframed}

	The poverty reduction impacts of microfinance was a firm belief in the early days of microfinance. Yet it suffered from a puzzling weak spot that microfinance is slow to reach the ultra poor, which is still debated today. Recently, even the poverty reduction impacts are subject to doubts, and it has been shown that the only borrowers with experience or skills are able to leap benefits. In this study, we examined the role of entrepreneurship in leaping benefits. We showed, under the rural setting, experiences or entrepreneurship seem to matter for participation. We note the usefulness of having managerial supports available for the prospective clients of MFIs when expanding the credit to the ultra poor. 

	This study employs a stepped-wedge design of multiple arms to isolate different functional attributes of loan contracts: Frontloading, a grace period, and in-kind loan with management supports. These functional attributes are intended to relax various constraints in productive investsmens by the poor: A liquidity constraint, a saving constraint, and an entrepreneurship constraint. Only frontloading the disbursement matters in all outcomes, which signifies the importance of a liquidity constraint. With evidence that borrowers with frontloaded arms invested in cattle while the borrowers under incremental lending invested in multiple, smaller projects, and the repayment rates are higher for the frontloaded arms, we conclude that there is a poverty trap which cannot be overcome by the traditional approach of microfinance. Under the study's setting, escaping from the poverty trap requires frontloading the lending, not lending incrementally as practiced by the majority of microfinance institutions. In addition, lending rather than a transfer may suffice to support the transition. 
	
	While we did not observe additional impacts of managerial supports, we found that more members with disadvantaged background participated. This implies that managerial supports can invite more disadvantaged prospective borrowers without adversely affecting the outcomes. To expand the coverage to the ultra poor, it may be useful to have managerial support services.
	
	We have witnessed that a binding domestic capacity constraint may impede potential borrowers from participation. This limits the potential benefit of lending a larger amount from the start of the program. While it in unclear why the outsourced labour cannot substitute the domestic labour, one can consider organising an arrangement in each group, tended by the group members, to collectively graze the cattle during the daytime. This partly eases the domestic labour and/or space constraints faced by small households. 

	We note that our study site is rich in rainfall, giving more advantages to cattle production over sheep/goat production. In contrast, if the climate is more arid, sheep and goats are better suited because of less water logging and their greater viability in relying on natural grass. This raises a concern that our results may not directly transferrable to more arid areas. However, the key lesson from the study is the presence of fixed inputs in scaling the herd size. While sheep/goats are easier to scale than cattle, it will require larger land and roofed facilities at some point as one increases the herd size. This can effectively form nonconvexity in the production set, and large enough finance may allow herders to go pass the threshold. 

	We have seen that borrowers accumulated assets, increased labour supplies, but not increasing the consumption. This is consistent with a high morale of repayment, which can partly be explained by the lack of alternative lenders in the study area. With stronger incentives to repay, the evidence on stronger repayment discipline of large sized arm members need not generalise in the areas outside the study site. On the other hand, the necessity of codifiable knowledge in participation even for a simple production process and the scope for escaping the poverty trap with large, frontloaded lending may be generalisable to other rural areas with liquidity constraints.

{\footnotesize\bibliographystyle{aer}
\setlength{\baselineskip}{8pt}
\bibliography{c:/seiro/settings/TeX/seiro}
}

\appendix
\setcounter{section}{0}
\setcounter{figure}{0}
\setcounter{table}{0}
\renewcommand{\thefigure}{\Alph{section}\arabic{figure}}
\renewcommand{\thetable}{\Alph{section}\arabic{table}}
\renewcommand{\thesection}{\Alph{section}}




\section{Randomisation checks}
\label{AppSecRandomisation}
\setcounter{table}{0}

% Created in PermutationTestsContents.rnw(87)
\hspace{-1.5cm}\begin{minipage}[t]{14cm}
\hfil\textsc{\normalsize Table \refstepcounter{table}\thetable: Permutation test results\label{tab perm}}\\
\setlength{\tabcolsep}{.5pt}
\setlength{\baselineskip}{8pt}
\renewcommand{\arraystretch}{.50}
\hfil\begin{tikzpicture}
\node (tbl) {\input{c:/data/GUK/analysis/save/EstimationMemo/PermutationTestResults.tex}};
\end{tikzpicture}\\
\renewcommand{\arraystretch}{.8}
\setlength{\tabcolsep}{1pt}
\begin{tabular}{>{\hfill\scriptsize}p{1cm}<{}>{\hfill\scriptsize}p{.25cm}<{}>{\scriptsize}p{12cm}<{\hfill}}
Source:& \multicolumn{2}{l}{\scriptsize Estimated with GUK administrative and survey data.}\\
Notes: & 1. & \textsf{R}'s package \textsf{coin} is used for baseline group mean covariates to conduct approximate permutation tests. Number of repetition is set to 100000. Number of groups is 72. Holm's step-down method is used to adjust for multiple testing of a multi-factor grouping variable.
\\
& 2. & See the footnote of \textsc{Table \ref{tab MainTextIRjecters}} for description of variables. 
\end{tabular}
\end{minipage}

\vspace{2ex}
\mpage{\linewidth}{
\renewcommand{\arraystretch}{.6}
\hfil\textsc{\normalsize Table \refstepcounter{table}\thetable: Anova results for cattle holding equality by arm\label{table anova CattleHoldingArm}}\\
\hfil\input{c:/data/GUK/analysis/program/table/EstimationMemo/anovaCowResults.tex}\\
\renewcommand{\arraystretch}{1}
\hfil\begin{tabular}{>{\hfill\scriptsize}p{1cm}<{}>{\scriptsize}p{12cm}<{\hfill}}
Source: & Survey data.\\
Note:& Each column uses respective year cattle ownership information. Columns (1) to (5) tests cattle holding equality for each survey rounds. In column (2), we edited the data by assigning 1 to members of \textsf{Cattle} arm who report holding is NA or zero. For ANOVA and Kruskal-Wallis, each entry indicates $p$ values. ANOVA tests for the null of equality of all means under normality. Kruskal-Wallis tests for the null of no stochastic dominance among samples without using the normality assumption. Tukey's honest significant tests show difference in means and $p$ values in parenthesis that account for multiple testing under the normality assumption.  \\[1ex]
\end{tabular}}





% Randomisation tests are done in PermutationTestsContents.rnw
In \textsc{\normalsize Table \ref{tab perm}}, we use \textsf{independence\_test} of \textsf{R}'s \textsf{coin} package: Approximate permutation tests by randomly resampling 100000 times. The test examines if arm $a$ mean to be different from rest of other arm means. This is done by permuting group labels to get permuted distribution of arm $a$ means. If the arm $a$ does not differ from non-$a$ arms, then the $p$ value becomes non small. All the values are relatively large, except for \textsf{TimePref1, TimePref2} of \textsf{Traditional} arm are around 20\%.

In \textsc{\normalsize Table \ref{table anova CattleHoldingArm}}, we show the cattle ownership ratios by each arm  at various points in time and examine their equality with ANOVA, Kruskal-Wallis, and Tukey Honest Significant Test. For ANOVA and Kruskal-Wallis, each entry indicates $p$ values. ANOVA tests for the null of equality of all means under normality. Kruskal-Wallis tests for the null of no stochastic dominance among samples without using the normality assumption. Tukey's honest significant tests show difference in means and $p$ values in parenthesis that account for multiple testing under the normality assumption.  

In column (1), for example, the final round cattle holding is tested. ANOVA and Kruskall-Wallis give .06\% and .07\%, respectively. Tukey HST is tested for each pair wise differences in ownewship ratios. \textsf{Large-Traditional} shows .5016 percentage points larger for \textsf{Large} arm relative to \textsf{Traditional}, and the null $p$ value of equality is .02\%. Likewise, \textsf{Large grace-Traditional, Cattle-Traditional} give $p$ values of 2.35\% and 6.90\%, respectively. Differences between arms with large loan size, \textsf{Large, Large grace, Cattle} show relatively large $p$ values. It shows the results are statistically different between \textsf{Traditional} and the other arms. 

Similarly, columns (2) to (5) show test results at each different points of time. In column (2), we edited the data by assigning 1 to members of \textsf{Cattle} arm who report holding is NA or zero at round 4. We did so because there is a possibility of misreporting and decided to check the sensitivity of permutation test results if we correct them. We see effectively no difference between (1) and (2) except the difference \textsf{Cattle-Traditional} becomes larger and associated $p$ value becomes smaller. Looking at (5), all the $p$ values are large and do not indicate statistically meaningful differences between arms.


\section{Rejection}
\label{AppSecRejection}
\setcounter{table}{0}



Among 776 observations, there are 40 whose villages are washed away and 70 who by group rejected the assigned arms (traditional, large, large grace with 40, 20, 10 individuals, respectively). There are 31, 9, 13, 37 individuals who individually rejected traditional, large, large grace, cow, respectively. %Among attrited HHs, when were they lost?

%Reasons for attrition and relation to flood damage.

Use \textsf{coin} package's \textsf{independence\_test}: Approximate permutation tests by randomly resampling 100000 times.

% below form permutation tables for all groups examined in AttritionTestContents2.rnw




For risks preference values, the larger the more risk averse. For time preferences values 1 and 2, larger the more impatient. If time preference value 1 (3 months) is larger than value 2 (1 year 3 monhts), time inconsistent, if 3 months $<$ 1 year 3 months, a future bias.\footnote{\textsf{RiskPrefVal} is the respondent's choice of the acceptable minimum excess monetary value of the risky option over a certainty option. Lower values indicate a greater risk tolerance. \textsf{TimePref1val} is the respondent's choice of the acceptable minimum excess monetary value in 3 months that is no smaller than present monetary benefit, and \textsf{TimePref2Val} is the the minimum excess value in 1 year and 3 months that is no smaller than monetary benefits of 1 year from now. Lower values indicate a greater patience. If a respondent's \textsf{TimePref1val} is greater than \textsf{TimePref2val}, the respondent is considered to be present-biased. \textsf{PresentBias} is an indicator function that takes the value of 1 if the respondent is considered to be present-biased, 0 otherwise. }
% First Set of options:
% 1. A: 1000 BDT tomorrow vs. B: 1000 BDT in 3 months
% 2. A: 1000 BDT tomorrow vs. B: 1100 BDT in 3 months
% 3. A: 1000 BDT tomorrow vs. B: 1200 BDT in 3 months   
% 4. A: 1000 BDT tomorrow vs. B: 1400 BDT in 3 months
% 5. A: 1000 BDT tomorrow vs. B: 1600 BDT in 3 months   
% 6. A: 1000 BDT tomorrow vs. B: 2000 BDT in 3 months  
% Second Set of options:
% 1. A: 1000 BDT in a year (12 months) vs. B: 1000 BDT in 1 year and 3 months (15 months)
% 2. A: 1000 BDT in a year (12 months) vs. B: 1200 BDT in 1 year and 3 months (15 months)   
% 3. A: 1000 BDT in a year (12 months) vs. B: 1400 BDT in 1 year and 3 months (15 months)
% 4. A: 1000 BDT in a year (12 months) vs. B: 1600 BDT in 1 year and 3 months (15 months)
% 5. A: 1000 BDT in a year (12 months) vs. B: 2000 BDT in 1 year and 3 months (15 months)




\hfil\begin{minipage}[t]{14cm}\hfil\textsc{\normalsize Table \refstepcounter{table}\thetable: Permutation test results of rejection\label{tab1 Permutation test results of rejection}}\\\setlength{\tabcolsep}{.5pt}\setlength{\baselineskip}{8pt}\renewcommand{\arraystretch}{.50}\hfil\begin{tikzpicture}\node (tbl) {\input{c:/data/GUK/analysis/save/PermutationTests/RejectedPermutationTestResultso800.tex}};\end{tikzpicture}\\\begin{tabular}{>{\hfill\scriptsize}p{1cm}<{}>{\hfill\scriptsize}p{.25cm}<{}>{\scriptsize}p{12cm}<{\hfill}}Source:& \multicolumn{2}{l}{\scriptsize Estimated with GUK administrative and survey data.}\\ Notes: & 1. & \textsf{R}'s package \textsf{coin} is used for baseline mean covariates to conduct approximate permutation tests. Number of repetition is set to 100000. Step-down method is used to adjust for multiple testing of a multi-factor grouping variable. The second and third columns show means of each group. For \textsf{Arm}, proportions of non-traditional arm between two groups are tested. \\& 2. & \textsf{p-value.lower}, \textsf{p-value.mid}, \textsf{p-value.upper} indicate lower-bound, mid point value, and upper-bound of the $p$ values for observed test statistic and the null distribution, expressed in per centage units. \\& 3. & \textsf{HeadLiteracy} is an indicator variable of household head literacy. \textsf{HeadAge} is age of household head. \textsf{HHsize} is total number of household members. \textsf{FloodInRd1} is an indicator variable of flood exposure. \textsf{HAssetAmount} and \textsf{PAssetAmount} are amount of household and productive assets, respectively, in BDT, \textsf{NumCows} is cattle holding per household. \textsf{NetValue} is net asset values in BDT per housheold using asset items observed in all 4 rounds. \textsf{BroadNetValue} is net asset values in BDT per housheold for all asset items. \textsf{Attrited} indicates attrition rates in the household survey, and \textsf{GRejected} and \textsf{IRejected} show group rejection rates and individual rejection rates to the lending program. \textsf{Non-attriting borrowers} indicates the ratio of non-attriting borrowers to all borrowers. Because attrition and rejection are separate events, a household can reject and attrit, so non-attrited borrowers $\geqslant$ total - (rejected members + attrited members). USD 1 is about BDT 80.\textsf{RiskPrefVal} is the respondent's choice of the acceptable minimum excess monetary value of the risky option over a certainty option. Lower values indicate a greater risk tolerance. \textsf{TimePref1val} is the respondent's choice of the acceptable minimum excess monetary value in 3 months that is no smaller than present monetary benefit, and \textsf{TimePref2Val} is the the minimum excess value in 1 year and 3 months that is no smaller than monetary benefits of 1 year from now. Lower values indicate a greater patience. If a respondent's \textsf{TimePref1val} is greater than \textsf{TimePref2val}, the respondent is considered to be present-biased. \textsf{PresentBias} is an indicator function that takes the value of 1 if the respondent is considered to be present-biased, 0 otherwise.\end{tabular}\end{minipage}\\\vspace{2ex}\hfil\begin{minipage}[t]{14cm}\hfil\textsc{\normalsize Table \refstepcounter{table}\thetable: Permutation test results of rejection among traditional arm\label{tab1 Permutation test results of rejection among traditional arm}}\\\setlength{\tabcolsep}{.5pt}\setlength{\baselineskip}{8pt}\renewcommand{\arraystretch}{.50}\hfil\begin{tikzpicture}\node (tbl) {\input{c:/data/GUK/analysis/save/PermutationTests/RejectedInTradPermutationTestResultso800.tex}};\end{tikzpicture}\\\begin{tabular}{>{\hfill\scriptsize}p{1cm}<{}>{\hfill\scriptsize}p{.25cm}<{}>{\scriptsize}p{12cm}<{\hfill}}Source:& \multicolumn{2}{l}{\scriptsize Estimated with GUK administrative and survey data.}\\ Notes: & 1. & \textsf{R}'s package \textsf{coin} is used for baseline mean covariates to conduct approximate permutation tests. Number of repetition is set to 100000.\\& 2. &  See footnotes of \textsc{Table \ref{tab1 Permutation test results of rejection}}. \end{tabular}\end{minipage}\\\vspace{2ex}\hfil\begin{minipage}[t]{14cm}\hfil\textsc{\normalsize Table \refstepcounter{table}\thetable: Permutation test results of rejection among non-traditional arm\label{tab1 Permutation test results of rejection among non-traditional arm}}\\\setlength{\tabcolsep}{.5pt}\setlength{\baselineskip}{8pt}\renewcommand{\arraystretch}{.50}\hfil\begin{tikzpicture}\node (tbl) {\input{c:/data/GUK/analysis/save/PermutationTests/RejectedInNonTradPermutationTestResultso800.tex}};\end{tikzpicture}\\\begin{tabular}{>{\hfill\scriptsize}p{1cm}<{}>{\hfill\scriptsize}p{.25cm}<{}>{\scriptsize}p{12cm}<{\hfill}}Source:& \multicolumn{2}{l}{\scriptsize Estimated with GUK administrative and survey data.}\\ Notes: & 1. & \textsf{R}'s package \textsf{coin} is used for baseline mean covariates to conduct approximate permutation tests. Number of repetition is set to 100000.\\& 2. &  See footnotes of \textsc{Table \ref{tab1 Permutation test results of rejection among traditional arm}}. \end{tabular}\end{minipage}\\\vspace{2ex}\hfil\begin{minipage}[t]{14cm}\hfil\textsc{\normalsize Table \refstepcounter{table}\thetable: Permutation test results of rejecters, traditional vs. non-traditional arm\label{tab1 Permutation test results of rejecters, traditional vs. non-traditional arm}}\\\setlength{\tabcolsep}{.5pt}\setlength{\baselineskip}{8pt}\renewcommand{\arraystretch}{.50}\hfil\begin{tikzpicture}\node (tbl) {\input{c:/data/GUK/analysis/save/PermutationTests/TradNonTradRejectedPermutationTestResultso800.tex}};\end{tikzpicture}\\\begin{tabular}{>{\hfill\scriptsize}p{1cm}<{}>{\hfill\scriptsize}p{.25cm}<{}>{\scriptsize}p{12cm}<{\hfill}}Source:& \multicolumn{2}{l}{\scriptsize Estimated with GUK administrative and survey data.}\\ Notes: & 1. & \textsf{R}'s package \textsf{coin} is used for baseline mean covariates to conduct approximate permutation tests. Number of repetition is set to 100000.\\& 2. &  See footnotes of \textsc{Table \ref{tab1 Permutation test results of rejection among traditional arm}}. \end{tabular}\end{minipage}\\\vspace{2ex}\hfil\begin{minipage}[t]{14cm}\hfil\textsc{\normalsize Table \refstepcounter{table}\thetable: Permutation test results of group rejection\label{tab1 Permutation test results of group rejection}}\\\setlength{\tabcolsep}{.5pt}\setlength{\baselineskip}{8pt}\renewcommand{\arraystretch}{.50}\hfil\begin{tikzpicture}\node (tbl) {\input{c:/data/GUK/analysis/save/PermutationTests/GRejectedPermutationTestResultso800.tex}};\end{tikzpicture}\\\begin{tabular}{>{\hfill\scriptsize}p{1cm}<{}>{\hfill\scriptsize}p{.25cm}<{}>{\scriptsize}p{12cm}<{\hfill}}Source:& \multicolumn{2}{l}{\scriptsize Estimated with GUK administrative and survey data.}\\ Notes: & 1. & \textsf{R}'s package \textsf{coin} is used for baseline mean covariates to conduct approximate permutation tests. Number of repetition is set to 100000.\\& 2. &  See footnotes of \textsc{Table \ref{tab1 Permutation test results of rejection}}. \end{tabular}\end{minipage}\\\vspace{2ex}\hfil\begin{minipage}[t]{14cm}\hfil\textsc{\normalsize Table \refstepcounter{table}\thetable: Permutation test results of group rejection among traditional arm\label{tab1 Permutation test results of group rejection among traditional arm}}\\\setlength{\tabcolsep}{.5pt}\setlength{\baselineskip}{8pt}\renewcommand{\arraystretch}{.50}\hfil\begin{tikzpicture}\node (tbl) {\input{c:/data/GUK/analysis/save/PermutationTests/GRejectedInTradPermutationTestResultso800.tex}};\end{tikzpicture}\\\begin{tabular}{>{\hfill\scriptsize}p{1cm}<{}>{\hfill\scriptsize}p{.25cm}<{}>{\scriptsize}p{12cm}<{\hfill}}Source:& \multicolumn{2}{l}{\scriptsize Estimated with GUK administrative and survey data.}\\ Notes: & 1. & \textsf{R}'s package \textsf{coin} is used for baseline mean covariates to conduct approximate permutation tests. Number of repetition is set to 100000.\\& 2. &  See footnotes of \textsc{Table \ref{tab1 Permutation test results of rejection among traditional arm}}. \end{tabular}\end{minipage}\\\vspace{2ex}\hfil\begin{minipage}[t]{14cm}\hfil\textsc{\normalsize Table \refstepcounter{table}\thetable: Permutation test results of group rejection among non-traditional arm\label{tab1 Permutation test results of group rejection among non-traditional arm}}\\\setlength{\tabcolsep}{.5pt}\setlength{\baselineskip}{8pt}\renewcommand{\arraystretch}{.50}\hfil\begin{tikzpicture}\node (tbl) {\input{c:/data/GUK/analysis/save/PermutationTests/GRejectedInNonTradPermutationTestResultso800.tex}};\end{tikzpicture}\\\begin{tabular}{>{\hfill\scriptsize}p{1cm}<{}>{\hfill\scriptsize}p{.25cm}<{}>{\scriptsize}p{12cm}<{\hfill}}Source:& \multicolumn{2}{l}{\scriptsize Estimated with GUK administrative and survey data.}\\ Notes: & 1. & \textsf{R}'s package \textsf{coin} is used for baseline mean covariates to conduct approximate permutation tests. Number of repetition is set to 100000.\\& 2. &  See footnotes of \textsc{Table \ref{tab1 Permutation test results of rejection among traditional arm}}. \end{tabular}\end{minipage}\\\vspace{2ex}\hfil\begin{minipage}[t]{14cm}\hfil\textsc{\normalsize Table \refstepcounter{table}\thetable: Permutation test results of group rejecters, traditional vs. non-traditional arm\label{tab1 Permutation test results of group rejecters, traditional vs. non-traditional arm}}\\\setlength{\tabcolsep}{.5pt}\setlength{\baselineskip}{8pt}\renewcommand{\arraystretch}{.50}\hfil\begin{tikzpicture}\node (tbl) {\input{c:/data/GUK/analysis/save/PermutationTests/TradNonTradGRejectedPermutationTestResultso800.tex}};\end{tikzpicture}\\\begin{tabular}{>{\hfill\scriptsize}p{1cm}<{}>{\hfill\scriptsize}p{.25cm}<{}>{\scriptsize}p{12cm}<{\hfill}}Source:& \multicolumn{2}{l}{\scriptsize Estimated with GUK administrative and survey data.}\\ Notes: & 1. & \textsf{R}'s package \textsf{coin} is used for baseline mean covariates to conduct approximate permutation tests. Number of repetition is set to 100000.\\& 2. &  See footnotes of \textsc{Table \ref{tab1 Permutation test results of rejection among traditional arm}}. \end{tabular}\end{minipage}\\\vspace{2ex}\hfil\begin{minipage}[t]{14cm}\hfil\textsc{\normalsize Table \refstepcounter{table}\thetable: Permutation test results of individual rejection\label{tab1 Permutation test results of individual rejection}}\\\setlength{\tabcolsep}{.5pt}\setlength{\baselineskip}{8pt}\renewcommand{\arraystretch}{.50}\hfil\begin{tikzpicture}\node (tbl) {\input{c:/data/GUK/analysis/save/PermutationTests/IRejectedPermutationTestResultso800.tex}};\end{tikzpicture}\\\begin{tabular}{>{\hfill\scriptsize}p{1cm}<{}>{\hfill\scriptsize}p{.25cm}<{}>{\scriptsize}p{12cm}<{\hfill}}Source:& \multicolumn{2}{l}{\scriptsize Estimated with GUK administrative and survey data.}\\ Notes: & 1. & \textsf{R}'s package \textsf{coin} is used for baseline mean covariates to conduct approximate permutation tests. Number of repetition is set to 100000.\\& 2. &  See footnotes of \textsc{Table \ref{tab1 Permutation test results of rejection}}. \end{tabular}\end{minipage}\\\vspace{2ex}\hfil\begin{minipage}[t]{14cm}\hfil\textsc{\normalsize Table \refstepcounter{table}\thetable: Permutation test results of individual rejection among traditional arm\label{tab1 Permutation test results of individual rejection among traditional arm}}\\\setlength{\tabcolsep}{.5pt}\setlength{\baselineskip}{8pt}\renewcommand{\arraystretch}{.50}\hfil\begin{tikzpicture}\node (tbl) {\input{c:/data/GUK/analysis/save/PermutationTests/IRejectedInTradPermutationTestResultso800.tex}};\end{tikzpicture}\\\begin{tabular}{>{\hfill\scriptsize}p{1cm}<{}>{\hfill\scriptsize}p{.25cm}<{}>{\scriptsize}p{12cm}<{\hfill}}Source:& \multicolumn{2}{l}{\scriptsize Estimated with GUK administrative and survey data.}\\ Notes: & 1. & \textsf{R}'s package \textsf{coin} is used for baseline mean covariates to conduct approximate permutation tests. Number of repetition is set to 100000.\\& 2. &  See footnotes of \textsc{Table \ref{tab1 Permutation test results of rejection among traditional arm}}. \end{tabular}\end{minipage}\\\vspace{2ex}\hfil\begin{minipage}[t]{14cm}\hfil\textsc{\normalsize Table \refstepcounter{table}\thetable: Permutation test results of individual rejection among non-traditional arm\label{tab1 Permutation test results of individual rejection among non-traditional arm}}\\\setlength{\tabcolsep}{.5pt}\setlength{\baselineskip}{8pt}\renewcommand{\arraystretch}{.50}\hfil\begin{tikzpicture}\node (tbl) {\input{c:/data/GUK/analysis/save/PermutationTests/IRejectedInNonTradPermutationTestResultso800.tex}};\end{tikzpicture}\\\begin{tabular}{>{\hfill\scriptsize}p{1cm}<{}>{\hfill\scriptsize}p{.25cm}<{}>{\scriptsize}p{12cm}<{\hfill}}Source:& \multicolumn{2}{l}{\scriptsize Estimated with GUK administrative and survey data.}\\ Notes: & 1. & \textsf{R}'s package \textsf{coin} is used for baseline mean covariates to conduct approximate permutation tests. Number of repetition is set to 100000.\\& 2. &  See footnotes of \textsc{Table \ref{tab1 Permutation test results of rejection among traditional arm}}. \end{tabular}\end{minipage}\\\vspace{2ex}\hfil\begin{minipage}[t]{14cm}\hfil\textsc{\normalsize Table \refstepcounter{table}\thetable: Permutation test results of individual rejecters, traditional vs. non-traditional arm\label{tab1 Permutation test results of individual rejecters, traditional vs. non-traditional arm}}\\\setlength{\tabcolsep}{.5pt}\setlength{\baselineskip}{8pt}\renewcommand{\arraystretch}{.50}\hfil\begin{tikzpicture}\node (tbl) {\input{c:/data/GUK/analysis/save/PermutationTests/TradNonTradIRejectedPermutationTestResultso800.tex}};\end{tikzpicture}\\\begin{tabular}{>{\hfill\scriptsize}p{1cm}<{}>{\hfill\scriptsize}p{.25cm}<{}>{\scriptsize}p{12cm}<{\hfill}}Source:& \multicolumn{2}{l}{\scriptsize Estimated with GUK administrative and survey data.}\\ Notes: & 1. & \textsf{R}'s package \textsf{coin} is used for baseline mean covariates to conduct approximate permutation tests. Number of repetition is set to 100000.\\& 2. &  See footnotes of \textsc{Table \ref{tab1 Permutation test results of rejection among traditional arm}}. \end{tabular}\end{minipage}\\\vspace{2ex}\hfil\begin{minipage}[t]{14cm}\hfil\textsc{\normalsize Table \refstepcounter{table}\thetable: Permutation test results of borrowers, cattle vs. non-cattle arms\label{tab1 Permutation test results of borrowers, cattle vs. non-cattle arms}}\\\setlength{\tabcolsep}{.5pt}\setlength{\baselineskip}{8pt}\renewcommand{\arraystretch}{.50}\hfil\begin{tikzpicture}\node (tbl) {\input{c:/data/GUK/analysis/save/PermutationTests/AcceptedCowNonCowPermutationTestResultso800.tex}};\end{tikzpicture}\\\begin{tabular}{>{\hfill\scriptsize}p{1cm}<{}>{\hfill\scriptsize}p{.25cm}<{}>{\scriptsize}p{12cm}<{\hfill}}Source:& \multicolumn{2}{l}{\scriptsize Estimated with GUK administrative and survey data.}\\ Notes: & 1. & \textsf{R}'s package \textsf{coin} is used for baseline mean covariates to conduct approximate permutation tests. Number of repetition is set to 100000.\\& 2. &  See footnotes of \textsc{Table \ref{tab1 Permutation test results of rejection among traditional arm}}. \end{tabular}\end{minipage}\\\vspace{2ex}


%	\noindent\textsc{\normalsize Table \ref{grep(TabLabelStrings[1], TabLabel1)}} to \textsc{\normalsize Table \ref{grep(TabLabelStrings[13], TabLabel1)}}: Trimmed sample.\\

	\textsc{\normalsize Table \ref{tab1 Permutation test results of rejection}} shows test results of independence between loan receivers and nonreceivers (group, individual rejecters) on the analysis sample of 776 members. It shows that lower head literacy, smaller household size, being affected by flood at the baseline, smaller livestock holding, and smaller net assets are correlated with opting out the offered type of lending. \textsc{\normalsize Table \ref{tab1 Permutation test results of rejection among traditional arm}} indicates that lower asset and livestock holding is more pronounced among \textsf{traditional} rejecters relative to loan receivers. It also shows that flood exposure is less frequent, contrary to \textsc{\normalsize Table \ref{tab1 Permutation test results of rejection}}, among the rejecters. \textsc{\normalsize Table \ref{tab1 Permutation test results of rejection among non-traditional arm}} indicates that lower head literacy, smaller household size, higher flood exposure, are more pronounced among non-\textsf{traditional} rejecters relative to loan receivers. It also shows that asset and livestock holding is no different relative to the receivers. Comparing rejecters of \textsf{traditional} arm, lower flood exposure may be the only stark difference against non-\textsf{traditional} arm members, and smaller asset and livestock holding is merely suggestive (\textsc{\normalsize Table \ref{tab1 Permutation test results of rejecters, traditional vs. non-traditional arm}}). 
	
	Group rejecters and non-group rejecters are compared in \textsc{\normalsize Table \ref{tab1 Permutation test results of group rejection}}. Marked differences are found in \textsf{Arm} (proportion of non-\textsf{traditional} arm members) and head literacy. \textsf{TimePrefVal1} and \textsf{TimePrefVal2} are values of premium required to give up the immediate gratification, now or 1 year from now, respectively. \textsf{TimePrefVal2} shows that group rejecters are less impatient than the non-group rejecters. In the meantime, group rejecters have a higher proportion of individuals with present bias as indicated in \textsf{PresentBias}. There are no difference in terms of risk tolerance in \textsf{RiskPrefVal}. Group rejecters tend to have smaller livestock assets, as indicated by \textsf{NumCows, LivestockValue} and smaller overall assets in \textsf{NarrowNetValue, NetValue}. 
	\textsc{\normalsize Table \ref{tab1 Permutation test results of group rejection among traditional arm}} compares group rejecters in \textsf{traditional} arm and finds smaller flood exposure and lower livestock and net asset holding are associated with group rejection. Group rejecters in non-\textsf{traditional} arm are examined in \textsc{\normalsize Table \ref{tab1 Permutation test results of group rejection among non-traditional arm}} and younger head age, flood at baseline, and smaller household asset holding are correlated with rejection. We also note group rejecters in non-\textsf{traditional} arms are less impatient but have a higher proportion of present biased members. 
	Comparing group rejecters between \textsf{traditional} and non-\textsf{traditional} arms in \textsc{\normalsize Table \ref{tab1 Permutation test results of group rejecters, traditional vs. non-traditional arm}}, younger head age, higher flood exposure, larger net asset values and livestock holding are noted among the non-\textsf{traditional} group rejecters. These hint that for non-\textsf{traditional} arm group rejecters, it is the smaller household size and the baseline flood that may have constrained them from participation, and for \textsf{traditional} group rejecters, it is the low asset levels.

	Acknowledging the reasons for rejection can be different, we tested the independence of each characteristics for individual rejecters (vs. non-individual rejeceters) in \textsc{\normalsize Table \ref{tab1 Permutation test results of individual rejection}}. Smaller \textsf{HHsize}, being affected with \textsf{FloodInRd1}, and smaller \textsf{LivestockValue}, \textsf{NumCows}, \textsf{NetValue}, \textsf{NarrowNetValue}, and smaller risk tolerance in \textsf{RiskPrefVal} are associated with individual rejecters. Individual decisions not to participate may be more straightforward than group rejection: Smaller household size may indicate difficulty in securing the cattle production labour in a household, being hit with a flood may have resulted in lower livestock and asset levels that would prompt them to reconsider partaking in another livestock project. 

	\textsc{\normalsize Table \ref{tab1 Permutation test results of individual rejection among traditional arm}} and \textsc{\normalsize Table \ref{tab1 Permutation test results of individual rejection among non-traditional arm}} compare individual rejecters and nonrejecters in \textsf{traditional} arm and non-\textsf{traditional} arms, respectively. For both \textsf{traditional} and non-\textsf{traditional} rejecters, livestock and other asset values are not correlated with rejection, but the values are similar to \textsc{\normalsize Table \ref{tab1 Permutation test results of individual rejection}} but with higher $p$ values, indicating the results are due to smaller sample size. For non-\textsf{traditional} arm rejecters, household size and flood exposure are correlated. Comparison of individual rejecters between \textsf{traditional} and non-\textsf{traditional} arms show no detectable difference (\textsc{\normalsize Table \ref{tab1 Permutation test results of individual rejecters, traditional vs. non-traditional arm}}). This suggests that indvidual rejecters in all arms were constrained with small household size and small asset holding. In \textsc{\normalsize Table \ref{tab1 Permutation test results of borrowers, cattle vs. non-cattle arms}}, we compare if the \textsf{cattle} arm participants (borrowers) differ from participants in other arms at the baseline. It is worth noting that partcipants of \textsf{cattle} arm differ from other arms in having less cattle rearing experience as observed in smaller initial cattle holding ($p$ value = .156) and in having lower net asset values ($p$ value = .058), weakly hinting that the \textsf{cattle} arm's managerial support programs may have encouraged participation of inexperienced or lower asset holders. The \textsf{cattle} arm participants are more impatient than non-\textsf{cattle} arm participants as indicated in \textsf{TimePrefVal1, TimePrefVal2}, and have higher proportion of head literacy. \label{RejectionLastPage}

	%	A closer look at the nonparticipation correlates among \textsf{traditional} arm mebers in \textsc{\normalsize Table \ref{tab reject trad perm}} and non-\textsf{traditional} arm members in \textsc{\normalsize Table \ref{tab reject nontrad perm}} reveal possible differences in the causes. Rejection among \textsf{traditional} members tend to be associated with lower livestock holding but not with higher flood exposure nor smaller household size, while rejecters among non-\textsf{traditional} members are more likely to have suffered from flood in baseline and have smaller household size. Since the offered arms were randomised, rejecters of \textsf{traditional} arm, who were not more exposed to flood and have similar household size at the mean, may have accepted the offer had they been offered non- \textsf{traditional} lending. Henceforth, we conjecture that flood exposure and household size are the potential impediments in larger size loans. This implies that there may not be minimum livestock and asset holding levels to partake the larger loans, and a poverty trap at this level may be overcome.



\section{Attrition}
\label{AppSecAttrition}
\setcounter{table}{0}


% Attrition tests are conducted in AttritionTests2.rnw
% TabLabel1 is from paste0(pathprogram, "AttritionPermutationTableHeaders.R") referred in the main rnw file


% TabLabelStrings is from RejectionTestsMainText4.rnw 
% I will modify the table headers for survival/surviving members to "non-attriting borrowers"
%   "survival$", "surviving.*race$", "surv.*other"
\hfil\begin{minipage}[t]{14cm}\hfil\textsc{\normalsize Table \refstepcounter{table}\thetable: Permutation test results of attrition\label{tab1 Permutation test results of attrition}}\\\setlength{\tabcolsep}{.5pt}\setlength{\baselineskip}{8pt}\renewcommand{\arraystretch}{.50}\hfil\begin{tikzpicture}\node (tbl) {\input{c:/data/GUK/analysis/save/PermutationTests/AttritedPermutationTestResultso800.tex}};\end{tikzpicture}\\\begin{tabular}{>{\hfill\scriptsize}p{1cm}<{}>{\hfill\scriptsize}p{.25cm}<{}>{\scriptsize}p{12cm}<{\hfill}}Source:& \multicolumn{2}{l}{\scriptsize Estimated with GUK administrative and survey data.}\\ Notes: & 1. & \textsf{R}'s package \textsf{coin} is used for baseline mean covariates to conduct approximate permutation tests. Number of repetition is set to 100000.\\& 2. &  See footnotes of \textsc{Table \ref{tab1 Permutation test results of rejection}}. \end{tabular}\end{minipage}\\\vspace{2ex}\hfil\begin{minipage}[t]{14cm}\hfil\textsc{\normalsize Table \refstepcounter{table}\thetable: Permutation test results of attrition among traditional arm\label{tab1 Permutation test results of attrition among traditional arm}}\\\setlength{\tabcolsep}{.5pt}\setlength{\baselineskip}{8pt}\renewcommand{\arraystretch}{.50}\hfil\begin{tikzpicture}\node (tbl) {\input{c:/data/GUK/analysis/save/PermutationTests/AttritedInTradPermutationTestResultso800.tex}};\end{tikzpicture}\\\begin{tabular}{>{\hfill\scriptsize}p{1cm}<{}>{\hfill\scriptsize}p{.25cm}<{}>{\scriptsize}p{12cm}<{\hfill}}Source:& \multicolumn{2}{l}{\scriptsize Estimated with GUK administrative and survey data.}\\ Notes: & 1. & \textsf{R}'s package \textsf{coin} is used for baseline mean covariates to conduct approximate permutation tests. Number of repetition is set to 100000.\\& 2. &  See footnotes of \textsc{Table \ref{tab1 Permutation test results of rejection among traditional arm}}. \end{tabular}\end{minipage}\\\vspace{2ex}\hfil\begin{minipage}[t]{14cm}\hfil\textsc{\normalsize Table \refstepcounter{table}\thetable: Permutation test results of attrition among non-traditional arm\label{tab1 Permutation test results of attrition among non-traditional arm}}\\\setlength{\tabcolsep}{.5pt}\setlength{\baselineskip}{8pt}\renewcommand{\arraystretch}{.50}\hfil\begin{tikzpicture}\node (tbl) {\input{c:/data/GUK/analysis/save/PermutationTests/AttritedInNonTradPermutationTestResultso800.tex}};\end{tikzpicture}\\\begin{tabular}{>{\hfill\scriptsize}p{1cm}<{}>{\hfill\scriptsize}p{.25cm}<{}>{\scriptsize}p{12cm}<{\hfill}}Source:& \multicolumn{2}{l}{\scriptsize Estimated with GUK administrative and survey data.}\\ Notes: & 1. & \textsf{R}'s package \textsf{coin} is used for baseline mean covariates to conduct approximate permutation tests. Number of repetition is set to 100000.\\& 2. &  See footnotes of \textsc{Table \ref{tab1 Permutation test results of rejection among traditional arm}}. \end{tabular}\end{minipage}\\\vspace{2ex}\hfil\begin{minipage}[t]{14cm}\hfil\textsc{\normalsize Table \refstepcounter{table}\thetable: Permutation test results of attriters of traditional and non-traditional arms\label{tab1 Permutation test results of attriters of traditional and non-traditional arms}}\\\setlength{\tabcolsep}{.5pt}\setlength{\baselineskip}{8pt}\renewcommand{\arraystretch}{.50}\hfil\begin{tikzpicture}\node (tbl) {\input{c:/data/GUK/analysis/save/PermutationTests/TradNonTradAttritedPermutationTestResultso800.tex}};\end{tikzpicture}\\\begin{tabular}{>{\hfill\scriptsize}p{1cm}<{}>{\hfill\scriptsize}p{.25cm}<{}>{\scriptsize}p{12cm}<{\hfill}}Source:& \multicolumn{2}{l}{\scriptsize Estimated with GUK administrative and survey data.}\\ Notes: & 1. & \textsf{R}'s package \textsf{coin} is used for baseline mean covariates to conduct approximate permutation tests. Number of repetition is set to 100000.\\& 2. &  See footnotes of \textsc{Table \ref{tab1 Permutation test results of rejection among traditional arm}}. \end{tabular}\end{minipage}\\\vspace{2ex}\hfil\begin{minipage}[t]{14cm}\hfil\textsc{\normalsize Table \refstepcounter{table}\thetable: Permutation test results of active status\label{tab1 Permutation test results of active status}}\\\setlength{\tabcolsep}{.5pt}\setlength{\baselineskip}{8pt}\renewcommand{\arraystretch}{.50}\hfil\begin{tikzpicture}\node (tbl) {\input{c:/data/GUK/analysis/save/PermutationTests/ActivePermutationTestResultso800.tex}};\end{tikzpicture}\\\begin{tabular}{>{\hfill\scriptsize}p{1cm}<{}>{\hfill\scriptsize}p{.25cm}<{}>{\scriptsize}p{12cm}<{\hfill}}Source:& \multicolumn{2}{l}{\scriptsize Estimated with GUK administrative and survey data.}\\ Notes: & 1. & \textsf{R}'s package \textsf{coin} is used for baseline mean covariates to conduct approximate permutation tests. Number of repetition is set to 100000.\\& 2. &  See footnotes of \textsc{Table \ref{tab1 Permutation test results of rejection}}. \end{tabular}\end{minipage}\\\vspace{2ex}\hfil\begin{minipage}[t]{14cm}\hfil\textsc{\normalsize Table \refstepcounter{table}\thetable: Permutation test results of active members of cattle and large grace arms\label{tab1 Permutation test results of active members of cattle and large grace arms}}\\\setlength{\tabcolsep}{.5pt}\setlength{\baselineskip}{8pt}\renewcommand{\arraystretch}{.50}\hfil\begin{tikzpicture}\node (tbl) {\input{c:/data/GUK/analysis/save/PermutationTests/ActiveCowLargeGracePermutationTestResultso800.tex}};\end{tikzpicture}\\\begin{tabular}{>{\hfill\scriptsize}p{1cm}<{}>{\hfill\scriptsize}p{.25cm}<{}>{\scriptsize}p{12cm}<{\hfill}}Source:& \multicolumn{2}{l}{\scriptsize Estimated with GUK administrative and survey data.}\\ Notes: & 1. & \textsf{R}'s package \textsf{coin} is used for baseline mean covariates to conduct approximate permutation tests. Number of repetition is set to 100000.\\& 2. &  See footnotes of \textsc{Table \ref{tab1 Permutation test results of rejection among traditional arm}}. \end{tabular}\end{minipage}\\\vspace{2ex}\hfil\begin{minipage}[t]{14cm}\hfil\textsc{\normalsize Table \refstepcounter{table}\thetable: Permutation test results of active members of cattle and all other arms\label{tab1 Permutation test results of active members of cattle and all other arms}}\\\setlength{\tabcolsep}{.5pt}\setlength{\baselineskip}{8pt}\renewcommand{\arraystretch}{.50}\hfil\begin{tikzpicture}\node (tbl) {\input{c:/data/GUK/analysis/save/PermutationTests/ActiveCowNonCowPermutationTestResultso800.tex}};\end{tikzpicture}\\\begin{tabular}{>{\hfill\scriptsize}p{1cm}<{}>{\hfill\scriptsize}p{.25cm}<{}>{\scriptsize}p{12cm}<{\hfill}}Source:& \multicolumn{2}{l}{\scriptsize Estimated with GUK administrative and survey data.}\\ Notes: & 1. & \textsf{R}'s package \textsf{coin} is used for baseline mean covariates to conduct approximate permutation tests. Number of repetition is set to 100000.\\& 2. &  See footnotes of \textsc{Table \ref{tab1 Permutation test results of rejection among traditional arm}}. \end{tabular}\end{minipage}\\\vspace{2ex}
% uptake: 105+171+177+163
% rejected: rjcted <- 71+29+23+37
% attrited: 32+8+29+23
% borrower nonattriting (survivors): 83+164+160+147=83+164+160+147
% borrower attriting: boratt <- (105-83)+(171-164)+(177-160)+(163-147)
% nonborrower nonattriting: 61+28+11+30
% nonborrower attriting: (40+31-36-25)+(20+9-19-9)+(10+13-11)+(37-39)
% rejected or attrited (nonsurvivors): borrower attriting + rejected = boratt + rjcted

%	\textsc{\normalsize Table \ref{TabLabel1[1]}} shows results from tests of independence between attriters and nonattriters. We see the moderate rate of attrition is not correlated with household level characteristics at the conventional $p$ value level. Productive asset amounts seem to differ between attriters and nonattriters, with the former being larger than the latter. This positive attrition selection can cause underestimation of impacts, if the asset values are positively correlated with entrepreneurial capacity. \textsc{\normalsize Table \ref{TabLabel1[grep("of attri.* traditional arm$", TabLabel1)]}} shows attrition in the \textsf{traditional} arm. Household heads of attriters are relatively less literate than nonattriters. \textsc{\normalsize Table \ref{TabLabel1[grep("of attri.* non-traditional arm$", TabLabel1)]}} compares attriters and nonattriters in the non-\textsf{traditional} arm. Unlike \textsf{traditional} arm attriters, non-\textsf{traditional} arm attriters have more literate household heads, have a larger household size, are more exposed to floods, and have larger productive assets. The \textsf{traditional} arm attriters may be less entrepreneurial, if anything, so their attrition may upwardly bias the positive gains of the arm, hence understate the impacts of non-\textsf{traditional} arm. These are explicitly shown in \textsc{\normalsize Table \ref{TabLabel1[grep("of attri.* of", TabLabel1)]}} where we compare attriters of \textsf{traditional} and non-\textsf{traditional} arms. Overall, attrition may have attenuated the impacts but is not likely to have inflated them.\footnote{So one can employ the Lee bounds for stronger results, but doing so will give us less precision and require more assumptions. We will not use the Lee bounds \textcolor{red}{[we can show them if necessary]}. }

%	For the microfinance institutions (MFIs), attrition of the loan receiving members poses a threat to their business continuation. Financial institutions often use observable characteristics, such as collateralisable assets, and easily surveyed chracteristics, such as job experiences and schooling of borrowers, and are likely to lend if the assets levels are greater and the borrowers have relevant job experiences and more schooling. We first examine if such screening variables have any predictive power in terms of loan rejection or borrower attrition under our lending. \textsc{\normalsize Table \ref{TabLabel1[grep("survival$", TabLabel1)]}} compares potential MFI targets (nonattriting borrowers, noted as \textsf{Survived}) vs. non-targets (attriting borrowers or loan rejecters, noted as \textsf{NonSurvived}) in all arms. It shows potential targets at the baseline have larger values in livestock and greater number of cattle, and are less affected by the flood, which conforms the conventional wisdom of lenders in using these aspects in their loan decisions. Next, we examine if the relationship of having ``less favourable'' values in these characteristics and attrition is mitigated under various loan characteristics. In \textsc{\normalsize Table \ref{gsub("surviving members", "non-attriting borrowers", TabLabel1[grep("surviving members of cattle and large grace", TabLabel1)])}}, we restrict our attention to the potential MFI targets, or the nonattriting borrowers, and compare between \textsf{cattle} and \textsf{large grace} arms, whose difference is efffectively the presence of managerial supports that the former provides. \label{PageOfAttrition}%Comparing the nonattriting borrower, characteristics are similar except that the \textsf{traditional} members are more exposed to the flood than the non-\textsf{traditional} members. 
%	Comparing against the \textsf{large grace} arm, nonattriting borrowers of the \textsf{cattle} arm are more exposed to the flood ($p=.055$), have less productive assets ($p=.003$), have lower net asset values ($p=.046$), and have fewer livestock ($p=.137$). This shows that the smaller livestock holders or less experienced individuals are encouraged to participate and continue to operate in the \textsf{cattle} arm that has a managerial support program, with all other features being equal. This is consistent with our analysis of participation in \textsc{\normalsize Table \ref{TabLabel1[grep("bo.*non-ca", TabLabel1)]}} which weakly hints that the \textsf{cattle} arm's managerial support programs may have encouraged participation of inexperienced or lower asset holders. This also underscores our interpretation that the current impact estimates may be downward biased, if any, as people who would otherwise attrit or reject in cattle arm stayed on. This result is confirmed with lower $p$ values due to a larger sample size when we compare the nonattriting borrowers between \textsf{cattle} arm with other arms in \textsc{\normalsize Table \ref{gsub("surviving members", "non-attriting borrowers", TabLabel1[grep("surviving members of cattle and all", TabLabel1)])}}. At the baseline, \textsf{cattle} arm nonattriting borrowers have smaller baseline livestock holding ($p$ value = .016) and smaller baseline net asset holding ($p$ value = .007) than other arms' nonattriting borrowers. 


	\textsc{\normalsize Table \ref{tab1 Permutation test results of attrition}} shows results from tests of independence between attriters and nonattriters. Attrition is defined as attrition from household surveys, not from the lending program. We see the moderate rate of attrition is not correlated with household level characteristics%, except less risk torelance in \textsf{RiskPrefVal}, 
	at the conventional $p$ value level. Productive asset amounts seem to differ between attriters and nonattriters at $p=.105$, with the former being larger than the latter. This positive attrition selection can cause underestimation of impacts, if the asset values are positively correlated with entrepreneurial capacity. We also see that the attriters are less risk tolerant in terms of minimum expected payoff to choose a risky option in \textsf{RiskPrefVal}. \textsc{\normalsize Table \ref{tab1 Permutation test results of attrition among traditional arm}} shows attrition in the \textsf{traditional} arm. Household heads of attriters are relatively less literate than nonattriters. We observe the \textsf{traditional} arm attriters are less risk tolerant the nonattriters.
	\textsc{\normalsize Table \ref{tab1 Permutation test results of attrition among non-traditional arm}} compares attriters and nonattriters in the non-\textsf{traditional} arm. Unlike \textsf{traditional} arm attriters, non-\textsf{traditional} arm attriters have more literate household heads, have a larger household size, are more exposed to floods, and have larger productive assets. The \textsf{traditional} arm attriters may be less entrepreneurial, if anything, so their attrition may upwardly bias the positive gains of the arm, hence understate the impacts of non-\textsf{traditional} arm. These are explicitly shown in \textsc{\normalsize Table \ref{tab1 Permutation test results of attriters of traditional and non-traditional arms}} where we compare attriters of \textsf{traditional} and non-\textsf{traditional} arms. Overall, attrition may have attenuated the impacts but is not likely to have inflated them.\footnote{So one can employ the Lee bounds for stronger results, but doing so will give us less precision and require more assumptions. We will not use the Lee bounds \textcolor{red}{[we can show them if necessary]}. } We observe the non-\textsf{traditional} arm attriters are also less risk tolerant than the nonattriters.

	For the microfinance institutions (MFIs), attrition of the loan receiving members poses a threat to their business continuation. Financial institutions often use observable characteristics, such as collateralisable assets, and easily surveyed chracteristics, such as job experiences and schooling of borrowers, and are likely to lend if the assets levels are greater and the borrowers have relevant job experiences and more schooling. We first examine if such screening variables have any predictive power in terms of loan rejection or borrower attrition under our lending. \textsc{\normalsize Table \ref{tab1 Permutation test results of active status}} compares potential MFI targets (nonattriting borrowers, noted as \textsf{Active}) vs. non-targets (attriting borrowers or loan rejecters, noted as \textsf{NonActive}) in all arms. It shows potential targets at the baseline have larger values in livestock and greater number of cattle, and are less affected by the flood, which conforms the conventional wisdom of lenders in using these aspects in their loan decisions. We also see that more risk torelant members are likely to be borrowers and do not attrit. Next, we examine if the relationship of having ``less favourable'' values in these characteristics and attrition is mitigated under various loan characteristics. In \textsc{\normalsize Table \ref{tab1 Permutation test results of active members of cattle and large grace arms}}, we restrict our attention to the potential MFI targets, or the nonattriting borrowers, and compare between \textsf{cattle} and \textsf{large grace} arms, whose difference is efffectively the presence of managerial supports that the former provides. \label{PageOfAttrition}%Comparing the nonattriting borrower, characteristics are similar except that the \textsf{traditional} members are more exposed to the flood than the non-\textsf{traditional} members. 
	Comparing against the \textsf{large grace} arm, nonattriting borrowers of the \textsf{cattle} arm are more exposed to the flood ($p=.055$), have less productive assets ($p=.003$), have lower net asset values ($p=.046$), and have fewer livestock ($p=.139$). This shows that the smaller livestock holders or individuals with less experienced in livestock are encouraged to participate and continue to operate in the \textsf{cattle} arm that has a managerial support program, with all other features being equal. This is consistent with our analysis of participation in \textsc{\normalsize Table \ref{tab1 Permutation test results of borrowers, cattle vs. non-cattle arms}} which weakly hints that the \textsf{cattle} arm's managerial support programs may have encouraged participation of inexperienced or lower asset holders. This also underscores our interpretation that the current impact estimates may be downwardly biased, if any, as people who would otherwise attrit or reject in the \textsf{cattle} arm stayed on. This result is confirmed with lower $p$ values due to a larger sample size when we compare the nonattriting borrowers between \textsf{cattle} arm with all other arms in \textsc{\normalsize Table \ref{tab1 Permutation test results of active members of cattle and all other arms}}. At the baseline, \textsf{cattle} arm nonattriting borrowers have smaller baseline livestock holding ($p$ value = .016) and smaller baseline net asset holding ($p$ value = .007) than other arms' nonattriting borrowers. 


%	Moved from main text.
	
%	Group rejecters of \textsf{traditional} and non-\textsf{traditional} arms differ in household characteristics. Lower livestock values, smaller cattle holding, and smaller net asset values are associated with group rejection for \textsf{traditional} arm (\textsc{\normalsize Table \ref{TabLabel1[grep("p rej.* traditional arm$", TabLabel1)]}}), while higher baseline flood exposure rates and younger household heads are associated with group rejection for non-\textsf{traditional} arms (\textsc{\normalsize Table \ref{TabLabel1[grep("p rej.*g non-traditional arm$", TabLabel1)]}}). Given randomisation, we conjecture that it is lack of \textsf{Upfront} liquidity that prevented smaller livestock holders of \textsf{traditional} arm from participating because they cannot purchase cattle due to insufficient net asset values or an insufficient resale value of owned livestock, when members of similar characteristics partcipated in non-\textsf{traditional} arms. %This is a real resource constraint that binds the households. This is different from a psychological constraint that, so long as there is a cost or a payment involved, albeit at a minimal level, there remains a group of households who would not take up the investment \citep{Ashraf2010, CohenDupas2010}. 
%	For non-\textsf{traditional} arm rejecters, it is the past flood that kept members from participating, even they are younger and have similar cattle holding as the non group-rejecters. 

%	Individual rejecters of \textsf{traditional} arm and non-\textsf{traditional} arms share similar characteristics (\textsc{\normalsize Table \ref{TabLabel1[grep("l rej.*vs", TabLabel1)]}}). In fact, they are not very different in all the variables considered. %This is consistent with the conjecture that, had the \textsf{traditional} arm group rejecters been offered any of the non-\textsf{traditional} arms, they, as a group, may have accepted it.  %It shows the latter is more exposed to flood in baseline and has larger livestock values. This implies that, once large enough sum of loan is disbursed, %there is no minimum livestock and asset holding level to partake in the larger loans, and 
	%despite a negative asset shock in flood and a poverty trap at this level may be overcome once household size and negative asset shocks are accounted for.
%	The common factors associated with nonparticipation are a smaller household size and smaller livestock holding (\textsc{\normalsize Table \ref{TabLabel1[grep("l rej.*g tra", TabLabel1)]}} and \textsc{\normalsize Table \ref{TabLabel1[grep("l rej.*g non-", TabLabel1)]}}), although the $p$ values for livestock holding difference between individual rejecters and non individual rejecters are around 7\% (\textsc{\normalsize Table \ref{TabLabel1[grep("l rejection$", TabLabel1)]}}). %In non-\textsf{traditional} arms, the individual rejecters have only marginally different mean values relative to individual nonrejecters (\textsc{\normalsize Table \ref{tab Ireject nontrad perm}}). 



\section{Impact estimation results}
\label{AppendixEstimationTables}
\setcounter{table}{0}

	In this section, the ANCOVA estimates on various outcomes using %\eqref{1steq} and 
	\eqref{EstimatingEqTimeVarying} are presented. In each table, the first column shows the covariate names and their means and standard deviations in the second column in the sample of the richest specification of the table. Specification (1) is OLS estimates on the intercept, $\bfdee_{i}$, and its period interactions. This is intended to provide a reference to ANCOVA estimates shown in the specification (2) onwards. Specification (2) follows the most basic specification under \eqref{EstimatingEqTimeVarying}. From (3), we progressively add more covariates to control for the differences in initial conditions in an attempt to get more precise ANCOVA estimates. In the figures (\textsc{\footnotesize Figure \ref{fig NarrowNetAssetsLivestockEffects}}, \textsc{\small Figure \ref{fig IncomeConsumptionEffects}}, \textsc{\footnotesize Figure \ref{fig SchoolingEffects}}) shown in main texts, we omit OLS estimates of specification (1). 
	
	We annotate the number of periods that a household is observed with \textsf{T}. The total number of households is shown for each values of \textsf{T}. \textsf{T=4} indicates the number of households with complete panel information, \textsf{T=3} indicates number of households observed three times, \textsf{T=2} indicates the number of households observed twice. \textsf{N} indicates total number of observations used in ANCOVA estimation, or \textsf{N$=$1$\times$(T=2)+2$\times$(T=3)+3$\times$(T=4)}. 
	
	For \textsc{\footnotesize Figure \ref{fig NarrowNetAssetsLivestockEffects}}, \textsc{\small Figure \ref{fig IncomeConsumptionEffects}}, we show cumulative impacts of the arm or functional attribute $k$ relative to the \textsf{traditional} arm as given by $b_{2k}$, $b_{2k}+b_{3k}$,  $b_{2k}+b_{4k}$ for periods 2, 3, and 4. In \textsc{\footnotesize Figure \ref{fig SchoolingEffects}}, we show contemporaneous impacts relative to the \textsf{traditional} arm as given by $b_{2k}$, $b_{3k}$,  $b_{4k}$ for periods 2, 3, and 4.

% Shortfall regression is done in ShortfallRegressionAtIndivLevelAndGroupLevel.rnw

% This file copies latex tables that are chosen by StrTablesToDisplay (see <<print est results sample, results = "asis">>= section ) in estimation_memo.rnw and ShortfallRegressionAtIndivLevelAndGroupLevel.rnw and print it in the draft tex file.


\hspace{-1cm}\begin{minipage}[t]{14cm} \hfil\textsc{\normalsize Table \refstepcounter{table}\thetable: ANCOVA estimation of net assets by period\label{tab ANCOVA narrow net assets timevarying}}\\ \setlength{\tabcolsep}{1pt}
  \setlength{\baselineskip}{8pt}
  \renewcommand{\arraystretch}{.55}
  \hfil\begin{tikzpicture}
  \node (tbl) {\input{ c:/data/GUK/analysis/save/EstimationMemo/NarrowNetAssetsTimeVaryingANCOVAEstimationResults.tex }};
%\input{c:/dropbox/data/ramadan/save/tablecolortemplate.tex}
\end{tikzpicture}\\
\renewcommand{\arraystretch}{.8}
\setlength{\tabcolsep}{1pt} \begin{tabular}{>{\hfill\scriptsize}p{1cm}<{}>{\hfill\scriptsize}p{.25cm}<{}>{\scriptsize}p{12cm}<{\hfill}} 
Source:& \multicolumn{2}{l}{\scriptsize Estimated with GUK administrative and survey data.}\\
Notes: & 1. & ANCOVA estimates using administrative and survey data. Post treatment regressands are regressed on categorical variables, pre-treatment regressand and other covariates. \textsf{FloodInRd1} and \textsf{HeadLiterate0} are indicator variables for the presence of self reported damage by a flood at the baseline, and literacy of household head, respectively. \textsf{HHsize0} is household size at the baseline. We annotate the number of periods that a household is observed with \textsf{T}. The total number of households is shown for each values of \textsf{T}. \textsf{T=4} indicates the number of households with complete panel information, \textsf{T=3} indicates number of households observed three times, \textsf{T=2} indicates the number of households observed twice. \textsf{N} indicates total number of observations used in ANCOVA estimation, or \textsf{N$=$1$\times$(T=2)+2$\times$(T=3)+3$\times$(T=4)}.  \textsf{Large}, \textsf{LargeGrace}, \textsf{Cattle} are indicator variables of the \textsf{large}, \textsf{large grace}, and \textsf{cattle} arms, respectively. The default arm category is \textsf{traditional} arm. Narrow net assets = Narrow assets + net saving - debt to GUK - debts to relatives and money lenders. Narrow assets use only items observed for all 4 rounds for household assets. Household assets do not include livestock. Regressions (1)-(3), (5)-(6) use only arm and calendar information. (4) and (7) use previous six month repayment and saving information which is lacking in rd 1, hence starts from rd 2.\\
& 2. &  $P$ values in percentages in parenthesises. Standard errors are clustered at group (village) level. %${}^{***}$, ${}^{**}$, ${}^{*}$ indicate statistical significance at 1\%, 5\%, 10\%, respetively. Standard errors are clustered at group (village) level.
 \end{tabular}
\end{minipage} \\\\\hspace{-1cm}\begin{minipage}[t]{14cm} \hfil\textsc{\normalsize Table \refstepcounter{table}\thetable: ANCOVA estimation of net assets by attributes and period\label{tab ANCOVA narrow net assets timevarying attributes}}\\ \setlength{\tabcolsep}{1pt}
  \setlength{\baselineskip}{8pt}
  \renewcommand{\arraystretch}{.55}
  \hfil\begin{tikzpicture}
  \node (tbl) {\input{ c:/data/GUK/analysis/save/EstimationMemo/NarrowNetAssetsTimeVaryingAttributesANCOVAEstimationResults.tex }};
%\input{c:/dropbox/data/ramadan/save/tablecolortemplate.tex}
\end{tikzpicture}\\
\renewcommand{\arraystretch}{.8}
\setlength{\tabcolsep}{1pt} \begin{tabular}{>{\hfill\scriptsize}p{1cm}<{}>{\hfill\scriptsize}p{.25cm}<{}>{\scriptsize}p{12cm}<{\hfill}} Notes : & \multicolumn{2}{l}{\scriptsize See footnotes of \textsc{Table D1}.} \end{tabular}
\end{minipage} \\\\\hspace{-1cm}\begin{minipage}[t]{14cm} \hfil\textsc{\normalsize Table \refstepcounter{table}\thetable: ANCOVA estimation of broad net assets by period\label{tab ANCOVA net assets timevarying}}\\ \setlength{\tabcolsep}{1pt}
  \setlength{\baselineskip}{8pt}
  \renewcommand{\arraystretch}{.55}
  \hfil\begin{tikzpicture}
  \node (tbl) {\input{ c:/data/GUK/analysis/save/EstimationMemo/NetAssetsTimeVaryingANCOVAEstimationResults.tex }};
%\input{c:/dropbox/data/ramadan/save/tablecolortemplate.tex}
\end{tikzpicture}\\
\renewcommand{\arraystretch}{.8}
\setlength{\tabcolsep}{1pt} \begin{tabular}{>{\hfill\scriptsize}p{1cm}<{}>{\hfill\scriptsize}p{.25cm}<{}>{\scriptsize}p{12cm}<{\hfill}} 
Source:& \multicolumn{2}{l}{\scriptsize Estimated with GUK administrative and survey data.}\\
Notes: & 1. & ANCOVA estimates using administrative and survey data. Post treatment regressands are regressed on categorical variables, pre-treatment regressand and other covariates. \textsf{FloodInRd1} and \textsf{HeadLiterate0} are indicator variables for the presence of self reported damage by a flood at the baseline, and literacy of household head, respectively. \textsf{HHsize0} is household size at the baseline. We annotate the number of periods that a household is observed with \textsf{T}. The total number of households is shown for each values of \textsf{T}. \textsf{T=4} indicates the number of households with complete panel information, \textsf{T=3} indicates number of households observed three times, \textsf{T=2} indicates the number of households observed twice. \textsf{N} indicates total number of observations used in ANCOVA estimation, or \textsf{N$=$1$\times$(T=2)+2$\times$(T=3)+3$\times$(T=4)}.  \textsf{Large}, \textsf{LargeGrace}, \textsf{Cattle} are indicator variables of the \textsf{large}, \textsf{large grace}, and \textsf{cattle} arms, respectively. The default arm category is \textsf{traditional} arm. Household assets do not include livestock. Regressions (1)-(3), (5)-(6) use only arm and calendar information. (4) and (7) use previous six month repayment and saving information which is lacking in rd 1, hence starts from rd 2.\\
& 2. &  $P$ values in percentages in parenthesises. Standard errors are clustered at group (village) level. %${}^{***}$, ${}^{**}$, ${}^{*}$ indicate statistical significance at 1\%, 5\%, 10\%, respetively. Standard errors are clustered at group (village) level.
 \end{tabular}
\end{minipage} \\\\\hspace{-1cm}\begin{minipage}[t]{14cm} \hfil\textsc{\normalsize Table \refstepcounter{table}\thetable: ANCOVA estimation of broad net assets by attributes and period\label{tab ANCOVA net assets timevarying attributes}}\\ \setlength{\tabcolsep}{1pt}
  \setlength{\baselineskip}{8pt}
  \renewcommand{\arraystretch}{.55}
  \hfil\begin{tikzpicture}
  \node (tbl) {\input{ c:/data/GUK/analysis/save/EstimationMemo/NetAssetsTimeVaryingAttributesANCOVAEstimationResults.tex }};
%\input{c:/dropbox/data/ramadan/save/tablecolortemplate.tex}
\end{tikzpicture}\\
\renewcommand{\arraystretch}{.8}
\setlength{\tabcolsep}{1pt} \begin{tabular}{>{\hfill\scriptsize}p{1cm}<{}>{\hfill\scriptsize}p{.25cm}<{}>{\scriptsize}p{12cm}<{\hfill}} Notes : & \multicolumn{2}{l}{\scriptsize See footnotes of \textsc{Table D3}.} \end{tabular}
\end{minipage} \\\\\hspace{-1cm}\begin{minipage}[t]{14cm} \hfil\textsc{\normalsize Table \refstepcounter{table}\thetable: ANCOVA estimation of net assets by period, cattle rearing experiences\label{tab ANCOVA narrow net assets Experience timevarying 1}}\\ \setlength{\tabcolsep}{1pt}
  \setlength{\baselineskip}{8pt}
  \renewcommand{\arraystretch}{.55}
  \hfil\begin{tikzpicture}
  \node (tbl) {\input{ c:/data/GUK/analysis/save/EstimationMemo/NarrowNetAssetsByExperience1TimeVaryingANCOVAEstimationResults.tex }};
%\input{c:/dropbox/data/ramadan/save/tablecolortemplate.tex}
\end{tikzpicture}\\
\renewcommand{\arraystretch}{.8}
\setlength{\tabcolsep}{1pt} \begin{tabular}{>{\hfill\scriptsize}p{1cm}<{}>{\hfill\scriptsize}p{.25cm}<{}>{\scriptsize}p{12cm}<{\hfill}} 
Source:& \multicolumn{2}{l}{\scriptsize Estimated with GUK administrative and survey data.}\\
Notes: & 1. & ANCOVA estimates using administrative and survey data. Post treatment regressands are regressed on categorical variables, pre-treatment regressand and other covariates. \textsf{FloodInRd1} and \textsf{HeadLiterate0} are indicator variables for the presence of self reported damage by a flood at the baseline, and literacy of household head, respectively. \textsf{HHsize0} is household size at the baseline. We annotate the number of periods that a household is observed with \textsf{T}. The total number of households is shown for each values of \textsf{T}. \textsf{T=4} indicates the number of households with complete panel information, \textsf{T=3} indicates number of households observed three times, \textsf{T=2} indicates the number of households observed twice. \textsf{N} indicates total number of observations used in ANCOVA estimation, or \textsf{N$=$1$\times$(T=2)+2$\times$(T=3)+3$\times$(T=4)}.  \textsf{Large}, \textsf{LargeGrace}, \textsf{Cattle} are indicator variables of the \textsf{large}, \textsf{large grace}, and \textsf{cattle} arms, respectively. The default arm category is \textsf{traditional} arm. Narrow net assets = Narrow assets + net saving - debt to GUK - debts to relatives and money lenders. Narrow assets use only items observed for all 4 rounds for household assets. Household assets do not include livestock. Regressions (1)-(3), (5)-(6) use only arm and calendar information. (4) and (7) use previous six month repayment and saving information which is lacking in rd 1, hence starts from rd 2.\\
& 2. &  $P$ values in percentages in parenthesises. Standard errors are clustered at group (village) level. %${}^{***}$, ${}^{**}$, ${}^{*}$ indicate statistical significance at 1\%, 5\%, 10\%, respetively. Standard errors are clustered at group (village) level.
 \end{tabular}
\end{minipage} \\\\\addtocounter{table}{-1}\hspace{-1cm}\begin{minipage}[t]{14cm} \hfil\textsc{\normalsize Table \refstepcounter{table}\thetable: ANCOVA estimation of net assets by period, cattle rearing experiences (continued) \label{tab ANCOVA narrow net assets Experience timevarying 2}}\\ \setlength{\tabcolsep}{1pt}
  \setlength{\baselineskip}{8pt}
  \renewcommand{\arraystretch}{.55}
  \hfil\begin{tikzpicture}
  \node (tbl) {\input{ c:/data/GUK/analysis/save/EstimationMemo/NarrowNetAssetsByExperience2TimeVaryingANCOVAEstimationResults.tex }};
%\input{c:/dropbox/data/ramadan/save/tablecolortemplate.tex}
\end{tikzpicture}\\
\renewcommand{\arraystretch}{.8}
\setlength{\tabcolsep}{1pt} \begin{tabular}{>{\hfill\scriptsize}p{1cm}<{}>{\hfill\scriptsize}p{.25cm}<{}>{\scriptsize}p{12cm}<{\hfill}} Notes : & \multicolumn{2}{l}{\scriptsize See footnotes of \textsc{Table D5}.} \end{tabular}
\end{minipage} \\\\\hspace{-1cm}\begin{minipage}[t]{14cm} \hfil\textsc{\normalsize Table \refstepcounter{table}\thetable: ANCOVA estimation of narrow net assets by attributes and period, cattle rearing experiences\label{tab ANCOVA narrow net assets Experience timevarying attributes 1}}\\ \setlength{\tabcolsep}{1pt}
  \setlength{\baselineskip}{8pt}
  \renewcommand{\arraystretch}{.55}
  \hfil\begin{tikzpicture}
  \node (tbl) {\input{ c:/data/GUK/analysis/save/EstimationMemo/NarrowNetAssetsByExperience1TimeVaryingAttributesANCOVAEstimationResults.tex }};
%\input{c:/dropbox/data/ramadan/save/tablecolortemplate.tex}
\end{tikzpicture}\\
\renewcommand{\arraystretch}{.8}
\setlength{\tabcolsep}{1pt} \begin{tabular}{>{\hfill\scriptsize}p{1cm}<{}>{\hfill\scriptsize}p{.25cm}<{}>{\scriptsize}p{12cm}<{\hfill}} Notes : & \multicolumn{2}{l}{\scriptsize See footnotes of \textsc{Table D6}.} \end{tabular}
\end{minipage} \\\\\addtocounter{table}{-1}\hspace{-1cm}\begin{minipage}[t]{14cm} \hfil\textsc{\normalsize Table \refstepcounter{table}\thetable: ANCOVA estimation of net assets by attributes and period, cattle rearing experiences (continued)\label{tab ANCOVA narrow net assets Experience timevarying attributes 2}}\\ \setlength{\tabcolsep}{1pt}
  \setlength{\baselineskip}{8pt}
  \renewcommand{\arraystretch}{.55}
  \hfil\begin{tikzpicture}
  \node (tbl) {\input{ c:/data/GUK/analysis/save/EstimationMemo/NarrowNetAssetsByExperience2TimeVaryingAttributesANCOVAEstimationResults.tex }};
%\input{c:/dropbox/data/ramadan/save/tablecolortemplate.tex}
\end{tikzpicture}\\
\renewcommand{\arraystretch}{.8}
\setlength{\tabcolsep}{1pt} \begin{tabular}{>{\hfill\scriptsize}p{1cm}<{}>{\hfill\scriptsize}p{.25cm}<{}>{\scriptsize}p{12cm}<{\hfill}} Notes : & \multicolumn{2}{l}{\scriptsize See footnotes of \textsc{Table D7}.} \end{tabular}
\end{minipage} \\\\\hspace{-1cm}\begin{minipage}[t]{14cm} \hfil\textsc{\normalsize Table \refstepcounter{table}\thetable: ANCOVA estimation of net non-livestock assets by attributes and period\label{tab ANCOVA NarrowNetNLAssets timevarying attributes}}\\ \setlength{\tabcolsep}{1pt}
  \setlength{\baselineskip}{8pt}
  \renewcommand{\arraystretch}{.55}
  \hfil\begin{tikzpicture}
  \node (tbl) {\input{ c:/data/GUK/analysis/save/EstimationMemo/NarrowNetNLAssetsTimeVaryingAttributesANCOVAEstimationResults.tex }};
%\input{c:/dropbox/data/ramadan/save/tablecolortemplate.tex}
\end{tikzpicture}\\
\renewcommand{\arraystretch}{.8}
\setlength{\tabcolsep}{1pt} \begin{tabular}{>{\hfill\scriptsize}p{1cm}<{}>{\hfill\scriptsize}p{.25cm}<{}>{\scriptsize}p{12cm}<{\hfill}} 
Source:& \multicolumn{2}{l}{\scriptsize Estimated with GUK administrative and survey data.}\\
Notes: & 1. & ANCOVA estimates using administrative and survey data. Post treatment regressands are regressed on categorical variables, pre-treatment regressand and other covariates. \textsf{FloodInRd1} and \textsf{HeadLiterate0} are indicator variables for the presence of self reported damage by a flood at the baseline, and literacy of household head, respectively. \textsf{HHsize0} is household size at the baseline. We annotate the number of periods that a household is observed with \textsf{T}. The total number of households is shown for each values of \textsf{T}. \textsf{T=4} indicates the number of households with complete panel information, \textsf{T=3} indicates number of households observed three times, \textsf{T=2} indicates the number of households observed twice. \textsf{N} indicates total number of observations used in ANCOVA estimation, or \textsf{N$=$1$\times$(T=2)+2$\times$(T=3)+3$\times$(T=4)}.  \textsf{Upfront} is an indicator variable of the arm with an upfront large disbursement, \textsf{WithGrace} is an indicator variable of the arm with a grace period, \textsf{InKind} is an indicator variable of the arm which lends a heifer. Narrow net assets = Narrow assets + net saving - debt to GUK - debts to relatives and money lenders. Narrow assets use only items observed for all 4 rounds for household assets. Household assets do not include livestock. Regressions (1)-(3), (5)-(6) use only arm and calendar information. (4) and (7) use previous six month repayment and saving information which is lacking in rd 1, hence starts from rd 2.\\
& 2. &  $P$ values in percentages in parenthesises. Standard errors are clustered at group (village) level. %${}^{***}$, ${}^{**}$, ${}^{*}$ indicate statistical significance at 1\%, 5\%, 10\%, respetively. Standard errors are clustered at group (village) level.
 \end{tabular}
\end{minipage} \\\\\hspace{-1cm}\begin{minipage}[t]{14cm} \hfil\textsc{\normalsize Table \refstepcounter{table}\thetable: ANCOVA estimation of cattle holding by arm and period\label{tab ANCOVA cow time varying}}\\ \setlength{\tabcolsep}{1pt}
  \setlength{\baselineskip}{8pt}
  \renewcommand{\arraystretch}{.55}
  \hfil\begin{tikzpicture}
  \node (tbl) {\input{ c:/data/GUK/analysis/save/EstimationMemo/NumCowsTimeVaryingANCOVAEstimationResults.tex }};
%\input{c:/dropbox/data/ramadan/save/tablecolortemplate.tex}
\end{tikzpicture}\\
\renewcommand{\arraystretch}{.8}
\setlength{\tabcolsep}{1pt} \begin{tabular}{>{\hfill\scriptsize}p{1cm}<{}>{\hfill\scriptsize}p{.25cm}<{}>{\scriptsize}p{12cm}<{\hfill}} 
Source:& \multicolumn{2}{l}{\scriptsize Estimated with GUK administrative and survey data.}\\
Notes: & 1. & ANCOVA estimates using administrative and survey data. Post treatment regressands are regressed on categorical variables, pre-treatment regressand and other covariates. \textsf{FloodInRd1} and \textsf{HeadLiterate0} are indicator variables for the presence of self reported damage by a flood at the baseline, and literacy of household head, respectively. \textsf{HHsize0} is household size at the baseline. We annotate the number of periods that a household is observed with \textsf{T}. The total number of households is shown for each values of \textsf{T}. \textsf{T=4} indicates the number of households with complete panel information, \textsf{T=3} indicates number of households observed three times, \textsf{T=2} indicates the number of households observed twice. \textsf{N} indicates total number of observations used in ANCOVA estimation, or \textsf{N$=$1$\times$(T=2)+2$\times$(T=3)+3$\times$(T=4)}.  \textsf{Large}, \textsf{LargeGrace}, \textsf{Cattle} are indicator variables of the \textsf{large}, \textsf{large grace}, and \textsf{cattle} arms, respectively. The default arm category is \textsf{traditional} arm. \textsf{rd2, rd3, rd4} are dummy variables for second, third, and fourth round of survey. Sample is continuing members and replacing members of early rejecters and received loans prior to 2015 Janunary. Regressand is \textsf{NumCows}, number of cattle holding. \\
& 2. & $P$ values in percentages in parenthesises. Standard errors are clustered at group (village) level. %${}^{***}$, ${}^{**}$, ${}^{*}$ indicate statistical significance at 1\%, 5\%, 10\%, respetively. Standard errors are clustered at group (village) level.
 \end{tabular}
\end{minipage} \\\\\hspace{-1cm}\begin{minipage}[t]{14cm} \hfil\textsc{\normalsize Table \refstepcounter{table}\thetable: ANCOVA estimation of cattle holding by attributes and period\label{tab ANCOVA cow time varying attributes}}\\ \setlength{\tabcolsep}{1pt}
  \setlength{\baselineskip}{8pt}
  \renewcommand{\arraystretch}{.55}
  \hfil\begin{tikzpicture}
  \node (tbl) {\input{ c:/data/GUK/analysis/save/EstimationMemo/NumCowsTimeVaryingAttributesANCOVAEstimationResults.tex }};
%\input{c:/dropbox/data/ramadan/save/tablecolortemplate.tex}
\end{tikzpicture}\\
\renewcommand{\arraystretch}{.8}
\setlength{\tabcolsep}{1pt} \begin{tabular}{>{\hfill\scriptsize}p{1cm}<{}>{\hfill\scriptsize}p{.25cm}<{}>{\scriptsize}p{12cm}<{\hfill}} Notes : & \multicolumn{2}{l}{\scriptsize See footnotes of \textsc{Table D10}.} \end{tabular}
\end{minipage} \\\\\hspace{-1cm}\begin{minipage}[t]{14cm} \hfil\textsc{\normalsize Table \refstepcounter{table}\thetable: ANCOVA estimation of consumption by period\label{tab ANCOVA consumption timevarying}}\\ \setlength{\tabcolsep}{1pt}
  \setlength{\baselineskip}{8pt}
  \renewcommand{\arraystretch}{.55}
  \hfil\begin{tikzpicture}
  \node (tbl) {\input{ c:/data/GUK/analysis/save/EstimationMemo/ConsumptionTimeVaryingANCOVAEstimationResults.tex }};
%\input{c:/dropbox/data/ramadan/save/tablecolortemplate.tex}
\end{tikzpicture}\\
\renewcommand{\arraystretch}{.8}
\setlength{\tabcolsep}{1pt} \begin{tabular}{>{\hfill\scriptsize}p{1cm}<{}>{\hfill\scriptsize}p{.25cm}<{}>{\scriptsize}p{12cm}<{\hfill}} 
Source:& \multicolumn{2}{l}{\scriptsize Estimated with GUK administrative and survey data.}\\
Notes: & 1. & ANCOVA estimates using administrative and survey data. Post treatment regressands are regressed on categorical variables, pre-treatment regressand and other covariates. \textsf{FloodInRd1} and \textsf{HeadLiterate0} are indicator variables for the presence of self reported damage by a flood at the baseline, and literacy of household head, respectively. \textsf{HHsize0} is household size at the baseline. We annotate the number of periods that a household is observed with \textsf{T}. The total number of households is shown for each values of \textsf{T}. \textsf{T=4} indicates the number of households with complete panel information, \textsf{T=3} indicates number of households observed three times, \textsf{T=2} indicates the number of households observed twice. \textsf{N} indicates total number of observations used in ANCOVA estimation, or \textsf{N$=$1$\times$(T=2)+2$\times$(T=3)+3$\times$(T=4)}.  \textsf{UltraPoor} is an indicator variable if the household is classified as the ultra poor. \textsf{Large}, \textsf{LargeGrace}, \textsf{Cattle} are indicator variables of the \textsf{large}, \textsf{large grace}, and \textsf{cattle} arms, respectively. The default arm category is \textsf{traditional} arm. Consumption is annualised values. \\
& 2. & $P$ values in percentages in parenthesises. Standard errors are clustered at group (village) level. %${}^{***}$, ${}^{**}$, ${}^{*}$ indicate statistical significance at 1\%, 5\%, 10\%, respetively. Standard errors are clustered at group (village) level.
 \end{tabular}
\end{minipage} \\\\\hspace{-1cm}\begin{minipage}[t]{14cm} \hfil\textsc{\normalsize Table \refstepcounter{table}\thetable: ANCOVA estimation of consumption by attributes and period\label{tab ANCOVA consumption timevarying attributes original HH}}\\ \setlength{\tabcolsep}{1pt}
  \setlength{\baselineskip}{8pt}
  \renewcommand{\arraystretch}{.55}
  \hfil\begin{tikzpicture}
  \node (tbl) {\input{ c:/data/GUK/analysis/save/EstimationMemo/ConsumptionTimeVaryingAttributesANCOVAEstimationResults.tex }};
%\input{c:/dropbox/data/ramadan/save/tablecolortemplate.tex}
\end{tikzpicture}\\
\renewcommand{\arraystretch}{.8}
\setlength{\tabcolsep}{1pt} \begin{tabular}{>{\hfill\scriptsize}p{1cm}<{}>{\hfill\scriptsize}p{.25cm}<{}>{\scriptsize}p{12cm}<{\hfill}} Notes : & \multicolumn{2}{l}{\scriptsize See footnotes of \textsc{Table D12}.} \end{tabular}
\end{minipage} \\\\\hspace{-1cm}\begin{minipage}[t]{14cm} \hfil\textsc{\normalsize Table \refstepcounter{table}\thetable: ANCOVA estimation of household labour incomes and farm incomes by period\label{tab ANCOVA labour incomes timevarying}}\\ \setlength{\tabcolsep}{1pt}
  \setlength{\baselineskip}{8pt}
  \renewcommand{\arraystretch}{.55}
  \hfil\begin{tikzpicture}
  \node (tbl) {\input{ c:/data/GUK/analysis/save/EstimationMemo/LabourIncomeTimeVaryingANCOVAEstimationResults.tex }};
%\input{c:/dropbox/data/ramadan/save/tablecolortemplate.tex}
\end{tikzpicture}\\
\renewcommand{\arraystretch}{.8}
\setlength{\tabcolsep}{1pt} \begin{tabular}{>{\hfill\scriptsize}p{1cm}<{}>{\hfill\scriptsize}p{.25cm}<{}>{\scriptsize}p{12cm}<{\hfill}} 
Source:& \multicolumn{2}{l}{\scriptsize Estimated with GUK administrative and survey data.}\\
Notes: & 1. & ANCOVA estimates using administrative and survey data. Post treatment regressands are regressed on categorical variables, pre-treatment regressand and other covariates. \textsf{FloodInRd1} and \textsf{HeadLiterate0} are indicator variables for the presence of self reported damage by a flood at the baseline, and literacy of household head, respectively. \textsf{HHsize0} is household size at the baseline. We annotate the number of periods that a household is observed with \textsf{T}. The total number of households is shown for each values of \textsf{T}. \textsf{T=4} indicates the number of households with complete panel information, \textsf{T=3} indicates number of households observed three times, \textsf{T=2} indicates the number of households observed twice. \textsf{N} indicates total number of observations used in ANCOVA estimation, or \textsf{N$=$1$\times$(T=2)+2$\times$(T=3)+3$\times$(T=4)}.  \textsf{Large}, \textsf{LargeGrace}, \textsf{Cattle} are indicator variables of the \textsf{large}, \textsf{large grace}, and \textsf{cattle} arms, respectively. The default arm category is \textsf{traditional} arm. \textsf{rd2, rd3, rd4} are dummy variables for second, third, and fourth round of survey. Labour incomes are in 1000 Tk units and are a sum of all earned labour incomes of household members. Farm revenues are in 1000 Tk units and are a total of agricultural produce sales. \\
& 2. & $P$ values in percentages in parenthesises. Standard errors are clustered at group (village) level. %${}^{***}$, ${}^{**}$, ${}^{*}$ indicate statistical significance at 1\%, 5\%, 10\%, respetively. Standard errors are clustered at group (village) level.
 \end{tabular}
\end{minipage} \\\\\hspace{-1cm}\begin{minipage}[t]{14cm} \hfil\textsc{\normalsize Table \refstepcounter{table}\thetable: ANCOVA estimation of household labour incomes and farm incomes by attributes and period\label{tab ANCOVA labour incomes timevarying attributes}}\\ \setlength{\tabcolsep}{1pt}
  \setlength{\baselineskip}{8pt}
  \renewcommand{\arraystretch}{.55}
  \hfil\begin{tikzpicture}
  \node (tbl) {\input{ c:/data/GUK/analysis/save/EstimationMemo/LabourIncomeTimeVaryingAttributesANCOVAEstimationResults.tex }};
%\input{c:/dropbox/data/ramadan/save/tablecolortemplate.tex}
\end{tikzpicture}\\
\renewcommand{\arraystretch}{.8}
\setlength{\tabcolsep}{1pt} \begin{tabular}{>{\hfill\scriptsize}p{1cm}<{}>{\hfill\scriptsize}p{.25cm}<{}>{\scriptsize}p{12cm}<{\hfill}} Notes : & \multicolumn{2}{l}{\scriptsize See footnotes of \textsc{Table D14}.} \end{tabular}
\end{minipage} \\\\\hspace{-1cm}\begin{minipage}[t]{14cm} \hfil\textsc{\normalsize Table \refstepcounter{table}\thetable: ANCOVA estimation of school enrollment by time\label{tab ANCOVA enroll time varying1}}\\ \setlength{\tabcolsep}{1pt}
  \setlength{\baselineskip}{8pt}
  \renewcommand{\arraystretch}{.55}
  \hfil\begin{tikzpicture}
  \node (tbl) {\input{ c:/data/GUK/analysis/save/EstimationMemo/SchoolingTimeVaryingANCOVAEstimationResults_1.tex }};
%\input{c:/dropbox/data/ramadan/save/tablecolortemplate.tex}
\end{tikzpicture}\\
\renewcommand{\arraystretch}{.8}
\setlength{\tabcolsep}{1pt} \begin{tabular}{>{\hfill\scriptsize}p{1cm}<{}>{\hfill\scriptsize}p{.25cm}<{}>{\scriptsize}p{12cm}<{\hfill}} 
Source:& \multicolumn{2}{l}{\scriptsize Estimated with GUK administrative and survey data.}\\
Notes: & 1. & ANCOVA estimates using administrative and survey data. Post treatment regressands are regressed on categorical variables, pre-treatment regressand and other covariates. \textsf{FloodInRd1} and \textsf{HeadLiterate0} are indicator variables for the presence of self reported damage by a flood at the baseline, and literacy of household head, respectively. \textsf{HHsize0} is household size at the baseline. We annotate the number of periods that a household is observed with \textsf{T}. The total number of households is shown for each values of \textsf{T}. \textsf{T=4} indicates the number of households with complete panel information, \textsf{T=3} indicates number of households observed three times, \textsf{T=2} indicates the number of households observed twice. \textsf{N} indicates total number of observations used in ANCOVA estimation, or \textsf{N$=$1$\times$(T=2)+2$\times$(T=3)+3$\times$(T=4)}.  \textsf{Large}, \textsf{LargeGrace}, \textsf{Cattle} are indicator variables of the \textsf{large}, \textsf{large grace}, and \textsf{cattle} arms, respectively. The default arm category is \textsf{traditional} arm. \textsf{Secondary} and \textsf{College} are indicator variables of secondary schooling (ages 13-15) and tertiary schooling (ages 16-18), both at the time of baseline. Default category is primary (ages 05-12). \textsf{rd2, rd3, rd4} are dummy variables for second, third, and fourth round of survey. Interaction terms of dummy variables are demeaned before interacting. The first column gives mean and standard deviation (in parenthesises) of each covariates before demeaning. \\
& 2. & $P$ values in percentages in parenthesises. Standard errors are clustered at group (village) level. %${}^{***}$, ${}^{**}$, ${}^{*}$ indicate statistical significance at 1\%, 5\%, 10\%, respetively. Standard errors are clustered at group (village) level.
 \end{tabular}
\end{minipage} \\\\\addtocounter{table}{-1}\hspace{-1cm}\begin{minipage}[t]{14cm} \hfil\textsc{\normalsize Table \refstepcounter{table}\thetable: ANCOVA estimation of school enrollment by time (continued)\label{tab ANCOVA enroll time varying2}}\\ \setlength{\tabcolsep}{1pt}
  \setlength{\baselineskip}{8pt}
  \renewcommand{\arraystretch}{.55}
  \hfil\begin{tikzpicture}
  \node (tbl) {\input{ c:/data/GUK/analysis/save/EstimationMemo/SchoolingTimeVaryingANCOVAEstimationResults_2.tex }};
%\input{c:/dropbox/data/ramadan/save/tablecolortemplate.tex}
\end{tikzpicture}\\
\renewcommand{\arraystretch}{.8}
\setlength{\tabcolsep}{1pt} \begin{tabular}{>{\hfill\scriptsize}p{1cm}<{}>{\hfill\scriptsize}p{.25cm}<{}>{\scriptsize}p{12cm}<{\hfill}} Notes : & \multicolumn{2}{l}{\scriptsize See footnotes of \textsc{Table D16}.} \end{tabular}
\end{minipage} \\\\\addtocounter{table}{-1}\hspace{-1cm}\begin{minipage}[t]{14cm} \hfil\textsc{\normalsize Table \refstepcounter{table}\thetable: ANCOVA estimation of school enrollment by time (continued 2)\label{tab ANCOVA enroll time varying3}}\\ \setlength{\tabcolsep}{1pt}
  \setlength{\baselineskip}{8pt}
  \renewcommand{\arraystretch}{.55}
  \hfil\begin{tikzpicture}
  \node (tbl) {\input{ c:/data/GUK/analysis/save/EstimationMemo/SchoolingTimeVaryingAttributesANCOVAEstimationResults_3.tex }};
%\input{c:/dropbox/data/ramadan/save/tablecolortemplate.tex}
\end{tikzpicture}\\
\renewcommand{\arraystretch}{.8}
\setlength{\tabcolsep}{1pt} \begin{tabular}{>{\hfill\scriptsize}p{1cm}<{}>{\hfill\scriptsize}p{.25cm}<{}>{\scriptsize}p{12cm}<{\hfill}} Notes: &\multicolumn{2}{l}{\scriptsize See footnotes of \textsc{Table \ref{tab ANCOVA enroll time varying1}}}.  \end{tabular}
\end{minipage} \\\\\hspace{-1cm}\begin{minipage}[t]{14cm} \hfil\textsc{\normalsize Table \refstepcounter{table}\thetable: ANCOVA estimation of school enrollment by attributes and time\label{tab ANCOVA enroll time varying attributes}}\\ \setlength{\tabcolsep}{1pt}
  \setlength{\baselineskip}{8pt}
  \renewcommand{\arraystretch}{.55}
  \hfil\begin{tikzpicture}
  \node (tbl) {\input{ c:/data/GUK/analysis/save/EstimationMemo/SchoolingTimeVaryingAttributesANCOVAEstimationResults_1.tex }};
%\input{c:/dropbox/data/ramadan/save/tablecolortemplate.tex}
\end{tikzpicture}\\
\renewcommand{\arraystretch}{.8}
\setlength{\tabcolsep}{1pt} \begin{tabular}{>{\hfill\scriptsize}p{1cm}<{}>{\hfill\scriptsize}p{.25cm}<{}>{\scriptsize}p{12cm}<{\hfill}} Notes : & \multicolumn{2}{l}{\scriptsize See footnotes of \textsc{Table D18}.} \end{tabular}
\end{minipage} \\\\\addtocounter{table}{-1}\hspace{-1cm}\begin{minipage}[t]{14cm} \hfil\textsc{\normalsize Table \refstepcounter{table}\thetable: ANCOVA estimation of school enrollment by attributes and time (continued)\label{tab ANCOVA enroll time varying attributes2}}\\ \setlength{\tabcolsep}{1pt}
  \setlength{\baselineskip}{8pt}
  \renewcommand{\arraystretch}{.55}
  \hfil\begin{tikzpicture}
  \node (tbl) {\input{ c:/data/GUK/analysis/save/EstimationMemo/SchoolingTimeVaryingAttributesANCOVAEstimationResults_2.tex }};
%\input{c:/dropbox/data/ramadan/save/tablecolortemplate.tex}
\end{tikzpicture}\\
\renewcommand{\arraystretch}{.8}
\setlength{\tabcolsep}{1pt} \begin{tabular}{>{\hfill\scriptsize}p{1cm}<{}>{\hfill\scriptsize}p{.25cm}<{}>{\scriptsize}p{12cm}<{\hfill}} Notes : & \multicolumn{2}{l}{\scriptsize See footnotes of \textsc{Table D19}.} \end{tabular}
\end{minipage} \\\\\addtocounter{table}{-1}\hspace{-1cm}\begin{minipage}[t]{14cm} \hfil\textsc{\normalsize Table \refstepcounter{table}\thetable: ANCOVA estimation of school enrollment by attributes and time (continued 2)\label{tab ANCOVA enroll time varying attributes3}}\\ \setlength{\tabcolsep}{1pt}
  \setlength{\baselineskip}{8pt}
  \renewcommand{\arraystretch}{.55}
  \hfil\begin{tikzpicture}
  \node (tbl) {\input{ c:/data/GUK/analysis/save/EstimationMemo/SchoolingTimeVaryingAttributesANCOVAEstimationResults_3.tex }};
%\input{c:/dropbox/data/ramadan/save/tablecolortemplate.tex}
\end{tikzpicture}\\
\renewcommand{\arraystretch}{.8}
\setlength{\tabcolsep}{1pt} \begin{tabular}{>{\hfill\scriptsize}p{1cm}<{}>{\hfill\scriptsize}p{.25cm}<{}>{\scriptsize}p{12cm}<{\hfill}} Notes : & \multicolumn{2}{l}{\scriptsize See footnotes of \textsc{Table D20}.} \end{tabular}
\end{minipage} \\\\\section{Correlates of repayment shortfall}\label{Sec App Repay}\hspace{-1cm}\begin{minipage}[t]{14cm} \hfil\textsc{\normalsize Table \refstepcounter{table}\thetable: Individual level effects of repayment shortfall\label{tab shortfall indiv o800}}\\ \setlength{\tabcolsep}{1pt}
  \setlength{\baselineskip}{8pt}
  \renewcommand{\arraystretch}{.55}
  \hfil\begin{tikzpicture}
  \node (tbl) {\input{ c:/data/GUK/analysis/save/EstimationMemo/Shortfallo800EstimationResults1.tex }};
%\input{c:/dropbox/data/ramadan/save/tablecolortemplate.tex}
\end{tikzpicture}\\
\renewcommand{\arraystretch}{.8}
\setlength{\tabcolsep}{1pt} \begin{tabular}{>{\hfill\scriptsize}p{1cm}<{}>{\hfill\scriptsize}p{.25cm}<{}>{\scriptsize}p{12cm}<{\hfill}} 
Source:& \multicolumn{2}{l}{\scriptsize Estimated with GUK administrative data.}\\
Notes: & 1. & Estimates of repayment shortfall controlling for group/village and year-month fixed effects using 48 month administrative records. The estimated model is $\tilde{y}_{it}=b_{1}+\bfb'_{1}\bfdee_{i}+b_{2}\mbox{\textsf{LY2}}+\bfb'_{2}\bfdee_{i}\mbox{\textsf{LY2}}+b_{3}\mbox{\textsf{LY3}}+\bfb'_{3}\bfdee_{i}\mbox{\textsf{LY3}}+b_{4}\mbox{\textsf{LY4}}+\bfb'_{4}\bfdee_{i}\mbox{\textsf{LY4}}+\tilde{e}_{it}$, where $\tilde{x}_{it}$ is group and time demeaned value of variable $x$, $t=1,\dots, 48$ is an ellapsed month index, $\bfdee_{i}$ is a three element vector of arms or functional attributes, $\mbox{\textsf{LY2}}, \mbox{\textsf{LY3}}, \mbox{\textsf{LY4}}$ are indicator variables of loan years 2, 3, 4. Loan years are defined with the ellapsed months since the first disbursement date, 13-24 for \textsf{LY2}, 25-36 for \textsf{LY3}, and 37-48 for \textsf{LY4}. Fixed effects are controlled by differencing out respecive means from the data matrix. Shortfall $y_{it}$ is (planned installment) - (actual repayment). \textsf{Group shortfall}$_{t-1}$ indicates a one month lagged mean shortfall amount of a group. \textsf{Per member group net saving}$_{t-1}$ and \textsf{Per member cumulative group net saving (BDT1000)}$_{t-1}$ give one month lagged average net saving in a group and their accumulated sums, respectively. Median group repayent shortfall rate is -1.42. 69 groups participated in the lending program.  \\
& 2. &  $P$ values in percentages in parenthesises. Standard errors are clustered at group (village) level. %${}^{***}$, ${}^{**}$, ${}^{*}$ indicate statistical significance at 1\%, 5\%, 10\%, respetively. Standard errors are clustered at group (village) level.
 \end{tabular}
\end{minipage} \\\\\addtocounter{table}{-1}\hspace{-1cm}\begin{minipage}[t]{14cm} \hfil\textsc{\normalsize Table \refstepcounter{table}\thetable: Individual level effects of repayment shortfall (continued)\label{tab shortfall indiv o800 2}}\\ \setlength{\tabcolsep}{1pt}
  \setlength{\baselineskip}{8pt}
  \renewcommand{\arraystretch}{.55}
  \hfil\begin{tikzpicture}
  \node (tbl) {\input{ c:/data/GUK/analysis/save/EstimationMemo/Shortfallo800EstimationResults2.tex }};
%\input{c:/dropbox/data/ramadan/save/tablecolortemplate.tex}
\end{tikzpicture}\\
\renewcommand{\arraystretch}{.8}
\setlength{\tabcolsep}{1pt} \begin{tabular}{>{\hfill\scriptsize}p{1cm}<{}>{\hfill\scriptsize}p{.25cm}<{}>{\scriptsize}p{12cm}<{\hfill}} Notes : & \multicolumn{2}{l}{\scriptsize See footnotes of \textsc{Table D22}.} \end{tabular}
\end{minipage} \\\\\addtocounter{table}{-1}\hspace{-1cm}\begin{minipage}[t]{14cm} \hfil\textsc{\normalsize Table \refstepcounter{table}\thetable: Individual level effects of repayment shortfall (continued)\label{tab shortfall indiv o800 3}}\\ \setlength{\tabcolsep}{1pt}
  \setlength{\baselineskip}{8pt}
  \renewcommand{\arraystretch}{.55}
  \hfil\begin{tikzpicture}
  \node (tbl) {\input{ c:/data/GUK/analysis/save/EstimationMemo/Shortfallo800EstimationResults3.tex }};
%\input{c:/dropbox/data/ramadan/save/tablecolortemplate.tex}
\end{tikzpicture}\\
\renewcommand{\arraystretch}{.8}
\setlength{\tabcolsep}{1pt} \begin{tabular}{>{\hfill\scriptsize}p{1cm}<{}>{\hfill\scriptsize}p{.25cm}<{}>{\scriptsize}p{12cm}<{\hfill}} Notes : & \multicolumn{2}{l}{\scriptsize See footnotes of \textsc{Table D23}.} \end{tabular}
\end{minipage} \\\\


\renewcommand{\arraystretch}{.6}
\mpage{\linewidth}{
\hfil\textsc{\footnotesize Figure \refstepcounter{figure}\thefigure: Assets by period\label{fig AssetCumRelativeToConcurrentTradEffects}}\\
\hfil\includegraphics[width = 12cm]{c:/data/GUK/analysis/program/figure/EstimationMemo/AssetsByPeriod.png}\\
\renewcommand{\arraystretch}{1}
\hfil\begin{tabular}{>{\hfill\scriptsize}p{1cm}<{}>{\scriptsize}p{12cm}<{\hfill}}
Source: & Tabulated with survey data.\\
Note:&  Red squares are means of respective data. \textsf{Net assets}=total assets - debts. Debts include outstanding loaned amount of the experiment. Total assets use items observed in all 4 rounds of household surveys. \textsf{Broad net assets}=total broad assets - debts, where total broad assets use all assets observed in the household surveys. \textsf{Broad net assets annual price} use annual median price for computing livestock values. \textsf{Net NL assets}=\textsf{net assets}-livestock asset values, \textsf{Broad net NL assets}=\textsf{Broad net assets}-livestock asset values. \textsf{Net assets, broad net assets} uses median baseline price for livestock.  All net assets are in logarithms, number of cattle is in natural numbers. \\[1ex]
\end{tabular}
}

\renewcommand{\arraystretch}{.6}
\mpage{\linewidth}{
\hfil\textsc{\footnotesize Figure \refstepcounter{figure}\thefigure: Cumulative impacts on assets relative to concurrent traditional arm\label{fig ImpactsOnAllAssetsCumRelativeToConcurrentTradEffects}}\\
\hfil\includegraphics[width = 14cm]{c:/data/GUK/analysis/program/figure/EstimationMemo/AssetCumRelativeToConcurrentTradEffects.pdf}\\
\renewcommand{\arraystretch}{1}
\hfil\begin{tabular}{>{\hfill\scriptsize}p{1cm}<{}>{\scriptsize}p{12cm}<{\hfill}}
Source: & Estimated with survey data.\\
Note:&  Cumulative impacts on various asset measures. \textsf{Net assets, broad net assets, net non livestock assets, land} are in BDT, cattle holding is in natural units.\\[1ex]
\end{tabular}
}

% \mpage{12cm}{
% \hfil\textsc{\footnotesize Figure \refstepcounter{figure}\thefigure: Cumulative effects on livestock, net assets, and narrow net assets\label{fig NetAssetsLivestockEffects}}\\
% 
% \vspace{2ex}
% \hfil\includegraphics[height = 8cm, width = 12cm]{      paste0(pathprogram, "figure/EstimationMemo/NarrowNetAssetsNetAssetsNumCowsCumRelativeToConcurrentTradEffects.eps")}\\
% \renewcommand{\arraystretch}{1}
% \setlength{\tabcolsep}{1pt}
% \hfil\begin{tabular}{>{\hfill\scriptsize}p{1cm}<{}>{\scriptsize}p{12.5cm}<{\hfill}}
% Source: & Constructed from ANCOVA estimation results \textsc{Table \ref{tab ANCOVA net assets timevarying}}, \textsc{Table \ref{tab ANCOVA net assets timevarying attributes}}, \textsc{Table \ref{tab ANCOVA cow time varying}}, \textsc{Table \ref{tab ANCOVA cow time varying attributes}}. \textsf{NarrowNetAssets} and \textsf{NetAssets} has 5 specifications (2-6), \textsf{NumCows} have 4 specifications (2-5).\\
% Note:& CumImpactText3 Net assets = assets + net saving - debt to GUK - debts to relatives and money lenders. \\[1ex]
% \end{tabular}
% }



% \begin{figure}
% \mpage{12cm}{
% \hfil\textsc{\footnotesize Figure \refstepcounter{figure}\thefigure: Effects on land, livestock, and net assets\label{fig LivestockEffects}}\\
% 
% \vspace{2ex}
% \hspace{-2em}\includegraphics[height = 12cm, width = 14cm]{      paste0(pathprogram, "figure/EstimationMemo/AssetRelativeToConcurrentTradEffects.eps")}\\
% \renewcommand{\arraystretch}{1}
% \setlength{\tabcolsep}{1pt}
% \hfil\begin{tabular}{>{\hfill\scriptsize}p{1cm}<{}>{\scriptsize}p{12.5cm}<{\hfill}}
% Source: & Constructed from ANCOVA estimation results.\\
% Note:& ConcurrentImpactText\\[1ex]
% \end{tabular}
% }
%\end{figure}




\end{document}
