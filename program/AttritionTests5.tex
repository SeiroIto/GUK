%  path0 <- "c:/data/GUK/"; path <- paste0(path0, "analysis/"); setwd(pathprogram <- paste0(path, "program/")); system("recycle c:/data/GUK/analysis/program/cache/AttritionTests5/"); library(knitr); knit("AttritionTests5.rnw", "AttritionTests5.tex"); system("platex AttritionTests5"); system("dvipdfmx AttritionTests5")

\input{c:/migrate/R/knitrPreamble/knitr_preamble.rnw}
\renewcommand\Routcolor{\color{gray30}}
\newtheorem{finding}{Finding}[section]
\makeatletter
\g@addto@macro{\UrlBreaks}{\UrlOrds}
\newcommand\gobblepars{%
    \@ifnextchar\par%
        {\expandafter\gobblepars\@gobble}%
        {}}
\newenvironment{lightgrayleftbar}{%
  \def\FrameCommand{\textcolor{lightgray}{\vrule width 1zw} \hspace{10pt}}% 
  \MakeFramed {\advance\hsize-\width \FrameRestore}}%
{\endMakeFramed}
\newenvironment{palepinkleftbar}{%
  \def\FrameCommand{\textcolor{palepink}{\vrule width 1zw} \hspace{10pt}}% 
  \MakeFramed {\advance\hsize-\width \FrameRestore}}%
{\endMakeFramed}
\makeatother
\usepackage{caption}
\usepackage{setspace}
\usepackage{framed}
\captionsetup[figure]{font={stretch=.6}} 
\def\pgfsysdriver{pgfsys-dvipdfm.def}
\usepackage{tikz}
\usetikzlibrary{calc, arrows, decorations, decorations.pathreplacing, backgrounds}
\usepackage{adjustbox}
\tikzstyle{toprow} =
[
top color = gray!20, bottom color = gray!50, thick
]
\tikzstyle{maintable} =
[
top color = blue!1, bottom color = blue!20, draw = white
%top color = green!1, bottom color = green!20, draw = white
]
\tikzset{
%Define standard arrow tip
>=stealth',
%Define style for different line styles
help lines/.style={dashed, thick},
axis/.style={<->},
important line/.style={thick},
connection/.style={thick, dotted},
}


\begin{document}
\setlength{\baselineskip}{12pt}








\hfil Permutation tests\\

\hfil\MonthDY\\
\hfil{\footnotesize\currenttime}\\

\hfil Seiro Ito

\setcounter{tocdepth}{3}
\tableofcontents

\setlength{\parindent}{1em}
\vspace{2ex}


Use the `trimmed' sample (has all 800 members) rather than the `initial' sample (has only 776 members after dropping members who received loans only twice). To set to the trimmed sample, set the parameter \textsf{UseTrimmedSample} to T.
\begin{Schunk}
\begin{Sinput}
UseTrimmedSample <- T
TestMedian <- F
\end{Sinput}
\end{Schunk}
There are 92 members who attrited.
\begin{Schunk}
\begin{Sinput}
asv <- readRDS(paste0(pathsaveHere, "DestatData.rds"))
addmargins(table0(asv[!grepl("tw|dou", TradGroup), .(Arm, Attrited)]))
\end{Sinput}
\end{Schunk}
Attrition of members who were not affected by floods nor rejected.
\begin{Schunk}
\begin{Sinput}
addmargins(table0(asv[!grepl("flo", BStatus) & Rejected == 0, .(Arm, Attrited)]))
\end{Sinput}
\begin{Soutput}
             Attrited
Arm             0   1 Sum
  traditional  83   2  85
  large       164   7 171
  large grace 160   7 167
  cow         147   6 153
  Sum         554  22 576
\end{Soutput}
\begin{Sinput}
# these are HHs with two disbursements under traditional; read_admin_data.rnw(472)
# adw[(loanamount1st == 5600 & loanamount2nd == 5600 & loanamount3rd == 5600) |
#   (!is.na(DisDate1) & !is.na(DisDate2) & !is.na(DisDate3)), 
#   TradGroup := "planned"]
# adw[loanamount1st == 5600 & loanamount2nd == 11200, 
#   TradGroup := "double"]
# adw[(loanamount1st == 7840 & loanamount2nd == 8960) | 
#   (!is.na(DisDate1) & !is.na(DisDate2) & is.na(DisDate3)), 
#   TradGroup := "twice"]
# adw[, TradGroup := factor(TradGroup, levels = c("planned", "twice", "double"))]
\end{Sinput}
\end{Schunk}
\begin{Schunk}
\begin{Sinput}
# data to use in each tests: TradNonTradAttrited, AttritedInTrad, TradNonTradRejected, IRejected, RejectedInTrad, RejectedInNonTrad
# drop 2 loan receivers
asv1 <- asv[!grepl("tw|dou", TradGroup), ]
# drop group rejecters
asv2 <- asv[!grepl("gr", BStatus), ]
# drop 2 loan receivers and group rejecters
asv3 <- asv[!grepl("gr", BStatus) & !grepl("tw|dou", TradGroup), ]
asvT <- asv[grepl("tra", Arm), ]
asvNT <- asv[!grepl("tra", Arm), ]
# data to be used for each tested variable
datalist <- rep("asv", length(vartobetested))
datalist1 <- paste0(datalist, 1) # drop 2 loan receivers
datalist2 <- paste0(datalist, 2) # drop group rejecters
datalist3 <- paste0(datalist, 3) # drop 2 loan receivers and group rejecters
datasets <- "asv"
datasets1 <- paste0(datasets, 1)
datasets2 <- paste0(datasets, 2)
datasets3 <- paste0(datasets, 3)
for (k in 1:3) {
  addchar <- c("f", "t", "j")[k]
  Datasets <- get(paste0("datasets", c("", 1, 2)[k]))
  for (dd in Datasets) {
    xdd <- get(dd)
    # all members all arms: attrited vs. nonattrited
    xa <- xdd
    assign(paste0(dd, "a", addchar), xa)
    # all in trad: attrition vs. nonattrition
    xTa <- xdd[grepl("trad", Arm), ]
    assign(paste0(dd, "Ta", addchar), xTa)
    # all in nontrad: attrition vs. nonattrition
    xNTa <- xdd[!grepl("trad", Arm), ]
    assign(paste0(dd, "NTa", addchar), xNTa)
    # attrited members in all arms: trad vs. nontrad
    xTNTa <- xdd[Attrited == 1L, ]
    xTNTa[, TradArm := 1L]; xTNTa[!grepl("trad", Arm), TradArm := 0L]
    assign(paste0(dd, "TNTa", addchar), xTNTa)
    # all members except flood victims: attrited vs. nonattrited
    xNFa <- xdd[!grepl("floo", BStatus), ]
    assign(paste0(dd, "NFa", addchar), xNFa)
    # all except flood victims in trad: attrition vs. nonattrition
    xNFTa <- xdd[!grepl("floo", BStatus) & grepl("trad", Arm), ]
    assign(paste0(dd, "NFTa", addchar), xNFTa)
    # all except flood victims in nontrad: attrition vs. nonattrition
    xNFNTa <- xdd[!grepl("floo", BStatus) & !grepl("trad", Arm), ]
    assign(paste0(dd, "NFNTa", addchar), xNFNTa)
    # attrited members except flood victims in all arms: trad vs. nontrad
    xNFTNTa <- xdd[!grepl("floo", BStatus) & Attrited == 1L, ]
    xNFTNTa[, TradArm := 1L]; xNFTNTa[!grepl("trad", Arm), TradArm := 0L]
    assign(paste0(dd, "NFTNTa", addchar), xNFTNTa)
    # attrited members except flood victims in all arms: cow vs. noncow
    xNFCNCa <- xdd[!grepl("floo", BStatus) & Attrited == 1L, ]
    xNFCNCa[, CowArm := 1L]; xNFCNCa[!grepl("cow", Arm), CowArm := 0L]
    assign(paste0(dd, "NFCNCa", addchar), xNFCNCa)
    # attrited members except flood victims: cow vs. large grace
    xNFCGa <- xdd[!grepl("floo", BStatus) & grepl("cow|gr", Arm) & Attrited == 1L, ]
    xNFCGa[, CowArm := 1L]; xNFCGa[!grepl("cow", Arm), CowArm := 0L]
    assign(paste0(dd, "NFCGa", addchar), xNFCGa)
    # being active (neither attrited nor rejected) members except flood victims 
    # (these people are considered not fit for the offered program)
    # active in all arms
    xs <- xdd
    assign(paste0(dd, "s", addchar), xs)
    # active in trad: attrition vs. nonattrition
    xTs <- xdd[grepl("trad", Arm), ]
    assign(paste0(dd, "Ts", addchar), xTs)
    # active in nontrad: attrition vs. nonattrition
    xNTs <- xdd[!grepl("trad", Arm), ]
    assign(paste0(dd, "NTs", addchar), xNTs)
    # active members in all arms: trad vs. nontrad
    xTNTs <- xdd[Active == 1L, ]
    xTNTs[, TradArm := 1L]; xTNTs[!grepl("trad", Arm), TradArm := 0L]
    assign(paste0(dd, "TNTs", addchar), xTNTs)
    # active members: cow vs. noncow
    xCNCs <- xdd[!grepl("floo", BStatus) & Active == 1L, ]
    xCNCs[, CowArm := 1L]; xCNCs[!grepl("cow", Arm), CowArm := 0L]
    assign(paste0(dd, "CNCs", addchar), xCNCs)
    # active members: cow vs. lsge grace
    xCGs <- xdd[!grepl("floo", BStatus) & grepl("cow|gr", Arm) & Active == 1L, ]
    xCGs[, CowArm := 1L]; xCGs[!grepl("cow", Arm), CowArm := 0L]
    assign(paste0(dd, "CGs", addchar), xCGs)
    # all rejection all arms: rejected vs. nonrejected
    xr <- xdd
    assign(paste0(dd, "r", addchar), xr)
    # all rejection in trad: rejected vs. nonrejected
    xTr <- xdd[grepl("trad", Arm), ]
    assign(paste0(dd, "Tr", addchar), xTr)
    # all rejection in nontrad: rejected vs. nonrejected
    xNTr <- xdd[!grepl("trad", Arm), ]
    assign(paste0(dd, "NTr", addchar), xNTr)
    # all rejection: trad rejecetd vs. nontrad rejected
    xTNTr <- xdd[Rejected == 1L, ]
    xTNTr[, TradArm := 1L]; xTNTr[!grepl("trad", Arm), TradArm := 0L]
    assign(paste0(dd, "TNTr", addchar), xTNTr)
    # all rejection: cow rejecetd vs. noncow rejected
    xCNCr <- xdd[Rejected == 1L, ]
    xCNCr[, CowArm := 1L]; xCNCr[!grepl("cow", Arm), CowArm := 0L]
    assign(paste0(dd, "CNCr", addchar), xCNCr)
    # all rejection: cow rejecetd vs. large grace rejected
    xCLGr <- xdd[grepl("cow|gr", Arm) & Rejected == 1L, ]
    xCLGr[, CowArm := 1L]; xCLGr[!grepl("cow", Arm), CowArm := 0L]
    assign(paste0(dd, "CLGr", addchar), xCLGr)
    # all acceptance: cow accepted vs. noncow accepted
    xCNCa <- xdd[Rejected == 0L, ]
    xCNCa[, CowArm := 1L]; xCNCa[!grepl("cow", Arm), CowArm := 0L]
    assign(paste0(dd, "CNCa", addchar), xCNCa)
    # all acceptance: cow accepted vs. large grace accepted
    xCLGa <- xdd[grepl("cow|gr", Arm) & Rejected == 0L, ]
    xCLGa[, CowArm := 1L]; xCLGa[!grepl("cow", Arm), CowArm := 0L]
    assign(paste0(dd, "CLGa", addchar), xCLGa)
    # group rejection in all arms: rejected vs. nonrejected
    xgr <- xdd
    assign(paste0(dd, "gr", addchar), xgr)
    # group rejection in trad: rejecters vs. nonrejecters
    xTgr <- xdd[grepl("tra", Arm), ]
    assign(paste0(dd, "Tgr", addchar), xTgr)
    # group rejection in nontrad: rejecters vs. nonrejecters
    xNTgr <- xdd[!grepl("tra", Arm), ]
    assign(paste0(dd, "NTgr", addchar), xNTgr)
    # group rejection: trad rejecters vs. nontrad rejecters
    xTNTgr <- xdd[GRejected == 1L, ]
    xTNTgr[, TradArm := 1L]; xTNTgr[!grepl("trad", Arm), TradArm := 0L]
    assign(paste0(dd, "TNTgr", addchar), xTNTgr)
    # individual rejection in all arms: rejected vs. nonrejected
    # individual rejecters vs. all except group rejecters
    # group rejecters are excluded because they preceded indiv rejection
    xir <- xdd[!grepl("gr", BStatus), ]
    assign(paste0(dd, "ir", addchar), xir)
    # individual rejection in trad: rejecters vs. nonrejecters
    xTir <- xdd[grepl("tra", Arm) & !grepl("gr", BStatus), ]
    assign(paste0(dd, "Tir", addchar), xTir)
    # individual rejection in nontrad: rejecters vs. nonrejecters
    xNTir <- xdd[!grepl("tra", Arm) & !grepl("gr", BStatus), ]
    assign(paste0(dd, "NTir", addchar), xNTir)
    # individual rejection: trad rejecters vs. nontrad rejecters
    xTNTir <- xdd[!grepl("gr", BStatus) & Rejected == 1L, ]
    xTNTir[, TradArm := 1L]; xTNTir[!grepl("trad", Arm), TradArm := 0L]
    assign(paste0(dd, "TNTir", addchar), xTNTir)
    # trad group rejecters vs. nontrad participants
    xTNTgrp <- xdd[(grepl("gr", BStatus) & grepl("trad", Arm) & Rejected == 1L) |
      (grepl("bo", BStatus) & !grepl("trad", Arm)), ]
    xTNTgrp[, TradArm := 1L]; xTNTgrp[!grepl("trad", Arm), TradArm := 0L]
    assign(paste0(dd, "TNTgrp", addchar), xTNTgrp)
    # trad group vs. nontrad group
    xTNTrandom <- xdd
    xTNTrandom[, TradArm := 1L]; xTNTrandom[!grepl("trad", Arm), TradArm := 0L]
    assign(paste0(dd, "TNTrandom", addchar), xTNTrandom)
  }
}
\end{Sinput}
\end{Schunk}
\begin{Schunk}
\begin{Sinput}
# data names: ..af, ..rf  (full), ..at, ..rt (drop 2 loan receivers), ..aj, ..rj (drop group rejeceters)
# data to use: datalist (full), datalist1 (drop 2 loan receivers), datalist2 (drop group rejeceters)
library(coin)
PM <- vector(mode = "list", length = 3)
for (k in 1:3) {
  addchar <- c("f", "t", "j")[k]
  dataList <- eval(parse(text=paste0("datalist", c("", 1:2))[k]))
  if (addchar == "j") M <- 9 else M <- length(selection.criteria)
  Pm <- vector(mode = "list", length = M)
  for (m in 1:M) {
    set.seed(100+m)
    if (grepl("^Attrited$", addtofilename[m])) 
      DataList <- gsub("$", paste0("a", addchar), dataList) else
    if (grepl("^AttritedInTrad", addtofilename[m])) 
      DataList <- gsub("$", paste0("Ta", addchar), dataList) else
    if (grepl("^AttritedInNonTrad", addtofilename[m])) 
      DataList <- gsub("$", paste0("NTa", addchar), dataList) else
    if (grepl("^TradNonTradAttrited$", addtofilename[m])) 
      DataList <- gsub("$", paste0("TNTa", addchar), dataList) else
    if (grepl("^NonFloodAttrited$", addtofilename[m])) 
      DataList <- gsub("$", paste0("NFa", addchar), dataList) else
    if (grepl("^NonFloodAttritedInTrad$", addtofilename[m])) 
      DataList <- gsub("$", paste0("NFTa", addchar), dataList) else
    if (grepl("^NonFloodAttritedInNonTrad$", addtofilename[m])) 
      DataList <- gsub("$", paste0("NFNTa", addchar), dataList) else
    if (grepl("^NonFloodTradNonTradAttrited$", addtofilename[m])) 
      DataList <- gsub("$", paste0("NFTNTa", addchar), dataList) else
    if (grepl("^NonFloodAttritedCowN", addtofilename[m])) 
      DataList <- gsub("$", paste0("NFCNCa", addchar), dataList) else
    if (grepl("^NonFloodAttritedCowL", addtofilename[m])) 
      DataList <- gsub("$", paste0("NFCGa", addchar), dataList) else
    if (grepl("^Active$", addtofilename[m])) 
      DataList <- gsub("$", paste0("s", addchar), dataList) else
    if (grepl("^ActiveInTrad", addtofilename[m])) 
      DataList <- gsub("$", paste0("Ts", addchar), dataList) else
    if (grepl("^ActiveInNonTrad", addtofilename[m])) 
      DataList <- gsub("$", paste0("NTs", addchar), dataList) else
    if (grepl("^ActiveTradNonTrad", addtofilename[m])) 
      DataList <- gsub("$", paste0("TNTs", addchar), dataList) else
    if (grepl("^ActiveCowN", addtofilename[m]))
      DataList <- gsub("$", paste0("CNCs", addchar), dataList) else
    if (grepl("^ActiveCowL", addtofilename[m])) 
      DataList <- gsub("$", paste0("CGs", addchar), dataList) else
    if (grepl("^Random", addtofilename[m])) 
      DataList <- gsub("$", paste0("TNTrandom", addchar), dataList) else
    if (grepl("^Rejected$", addtofilename[m])) 
      DataList <- gsub("$", paste0("r", addchar), dataList) else
    if (grepl("^Rej.*InTrad$", addtofilename[m])) 
      DataList <- gsub("$", paste0("Tr", addchar), dataList) else
    if (grepl("^Rej.*InNonTrad$", addtofilename[m])) 
      DataList <- gsub("$", paste0("NTr", addchar), dataList) else
    if (grepl("^TradNonTradR", addtofilename[m])) 
      DataList <- gsub("$", paste0("TNTr", addchar), dataList) else
    if (grepl("^GRejected$", addtofilename[m])) 
      DataList <- gsub("$", paste0("gr", addchar), dataList) else
    if (grepl("^GRej.*InTrad$", addtofilename[m])) 
      DataList <- gsub("$", paste0("Tgr", addchar), dataList) else
    if (grepl("^GRej.*InNonTrad$", addtofilename[m])) 
      DataList <- gsub("$", paste0("NTgr", addchar), dataList) else
    if (grepl("^TradNonTradGR", addtofilename[m])) 
      DataList <- gsub("$", paste0("TNTgr", addchar), dataList) else
    if (grepl("^IRejected$", addtofilename[m])) 
      DataList <- gsub("$", paste0("ir", addchar), dataList) else
    if (grepl("^IRej.*InTrad$", addtofilename[m])) 
      DataList <- gsub("$", paste0("Tir", addchar), dataList) else
    if (grepl("^IRej.*InNonTrad$", addtofilename[m])) 
      DataList <- gsub("$", paste0("NTir", addchar), dataList) else
    if (grepl("^TradNonTradIR", addtofilename[m])) 
      DataList <- gsub("$", paste0("TNTir", addchar), dataList) else
    if (grepl("^GRejectedTradPar", addtofilename[m])) 
      DataList <- gsub("$", paste0("TNTgrp", addchar), dataList) else
    if (grepl("^RejectedCowN", addtofilename[m])) 
      DataList <- gsub("$", paste0("CNCr", addchar), dataList) else
    if (grepl("^RejectedCowLa", addtofilename[m])) 
      DataList <- gsub("$", paste0("CLGr", addchar), dataList) else
    if (grepl("^AcceptedCowN", addtofilename[m])) 
      DataList <- gsub("$", paste0("CNCa", addchar), dataList) else
    if (grepl("^AcceptedCowLa", addtofilename[m])) 
      DataList <- gsub("$", paste0("CLGa", addchar), dataList) else
      DataList <- gsub("$", addchar, dataList)
    pmresults <- permmedian <- vector(mode = "list", length(vartobetested))
    for (i in 1:length(vartobetested)) {
      # if specific arm is selected, Arm is not compared in permutation
      if (grepl("Trad$|TradArm|Cow", addtofilename[m]) & 
        vartobetested[i] == "Arm") next
      pmdata <- get(DataList[i])
      # drop NAs in vartobetested[i]
      pmdata <- pmdata[!is.na(eval(parse(text=vartobetested[i]))), ]
      # NULL if vartobetested[i] has uniform values (otherwise returns an error)
      if (length(unique(unlist(pmdata[, vartobetested[i], with = F]))) == 1) 
        pmresults[[i]] <- NULL else
        pmresults[[i]] <- independence_test(eval(parse(text=
          paste(vartobetested[i], "~ as.factor(", selection.criteria[m], ")")
          )), 
          data = pmdata, 
          distribution = approximate(nresample=PermRepTimes))
      if (!TestMedian) next
      if (vartobetested[i] == "Arm" | length(unique(unlist(pmdata[, vartobetested[i], with = F]))) == 1) 
        permmedian[[i]] <- NULL else 
        permmedian[[i]] <- median_test(eval(parse(text=
          paste(vartobetested[i], "~ as.factor(", selection.criteria[m], ")"))), 
          data = pmdata,
          mid.score = "0.5",
          distribution = approximate(nresample=PermRepTimes))
    }
    #pmresults[[1]]@statistic@teststatistic
    Pmtresults <- NULL
    for (i in 1:length(vartobetested)) 
    {
      if (grepl("Trad$|TradArm|Cow", addtofilename[m]) & 
        vartobetested[i] == "Arm") next
      z <- get(DataList[i])
      z <- z[!is.na(eval(parse(text=vartobetested[i]))), ]
      if (vartobetested[i] == "Arm") {
        Pmtresults <- rbind(Pmtresults, 
          c(vartobetested[i], 
            sum(!grepl("trad", unlist(z[eval(parse(text = selection.criteria[m])) == 0L, 
              vartobetested[i], with = F])))/
              nrow(z[eval(parse(text = selection.criteria[m])) == 0L, ]),
            sum(!grepl("trad", unlist(z[eval(parse(text = selection.criteria[m])) == 1L, 
              vartobetested[i], with = F])))/
              nrow(z[eval(parse(text = selection.criteria[m])) == 1L, ]),
            midpvalue(pmresults[[i]]), 
            pvalue_interval(pmresults[[i]])))
      } else if (length(unique(unlist(z[, vartobetested[i], with = F]))) == 1) 
      {
        # if both groups have no different values, 
        # use 0 for all zero entries or 1 for unique nonzero entries
        if (allzerovalues <- unique(unlist(z[, vartobetested[i], with = F])) == 0)
          Pmtresults <- rbind(Pmtresults, 
            c(vartobetested[i], 0, 0, rep(NA, 3))) else
          Pmtresults <- rbind(Pmtresults, 
            c(vartobetested[i], 1, 1, rep(NA, 3)))
      } else {
        Pmtresults <- rbind(Pmtresults, 
          c(vartobetested[i], 
            mean(unlist(z[eval(parse(text = selection.criteria[m])) == 0L, 
              vartobetested[i], with = F]), na.rm = T),
            mean(unlist(z[eval(parse(text = selection.criteria[m])) == 1L, 
              vartobetested[i], with = F]), na.rm = T),
            midpvalue(pmresults[[i]]), 
            pvalue_interval(pmresults[[i]])))
        if (TestMedian)
          Pmtresults <- rbind(Pmtresults, 
            c("", 
              median(unlist(z[eval(parse(text = selection.criteria[m])) == 0L, 
                vartobetested[i], with = F]), na.rm = T),
              median(unlist(z[eval(parse(text = selection.criteria[m])) == 1L, 
                vartobetested[i], with = F]), na.rm = T),
              midpvalue(permmedian[[i]]), 
              pvalue_interval(permmedian[[i]]) 
            ))
       }
    }
    Pmtresults <- data.table(Pmtresults)
    setnames(Pmtresults, c("variables", paste0(c("Non", ""), selection.criteria[m]), 
      "p-value.mid", "p-value.lower", "p-value.upper"))
    Pmtresults[grepl("Impute", variables), 
      variables := gsub("To.*", "LivestockValue", variables)]
    cols <- grepout("p|er|ttr|eje|TradArm|CowA|Acti", colnames(Pmtresults))
    Pmtresults[,  (cols) := lapply(.SD, as.numeric), .SDcols = cols]
    Pmtresults[,  (cols) := lapply(.SD, formatC, digits = 3, format = "f"), .SDcols = cols]
    cols <- grepout("ed$|TradArm|CowA", colnames(Pmtresults))
    Pmtresults[grepl("Ass|Liv|NetV", variables),  
      (cols) := lapply(.SD, function(x) formatC(as.numeric(x), digits = 0, format = "f")), 
      .SDcols = cols]
    setcolorder(Pmtresults, c("variables", paste0(c("Non", ""), selection.criteria[m]), 
      "p-value.lower", "p-value.mid", "p-value.upper"))
    obs0L <- nrow(get(DataList[1])[eval(parse(text = selection.criteria[m])) == 0L, ])
    obs1L <- nrow(get(DataList[1])[eval(parse(text = selection.criteria[m])) == 1L, ])
    nobs <- t(c(NA, obs0L, obs1L, NA, obs1L/(obs0L+obs1L), NA))
    Pmtresults[, variables := paste0("\\makebox[2.5cm]{\\hfill ", variables, "}")]
    Pmtresults0 <- rbind(Pmtresults, nobs, use.names = F)
    Pmtresults0[nrow(Pmtresults0), variables := "\\makebox[2.5cm]{\\hfill n}"]
    Pm[[m]] <- Pmtresults0
    if (grepl("InNon|InTra|^TradNon|Cow", addtofilename[m]))
      Pmtresults <- Pmtresults[!grepl("Arm", variables), ]
    pmt <- latextab(as.matrix(Pmtresults), 
      hleft = "\\scriptsize\\hfil$", 
      hcenter = c(3, rep(1.5, ncol(Pmtresults)-1)), 
      hright = "$", 
      headercolor = "gray80", adjustlineskip = "-.2ex", delimiterline= NULL,
      alternatecolor = "gray90")
    pmt <- rbind(pmt[1:(nrow(pmt)-1), , drop = F], 
        paste(c("\\makebox[2.5cm]{\\hfill n}", 
         obs0L, obs1L, paste0("\\multicolumn{3}{l}{\\makebox[4.5cm]{\\scriptsize (rate: ", 
         formatC(obs1L/(obs0L+obs1L), digits = 3, format = "f"), ")\\hfill}}")), 
         collapse = " & "),
        pmt[nrow(pmt), , drop = F] 
      )
    write.tablev(pmt,  
      paste0(pathsaveHere, addtofilename[m], 
        c("Full", "", "DropGroupRejecters")[k], "PermutationTestResultso800.tex")
    , colnamestrue = F)
  }
  names(Pm) <- addtofilename[1:M]
  PM[[k]] <- Pm
}
names(PM) <- c("Full", "Drop2LoanReceivers", "DropGroupRejecters")
saveRDS(PM, paste0(pathsaveHere, "AllPermutationTestResults.rds"))
PM <- readRDS(paste0(pathsaveHere, "AllPermutationTestResults.rds"))
# indiv rejecters
Irej <- c("IRejectedInTrad", "IRejectedInNonTrad", "^IRejected$")
ir12 <-  cbind(
    PM[[2]][[grep(Irej[1], addtofilename)]][, c(1:3, 5)],
    PM[[2]][[grep(Irej[2], addtofilename)]][, c(2:3, 5)])
setnames(ir12, c("variables", 1:(ncol(ir12)-1)))
ir3 <- PM[[2]][[grep(Irej[3], addtofilename)]][, c(1:3, 5)]
setnames(ir3, c("variables", 10+1:(ncol(ir3)-1)))
ir3rows <- data.table(variables = ir3[, variables])
setkey(ir12, variables)
setkey(ir3, variables)
ir123 <- ir12[ir3]
ir123 <- ir123[ir3rows]
setnames(ir123, c("variables", paste0("v", 1:(ncol(ir123)-1))))
for (i in paste0("v", c(3, 6, 9))) 
  ir123[nrow(ir123), (i) := 
    paste0("(\\mbox{rate }", formatC(as.numeric(eval(parse(text=i))), digits = 3, format = "f"), ")") ]
#cnm <- t(c("\\makebox[3cm]{\\hfil variables}", 
#  paste0("\\makebox[1.5cm]{\\hfil ", rep(c("Yes", "No", "$p$ value"), 3), "}")))
cnm <- t(c("\\makebox[2.5cm]{\\hfil }", 
  paste0("\\makebox[1.2cm]{(", 1:(ncol(ir123)-1), ")}")))
irj <- as.matrix(rbind(cnm, ir123, use.names = F))
irj[is.na(irj)] <- ""
colnames(irj) <- c("variables", rep(c("Not rejected", "Rejected", "$p$ value"), 3))
irj <- latextab(irj, 
  hleft = "\\scriptsize\\hfil$", 
  hcenter = c(2.5, rep(1.2, ncol(Pmtresults)-1)), 
  hright = "$", 
  headercolor = "gray80", adjustlineskip = "-.2ex", delimiterline= NULL,
  alternatecolor = "gray90",
  addseparatingcols = c(3, 6), 
  separatingcolwidth = rep(.1, 2),
  separatingcoltitle = c("\\textsf{Traditional} arm", "non-\\textsf{Traditional} arms", "All arms"),
  addsubcoltitlehere = T
  )
write.tablev(irj,  
  paste0(pathsaveHere, "IndividualRejectionTestResults.tex")
, colnamestrue = F)
# active
Suv <- c("Acc.*NonCow", "Act.*NonCow")
sv12 <-  cbind(
    PM[[2]][[grep(Suv[1], addtofilename)]][, c(1, 3, 2, 5)],
    PM[[2]][[grep(Suv[2], addtofilename)]][, c(3, 2, 5)])
setnames(sv12, c("variables", paste0("v", 1:(ncol(sv12)-1))))
for (i in paste0("v", c(3, 6))) 
  sv12[nrow(sv12), (i) := 
    paste0("(\\mbox{rate }", formatC(as.numeric(eval(parse(text=i))), digits = 3, format = "f"), ")") ]
cnm <- t(c("\\makebox[2.5cm]{\\hfil }", 
  paste0("\\makebox[1.2cm]{(", 1:(ncol(sv12)-1), ")}")))
suv <- as.matrix(rbind(cnm, sv12, use.names = F))
colnames(suv) <- c("variables", rep(c("Cattle arm", "Other arms", "$p$ value"), 2))
suv <- latextab(suv, 
  hleft = "\\scriptsize\\hfil$", 
  hcenter = c(2.5, rep(1.2, ncol(suv)-1)), 
  hright = "$", 
  headercolor = "gray80", adjustlineskip = "-.2ex", delimiterline= NULL,
  alternatecolor = "gray90",
  addseparatingcols = 3, 
  separatingcolwidth = .1, 
  separatingcoltitle = c("Borrowers", "Non-attriting borrowers"),
  addsubcoltitlehere = T
  )
write.tablev(suv,  
  paste0(pathsaveHere, "CowVsNonCowTestResults.tex")
, colnamestrue = F)
\end{Sinput}
\end{Schunk}

% Attrition tests are conducted in AttritionTests5.rnw
% TabLabel1 is from paste0(pathprogram, "AttritionPermutationTableHeaders5.R") referred in the main rnw file
\begin{Schunk}
\begin{Sinput}
HeaderDescription[11:16] <- c(
   "Permutation test results of attrition among borrowers", 
     "Permutation test results of attrition among traditional arm borrowers",
     "Permutation test results of attrition among non-traditional arm borrowers",
     "Permutation test results of non-attriting members of traditional and non-traditional arm borrowers",
     "Permutation test results of non-attriting members of cattle and all other arm borrowers",
     "Permutation test results of non-attriting members of cattle and large grace arm borrowers",
    )
TabLabel1 <- paste("tab1", HeaderDescription)
% TabLabelStrings is from RejectionTestsMainText4.rnw 
% I will modify the table headers for survival/surviving members to "non-attriting borrowers"
%   "survival$", "surviving.*race$", "surv.*other"
\end{Sinput}
\end{Schunk}
\begin{Schunk}
\begin{Sinput}
TabLabelStrings <- 
c(
  "Per.* of rejection$", "of rej.*ng traditional arm$", "of rej.*ng non-traditional arm$", 
  "of rej.*l vs", "p rejection$", "p rej.* traditional arm$", 
  "p rej.*g non-traditional arm$", "p rejecters.*vs", "l rejection$", 
  "l rej.*g tra",  "l rej.*g non-", "l rej.*vs", 
  "bo.*non-ca", 
  "of attrition$", "of attri.* traditional arm$", 
  "of attri.* non-traditional arm$", "of attri.* of",
  "active status$", "active.*race$", "active.*other"
  )
\end{Sinput}
\end{Schunk}


\begin{Schunk}
\begin{Sinput}
tb1 <- "\\hfil\\begin{minipage}[t]{14cm}\\hfil\\textsc{\\normalsize Table \\refstepcounter{table}\\thetable: "
tb2 <- "}\\\\\\setlength{\\tabcolsep}{.5pt}\\setlength{\\baselineskip}{8pt}\\renewcommand{\\arraystretch}{.50}\\hfil\\begin{tikzpicture}\\node (tbl) {\\input{"
tb3 <- "}};\\end{tikzpicture}\\\\\\begin{tabular}{>{\\hfill\\scriptsize}p{1cm}<{}>{\\hfill\\scriptsize}p{.25cm}<{}>{\\scriptsize}p{12cm}<{\\hfill}}Source:& \\multicolumn{2}{l}{\\scriptsize Estimated with GUK administrative and survey data.}\\\\ Notes: & 1. & \\textsf{R}'s package \\textsf{coin} is used for baseline mean covariates to conduct approximate permutation tests. Number of repetition is set to "
#tb4 <- ". Step-down method is used to adjust for multiple testing of a multi-factor grouping variable. \\textsf{Attrited} and \\textsf{Nonattrited} columns show means of each group. For \\textsf{Arm}, proportions of non-traditional arm are given. \\\\& 2. & ${}^{***}$, ${}^{**}$, ${}^{*}$ indicate statistical significance at 1\\%, 5\\%, 10\\%, respetively. Standard errors are clustered at group (village) level.  \\\\& 3. & See the footnote of \\textsc{Table \\ref{tab MainTextIRjecters}} for description of variables. \\end{tabular}\\end{minipage}\\\\\\vspace{2ex}"
tb42 <- ". Step-down method is used to adjust for multiple testing of a multi-factor grouping variable. The second and third columns show means of each group. For \\textsf{Arm}, proportions of non-traditional arm between two groups are tested. \\\\& 2. & \\textsf{p-value.lower}, \\textsf{p-value.mid}, \\textsf{p-value.upper} indicate lower-bound, mid point value, and upper-bound of the $p$ values for observed test statistic and the null distribution, expressed in per centage units. \\\\& 3. & "
tb41 <- paste(tb42, "See the footnote of \\textsc{Table \\ref{tab MainTextIRjecters}} for description of variables. \\end{tabular}\\end{minipage}\\\\\\vspace{2ex} ")
tb43 <- ".\\\\& 2. &  See footnotes of \\textsc{Table \\ref{tab1 Permutation test results of rejection}}. \\end{tabular}\\end{minipage}\\\\\\vspace{2ex}"
tb44 <- ".\\\\& 2. &  See footnotes of \\textsc{Table \\ref{tab1 Permutation test results of rejection among traditional arm}}. \\end{tabular}\\end{minipage}\\\\\\vspace{2ex}"
for (k in 1:3) 
  for (i in 1:length(HeaderDescription))
    assign(paste0("Tb", k, i), 
      paste0(
        tb1
        , 
        HeaderDescription[i]
        ,
        paste0("\\label{", get(paste0("TabLabel", k))[i], "}")
        ,
        tb2
        , 
        paste0(pathsavePerm, addtofilename[i], c("", "Full", "DropGroupRejecters")[k],
          "PermutationTestResultso800.tex")
        , 
        tb3
        ,
        PermRepTimes
        ,
        #if (i %in% c(1, 5, 11, 17, 21, 25)) tb42 else tb41
        if (i==17) 
        paste0(tb42, TabVariableDescription, PrefTestsDefinitions1, "\\end{tabular}\\end{minipage}\\\\\\vspace{2ex}") else 
        #if (i==18) tb41 else 
        if (i %in% c(1, 5, 11, 18, 21, 25)) tb43 else 
        tb44
        )
      )
\end{Sinput}
\end{Schunk}

\begin{Schunk}
\begin{Sinput}
TabLabelStrings[19:20] <- c("active.*race arms$", "activ.*other arms$")
\end{Sinput}
\end{Schunk}
\begin{Sinput}
for (i in 14:length(TabLabelStrings)) {
  ii <- grep(TabLabelStrings[i], TabLabel1)
  if (grepl("active", TabLabelStrings[i])){
   tblatt <- eval(parse(text=paste0("Tb1", ii)))
   cat(gsub("active\\\\", "attrited or rejected (NonActive) and other (Active) borrowers\\\\", 
     tblatt)) 
   rm(tblatt)
  } else
  if (grepl("active|activ.*o", TabLabelStrings[i])) {
   tblatt <- eval(parse(text=paste0("Tb1", ii)))
   cat(gsub("active members", "non-attriting borrowers", 
     tblatt)) 
    rm(tblatt)
  } else
   cat(eval(parse(text=paste0("Tb1", ii))))
}
\end{Sinput}
\hfil\begin{minipage}[t]{14cm}\hfil\textsc{\normalsize Table \refstepcounter{table}\thetable: Permutation test results of attrition\label{tab1 Permutation test results of attrition}}\\\setlength{\tabcolsep}{.5pt}\setlength{\baselineskip}{8pt}\renewcommand{\arraystretch}{.50}\hfil\begin{tikzpicture}\node (tbl) {\input{c:/data/GUK/analysis/save/PermutationTests/AttritedPermutationTestResultso800.tex}};\end{tikzpicture}\\\begin{tabular}{>{\hfill\scriptsize}p{1cm}<{}>{\hfill\scriptsize}p{.25cm}<{}>{\scriptsize}p{12cm}<{\hfill}}Source:& \multicolumn{2}{l}{\scriptsize Estimated with GUK administrative and survey data.}\\ Notes: & 1. & \textsf{R}'s package \textsf{coin} is used for baseline mean covariates to conduct approximate permutation tests. Number of repetition is set to 100000.\\& 2. &  See footnotes of \textsc{Table \ref{tab1 Permutation test results of rejection}}. \end{tabular}\end{minipage}\\\vspace{2ex}\hfil\begin{minipage}[t]{14cm}\hfil\textsc{\normalsize Table \refstepcounter{table}\thetable: Permutation test results of attrition among traditional arm\label{tab1 Permutation test results of attrition among traditional arm}}\\\setlength{\tabcolsep}{.5pt}\setlength{\baselineskip}{8pt}\renewcommand{\arraystretch}{.50}\hfil\begin{tikzpicture}\node (tbl) {\input{c:/data/GUK/analysis/save/PermutationTests/AttritedInTradPermutationTestResultso800.tex}};\end{tikzpicture}\\\begin{tabular}{>{\hfill\scriptsize}p{1cm}<{}>{\hfill\scriptsize}p{.25cm}<{}>{\scriptsize}p{12cm}<{\hfill}}Source:& \multicolumn{2}{l}{\scriptsize Estimated with GUK administrative and survey data.}\\ Notes: & 1. & \textsf{R}'s package \textsf{coin} is used for baseline mean covariates to conduct approximate permutation tests. Number of repetition is set to 100000.\\& 2. &  See footnotes of \textsc{Table \ref{tab1 Permutation test results of rejection among traditional arm}}. \end{tabular}\end{minipage}\\\vspace{2ex}\hfil\begin{minipage}[t]{14cm}\hfil\textsc{\normalsize Table \refstepcounter{table}\thetable: Permutation test results of attrition among non-traditional arm\label{tab1 Permutation test results of attrition among non-traditional arm}}\\\setlength{\tabcolsep}{.5pt}\setlength{\baselineskip}{8pt}\renewcommand{\arraystretch}{.50}\hfil\begin{tikzpicture}\node (tbl) {\input{c:/data/GUK/analysis/save/PermutationTests/AttritedInNonTradPermutationTestResultso800.tex}};\end{tikzpicture}\\\begin{tabular}{>{\hfill\scriptsize}p{1cm}<{}>{\hfill\scriptsize}p{.25cm}<{}>{\scriptsize}p{12cm}<{\hfill}}Source:& \multicolumn{2}{l}{\scriptsize Estimated with GUK administrative and survey data.}\\ Notes: & 1. & \textsf{R}'s package \textsf{coin} is used for baseline mean covariates to conduct approximate permutation tests. Number of repetition is set to 100000.\\& 2. &  See footnotes of \textsc{Table \ref{tab1 Permutation test results of rejection among traditional arm}}. \end{tabular}\end{minipage}\\\vspace{2ex}\hfil\begin{minipage}[t]{14cm}\hfil\textsc{\normalsize Table \refstepcounter{table}\thetable: Permutation test results of attriters of traditional and non-traditional arms\label{tab1 Permutation test results of attriters of traditional and non-traditional arms}}\\\setlength{\tabcolsep}{.5pt}\setlength{\baselineskip}{8pt}\renewcommand{\arraystretch}{.50}\hfil\begin{tikzpicture}\node (tbl) {\input{c:/data/GUK/analysis/save/PermutationTests/TradNonTradAttritedPermutationTestResultso800.tex}};\end{tikzpicture}\\\begin{tabular}{>{\hfill\scriptsize}p{1cm}<{}>{\hfill\scriptsize}p{.25cm}<{}>{\scriptsize}p{12cm}<{\hfill}}Source:& \multicolumn{2}{l}{\scriptsize Estimated with GUK administrative and survey data.}\\ Notes: & 1. & \textsf{R}'s package \textsf{coin} is used for baseline mean covariates to conduct approximate permutation tests. Number of repetition is set to 100000.\\& 2. &  See footnotes of \textsc{Table \ref{tab1 Permutation test results of rejection among traditional arm}}. \end{tabular}\end{minipage}\\\vspace{2ex}\hfil\begin{minipage}[t]{14cm}\hfil\textsc{\normalsize Table \refstepcounter{table}\thetable: Permutation test results of active status\label{tab1 Permutation test results of active status}}\\\setlength{\tabcolsep}{.5pt}\setlength{\baselineskip}{8pt}\renewcommand{\arraystretch}{.50}\hfil\begin{tikzpicture}\node (tbl) {\input{c:/data/GUK/analysis/save/PermutationTests/ActivePermutationTestResultso800.tex}};\end{tikzpicture}\\\begin{tabular}{>{\hfill\scriptsize}p{1cm}<{}>{\hfill\scriptsize}p{.25cm}<{}>{\scriptsize}p{12cm}<{\hfill}}Source:& \multicolumn{2}{l}{\scriptsize Estimated with GUK administrative and survey data.}\\ Notes: & 1. & \textsf{R}'s package \textsf{coin} is used for baseline mean covariates to conduct approximate permutation tests. Number of repetition is set to 100000.\\& 2. &  See footnotes of \textsc{Table \ref{tab1 Permutation test results of rejection}}. \end{tabular}\end{minipage}\\\vspace{2ex}\hfil\begin{minipage}[t]{14cm}\hfil\textsc{\normalsize Table \refstepcounter{table}\thetable: Permutation test results of active members of cattle and large grace arms\label{tab1 Permutation test results of active members of cattle and large grace arms}}\\\setlength{\tabcolsep}{.5pt}\setlength{\baselineskip}{8pt}\renewcommand{\arraystretch}{.50}\hfil\begin{tikzpicture}\node (tbl) {\input{c:/data/GUK/analysis/save/PermutationTests/ActiveCowLargeGracePermutationTestResultso800.tex}};\end{tikzpicture}\\\begin{tabular}{>{\hfill\scriptsize}p{1cm}<{}>{\hfill\scriptsize}p{.25cm}<{}>{\scriptsize}p{12cm}<{\hfill}}Source:& \multicolumn{2}{l}{\scriptsize Estimated with GUK administrative and survey data.}\\ Notes: & 1. & \textsf{R}'s package \textsf{coin} is used for baseline mean covariates to conduct approximate permutation tests. Number of repetition is set to 100000.\\& 2. &  See footnotes of \textsc{Table \ref{tab1 Permutation test results of rejection among traditional arm}}. \end{tabular}\end{minipage}\\\vspace{2ex}\hfil\begin{minipage}[t]{14cm}\hfil\textsc{\normalsize Table \refstepcounter{table}\thetable: Permutation test results of non-attriting borrowers of cattle and all other arms\label{tab1 Permutation test results of non-attriting borrowers of cattle and all other arms}}\\\setlength{\tabcolsep}{.5pt}\setlength{\baselineskip}{8pt}\renewcommand{\arraystretch}{.50}\hfil\begin{tikzpicture}\node (tbl) {\input{c:/data/GUK/analysis/save/PermutationTests/ActiveCowNonCowPermutationTestResultso800.tex}};\end{tikzpicture}\\\begin{tabular}{>{\hfill\scriptsize}p{1cm}<{}>{\hfill\scriptsize}p{.25cm}<{}>{\scriptsize}p{12cm}<{\hfill}}Source:& \multicolumn{2}{l}{\scriptsize Estimated with GUK administrative and survey data.}\\ Notes: & 1. & \textsf{R}'s package \textsf{coin} is used for baseline mean covariates to conduct approximate permutation tests. Number of repetition is set to 100000.\\& 2. &  See footnotes of \textsc{Table \ref{tab1 Permutation test results of rejection among traditional arm}}. \end{tabular}\end{minipage}\\\vspace{2ex}
% uptake: 105+171+177+163
% rejected: rjcted <- 71+29+23+37
% attrited: 32+8+29+23
% borrower nonattriting (survivors): 83+164+160+147=83+164+160+147
% borrower attriting: boratt <- (105-83)+(171-164)+(177-160)+(163-147)
% nonborrower nonattriting: 61+28+11+30
% nonborrower attriting: (40+31-36-25)+(20+9-19-9)+(10+13-11)+(37-39)
% rejected or attrited (nonsurvivors): borrower attriting + rejected = boratt + rjcted


	\textsc{\normalsize Table \ref{tab1 Permutation test results of attrition}} shows results from tests of independence between attriters and nonattriters. Attrition is defined as attrition from household surveys, not from the lending program. We see the moderate rate of attrition is not correlated with household level characteristics%, except less risk torelance in \textsf{RiskPrefVal}, 
	at the conventional $p$ value level. Productive asset amounts seem to differ between attriters and nonattriters at $p=.105$, with the former being larger than the latter. This positive attrition selection can cause underestimation of impacts, if the asset values are positively correlated with entrepreneurial capacity. We also see that the attriters are less risk tolerant in terms of minimum expected payoff to choose a risky option in \textsf{RiskPrefVal}. \textsc{\normalsize Table \ref{tab1 Permutation test results of attrition among traditional arm}} shows attrition in the \textsf{traditional} arm. Household heads of attriters are relatively less literate than nonattriters. We observe the \textsf{traditional} arm attriters are less risk tolerant the nonattriters.
	\textsc{\normalsize Table \ref{tab1 Permutation test results of attrition among non-traditional arm}} compares attriters and nonattriters in the non-\textsf{traditional} arm. Unlike \textsf{traditional} arm attriters, non-\textsf{traditional} arm attriters have more literate household heads, have a larger household size, are more exposed to floods, and have larger productive assets. The \textsf{traditional} arm attriters may be less entrepreneurial, if anything, so their attrition may upwardly bias the positive gains of the arm, hence understate the impacts of non-\textsf{traditional} arm. These are explicitly shown in \textsc{\normalsize Table \ref{tab1 Permutation test results of attriters of traditional and non-traditional arms}} where we compare attriters of \textsf{traditional} and non-\textsf{traditional} arms. Overall, attrition may have attenuated the impacts but is not likely to have inflated them.\footnote{So one can employ the Lee bounds for stronger results, but doing so will give us less precision and require more assumptions. We will not use the Lee bounds \textcolor{red}{[we can show them if necessary]}. } We observe the non-\textsf{traditional} arm attriters are also less risk tolerant than the nonattriters.

	For the microfinance institutions (MFIs), attrition of the loan receiving members poses a threat to their business continuation. Financial institutions often use observable characteristics, such as collateralisable assets, and easily surveyed chracteristics, such as job experiences and schooling of borrowers, and are likely to lend if the assets levels are greater and the borrowers have relevant job experiences and more schooling. We first examine if such screening variables have any predictive power in terms of loan rejection or borrower attrition under our lending. \textsc{\normalsize Table \ref{tab1 Permutation test results of active status}} compares potential MFI targets (nonattriting borrowers, noted as \textsf{Active}) vs. non-targets (attriting borrowers or loan rejecters, noted as \textsf{NonSurvived}) in all arms. It shows potential targets at the baseline have larger values in livestock and greater number of cattle, and are less affected by the flood, which conforms the conventional wisdom of lenders in using these aspects in their loan decisions. We also see that more risk torelant members are likely to be borrowers and do not attrit. Next, we examine if the relationship of having ``less favourable'' values in these characteristics and attrition is mitigated under various loan characteristics. In \textsc{\normalsize Table \ref{tab1 Permutation test results of active members of cattle and large grace arms}}, we restrict our attention to the potential MFI targets, or the nonattriting borrowers, and compare between \textsf{cattle} and \textsf{large grace} arms, whose difference is efffectively the presence of managerial supports that the former provides. \label{PageOfAttrition}%Comparing the nonattriting borrower, characteristics are similar except that the \textsf{traditional} members are more exposed to the flood than the non-\textsf{traditional} members. 
	Comparing against the \textsf{large grace} arm, nonattriting borrowers of the \textsf{cattle} arm are more exposed to the flood ($p=.055$), have less productive assets ($p=.003$), have lower net asset values ($p=.046$), and have fewer livestock ($p=.139$). This shows that the smaller livestock holders or individuals with less experienced in livestock are encouraged to participate and continue to operate in the \textsf{cattle} arm that has a managerial support program, with all other features being equal. This is consistent with our analysis of participation in \textsc{\normalsize Table \ref{tab1 Permutation test results of borrowers, cattle vs. non-cattle arms}} which weakly hints that the \textsf{cattle} arm's managerial support programs may have encouraged participation of inexperienced or lower asset holders. This also underscores our interpretation that the current impact estimates may be downwardly biased, if any, as people who would otherwise attrit or reject in the \textsf{cattle} arm stayed on. This result is confirmed with lower $p$ values due to a larger sample size when we compare the nonattriting borrowers between \textsf{cattle} arm with all other arms in \textsc{\normalsize Table \ref{tab1 Permutation test results of active members of cattle and all other arms}}. At the baseline, \textsf{cattle} arm nonattriting borrowers have smaller baseline livestock holding ($p$ value = .016) and smaller baseline net asset holding ($p$ value = .007) than other arms' nonattriting borrowers. 

%	\textsc{\normalsize Table \ref{TabLabel1[1]}} shows results from tests of independence between attriters and nonattriters. We see the moderate rate of attrition is not correlated with household level characteristics at the conventional $p$ value level. Productive asset amounts seem to differ between attriters and nonattriters, with the former being larger than the latter. This positive attrition selection can cause underestimation of impacts, if the asset values are positively correlated with entrepreneurial capacity. \textsc{\normalsize Table \ref{TabLabel1[grep("of attri.* traditional arm$", TabLabel1)]}} shows attrition in the \textsf{traditional} arm. Household heads of attriters are relatively less literate than nonattriters. \textsc{\normalsize Table \ref{TabLabel1[grep("of attri.* non-traditional arm$", TabLabel1)]}} compares attriters and nonattriters in the non-\textsf{traditional} arm. Unlike \textsf{traditional} arm attriters, non-\textsf{traditional} arm attriters have more literate household heads, have a larger household size, are more exposed to floods, and have larger productive assets. The \textsf{traditional} arm attriters may be less entrepreneurial, if anything, so their attrition may upwardly bias the positive gains of the arm, hence understate the impacts of non-\textsf{traditional} arm. These are explicitly shown in \textsc{\normalsize Table \ref{TabLabel1[grep("of attri.* of", TabLabel1)]}} where we compare attriters of \textsf{traditional} and non-\textsf{traditional} arms. Overall, attrition may have attenuated the impacts but is not likely to have inflated them.\footnote{So one can employ the Lee bounds for stronger results, but doing so will give us less precision and require more assumptions. We will not use the Lee bounds \textcolor{red}{[we can show them if necessary]}. }

%	For the microfinance institutions (MFIs), attrition of the loan receiving members poses a threat to their business continuation. Financial institutions often use observable characteristics, such as collateralisable assets, and easily surveyed chracteristics, such as job experiences and schooling of borrowers, and are likely to lend if the assets levels are greater and the borrowers have relevant job experiences and more schooling. We first examine if such screening variables have any predictive power in terms of loan rejection or borrower attrition under our lending. \textsc{\normalsize Table \ref{TabLabel1[grep("survival$", TabLabel1)]}} compares potential MFI targets (nonattriting borrowers, noted as \textsf{Survived}) vs. non-targets (attriting borrowers or loan rejecters, noted as \textsf{NonSurvived}) in all arms. It shows potential targets at the baseline have larger values in livestock and greater number of cattle, and are less affected by the flood, which conforms the conventional wisdom of lenders in using these aspects in their loan decisions. Next, we examine if the relationship of having ``less favourable'' values in these characteristics and attrition is mitigated under various loan characteristics. In \textsc{\normalsize Table \ref{gsub("surviving members", "non-attriting borrowers", TabLabel1[grep("surviving members of cattle and large grace", TabLabel1)])}}, we restrict our attention to the potential MFI targets, or the nonattriting borrowers, and compare between \textsf{cattle} and \textsf{large grace} arms, whose difference is efffectively the presence of managerial supports that the former provides. \label{PageOfAttrition}%Comparing the nonattriting borrower, characteristics are similar except that the \textsf{traditional} members are more exposed to the flood than the non-\textsf{traditional} members. 
%	Comparing against the \textsf{large grace} arm, nonattriting borrowers of the \textsf{cattle} arm are more exposed to the flood ($p=.055$), have less productive assets ($p=.003$), have lower net asset values ($p=.046$), and have fewer livestock ($p=.137$). This shows that the smaller livestock holders or less experienced individuals are encouraged to participate and continue to operate in the \textsf{cattle} arm that has a managerial support program, with all other features being equal. This is consistent with our analysis of participation in \textsc{\normalsize Table \ref{TabLabel1[grep("bo.*non-ca", TabLabel1)]}} which weakly hints that the \textsf{cattle} arm's managerial support programs may have encouraged participation of inexperienced or lower asset holders. This also underscores our interpretation that the current impact estimates may be downward biased, if any, as people who would otherwise attrit or reject in cattle arm stayed on. This result is confirmed with lower $p$ values due to a larger sample size when we compare the nonattriting borrowers between \textsf{cattle} arm with other arms in \textsc{\normalsize Table \ref{gsub("surviving members", "non-attriting borrowers", TabLabel1[grep("surviving members of cattle and all", TabLabel1)])}}. At the baseline, \textsf{cattle} arm nonattriting borrowers have smaller baseline livestock holding ($p$ value = .016) and smaller baseline net asset holding ($p$ value = .007) than other arms' nonattriting borrowers. 

%	Moved from main text.
	
%	Group rejecters of \textsf{traditional} and non-\textsf{traditional} arms differ in household characteristics. Lower livestock values, smaller cattle holding, and smaller net asset values are associated with group rejection for \textsf{traditional} arm (\textsc{\normalsize Table \ref{TabLabel1[grep("p rej.* traditional arm$", TabLabel1)]}}), while higher baseline flood exposure rates and younger household heads are associated with group rejection for non-\textsf{traditional} arms (\textsc{\normalsize Table \ref{TabLabel1[grep("p rej.*g non-traditional arm$", TabLabel1)]}}). Given randomisation, we conjecture that it is lack of \textsf{Upfront} liquidity that prevented smaller livestock holders of \textsf{traditional} arm from participating because they cannot purchase cattle due to insufficient net asset values or an insufficient resale value of owned livestock, when members of similar characteristics partcipated in non-\textsf{traditional} arms. %This is a real resource constraint that binds the households. This is different from a psychological constraint that, so long as there is a cost or a payment involved, albeit at a minimal level, there remains a group of households who would not take up the investment \citep{Ashraf2010, CohenDupas2010}. 
%	For non-\textsf{traditional} arm rejecters, it is the past flood that kept members from participating, even they are younger and have similar cattle holding as the non group-rejecters. 

%	Individual rejecters of \textsf{traditional} arm and non-\textsf{traditional} arms share similar characteristics (\textsc{\normalsize Table \ref{TabLabel1[grep("l rej.*vs", TabLabel1)]}}). In fact, they are not very different in all the variables considered. %This is consistent with the conjecture that, had the \textsf{traditional} arm group rejecters been offered any of the non-\textsf{traditional} arms, they, as a group, may have accepted it.  %It shows the latter is more exposed to flood in baseline and has larger livestock values. This implies that, once large enough sum of loan is disbursed, %there is no minimum livestock and asset holding level to partake in the larger loans, and 
	%despite a negative asset shock in flood and a poverty trap at this level may be overcome once household size and negative asset shocks are accounted for.
%	The common factors associated with nonparticipation are a smaller household size and smaller livestock holding (\textsc{\normalsize Table \ref{TabLabel1[grep("l rej.*g tra", TabLabel1)]}} and \textsc{\normalsize Table \ref{TabLabel1[grep("l rej.*g non-", TabLabel1)]}}), although the $p$ values for livestock holding difference between individual rejecters and non individual rejecters are around 7\% (\textsc{\normalsize Table \ref{TabLabel1[grep("l rejection$", TabLabel1)]}}). %In non-\textsf{traditional} arms, the individual rejecters have only marginally different mean values relative to individual nonrejecters (\textsc{\normalsize Table \ref{tab Ireject nontrad perm}}). 



\end{document}
