% library(knitr); path0 <- "c:/data/GUK/"; path <- paste0(path0, "analysis/"); setwd(pathprogram <- paste0(path, "program/")); pathsource.mar <- paste0(path, "source/mar/"); pathreceived.nar <- paste0(path0, "received/mar/")
%  library(knitr); path0 <- "c:/data/GUK/"; path <- paste0(path0, "analysis/"); setwd(pathprogram <- paste0(path, "program/")); pathsource.mar <- paste0(path, "source/mar/"); pathreceived.nar <- paste0(path0, "received/mar/"); system("recycle c:/data/GUK/analysis/program/cache/ier_slides/"); library(knitr); knit("ier_slides.rnw", "ier_slides.tex"); system("platex ier_slides"); system("pbibtex ier_slides"); system("dvipdfmx ier_slides")


%\input{c:/data/knitr_beamer_preamble.rnw}
\input{c:/data/knitr_beamer_preamble_ho.rnw}

\newcounter{armindex}
\newcommand{\mpagec}[2]{\begin{minipage}[c]{#1}#2\end{minipage}}

\begin{document}
\setlength{\baselineskip}{12pt}






\title[MF ultra poor]{\large An assessment of microfinance interventions on the ultra poor}
\author[Ito, Kurosaki, Shonchoy, Takahashi]{Seiro Ito, Takashi Kurosaki, Abu Shonchoy, Kazushi Takahashi}
\institute[IDE]{Institute of Developing Economies, Hitotsubashi University, Institute of Developing Economies, Sophia University}
\date[March 15, 2017]{March 15, 2017\\\vspace{1ex} IER, Hitotsubashi University}
\logo{IDE, IER, SU}


\frame{\titlepage}
\setcounter{page}{1}

\setbeamercovered{transparent = 0}
%\setbeamercolor{normal text}{fg=turquoiseblue2, bg=}
%\setbeamercolor{alerted text}{fg=black, bg=}
%\usebeamercolor{normal text}

\begin{frame}[label = background]{}
Motivation 1: Costly, risky, and stressful?
\begin{itemize}
\vspace{1.0ex}\setlength{\itemsep}{3.0ex}\setlength{\baselineskip}{12pt}
\pause
\item	Why is there so few evidence on positive microfinance impacts?
\pause
\item	Why is the outreach of MF limited to the relatively wealthy?
\pause
\item	Supply: Capital constraint is only one of the constraints.
\pause
\item	Demand: Managerial ability? \pause Not all of us are entrepreneurs.
\end{itemize}
\end{frame}

\begin{frame}[label = background2]{}
Motivation 2: Adult trainability?
\begin{itemize}
\vspace{1.0ex}\setlength{\itemsep}{3.0ex}\setlength{\baselineskip}{12pt}
\pause
\item	Some argued that cognitive skills interventions must take place early in life.
\pause
\item	May be true in high income countries where a school system is efficient that leaves no low hanging fruit.
\pause
\item	In low income countries, schools are inefficient that can leave some room for interventions in later life.
\pause
\item	Mixed evidence on adult training impacts. \pause Can we supplement elementary managerial capacity with a tweak in program archtecture?
\end{itemize}
\end{frame}

\begin{frame}{}
Related literature: 
\begin{itemize}
\vspace{1.0ex}\setlength{\itemsep}{1.0ex}\setlength{\baselineskip}{12pt}
\item	MF is not successful in reaching out to the poorest of the poor, or the ultra poor \citep[][]{Scully2004}. Empirical evidence in \citet{Yaron1994, Navajas2000, RahmanRazzaque2000, AghionMorduch2007} supports this claim. Some authors discuss the tradeoff between sustainability and outreach for microfinance institutions (MFIs) \citet{HermesLensink2011, HermesLensinkMeesters2011, Cull2011}. 
\item	There is a growing interest on managerial impacts on firm growth in LDCs \citep[][]{deMel2008, Calderon2011, KarlanValvidia2011, Berge2012, Bruhn2012, MckenzieWoodruff2012, Bloometal2013}. Evidence on profits is mixed: \citet{Calderon2011, Berge2012, Bloometal2013} report effectiveness, while \citet{KarlanValvidia2011, Bruhn2012, KarlanKnightUdry2012} conclude that they are ineffective. 
\item	\citet{MckenzieWoodruff2012}: These studies are too different to compare, in terms of population, interventions, measurement (variables, timing), and most importantly, implied statistical power in the design. 
\end{itemize}
\end{frame}


\begin{frame}{}
We conjecture that the previous studies might have a too limited time frame, a too wide variety of investment options, and a lesser degree of control in isulating from outside options.

\vspace{2ex}
\pause
We want:
\begin{itemize}
\vspace{1.0ex}\setlength{\itemsep}{1.0ex}\setlength{\baselineskip}{12pt}
\pause
\item	A long enough follow up period.
\pause
\item	Less heterogeneity, more statistical power.
\pause
\item	Less uncontrolled treatments.
\end{itemize}
\vspace{2ex}
\pause
NGOs and CBOs: Livestock (usually calf) is the most popular and plausibly the only viable investment option. 

\vspace{2ex}
\pause
Our study area is ideal for impact evaluation:
\begin{itemize}
\vspace{1.0ex}\setlength{\itemsep}{1.0ex}\setlength{\baselineskip}{12pt}
\pause
\item	A 4-year (unexpectedly extended to 5-year) project.
\pause
\item	A smaller choice set.
\pause
\item	Little or no MFI/NGOs presence.
\end{itemize} 
\end{frame}

\begin{frame}{}
Theretical considerations in related literature: Management as a technology
\begin{itemize}
\vspace{1.0ex}\setlength{\itemsep}{1.0ex}\setlength{\baselineskip}{12pt}
\pause
\item	Bloom et al (2016): theory and global evidence supporting the theory. RCTs to improve management skills of micro entrepreneurs, resulting in higher productivity or adoption of good management practices.
\pause
\item	Bloom et al. (2016 QJE): inventory control and factory information training in India. Higuchi et al. (2015 JEBO): KAIZEN training in Vietnam. More KAIZEN RCTs in Africa by Sonobe and his co.
\end{itemize}
\pause
(Less related) Behavioral foundation of entrepreneurship
\begin{itemize}
\vspace{1.0ex}\setlength{\itemsep}{1.0ex}\setlength{\baselineskip}{12pt}
\item	Astebro et al. (2014): review paper. Fafchamps and Quinn (forthcoming): random assignment of peers to entrepreneurs in Ghana. Kremer et al. (2013, AER P\&P): behavior biases' impact on firm performance in Kenyan retail shops; using risk games. Fehr and List (2004, JEEA): Trust games to CEOs in Costa Rica
\item	Reminder as a tool to overcome limited attention issues: Several RCTs in the context of savings (e.g., Karlan et al. 2016; Kast et al. 2012). On-going RCT on micro entrepreneurs in India by Goto et al. (2017)
\end{itemize}
\end{frame}

\begin{frame}{}
We want to measure the impacts of managerial capacity in MF uptake and outcomes.\\~\\
\pause
But managerial capacity is unobservable.\\~\\
\pause
We design the study to allow us to infer if managerial capacity matters for successes.\\~\\
\pause
We make the managerial capacity useless by offering a credit under the environment which requires a minimal level of managerial capacity.\\~\\
\pause
Success rates in this ``easy'' credit $\simeq$ success rates in a comparable credit\\
\pause
$\Rightarrow$ Managerial capacity does not causally affect MF success \pause (or our provision of help was hoplessly useless that requires managerial capacity in ``easy'' credit)\\~\\

``Easy'': Provide a calf, supplementary services, and training.
\end{frame}



\begin{frame}{}
A 4$\times$4 factorial, stepped wedge design with placebo controls\\

\vspace{2ex}
\pause
\hfil\begin{tabular}{>{\footnotesize\hfill}p{2cm}<{}
>{\footnotesize\hfil}p{2cm}<{}
>{\footnotesize\hfil}p{2cm}<{}
>{\footnotesize\hfil}p{2cm}<{}}
					& \cellcolor{paleblue}large, grace 			& \cellcolor{paleblue}large & \cellcolor{paleblue}small \\\cellcolor{paleblue}
cow 				& \mpage{2cm}{\hfil managerial\\\hfil capacity} &\cellcolor{gray50} &\cellcolor{gray50}\\\cellcolor{paleblue}
large, grace &\cellcolor{gray50} 	&  \mpage{2cm}{\hfil saving\\\hfil constraint} & \cellcolor{gray50}\\\cellcolor{paleblue}
large 			&\cellcolor{gray50} 	&\cellcolor{gray50}& \mpage{2cm}{\hfil convex\\\hfil technology}\\\cellcolor{pink}
control & \multicolumn{3}{c}{\cellcolor{green}level \hspace{1em} impacts}
\end{tabular}

\pause
\vspace{2ex}
We will report the ``level impacts'' as a linear function of treatment dose difference $t_{1}-t_{0}$ between the control and the treated. 
\[
\mbox{impacts}=\E[y|\mbox{treated}]-\E[y|\mbox{control}]=\E[y|t_{1}]-\E[y|t_{0}]=b(t_{1}-t_{0}).
\]
\end{frame}

\begin{frame}{}

\vspace{-2ex}
\begin{adjustbox}{max size={1.0\textwidth}{1.2\textheight}}
\hspace{-1cm}% Set the overall layout of the tree
\tikzstyle{level 1}=[level distance = 1.75cm, sibling distance = 2.1cm]
\tikzstyle{level 2}=[level distance = 3.00cm, sibling distance = 2.0cm]
\tikzstyle{level 3}=[level distance = 1.5cm, sibling distance = 0.85cm]
\tikzstyle{level 4}=[level distance = 2.5cm, sibling distance = 1cm]
\tikzstyle{level 5}=[level distance = 3cm, sibling distance = 2cm]
\tikzstyle{level 6}=[level distance = 3cm, sibling distance = 2cm]
\tikzstyle{level 7}=[level distance = 3cm, sibling distance = 2cm]
\tikzstyle{level 8}=[level distance = 2.25cm, sibling distance = 2cm]
% Define styles for bags and leafs
\tikzstyle{bag} = [text width = 5em, text centered, thick]
\tikzstyle{testbag} = [minimum width=2.5em, text centered, thick, top color = darkgreen, bottom color = green]
\tikzstyle{Hbbag} = [text width=3em, text centered, thick, top color = blue!40, bottom color = blue!40, color = black] 
\tikzstyle{govbag} = [minimum width=2.5em, text centered, thick, top color = gray!30, bottom color = gray!10, color = black]
%\tikzstyle{end} = [circle, minimum width = 3pt, fill, inner sep = 0pt]
\tikzstyle{end} = [text width = 4em, text centered, thick]
\tikzstyle{endbox} = [draw, very thick, fill = white, rectangle, 
	minimum height = 3em, minimum width=7em, 
	node distance = 8em, font = {\sffamily\bfseries}]
\tikzstyle{endboxEdgePortion} = [black, thick]
\tikzstyle{endboxEdge} = [endboxEdgePortion, ->]
\tikzstyle{edgeLabel} = [pos = 0.5, text centered, font = {\sffamily\small}]
\tikzstyle{bracebag} = [decoration = {brace}, decorate]
\tikzstyle{surveybag} = [edgeLabel, xshift = 0em, shading = ball,
				top color = green, bottom color = green, color = black]
\tikzstyle{nobackground} = [text centered, thick, top color =yellow!30, 
				bottom color = yellow!30, opacity = 50, color = black]
% The sloped option gives rotated edge labels. Personally
% I find sloped labels a bit difficult to read. Remove the sloped options
% to get horizontal labels. 
\hspace{-2cm}\begin{tikzpicture}
[	grow = right, sloped, 
%	every node/.style={top color = blue, bottom color = blue!30, rounded corners, text = white}
	every node/.style = {shading = ball, rounded corners, text = white},
	%background rectangle/.style = {fill = gray, opacity = .2}, 
	%framed
	decoration={brace, amplitude = 2pt},
]

\node (randomize) [testbag] {randomize}
child 
{
	%	start packaged loans arm
	node (packaged) {}
		child
		{
			node (ps) {}
				child 
				{
					node (pc1) [Hbbag] {control}
					child { 
							node (pc2) {}
								child { node (pc3) {}
									child { node (pcend) {} }
									}
						}
				}
				child 
				{
					node (pt1) [Hbbag] {cows}
					child { 
							node (pt2) {}
								child { node (pt3) {}
									child { node (ptend) {} }
									}
						}
				}
			edge from parent
		}
	edge from parent
}
child 
{
	%	start large loans with grace arm
	node (grace) {}
		child 
		{
			node (gs) {}
				child 
				{
					node (gc1) [Hbbag] {control}
					child { 
							node (gc2) {}
								child { node (gc3) {}
									child { node (gcend) {} }
									}
						}
				}
				child 
				{
					node (gt1) [Hbbag] {large}
					child { 
							node (gt2) {}
								child { node (gt3) {}
									child { node (gtend) {} }
									}
						}
				}
			edge from parent
		}
	edge from parent
}
child 
{
	%	start large loans arm
	node (large) {}
		child 
		{
			node (ls) {}
				child 
				{
					node (lc1) [Hbbag] {control}
					child { 
							node (lc2) {}
								child { node (lc3) {}
									child { node (lcend) {} }
									}
						}
				}
				child 
				{
					node (lt1) [Hbbag] {large}
					child { 
							node (lt2) {}
								child { node (lt3) {}
									child { node (ltend) {} }
									}
						}
				}
			edge from parent
		}
	edge from parent
}
child 
{
	%	start small loans arm
	node (traditional) {}
		child 
		{
			node (ts) {}
				child 
				{
					node (tc1) [Hbbag] {control}
					child { 
							node (tc2) {}
								child { node (tc3) {}
									child { node (tcend) {} }
									}
						}
				}
				child 
				{
					node (tt1) [Hbbag] {small}
					child { 
							node (tt2) {}
								child { node (tt3) {}
									child { node (ttend) {} }
									}
						}
				}
			edge from parent
		}
	edge from parent
}
;

% time line
%\node (calendar1) at ($(baseline) + (0, -5cm)$){};
%\node (calendar2) at ($(f3) + (0, -5cm)$){};
%\draw (calendar1) -- (calendar2);
\setcounter{armindex}{2011}
\def\ytimeline{-5.35}
\def\ytimelinetwo{1}
\def\ytimelinethree{8}
\draw [thick] (1cm, \ytimeline cm) -- (15cm, \ytimeline cm);
\foreach \x in {1, 7, 10.35, 13.35}
{
	\stepcounter{armindex}
	\draw [thick] (\x cm, \ytimeline) -- ($(\x cm, \ytimeline) + (0, .5cm)$);
	\draw ($(\x cm, \ytimeline) + (0, .75cm)$) node [text centered] {\thearmindex};
	\ifthenelse{\equal{\thearmindex}{12}}{\setcounter{armindex}{0}}{};
	\draw [thick] (\x cm, \ytimeline) -- ($(\x cm, \ytimeline) + (0, .25cm)$);
};

%  target
\node (peri) at ($(randomize) + (0, 2cm)$)
	{\hfil\begin{minipage}[t]{1.75cm}
	\hfil villages\\
	\hfil on chars\setlength{\baselineskip}{12pt}
	\end{minipage}};
\draw[->, >= latex', thick] (peri.south) -- (randomize.north);

%	%  curly braced texts
%		%	mirror: turn brace to opposite side
%	\path (cp1.west |- cn.north)+(0, .5ex) node (phase1topleft) {};
%	\path (tp2.east |- cn.north)+(0, .5ex) node (phase1topright) {};
%	\path (tp4.west |- cn.north)+(0, .5ex) node (phase2topleft) {};
%	\path (tp5.east |- cn.north)+(0, .5ex) node (phase2topright) {};
%	\draw[bracebag] let \p1=(phase1topleft), \p2=(phase1topright) in
%		($(\x1, \y1)$) -- ($(\x2, \y1)$) 
%		node (ph1) [midway, govbag, above = 2pt] {Phase 1};
%	\draw[bracebag] let \p1=(phase2topleft), \p2=(phase2topright) in
%		($(\x1, \y1)$) -- ($(\x2, \y1)$) 
%		node (ph2) [midway, govbag, above = 2pt] {Phase 2};

% composition of arms
\pause;
\node (armcomp)[govbag] at ($(tt1.north |- tt1.east) + (10em, 10ex)$)
	{\mpage{3.5cm}{7 ultra poor\hfill\\ 3 moderately poor\hfill\setlength{\baselineskip}{10pt}}\mpage{2cm}{per village}};
\draw (armcomp.west) edge[out = 180, in = 90, ->, >= latex', thick] 
	node[midway, above, nobackground] {} (tt1.north);
\pause;
\node (armsize)[govbag] at ($(tt1.north |- tt1.east) + (-10em, 10ex)$)
	{\mpage{3.75cm}{20 HHs$\times$20 villages\hfill\\ = 400 HHs\hfill\setlength{\baselineskip}{10pt}} \mpage{2cm}{per arm}};
\draw (armsize.east) edge[out = 0, in = 90, ->, >= latex', thick] 
	node[midway, above, nobackground] {} (tt1.north);
%\node (armcomp)[govbag] at ($(tt1.south) + (1em, -3ex)$){400 HHs};
%\node (armcomp)[govbag] at ($(gt1.south) + (1em, -3ex)$){400 HHs};
%\node (armcomp)[govbag] at ($(lt1.south) + (1em, -3ex)$){400 HHs};
%\node (armcomp)[govbag] at ($(pt1.south) + (1em, -3ex)$){400 HHs};

%	breakaway control ==> treated
\pause;
\draw (tc1.east) edge[out = 0, in = 180, ->, >= latex', thick] 
	node[midway, above, nobackground] {} (tt2.west);
\pause;
\draw (tc2.east) edge[out = 0, in = 180, ->, >= latex', thick] 
	node[midway, above, nobackground] {} (tt3.west);
\pause;
\draw (lc1.east) edge[out = 0, in = 180, ->, >= latex', thick] 
	node[midway, above, nobackground] {} (lt2.west);
\pause;
\draw (lc2.east) edge[out = 0, in = 180, ->, >= latex', thick] 
	node[midway, above, nobackground] {} (lt3.west);
\pause;
\draw (gc1.east) edge[out = 0, in = 180, ->, >= latex', dotted] 
	node[midway, above, nobackground] {} (gt2.west);
\pause;
\draw (gc2.east) edge[out = 0, in = 180, ->, >= latex', thick] 
	node[midway, above, nobackground] {} (gt3.west);
\pause;
\draw (pc1.east) edge[out = 0, in = 180, ->, >= latex', thick] 
	node[midway, above, nobackground] {} (pt2.west);
\pause;
\draw (pc2.east) edge[out = 0, in = 180, ->, >= latex', thick] 
	node[midway, above, nobackground] {} (pt3.west);

%	common events
\pause;
\draw[->, >= latex', thick, dashed]
	($(large) + (3ex, -12.0ex)$)
	edge[endboxEdge] node (listing) [edgeLabel, xshift = 0em, shading = ball,
				top color = orange, bottom color = orange, color = black]
		{\mpage{9cm}{\hfil household census and gradation}} 
	($(large) + (3ex, 3.5ex)$);
\pause;
\draw[->, >= latex', thick, dashed]
	($(large) + (7.5ex, -12.0ex)$)
	edge[endboxEdge] node (baseline) [surveybag]
		{\mpage{9cm}{\hfil baseline survey}} 
	($(large) + (7.5ex, 3.5ex)$);
\pause;
\draw[->, >= latex', thick, dashed]
	($(large) + (12ex, -12.0ex)$)
	edge[endboxEdge] node (selection) [edgeLabel, xshift = 0em, shading = ball,
				top color = purple, bottom color = purple]
		{\mpage{9cm}{\hfil randomized selection}} 
	($(large) + (12ex, 3.5ex)$);
\pause;
\draw[->, >= latex', thick, dashed]
	($(large) + (16ex, -12.0ex)$)
	edge[endboxEdge] node (training) [edgeLabel, xshift = 0em, shading = ball,
				top color = darkpurple, bottom color = purple]
		{\mpage{9cm}{\hfil training}} 
	($(large) + (16ex, 3.5ex)$);
%	\draw[->, >= latex', thick, dashed]
%		($(large) + (30ex, -0.5ex)$)
%		edge[endboxEdge] node (selection) [edgeLabel, xshift = 0em, shading = ball,
%					top color = green, bottom color = green]
%			{\mpage{8cm}{\hfil follow up survey 1}} 
%		($(large) + (30ex, 1.5ex)$);

% traditinal loans arm
\pause;
\draw[fill = red, opacity = .15] 
	($(tt1) + (-1cm, -1.4cm)$) rectangle 
	($(ttend) + (.5cm, .75cm)$);
\pause;
\node (trepay1) at ($(tt1.north |- tt1.east) + (3.5em, -2ex)$){repay};
\pause;
\node (trad1) at ($(tt2.north |- tt2.east) + (3.5em, 2ex)$){loan 2};
\pause;
\node (trepay2) at ($(tt2.north |- tt2.east) + (3.5em, -2ex)$){repay};
\pause;
\node (trad2) at ($(tt3.north |- tt3.east) + (3.5em, 2ex)$){loan 3};
\pause;
\node (trepay3) at ($(tt3.north |- tt3.east) + (3.5em, -2ex)$){repay};
% large loans arm
\pause;
\draw[fill = green, opacity = .15] 
	($(lt1) + (-1cm, -1.4cm)$) rectangle 
	($(ltend) + (.5cm, .6cm)$);
\pause;
\node (lrepay1) at ($(lt1.north |- lt1.east) + (3.5em, -2ex)$){repay};
\pause;
\node (lrepay2) at ($(lt2.north |- lt2.east) + (3.5em, -2ex)$){repay};
\pause;
\node (lrepay3) at ($(lt3.north |- lt3.east) + (3.5em, -2ex)$){repay};
% large loans with grace arm
\pause;
\draw[fill = red, opacity = .15] 
	($(gt1) + (-1cm, -1.4cm)$) rectangle 
	($(gtend) + (.5cm, .6cm)$);
\pause;
\node (lrepay2) at ($(gt2.north |- gt2.east) + (3.5em, -2ex)$){repay};
\pause;
\node (lrepay3) at ($(gt3.north |- gt3.east) + (3.5em, -2ex)$){repay};
% packaged loans arm
\pause;
\draw[fill = yellow, opacity = .15] 
	($(pt1) + (-1cm, -1.4cm)$) rectangle 
	($(ptend) + (.5cm, .6cm)$);
\pause;
\node (prepay2) at ($(pt2.north |- pt2.east) + (3.5em, -2ex)$){repay};
\pause;
\node (prepay3) at ($(pt3.north |- pt3.east) + (3.5em, -2ex)$){repay};

% surveys
\pause;
\draw[->, >= latex', thick, dashed]
	($(large) + (7.0cm, -12.0ex)$)
	edge[endboxEdge] node (f1) [surveybag]
		{\mpage{9cm}{\hfil follow up survey 1}} 
	($(large) + (7.0cm, 3.5ex)$);
\pause;
\draw[->, >= latex', thick, dashed]
	($(large) + (10.0cm, -12.0ex)$)
	edge[endboxEdge] node (f2) [surveybag]
		{\mpage{9cm}{\hfil follow up survey 2}} 
	($(large) + (10.0cm, 3.5ex)$);
\pause;
\draw[->, >= latex', thick, dashed]
	($(large) + (13.0cm, -12.0ex)$)
	edge[endboxEdge] node (f3) [surveybag]
		{\mpage{9cm}{\hfil follow up survey 3}} 
	($(large) + (13.0cm, 3.5ex)$);

	% Draw the background
% \begin{pgfonlayer}{background}
% 	\path (peri.west |- peri.north)+(-0.5em, 8.5em) 
% 		node (topleft1) {};
% 	\path (pcend.east |- pcend.south)+(0.5cm, -4.5em) 
% 		node (bottomright1) {};
% 	\path[fill = yellow!30, rounded corners]
% 		(topleft1) rectangle (bottomright1);
% \end{pgfonlayer}

\end{tikzpicture}

\end{adjustbox}
\end{frame}


\begin{frame}{}
Disbursement\\
\hfil\includegraphics[width=12cm]{figure/compare_outcomesunnamed-chunk-36-1.eps}
\end{frame}

\begin{frame}{}
Stated use of credit\\
\hfil\includegraphics[width=12cm]{figure/compare_outcomesunnamed-chunk-29-1.eps} \\
\pause
As expected, livestock is the majority investments.
\end{frame}


\begin{frame}{}
Currently, we are collecting the final round data.\\~\\
\pause
I will report the intermediate results up to round 3.\\~\\

\pause
\textsf{control / treated}: Initial treatment assignment. All the control will be given a chance to be treated in our stepped wedge design.\\~\\

\pause
\textsf{credit}: Actual disbursement of credit.\\~\\

\pause
\textsf{elapsed days}: Number of days since receiving a credit. About 200 + day difference between \textsf{cotrol} and \textsf{treated} in a same cluster. We consider this as a measure of continous treatment (as in dose-response). \pause We have not exploited it yet.
\end{frame}

\begin{frame}{}
\hfil\textsc{\footnotesize Table \refstepcounter{table}\thetable: FD estimates of three meals per day, round 2, 3\label{FD3meals}}\\
\setlength{\tabcolsep}{1pt}
\renewcommand{\arraystretch}{.6}
\hfil\begin{tikzpicture}
\node (tbl) {\input{c:/data/GUK/analysis/save/3meals_slides.tex}};
\begin{pgfonlayer}{background}
\shade[toprow]
	($(tbl.north west)+(0.13, -0.8ex)$)
	rectangle ($(tbl.north east)-(0.13, 0.45)$);
%\draw[rounded corners, top color = white, bottom color = black,
%	middle color = red, draw = blue!20] 
%	($(tbl.south west)+(0.12, 0.5)$) 
%	rectangle ($(tbl.south east)-(0.12, 0)$);
\draw[maintable]
    ($(tbl.north east)-(0.13, .85)$)
    rectangle ($(tbl.south west)+(0.13, 0.1)$);
\end{pgfonlayer}

\end{tikzpicture}\\
\renewcommand{\arraystretch}{1}
\end{frame}

\begin{frame}{}
There may not be much of poverty alleviation impacts on the meal intake. \\~\\
\pause
\vspace{2ex}
This is contrary to our expectation, because the area is known for a hunger season called \textit{monga}.
\end{frame}

\begin{frame}{}
Nonfarm asset accumulation by arms: Within cluster comparisons\\
\hfil\includegraphics[width=12cm, height = 7.5cm]{figure/compare_outcomesunnamed-chunk-48-1.eps} \\

\pause
Almost zero gradient.
\end{frame}

\begin{frame}{}
There is no poverty alleviation impact on nonfarm asset accumulation. \\~\\
\pause
\vspace{2ex}
This was expected, because most of the investments were on livestock.
\end{frame}

\begin{frame}{}
Livestock accumulation: Within cluster comparisons\\
\hfil\includegraphics[width=12cm, height = 7.5cm]{figure/compare_outcomesunnamed-chunk-62-1.eps}\\
\pause
Zero difference at start but becomes increasingly more heterogenous in the later rounds.
\end{frame}

\begin{frame}{}
Total asset accumulation by arms: Within cluster comparisons\\
\hfil\includegraphics[width=12cm, height = 7.5cm]{figure/compare_outcomesunnamed-chunk-66-1.eps} \\
\pause
Zero difference at start but becomes increasingly more heterogenous in the later rounds. Dose-response is not monotonic.
\end{frame}

\begin{frame}{}
\hfil\textsc{\footnotesize Table \refstepcounter{table}\thetable: Descriptive statistics of asset regression data\label{destat.alr.sss}}\\
\setlength{\tabcolsep}{1pt}
\renewcommand{\arraystretch}{.6}
\hfil\begin{tikzpicture}
\node (tbl) {\input{c:/data/GUK/analysis/save/destat_alr_sss_slide.tex}};
\begin{pgfonlayer}{background}
\shade[toprow]
	($(tbl.north west)+(0.13, -0.8ex)$)
	rectangle ($(tbl.north east)-(0.13, 0.45)$);
%\draw[rounded corners, top color = white, bottom color = black,
%	middle color = red, draw = blue!20] 
%	($(tbl.south west)+(0.12, 0.5)$) 
%	rectangle ($(tbl.south east)-(0.12, 0)$);
\draw[maintable]
    ($(tbl.north east)-(0.13, .85)$)
    rectangle ($(tbl.south west)+(0.13, 0.1)$);
\end{pgfonlayer}

\end{tikzpicture}
\end{frame}

\begin{frame}{}
%\hspace{-1.5cm}\begin{minipage}[t]{9cm}
\hfil\textsc{\footnotesize Table \refstepcounter{table}\thetable: DID estimates of asset impacts\label{FDasset}}\\
\setlength{\tabcolsep}{1pt}
\renewcommand{\arraystretch}{.6}
\hfil\begin{tikzpicture}
\node (tbl) {\input{c:/data/GUK/analysis/save/asset_regression_alr_sss_slide.tex}};
\begin{pgfonlayer}{background}
\shade[toprow]
	($(tbl.north west)+(0.13, -0.8ex)$)
	rectangle ($(tbl.north east)-(0.13, 0.45)$);
%\draw[rounded corners, top color = white, bottom color = black,
%	middle color = red, draw = blue!20] 
%	($(tbl.south west)+(0.12, 0.5)$) 
%	rectangle ($(tbl.south east)-(0.12, 0)$);
\draw[maintable]
    ($(tbl.north east)-(0.13, .85)$)
    rectangle ($(tbl.south west)+(0.13, 0.1)$);
\end{pgfonlayer}

\end{tikzpicture}\\
\renewcommand{\arraystretch}{1}
%\end{minipage}
\end{frame}

\begin{frame}{}
There is no poverty alleviation impact on total asset accumulation. \\~\\
\pause
\vspace{2ex}
This was not expected, because most of the investments were on livestock.
\end{frame}

\begin{frame}{}
\hfil\textsc{\footnotesize Table \refstepcounter{table}\thetable: DID estimates of livestock impacts\label{FDlivestock}}\\
\setlength{\tabcolsep}{1pt}
\renewcommand{\arraystretch}{.6}
\hfil\begin{tikzpicture}
\node (tbl) {\input{c:/data/GUK/analysis/save/livestock_regression_alr_sss_slide.tex}};
\begin{pgfonlayer}{background}
\shade[toprow]
	($(tbl.north west)+(0.13, -0.8ex)$)
	rectangle ($(tbl.north east)-(0.13, 0.45)$);
%\draw[rounded corners, top color = white, bottom color = black,
%	middle color = red, draw = blue!20] 
%	($(tbl.south west)+(0.12, 0.5)$) 
%	rectangle ($(tbl.south east)-(0.12, 0)$);
\draw[maintable]
    ($(tbl.north east)-(0.13, .85)$)
    rectangle ($(tbl.south west)+(0.13, 0.1)$);
\end{pgfonlayer}

\end{tikzpicture}
\end{frame}

\begin{frame}{}
There is some poverty alleviation impact on livestock asset accumulation for educated male headed households. \\~\\
\pause
\vspace{2ex}
It shows livestock accumulation continued for the two years (up to 2015) after we started to disbursing loans in 2013.
\end{frame}

\begin{frame}{}
We find almost zero impact of loan disbursement when compared with the late loan takes in the same cluster. \\~\\
\pause
This might have been expected to some people because one calf/cow with repayment would not drastically change the asset position of the UP.\\~\\
\pause
It was disappointingly surprising to me, nonetheless.\\~\\
\pause
We will need to:
\begin{itemize}
\vspace{1.0ex}\setlength{\itemsep}{1.0ex}\setlength{\baselineskip}{12pt}
\item	Consider impacts on flows, not stocks.
\item	Incorporate generalised propensity score for continuous treatments \citet{Imbens2000, HiranoImbens2004, ImaiVanDyk2004, Egger2013}.
\item	Make use of factorial design to test the main hypothesis.
\end{itemize}
\end{frame}

\begin{frame}{}
\vspace{3ex}
\hfil{\Large Thank you very much.}
\vspace{3ex}
I would like to thank my coauthors and:
\begin{itemize}
\vspace{1.0ex}\setlength{\itemsep}{1.0ex}\setlength{\baselineskip}{12pt}
\item	People at GUK.
\item	Our RAs at MOMODa Foundation.
\item	JSPF funding.
\item	Hitotsubashi Univeristy, IER funding. Okayasu-san, Aoki-san.
\item	IDE funding.
\end{itemize}

\end{frame}

\begin{frame}[allowframebreaks]{}
\footnotesize\bibliographystyle{aer}
\setlength{\baselineskip}{10pt}
\bibliography{c:/dropbox/docs/notes/seiro}
\end{frame}

\end{document}

