% path0 <- "c:/data/GUK/"; path <- paste0(path0, "analysis/"); setwd(pathprogram <- paste0(path, "program/")); pathsource.mar <- paste0(path, "source/mar/"); pathreceived.mar <- paste0(path0, "received/mar/")
%  path0 <- "c:/data/GUK/"; path <- paste0(path0, "analysis/"); setwd(pathprogram <- paste0(path, "program/")); pathsource.mar <- paste0(path, "source/mar/"); pathreceived.mar <- paste0(path0, "received/mar/"); library(knitr); knit("ImpactEstimationMembership1or4.rnw", "ImpactEstimationMembership1or4.tex"); system("platex ImpactEstimationMembership1or4"); system("dvipdfmx ImpactEstimationMembership1or4")
%   path0 <- "c:/data/GUK/"; path <- paste0(path0, "analysis/"); setwd(pathprogram <- paste0(path, "program/")); system("recycle c:/data/GUK/analysis/program/cache/ImpactEstimationMembership1or4/"); library(knitr); knit("ImpactEstimationMembership1or4.rnw", "ImpactEstimationMembership1or4.tex"); system("platex ImpactEstimationMembership1or4"); system("dvipdfmx ImpactEstimationMembership1or4")

\input{c:/data/knitr_preamble.rnw}
\renewcommand\Routcolor{\color{gray30}}
\newtheorem{finding}{Finding}[section]
\makeatletter
\g@addto@macro{\UrlBreaks}{\UrlOrds}
\newcommand\gobblepars{%
    \@ifnextchar\par%
        {\expandafter\gobblepars\@gobble}%
        {}}
\newenvironment{lightgrayleftbar}{%
  \def\FrameCommand{\textcolor{lightgray}{\vrule width 1zw} \hspace{10pt}}% 
  \MakeFramed {\advance\hsize-\width \FrameRestore}}%
{\endMakeFramed}
\newenvironment{palepinkleftbar}{%
  \def\FrameCommand{\textcolor{palepink}{\vrule width 1zw} \hspace{10pt}}% 
  \MakeFramed {\advance\hsize-\width \FrameRestore}}%
{\endMakeFramed}
\makeatother
\usepackage{caption}
\usepackage{setspace}
\usepackage{framed}
\captionsetup[figure]{font={stretch=.6}} 
\def\pgfsysdriver{pgfsys-dvipdfm.def}
\usepackage{tikz}
\usetikzlibrary{calc, arrows, decorations, decorations.pathreplacing, backgrounds}
\usepackage{adjustbox}
\tikzstyle{toprow} =
[
top color = gray!20, bottom color = gray!50, thick
]
\tikzstyle{maintable} =
[
top color = blue!1, bottom color = blue!20, draw = white
%top color = green!1, bottom color = green!20, draw = white
]
\tikzset{
%Define standard arrow tip
>=stealth',
%Define style for different line styles
help lines/.style={dashed, thick},
axis/.style={<->},
important line/.style={thick},
connection/.style={thick, dotted},
}


\begin{document}
\setlength{\baselineskip}{12pt}








%  path0 <- "c:/data/GUK/"; path <- paste0(path0, "analysis/"); setwd(pathprogram <- paste0(path, "program/")); system("recycle c:/data/GUK/analysis/program/cache/ImpactEstimationMembership1or4/"); library(knitr); knit("ImpactEstimationMembership1or4.rnw", "ImpactEstimationMembership1or4.tex"); system("platex ImpactEstimationMembership1or4"); system("dvipdfmx ImpactEstimationMembership1or4")

\hfil Estimating lending impacts using \textsf{membership} == 1, 4\\

\hfil\MonthDY\\
\hfil{\footnotesize\currenttime}\\

\hfil Seiro Ito

\setcounter{tocdepth}{3}
\tableofcontents
\newpage

\setlength{\parindent}{1em}
\vspace{2ex}


\begin{Schunk}
\begin{Soutput}
File not found.
\end{Soutput}
\end{Schunk}


\section{Summary}

\begin{description}
\vspace{1.0ex}\setlength{\itemsep}{1.0ex}\setlength{\baselineskip}{12pt}
\item[Schooling]	Increased in \textsf{Cow} arms for girls in rd 1 vs rd 4 comparisons.
\item[Net saving and repayments]	Smaller in \textsf{traditional} arm.
\item[Assets]	Increased in all arms. Initially increased then decreased. There might have been liquidation of assets to repay the loans.
\item[Labour incomes]	Increased steadily during rd 2-4 in all arms. 
\item[Consumption]	Increased during rd 2-4 in all arms.
\item[IGAs]	Multiple IGAs for \textsf{Tradtional} arm. Everyone else chose to invest in cows, suggesting entrepreneurship does not seem to matter in the uptake of loans. It is consistent with the presence of a poverty trap induced by a liquidity constraint and convexity in livestock production technology.
\end{description}
One sees changes in investment choices when one compares \textsf{traditional} and all other arms. However, consumption does not seem to differ. Repayments and asset holding are greater in all other arms. These are consistent with households are enforcing the repayment disciplines and reinvesting the proceeds rather than increasing consumption. 


A more detailed summary:
\begin{description}
\vspace{1.0ex}\setlength{\itemsep}{1.0ex}\setlength{\baselineskip}{12pt}
\item[Low repayment rates]	Repayment was poor. Net saving was forfeit for repayment. Mean raw loan recovery rate (counting only repayments) measured at the end of third year was 0.67 overall, and was lowest for \textsf{traditional} at 0.48. Counting also net saving, these numbers change to 0.85, 0.59, respectively.
\item[Large-sized or grace period loans resulted in higher repayment rates]	Controlling for the loan size, larger initial lending resulted in larger repayment and net saving. As opposed to GUK's anxiety, lending was relatively less risky with large loans and loans with a grace period.
\item[No difference in repayment risk by poverty status] Raw loan recovery rates are 0.67, 0.67, respectively, for ultra poor and moderately poor. Also no statitically meaningful difference is found for cumulative repayment plus cumulative net saving.
\item[No difference in household assets]	Household assets increased in rd 1 - 3, then reduced in rd 4 (possibly liquidating for repayment purpose), with the overall impact of increased household asset values yet no statistically significant difference between arms. 
\item[No difference in labour incomes, per member consumption, marriage rates]	Per member consumption increased in all arms with no difference between arms. Marriage rates do not difffer between arms. A greater swing in labour incomes for \textsf{large}.
\end{description}

\section{Read files}


\subsection{Read from a list}

In reading raw files, I added ID information (\textsf{\footnotesize ./ID/ID\_Updated\_received\_from\_Abu.dta}) to all pages. I will further add HH ID information from the admin file if possible.







Description of data:
\begin{description}
\vspace{1.0ex}\setlength{\itemsep}{1.0ex}\setlength{\baselineskip}{12pt}
\item[ad]	Administrative data: Up to [-24, 48] months after first loan disbursement. This file has not been used in \textsf{read\_cleaned\_data.rnw}.
\item[sch1]	Schooling panel with attrition. Aged 6-18 in rd1. \textsf{Enrolled=\{0,1\}} is defined for children aged 6-18 in rd1 by referencing to \textsf{currently\_enrolled} and age information.
\item[sch2]	Schooling panel after augmenting attrited children to \textsf{sch1}. Attrited children are augmented by assuming to be out of school. \textsf{AssignRegression} is group classification: Number of observation is \textsf{618, 633, 594, 593, 363, 100} for \textsf{traditional, large, largeGrace, cow, dropOuts, forcedDropOuts}, respectively.
\item[ros]	 \textsf{roster} to condition the initial status prior to participation.
\item[ass]	 Assets. Household assets (houses, durables) and productive assets (machines, tools). 
\item[lvo]	Livestock holding. Rd 3 data is not entered yet.
\item[lab]	Labour incomes.
\item[far]	Farming revenues (no costs reported).
\item[con]	Household consumption. Food expenditure asks both bought and consumed volumes and prices. We impute consumption values by using median prices. All quantity is set to annualised quantity.
\item[shk]	Shocks. 
\end{description}

\subsection{Sample selection and treament assignment}

\subsubsection{Admin info}

\begin{Schunk}
\begin{Sinput}
adw2 <- readRDS(paste0(path1234, "admin_data_wide2.rds"))
idfu <- readRDS(paste0(pathsave, "idfu.rds"))
\end{Sinput}
\end{Schunk}
Redefine arms to include \textsf{DropOuts} in original arms.

\begin{Schunk}
\begin{Sinput}
setkey(idfu, hhid)
setkey(adw2, hhid)
adw3 <- idfu[adw2]
adw3[, MemNum := 1:.N, by = .(hhid, Year)]
#table0(adw3[MemNum==1, .(ArmInidfu, randomArm)])
adw3[, RArm := Arm]
adw3[grepl("^drop", Arm) & grepl("con", randomArm), RArm := "traditional"]
adw3[grepl("^drop", Arm) & grepl("^La.*t$", randomArm), RArm := "large"]
adw3[grepl("^drop", Arm) & grepl("^La.*gr", randomArm), RArm := "large grace"]
adw3[grepl("^drop", Arm) & grepl("^pack", randomArm), RArm := "cow"]
ad0 <- adw3[, 
  .(RArm, Arm, randomArm, groupid, hhid, TradGroup, 
    creditstatus, Mem, povertystatus,
    Date, Year, Month, DisDate1, MonthsElapsed, MonthsRepaid, LoanYear,
    EffectiveRepayment, value.repay, value.NetSaving, value.missw,
    OtherRepaid, OtherNetSaving, OtherMisses, CumOtherMisses,
    CumRepaid, CumEffectiveRepayment, CumNetSaving, CumPlannedInstallment,
    CumOtherRepaid, CumOtherNetSaving, CumMisses, EffectivelyFullyRepaid,
    CumRepaidRate, CumEffectiveRepaidRate)]
\end{Sinput}
\end{Schunk}



\subsubsection{Merge admin and roster files}

How I combined between pages: First, merge time-invariant portion of admin data \textsf{adbase} with roster data \textsf{ros} with \textsf{hhid} as a key. Then it is merged with time-variant portion of admin data \textsf{adrest} with \textsf{hhid, Year, Month} as keys. %Keep only dates when survey data match. 
Second, merge \textsf{adbase+adrest+ros} with other data \textsf{sch1}, \textsf{sch2}, \textsf{ass}, ... 

By merging in this way, I have \textsf{RArm} information for each HH in survey 1:
\begin{Schunk}
\begin{Soutput}

traditional       large large grace         cow        <NA> 
        485         464         467         487         220 
\end{Soutput}
\begin{Soutput}
             AssignOriginal
Arm           traditional large large grace  cow <NA>
  traditional        7812     0           0    0    0
  large                 0  7596           0    0    0
  large grace           0     0        7453    0    0
  cow                   0     0           0 7597    0
  <NA>                  0     0           0    0 2765
\end{Soutput}
\end{Schunk}
Observations with no \textsf{povertystatus} are drop outs and rejecters.
\begin{Schunk}
\begin{Sinput}
table0(ar.0[, povertystatus])
\end{Sinput}
\begin{Soutput}

   Ultra Poor Moderate Poor          <NA> 
        21203          9255          2765 
\end{Soutput}
\begin{Sinput}
table0(ar.0[is.na(povertystatus), .(Mstatus, survey)])
\end{Sinput}
\begin{Soutput}
              survey
Mstatus          1   2   3   4
  gErosion     344 229 233   0
  gRejection   560 487 466   0
  iRejection     0   0   0 446
  iReplacement   0   0   0   0
  newGroup       0   0   0   0
  oldMember      0   0   0   0
\end{Soutput}
\begin{Sinput}
summary(ar.0[hhid %in% hhid[is.na(povertystatus)], 
  .(hhid, Mstatus, survey, creditstatus)])
\end{Sinput}
\begin{Soutput}
      hhid                  Mstatus         survey              creditstatus 
 Min.   : 7020501   gErosion    : 806   Min.   :1.00   Yes            :   0  
 1st Qu.: 7031914   gRejection  :1513   1st Qu.:1.00   No             :   0  
 Median : 7085811   iRejection  : 446   Median :2.00   Replaced Member:   0  
 Mean   :13884824   iReplacement:   0   Mean   :2.25   NA's           :2765  
 3rd Qu.: 8148314   newGroup    :   0   3rd Qu.:3.00                         
 Max.   :81710220   oldMember   :   0   Max.   :4.00                         
\end{Soutput}
\end{Schunk}


There are 46 members (\textsf{newGroup} in \textsf{Mstatus}) who did not borrow but only saved. 
\begin{Schunk}
\begin{Soutput}
     survey     DisDate1            creditstatus         Mstatus   
 Min.   :1   Min.   :NA    Yes            :  0   gErosion    : 80  
 1st Qu.:1   1st Qu.:NA    No             :208   gRejection  :140  
 Median :1   Median :NA    Replaced Member:  0   iRejection  :159  
 Mean   :1   Mean   :NA    NA's           :220   iReplacement:  3  
 3rd Qu.:1   3rd Qu.:NA                          newGroup    : 20  
 Max.   :1   Max.   :NA                          oldMember   : 26  
             NA's   :428                                           
\end{Soutput}
\end{Schunk}
So are the same with 104 \textsf{oldMember} in \textsf{Mstatus}:
\begin{Schunk}
\begin{Soutput}
  groupid       survey        DisDate1            creditstatus
 70425:20   Min.   :1.00   Min.   :NA    Yes            :  0  
 70650:12   1st Qu.:1.75   1st Qu.:NA    No             :104  
 70861:28   Median :2.50   Median :NA    Replaced Member:  0  
 71166: 8   Mean   :2.50   Mean   :NA                         
 71372:12   3rd Qu.:3.25   3rd Qu.:NA                         
 81693:24   Max.   :4.00   Max.   :NA                         
                           NA's   :104                        
         Mstatus      CumRepaid      CumNetSaving            Arm     
 gErosion    :  0   Min.   :    0   Min.   :-2780   traditional:104  
 gRejection  :  0   1st Qu.:    0   1st Qu.:    0   large      :  0  
 iRejection  :  0   Median :    0   Median :  462   large grace:  0  
 iReplacement:  0   Mean   :  844   Mean   :  487   cow        :  0  
 newGroup    :  0   3rd Qu.:    0   3rd Qu.:  958                    
 oldMember   :104   Max.   :16800   Max.   : 1804                    
                    NA's   :26      NA's   :26                       
\end{Soutput}
\end{Schunk}
There are 12 members (\textsf{iReplacement} in \textsf{Mstatus}) who did not borrow but only saved. 
\begin{Schunk}
\begin{Soutput}
  groupid       survey        DisDate1           creditstatus         Mstatus  
 70650:12   Min.   :1.00   Min.   :NA   Yes            : 0    gErosion    : 0  
            1st Qu.:1.75   1st Qu.:NA   No             :12    gRejection  : 0  
            Median :2.50   Median :NA   Replaced Member: 0    iRejection  : 0  
            Mean   :2.50   Mean   :NA                         iReplacement:12  
            3rd Qu.:3.25   3rd Qu.:NA                         newGroup    : 0  
            Max.   :4.00   Max.   :NA                         oldMember   : 0  
                           NA's   :12                                          
   CumRepaid  CumNetSaving           Arm    
 Min.   :0   Min.   :  60   traditional:12  
 1st Qu.:0   1st Qu.: 150   large      : 0  
 Median :0   Median : 220   large grace: 0  
 Mean   :0   Mean   : 481   cow        : 0  
 3rd Qu.:0   3rd Qu.: 585                   
 Max.   :0   Max.   :1415                   
                                            
\end{Soutput}
\end{Schunk}
Create \textsf{BorrowerStatus} to indicate these guys. \gobblepars

Set \textsf{No} in \textsf{creditstatus} if NA in \textsf{DisDate1}.
\begin{Schunk}
\begin{Soutput}
     survey     DisDate1            creditstatus         Mstatus   
 Min.   :1   Min.   :NA    Yes            :  0   gErosion    : 80  
 1st Qu.:1   1st Qu.:NA    No             :428   gRejection  :140  
 Median :1   Median :NA    Replaced Member:  0   iRejection  :159  
 Mean   :1   Mean   :NA                          iReplacement:  3  
 3rd Qu.:1   3rd Qu.:NA                          newGroup    : 20  
 Max.   :1   Max.   :NA                          oldMember   : 26  
             NA's   :428                                           
    BorrowerStatus
 borrower  :379   
 pure saver: 49   
                  
                  
                  
                  
                  
\end{Soutput}
\end{Schunk}

Need to merge in 2 steps: Merge admin (time-invariant) with roster with \textsf{hhid} as a key, then merge to admin (time-variant {\footnotesize [e.g., OtherRepaid, OtherNetSaving, OtherMisses, CumOtherMisses, CumRepaid, CumEffectiveRepayment, CumNetSaving, CumPlannedInstallment, CumOtherRepaid, CumOtherNetSaving, CumMisses, CumRepaidRate, CumEffectiveRepaidRate, RMOtherNetSaving, RMOtherRepaid]}) with \textsf{hhid, Year, Month} as keys. This is because there are 8398 non-matching cases if we merge using \textsf{Year, Month} of \textsf{IntDate} in roster data and \textsf{Year, Month} of \textsf{Date} in admin data. This is inevitable because survey precedes the first meeting of borrowers: The admin data starts from 2013-05-01 while survey data starts from 2011-10-09 and rd 1 ends at 2013-10-12 for \textsf{oldMember}s with the median date 2012-10-20. Below gives \textsf{Year, Month} in roster data in rd 1 with no match in admin data.
\begin{Schunk}
\begin{Soutput}

  2011-October  2011-November   2012-January   2012-October  2012-November 
             6              1             19           1146            327 
 2012-December 2013-September   2013-October   2014-January   2014-October 
            79              6             19             12             83 
 2014-November  2014-December  2015-November  2015-December   2016-January 
            43             36            111             40             26 
  2017-January  2017-February     2017-March     2017-April          NA-NA 
            44             97             17             17             21 
\end{Soutput}
\end{Schunk}
After 2014, it is mostly drop out members who do not match with admin data because they do not attend the meeting.
\begin{Schunk}
\begin{Sinput}
table0(ar00[is.na(MonthsElapsed) & MemNum == 1 & Year >= 2014, 
  Mgroup])
\end{Sinput}
\begin{Soutput}

   continued    drop outs    new group replacements 
          78          381           58            9 
\end{Soutput}
\end{Schunk}
No additional match if matching only with \textsf{Year}. 
\begin{Schunk}
\begin{Soutput}
               FALSE TRUE
YearMonthMatch  2055 5958
YearMatch       2055 5958
\end{Soutput}
\end{Schunk}
In \textsf{roster + admin} (base: roster): Tabulate \textsf{hhid} observations by \textsf{survey} round and \textsf{RArm} before supplementing with \textsf{AssignOriginal} and \textsf{VArm}. Note: 220 observations with NA are also pointed in \textsf{read\_cleaned\_data.rnw} and are going to be dealt with in the next subsection.
\begin{Schunk}
\begin{Soutput}
      RArm
survey traditional large large grace cow <NA>
     1         485   464         467 487  220
     2         472   445         447 446  173
     3         472   448         452 453  168
     4         465   444         447 444  114
\end{Soutput}
\begin{Soutput}
      AssignOriginal
survey traditional large large grace cow <NA>
     1         485   464         467 487  220
     2         472   445         447 446  173
     3         472   448         452 453  168
     4         465   444         447 444  114
\end{Soutput}
\end{Schunk}

\subsubsection{Merge village level info}

\begin{Schunk}
\begin{Sinput}
library(readstata13)
vr <- read.dta13(paste0(pathcleaned, "RCT_village.dta"), 
  generate.factors = T, nonint.factors = T)
vr <- data.table(vr)
vr[, GroupStatus := "accepted"]
vr[grepl("De", comment), GroupStatus := "group rejection"]
vr[grepl("Ero", comment), GroupStatus := "erosion"]
setnames(vr, c("comment", "randomization"), c("GroupComment", "VArm"))
\end{Sinput}
\end{Schunk}

\begin{Schunk}
\begin{Sinput}
setkey(vr, groupid)
setkey(ar.1, groupid)
ar <- vr[ar.1]
# individual replacing members: GroupStatus: NA => accepted
ar[grepl("Rep", Mstatus), GroupStatus := "accepted"]
\end{Sinput}
\end{Schunk}
Tabulation of \textsf{AssignOriginal} against \textsf{VArm}. It shows complementarity so I can use one variable to fill in NAs in another.
\begin{Schunk}
\begin{Soutput}
              VArm
AssignOriginal traditional large large grace  cow <NA>
   traditional        1244     0           0    0  650
   large                 0  1423           0    0  378
   large grace           0     0        1437    0  376
   cow                   0     0           0 1631  199
   <NA>                418   158          40   59    0
\end{Soutput}
\end{Schunk}
Tabulation of \textsf{RArm} after supplementing with \textsf{AssignOriginal} and \textsf{VArm}.
\begin{Schunk}
\begin{Sinput}
ar[is.na(RArm) & !is.na(AssignOriginal), RArm := AssignOriginal]
ar[is.na(RArm) & !is.na(VArm), RArm := VArm]
\end{Sinput}
\end{Schunk}
\begin{Schunk}
\begin{Soutput}
      RArm
survey traditional large large grace cow
     1         605   504         507 507
     2         585   485         447 466
     3         582   487         452 472
     4         540   483         447 444
\end{Soutput}
\end{Schunk}
Below is what is supplemented from \textsf{VArm} of village level information to the 220 NAs.
\begin{Schunk}
\begin{Soutput}
                 RArm
BorrowerStatus    traditional large large grace cow
  borrower                  0     0           0   0
  pure saver                0     0           0   0
  quit membership         120    40          40  20
\end{Soutput}
\end{Schunk}

\subsection{Merge admin-roster with other files}

\subsubsection{Choosing sample in admin-roster}

Tabulation of \textsf{RArm} when dropping \textsf{twice, double} in traditional arm.
\begin{Schunk}
\begin{Soutput}
  traditional large large grace cow total
1         441   504         507 507  1959
2         319   485         447 466  1717
3         316   487         452 472  1727
4         278   483         447 444  1652
\end{Soutput}
\end{Schunk}
Tabulation of \textsf{RArm} when dropping \textsf{twice} in traditional arm. This may make most sense but a large attrition between rd 1 and 2.
\begin{Schunk}
\begin{Soutput}
  traditional large large grace cow total
1         505   504         507 507  2023
2         430   485         447 466  1828
3         426   487         452 472  1837
4         388   483         447 444  1762
\end{Soutput}
\end{Schunk}
Tabulation of \textsf{RArm} when dropping dirbursement after 2015-01-01. This has less attrition but includes heterogenous treatment among traditional.
\begin{Schunk}
\begin{Soutput}
  traditional large large grace cow total
1         328   385         359 328  1400
2         323   371         350 316  1360
3         323   372         349 318  1362
4         321   370         345 312  1348
\end{Soutput}
\end{Schunk}
In \textsf{roster + admin} 1: Tabulate observations after keeping only observations used in estimation: Keep if \textsf{Mstatus} includes strings old, iRej, gEro, gRej, \& \textsf{DisDate1} is before 2015-01-01, \& \textsf{TradGroup} does not include strings tw.
\begin{Schunk}
\begin{Soutput}
  traditional large large grace cow total
1         170   296         278 248   992
2         137   285         270 240   932
3         137   286         270 239   932
4         136   284         266 235   921
\end{Soutput}
\end{Schunk}
In \textsf{roster + admin} 2: Keep if \textsf{Mstatus} includes strings old, iRej, gEro, gRej, \& \textsf{TradGroup} does not include strings tw (relaxing  \textsf{DisDate1} is before 2015-01-01). \textcolor{blue}{This the data used in this note.} This also shows a lower attrition rate for \textsf{large} arm. \gobblepars
\begin{Schunk}
\begin{Soutput}
  traditional large large grace cow total
1         400   400         400 400  1600
2         327   384         342 366  1419
3         324   386         348 366  1424
4         287   382         343 342  1354
\end{Soutput}
\end{Schunk}
Create \textsf{o1600} to indicate the original 1600 HHs.
\begin{Schunk}
\begin{Sinput}
# tabulation of total by o1600 and round
table0(ar[, .(o1600, survey)])
\end{Sinput}
\begin{Soutput}
     survey
o1600    1    2    3    4
    0 2101 2510 2543 2457
    1 6532 5817 5843 5420
\end{Soutput}
\begin{Sinput}
# tabulation of 1 obs per HH by o1600 and round. o1600 == 0 is added HHs through newGroup and iReplacement.
table0(ar[MemNum==1, .(o1600, survey)])
\end{Sinput}
\begin{Soutput}
     survey
o1600    1    2    3    4
    0  523  611  616  607
    1 1600 1372 1377 1307
\end{Soutput}
\end{Schunk}

Tabulate observations without disbursement date info. Note: \textsf{iReplacement} are borrower in \textsf{BorrowerStatus}. (Did they remain as a member?)
\begin{Schunk}
\begin{Soutput}
     survey     DisDate1            creditstatus         Mstatus   
 Min.   :1   Min.   :NA    Yes            :  0   gErosion    : 80  
 1st Qu.:1   1st Qu.:NA    No             :428   gRejection  :140  
 Median :1   Median :NA    Replaced Member:  0   iRejection  :159  
 Mean   :1   Mean   :NA                          iReplacement:  3  
 3rd Qu.:1   3rd Qu.:NA                          newGroup    : 20  
 Max.   :1   Max.   :NA                          oldMember   : 26  
             NA's   :428                                           
         BorrowerStatus
 borrower       :159   
 pure saver     : 49   
 quit membership:220   
                       
                       
                       
                       
\end{Soutput}
\end{Schunk}
These are people who rejected loans. Add \textsf{RejectedLoans} to \textsf{FirstDisPeriod}. \gobblepars
\begin{Schunk}
\begin{Soutput}
     survey     DisDate1                         FirstDisPeriod
 Min.   :1   Min.   :2013-05-01 00:00:00   BeforeJan2015:1400  
 1st Qu.:1   1st Qu.:2013-07-01 00:00:00   Year2015     : 295  
 Median :1   Median :2013-11-01 00:00:00   Year2016     :   0  
 Mean   :1   Mean   :2014-03-23 17:07:57   AfterJan2017 :   0  
 3rd Qu.:1   3rd Qu.:2014-12-01 00:00:00   RejectedLoans: 428  
 Max.   :1   Max.   :2015-12-01 00:00:00                       
             NA's   :428                                       
          creditstatus          Mstatus             BorrowerStatus
 Yes            :1695   gErosion    :  80   borrower       :1854  
 No             : 428   gRejection  : 140   pure saver     :  49  
 Replaced Member:   0   iRejection  : 160   quit membership: 220  
                        iReplacement: 115                         
                        newGroup    : 408                         
                        oldMember   :1220                         
                                                                  
\end{Soutput}
\end{Schunk}
%This tabulation of \textsf{survey} vs. \textsf{Arm} shows addition from \textsf{roster+admin} 1 is mostly in round 1 for \textsf{traditional} but in all rds for other arms. \textsf{FirstDisPeriod} gives the period of first disbursement, and all credit reeceivers received loans by the end of 2015.

Breakdown of first disbursement by \textsf{RArm} at rd 1 in \textsf{roster + admin} 2.
\begin{Schunk}
\begin{Soutput}
              traditional large large grace cow
BeforeJan2015         170   296         278 248
Year2015               31    52          60  60
Year2016                0     0           0   0
AfterJan2017            0     0           0   0
RejectedLoans         199    52          62  92
total                 400   400         400 400
\end{Soutput}
\end{Schunk}
%Same tabulation if no conditioning on \textsf{Mstatus} or  \textsf{TradGroup}.

Tabulation of membership status against \textsf{GroupStatus} from \textsf{"RCT\_village.dta"}.
\begin{Schunk}
\begin{Soutput}
              GroupStatus
Mstatus        accepted erosion group rejection
  gErosion            0     189               0
  gRejection          0       0             372
  iRejection        543       0             114
  iReplacement      445       0               0
  newGroup         1603       0               0
  oldMember        4747       0               0
\end{Soutput}
\end{Schunk}
There are 114 cases of group rejections in \textsf{GroupStatus} classified as individual rejections in \textsf{Mstatus}. Overwrite \textsf{Mstatus} with \textsf{GroupStatus} in these cases.
\begin{Schunk}
\begin{Soutput}
              GroupStatus
Mstatus        accepted erosion group rejection
  gErosion            0     189               0
  gRejection          0       0             486
  iRejection        543       0               0
  iReplacement      445       0               0
  newGroup         1603       0               0
  oldMember        4747       0               0
\end{Soutput}
\begin{Soutput}
                traditional large large grace  cow total
accepted               1894  1801        1813 1830  7338
erosion                 110     0          20   59   189
group rejection         308   158          20    0   486
\end{Soutput}
\end{Schunk}
As one can see below, \textsf{gRejection} is more frequent in \textsf{traditional} and \textsf{large}, while there is none in \textsf{cow}. \textsf{traditional, cow} have more frequent \textsf{iRejection}. So \textsf{traditional} was disliked both at group and individual levels, \textsf{large} was disliked as a group, \textsf{cow} was disliked at an individual level, and \textsf{large grace} were well received at both group and individual levels. This indicates attractiveness of a grace period at least at the group level, and a large cash form (over small cash or in-kind) at the individual level.
\begin{Schunk}
\begin{Soutput}
              RArm
Mstatus        traditional large large grace cow
  gErosion              40     0          20  20
  gRejection            80    40          20   0
  iRejection            54    12          22  72
  iReplacement          39     8          11  57
  newGroup             166    96          96  50
  oldMember            226   348         338 308
\end{Soutput}
\end{Schunk}
\begin{Schunk}
\begin{Soutput}
              RArm
Mstatus        traditional large large grace  cow
  gErosion            0.07  0.00        0.03 0.04
  gRejection          0.16  0.08        0.04 0.00
  iRejection          0.11  0.02        0.04 0.12
  iReplacement        0.08  0.02        0.02 0.11
  newGroup            0.27  0.19        0.16 0.10
  oldMember           0.45  0.69        0.67 0.61
\end{Soutput}
\end{Schunk}

Create time-invariant \textsf{HHinfo} from \textsf{ar}. \gobblepars

Create roster member total \textsf{RosterMemTotal}. 
\begin{Schunk}
\begin{Sinput}
ar[, RosterMemTotal := .N, by = .(hhid, survey, IntDate)]
# HH member orders
table0(ar[, .(MemNum, survey)])
\end{Sinput}
\begin{Soutput}
      survey
MemNum    1    2    3    4
    1  2123 1983 1993 1914
    2  2061 1930 1930 1841
    3  1874 1781 1782 1691
    4  1414 1408 1415 1324
    5   744  778  803  734
    6   290  302  311  260
    7    88   96   98   79
    8    32   38   40   29
    9     6   10   10    4
    10    1    1    3    1
    11    0    0    1    0
\end{Soutput}
\begin{Sinput}
# HH size distribution
table0(ar[MemNum == RosterMemTotal, .(MemNum, survey)])
\end{Sinput}
\begin{Soutput}
      survey
MemNum   1   2   3   4
    1   62  53  63  73
    2  187 149 148 150
    3  460 373 367 367
    4  670 630 612 590
    5  454 476 492 474
    6  202 206 213 181
    7   56  58  58  50
    8   26  28  30  25
    9    5   9   7   3
    10   1   1   2   1
    11   0   0   1   0
\end{Soutput}
\begin{Sinput}
# single member HHs
ar[hhid %in% hhid[RosterMemTotal == 1], 
  .(hhid, mid, survey, IntDate, sex, Age_1, marital, HeadAge)]
\end{Sinput}
\begin{Soutput}
            hhid mid survey    IntDate    sex Age_1  marital HeadAge
  1:     7020405   1      1 2012-10-07 Female    55  widowed      55
  2:     7020405   1      2 2014-10-14 Female    55        3      55
  3:     7020405   1      3 2015-12-31 Female    55        3      55
  4:     7020405   1      4 2017-04-26 Female    55  widowed      55
  5:     7020413   1      1 2012-10-10 Female    55  widowed      55
 ---                                                                
485: 99081912103   1      4 2017-03-30 Female    20  married      20
486: 99081912103   2      4 2017-03-30   Male    24  married      20
487: 99081912406   1      1 2013-09-08 Female    50 divorced      50
488: 99081912406   1      3 2016-01-11 Female    50        3      50
489: 99081912406   1      4 2017-04-05 Female    50  widowed      50
\end{Soutput}
\end{Schunk}
Save roster-admin data.
\begin{Schunk}
\begin{Sinput}
saveRDS(ar, paste0(pathsavemembership1or4, "RosterAdminData.rds"))
fwrite(ar, paste0(pathsavemembership1or4, "RosterAdminData.prn"), sep = "\t", quote = F)
\end{Sinput}
\end{Schunk}


Schooling. \gobblepars

Schooling pattern in sch1.
\begin{Schunk}
\begin{Soutput}

0000 0001 000n 0011 001n 00nn 0100 0101 010n 0111 011n 01nn 0nnn 1000 1001 100n 
 208   36  216  152   33  192   16    4    9  840  105   70  316   64    8   45 
1011 101n 10nn 1100 1101 110n 1110 1111 111n 11nn 1nnn 
  56   24   86   48   16   84   28 5172  654  326  199 
\end{Soutput}
\end{Schunk}



In \textsf{sch1}: Number of unique \textsf{hhid}s by \textsf{year} (original entry) or \textsf{Year} (extracted from \textsf{IntDate}).
\begin{Schunk}
\begin{Soutput}
      Year
year   2011 2012 2013 2014 2015 2016 2017 <NA>
  2012    4 1069    1    0    0    0    0  168
  2013    0    0  359    0    0    0    0  100
  2014    0    0    0 1251    0    0    0    0
  2015    0    0    0    1  849  358    0    2
  2017    0    0    0    0    0    1 1118    8
\end{Soutput}
\end{Schunk}
In \textsf{sch1}: Number of observations tabulated by \textsf{year} (original entry) and round (\textsf{survey}).
\begin{Schunk}
\begin{Soutput}
      survey
year      1    2    3    4
  2012 1931    0    0    0
  2013  651    0    0    0
  2014    0 2059    0    0
  2015    0    0 1911    0
  2017    0    0    0 1696
\end{Soutput}
\end{Schunk}
In \textsf{sch1}: RoundOrder is 1 if individual is observed for the first time in data, 2 if for the second time, ...
\begin{Schunk}
\begin{Soutput}
      RoundOrder
year      1    2    3    4
  2012 2098    0    0    0
  2013  806    0    0    0
  2014    0 2282    0    0
  2015    0   79 1945    0
  2017    0   28  107 1662
\end{Soutput}
\end{Schunk}
In \textsf{sch2}: Number of observations tabulated by \textsf{year} (original entry) and round (\textsf{survey}).
\begin{Schunk}
\begin{Soutput}
      survey
year      1    2    3    4
  2012 1931    0    0    0
  2013  651    0    0    0
  2014    0 2417    0    0
  2015    0    0 2347    0
  2017    0    0    0 2202
\end{Soutput}
\end{Schunk}
In \textsf{sch2}: RoundOrder.
\begin{Schunk}
\begin{Soutput}
      RoundOrder
year      1    2    3    4    5
  2012 2904    0    0    0    0
  2013    0 2904    0    0    0
  2014    0    0 2904    0    0
  2015    0    0    0 2904    0
  2017    0    0    0    0 2904
\end{Soutput}
\end{Schunk}
In \textsf{sch1}: Number of observations tabulated by \textsf{year} (original entry) and age (\textsf{AgeComputed}).
\begin{Schunk}
\begin{Soutput}
      AgeComputed
year     6   7   8   9  10  11  12  13  14  15  16  17  18  19  20  21  22  23
  2012 168 264 279 114 333  77 237 109 104 173 103  43  94   0   0   0   0   0
  2013  48  93  90  61 118  60  79  55  46  58  46  14  38   0   0   0   0   0
  2014   0  43 222 317 298 211 346 131 234 121 124 152  62  15   6   0   0   0
  2015   0   0  42 225 311 291 198 302 118 192 100  93  95  38  11   8   0   0
  2017   0   0   0   0  40 218 289 279 186 272 110 171  90  64  51  22   4   1
\end{Soutput}
\end{Schunk}
In \textsf{sch2}: Number of observations tabulated by \textsf{year} (original entry) and age (\textsf{AgeComputed}).
\begin{Schunk}
\begin{Soutput}
      AgeComputed
year     5   6   7   8   9  10  11  12  13  14  15  16  17  18  19  20  21  22
  2012  48 261 354 340 232 393 156 292 155 162 219 117  81  94   0   0   0   0
  2013   0  48 261 354 340 232 393 156 292 155 162 219 117  81  94   0   0   0
  2014   0   0  48 261 354 340 232 393 156 292 155 162 219 117  81  94   0   0
  2015   0   0   0  48 261 354 340 232 393 156 292 155 162 219 117  81  94   0
  2017   0   0   0   0   0  48 261 354 340 232 393 156 292 155 162 219 117  81
      AgeComputed
year    23
  2012   0
  2013   0
  2014   0
  2015   0
  2017  94
\end{Soutput}
\end{Schunk}

\subsubsection{Attach variables from admin-roster to other files}


Attach \textsf{RArm, Arm, TradGroup, Mem, ObPattern, AttritIn, o1600, Mstatus, BorrowerStatus, creditstatus, povertystatus, HHsize, HeadLiteracy, IntDate, DisDate1} from \textsf{ar}.

\begin{Schunk}
\begin{Sinput}
vartoattach <- c("RArm", "Arm", "TradGroup", "Mem", "ObPattern", "AttritIn", 
  "o1600", "Mstatus", "BorrowerStatus", "creditstatus", "povertystatus", 
  "HHsize", "HeadLiteracy", "IntDate", "DisDate1")
dfiles <- c("ass", "s1", "s2", "lvo", "lab", "far", "con", "shk")
for (j in 1:length(dfiles)) {
  dd <- get(dfiles[j])
  dd[, groupid := as.integer(as.numeric(as.character(gid)))]
  dd[, gid := NULL]
  dd[, Year :=  as.numeric(format(as.Date(IntDate), "%Y"))]
  dd[, Month := as.character(format(as.Date(IntDate), "%B"))]
  dd[Year <= 2010, Year := Year + 10]
  # drop all variables in each page before copying from ar0
  dd[, (vartoattach) := NULL]
  setorder(dd, groupid, hhid, survey, Year, Month)
  setkey(dd, groupid, hhid, survey)
  if (j < length(dfiles)) dd <- ar0[dd]
  assign(dfiles[j], dd)
}
\end{Sinput}
\end{Schunk}
Create \textsf{Arm*HadCows}, \textsf{Arm*HadCows*Time} interactions in \textsf{lvo}. \gobblepars

Check number of HHs in assets by \textsf{o1600}:
\begin{Schunk}
\begin{Sinput}
table(ass[, .(creditstatus, survey, o1600)])
\end{Sinput}
\begin{Soutput}
, , o1600 = 0

                 survey
creditstatus         1    2    3    4
  Yes              478  588  593  586
  No                23   23   23   21
  Replaced Member    0    0    0    0

, , o1600 = 1

                 survey
creditstatus         1    2    3    4
  Yes             1192 1047 1054 1039
  No               403  323  323  268
  Replaced Member    0    0    0    0
\end{Soutput}
\begin{Sinput}
#table0(ass[o1600 == 0L, .(creditstatus, survey)])
\end{Sinput}
\end{Schunk}
Save all data.

\begin{Schunk}
\begin{Sinput}
fwrite(s1, paste0(pathsavemembership1or4, "RosterAdminSchoolingData.prn"), sep = "\t", quote = F)
fwrite(s2, paste0(pathsavemembership1or4, "RosterAdminSchoolingAugmentedData.prn"), sep = "\t", quote = F)
fwrite(ass, paste0(pathsavemembership1or4, "AssetAdminData.prn"), sep = "\t", quote = F)
fwrite(lvo, paste0(pathsavemembership1or4, "LivestockAdminData.prn"), sep = "\t", quote = F)
fwrite(lab, paste0(pathsavemembership1or4, "LabourIncomeAdminData.prn"), sep = "\t", quote = F)
fwrite(far, paste0(pathsavemembership1or4, "FarmRevenueAdminData.prn"), sep = "\t", quote = F)
fwrite(con, paste0(pathsavemembership1or4, "ConsumptionAdminData.prn"), sep = "\t", quote = F)
fwrite(shk, paste0(pathsavemembership1or4, "Shocks.prn"), sep = "\t", quote = F)
\end{Sinput}
\end{Schunk}




Further data preparations (trimming, adding shocks, round numbering, creating dummy vectors, interaction terms) for estimation. Produces files: \textsf{\footnotesize RosterAdminDataUsedForEstimation.prn, AssetAdminDataUsedForEstimation.prn, LivestockAdminDataUsedForEstimation.prn, LabourIncomeAdminDataUsedForEstimation.prn, FarmRevenueAdminDataUsedForEstimation.prn, ConsumptionAdminDataUsedForEstimation.prn, ShocksAdminDataUsedForEstimation.prn}.

\hspace{-1cm}\begin{minipage}[t]{14cm}
\hfil\textsc{\normalsize Table \refstepcounter{table}\thetable: Data trimming results\label{tab trim}}\\
\setlength{\tabcolsep}{1pt}
\setlength{\baselineskip}{8pt}
\renewcommand{\arraystretch}{.48}
\hfil\begin{tikzpicture}
\node (tbl) {\input{c:/data/GUK/analysis/save/membership1or4/TrimmingNumObsTable.tex}};
%\input{c:/dropbox/data/ramadan/save/tablecolortemplate.tex}
\end{tikzpicture}\\
\renewcommand{\arraystretch}{.8}
\setlength{\tabcolsep}{1pt}
\begin{tabular}{>{\hfill\scriptsize}p{1cm}<{}>{\hfill\scriptsize}p{.25cm}<{}>{\scriptsize}p{12cm}<{\hfill}}
Source:& \multicolumn{2}{l}{\scriptsize GUK survey data.}\\
Notes: & 1. & Top panel is observations for all rounds. Bottom panel is observations for round 1 only. We aim for ITT estimates and need to retain original sampled individuals. old$|$iRej$|$\^{}g in \textsf{Mstatus} are strings for old members, individual rejecters, group rejecters, group erosion. con$|$\^{}dro$|$\^{}rep in \textsf{Mgroup} indicates continuing, dropouts, replacing members. tw$|$dou in \textsf{TradGroup} are members who received loans twice and double amount in the 2nd loans. They are omitted from analysis because they are under a different treatment arm.\\
& 2. & 
\end{tabular}
\end{minipage}

Number of observations after trimming: 1. Keep only membership = 1 or 4, which corresponds to Mstatus old, iRej, gR, gE; 2. Keep only continuing, dropouts members in Mgroup.
\begin{Schunk}
\begin{Soutput}
    file tee traditional large large grace cow
 1:   ar   1         400   400         400 400
 2:   ar   2         398   400         400 398
 3:   ar   3         379   399         398 394
 4:   ar   4         347   395         389 387
 5:   ar   5         307   378         369 370
 6:   ar   6         289   376         355 369
 7:   ar   7         270   374         340 360
 8:   ar   8         267   371         337 359
 9:   ar   9         248   351         317 335
10:   ar  10         241   350         316 330
11:   ar  11         232   338         311 322
12:   ar  12         225   334         300 318
13:   ar  13         187   287         254 269
14:   ar  14         183   283         250 267
15:   ar  15         173   274         234 251
16:   ar  16         156   250         217 229
17:   ar  17          99   169         147 166
18:   ar  18          94   162         142 159
19:   ar  19          86   146         126 138
20:   ar  20          77   131         110 120
21:   ar  21          41    65          64  61
22:   ar  22          39    64          60  57
23:   ar  23          33    55          50  44
24:   ar  24          28    48          39  39
25:   ar  25          12    25          18  18
26:   ar  26          11    25          14  16
27:   ar  27           9    24          13  10
28:   ar  28           8    19          12   8
29:   ar  29           5    12           8   2
30:   ar  30           5    12           7   1
31:   ar  31           4     8           6  NA
32:   ar  32           4     6           4  NA
33:   ar  33           2     2           2  NA
34:   ar  34           2     2           2  NA
35:   ar  35           1     1           1  NA
36:   ar  36           1    NA          NA  NA
37:   ar  37           1    NA          NA  NA
38:   ar  38           1    NA          NA  NA
39:   ar  39           1    NA          NA  NA
40:   ar  40           1    NA          NA  NA
41:  ass   1         398   400         399 399
42:  ass   2         283   389         353 378
43:  ass   3         276   384         349 365
44:  ass   4         238   378         330 330
45:  con   1         283   388         352 378
46:  con   2         276   383         349 365
47:  con   3         238   377         331 331
48:  far   1          21    96          52  57
49:  far   2           5    51          28  27
50:  far   3           2    22          17  12
51:  far   4          NA     2          NA   1
52:  lab   1         398   400         399 399
53:  lab   2         396   400         400 397
54:  lab   3         378   399         398 394
55:  lab   4         351   394         387 385
56:  lab   5         305   374         366 363
57:  lab   6         258   347         327 332
58:  lab   7         191   283         250 271
59:  lab   8         119   187         173 170
60:  lab   9          71   121         104 100
61:  lab  10          39    86          67  64
62:  lab  11          29    57          44  45
63:  lab  12          21    40          27  29
64:  lab  13          14    21          19  18
65:  lab  14           9    13          15   8
66:  lab  15           8    10           9   6
67:  lab  16           5     8           5   3
68:  lab  17           3     3           3   1
69:  lab  18           1     1           1  NA
70:  lab  19           1    NA           1  NA
71:  lab  20           1    NA           1  NA
72:  lvo   1         398   399         379 398
73:  lvo   2         283   390         373 379
74:  lvo   3         276   384         348 365
75:  lvo   4         238   377         330 328
76: sch1   1         460   479         505 487
77: sch1   2         300   396         369 403
78: sch1   3         266   356         340 351
79: sch1   4         204   306         282 277
80: sch2   1         460   479         505 487
81: sch2   3         336   460         422 453
82: sch2   4         325   448         417 434
83: sch2   5         278   439         401 389
    file tee traditional large large grace cow
\end{Soutput}
\end{Schunk}



\hfil\begin{minipage}[t]{12cm}
\hfil\textsc{\normalsize Table \refstepcounter{table}\thetable: Number of observations  in each file at round 1 from HHs with single treatment\label{tab NObsOHall}}\\
\setlength{\tabcolsep}{.5pt}
\setlength{\baselineskip}{10pt}
\renewcommand{\arraystretch}{.7}
\hfil\begin{tikzpicture}
\node (tbl) {\input{c:/data/GUK/analysis/save/membership1or4/NumObsOriginalHHs_all.tex}};
%\input{c:/dropbox/data/ramadan/save/tablecolortemplate.tex}
\end{tikzpicture}\\
\renewcommand{\arraystretch}{.8}
\setlength{\tabcolsep}{1pt}
\begin{tabular}{>{\hfill\scriptsize}p{1cm}<{}>{\hfill\scriptsize}p{.25cm}<{}>{\scriptsize}p{10cm}<{\hfill}}
Source:& \multicolumn{2}{l}{\scriptsize Estimated with GUK administrative and survey data.}\\
Notes: & 1. & Sample is all households: Original 1600 and added households through new groups and individuals replacing opt-out members. All households in traditional arm who received more than one loan are excluded.\\
& 2. &  
\end{tabular}
\end{minipage}

\hfil\begin{minipage}[t]{12cm}
\hfil\textsc{\normalsize Table \refstepcounter{table}\thetable: Number of observations in each file at round 1 from original 1600 HHs\label{tab NObsOHo1600}}\\
\setlength{\tabcolsep}{.5pt}
\setlength{\baselineskip}{10pt}
\renewcommand{\arraystretch}{.7}
\hfil\begin{tikzpicture}
\node (tbl) {\input{c:/data/GUK/analysis/save/membership1or4/NumObsOriginalHHs_o1600.tex}};
%\input{c:/dropbox/data/ramadan/save/tablecolortemplate.tex}
\end{tikzpicture}\\
\renewcommand{\arraystretch}{.8}
\setlength{\tabcolsep}{1pt}
\begin{tabular}{>{\hfill\scriptsize}p{1cm}<{}>{\hfill\scriptsize}p{.25cm}<{}>{\scriptsize}p{10cm}<{\hfill}}
Source:& \multicolumn{2}{l}{\scriptsize Estimated with GUK administrative and survey data.}\\
Notes: & 1. & Sample is original 1600 households who agree to join the group. This includes households who later dropped out due to flood, group rejections, and individual rejections. All original 1600 households are tracked but some attrit from the sample.\\
& 2. &  
\end{tabular}
\end{minipage}


\section{Descriptive statistics}







\section{Estimation}


\subsection{Schooling}


% This file is a pure replicate of read, trim of data in the main file (ImpactEstimation_body3.rnw). 
\begin{Schunk}
\begin{Soutput}
   tee traditional large large grace cow total
1:   1         243   241         217 248   949
2:   2         180   240         217 247   884
3:   3         165   225         210 225   825
4:   4         133   206         181 190   710
\end{Soutput}
\end{Schunk}
If using \textsf{s1}, retain only the complete portion of panel. \textsf{sch1} has 5781 rows. Drop 370 observations in \textsf{sch1} with nnn in \textsf{Spattern} and 8 observations with 1001 in \textsf{EnrollPattern} because they are likely to be errors. This leaves us with 5403 rows. 
With OLS,  93, 154, 246, 1066 individuals are repeatedly observed for 1, 2, 3, 4 times, respectively. With FD, \textsf{s1} is reduced to 3844 rows after first-differencing with 246, 1066 individuals with repeatedly observed for 3, 4 times, respectively.
\begin{Schunk}
\begin{Soutput}

0000 0001 000n 0011 001n 00nn 0100 0101 010n 0111 011n 01nn 0nnn 1000 1001 100n 
  41    8   48   27    6   58    2    0    2  184   27   19    0   13    0   12 
1011 101n 10nn 1100 1101 110n 1110 1111 111n 11nn 1nnn 
   9    6   20   11    1   18    5  833  137   72    0 
\end{Soutput}
\end{Schunk}





\hspace{-1.5cm}\begin{minipage}[t]{14cm}
\hfil\textsc{\normalsize Table \refstepcounter{table}\thetable: OLS estimation of school enrollment\label{tab ols school}}\\
\setlength{\tabcolsep}{.5pt}
\setlength{\baselineskip}{8pt}
\renewcommand{\arraystretch}{.50}
\hfil\begin{tikzpicture}
\node (tbl) {\input{c:/data/GUK/analysis/save/membership1or4/SchoolingOLSEstimationResults.tex}};
%\input{c:/dropbox/data/ramadan/save/tablecolortemplate.tex}
\end{tikzpicture}\\
\renewcommand{\arraystretch}{.8}
\setlength{\tabcolsep}{1pt}
\begin{tabular}{>{\hfill\scriptsize}p{1cm}<{}>{\hfill\scriptsize}p{.25cm}<{}>{\scriptsize}p{12cm}<{\hfill}}
Source:& \multicolumn{2}{l}{\scriptsize Estimated with GUK administrative and survey data.}\\
Notes: & 1. & Intercept terms are omitted in estimating equations. Year effects are included in estimation (not shown). \textsf{s1} is complete portion of panel. \textsf{s2} is a panel data augmenting attrited members in \textsf{s1} with an assumption that they are out of school unless it is explicitly stated as attending school by family members. Number of observations decreases in (2) relative to (1) because of NAs in \textsf{Schooling} (because \textsf{Age\_1} is missing) erroneous entries in calendar \textsf{Year} in \textsf{IntDate} which result in NAs. \\
& 2. & ${}^{***}$, ${}^{**}$, ${}^{*}$ indicate statistical significance at 1\%, 5\%, 10\%, respetively. Standard errors are clustered at group (village) level.
\end{tabular}
\end{minipage}

Check number of observations in each cell:






\hspace{-1.0cm}\begin{minipage}[t]{14cm}
\hfil\textsc{\normalsize Table \refstepcounter{table}\thetable: Number of observations in each cells of schooling regression in Table \ref{tab ols school}\label{tab num obs ols school}}\\
\setlength{\tabcolsep}{1pt}
\setlength{\baselineskip}{8pt}
\renewcommand{\arraystretch}{.45}
\hfil\begin{tikzpicture}
\node (tbl) {\input{c:/data/GUK/analysis/save/membership1or4/NumObsSchoolingOLS.tex}};
%\input{c:/dropbox/data/ramadan/save/tablecolortemplate.tex}
\end{tikzpicture}\\
\renewcommand{\arraystretch}{.8}
\setlength{\tabcolsep}{1pt}
\begin{tabular}{>{\hfill\scriptsize}p{1cm}<{}>{\hfill\scriptsize}p{.25cm}<{}>{\scriptsize}p{12cm}<{\hfill}}
Source:& \multicolumn{2}{l}{\scriptsize GUK administrative and survey data.}\\
Notes: & 1. &  Original data is schooling panel data with attrition. Augmented data is original data plus imputed values of schooling for attrited individuals. \\
& 2. & %${}^{***}$, ${}^{**}$, ${}^{*}$ indicate statistical significance at 1\%, 5\%, 10\%, respetively. Standard errors are clustered at group (village) level.
\end{tabular}
\end{minipage}

\hspace{-1.5cm}\begin{minipage}[t]{14cm}
\hfil\textsc{\normalsize Table \refstepcounter{table}\thetable: OLS estimation of school enrollment, different grouping\label{tab ols school2}}\\
\setlength{\tabcolsep}{.5pt}
\setlength{\baselineskip}{8pt}
\renewcommand{\arraystretch}{.55}
\hfil\begin{tikzpicture}
\node (tbl) {\input{c:/data/GUK/analysis/save/membership1or4/SchoolingOLSEstimationResults2.tex}};
%\input{c:/dropbox/data/ramadan/save/tablecolortemplate.tex}
\end{tikzpicture}\\
\renewcommand{\arraystretch}{.8}
\setlength{\tabcolsep}{1pt}
\begin{tabular}{>{\hfill\scriptsize}p{1cm}<{}>{\hfill\scriptsize}p{.25cm}<{}>{\scriptsize}p{12cm}<{\hfill}}
Source:& \multicolumn{2}{l}{\scriptsize Estimated with GUK administrative and survey data.}\\
Notes: & 1. & Intercept terms are omitted in estimating equations. Year effects are included in estimation (not shown). \textsf{s1} is complete portion of panel. \textsf{s2} is a panel data augmenting attrited members in \textsf{s1} with an assumption that they are out of school unless it is explicitly stated as attending school by family members. \textsf{SmallSize} includes \textsf{Traditional}, \textsf{LargeSize} includes \textsf{Large, Large grace, Cow}. \textsf{WithoutGrace} includes \textsf{Traditional, Large}, \textsf{WithGrace} includes \textsf{Large grace, cow}.\\
& 2. & ${}^{***}$, ${}^{**}$, ${}^{*}$ indicate statistical significance at 1\%, 5\%, 10\%, respetively. Standard errors are clustered at group (village) level.
\end{tabular}
\end{minipage}


\hspace{-1.5cm}\begin{minipage}[t]{14cm}
\hfil\textsc{\normalsize Table \refstepcounter{table}\thetable: OLS estimation of school enrollment, small vs. large sized \label{tab ols school3}}\\
\setlength{\tabcolsep}{.5pt}
\setlength{\baselineskip}{8pt}
\renewcommand{\arraystretch}{.55}
\hfil\begin{tikzpicture}
\node (tbl) {\input{c:/data/GUK/analysis/save/membership1or4/SchoolingOLSEstimationResults3.tex}};
%\input{c:/dropbox/data/ramadan/save/tablecolortemplate.tex}
\end{tikzpicture}\\
\renewcommand{\arraystretch}{.8}
\setlength{\tabcolsep}{1pt}
\begin{tabular}{>{\hfill\scriptsize}p{1cm}<{}>{\hfill\scriptsize}p{.25cm}<{}>{\scriptsize}p{12cm}<{\hfill}}
Source:& \multicolumn{2}{l}{\scriptsize Estimated with GUK administrative and survey data.}\\
Notes: & 1. & Intercept terms are omitted in estimating equations. Year effects are included in estimation (not shown). \textsf{s1} is complete portion of panel. \textsf{s2} is a panel data augmenting attrited members in \textsf{s1} with an assumption that they are out of school unless it is explicitly stated as attending school by family members. \textsf{SmallSize} includes \textsf{Traditional}, \textsf{LargeSize} includes \textsf{Large, Large grace, Cow}. \textsf{WithoutGrace} includes \textsf{Traditional, Large}, \textsf{WithGrace} includes \textsf{Large grace, cow}.\\
& 2. & ${}^{***}$, ${}^{**}$, ${}^{*}$ indicate statistical significance at 1\%, 5\%, 10\%, respetively. Standard errors are clustered at group (village) level.
\end{tabular}
\end{minipage}

\hspace{-1cm}\begin{minipage}[t]{14cm}
\hfil\textsc{\normalsize Table \refstepcounter{table}\thetable: Number of observations in each cells of schooling regression in Table \ref{tab ols school2}, \ref{tab ols school3}\label{tab num obs ols school3}}\\
\setlength{\tabcolsep}{1pt}
\setlength{\baselineskip}{8pt}
\renewcommand{\arraystretch}{.45}
\hfil\begin{tikzpicture}
\node (tbl) {
\input{c:/data/GUK/analysis/save/membership1or4/NumObsSchoolingOLS2.tex}};
\end{tikzpicture}\\

\hfil\begin{tikzpicture}
\node (tbl2) {
\input{c:/data/GUK/analysis/save/membership1or4/NumObsSchoolingOLS4.tex}};
\end{tikzpicture}\\
\renewcommand{\arraystretch}{.8}
\setlength{\tabcolsep}{1pt}
\begin{tabular}{>{\hfill\scriptsize}p{1cm}<{}>{\hfill\scriptsize}p{.25cm}<{}>{\scriptsize}p{12cm}<{\hfill}}
Source:& \multicolumn{2}{l}{\scriptsize GUK administrative and survey data.}\\
Notes: & 1. &  \\
& 2. & %${}^{***}$, ${}^{**}$, ${}^{*}$ indicate statistical significance at 1\%, 5\%, 10\%, respetively. Standard errors are clustered at group (village) level.
\end{tabular}
\end{minipage}


\hspace{-1.5cm}\begin{minipage}[t]{14cm}
\hfil\textsc{\normalsize Table \refstepcounter{table}\thetable: OLS estimation of school enrollment, ultra poor vs. moderately poor\label{tab ols school4}}\\
\setlength{\tabcolsep}{1pt}
\setlength{\baselineskip}{8pt}
\renewcommand{\arraystretch}{.55}
\hfil\begin{tikzpicture}
\node (tbl) {\input{c:/data/GUK/analysis/save/membership1or4/SchoolingOLSEstimationPovertystatusResults.tex}};
%\input{c:/dropbox/data/ramadan/save/tablecolortemplate.tex}
\end{tikzpicture}\\
\renewcommand{\arraystretch}{.8}
\setlength{\tabcolsep}{1pt}
\begin{tabular}{>{\hfill\scriptsize}p{1cm}<{}>{\hfill\scriptsize}p{.25cm}<{}>{\scriptsize}p{12cm}<{\hfill}}
Source:& \multicolumn{2}{l}{\scriptsize Estimated with GUK administrative and survey data.}\\
Notes: & 1. & Intercept terms are omitted in estimating equations. Year effects are included in estimation (not shown). \textsf{s1} is complete portion of panel. \textsf{s2} is a panel data augmenting attrited members in \textsf{s1} with an assumption that they are out of school unless it is explicitly stated as attending school by family members. \\
& 2. & ${}^{***}$, ${}^{**}$, ${}^{*}$ indicate statistical significance at 1\%, 5\%, 10\%, respetively. Standard errors are clustered at group (village) level.
\end{tabular}
\end{minipage}

\begin{palepinkleftbar}
\begin{finding}
\textit{The following does not apply to first-difference estimation.} \sout{\textsc{\small Table \ref{tab ols school}} shows school enrollment is higher for \textsf{s1} than \textsf{s2}, indicating nonattriting members are school goers. When using \textsf{s1} data, \textsf{cow} and \textsf{large grace} show negative impacts for older children, yet not for girls in junior high schools. In fact, (1) shows that girles in high school have higher enrollment in \textsf{s1} and both junior and high schools for \textsf{s2} in (4). This may be due to increased labour demand within a family for boys. Similar patterns are found in \textsf{s2} data, yet not statistically significant, probably because data augmentation introduces more school dropouts among older girls. In \textsc{\small Table \ref{tab ols school2}} when using with grace/without grace grouping, the pattern becomes statistically significant for both \textsf{s1} and \textsf{s2}. Large size vs. small size contrast has smaller statistical power that more subtle outcomes cannot be detected. No difference between ultra and moderately poor is found in \textsc{\small Table \ref{tab ols school4}}.} In first-difference estimation, rd 1 and rd 4 comparison of \textsc{\normalsize Table \ref{tab FD enroll5}} (2) - (4) show a marginally statistically significant higher enrollment rates for girls of \textsf{Cow}. 
\end{finding}
\end{palepinkleftbar}




\begin{Schunk}
\begin{Soutput}
   RArm
tee traditional large large grace cow
  1         390   397         369 403
  2         293   378         350 380
  3         263   357         337 348
  4         214   320         304 300
\end{Soutput}
\begin{Soutput}
Warning in `[.data.table`(s1, , `:=`(c("Schooling", "hhid", "mid", "sex", : Adding new column 'en' then assigning NULL (deleting it).
\end{Soutput}
\begin{Soutput}
Warning in `[.data.table`(s2, , `:=`(c("Schooling", "hhid", "mid", "sex", : Adding new column 'en' then assigning NULL (deleting it).
\end{Soutput}
\begin{Soutput}
Dropped 93 obs due to T<2.
Dropped 1466 obs due to NA.
Dropped 93 obs due to T<2.
Dropped 1466 obs due to NA.
Dropped 230 obs due to T<2.
Dropped 1701 obs due to NA.
Dropped 230 obs due to T<2.
Dropped 1701 obs due to NA.
Dropped 421 obs due to T<2.
Dropped 1138 obs due to NA.
Dropped 424 obs due to T<2.
Dropped 1507 obs due to NA.
\end{Soutput}
\end{Schunk}
If using \textsf{s1}, retain only the complete portion of panel. \textsf{sch1} has 5781 rows. Drop 370 observations in \textsf{sch1} with nnn in \textsf{Spattern} and 8 observations with 1001 in \textsf{EnrollPattern} because they are likely to be errors. This leaves us with 5403 rows. 

With OLS,  1559 individuals are repeatedly observed for 1 times, respectively. With FD, \textsf{s1} is reduced to 3844 rows after first-differencing with 133, 202, 1066 individuals with repeatedly observed for 2, 3, 4 times, respectively.
Individuals with NAs in \textsf{Schooling}. Mostly older children (15.5 in \textsf{s1}, 16.5 in \textsf{s.2}) but with a high reported enrollment rate (0.7 for \textsf{s1}, 0.2 for \textsf{s.2}) at rd 4. We will substitute relevant schooling levels to \textsf{Schooling}.
\begin{Schunk}
\begin{Soutput}

   1 
5403 
\end{Soutput}
\begin{Soutput}

   1 
5403 
\end{Soutput}
\end{Schunk}
An example of dummy interactions: \textsf{\footnotesize dummyLargeSize.dummyPrimary.Time.2, dummySmallSize.dummyPrimary.Time.2, dummyLargeSize.dummyJunior.Time.2, dummySmallSize.dummyJunior.Time.2, dummyLargeSize.dummyHigh.Time.2, dummySmallSize.dummyHigh.Time.2, dummyLargeSize.dummyPrimary.Time.3, dummySmallSize.dummyPrimary.Time.3, dummyLargeSize.dummyJunior.Time.3, dummySmallSize.dummyJunior.Time.3, dummyLargeSize.dummyHigh.Time.3, dummySmallSize.dummyHigh.Time.3, dummyLargeSize.dummyPrimary.Time.4, dummySmallSize.dummyPrimary.Time.4, dummyLargeSize.dummyJunior.Time.4, dummySmallSize.dummyJunior.Time.4, dummyLargeSize.dummyHigh.Time.4, dummySmallSize.dummyHigh.Time.4}.
Obs for \textsf{s1}.
\begin{Schunk}
\begin{Soutput}

   2    3    4 
1401 1305 1138 
\end{Soutput}
\end{Schunk}
Obs for \textsf{s1} and admin repayment data.
\begin{Schunk}
\begin{Soutput}

   2    3    4 
1401 1305 1138 
\end{Soutput}
\begin{Soutput}

   2    3    4 
1671 1624 1507 
\end{Soutput}
\end{Schunk}
Obs for survey \textsf{s2}.
\begin{Schunk}
\begin{Soutput}

   2    3    4 
1671 1624 1507 
\end{Soutput}
\end{Schunk}
Obs for survey \textsf{s2} and admin repayment data.
\begin{Schunk}
\begin{Soutput}

   2    3    4 
1671 1624 1507 
\end{Soutput}
\end{Schunk}
\begin{Schunk}
\begin{Sinput}
arsuffixes <- c("", "g", "p", "s")
source(paste0(pathprogram, "SchoolingCovariateSelection.R"))
\end{Sinput}
\end{Schunk}





\hspace{-1cm}\begin{minipage}[t]{14cm}
\hfil\textsc{\normalsize Table \refstepcounter{table}\thetable: FD estimation of school enrollment\label{tab FD enroll}}\\
\setlength{\tabcolsep}{1pt}
\setlength{\baselineskip}{8pt}
\renewcommand{\arraystretch}{.48}
\hfil\begin{tikzpicture}
\node (tbl) {\input{c:/data/GUK/analysis/save/membership1or4/SchoolingFDEstimationResults.tex}};
%\input{c:/dropbox/data/ramadan/save/tablecolortemplate.tex}
\end{tikzpicture}\\
\renewcommand{\arraystretch}{.8}
\setlength{\tabcolsep}{1pt}
\begin{tabular}{>{\hfill\scriptsize}p{1cm}<{}>{\hfill\scriptsize}p{.25cm}<{}>{\scriptsize}p{12cm}<{\hfill}}
Source:& \multicolumn{2}{l}{\scriptsize Estimated with GUK administrative and survey data.}\\
Notes: & 1. & First-difference estimates. A first-difference is defined as $\Delta x_{t+1}\equiv x_{t+1} - x_{t}$. First-differenced regressands are regressed on categorical and time-variant covariates. Net saving is taken from administrative data and merged with survey data at Year-Month of survey interviews. Head age and literacy are from baseline data. Intercept terms are omitted in estimating equations. Net saving is saving - withdrawal. \\
& 2. & ${}^{***}$, ${}^{**}$, ${}^{*}$ indicate statistical significance at 1\%, 5\%, 10\%, respetively. Standard errors are clustered at group (village) level.
\end{tabular}
\end{minipage}

\hspace{-1cm}\begin{minipage}[t]{14cm}
\hfil\textsc{\normalsize Table \refstepcounter{table}\thetable: FD estimation of net school enrollment, ultra poor vs. moderately poor\label{tab FD enroll2}}\\
\setlength{\tabcolsep}{1pt}
\setlength{\baselineskip}{8pt}
\renewcommand{\arraystretch}{.55}
\hfil\begin{tikzpicture}
\node (tbl) {\input{c:/data/GUK/analysis/save/membership1or4/SchoolingPovertystatusFDEstimationResults.tex}};
%\input{c:/dropbox/data/ramadan/save/tablecolortemplate.tex}
\end{tikzpicture}\\
\renewcommand{\arraystretch}{.8}
\setlength{\tabcolsep}{1pt}
\begin{tabular}{>{\hfill\scriptsize}p{1cm}<{}>{\hfill\scriptsize}p{.25cm}<{}>{\scriptsize}p{12cm}<{\hfill}}
Source:& \multicolumn{2}{l}{\scriptsize Estimated with GUK administrative and survey data.}\\
Notes: & 1. & First-difference estimates. A first-difference is defined as $\Delta x_{t+1}\equiv x_{t+1} - x_{t}$. First-differenced regressands are regressed on categorical and time-variant covariates. Net saving is taken from administrative data and merged with survey data at Year-Month of survey interviews. Head age and literacy are from baseline data. Intercept terms are omitted in estimating equations. Net saving is saving - withdrawal. \\
& 2. & ${}^{***}$, ${}^{**}$, ${}^{*}$ indicate statistical significance at 1\%, 5\%, 10\%, respetively. Standard errors are clustered at group (village) level.
\end{tabular}
\end{minipage}

\hspace{-1cm}\begin{minipage}[t]{14cm}
\hfil\textsc{\normalsize Table \refstepcounter{table}\thetable: FD estimation of school enrollment, with vs. without a grace period\label{tab FD enroll3}}\\
\setlength{\tabcolsep}{1pt}
\setlength{\baselineskip}{8pt}
\renewcommand{\arraystretch}{.55}
\hfil\begin{tikzpicture}
\node (tbl) {\input{c:/data/GUK/analysis/save/membership1or4/SchoolingGraceFDEstimationResults.tex}};
%\input{c:/dropbox/data/ramadan/save/tablecolortemplate.tex}
\end{tikzpicture}\\
\renewcommand{\arraystretch}{.8}
\setlength{\tabcolsep}{1pt}
\begin{tabular}{>{\hfill\scriptsize}p{1cm}<{}>{\hfill\scriptsize}p{.25cm}<{}>{\scriptsize}p{12cm}<{\hfill}}
Source:& \multicolumn{2}{l}{\scriptsize Estimated with GUK administrative and survey data.}\\
Notes: & 1. & First-difference estimates. A first-difference is defined as $\Delta x_{t+1}\equiv x_{t+1} - x_{t}$. First-differenced regressands are regressed on categorical and time-variant covariates. Net saving is taken from administrative data and merged with survey data at Year-Month of survey interviews. Head age and literacy are from baseline data. Intercept terms are omitted in estimating equations. Net saving is saving - withdrawal. All dummy interaction terms are first demeaned and then interacted.\\
& 2. & ${}^{***}$, ${}^{**}$, ${}^{*}$ indicate statistical significance at 1\%, 5\%, 10\%, respetively. Standard errors are clustered at group (village) level.
\end{tabular}
\end{minipage}

\hspace{-1cm}\begin{minipage}[t]{14cm}
\hfil\textsc{\normalsize Table \refstepcounter{table}\thetable: FD estimation of school enrollment, small size vs. large size\label{tab FD enroll4}}\\
\setlength{\tabcolsep}{1pt}
\setlength{\baselineskip}{8pt}
\renewcommand{\arraystretch}{.55}
\hfil\begin{tikzpicture}
\node (tbl) {\input{c:/data/GUK/analysis/save/membership1or4/SchoolingSizeFDEstimationResults.tex}};
%\input{c:/dropbox/data/ramadan/save/tablecolortemplate.tex}
\end{tikzpicture}\\
\renewcommand{\arraystretch}{.8}
\setlength{\tabcolsep}{1pt}
\begin{tabular}{>{\hfill\scriptsize}p{1cm}<{}>{\hfill\scriptsize}p{.25cm}<{}>{\scriptsize}p{12cm}<{\hfill}}
Source:& \multicolumn{2}{l}{\scriptsize Estimated with GUK administrative and survey data.}\\
Notes: & 1. & First-difference estimates. A first-difference is defined as $\Delta x_{t+1}\equiv x_{t+1} - x_{t}$. First-differenced regressands are regressed on categorical and time-variant covariates. Net saving is taken from administrative data and merged with survey data at Year-Month of survey interviews. Head age and literacy are from baseline data. Intercept terms are omitted in estimating equations. Net saving is saving - withdrawal. All dummy interaction terms are first demeaned and then interacted.\\
& 2. & ${}^{***}$, ${}^{**}$, ${}^{*}$ indicate statistical significance at 1\%, 5\%, 10\%, respetively. Standard errors are clustered at group (village) level.
\end{tabular}
\end{minipage}


\hspace{-1cm}\begin{minipage}[t]{14cm}
\hfil\textsc{\normalsize Table \refstepcounter{table}\thetable: FD estimation of school enrollment, round 1 vs. round 4 differences\label{tab FD enroll5}}\\
\setlength{\tabcolsep}{1pt}
\setlength{\baselineskip}{8pt}
\renewcommand{\arraystretch}{.55}
\hfil\begin{tikzpicture}
\node (tbl) {\input{c:/data/GUK/analysis/save/membership1or4/SchoolingRd14DiffFDEstimationResults.tex}};
%\input{c:/dropbox/data/ramadan/save/tablecolortemplate.tex}
\end{tikzpicture}\\
\renewcommand{\arraystretch}{.8}
\setlength{\tabcolsep}{1pt}
\begin{tabular}{>{\hfill\scriptsize}p{1cm}<{}>{\hfill\scriptsize}p{.25cm}<{}>{\scriptsize}p{12cm}<{\hfill}}
Source:& \multicolumn{2}{l}{\scriptsize Estimated with GUK administrative and survey data.}\\
Notes: & 1. & First-difference estimates. A first-difference is defined as $\Delta x_{t+1}\equiv x_{t+1} - x_{t}$. First-differenced regressands are regressed on categorical and time-variant covariates. Net saving is taken from administrative data and merged with survey data at Year-Month of survey interviews. Head age and literacy are from baseline data. Intercept terms are omitted in estimating equations. Net saving is saving - withdrawal. All dummy interaction terms are first demeaned and then interacted.\\
& 2. & ${}^{***}$, ${}^{**}$, ${}^{*}$ indicate statistical significance at 1\%, 5\%, 10\%, respetively. Standard errors are clustered at group (village) level.
\end{tabular}
\end{minipage}


\hspace{-1cm}\begin{minipage}[t]{14cm}
\hfil\textsc{\normalsize Table \refstepcounter{table}\thetable: FD estimation of school enrollment, round 1 vs. round 4 differences, grace period\label{tab FD enroll6 14}}\\
\setlength{\tabcolsep}{1pt}
\setlength{\baselineskip}{8pt}
\renewcommand{\arraystretch}{.55}
\hfil\begin{tikzpicture}
\node (tbl) {\input{c:/data/GUK/analysis/save/membership1or4/SchoolingRd14DiffGraceFDEstimationResults.tex}};
%\input{c:/dropbox/data/ramadan/save/tablecolortemplate.tex}
\end{tikzpicture}\\
\renewcommand{\arraystretch}{.8}
\setlength{\tabcolsep}{1pt}
\begin{tabular}{>{\hfill\scriptsize}p{1cm}<{}>{\hfill\scriptsize}p{.25cm}<{}>{\scriptsize}p{12cm}<{\hfill}}
Source:& \multicolumn{2}{l}{\scriptsize Estimated with GUK administrative and survey data.}\\
Notes: & 1. & First-difference estimates. A first-difference is defined as $\Delta x_{t+1}\equiv x_{t+1} - x_{t}$. First-differenced regressands are regressed on categorical and time-variant covariates. Net saving is taken from administrative data and merged with survey data at Year-Month of survey interviews. Head age and literacy are from baseline data. Intercept terms are omitted in estimating equations. Net saving is saving - withdrawal. All dummy interaction terms are first demeaned and then interacted.\\
& 2. & ${}^{***}$, ${}^{**}$, ${}^{*}$ indicate statistical significance at 1\%, 5\%, 10\%, respetively. Standard errors are clustered at group (village) level.
\end{tabular}
\end{minipage}


\hspace{-1cm}\begin{minipage}[t]{14cm}
\hfil\textsc{\normalsize Table \refstepcounter{table}\thetable: FD estimation of school enrollment, period interactions\label{tab FD enroll6}}\\
\setlength{\tabcolsep}{1pt}
\setlength{\baselineskip}{8pt}
\renewcommand{\arraystretch}{.45}
\hfil\begin{tikzpicture}
\node (tbl) {\input{c:/data/GUK/analysis/save/membership1or4/SchoolingTIntFDEstimationResults1.tex}};
\end{tikzpicture}\\
\renewcommand{\arraystretch}{.8}
\setlength{\tabcolsep}{1pt}
\begin{tabular}{>{\hfill\scriptsize}p{1cm}<{}>{\hfill\scriptsize}p{.25cm}<{}>{\scriptsize}p{12cm}<{\hfill}}
Source:& \multicolumn{2}{l}{\scriptsize Estimated with GUK administrative and survey data.}\\
Notes: & 1. & First-difference estimates. A first-difference is defined as $\Delta x_{t+1}\equiv x_{t+1} - x_{t}$. First-differenced regressands are regressed on categorical and time-variant covariates. Net saving is taken from administrative data and merged with survey data at Year-Month of survey interviews. Head age and literacy are from baseline data. Intercept terms are omitted in estimating equations. Net saving is saving - withdrawal. All dummy interaction terms are first demeaned and then interacted.\\
& 2. & ${}^{***}$, ${}^{**}$, ${}^{*}$ indicate statistical significance at 1\%, 5\%, 10\%, respetively. Standard errors are clustered at group (village) level.
\end{tabular}
\end{minipage}

\hspace{-1cm}\begin{minipage}[t]{14cm}
\hfil\textsc{\normalsize Table \ref{tab FD enroll6}: FD estimation of school enrollment, period interactions, continued \label{tab FD enroll6c}}\\
\setlength{\tabcolsep}{1pt}
\setlength{\baselineskip}{8pt}
\renewcommand{\arraystretch}{.45}
\hfil\begin{tikzpicture}
\node (tbl) {\input{c:/data/GUK/analysis/save/membership1or4/SchoolingTIntFDEstimationResults2.tex}};
\end{tikzpicture}\\
\renewcommand{\arraystretch}{.8}
\setlength{\tabcolsep}{1pt}
\begin{tabular}{>{\hfill\scriptsize}p{1cm}<{}>{\hfill\scriptsize}p{.25cm}<{}>{\scriptsize}p{12cm}<{\hfill}}
Source:& \multicolumn{2}{l}{\scriptsize Estimated with GUK administrative and survey data.}\\
Notes: & 1. & First-difference estimates. A first-difference is defined as $\Delta x_{t+1}\equiv x_{t+1} - x_{t}$. First-differenced regressands are regressed on categorical and time-variant covariates. Net saving is taken from administrative data and merged with survey data at Year-Month of survey interviews. Head age and literacy are from baseline data. Intercept terms are omitted in estimating equations. Net saving is saving - withdrawal. All dummy interaction terms are first demeaned and then interacted.\\
& 2. & ${}^{***}$, ${}^{**}$, ${}^{*}$ indicate statistical significance at 1\%, 5\%, 10\%, respetively. Standard errors are clustered at group (village) level.
\end{tabular}
\end{minipage}

\hspace{-1cm}\begin{minipage}[t]{14cm}
\hfil\textsc{\normalsize Table \refstepcounter{table}\thetable: FD estimation of school enrollment, period interactions, grace period\label{tab FD enroll7}}\\
\setlength{\tabcolsep}{1pt}
\setlength{\baselineskip}{8pt}
\renewcommand{\arraystretch}{.5}
\hfil\begin{tikzpicture}
\node (tbl) {\input{c:/data/GUK/analysis/save/membership1or4/SchoolingTIntGraceFDEstimationResults.tex}};
%\input{c:/dropbox/data/ramadan/save/tablecolortemplate.tex}
\end{tikzpicture}\\
\renewcommand{\arraystretch}{.8}
\setlength{\tabcolsep}{1pt}
\begin{tabular}{>{\hfill\scriptsize}p{1cm}<{}>{\hfill\scriptsize}p{.25cm}<{}>{\scriptsize}p{12cm}<{\hfill}}
Source:& \multicolumn{2}{l}{\scriptsize Estimated with GUK administrative and survey data.}\\
Notes: & 1. & First-difference estimates. A first-difference is defined as $\Delta x_{t+1}\equiv x_{t+1} - x_{t}$. First-differenced regressands are regressed on categorical and time-variant covariates. Net saving is taken from administrative data and merged with survey data at Year-Month of survey interviews. Head age and literacy are from baseline data. Intercept terms are omitted in estimating equations. Net saving is saving - withdrawal. All dummy interaction terms are first demeaned and then interacted.\\
& 2. & ${}^{***}$, ${}^{**}$, ${}^{*}$ indicate statistical significance at 1\%, 5\%, 10\%, respetively. Standard errors are clustered at group (village) level.
\end{tabular}
\end{minipage}


\hspace{-1cm}\begin{minipage}[t]{14cm}
\hfil\textsc{\normalsize Table \refstepcounter{table}\thetable: FD estimation of school enrollment, period interactions, small vs. large sized loans\label{tab FD enroll8}}\\
\setlength{\tabcolsep}{1pt}
\setlength{\baselineskip}{8pt}
\renewcommand{\arraystretch}{.5}
\hfil\begin{tikzpicture}
\node (tbl) {\input{c:/data/GUK/analysis/save/membership1or4/SchoolingTIntSizeFDEstimationResults.tex}};
%\input{c:/dropbox/data/ramadan/save/tablecolortemplate.tex}
\end{tikzpicture}\\
\renewcommand{\arraystretch}{.8}
\setlength{\tabcolsep}{1pt}
\begin{tabular}{>{\hfill\scriptsize}p{1cm}<{}>{\hfill\scriptsize}p{.25cm}<{}>{\scriptsize}p{12cm}<{\hfill}}
Source:& \multicolumn{2}{l}{\scriptsize Estimated with GUK administrative and survey data.}\\
Notes: & 1. & First-difference estimates. A first-difference is defined as $\Delta x_{t+1}\equiv x_{t+1} - x_{t}$. First-differenced regressands are regressed on categorical and time-variant covariates. Net saving is taken from administrative data and merged with survey data at Year-Month of survey interviews. Head age and literacy are from baseline data. Intercept terms are omitted in estimating equations. Net saving is saving - withdrawal. All dummy interaction terms are first demeaned and then interacted.\\
& 2. & ${}^{***}$, ${}^{**}$, ${}^{*}$ indicate statistical significance at 1\%, 5\%, 10\%, respetively. Standard errors are clustered at group (village) level.
\end{tabular}
\end{minipage}


\subsection{Repayment and net saving}


\begin{Schunk}
\begin{Soutput}
Dropped 198 obs due to T<2.
Dropped 3084 obs due to NA.
\end{Soutput}
\begin{Soutput}
Dropped 198 obs due to T<2.
Dropped 3084 obs due to NA.
\end{Soutput}
\begin{Soutput}
   tee
Tee    2    3    4
  1    0   14   28
  2    0 1163 1163
  3    2    2    2
\end{Soutput}
\end{Schunk}
Repayment started in round 2. So taking a first-difference leaves us with period 2-3 and period 3-4. Drop 2 observations in \textsf{ar} that have round 1 data (for unknown reasons). After first-differencing, \textsf{ar} has 2372 rows with 42, 2326, 4 individuals with repeatedly observed for 2, 3, 4 times, respectively. 4 individuals observed for 4 times started repayment even before official disbursement date, so its round 1 will be dropped.

Plot repayment by date. (Simple plotting does not work because repayment is ighly erratic.)


Mean group repayment shortfall shows members, on average, repay by the end of 3rd year.


\mpage{\linewidth}{
\hfil\textsc{\footnotesize Figure \refstepcounter{figure}\thefigure: Weekly net saving \label{fig netsaving}}\\
\hfil\includegraphics{c:/data/GUK/analysis/program/figure/ImpactEstimationMembership1or4/CumulativeWeeklyNetSaving.png}\\
\renewcommand{\arraystretch}{1}
\hfil\begin{tabular}{>{\hfill\scriptsize}p{1cm}<{}>{\scriptsize}p{12cm}<{\hfill}}
Source: & Administrative data.\\
Note:& \\[1ex]
\end{tabular}

\hfil\textsc{\footnotesize Figure \refstepcounter{figure}\thefigure: Normalised cumulative weekly repayment \label{fig repayment}}\\
\hfil\includegraphics{c:/data/GUK/analysis/program/figure/ImpactEstimationMembership1or4/CumulativeWeeklyNormalisedRepayment.png}\\
\renewcommand{\arraystretch}{1}
\hfil\begin{tabular}{>{\hfill\scriptsize}p{1cm}<{}>{\scriptsize}p{12cm}<{\hfill}}
Source: & Administrative data.\\
Note:& Weekly repayments are normalised by the number of weeks in each month. \\[1ex]
\end{tabular}
}


Note all binary interaction terms are demeaned and then interacted.

NAs in \textsf{CumRepaid}.
\begin{Schunk}
\begin{Soutput}
      Arm
survey traditional large large grace cow
     1         398   400         398 400
     2         113    41           0  20
     3         110    39           0  19
     4          75    39           0   0
\end{Soutput}
\end{Schunk}
Tabulation at rd 1:
\begin{Schunk}
\begin{Soutput}
              RArm
Mstatus        traditional large large grace cow
  gErosion              40     0          20  20
  gRejection            80    40          20   0
  iRejection            54    12          22  72
  iReplacement           0     0           0   0
  newGroup               0     0           0   0
  oldMember            226   348         338 308
\end{Soutput}
\end{Schunk}











\hspace{-1cm}\begin{minipage}[t]{14cm}
\hfil\textsc{\normalsize Table \refstepcounter{table}\thetable: FD estimation of cumulative net saving and repayment\label{tab FD saving}}\\
\setlength{\tabcolsep}{1pt}
\setlength{\baselineskip}{8pt}
\renewcommand{\arraystretch}{.55}
\hfil\begin{tikzpicture}
\node (tbl) {\input{c:/data/GUK/analysis/save/membership1or4/SavingFDEstimationResults.tex}};
%\input{c:/dropbox/data/ramadan/save/tablecolortemplate.tex}
\end{tikzpicture}\\
\renewcommand{\arraystretch}{.8}
\setlength{\tabcolsep}{1pt}
\begin{tabular}{>{\hfill\scriptsize}p{1cm}<{}>{\hfill\scriptsize}p{.25cm}<{}>{\scriptsize}p{12cm}<{\hfill}}
Source:& \multicolumn{2}{l}{\scriptsize Estimated with GUK administrative and survey data.}\\
Notes: & 1. & First-difference estimates using rd 2 - rd 4 data. First-differenced ($\Delta x_{t+1}\equiv x_{t+1} - x_{t}$) regressands are regressed on categorical and time-variant covariates. Net saving is taken from administrative data and merged with survey data at Year-Month of survey interviews. Head age and literacy are from baseline data. Intercept terms are omitted in estimating equations. Net saving is saving - withdrawal. \\
& 2. & ${}^{***}$, ${}^{**}$, ${}^{*}$ indicate statistical significance at 1\%, 5\%, 10\%, respetively. Standard errors are clustered at group (village) level.
\end{tabular}
\end{minipage}

\hspace{-1cm}\begin{minipage}[t]{14cm}
\hfil\textsc{\normalsize Table \refstepcounter{table}\thetable: FD estimation of net cumulative saving and repayment, ultra poor vs. moderately poor\label{tab FD saving2}}\\
\setlength{\tabcolsep}{1pt}
\setlength{\baselineskip}{8pt}
\renewcommand{\arraystretch}{.55}
\hfil\begin{tikzpicture}
\node (tbl) {\input{c:/data/GUK/analysis/save/membership1or4/SavingPovertyStatusFDEstimationResults.tex}};
%\input{c:/dropbox/data/ramadan/save/tablecolortemplate.tex}
\end{tikzpicture}\\
\renewcommand{\arraystretch}{.8}
\setlength{\tabcolsep}{1pt}
\begin{tabular}{>{\hfill\scriptsize}p{1cm}<{}>{\hfill\scriptsize}p{.25cm}<{}>{\scriptsize}p{12cm}<{\hfill}}
Source:& \multicolumn{2}{l}{\scriptsize Estimated with GUK administrative and survey data.}\\
Notes: & 1. & First-difference estimates using rd 2 - rd 4 data. First-differenced ($\Delta x_{t+1}\equiv x_{t+1} - x_{t}$) regressands are regressed on categorical and time-variant covariates. Net saving is taken from administrative data and merged with survey data at Year-Month of survey interviews. Head age and literacy are from baseline data. Intercept terms are omitted in estimating equations. Net saving is saving - withdrawal. \\
& 2. & ${}^{***}$, ${}^{**}$, ${}^{*}$ indicate statistical significance at 1\%, 5\%, 10\%, respetively. Standard errors are clustered at group (village) level.
\end{tabular}
\end{minipage}

\hspace{-1cm}\begin{minipage}[t]{14cm}
\hfil\textsc{\normalsize Table \refstepcounter{table}\thetable: FD estimation of net cumulative saving and repayment, with vs. without a grace period\label{tab FD saving3}}\\
\setlength{\tabcolsep}{1pt}
\setlength{\baselineskip}{8pt}
\renewcommand{\arraystretch}{.55}
\hfil\begin{tikzpicture}
\node (tbl) {\input{c:/data/GUK/analysis/save/membership1or4/SavingGraceFDEstimationResults.tex}};
%\input{c:/dropbox/data/ramadan/save/tablecolortemplate.tex}
\end{tikzpicture}\\
\renewcommand{\arraystretch}{.8}
\setlength{\tabcolsep}{1pt}
\begin{tabular}{>{\hfill\scriptsize}p{1cm}<{}>{\hfill\scriptsize}p{.25cm}<{}>{\scriptsize}p{12cm}<{\hfill}}
Source:& \multicolumn{2}{l}{\scriptsize Estimated with GUK administrative and survey data.}\\
Notes: & 1. & First-difference estimates using rd 2 - rd 4 data. First-differenced ($\Delta x_{t+1}\equiv x_{t+1} - x_{t}$) regressands are regressed on categorical and time-variant covariates. Net saving is taken from administrative data and merged with survey data at Year-Month of survey interviews. Head age and literacy are from baseline data. Intercept terms are omitted in estimating equations. Net saving is saving - withdrawal. All dummy interaction terms are first demeaned and then interacted.\\
& 2. & ${}^{***}$, ${}^{**}$, ${}^{*}$ indicate statistical significance at 1\%, 5\%, 10\%, respetively. Standard errors are clustered at group (village) level.
\end{tabular}
\end{minipage}

\hspace{-1cm}\begin{minipage}[t]{14cm}
\hfil\textsc{\normalsize Table \refstepcounter{table}\thetable: FD estimation of net cumulative saving and repayment, small size vs. large size\label{tab FD saving4}}\\
\setlength{\tabcolsep}{1pt}
\setlength{\baselineskip}{8pt}
\renewcommand{\arraystretch}{.55}
\hfil\begin{tikzpicture}
\node (tbl) {\input{c:/data/GUK/analysis/save/membership1or4/SavingSizeFDEstimationResults.tex}};
%\input{c:/dropbox/data/ramadan/save/tablecolortemplate.tex}
\end{tikzpicture}\\
\renewcommand{\arraystretch}{.8}
\setlength{\tabcolsep}{1pt}
\begin{tabular}{>{\hfill\scriptsize}p{1cm}<{}>{\hfill\scriptsize}p{.25cm}<{}>{\scriptsize}p{12cm}<{\hfill}}
Source:& \multicolumn{2}{l}{\scriptsize Estimated with GUK administrative and survey data.}\\
Notes: & 1. & First-difference estimates using rd 2 - rd 4 data. First-differenced ($\Delta x_{t+1}\equiv x_{t+1} - x_{t}$) regressands are regressed on categorical and time-variant covariates. Net saving is taken from administrative data and merged with survey data at Year-Month of survey interviews. Head age and literacy are from baseline data. Intercept terms are omitted in estimating equations. Net saving is saving - withdrawal. All dummy interaction terms are first demeaned and then interacted.\\
& 2. & ${}^{***}$, ${}^{**}$, ${}^{*}$ indicate statistical significance at 1\%, 5\%, 10\%, respetively. Standard errors are clustered at group (village) level.
\end{tabular}
\end{minipage}


\begin{palepinkleftbar}
\begin{finding}
\textsc{\small Table \ref{tab FD saving}} (1) shows net saving increases, (2) shows that initially a larger then a smaller extent in the later rounds. This reduction may reflect the use of saving for repayment. \textsf{traditional} arm has the lowest repayment rates. Ultra poor and moderately poor have similar repayment rates as indicated in \textsc{\small Table \ref{tab FD saving2}}. \textsc{Table \ref{tab FD saving3}} (2) shows having a grace period increases the repayment amount while reduces net saving in later rounds. (4) and (5) show cumulative repayment is greater for with grace because each installment is larger. These are all by design that they do not repay in rd 1 so saving increases then they tap in these saving for repayment. 
\end{finding}
\end{palepinkleftbar}


\subsection{Assets}

Assets reportd in rd 1 is too small, indicating possible errors or different way of reporting only in rd 1. So we also examine rd 2 vs. rd 4 differences (\textsf{as3, as4}).


\begin{Schunk}
\begin{Soutput}
Dropped 196 obs due to T<2.
Dropped 1402 obs due to NA.
Dropped 196 obs due to T<2.
Dropped 1402 obs due to NA.
Dropped 196 obs due to T<2.
Dropped 1402 obs due to NA.
Dropped 196 obs due to T<2.
Dropped 1402 obs due to NA.
Dropped 130 obs due to T<2.
Dropped 1274 obs due to NA.
Dropped 130 obs due to T<2.
Dropped 1274 obs due to NA.
Dropped 130 obs due to T<2.
Dropped 1274 obs due to NA.
Dropped 130 obs due to T<2.
Dropped 1274 obs due to NA.
\end{Soutput}
\end{Schunk}

Main assets are household assets (\textsf{HAssetAmount}) and production assets (\textsf{PAssetAmount}) both with 5648 observations. After first-differencing, they become 4050 observations, with 28, 200, 3822 households observed for 2, 3, 4 times. We also examine rd 2 vs. rd 4 differences, which has 2678 observations. After first-differencing, they become 1274 observations.


\begin{Schunk}
\begin{Sinput}
source(paste0(pathprogram, "AssetCovariateSelection.R"))
\end{Sinput}
\end{Schunk}

\begin{Schunk}
\begin{Sinput}
# Compare asset changes between arms and "pure control" (loan nonreceivers)
source(paste0(pathprogram, "AssetCovariateSelectionRobustness.R"))
\end{Sinput}
\end{Schunk}



\hspace{-1cm}\begin{minipage}[t]{14cm}
\hfil\textsc{\normalsize Table \refstepcounter{table}\thetable: FD estimation of assets\label{tab FD assets}}\\
\setlength{\tabcolsep}{1pt}
\setlength{\baselineskip}{8pt}
\renewcommand{\arraystretch}{.55}
\hfil\begin{tikzpicture}
\node (tbl) {\input{c:/data/GUK/analysis/save/membership1or4/AssetFDEstimationResults.tex}};
%\input{c:/dropbox/data/ramadan/save/tablecolortemplate.tex}
\end{tikzpicture}\\
\renewcommand{\arraystretch}{.8}
\setlength{\tabcolsep}{1pt}
\begin{tabular}{>{\hfill\scriptsize}p{1cm}<{}>{\hfill\scriptsize}p{.25cm}<{}>{\scriptsize}p{12cm}<{\hfill}}
Source:& \multicolumn{2}{l}{\scriptsize Estimated with GUK administrative and survey data.}\\
Notes: & 1. & First-difference estimates. A first-difference is defined as $\Delta x_{t+k}\equiv x_{t+k} - x_{t}$  for $k=1, 2, \dots$. Saving and repayment misses are taken from administrative data and merged with survey data at Year-Month of survey interviews. Intercept terms are omitted in estimating equations. Sample is continuing members and replacing members of early rejecters and received loans prior to 2015 Janunary. Household assets do not include livestock. Regressions (1)-(3), (5)-(6) use only arm and calendar information. (4) and (7) use previous six month repayment and saving information which is lacking in rd 1, hence starts from rd 2.\\
& 2. & ${}^{***}$, ${}^{**}$, ${}^{*}$ indicate statistical significance at 1\%, 5\%, 10\%, respetively. Standard errors are clustered at group (village) level.
\end{tabular}
\end{minipage}

\hspace{-1cm}\begin{minipage}[t]{14cm}
\hfil\textsc{\normalsize Table \refstepcounter{table}\thetable: FD estimation of assets, moderately poor vs. ultra poor\label{tab FD assets2}}\\
\setlength{\tabcolsep}{1pt}
\setlength{\baselineskip}{8pt}
\renewcommand{\arraystretch}{.55}
\hfil\begin{tikzpicture}
\node (tbl) {\input{c:/data/GUK/analysis/save/membership1or4/AssetPovertyStatusFDEstimationResults.tex}};
%\input{c:/dropbox/data/ramadan/save/tablecolortemplate.tex}
\end{tikzpicture}\\
\renewcommand{\arraystretch}{.8}
\setlength{\tabcolsep}{1pt}
\begin{tabular}{>{\hfill\scriptsize}p{1cm}<{}>{\hfill\scriptsize}p{.25cm}<{}>{\scriptsize}p{12cm}<{\hfill}}
Source:& \multicolumn{2}{l}{\scriptsize Estimated with GUK administrative and survey data.}\\
Notes: & 1. & First-difference estimates. A first-difference is defined as $\Delta x_{t+k}\equiv x_{t+k} - x_{t}$  for $k=1, 2, \dots$. Saving and repayment misses are taken from administrative data and merged with survey data at Year-Month of survey interviews. Intercept terms are omitted in estimating equations. Sample is continuing members and replacing members of early rejecters and received loans prior to 2015 Janunary. Household assets do not include livestock. \\
& 2. & ${}^{***}$, ${}^{**}$, ${}^{*}$ indicate statistical significance at 1\%, 5\%, 10\%, respetively. Standard errors are clustered at group (village) level.
\end{tabular}
\end{minipage}

\hspace{-1cm}\begin{minipage}[t]{14cm}
\hfil\textsc{\normalsize Table \refstepcounter{table}\thetable: FD estimation of assets, small vs. large size loans\label{tab FD assets3}}\\
\setlength{\tabcolsep}{1pt}
\setlength{\baselineskip}{8pt}
\renewcommand{\arraystretch}{.55}
\hfil\begin{tikzpicture}
\node (tbl) {\input{c:/data/GUK/analysis/save/membership1or4/AssetSizeFDEstimationResults.tex}};
%\input{c:/dropbox/data/ramadan/save/tablecolortemplate.tex}
\end{tikzpicture}\\
\renewcommand{\arraystretch}{.8}
\setlength{\tabcolsep}{1pt}
\begin{tabular}{>{\hfill\scriptsize}p{1cm}<{}>{\hfill\scriptsize}p{.25cm}<{}>{\scriptsize}p{12cm}<{\hfill}}
Source:& \multicolumn{2}{l}{\scriptsize Estimated with GUK administrative and survey data.}\\
Notes: & 1. & First-difference estimates. A first-difference is defined as $\Delta x_{t+k}\equiv x_{t+k} - x_{t}$  for $k=1, 2, \dots$. Saving and repayment misses are taken from administrative data and merged with survey data at Year-Month of survey interviews. Intercept terms are omitted in estimating equations. Sample is continuing members and replacing members of early rejecters and received loans prior to 2015 Janunary. Household assets do not include livestock. \\
& 2. & ${}^{***}$, ${}^{**}$, ${}^{*}$ indicate statistical significance at 1\%, 5\%, 10\%, respetively. Standard errors are clustered at group (village) level.
\end{tabular}
\end{minipage}

\hspace{-1cm}\begin{minipage}[t]{14cm}
\hfil\textsc{\normalsize Table \refstepcounter{table}\thetable: FD estimation of assets, with vs. without a grace period\label{tab FD assets4}}\\
\setlength{\tabcolsep}{1pt}
\setlength{\baselineskip}{8pt}
\renewcommand{\arraystretch}{.55}
\hfil\begin{tikzpicture}
\node (tbl) {\input{c:/data/GUK/analysis/save/membership1or4/AssetGraceFDEstimationResults.tex}};
%\input{c:/dropbox/data/ramadan/save/tablecolortemplate.tex}
\end{tikzpicture}\\
\renewcommand{\arraystretch}{.8}
\setlength{\tabcolsep}{1pt}
\begin{tabular}{>{\hfill\scriptsize}p{1cm}<{}>{\hfill\scriptsize}p{.25cm}<{}>{\scriptsize}p{12cm}<{\hfill}}
Source:& \multicolumn{2}{l}{\scriptsize Estimated with GUK administrative and survey data.}\\
Notes: & 1. & First-difference estimates. A first-difference is defined as $\Delta x_{t+k}\equiv x_{t+k} - x_{t}$  for $k=1, 2, \dots$. Saving and repayment misses are taken from administrative data and merged with survey data at Year-Month of survey interviews. Intercept terms are omitted in estimating equations. Sample is continuing members and replacing members of early rejecters and received loans prior to 2015 Janunary. Household assets do not include livestock. \\
& 2. & ${}^{***}$, ${}^{**}$, ${}^{*}$ indicate statistical significance at 1\%, 5\%, 10\%, respetively. Standard errors are clustered at group (village) level.
\end{tabular}
\end{minipage}

\hspace{-1cm}\begin{minipage}[t]{14cm}
\hfil\textsc{\normalsize Table \refstepcounter{table}\thetable: FD estimation of assets, round 2 and 4 comparison\label{tab FD assets rd24}}\\
\setlength{\tabcolsep}{1pt}
\setlength{\baselineskip}{8pt}
\renewcommand{\arraystretch}{.55}
\hfil\begin{tikzpicture}
\node (tbl) {\input{c:/data/GUK/analysis/save/membership1or4/AssetRd24DiffFDEstimationResults.tex}};
%\input{c:/dropbox/data/ramadan/save/tablecolortemplate.tex}
\end{tikzpicture}\\
\renewcommand{\arraystretch}{.8}
\setlength{\tabcolsep}{1pt}
\begin{tabular}{>{\hfill\scriptsize}p{1cm}<{}>{\hfill\scriptsize}p{.25cm}<{}>{\scriptsize}p{12cm}<{\hfill}}
Source:& \multicolumn{2}{l}{\scriptsize Estimated with GUK administrative and survey data.}\\
Notes: & 1. & First-difference estimates between round 2 and 4. A first-difference is defined as $\Delta x_{t+k}\equiv x_{t+k} - x_{t}$ for $k=1, 2, \dots$. Saving and repayment misses are taken from administrative data and merged with survey data at Year-Month of survey interviews. Intercept terms are omitted in estimating equations. Sample is continuing members and replacing members of early rejecters and received loans prior to 2015 Janunary. Household assets do not include livestock. Regressions (1)-(3), (5)-(6) use only arm and calendar information. (4) and (7) use previous six month repayment and saving information which is lacking in rd 1, hence starts from rd 2.\\
& 2. & ${}^{***}$, ${}^{**}$, ${}^{*}$ indicate statistical significance at 1\%, 5\%, 10\%, respetively. Standard errors are clustered at group (village) level.
\end{tabular}
\end{minipage}

\hspace{-1cm}\begin{minipage}[t]{14cm}
\hfil\textsc{\normalsize Table \refstepcounter{table}\thetable: FD estimation of assets, round 2 and 4 comparison, grace period\label{tab FD assets rd24 grace}}\\
\setlength{\tabcolsep}{1pt}
\setlength{\baselineskip}{8pt}
\renewcommand{\arraystretch}{.55}
\hfil\begin{tikzpicture}
\node (tbl) {\input{c:/data/GUK/analysis/save/membership1or4/AssetRd24GraceFDEstimationResults.tex}};
%\input{c:/dropbox/data/ramadan/save/tablecolortemplate.tex}
\end{tikzpicture}\\
\renewcommand{\arraystretch}{.8}
\setlength{\tabcolsep}{1pt}
\begin{tabular}{>{\hfill\scriptsize}p{1cm}<{}>{\hfill\scriptsize}p{.25cm}<{}>{\scriptsize}p{12cm}<{\hfill}}
Source:& \multicolumn{2}{l}{\scriptsize Estimated with GUK administrative and survey data.}\\
Notes: & 1. & First-difference estimates between round 2 and 4. A first-difference is defined as $\Delta x_{t+k}\equiv x_{t+k} - x_{t}$ for $k=1, 2, \dots$. Saving and repayment misses are taken from administrative data and merged with survey data at Year-Month of survey interviews. Intercept terms are omitted in estimating equations. Sample is continuing members and replacing members of early rejecters and received loans prior to 2015 Janunary. Household assets do not include livestock. Regressions (1)-(3), (5)-(6) use only arm and calendar information. (4) and (7) use previous six month repayment and saving information which is lacking in rd 1, hence starts from rd 2.\\
& 2. & ${}^{***}$, ${}^{**}$, ${}^{*}$ indicate statistical significance at 1\%, 5\%, 10\%, respetively. Standard errors are clustered at group (village) level.
\end{tabular}
\end{minipage}


Robustness: \textsc{\small Table \ref{tab FD assets pure control}} shows that the pure controls also experience similar increase-increase-decrease pattern. This suggests the pattern observed among the loan recipients may be a systemic pattern of the area, not necessarily reflecting the repayment burdern.

\hspace{-1cm}\begin{minipage}[t]{14cm}
\hfil\textsc{\normalsize Table \refstepcounter{table}\thetable: FD estimation of assets, loan recipients vs. pure control\label{tab FD assets pure control}}\\
\setlength{\tabcolsep}{1pt}
\setlength{\baselineskip}{8pt}
\renewcommand{\arraystretch}{.55}
\hfil\begin{tikzpicture}
\node (tbl) {\input{c:/data/GUK/analysis/save/membership1or4/AssetRobustnessFDEstimationResults.tex}};
%\input{c:/dropbox/data/ramadan/save/tablecolortemplate.tex}
\end{tikzpicture}\\
\renewcommand{\arraystretch}{.8}
\setlength{\tabcolsep}{1pt}
\begin{tabular}{>{\hfill\scriptsize}p{1cm}<{}>{\hfill\scriptsize}p{.25cm}<{}>{\scriptsize}p{12cm}<{\hfill}}
Source:& \multicolumn{2}{l}{\scriptsize Estimated with GUK administrative and survey data.}\\
Notes: & 1. & First-difference estimates between round 2 and 4. A first-difference is defined as $\Delta x_{t+k}\equiv x_{t+k} - x_{t}$ for $k=1, 2, \dots$. Saving and repayment misses are taken from administrative data and merged with survey data at Year-Month of survey interviews. Intercept terms are omitted in estimating equations. Sample is continuing members and replacing members of early rejecters and received loans prior to 2015 Janunary. Household assets do not include livestock. Regressions (1)-(3), (5)-(6) use only arm and calendar information. (4) and (7) use previous six month repayment and saving information which is lacking in rd 1, hence starts from rd 2. Pure controls are households who rejected to receive a loan.\\
& 2. & ${}^{***}$, ${}^{**}$, ${}^{*}$ indicate statistical significance at 1\%, 5\%, 10\%, respetively. Standard errors are clustered at group (village) level.
\end{tabular}
\end{minipage}


\begin{palepinkleftbar}
\begin{finding}
\textsc{\small Table \ref{tab FD assets}} (1) shows household assets increase after receiving the loans in all arms. Total incremant is largest among the \textsf{large grace} arm as indicated in (2). In (3), increments are positive in rd 2 - 3, suggesting substantial purchase after receiving a loan. Significant decreases in rd 3 - 4 for all arms indicate liquidation of assets for repayment. Productive assets of large size loan arms decrease in rd 3 - 4 in \textsc{\small Table \ref{tab FD assets3}} (6). These may indicate forced liquidation for repayment, which can entail efficiency losses.
\end{finding}
\end{palepinkleftbar}



\subsection{Livestock}

\subsubsection{Livestock}


\begin{Schunk}
\begin{Soutput}
   Arm
tee traditional large large grace cow
  1         398   399         379 398
  2         283   390         373 379
  3         276   384         348 365
  4         238   377         330 328
\end{Soutput}
\begin{Soutput}
   Arm
tee traditional large large grace cow
  1          66    78          81  63
  2         151   254         258 283
  3         189   348         323 324
  4         156   328         291 287
\end{Soutput}
\begin{Soutput}
Dropped 196 obs due to T<2.
Dropped 1402 obs due to NA.
Dropped 196 obs due to T<2.
Dropped 1402 obs due to NA.
Dropped 154 obs due to T<2.
Dropped 1272 obs due to NA.
Dropped 154 obs due to T<2.
Dropped 1272 obs due to NA.
\end{Soutput}
\end{Schunk}
An example FD estimation.
\begin{Schunk}
\begin{Soutput}

Call:
lm(formula = TotalImputedValue ~ dummyWithGrace + Time.2 + dummyWithGrace.Time2 + 
    Time.3 + dummyWithGrace.Time3, data = dat)

Residuals:
   Min     1Q Median     3Q    Max 
-88909  -9129  -3499   9146  76281 

Coefficients:
                     Estimate Std. Error t value Pr(>|t|)  
(Intercept)              2281       4801    0.48    0.636  
dummyWithGrace           1759       6048    0.29    0.772  
Time.2                  16276       6902    2.36    0.021 *
dummyWithGrace.Time2     6862      13658    0.50    0.617  
Time.3                   7126       7034    1.01    0.315  
dummyWithGrace.Time3     2911      13905    0.21    0.835  
---
Signif. codes:  0 '***' 0.001 '**' 0.01 '*' 0.05 '.' 0.1 ' ' 1

Residual standard error: 21900 on 66 degrees of freedom
Multiple R-squared:  0.0855,	Adjusted R-squared:  0.0162 
F-statistic: 1.23 on 5 and 66 DF,  p-value: 0.303
\end{Soutput}
\end{Schunk}
%Estimates are negative on Time.2 and dummyWithGrace.Time2 because period 1-2 growth was very high so any growth in period 2-3 (Time.2) is reduction from it.

\begin{Schunk}
\begin{Sinput}
source(paste0(pathprogram, "LivestockCovariateSelection.R"))
\end{Sinput}
\end{Schunk}

\begin{Schunk}
\begin{figure}

{\centering \includegraphics[width=\maxwidth]{figure/ImpactEstimationMembership1or4/Total_number_of_cows_by_cow_holding_at_rd_1-1} 

}

\caption{Number of cows owned\\ {\footnotesize Cow ownership is defined at the baseline.\setlength{\baselineskip}{8pt}}}\label{Figure Total number of cows by cow holding at rd 1}
\end{figure}
\end{Schunk}
\begin{Schunk}
\begin{Soutput}
, , Arm = traditional

   dummyHadCows
tee do not own a cow own a cow
  1              332        66
  2              236        47
  3              230        46
  4              197        41

, , Arm = large

   dummyHadCows
tee do not own a cow own a cow
  1              323        76
  2              314        76
  3              309        75
  4              302        75

, , Arm = large grace

   dummyHadCows
tee do not own a cow own a cow
  1              299        80
  2              295        78
  3              274        74
  4              259        71

, , Arm = cow

   dummyHadCows
tee do not own a cow own a cow
  1              335        63
  2              316        63
  3              303        62
  4              279        49
\end{Soutput}
\begin{figure}

{\centering \includegraphics[width=\maxwidth]{figure/ImpactEstimationMembership1or4/Fig_cows-1} 

}

\caption{Number of cows owned by loan receipt\\ {\footnotesize Cow ownership is defined at the baseline.\setlength{\baselineskip}{8pt}}}\label{Figure Fig cows}
\end{figure}
\end{Schunk}
\clearpage
\hspace{-1cm}\begin{minipage}[t]{14cm}
\hfil\textsc{\normalsize Table \refstepcounter{table}\thetable: FD estimation of livestock holding values\label{tab FD livestock}}\\
\setlength{\tabcolsep}{1pt}
\setlength{\baselineskip}{8pt}
\renewcommand{\arraystretch}{.55}
\hfil\begin{tikzpicture}
\node (tbl) {\input{c:/data/GUK/analysis/save/membership1or4/LivestockFDEstimationResults.tex}};
%\input{c:/dropbox/data/ramadan/save/tablecolortemplate.tex}
\end{tikzpicture}\\
\renewcommand{\arraystretch}{.8}
\setlength{\tabcolsep}{1pt}
\begin{tabular}{>{\hfill\scriptsize}p{1cm}<{}>{\hfill\scriptsize}p{.25cm}<{}>{\scriptsize}p{12cm}<{\hfill}}
Source:& \multicolumn{2}{l}{\scriptsize Estimated with GUK administrative and survey data.}\\
Notes: & 1. & First-difference estimates. A first-difference is defined as $\Delta x_{t+1}\equiv x_{t+1} - x_{t}$. Saving and repayment misses are taken from administrative data and merged with survey data at Year-Month of survey interviews. Intercept terms are omitted in estimating equations. Sample is continuing members and replacing members of early rejecters and received loans prior to 2015 Janunary. Regressand is \textsf{TotalImputedValue}, a sum of all livestock holding values evaluated at respective median market prices in the same year. \\
& 2. & ${}^{***}$, ${}^{**}$, ${}^{*}$ indicate statistical significance at 1\%, 5\%, 10\%, respetively. Standard errors are clustered at group (village) level.
\end{tabular}
\end{minipage}

\hspace{-1cm}\begin{minipage}[t]{14cm}
\hfil\textsc{\normalsize Table \refstepcounter{table}\thetable: FD estimation of livestock holding values, with vs. without a grace period\label{tab FD livestock2}}\\
\setlength{\tabcolsep}{1pt}
\setlength{\baselineskip}{8pt}
\renewcommand{\arraystretch}{.55}
\hfil\begin{tikzpicture}
\node (tbl) {\input{c:/data/GUK/analysis/save/membership1or4/LivestockGraceFDEstimationResults.tex}};
%\input{c:/dropbox/data/ramadan/save/tablecolortemplate.tex}
\end{tikzpicture}\\
\renewcommand{\arraystretch}{.8}
\setlength{\tabcolsep}{1pt}
\begin{tabular}{>{\hfill\scriptsize}p{1cm}<{}>{\hfill\scriptsize}p{.25cm}<{}>{\scriptsize}p{12cm}<{\hfill}}
Source:& \multicolumn{2}{l}{\scriptsize Estimated with GUK administrative and survey data.}\\
Notes: & 1. & First-difference estimates. A first-difference is defined as $\Delta x_{t+1}\equiv x_{t+1} - x_{t}$. Saving and repayment misses are taken from administrative data and merged with survey data at Year-Month of survey interviews. Intercept terms are omitted in estimating equations. Sample is continuing members and replacing members of early rejecters and received loans prior to 2015 Janunary. Regressand is \textsf{TotalImputedValue}, a sum of all livestock holding values evaluated at respective median market prices in the same year. \\
& 2. & ${}^{***}$, ${}^{**}$, ${}^{*}$ indicate statistical significance at 1\%, 5\%, 10\%, respetively. Standard errors are clustered at group (village) level.
\end{tabular}
\end{minipage}

\hspace{-1cm}\begin{minipage}[t]{14cm}
\hfil\textsc{\normalsize Table \refstepcounter{table}\thetable: FD estimation of livestock holding values, rd 1 vs. rd 4 comparison\label{tab FD livestock3}}\\
\setlength{\tabcolsep}{1pt}
\setlength{\baselineskip}{8pt}
\renewcommand{\arraystretch}{.55}
\hfil\begin{tikzpicture}
\node (tbl) {\input{c:/data/GUK/analysis/save/membership1or4/LivestockRd14DiffFDEstimationResults.tex}};
%\input{c:/dropbox/data/ramadan/save/tablecolortemplate.tex}
\end{tikzpicture}\\
\renewcommand{\arraystretch}{.8}
\setlength{\tabcolsep}{1pt}
\begin{tabular}{>{\hfill\scriptsize}p{1cm}<{}>{\hfill\scriptsize}p{.25cm}<{}>{\scriptsize}p{12cm}<{\hfill}}
Source:& \multicolumn{2}{l}{\scriptsize Estimated with GUK administrative and survey data.}\\
Notes: & 1. & First-difference estimates. A first-difference is defined as $\Delta x_{t+1}\equiv x_{t+1} - x_{t}$. Saving and repayment misses are taken from administrative data and merged with survey data at Year-Month of survey interviews. Intercept terms are omitted in estimating equations. Sample is continuing members and replacing members of early rejecters and received loans prior to 2015 Janunary. Regressand is \textsf{TotalImputedValue}, a sum of all livestock holding values evaluated at respective median market prices in the same year. \\
& 2. & ${}^{***}$, ${}^{**}$, ${}^{*}$ indicate statistical significance at 1\%, 5\%, 10\%, respetively. Standard errors are clustered at group (village) level.
\end{tabular}
\end{minipage}

Check quickly if the estimated results make sense. \textsf{xid} is from the file I received as updated id file (Oct 18, 2017).
\begin{Schunk}
\begin{Soutput}
                      min   25\\%  median   75\\%      max       mean
hhid              7010101 7042116 7065014 8147812 81710520 12189046.0
TotalImputedValue       0       0   20000   40000   300000    19913.8
                         std   0s NAs    n
hhid              18472118.6    0   0 5645
TotalImputedValue    22457.6 1596   0 5645
\end{Soutput}
\end{Schunk}
\begin{Schunk}
\begin{figure}

{\centering \includegraphics[width=\maxwidth]{figure/ImpactEstimationMembership1or4/Total_imputed_value_histogram-1} 

}

\caption{Total imputed value of livestock holding\\ {\footnotesize Livestock holding values are computed by using respective median prices of each year.\setlength{\baselineskip}{8pt}}}\label{Figure Total imputed value histogram}
\end{figure}
\end{Schunk}
\begin{Schunk}
\begin{figure}

{\centering \includegraphics[width=\maxwidth]{figure/ImpactEstimationMembership1or4/Histogram_of_livestock_holding_classes-1} 

}

\caption{Histogram of livestock holding classes\\ {\footnotesize Livestock holding values are computed by using respective median prices of each year.\setlength{\baselineskip}{8pt}}}\label{Figure Histogram of livestock holding classes}
\end{figure}
\end{Schunk}
\begin{Schunk}
\begin{figure}

{\centering \includegraphics[width=\maxwidth]{figure/ImpactEstimationMembership1or4/Histogram_of_livestock_holding_classes_with_attrition-1} 

}

\caption{Histogram of livestock holding classes\\ {\footnotesize Livestock holding values are computed by using respective median prices of each year. Boxes without a colour accounts for attrited households.\setlength{\baselineskip}{8pt}}}\label{Figure Histogram of livestock holding classes with attrition}
\end{figure}
\end{Schunk}
\begin{Schunk}
\begin{figure}

{\centering \includegraphics[width=\maxwidth]{figure/ImpactEstimationMembership1or4/Histogram_of_livestock_holding_classes_by_year-1} 

}

\caption{Histogram of livestock holding classes by year\\ {\footnotesize Livestock holding values are computed by using respective median prices of each year.\setlength{\baselineskip}{8pt}}}\label{Figure Histogram of livestock holding classes by year}
\end{figure}
\end{Schunk}
\begin{Schunk}
\begin{figure}

{\centering \includegraphics[width=\maxwidth]{figure/ImpactEstimationMembership1or4/Histogram_of_livestock_holding_classes_with_attrition_by_year-1} 

}

\caption{Histogram of livestock holding classes by year\\ {\footnotesize Livestock holding values are computed by using respective median prices of each year. Boxes without a colour accounts for attrited households.\setlength{\baselineskip}{8pt}}}\label{Figure Histogram of livestock holding classes with attrition by year}
\end{figure}
\end{Schunk}
\begin{itemize}
\vspace{1.0ex}\setlength{\itemsep}{1.0ex}\setlength{\baselineskip}{12pt}
\item	Why does \textsf{cow} report below 1000 holding in rds 2-4?
\end{itemize}
\begin{Schunk}
\begin{Soutput}
            Arm survey MeanImputedVal MeanNumCows   N
 1: traditional      1        5065.33    0.233668 398
 2: traditional      2       15854.00    0.817844 280
 3: traditional      3       20179.62    1.022059 277
 4: traditional      4       21233.75    1.050000 240
 5:       large      1        6092.42    0.275689 399
 6:       large      3       31056.41    1.625000 386
 7:       large      2       24992.86    1.278820 383
 8:       large      4       32686.07    1.630890 382
 9: large grace      1        7392.54    0.333333 399
10: large grace      2       21510.32    1.150943 341
11: large grace      3       27565.65    1.422619 347
12: large grace      4       30276.97    1.528024 343
13:         cow      1        4997.68    0.218045 399
14:         cow      2       20550.29    1.078035 364
15:         cow      3       25399.62    1.300562 365
16:         cow      4       28700.23    1.436950 342
\end{Soutput}
\end{Schunk}
\begin{Schunk}
\begin{figure}

{\centering \includegraphics[width=\maxwidth]{figure/ImpactEstimationMembership1or4/Number_of_cows_by_year-1} 

}

\caption{Number of cows/oxen by year\\ {\footnotesize Means are mean holding among the owners. Totals are total number of cows/oxen owned. Mean and total number of cows/oxen may diverge because the number of owners differ across round.\setlength{\baselineskip}{8pt}}}\label{Figure Number of cows by year}
\end{figure}
\end{Schunk}
\begin{Schunk}
\begin{figure}

{\centering \includegraphics[width=\maxwidth]{figure/ImpactEstimationMembership1or4/Number_of_cows_by_rd-1} 

}

\caption{Number of cows/oxen by survey round\\ {\footnotesize Means are mean holding among the owners. Totals are total number of cows/oxen owned. Mean and total number of cows/oxen may diverge because the number of owners differ across round.\setlength{\baselineskip}{8pt}}}\label{Figure Number of cows by rd}
\end{figure}
\end{Schunk}
\begin{palepinkleftbar}
\begin{finding}
\textsc{\small Figure \ref{Figure Total imputed value histogram}} shows a general increase in upper holding classes round 3 and further upper holding classes in round 4. \textsc{\small Figure \ref{Figure Number of cows by year}} shows livestock type is not entered (yet collected) in rd3. At this moment, one needs to omit rd 3. All estimation results by far are subject to this omission.
\end{finding}
\end{palepinkleftbar}

\subsubsection{Traditional arm households who buy cows}

Some traditional arm households buy cows. Characteristics of these households.


\begin{Schunk}
\begin{Soutput}
Warning in `[.data.table`(lvNLl1, , `:=`(Variable, "total")): Invalid .internal.selfref detected and fixed by taking a (shallow) copy of the data.table so that := can add this new column by reference. At an earlier point, this data.table has been copied by R (or was created manually using structure() or similar). Avoid key<-, names<- and attr<- which in R currently (and oddly) may copy the whole data.table. Use set* syntax instead to avoid copying: ?set, ?setnames and ?setattr. If this message doesn't help, please report your use case to the data.table issue tracker so the root cause can be fixed or this message improved.
\end{Soutput}
\begin{Soutput}
Warning in `[.data.table`(lvNLlS, , `:=`(Variable, "solddied")): Invalid .internal.selfref detected and fixed by taking a (shallow) copy of the data.table so that := can add this new column by reference. At an earlier point, this data.table has been copied by R (or was created manually using structure() or similar). Avoid key<-, names<- and attr<- which in R currently (and oddly) may copy the whole data.table. Use set* syntax instead to avoid copying: ?set, ?setnames and ?setattr. If this message doesn't help, please report your use case to the data.table issue tracker so the root cause can be fixed or this message improved.
\end{Soutput}
\begin{Soutput}
Warning in `[.data.table`(lvNLlB, , `:=`(Variable, "born")): Invalid .internal.selfref detected and fixed by taking a (shallow) copy of the data.table so that := can add this new column by reference. At an earlier point, this data.table has been copied by R (or was created manually using structure() or similar). Avoid key<-, names<- and attr<- which in R currently (and oddly) may copy the whole data.table. Use set* syntax instead to avoid copying: ?set, ?setnames and ?setattr. If this message doesn't help, please report your use case to the data.table issue tracker so the root cause can be fixed or this message improved.
\end{Soutput}
\end{Schunk}
Merge non receivers and receivers of loans.

Number of HHs in \textsf{traditional} arm with inconsistent (e.g., rd1 2 cows, rd2 0 cow, rd3 2 cows, etc.) reporting of cow ownership across rounds:
\begin{Schunk}
\begin{Soutput}
[1] 216
\end{Soutput}
\end{Schunk}

There are 1598 HHs in the livestock data, of which: 755 increased and did not decrease cow ownership (strictly increasing), 416 increased and also decreased cow ownership, and 427 did not increase (decreased or no change), totaling to 1598. Within \textsf{traditional} arm, there are 164 HHs without a cow throughout the survey periods, and 234 HHs with a cow at least once. Among the HHs with a cow at least once, 
73 
[40 with a loan, 
33 without a loan] 
increased and did not decrease cow ownership (strictly increasing), 
69 increased and also decreased cow ownership [
33 with a loan, 
36 without a loan], and 
92 did not increase (decreased or no change) [
28 with a loan, 
64 without a loan], 
totaling to 
234.

1195 and 403 HHs received and did not receive a loan, respectively.


Number of HHs in \textsf{traditional} arm with an increase in cows at least once among nonzero cow ownership:
\begin{Schunk}
\begin{Soutput}
[1] 142
\end{Soutput}
\end{Schunk}
Number of HHs in \textsf{traditional} arm with an increase in cows at least once and with a decrease in in cows at least once:
\begin{Schunk}
\begin{Soutput}
[1] 69
\end{Soutput}
\end{Schunk}
There are many increase-decrease-increase patterns in cow ownership, which implies puchase-sales-puchase... which is implausible. Checking against livestock sales data. \textsf{CowOx.diff} is contemporaneous difference between births and sales/deaths, \textsf{CowOx.totdiff} is a first-difference of \textsf{CowOx.total} which is a sum of livestock holding by cow/ox, goat/sheep, chicken/duck. %\textsf{DCow} = \textsf{CowOx.totdiff}.
\begin{Schunk}
\begin{Soutput}
             Arm     hhid survey CowOx.born CowOx.solddied CowOx.total
  1: traditional  7020802      2          0              0           2
  2: traditional  7020804      2          0              0           0
  3: traditional  7020804      3          0              0           1
  4: traditional  7020804      4          0              0           0
  5: traditional  7020806      2          0              0           2
 ---                                                                  
347: traditional 81710219      2          0              0           0
348: traditional 81710219      3          0              0           0
349: traditional 81710219      4          0              0           0
350: traditional 81710220      3          0              0           0
351: traditional 81710220      4          0              0           0
     CowOx.diff CowOx.totdiff CowOx.inconsistent NoCow
  1:          0             2                  2 FALSE
  2:          0            -1                 -1 FALSE
  3:          0             1                  1 FALSE
  4:          0            -1                 -1 FALSE
  5:          0             2                  2 FALSE
 ---                                                  
347:          0             2                  2 FALSE
348:          0            -1                 -1 FALSE
349:          0            -1                 -1 FALSE
350:          0             1                  1 FALSE
351:          0            -1                 -1 FALSE
\end{Soutput}
\end{Schunk}
There are 188 out of 398 members in \textsf{traditional} arm who increased the cow ownership at least once (of which 88 have decreased at least once). Examining HH characteristics shows that all who increased the ownership received a loan while among who did not increase the cow ownership, 39.05\% did not receive a loan. \textsc{\normalsize Table \ref{tab permutation trad cow}} shows that it is one of the largest contrasting difference of \textsf{traditional} arm members in baseline asset holding, poverty grades, and household structure. While receiving a loan is a choice variable hence is endogenous to the outcomes, it implies that, even when the loan amount is small, members who are willing to take a loan is more likely to increase cow ownership than who are not. So the small amount lending may still have a role in poverty reduction through livestock accumulation.


Members who received a loan in \textsf{traditional}:
\begin{Schunk}
\begin{Soutput}
            notincreasedCow
increasedCow FALSE TRUE
       FALSE   100   28
       TRUE     73    0
\end{Soutput}
\begin{Soutput}

Yes  No 
 73  69 
\end{Soutput}
\begin{Soutput}

Yes  No 
 28  64 
\end{Soutput}
\end{Schunk}
Members who did not receive a loan in \textsf{traditional}:
\begin{Schunk}
\begin{Soutput}
            notincreasedCow
increasedCow FALSE TRUE
       FALSE    64   64
       TRUE     69    0
\end{Soutput}
\begin{Soutput}
      hhid              survey           creditstatus         Mstatus  
 Min.   : 7031502   Min.   :1   Yes            :  0   gErosion    :40  
 1st Qu.: 7054405   1st Qu.:1   No             :197   gRejection  :80  
 Median : 7086107   Median :1   Replaced Member:  0   iRejection  :51  
 Mean   :14956776   Mean   :1                         iReplacement: 0  
 3rd Qu.: 8148317   3rd Qu.:1                         newGroup    : 0  
 Max.   :81710220   Max.   :1                         oldMember   :26  
    NumCows     
 Min.   :0.000  
 1st Qu.:0.000  
 Median :0.000  
 Mean   :0.168  
 3rd Qu.:0.000  
 Max.   :3.000  
\end{Soutput}
\end{Schunk}


\begin{Schunk}
\begin{Soutput}
Warning in mean.default(dd[x, ], ...): argument is not numeric or logical: returning NA
\end{Soutput}
\begin{Soutput}
Warning in mean.default(dd[x, ], ...): argument is not numeric or logical: returning NA
\end{Soutput}
\begin{Soutput}
Warning in mean.default(dd[x, ], ...): argument is not numeric or logical: returning NA
\end{Soutput}
\begin{Soutput}
Warning in mean.default(dd[x, ], ...): argument is not numeric or logical: returning NA
\end{Soutput}
\begin{Soutput}
Warning in mean.default(dd[x, ], ...): argument is not numeric or logical: returning NA
\end{Soutput}
\begin{Soutput}
Warning in mean.default(dd[x, ], ...): argument is not numeric or logical: returning NA
\end{Soutput}
\begin{Soutput}
Warning in mean.default(dd[x, ], ...): argument is not numeric or logical: returning NA
\end{Soutput}
\begin{Soutput}
Warning in mean.default(dd[x, ], ...): argument is not numeric or logical: returning NA
\end{Soutput}
\end{Schunk}

\hspace{-1cm}\begin{minipage}[t]{14cm}
\hfil\textsc{\normalsize Table \refstepcounter{table}\thetable: Permutation tests of \textsf{traditional} arm members that increased vs. not-increased cow ownership\label{tab permutation trad cow}}\\
\setlength{\tabcolsep}{1pt}
\setlength{\baselineskip}{8pt}
\renewcommand{\arraystretch}{.55}
\hfil\begin{tikzpicture}
\node (tbl) {\input{c:/data/GUK/analysis/save/membership1or4/TradIncreased_vs_NotIncreasedPermutationTestResults.tex}};
%\input{c:/dropbox/data/ramadan/save/tablecolortemplate.tex}
\end{tikzpicture}\\
\renewcommand{\arraystretch}{.8}
\setlength{\tabcolsep}{1pt}
\begin{tabular}{>{\hfill\scriptsize}p{1cm}<{}>{\hfill\scriptsize}p{.25cm}<{}>{\scriptsize}p{12cm}<{\hfill}}
Source:& \multicolumn{2}{l}{\scriptsize Estimated with GUK administrative and survey data.}\\
Notes: & 1. & Permutation tests of each variables between members who increased cow ownership and who did not increase cow ownership in the \textsf{traditional} arm. Number of repetition is set at 10000. \textsf{R} package \textsf{coin} is used, and \textsf{global} option is used in the \textsf{pvalue} function. Columns under \textsf{mean} show means of both groups for each variates. Columns under \textsf{N} show number of observations of both groups for each variates. Columns under \textsf{p-values} show $p$ values of the null hypothesis of equal means. \\
& 2. & Baseline information is used for \textsf{HeadyLiteracy, HeadAge, HHsize, FloodInd1} with numbers of observation 133, 70, baseline information for \textsf{HAssetAmount, PAssetAmount, AssetAmount} with number of observations 88, 55, final round information is used for \textsf{dummyHadCows, NumCowsOwnedAtRd1, ReceivedCredit} with number of observations 88, 55 for increased and no change in cow ownership groups, respectively. The number of observations differ due to missingness of information and attrition.
\end{tabular}
\end{minipage}

\hspace{-1cm}\begin{minipage}[t]{14cm}
\hfil\textsc{\normalsize Table \refstepcounter{table}\thetable: Permutation tests of \textsf{traditional} arm members who received credits and only increased vs. increased-and-decreased cow ownership\label{tab permutation trad cow 2}}\\
\setlength{\tabcolsep}{1pt}
\setlength{\baselineskip}{8pt}
\renewcommand{\arraystretch}{.55}
\hfil\begin{tikzpicture}
\node (tbl) {\input{c:/data/GUK/analysis/save/membership1or4/TradIncreased_vs_IncreasedDecreasedPermutationTestResults.tex}};
%\input{c:/dropbox/data/ramadan/save/tablecolortemplate.tex}
\end{tikzpicture}\\
\renewcommand{\arraystretch}{.8}
\setlength{\tabcolsep}{1pt}
\begin{tabular}{>{\hfill\scriptsize}p{1cm}<{}>{\hfill\scriptsize}p{.25cm}<{}>{\scriptsize}p{12cm}<{\hfill}}
Source:& \multicolumn{2}{l}{\scriptsize Estimated with GUK administrative and survey data.}\\
Notes: & 1. & Permutation tests of each variables between loan receiving members who increased but never decreased cow ownership and who increased and decreased cow ownership in the \textsf{traditional} arm. Number of repetition is set at 10000. \textsf{R} package \textsf{coin} is used, and \textsf{global} option is used in the \textsf{pvalue} function. Columns under \textsf{mean} show means of both groups for each variates. Columns under \textsf{N} show number of observations of both groups for each variates. Columns under \textsf{p-values} show $p$ values of the null hypothesis of equal means. \\
& 2. & Baseline information is used for \textsf{HeadyLiteracy, HeadAge, HHsize, FloodInd1} with numbers of observation 2885, 366, baseline information for \textsf{HAssetAmount, PAssetAmount, AssetAmount} with number of observations 1038, 125, final round information is used for \textsf{dummyHadCows, NumCowsOwnedAtRd1, ReceivedCredit} with number of observations 1038, 125 for increased and no change in cow ownership groups, respectively. The number of observations differ due to missingness of information and attrition.
\end{tabular}
\end{minipage}

\hspace{-1cm}\begin{minipage}[t]{14cm}
\hfil\textsc{\normalsize Table \refstepcounter{table}\thetable: Permutation tests of all members who received credits and only increased vs. increased-and-decreased cow ownership\label{tab permutation trad cow 3}}\\
\setlength{\tabcolsep}{1pt}
\setlength{\baselineskip}{8pt}
\renewcommand{\arraystretch}{.55}
\hfil\begin{tikzpicture}
\node (tbl) {\input{c:/data/GUK/analysis/save/membership1or4/Increased_vs_IncreasedDecreasedPermutationTestResults.tex}};
%\input{c:/dropbox/data/ramadan/save/tablecolortemplate.tex}
\end{tikzpicture}\\
\renewcommand{\arraystretch}{.8}
\setlength{\tabcolsep}{1pt}
\begin{tabular}{>{\hfill\scriptsize}p{1cm}<{}>{\hfill\scriptsize}p{.25cm}<{}>{\scriptsize}p{12cm}<{\hfill}}
Source:& \multicolumn{2}{l}{\scriptsize Estimated with GUK administrative and survey data.}\\
Notes: & 1. & Permutation tests of each variables between loan receiving members who increased but never decreased cow ownership and who increased and decreased cow ownership in all arms. Number of repetition is set at 10000. \textsf{R} package \textsf{coin} is used, and \textsf{global} option is used in the \textsf{pvalue} function. Columns under \textsf{mean} show means of both groups for each variates. Columns under \textsf{N} show number of observations of both groups for each variates. Columns under \textsf{p-values} show $p$ values of the null hypothesis of equal means. \\
& 2. & Baseline information is used for \textsf{HeadyLiteracy, HeadAge, HHsize, FloodInd1} with numbers of observation 2885, 366, baseline information for \textsf{HAssetAmount, PAssetAmount, AssetAmount} with number of observations 1038, 125, final round information is used for \textsf{dummyHadCows, NumCowsOwnedAtRd1, ReceivedCredit} with number of observations 1038, 125 for increased and no change in cow ownership groups, respectively. The number of observations differ due to missingness of information and attrition.
\end{tabular}
\end{minipage}

\hspace{-1cm}\begin{minipage}[t]{14cm}
\hfil\textsc{\normalsize Table \refstepcounter{table}\thetable: Permutation tests of all members who received credits vs. not received credits\label{tab permutation cow 4}}\\
\setlength{\tabcolsep}{1pt}
\setlength{\baselineskip}{8pt}
\renewcommand{\arraystretch}{.55}
\hfil\begin{tikzpicture}
\node (tbl) {\input{c:/data/GUK/analysis/save/membership1or4/Received_vs_NotReceivedPermutationTestResults.tex}};
%\input{c:/dropbox/data/ramadan/save/tablecolortemplate.tex}
\end{tikzpicture}\\
\renewcommand{\arraystretch}{.8}
\setlength{\tabcolsep}{1pt}
\begin{tabular}{>{\hfill\scriptsize}p{1cm}<{}>{\hfill\scriptsize}p{.25cm}<{}>{\scriptsize}p{12cm}<{\hfill}}
Source:& \multicolumn{2}{l}{\scriptsize Estimated with GUK administrative and survey data.}\\
Notes: & 1. & Permutation tests of each variables between loan receiving members and non receiving members in all arms. Number of repetition is set at 10000. \textsf{R} package \textsf{coin} is used, and \textsf{global} option is used in the \textsf{pvalue} function. Columns under \textsf{mean} show means of both groups for each variates. Columns under \textsf{N} show number of observations of both groups for each variates. Columns under \textsf{p-values} show $p$ values of the null hypothesis of equal means. \\
& 2. & Baseline information is used for \textsf{HeadyLiteracy, HeadAge, HHsize, FloodInd1} with numbers of observation 2885, 366, baseline information for \textsf{HAssetAmount, PAssetAmount, AssetAmount} with number of observations 1038, 125, final round information is used for \textsf{dummyHadCows, NumCowsOwnedAtRd1, ReceivedCredit} with number of observations 1038, 125 for increased and no change in cow ownership groups, respectively. The number of observations differ due to missingness of information and attrition.
\end{tabular}
\end{minipage}


Do these household who increased cow ownership report cows as IGA? Among who increased cow ownership, 58 report cow or ox as IGA at least once and 60 (51\%) do not. \gobblepars

Below tabulation shows reported IGA as cows/oxen against cow/ox ownership.
\begin{Schunk}
\begin{Soutput}
        Cow
CowAsIGA FALSE TRUE <NA>
   FALSE   147  167    2
   TRUE     83  965   14
   <NA>     77  138    5
\end{Soutput}
\end{Schunk}
We need to modify IGA summary by using livestock ownership data.



\subsection{Assets+Livestock}




\begin{Schunk}
\begin{Soutput}
                 creditstatus
BorrowerStatus     Yes   No
  borrower        1192  157
  pure saver         0   26
  quit membership    0  220
\end{Soutput}
\begin{Soutput}
              creditstatus
Mstatus         Yes   No
  gErosion        0   80
  gRejection      0  140
  iRejection      1  157
  iReplacement    0    0
  newGroup        0    0
  oldMember    1191   26
\end{Soutput}
\begin{Soutput}
Dropped 196 obs due to T<2.
Dropped 1402 obs due to NA.
Dropped 196 obs due to T<2.
Dropped 1402 obs due to NA.
Dropped 196 obs due to T<2.
Dropped 1402 obs due to NA.
Dropped 196 obs due to T<2.
Dropped 1402 obs due to NA.
Dropped 130 obs due to T<2.
Dropped 1274 obs due to NA.
Dropped 130 obs due to T<2.
Dropped 1274 obs due to NA.
Dropped 130 obs due to T<2.
Dropped 1274 obs due to NA.
Dropped 130 obs due to T<2.
Dropped 1274 obs due to NA.
\end{Soutput}
\end{Schunk}

\begin{Schunk}
\begin{Soutput}
   Arm
tee traditional large large grace cow
  1         398   399         379 398
  2         283   390         373 379
  3         276   384         348 365
  4         238   377         330 328
\end{Soutput}
\begin{Soutput}
   Arm
tee traditional large large grace cow
  1          66    78          81  63
  2         151   254         258 283
  3         189   348         323 324
  4         156   328         291 287
\end{Soutput}
\begin{Soutput}
Dropped 196 obs due to T<2.
Dropped 1402 obs due to NA.
Dropped 196 obs due to T<2.
Dropped 1402 obs due to NA.
Dropped 154 obs due to T<2.
Dropped 1272 obs due to NA.
Dropped 154 obs due to T<2.
Dropped 1272 obs due to NA.
\end{Soutput}
\end{Schunk}

\begin{Schunk}
\begin{Soutput}
      hhid tee HAssetAmount PAssetAmount
1: 7043715   1         1400          280
2: 7043715   2            0            0
3: 7043715   3         1200          400
4: 7043715   4         5700          400
\end{Soutput}
\begin{Soutput}
logical(0)
\end{Soutput}
\begin{Soutput}
      hhid FloodInRd1 groupid   Arm HeadLiteracy Time.2 Time.3 Time.4
1: 7043715          1   70437 large            0      0      0      0
2: 7043715          1   70437 large            0      0      1      0
3: 7043715          1   70437 large            0      0      0      1
   HAssetAmount PAssetAmount tee teeyr dummyTraditional dummyLarge
1:         1400          280   1     1                0          1
2:         1200          400   3     3                0          1
3:         5700          400   4     4                0          1
   dummyLargeGrace dummyCow dummyUltraPoor dummyModeratelyPoor dummySmallSize
1:               0        0              0                   1              0
2:               0        0              0                   1              0
3:               0        0              0                   1              0
   dummyLargeSize dummyWithGrace dummyWithoutGrace dummyTraditional.Time2
1:              1              0                 1              0.0512659
2:              1              0                 1              0.0512659
3:              1              0                 1              0.0512659
   dummyLarge.Time2 dummyLargeGrace.Time2 dummyCow.Time2 dummyUltraPoor.Time2
1:        -0.175805             0.0613903      0.0631492             0.149937
2:        -0.175805             0.0613903      0.0631492             0.149937
3:        -0.175805             0.0613903      0.0631492             0.149937
   dummyModeratelyPoor.Time2 dummySmallSize.Time2 dummyLargeSize.Time2
1:                 -0.178894            0.0512659           -0.0512659
2:                 -0.178894            0.0512659           -0.0512659
3:                 -0.178894            0.0512659           -0.0512659
   dummyWithoutGrace.Time2 dummyWithGrace.Time2 dummyTraditional.Time3
1:                -0.12454              0.12454              0.0515654
2:                -0.12454              0.12454             -0.1599764
3:                -0.12454              0.12454              0.0515654
   dummyLarge.Time3 dummyLargeGrace.Time3 dummyCow.Time3 dummyUltraPoor.Time3
1:        -0.176833             0.0617491      0.0635183             0.150813
2:         0.548605            -0.1915701     -0.1970588            -0.467881
3:        -0.176833             0.0617491      0.0635183             0.150813
   dummyModeratelyPoor.Time3 dummySmallSize.Time3 dummyLargeSize.Time3
1:                 -0.179940            0.0515654           -0.0515654
2:                  0.558244           -0.1599764            0.1599764
3:                 -0.179940            0.0515654           -0.0515654
   dummyWithoutGrace.Time3 dummyWithGrace.Time3 dummyTraditional.Time4
1:               -0.125267             0.125267              0.0489441
2:                0.388629            -0.388629              0.0489441
3:               -0.125267             0.125267             -0.1625978
   dummyLarge.Time4 dummyLargeGrace.Time4 dummyCow.Time4 dummyUltraPoor.Time4
1:        -0.167843              0.058610      0.0602893             0.143146
2:        -0.167843              0.058610      0.0602893             0.143146
3:         0.557595             -0.194709     -0.2002878            -0.475547
   dummyModeratelyPoor.Time4 dummySmallSize.Time4 dummyLargeSize.Time4
1:                 -0.170792            0.0489441           -0.0489441
2:                 -0.170792            0.0489441           -0.0489441
3:                  0.567391           -0.1625978            0.1625978
   dummyWithoutGrace.Time4 dummyWithGrace.Time4
1:               -0.118899             0.118899
2:               -0.118899             0.118899
3:                0.394997            -0.394997
\end{Soutput}
\begin{Soutput}
Dropped 177 obs due to T<2.
Dropped 1399 obs due to NA.
Dropped 130 obs due to T<2.
Dropped 1271 obs due to NA.
Dropped 177 obs due to T<2.
Dropped 1399 obs due to NA.
Dropped 130 obs due to T<2.
Dropped 1271 obs due to NA.
\end{Soutput}
\end{Schunk}

\begin{Schunk}
\begin{Sinput}
source(paste0(pathprogram, "AssetLivestockCovariateSelection.R"))
\end{Sinput}
\end{Schunk}

\begin{Schunk}
\begin{figure}

{\centering \includegraphics[width=\maxwidth]{figure/ImpactEstimationMembership1or4/Total_values-1} 

}

\caption{Total asset values\\ {\footnotesize Sum of assets and livestock holding values.\setlength{\baselineskip}{8pt}}}\label{Figure Total values}
\end{figure}
\end{Schunk}
\begin{Schunk}
\begin{figure}

{\centering \includegraphics[width=\maxwidth]{figure/ImpactEstimationMembership1or4/Total_values_by_Hadcows-1} 

}

\caption{Total asset values by cow holding at baseline\\ {\footnotesize Sum of assets and livestock holding values.\setlength{\baselineskip}{8pt}}}\label{Figure Total values by Hadcows}
\end{figure}
\end{Schunk}


\clearpage
\hspace{-1cm}\begin{minipage}[t]{14cm}
\hfil\textsc{\normalsize Table \refstepcounter{table}\thetable: FD estimation of total assets\label{tab FD total assets}}\\
\setlength{\tabcolsep}{1pt}
\setlength{\baselineskip}{8pt}
\renewcommand{\arraystretch}{.55}
\hfil\begin{tikzpicture}
\node (tbl) {\input{c:/data/GUK/analysis/save/membership1or4/AssetLivestockFDEstimationResults.tex}};
%\input{c:/dropbox/data/ramadan/save/tablecolortemplate.tex}
\end{tikzpicture}\\
\renewcommand{\arraystretch}{.8}
\setlength{\tabcolsep}{1pt}
\begin{tabular}{>{\hfill\scriptsize}p{1cm}<{}>{\hfill\scriptsize}p{.25cm}<{}>{\scriptsize}p{12cm}<{\hfill}}
Source:& \multicolumn{2}{l}{\scriptsize Estimated with GUK administrative and survey data.}\\
Notes: & 1. & First-difference estimates. A first-difference is defined as $\Delta x_{t+k}\equiv x_{t+k} - x_{t}$  for $k=1, 2, \dots$. Saving and repayment misses are taken from administrative data and merged with survey data at Year-Month of survey interviews. Intercept terms are omitted in estimating equations. Sample is continuing members and replacing members of early rejecters and received loans prior to 2015 Janunary. Household assets do not include livestock. Regressions (1)-(3), (5)-(6) use only arm and calendar information. (4) and (7) use previous six month repayment and saving information which is lacking in rd 1, hence starts from rd 2.\\
& 2. & ${}^{***}$, ${}^{**}$, ${}^{*}$ indicate statistical significance at 1\%, 5\%, 10\%, respetively. Standard errors are clustered at group (village) level.
\end{tabular}
\end{minipage}

\hspace{-1cm}\begin{minipage}[t]{14cm}
\hfil\textsc{\normalsize Table \refstepcounter{table}\thetable: FD estimation of total assets, moderately poor vs. ultra poor\label{tab FD total assets2}}\\
\setlength{\tabcolsep}{1pt}
\setlength{\baselineskip}{8pt}
\renewcommand{\arraystretch}{.55}
\hfil\begin{tikzpicture}
\node (tbl) {\input{c:/data/GUK/analysis/save/membership1or4/AssetLivestockPovertyStatusFDEstimationResults.tex}};
%\input{c:/dropbox/data/ramadan/save/tablecolortemplate.tex}
\end{tikzpicture}\\
\renewcommand{\arraystretch}{.8}
\setlength{\tabcolsep}{1pt}
\begin{tabular}{>{\hfill\scriptsize}p{1cm}<{}>{\hfill\scriptsize}p{.25cm}<{}>{\scriptsize}p{12cm}<{\hfill}}
Source:& \multicolumn{2}{l}{\scriptsize Estimated with GUK administrative and survey data.}\\
Notes: & 1. & First-difference estimates. A first-difference is defined as $\Delta x_{t+k}\equiv x_{t+k} - x_{t}$  for $k=1, 2, \dots$. Saving and repayment misses are taken from administrative data and merged with survey data at Year-Month of survey interviews. Intercept terms are omitted in estimating equations. Sample is continuing members and replacing members of early rejecters and received loans prior to 2015 Janunary. Household assets do not include livestock. \\
& 2. & ${}^{***}$, ${}^{**}$, ${}^{*}$ indicate statistical significance at 1\%, 5\%, 10\%, respetively. Standard errors are clustered at group (village) level.
\end{tabular}
\end{minipage}

\hspace{-1cm}\begin{minipage}[t]{14cm}
\hfil\textsc{\normalsize Table \refstepcounter{table}\thetable: FD estimation of total assets, small vs. large size loans\label{tab FD total assets3}}\\
\setlength{\tabcolsep}{1pt}
\setlength{\baselineskip}{8pt}
\renewcommand{\arraystretch}{.55}
\hfil\begin{tikzpicture}
\node (tbl) {\input{c:/data/GUK/analysis/save/membership1or4/AssetLivestockSizeFDEstimationResults.tex}};
%\input{c:/dropbox/data/ramadan/save/tablecolortemplate.tex}
\end{tikzpicture}\\
\renewcommand{\arraystretch}{.8}
\setlength{\tabcolsep}{1pt}
\begin{tabular}{>{\hfill\scriptsize}p{1cm}<{}>{\hfill\scriptsize}p{.25cm}<{}>{\scriptsize}p{12cm}<{\hfill}}
Source:& \multicolumn{2}{l}{\scriptsize Estimated with GUK administrative and survey data.}\\
Notes: & 1. & First-difference estimates. A first-difference is defined as $\Delta x_{t+k}\equiv x_{t+k} - x_{t}$  for $k=1, 2, \dots$. Saving and repayment misses are taken from administrative data and merged with survey data at Year-Month of survey interviews. Intercept terms are omitted in estimating equations. Sample is continuing members and replacing members of early rejecters and received loans prior to 2015 Janunary. Household assets do not include livestock. \\
& 2. & ${}^{***}$, ${}^{**}$, ${}^{*}$ indicate statistical significance at 1\%, 5\%, 10\%, respetively. Standard errors are clustered at group (village) level.
\end{tabular}
\end{minipage}

\hspace{-1cm}\begin{minipage}[t]{14cm}
\hfil\textsc{\normalsize Table \refstepcounter{table}\thetable: FD estimation of total assets, with vs. without a grace period\label{tab FD total assets4}}\\
\setlength{\tabcolsep}{1pt}
\setlength{\baselineskip}{8pt}
\renewcommand{\arraystretch}{.55}
\hfil\begin{tikzpicture}
\node (tbl) {\input{c:/data/GUK/analysis/save/membership1or4/AssetLivestockGraceFDEstimationResults.tex}};
%\input{c:/dropbox/data/ramadan/save/tablecolortemplate.tex}
\end{tikzpicture}\\
\renewcommand{\arraystretch}{.8}
\setlength{\tabcolsep}{1pt}
\begin{tabular}{>{\hfill\scriptsize}p{1cm}<{}>{\hfill\scriptsize}p{.25cm}<{}>{\scriptsize}p{12cm}<{\hfill}}
Source:& \multicolumn{2}{l}{\scriptsize Estimated with GUK administrative and survey data.}\\
Notes: & 1. & First-difference estimates. A first-difference is defined as $\Delta x_{t+k}\equiv x_{t+k} - x_{t}$  for $k=1, 2, \dots$. Saving and repayment misses are taken from administrative data and merged with survey data at Year-Month of survey interviews. Intercept terms are omitted in estimating equations. Sample is continuing members and replacing members of early rejecters and received loans prior to 2015 Janunary. Household assets do not include livestock. \\
& 2. & ${}^{***}$, ${}^{**}$, ${}^{*}$ indicate statistical significance at 1\%, 5\%, 10\%, respetively. Standard errors are clustered at group (village) level.
\end{tabular}
\end{minipage}

\hspace{-1cm}\begin{minipage}[t]{14cm}
\hfil\textsc{\normalsize Table \refstepcounter{table}\thetable: FD estimation of total assets, round 2 and 4 comparison\label{tab FD total assets rd24}}\\
\setlength{\tabcolsep}{1pt}
\setlength{\baselineskip}{8pt}
\renewcommand{\arraystretch}{.55}
\hfil\begin{tikzpicture}
\node (tbl) {\input{c:/data/GUK/analysis/save/membership1or4/AssetLivestockRd24DiffFDEstimationResults.tex}};
%\input{c:/dropbox/data/ramadan/save/tablecolortemplate.tex}
\end{tikzpicture}\\
\renewcommand{\arraystretch}{.8}
\setlength{\tabcolsep}{1pt}
\begin{tabular}{>{\hfill\scriptsize}p{1cm}<{}>{\hfill\scriptsize}p{.25cm}<{}>{\scriptsize}p{12cm}<{\hfill}}
Source:& \multicolumn{2}{l}{\scriptsize Estimated with GUK administrative and survey data.}\\
Notes: & 1. & First-difference estimates between round 2 and 4. A first-difference is defined as $\Delta x_{t+k}\equiv x_{t+k} - x_{t}$ for $k=1, 2, \dots$. Saving and repayment misses are taken from administrative data and merged with survey data at Year-Month of survey interviews. Intercept terms are omitted in estimating equations. Sample is continuing members and replacing members of early rejecters and received loans prior to 2015 Janunary. Household assets do not include livestock. Regressions (1)-(3), (5)-(6) use only arm and calendar information. (4) and (7) use previous six month repayment and saving information which is lacking in rd 1, hence starts from rd 2.\\
& 2. & ${}^{***}$, ${}^{**}$, ${}^{*}$ indicate statistical significance at 1\%, 5\%, 10\%, respetively. Standard errors are clustered at group (village) level.
\end{tabular}
\end{minipage}

\hspace{-1cm}\begin{minipage}[t]{14cm}
\hfil\textsc{\normalsize Table \refstepcounter{table}\thetable: FD estimation of total assets, round 2 and 4 comparison, grace period\label{tab FD total assets rd24 grace}}\\
\setlength{\tabcolsep}{1pt}
\setlength{\baselineskip}{8pt}
\renewcommand{\arraystretch}{.55}
\hfil\begin{tikzpicture}
\node (tbl) {\input{c:/data/GUK/analysis/save/membership1or4/AssetLivestockRd24DiffGraceFDEstimationResults.tex}};
%\input{c:/dropbox/data/ramadan/save/tablecolortemplate.tex}
\end{tikzpicture}\\
\renewcommand{\arraystretch}{.8}
\setlength{\tabcolsep}{1pt}
\begin{tabular}{>{\hfill\scriptsize}p{1cm}<{}>{\hfill\scriptsize}p{.25cm}<{}>{\scriptsize}p{12cm}<{\hfill}}
Source:& \multicolumn{2}{l}{\scriptsize Estimated with GUK administrative and survey data.}\\
Notes: & 1. & First-difference estimates between round 2 and 4. A first-difference is defined as $\Delta x_{t+k}\equiv x_{t+k} - x_{t}$ for $k=1, 2, \dots$. Saving and repayment misses are taken from administrative data and merged with survey data at Year-Month of survey interviews. Intercept terms are omitted in estimating equations. Sample is continuing members and replacing members of early rejecters and received loans prior to 2015 Janunary. Household assets do not include livestock. Regressions (1)-(3), (5)-(6) use only arm and calendar information. (4) and (7) use previous six month repayment and saving information which is lacking in rd 1, hence starts from rd 2.\\
& 2. & ${}^{***}$, ${}^{**}$, ${}^{*}$ indicate statistical significance at 1\%, 5\%, 10\%, respetively. Standard errors are clustered at group (village) level.
\end{tabular}
\end{minipage}

\hspace{-1cm}\begin{minipage}[t]{14cm}
\hfil\textsc{\normalsize Table \refstepcounter{table}\thetable: FD estimation of total assets, round 2 and 4 comparison, ultra poor vs. moderately poor\label{tab FD total assets rd24 poor}}\\
\setlength{\tabcolsep}{1pt}
\setlength{\baselineskip}{8pt}
\renewcommand{\arraystretch}{.55}
\hfil\begin{tikzpicture}
\node (tbl) {\input{c:/data/GUK/analysis/save/membership1or4/AssetLivestockRd24DiffPovertyStatusFDEstimationResults.tex}};
%\input{c:/dropbox/data/ramadan/save/tablecolortemplate.tex}
\end{tikzpicture}\\
\renewcommand{\arraystretch}{.8}
\setlength{\tabcolsep}{1pt}
\begin{tabular}{>{\hfill\scriptsize}p{1cm}<{}>{\hfill\scriptsize}p{.25cm}<{}>{\scriptsize}p{12cm}<{\hfill}}
Source:& \multicolumn{2}{l}{\scriptsize Estimated with GUK administrative and survey data.}\\
Notes: & 1. & First-difference estimates between round 2 and 4. A first-difference is defined as $\Delta x_{t+k}\equiv x_{t+k} - x_{t}$ for $k=1, 2, \dots$. Saving and repayment misses are taken from administrative data and merged with survey data at Year-Month of survey interviews. Intercept terms are omitted in estimating equations. Sample is continuing members and replacing members of early rejecters and received loans prior to 2015 Janunary. Household assets do not include livestock. Regressions (1)-(3), (5)-(6) use only arm and calendar information. (4) and (7) use previous six month repayment and saving information which is lacking in rd 1, hence starts from rd 2.\\
& 2. & ${}^{***}$, ${}^{**}$, ${}^{*}$ indicate statistical significance at 1\%, 5\%, 10\%, respetively. Standard errors are clustered at group (village) level.
\end{tabular}
\end{minipage}


\begin{palepinkleftbar}
\begin{finding}
\ref{Figure Total values by Hadcows} seems to show that more experienced (or wealthier) members under \textsf{large grace} and \textsf{cow} arms did not increase the asset holding as much as their counterpart who are less experienced. More experienced members under \textsf{traditional} arm show higher increases in assets relative to their less experienced counterpart.
\end{finding}
\end{palepinkleftbar}

\subsection{Incomes}



\begin{Schunk}
\begin{Soutput}
Warning in `[.data.table`(lab, , `:=`(grepout("RM", colnames(lab)), NULL)): length(LHS)==0; no columns to delete or assign RHS to.
\end{Soutput}
\begin{Soutput}
Warning in `[.data.table`(far, , `:=`(grepout("RM", colnames(far)), NULL)): length(LHS)==0; no columns to delete or assign RHS to.
\end{Soutput}
\begin{Soutput}
Dropped 436 obs due to T<2.
Dropped 1463 obs due to NA.
Dropped 436 obs due to T<2.
Dropped 1463 obs due to NA.
Dropped 116 obs due to T<2.
Dropped 111 obs due to NA.
Dropped 116 obs due to T<2.
Dropped 111 obs due to NA.
\end{Soutput}
\end{Schunk}

Income sources are mainly labour incomes (\textsf{lab}) and farm revenues (\textsf{far}) with 5649 and 393 observations, respectively. After first-differencing, they become 3750 and 158 observations, with 3750 households observed for 3751 times. 

Obs for survey labour income.
\begin{Schunk}
\begin{Soutput}

   1    2    3    4 
   1 1170 1303 1276 
\end{Soutput}
\end{Schunk}
Obs for survey labour income and admin repayment data.
\begin{Schunk}
\begin{Soutput}

   3    4 
1303 1276 
\end{Soutput}
\end{Schunk}
\begin{Schunk}
\begin{Soutput}

 3  4 
85 73 
\end{Soutput}
\end{Schunk}
Obs for survey farm revenue.
\begin{Schunk}
\begin{Soutput}

 3  4 
85 73 
\end{Soutput}
\end{Schunk}
Obs for survey farm revenue and admin repayment data.
\begin{Schunk}
\begin{Soutput}

 3  4 
85 73 
\end{Soutput}
\end{Schunk}

\begin{Schunk}
\begin{Sinput}
source(paste0(pathprogram, "IncomeCovariateSelection.R"))
\end{Sinput}
\end{Schunk}

\begin{Schunk}
\begin{Sinput}
source(paste0(pathprogram, "IncomeCovariateSelectionRobustness.R"))
\end{Sinput}
\end{Schunk}



\hspace{-1cm}\begin{minipage}[t]{14cm}
\hfil\textsc{\normalsize Table \refstepcounter{table}\thetable: FD estimation of incomes\label{tab FD incomes}}\\
\setlength{\tabcolsep}{1pt}
\setlength{\baselineskip}{8pt}
\renewcommand{\arraystretch}{.55}
\hfil\begin{tikzpicture}
\node (tbl) {\input{c:/data/GUK/analysis/save/membership1or4/IncomesFDEstimationResults.tex}};
%\input{c:/dropbox/data/ramadan/save/tablecolortemplate.tex}
\end{tikzpicture}\\
\renewcommand{\arraystretch}{.8}
\setlength{\tabcolsep}{1pt}
\begin{tabular}{>{\hfill\scriptsize}p{1cm}<{}>{\hfill\scriptsize}p{.25cm}<{}>{\scriptsize}p{12cm}<{\hfill}}
Source:& \multicolumn{2}{l}{\scriptsize Estimated with GUK administrative and survey data.}\\
Notes: & 1. & First-difference estimates. A first-difference is defined as $\Delta x_{t+1}\equiv x_{t+1} - x_{t}$. Saving and repayment misses are taken from administrative data and merged with survey data at Year-Month of survey interviews. Intercept terms are omitted in estimating equations. Sample is continuing members and replacing members of early rejecters and received loans prior to 2015 Janunary. Labour income is in 1000 Tk unit andis sum of all earned labour incomes. Farm revenue is total of agricultural produce sales. \\
& 2. & ${}^{***}$, ${}^{**}$, ${}^{*}$ indicate statistical significance at 1\%, 5\%, 10\%, respetively. Standard errors are clustered at group (village) level.
\end{tabular}
\end{minipage}

\hspace{-1cm}\begin{minipage}[t]{14cm}
\hfil\textsc{\normalsize Table \refstepcounter{table}\thetable: FD estimation of incomes, moderately poor vs. ultra poor\label{tab FD incomes2}}\\
\setlength{\tabcolsep}{1pt}
\setlength{\baselineskip}{8pt}
\renewcommand{\arraystretch}{.55}
\hfil\begin{tikzpicture}
\node (tbl) {\input{c:/data/GUK/analysis/save/membership1or4/IncomesPovertyStatusFDEstimationResults.tex}};
%\input{c:/dropbox/data/ramadan/save/tablecolortemplate.tex}
\end{tikzpicture}\\
\renewcommand{\arraystretch}{.8}
\setlength{\tabcolsep}{1pt}
\begin{tabular}{>{\hfill\scriptsize}p{1cm}<{}>{\hfill\scriptsize}p{.25cm}<{}>{\scriptsize}p{12cm}<{\hfill}}
Source:& \multicolumn{2}{l}{\scriptsize Estimated with GUK administrative and survey data.}\\
Notes: & 1. & First-difference estimates. A first-difference is defined as $\Delta x_{t+1}\equiv x_{t+1} - x_{t}$. Saving and repayment misses are taken from administrative data and merged with survey data at Year-Month of survey interviews. Intercept terms are omitted in estimating equations. Sample is continuing members and replacing members of early rejecters and received loans prior to 2015 Janunary. Labour income is in 1000 Tk unit andis sum of all earned labour incomes. Farm revenue is total of agricultural produce sales. \\
& 2. & ${}^{***}$, ${}^{**}$, ${}^{*}$ indicate statistical significance at 1\%, 5\%, 10\%, respetively. Standard errors are clustered at group (village) level.
\end{tabular}
\end{minipage}

\hspace{-1cm}\begin{minipage}[t]{14cm}
\hfil\textsc{\normalsize Table \refstepcounter{table}\thetable: FD estimation of incomes, loan size\label{tab FD incomes3}}\\
\setlength{\tabcolsep}{1pt}
\setlength{\baselineskip}{8pt}
\renewcommand{\arraystretch}{.55}
\hfil\begin{tikzpicture}
\node (tbl) {\input{c:/data/GUK/analysis/save/membership1or4/IncomesSizeFDEstimationResults.tex}};
%\input{c:/dropbox/data/ramadan/save/tablecolortemplate.tex}
\end{tikzpicture}\\
\renewcommand{\arraystretch}{.8}
\setlength{\tabcolsep}{1pt}
\begin{tabular}{>{\hfill\scriptsize}p{1cm}<{}>{\hfill\scriptsize}p{.25cm}<{}>{\scriptsize}p{12cm}<{\hfill}}
Source:& \multicolumn{2}{l}{\scriptsize Estimated with GUK administrative and survey data.}\\
Notes: & 1. & First-difference estimates. A first-difference is defined as $\Delta x_{t+1}\equiv x_{t+1} - x_{t}$. Saving and repayment misses are taken from administrative data and merged with survey data at Year-Month of survey interviews. Intercept terms are omitted in estimating equations. Sample is continuing members and replacing members of early rejecters and received loans prior to 2015 Janunary. Labour income is in 1000 Tk unit andis sum of all earned labour incomes. Farm revenue is total of agricultural produce sales. \\
& 2. & ${}^{***}$, ${}^{**}$, ${}^{*}$ indicate statistical significance at 1\%, 5\%, 10\%, respetively. Standard errors are clustered at group (village) level.
\end{tabular}
\end{minipage}

\hspace{-1cm}\begin{minipage}[t]{14cm}
\hfil\textsc{\normalsize Table \refstepcounter{table}\thetable: FD estimation of incomes, with vs. without a grace period\label{tab FD incomes4}}\\
\setlength{\tabcolsep}{1pt}
\setlength{\baselineskip}{8pt}
\renewcommand{\arraystretch}{.55}
\hfil\begin{tikzpicture}
\node (tbl) {\input{c:/data/GUK/analysis/save/membership1or4/IncomesGraceFDEstimationResults.tex}};
%\input{c:/dropbox/data/ramadan/save/tablecolortemplate.tex}
\end{tikzpicture}\\
\renewcommand{\arraystretch}{.8}
\setlength{\tabcolsep}{1pt}
\begin{tabular}{>{\hfill\scriptsize}p{1cm}<{}>{\hfill\scriptsize}p{.25cm}<{}>{\scriptsize}p{12cm}<{\hfill}}
Source:& \multicolumn{2}{l}{\scriptsize Estimated with GUK administrative and survey data.}\\
Notes: & 1. & First-difference estimates. A first-difference is defined as $\Delta x_{t+1}\equiv x_{t+1} - x_{t}$. Saving and repayment misses are taken from administrative data and merged with survey data at Year-Month of survey interviews. Intercept terms are omitted in estimating equations. Sample is continuing members and replacing members of early rejecters and received loans prior to 2015 Janunary. Labour income is in 1000 Tk unit andis sum of all earned labour incomes. Farm revenue is total of agricultural produce sales. \\
& 2. & ${}^{***}$, ${}^{**}$, ${}^{*}$ indicate statistical significance at 1\%, 5\%, 10\%, respetively. Standard errors are clustered at group (village) level.
\end{tabular}
\end{minipage}

\hspace{-1cm}\begin{minipage}[t]{14cm}
\hfil\textsc{\normalsize Table \refstepcounter{table}\thetable: FD estimation of incomes, small vs. large size loans\label{tab FD incomes5}}\\
\setlength{\tabcolsep}{1pt}
\setlength{\baselineskip}{8pt}
\renewcommand{\arraystretch}{.55}
\hfil\begin{tikzpicture}
\node (tbl) {\input{c:/data/GUK/analysis/save/membership1or4/IncomesSizeFDEstimationResults.tex}};
%\input{c:/dropbox/data/ramadan/save/tablecolortemplate.tex}
\end{tikzpicture}\\
\renewcommand{\arraystretch}{.8}
\setlength{\tabcolsep}{1pt}
\begin{tabular}{>{\hfill\scriptsize}p{1cm}<{}>{\hfill\scriptsize}p{.25cm}<{}>{\scriptsize}p{12cm}<{\hfill}}
Source:& \multicolumn{2}{l}{\scriptsize Estimated with GUK administrative and survey data.}\\
Notes: & 1. & First-difference estimates. A first-difference is defined as $\Delta x_{t+1}\equiv x_{t+1} - x_{t}$. Saving and repayment misses are taken from administrative data and merged with survey data at Year-Month of survey interviews. Intercept terms are omitted in estimating equations. Sample is continuing members and replacing members of early rejecters and received loans prior to 2015 Janunary. Labour income is in 1000 Tk unit andis sum of all earned labour incomes. Farm revenue is total of agricultural produce sales. \\
& 2. & ${}^{***}$, ${}^{**}$, ${}^{*}$ indicate statistical significance at 1\%, 5\%, 10\%, respetively. Standard errors are clustered at group (village) level.
\end{tabular}
\end{minipage}

Robustness: \textsc{\small Table \ref{tab FD incomes HH size}} shows that members from larger household size (defined as more than 2 adults) have a higher labour income increase in rd 2-3 and 3-4. This suggests existence of surplus labour in households and local employment opportunities. 


\hspace{-1cm}\begin{minipage}[t]{14cm}
\hfil\textsc{\normalsize Table \refstepcounter{table}\thetable: FD estimation of incomes, small vs. large HH size samples\label{tab FD incomes HH size}}\\
\setlength{\tabcolsep}{1pt}
\setlength{\baselineskip}{8pt}
\renewcommand{\arraystretch}{.55}
\hfil\begin{tikzpicture}
\node (tbl) {\input{c:/data/GUK/analysis/save/membership1or4/IncomesRobustnessFDEstimationResults.tex}};
%\input{c:/dropbox/data/ramadan/save/tablecolortemplate.tex}
\end{tikzpicture}\\
\renewcommand{\arraystretch}{.8}
\setlength{\tabcolsep}{1pt}
\begin{tabular}{>{\hfill\scriptsize}p{1cm}<{}>{\hfill\scriptsize}p{.25cm}<{}>{\scriptsize}p{12cm}<{\hfill}}
Source:& \multicolumn{2}{l}{\scriptsize Estimated with GUK administrative and survey data.}\\
Notes: & 1. & First-difference estimates. A first-difference is defined as $\Delta x_{t+1}\equiv x_{t+1} - x_{t}$. Saving and repayment misses are taken from administrative data and merged with survey data at Year-Month of survey interviews. Intercept terms are omitted in estimating equations. Sample is continuing members and replacing members of early rejecters and received loans prior to 2015 Janunary. Labour income is in 1000 Tk unit andis sum of all earned labour incomes. Farm revenue is total of agricultural produce sales. \\
& 2. & ${}^{***}$, ${}^{**}$, ${}^{*}$ indicate statistical significance at 1\%, 5\%, 10\%, respetively. Standard errors are clustered at group (village) level.
\end{tabular}
\end{minipage}


\begin{palepinkleftbar}
\begin{finding}
\textsc{\small Table \ref{tab FD incomes}} (1) and (3) show a general decrease in rd 1 - 2 period and a general increase in rd 2 - 4 periods for labour incomes. (2) and (4) suggest \textsf{Large grace} arm saw a greater swing (decrease and increases) which resulted in overall significant mean increase of -4.79 (at $p$ value of 11.07\%), yet not statistically different from \textsf{traditional}, while other arms have estimates closer to \textsf{traditional}. This labour income response can be due to the flood in rd 1 which reduced the labour incomes while repayment burden in later rounds prompted households to earn more labour incomes. Strong positive correlation with other members' previous 6 month repayment in (4) may be due to concerted peer efforts in repayment. Farm revenues do not show any systematic trend.
\end{finding}
\end{palepinkleftbar}


\subsection{Consumption}



%Number of HHs with consumption before the loan is disbursed (\textsf{ConsumptionBaseline} == 1) is small.
\begin{Schunk}
\begin{Soutput}
             ConsumptionBaseline
Arm             0   1
  traditional 797   0
  large       892 256
  large grace 814 218
  cow         829 245
\end{Soutput}
\begin{Soutput}
Warning in `[.data.table`(con, , `:=`(grepout("RM", colnames(con)), NULL)): length(LHS)==0; no columns to delete or assign RHS to.
\end{Soutput}
\begin{Soutput}
Dropped 28 obs due to T<2.
Dropped 1373 obs due to NA.
Dropped 28 obs due to T<2.
Dropped 1373 obs due to NA.
\end{Soutput}
\end{Schunk}

Consumption is observed in rd 2-4. There are 4051 observations, with first-differencing, it becomes 2650 observations with 96, 2554 households observed for 2, 3 times. 


\begin{Schunk}
\begin{Sinput}
source(paste0(pathprogram, "ConsumptionCovariateSelection.R"))
\end{Sinput}
\end{Schunk}

\begin{Schunk}
\begin{Sinput}
source(paste0(pathprogram, "ConsumptionCovariateSelectionRobustness.R"))
\end{Sinput}
\end{Schunk}




\hspace{-1cm}\begin{minipage}[t]{14cm}
\hfil\textsc{\normalsize Table \refstepcounter{table}\thetable: FD estimation of consumption\label{tab FD consumption}}\\
\setlength{\tabcolsep}{1pt}
\setlength{\baselineskip}{8pt}
\renewcommand{\arraystretch}{.55}
\hfil\begin{tikzpicture}
\node (tbl) {\input{c:/data/GUK/analysis/save/membership1or4/consumptionFDEstimationResults.tex}};
%\input{c:/dropbox/data/ramadan/save/tablecolortemplate.tex}
\end{tikzpicture}\\
\renewcommand{\arraystretch}{.8}
\setlength{\tabcolsep}{1pt}
\begin{tabular}{>{\hfill\scriptsize}p{1cm}<{}>{\hfill\scriptsize}p{.25cm}<{}>{\scriptsize}p{12cm}<{\hfill}}
Source:& \multicolumn{2}{l}{\scriptsize Estimated with GUK administrative and survey data.}\\
Notes: & 1. & First-difference estimates. A first-difference is defined as $\Delta x_{t+1}\equiv x_{t+1} - x_{t}$. Saving and repayment misses are taken from administrative data and merged with survey data at Year-Month of survey interviews. Intercept terms are omitted in estimating equations. Sample is continuing members and replacing members of early rejecters and received loans prior to 2015 Janunary. Consumption is annualised values. \\
& 2. & ${}^{***}$, ${}^{**}$, ${}^{*}$ indicate statistical significance at 1\%, 5\%, 10\%, respetively. Standard errors are clustered at group (village) level.
\end{tabular}
\end{minipage}


\hspace{-1cm}\begin{minipage}[t]{14cm}
\hfil\textsc{\normalsize Table \refstepcounter{table}\thetable: FD estimation of consumption, moderately poor vs. ultra poor\label{tab FD consumption2}}\\
\setlength{\tabcolsep}{1pt}
\setlength{\baselineskip}{8pt}
\renewcommand{\arraystretch}{.55}
\hfil\begin{tikzpicture}
\node (tbl) {\input{c:/data/GUK/analysis/save/membership1or4/consumptionPovertyStatusFDEstimationResults.tex}};
%\input{c:/dropbox/data/ramadan/save/tablecolortemplate.tex}
\end{tikzpicture}\\
\renewcommand{\arraystretch}{.8}
\setlength{\tabcolsep}{1pt}
\begin{tabular}{>{\hfill\scriptsize}p{1cm}<{}>{\hfill\scriptsize}p{.25cm}<{}>{\scriptsize}p{12cm}<{\hfill}}
Source:& \multicolumn{2}{l}{\scriptsize Estimated with GUK administrative and survey data.}\\
Notes: & 1. & First-difference estimates. A first-difference is defined as $\Delta x_{t+1}\equiv x_{t+1} - x_{t}$. Saving and repayment misses are taken from administrative data and merged with survey data at Year-Month of survey interviews. Intercept terms are omitted in estimating equations. Sample is continuing members and replacing members of early rejecters and received loans prior to 2015 Janunary. Consumption is annualised values. \\
& 2. & ${}^{***}$, ${}^{**}$, ${}^{*}$ indicate statistical significance at 1\%, 5\%, 10\%, respetively. Standard errors are clustered at group (village) level.
\end{tabular}
\end{minipage}


\hspace{-1cm}\begin{minipage}[t]{14cm}
\hfil\textsc{\normalsize Table \refstepcounter{table}\thetable: FD estimation of consumption, large vs. small size loans\label{tab FD consumption3}}\\
\setlength{\tabcolsep}{1pt}
\setlength{\baselineskip}{8pt}
\renewcommand{\arraystretch}{.55}
\hfil\begin{tikzpicture}
\node (tbl) {\input{c:/data/GUK/analysis/save/membership1or4/consumptionSizeFDEstimationResults.tex}};
%\input{c:/dropbox/data/ramadan/save/tablecolortemplate.tex}
\end{tikzpicture}\\
\renewcommand{\arraystretch}{.8}
\setlength{\tabcolsep}{1pt}
\begin{tabular}{>{\hfill\scriptsize}p{1cm}<{}>{\hfill\scriptsize}p{.25cm}<{}>{\scriptsize}p{12cm}<{\hfill}}
Source:& \multicolumn{2}{l}{\scriptsize Estimated with GUK administrative and survey data.}\\
Notes: & 1. & First-difference estimates. A first-difference is defined as $\Delta x_{t+1}\equiv x_{t+1} - x_{t}$. Saving and repayment misses are taken from administrative data and merged with survey data at Year-Month of survey interviews. Intercept terms are omitted in estimating equations. Sample is continuing members and replacing members of early rejecters and received loans prior to 2015 Janunary. Consumption is annualised values. \\
& 2. & ${}^{***}$, ${}^{**}$, ${}^{*}$ indicate statistical significance at 1\%, 5\%, 10\%, respetively. Standard errors are clustered at group (village) level.
\end{tabular}
\end{minipage}

\hspace{-1cm}\begin{minipage}[t]{14cm}
\hfil\textsc{\normalsize Table \refstepcounter{table}\thetable: FD estimation of consumption, with vs. without a grace period\label{tab FD consumption4}}\\
\setlength{\tabcolsep}{1pt}
\setlength{\baselineskip}{8pt}
\renewcommand{\arraystretch}{.55}
\hfil\begin{tikzpicture}
\node (tbl) {\input{c:/data/GUK/analysis/save/membership1or4/consumptionGraceFDEstimationResults.tex}};
%\input{c:/dropbox/data/ramadan/save/tablecolortemplate.tex}
\end{tikzpicture}\\
\renewcommand{\arraystretch}{.8}
\setlength{\tabcolsep}{1pt}
\begin{tabular}{>{\hfill\scriptsize}p{1cm}<{}>{\hfill\scriptsize}p{.25cm}<{}>{\scriptsize}p{12cm}<{\hfill}}
Source:& \multicolumn{2}{l}{\scriptsize Estimated with GUK administrative and survey data.}\\
Notes: & 1. & First-difference estimates. A first-difference is defined as $\Delta x_{t+1}\equiv x_{t+1} - x_{t}$. Saving and repayment misses are taken from administrative data and merged with survey data at Year-Month of survey interviews. Intercept terms are omitted in estimating equations. Sample is continuing members and replacing members of early rejecters and received loans prior to 2015 Janunary. Consumption is annualised values. \\
& 2. & ${}^{***}$, ${}^{**}$, ${}^{*}$ indicate statistical significance at 1\%, 5\%, 10\%, respetively. Standard errors are clustered at group (village) level.
\end{tabular}
\end{minipage}

\hspace{-1cm}\begin{minipage}[t]{14cm}
\hfil\textsc{\normalsize Table \refstepcounter{table}\thetable: FD estimation of consumption, loan recipients vs. pure control\label{tab FD consumption pure control}}\\
\setlength{\tabcolsep}{1pt}
\setlength{\baselineskip}{8pt}
\renewcommand{\arraystretch}{.55}
\hfil\begin{tikzpicture}
\node (tbl) {\input{c:/data/GUK/analysis/save/membership1or4/ConsumptionRobustnessFDEstimationResults.tex}};
%\input{c:/dropbox/data/ramadan/save/tablecolortemplate.tex}
\end{tikzpicture}\\
\renewcommand{\arraystretch}{.8}
\setlength{\tabcolsep}{1pt}
\begin{tabular}{>{\hfill\scriptsize}p{1cm}<{}>{\hfill\scriptsize}p{.25cm}<{}>{\scriptsize}p{12cm}<{\hfill}}
Source:& \multicolumn{2}{l}{\scriptsize Estimated with GUK administrative and survey data.}\\
Notes: & 1. & First-difference estimates of round 2 - 4. A first-difference is defined as $\Delta x_{t+1}\equiv x_{t+1} - x_{t}$. Saving and repayment misses are taken from administrative data and merged with survey data at Year-Month of survey interviews. Intercept terms are omitted in estimating equations. Sample is continuing members and replacing members of early rejecters and received loans prior to 2015 Janunary. Regressions (1)-(3), (5)-(6) use only arm and calendar information. (4) and (7) use previous six month repayment and saving information which is lacking in rd 1, hence starts from rd 2. Pure controls are households who rejected to receive a loan.\\
& 2. & ${}^{***}$, ${}^{**}$, ${}^{*}$ indicate statistical significance at 1\%, 5\%, 10\%, respetively. Standard errors are clustered at group (village) level.
\end{tabular}
\end{minipage}


\begin{palepinkleftbar}
\begin{finding}
\textsc{\small Table \ref{tab FD consumption}} uses rd 2 - 4 data and shows an increase in per member consumption in rd 2 - 3 period. The estimates are imprecise for all interaction terms. Continued increases in consumption hints welfare gains, but do not differ by arms. Per member food consumption increases in rd 2- 3 period but decreases in rd 3 - 4 period.
\end{finding}
\end{palepinkleftbar}


\subsection{IGA}


As written in the above at livestock section, IGA is misreported. 
\begin{Schunk}
\begin{Soutput}
        PositiveCows
CowAsIGA FALSE TRUE <NA>
   FALSE   524  430    5
   TRUE   1331 2588   93
   <NA>    334  328   12
\end{Soutput}
\begin{Soutput}
, , survey = 1

        PositiveCows
CowAsIGA FALSE TRUE <NA>
   FALSE   258   57    0
   TRUE    870  190    0
   <NA>    178   42    0

, , survey = 2

        PositiveCows
CowAsIGA FALSE TRUE <NA>
   FALSE    99  109    4
   TRUE    287  648   49
   <NA>     69   94    9

, , survey = 3

        PositiveCows
CowAsIGA FALSE TRUE <NA>
   FALSE    82  133    1
   TRUE     62  890   39
   <NA>     53  112    3

, , survey = 4

        PositiveCows
CowAsIGA FALSE TRUE <NA>
   FALSE    85  131    0
   TRUE    112  860    5
   <NA>     34   80    0
\end{Soutput}
\end{Schunk}
\textsf{CowAsIGA} = T, \textsf{NumCows} $=0$: \textsf{CowAsIGA} = T may be reported as an intention, not an actual activity.
\begin{Schunk}
\begin{Soutput}
      Arm
survey traditional large large grace cow
     1          82   260         263 265
     2          27   107          89  64
     3           9    14          18  21
     4          16    31          36  29
\end{Soutput}
\end{Schunk}
\textsf{CowAsIGA} = F, \textsf{NumCows} $>0$: 
\begin{Schunk}
\begin{Soutput}
      Arm
survey traditional large large grace cow
     1          33     7           8   9
     2          47    20          13  29
     3          61    25          16  31
     4          55    28          15  33
\end{Soutput}
\end{Schunk}
Revise IGA:
\begin{itemize}
\vspace{1.0ex}\setlength{\itemsep}{1.0ex}\setlength{\baselineskip}{12pt}
\item	\textsf{CowAsIGA} = T, \textsf{NumCows} $=0$ for all rounds: \textsf{CowAsIGA} = F.
\item	\textsf{CowAsIGA} = F, \textsf{NumCows} $>0$ for any round: \textsf{CowAsIGA} = T.
\end{itemize}

Find HHs who do not report cows as IGA and copy \textsf{CowAsIGA}.


\begin{Schunk}
\begin{figure}

{\centering \includegraphics[width=\maxwidth]{figure/ImpactEstimationMembership1or4/All_IGA_choices-1} 

}

\caption{All income generatng activity choices\\ {\footnotesize All of multiple investment choices are summed by arms and the number of IGAs are plotted as bars. Cow as IGA is corrected with livestock ownership information. \setlength{\baselineskip}{8pt}}}\label{Figure All IGA choices}
\end{figure}
\end{Schunk}
\begin{Schunk}
\begin{figure}

{\centering \includegraphics[width=\maxwidth]{figure/ImpactEstimationMembership1or4/All_IGA_choices_collapsed-1} 

}

\caption{All income generatng activity choices collapsed over different number of IGAs\\ {\footnotesize All of multiple investment choices are summed by arms and plotted as bars. Cow as IGA is corrected with livestock ownership information.\setlength{\baselineskip}{8pt}}}\label{Figure All IGA choices collapsed}
\end{figure}
\end{Schunk}


Given that it was minority households who owned a cow at baseline, cow as reported IGA in the first round indicates it is likely to include member's intention, not just actual ownership. So we base IGA according to ownership. \textsc{\small Figure \ref{Figure all livestock holding}} shows that fewer members own poultry and goat/sheep in all arms, while cow ownership exapanded in all arms but least pronounced in \textsf{traditional} arm (see the close-up plots in \textsc{\small Figure \ref{Figure cow holding}}). This suggests a loan triggered the substitution of smaller, less profitable livestock with bigger, and more profitable livestock.

\begin{Schunk}
\begin{figure}

{\centering \includegraphics[width=\maxwidth]{figure/ImpactEstimationMembership1or4/all_livestock_holding-1} 

}

\caption[Livestock holding\\ {\footnotesize \setlength{\baselineskip}{8pt}}]{Livestock holding\\ {\footnotesize \setlength{\baselineskip}{8pt}}}\label{Figure all livestock holding}
\end{figure}
\end{Schunk}
\begin{Schunk}
\begin{figure}

{\centering \includegraphics[width=\maxwidth]{figure/ImpactEstimationMembership1or4/cow_holding-1} 

}

\caption[Livestock holding\\ {\footnotesize \setlength{\baselineskip}{8pt}}]{Livestock holding\\ {\footnotesize \setlength{\baselineskip}{8pt}}}\label{Figure cow holding}
\end{figure}
\end{Schunk}

\begin{Schunk}
\begin{figure}

{\centering \includegraphics[width=\maxwidth]{figure/ImpactEstimationMembership1or4/IGA_choices_unrevised-1} 

}

\caption{Income generatng activity choices\\ {\footnotesize The first income generating activity choices are plotted.\setlength{\baselineskip}{8pt}}}\label{Figure IGA choices unrevised}
\end{figure}
\end{Schunk}
\begin{Schunk}
\begin{figure}

{\centering \includegraphics[width=\maxwidth]{figure/ImpactEstimationMembership1or4/All_IGA_choices_unrevised-1} 

}

\caption{All income generatng activity choices\\ {\footnotesize All of multiple investment choices are summed by arms and the number of IGAs are plotted as bars. \setlength{\baselineskip}{8pt}}}\label{Figure All IGA choices unrevised}
\end{figure}
\end{Schunk}
\begin{Schunk}
\begin{figure}

{\centering \includegraphics[width=\maxwidth]{figure/ImpactEstimationMembership1or4/All_IGA_choices_collapsed_unrevised-1} 

}

\caption{All income generatng activity choices collapsed over different number of IGAs\\ {\footnotesize All of multiple investment choices are summed by arms and plotted as bars. \setlength{\baselineskip}{8pt}}}\label{Figure All IGA choices collapsed unrevised}
\end{figure}
\end{Schunk}

\begin{palepinkleftbar}
\begin{finding}
\textsc{\small Figure \ref{Figure All IGA choices}, \ref{Figure All IGA choices collapsed}} show that there are very few members who chose to invest in more than one project for the ``large'' arms, while in the \textsf{traditional} arm, almost no one invested only in one project. Goat/sheep and small trades are the top choices for the first IGA in \textsf{traditional}. This indicates the exitence of both a liquidity constraint and convexity in the production technology of large domestic animals. This also validates our supposition that dairy livestock production is the most preferred and probably the only economically viable investment choice. It reduces a concern that the \textsf{cow} arm may have imposed an unnecessary restriction in an investment choice by forcing to receive a cow. \textsc{\small Figure \ref{Figure All IGA choices collapsed}} shows there are a significant number of cases in the \textsf{traditional} arm that members reportedly raise cows, yet they are also accompanied by pararell projects in smaller livestock production and small trades. Contrasting \textsf{large}, \textsf{large grace} with \textsf{cow} arms, it suggests that entrepreneurship (to the extent that is necessary for dairy livestock production) may not be an impediment for a microfinance loan uptake among members.
\end{finding}
\end{palepinkleftbar}

Together with \textsc{\small Table \ref{tab FD saving}} showing smaller net saving and repayment among \textsf{traditional}, the restriction on a project choice induced by a smaller loaned sum resulted in smaller returns. Between with or no grace period loans, cumulative net saving and repayment are both larger with loans with a grace period. No such difference is found between \textsf{cow} and other arms.


\subsection{Marriage}

\begin{Schunk}
\begin{Soutput}
                 TradGroup
creditstatus      planned twice double <NA>
  Yes                  63   409    265 6182
  No                    0     0      0 1714
  Replaced Member       0     0      0    0
\end{Soutput}
\begin{Soutput}
           Arm NumEligible.1 NumEligible.2 NumEligible.3 NumEligible.4
1:        <NA>            59             1             0            59
2: traditional            87             0             0           214
3:       large           110             1             0           222
4: large grace           124             1             2           246
5:         cow           115             0             0           254
\end{Soutput}
\end{Schunk}
Tabulate marriage for \textsf{sex} == ``Female" \& \textsf{ReadyToMarry}, where the latter is unmarried females with ages between 10 and 40.

When we compare the marriage rates, we need to define the denominator sensibly. It should be all relevant aged females that are present in baseline. As we do not want to include marriages immediately after receiving loans, we need to take off some period to count the marriage cases. We will consider 1 year, 2 years, and 3 years. At the same time, there are househods who chose not to receive a loan. 
Then, we need the denominator to be relevant aged females who do not attrit by:
\begin{itemize}
\vspace{1.0ex}\setlength{\itemsep}{1.0ex}\setlength{\baselineskip}{12pt}
\item	1 year (499 individuals), or,
\item	2 years (334 individuals), or,
\item	3 years (242 individuals).
\end{itemize}
\begin{Schunk}
\begin{Soutput}
            Arm AttritedBefore NumEligible Married MarriageRate
 1:        <NA>         year 1         119       0            0
 2: traditional         year 1          66       0            0
 3: traditional         year 2          23       0            0
 4: traditional         year 3          75       0            0
 5: traditional          never         137       0            0
 6:       large         year 1          12       0            0
 7:       large         year 2          30       0            0
 8:       large         year 3          68       0            0
 9:       large          never         223       0            0
10: large grace         year 1          16       0            0
11: large grace         year 2          74       0            0
12: large grace         year 3          80       0            0
13: large grace          never         203       0            0
14:         cow         year 1          39       0            0
15:         cow         year 2          58       0            0
16:         cow         year 3          96       0            0
17:         cow          never         176       0            0
\end{Soutput}
\end{Schunk}
\begin{palepinkleftbar}
\begin{finding}
There is very small difference in marriage rates between arms with grace and without grace.
\end{finding}
\end{palepinkleftbar}


\newpage
\section{Descriptive statistics of original 1600 HHs}




\begin{description}
\vspace{1.0ex}\setlength{\itemsep}{1.0ex}\setlength{\baselineskip}{12pt}
\item[c]	continuing members.
\item[d]	drop out members.
\item[a]	absence.
\item[n]	members of a new group.
\item[r]	replacing members.
\end{description}
\begin{Schunk}
\begin{Soutput}
         Mpattern
ObPattern caaa caca cacc ccaa ccac ccca cccc daaa dada dadd ddaa ddda dddd naaa
     0111    0    0   14    0    0    0    0    0    0   13    0    0    0    0
     1000   25    0    0    0    0    0    0   68    0    0    0    0    0    5
     1010    0    4    0    0    0    0    0    0    1    0    0    0    0    0
     1011    0    0    0    0    1    0    0    0    0    0    0    0    0    0
     1100    0    0    0   11    0    0    0    0    0    0   14    0    0    0
     1110    0    0    0    0    0   13    0    0    0    0    0   54    0    0
     1111    0    0    0    0    0    0 1153    0    0    0    0    0  229    0
         Mpattern
ObPattern nann nnaa nnna nnnn raaa rara rarr rraa rrra rrrr
     0111    4    0    0    0    0    0    5    0    0    0
     1000    0    0    0    0    2    0    0    0    0    0
     1010    0    0    0    0    0    1    0    0    0    0
     1011    0    0    0    0    0    0    0    0    0    0
     1100    0    2    0    0    0    0    0    1    0    0
     1110    0    0    9    0    0    0    0    0    6    0
     1111    0    0    0  440    0    0    0    0    0  144
\end{Soutput}
\end{Schunk}
\textsf{AttritIn}: Attrition round. 9 is nonattriting members.
\begin{Schunk}
\begin{Soutput}

   2    3    4    9 
 100   56  258 7975 
\end{Soutput}
\begin{Soutput}
        ObPattern
AttritIn 0111 1000 1010 1011 1100 1110 1111
       2    0  100    0    0    0    0    0
       3    0    0    0    0   28    0    0
       4    0    0    6    0    0   82    0
       9   36    0    0    1    0    0 1966
\end{Soutput}
\begin{Soutput}
        survey
AttritIn    1    2    3    4
       2  100    0    0    0
       3   28   28    0    0
       4   88   82   88    0
       9 2003 1967 2002 2003
\end{Soutput}
\end{Schunk}
\textsf{Mstatus} changes for some \textsf{groupid}s. Correct \textsf{Mstatus} by checking \textsf{comment} for dropping out (taken from CharRandomization2012.prn).
\begin{Schunk}
\begin{Soutput}
              survey
Mstatus          1   2   3   4
  gErosion       0   0   0   0
  gRejection   114 114 114   0
  iRejection     1   1   1 114
  iReplacement   0   0   0   0
  newGroup       0   0   0   0
  oldMember      0   0   0   1
\end{Soutput}
\end{Schunk}
See how \textsf{Mstatus} changes at rd 4: This suggests \textsf{iRejection} needs to change to \textsf{gRejection}, and \textsf{iRejection} to \textsf{oldMember}.
\begin{Schunk}
\begin{Soutput}
              survey
Mstatus          1   2   3   4
  gErosion      80  55  54   0
  gRejection   140 118 114   0
  iRejection     7   7   5 118
  iReplacement   6   6   6   6
  newGroup       0   0   0   0
  oldMember     13  13  13  14
\end{Soutput}
\end{Schunk}
\textsf{group.id} (created from first characters of \textsf{hhid}) and their reasons for dropping out.
\begin{Schunk}
\begin{Soutput}
        comment
group.id denial <NA>
   70317     19    0
   70319     20    0
   70539     16    0
   70858     20    0
   71372      0    1
   81483     20    0
   81697     19    0
\end{Soutput}
\end{Schunk}
Correct \textsf{Mstatus} in rd 4 from \textsf{iRejection} to \textsf{gRejection} if denial is the \textsf{comment}. \gobblepars
\begin{Schunk}
\begin{Soutput}
              survey
Mstatus          1   2   3   4
  gErosion       0   0   0   0
  gRejection   114 114 114 114
  iRejection     1   1   1   0
  iReplacement   0   0   0   0
  newGroup       0   0   0   0
  oldMember      0   0   0   1
\end{Soutput}
\end{Schunk}
Correct \textsf{Mstatus} in rd 1-3 from \textsf{iRejection} to \textsf{oldMember} if NA is the \textsf{comment}. \gobblepars
\begin{Schunk}
\begin{Soutput}
      hhid    Mstatus survey creditstatus
1: 7137220 iRejection      1          Yes
2: 7137220 iRejection      2          Yes
3: 7137220 iRejection      3          Yes
4: 7137220  oldMember      4          Yes
\end{Soutput}
\begin{Soutput}
              survey
Mstatus        1 2 3 4
  gErosion     0 0 0 0
  gRejection   0 0 0 0
  iRejection   1 1 1 0
  iReplacement 0 0 0 0
  newGroup     0 0 0 0
  oldMember    0 0 0 1
\end{Soutput}
\begin{Soutput}
              survey
Mstatus        1 2 3 4
  gErosion     0 0 0 0
  gRejection   0 0 0 0
  iRejection   0 0 0 0
  iReplacement 0 0 0 0
  newGroup     0 0 0 0
  oldMember    1 1 1 1
\end{Soutput}
\end{Schunk}
Original 1600 HHs (original sample) by arm and membership status.
\begin{Schunk}
\begin{Soutput}
              AssignOriginal
Mstatus        traditional large large grace cow
  gErosion              40     0          20  20
  gRejection            80    40          20   0
  iRejection            53    12          22  72
  iReplacement           0     0           0   0
  newGroup               0     0           0   0
  oldMember            227   348         338 308
\end{Soutput}
\end{Schunk}
Including \textsf{r} or individually replacing HHs (replacing sample): 1759
\begin{Schunk}
\begin{Soutput}
              AssignOriginal
Mstatus        traditional large large grace cow
  gErosion              40     0          20  20
  gRejection            80    40          20   0
  iRejection            53    12          22  72
  iReplacement          53    12          22  72
  newGroup               0     0           0   0
  oldMember            227   348         338 308
\end{Soutput}
\end{Schunk}
First disbursement year of individual and replacing samples. We have about 100+ in 2013 for replacing sample.
\begin{Schunk}
\begin{Soutput}
          2013 2014 2015 <NA>
original   679  313  203  405
replacing  771  348  232  408
\end{Soutput}
\end{Schunk}
Use original sample. \gobblepars

Attrition. % (\textsf{WillAttrit} is just a check if reshaping forced 4 obs per HH.)
\begin{Schunk}
\begin{Soutput}
   WillAttrit
tee    0    1
  1 1410  190
  2 1410  190
  3 1410  190
  4 1410  190
\end{Soutput}
\end{Schunk}
Merge \textsf{xid} with other files. Keep \textsf{all==T}.
\begin{Schunk}
\begin{Sinput}
xid[, Fromxid := T]
datafiles <- c("s1", "s2", "ar", "ass", "lvo", "lab", "far", "con")
Datafiles <- c("S1", "S2", "Ar", "Ass", "Lvo", "Lab", "Far", "Con")
DataFileNames <- c(
  "Schooling", "AugmentedSchooling", "Repayment", "Asset", "Livestock", 
  "LabourIncome", "FarmIncome", "Consumption")
#lapply(datafiles, function(x) 
#  grepout("AssignO|^Arm$|groupi|hhid|tee", colnames(get(x))))
# use only rd 1 characteristics
xid[, c("year") := NULL]
setkey(xid, AssignOriginal, groupid, hhid, tee)
# tee numbering is not in line with survey. This causes multiple matches per hhid-tee below. Correct.
corrtee <- c("ar", "ass", "lvo")
for (i in corrtee) {
  this <- get(i)
  setkey(this, hhid, survey)
  this[, tee := NULL]
  this[, tee := 1:.N, by = hhid]
  assign(i, this)
}
for (i in 1:length(datafiles)) {
  X <- get(datafiles[i])
  X[, FromFile := 1L]
  # files up to livestock do not have AssignOriginal
  if (i >= 5)
    xx <- merge(xid, X, by = key(xid)[-1], all = T, 
      suffixes = c("", paste0("From", Datafiles[i]))) else
    xx <- merge(xid, X, by = key(xid), all = T, 
      suffixes = c("", paste0("From", Datafiles[i])))
  xx[is.na(FromFile), FromFile := 0L]
  assign(paste0(datafiles[i], "x"), xx)
  saveRDS(xx, paste0(pathsaveHere, "Roster", DataFileNames[i],
     "AdminOriginalHHsDataUsedForEstimation.rds"))
}
\end{Sinput}
\end{Schunk}

%Attririon by membership status in repayment-saving:

Membership status in schooling: Schooling files have multiple observations per household.
\begin{Schunk}
\begin{Soutput}

  gErosion gRejection iRejection  oldMember 
        80        140        234       1872 
\end{Soutput}
\end{Schunk}
Number of obs per survey round in the schooling file:
\begin{Schunk}
\begin{Soutput}
      tee
teenum    1    2    3    4
     1 1600 1600 1600 1600
     2  682  511  446  322
     3  248  150  120   83
     4   50   26   17   11
     5   13    3    2    2
     6    2    0    0    0
\end{Soutput}
\end{Schunk}
Assets: Original arm assignment by membership status in rd 1: 1820 households.
\begin{Schunk}
\begin{Soutput}
              AssignOriginal
Mstatus        traditional large large grace cow <NA>
  gErosion              40     0          20  20    0
  gRejection            80    40          20   0    0
  iRejection            53    12          22  72    0
  iReplacement           0     0           0   0    0
  newGroup               0     0           0   0    0
  oldMember            227   348         338 308    0
  <NA>                   0     0           0   0  220
\end{Soutput}
\end{Schunk}


\hfil\begin{minipage}[t]{12cm}
\hfil\textsc{\normalsize Table \refstepcounter{table}\thetable: Number of observations from original 1600 HHs in round 1\label{tab NObsOH}}\\
\setlength{\tabcolsep}{.5pt}
\setlength{\baselineskip}{10pt}
\renewcommand{\arraystretch}{.7}
\hfil\begin{tikzpicture}
\node (tbl) {\input{c:/data/GUK/analysis/save/membership1or4/NumObsOriginalHHs.tex}};
%\input{c:/dropbox/data/ramadan/save/tablecolortemplate.tex}
\end{tikzpicture}\\
\renewcommand{\arraystretch}{.8}
\setlength{\tabcolsep}{1pt}
\begin{tabular}{>{\hfill\scriptsize}p{1cm}<{}>{\hfill\scriptsize}p{.25cm}<{}>{\scriptsize}p{10cm}<{\hfill}}
Source:& \multicolumn{2}{l}{\scriptsize Estimated with GUK administrative and survey data.}\\
Notes: & 1. & \\
& 2. &  
\end{tabular}
\end{minipage}







\section{Estimation using original 1600 HHs}


\subsection{Schooling}




Enrollment pattern in original schooling panel. `n' indicates NA (either attrition or not reported).
\begin{Schunk}
\begin{Soutput}
         SchPattern
ObPattern 0000 0001 000n 0011 001n 00nn 0100 010n 0111 011n 01nn 0nnn 1000 1001
     0111    0    0    0    0    0    0    0    0    0    2    2    6    0    0
     1000    0    0    0    0    0    0    0    0    0    0    0   17    0    0
     1010    0    0    0    0    0    1    0    0    0    0    0    2    0    0
     1011    0    0    0    0    0    0    0    0    0    0    0    0    0    0
     1100    0    0    0    0    0    1    0    0    0    0    5    2    0    0
     1110    0    0    0    0    1    0    0    0    0    3    0    0    0    0
     1111   27    5   35   20    4   44    2    2  158   15   10  167   12    2
     <NA>   13    2   13    5    1    9    0    0   15    5    1   66    1    0
         SchPattern
ObPattern 100n 1011 101n 10nn 1100 1101 110n 1110 1111 111n 11n1 11nn 1nnn nnnn
     0111    0    0    0    1    0    0    0    0    0   12    0    0    5   12
     1000    0    0    0    0    0    0    0    0    0    0    0    0   30   64
     1010    0    0    0    0    0    0    0    0    0    0    0    0    3    2
     1011    0    0    0    0    0    0    0    0    0    0    0    1    0    0
     1100    0    0    0    0    0    0    0    0    0    0    0   12    3   14
     1110    0    0    1    0    0    0    0    0    0    8    0    4    0   59
     1111    8    9    4   15   10    1   16    4  725   75    1   38  131  513
     <NA>    3    0    0    2    1    0    1    0   56   36    0    8   31    0
\end{Soutput}
\end{Schunk}
Enrollment pattern in augmented panel.
\begin{Schunk}
\begin{Soutput}
            SchPattern
SchObPattern 000 0000 0001 001 0010 0011 010 0100 0101 011 0110 0111 0n00 0n01
        1      0    0    0   0    0    0   0    0    0   0    0    0    0    0
        1234   0    0    0   0    0    0   0    0    0   0    0    0    0    0
        13     0    0    0   0    0    0   0    0    0   0    0    0    0    0
        134    0    0    0   0    0    0   0    0    0   0    0    0   32    2
        1345   0  234   10   0    5   33   0    9    2   0    4  173    0    0
        135    4    0    0   0    0    0   0    0    0   0    0    0    0    0
        14     0    0    0   0    0    0   0    0    0   0    0    0    0    0
        145    6    0    0   1    0    0   1    0    0   2    0    0    0    0
            SchPattern
SchObPattern 0n0n 0n10 0n11 0n1n 0nn0 0nnn 100 1000 1001 1010 1011 110 1100
        1       0    0    0    0    0  101   0    0    0    0    0   0    0
        1234    0    0    0    0    0    0   0    0    0    0    0   0    0
        13     20    0    0    9    0    0   0    0    0    0    0   0    0
        134     0    2   10    0    0    0   0    0    0    0    0   0    0
        1345    0    0    0    0    0    0   0   74    7    2   25   0   51
        135     0    0    0    0    0    0   2    0    0    0    0   3    0
        14      0    0    0    0    1    0   0    0    0    0    0   0    0
        145     0    0    0    0    0    0   3    0    0    0    0   0    0
            SchPattern
SchObPattern 1101 111 1110 1111 1n00 1n01 1n0n 1n10 1n11 1n1n 1nn0 1nn1 1nnn
        1       0   0    0    0    0    0    0    0    0    0    0    0  129
        1234    0   0    0    0    0    0    0    0    0    0    0    0    0
        13      0   0    0    0    0    0   19    0    0   19    0    0    0
        134     0   0    0    0    9    4    0   14   49    0    0    0    0
        1345    7   0   54  782    0    0    0    0    0    0    0    0    0
        135     0   1    0    0    0    0    0    0    0    0    0    0    0
        14      0   0    0    0    0    0    0    0    0    0    3    1    0
        145     0  12    0    0    0    0    0    0    0    0    0    0    0
            SchPattern
SchObPattern nnnn
        1       0
        1234  664
        13      0
        134     0
        1345    0
        135     0
        14      0
        145     0
\end{Soutput}
\end{Schunk}
Drop any string with \textsf{nnn} in \textsf{SchPattern} as it does not form a panel.

Left panel is before dropping \textsf{nnn}, right panel is after: Original panel.
\begin{Schunk}
\begin{Soutput}
  traditional large large grace cow <NA> traditional large large grace cow
1         460   479         505 487  664         300   396         369 403
2         300   396         369 403  822         300   396         369 403
3         266   356         340 351  872         266   356         340 351
4         204   306         282 277  949         204   306         282 277
\end{Soutput}
\end{Schunk}
Augmented panel.
\begin{Schunk}
\begin{Soutput}
  traditional large large grace cow <NA> traditional large large grace cow <NA>
1         460   479         505 487  664         338   466         433 464    0
2           0     0           0   0 1600           0     0           0   0   15
3         336   460         422 453  779         336   460         422 453   15
4         325   448         417 434  787         325   448         417 434    0
5         278   439         401 389    0         278   439         401 389    0
\end{Soutput}
\end{Schunk}
\begin{Schunk}
\begin{Soutput}
Dropped 1520 obs due to NA.
Dropped 1520 obs due to NA.
Dropped 1796 obs due to NA.
Dropped 1796 obs due to NA.
Dropped 320 obs due to T<2.
Dropped 1037 obs due to NA.
Dropped 64 obs due to T<2.
Dropped 1563 obs due to NA.
\end{Soutput}
\end{Schunk}
If using \textsf{s1x}, retain only the complete portion of panel. \textsf{sch1} has 9088 rows. Drop 3770 observations in \textsf{sch1} with nnn in \textsf{SchPattern}. In augmented schooling panel, \textsf{sch2} has 10563 rows. Drop 4030 observations in \textsf{sch2} with nnn in \textsf{SchPattern}.
%and nrow(s.1x[!grepl("nnn", Spattern) & grepl("1001", EnrollPattern), ]) observations with 1001 in \textsf{EnrollPattern} because they are likely to be errors. This leaves us with nrow(s1x) rows. 
\begin{Schunk}
\begin{Soutput}
Warning in `[.data.table`(s1x, , `:=`(Fromxid, NULL)): Adding new column 'Fromxid' then assigning NULL (deleting it).
\end{Soutput}
\begin{Soutput}
Warning in `[.data.table`(s2x, , `:=`(Fromxid, NULL)): Adding new column 'Fromxid' then assigning NULL (deleting it).
\end{Soutput}
\end{Schunk}
With OLS,  134, 187, 975 individuals are repeatedly observed for 2, 3, 4 times, respectively. With FD, \textsf{s1x} is reduced to 3209 rows after first-differencing with 121, 176, 907 individuals with repeatedly observed for 2, 3, 4 times, respectively.
Individuals with NAs in \textsf{Enrolled}. 0, 0 obs for \textsf{s1x} and \textsf{s2x}. 
%Mostly older children (round(mean(s.1[is.na(Enrolled), Age_1], na.rm = T), 1) in \textsf{s1x}, round(mean(s.2[is.na(Enrolled), Age_1], na.rm = T), 1) in \textsf{s.2}) but with a high reported enrollment rate (round(mean(s.1[is.na(Enrolled) & tee == 4, Enrolled]), 1) for \textsf{s1x}, round(mean(s.2[is.na(Enrolled) & tee == 4, Enrolled]), 1) for \textsf{s.2}) at rd 4. We will substitute relevant schooling levels to \textsf{Enrolled}.
Check missingness in schooling level information.
\begin{Schunk}
\begin{Soutput}
   1 
4729 
\end{Soutput}
\end{Schunk}
Check missingness in arm information.
\begin{Schunk}
\begin{Soutput}
   1 
4729 
\end{Soutput}
\end{Schunk}
Drop 0 obs without school level information.

An example of dummy interactions: \textsf{\footnotesize dummyLargeSize.dummyPrimary.Time.2, dummySmallSize.dummyPrimary.Time.2, dummyLargeSize.dummyJunior.Time.2, dummySmallSize.dummyJunior.Time.2, dummyLargeSize.dummyHigh.Time.2, dummySmallSize.dummyHigh.Time.2, dummyLargeSize.dummyPrimary.Time.3, dummySmallSize.dummyPrimary.Time.3, dummyLargeSize.dummyJunior.Time.3, dummySmallSize.dummyJunior.Time.3, dummyLargeSize.dummyHigh.Time.3, dummySmallSize.dummyHigh.Time.3, dummyLargeSize.dummyPrimary.Time.4, dummySmallSize.dummyPrimary.Time.4, dummyLargeSize.dummyJunior.Time.4, dummySmallSize.dummyJunior.Time.4, dummyLargeSize.dummyHigh.Time.4, dummySmallSize.dummyHigh.Time.4}.
Obs for \textsf{s1x}.
\begin{Schunk}
\begin{Soutput}

   2    3    4 
1204 1087  918 
\end{Soutput}
\end{Schunk}
Obs for \textsf{s1x} and admin repayment data.
\begin{Schunk}
\begin{Soutput}

   2    3    4 
1204 1087  918 
\end{Soutput}
\begin{Soutput}

   3    4 
1346 1314 
\end{Soutput}
\end{Schunk}
Obs for survey \textsf{s2x}.
\begin{Schunk}
\begin{Soutput}

   3    4 
1346 1314 
\end{Soutput}
\end{Schunk}
Obs for survey \textsf{s2x} and admin repayment data.
\begin{Schunk}
\begin{Soutput}

   3    4 
1346 1314 
\end{Soutput}
\end{Schunk}
\begin{Schunk}
\begin{Sinput}
arsuffixes <- c("", "g", "p", "s")
source(paste0(pathprogram, "SchoolingCovariateSelection.R"))
\end{Sinput}
\end{Schunk}



\hspace{-1cm}\begin{minipage}[t]{14cm}
\hfil\textsc{\normalsize Table \refstepcounter{table}\thetable: FD estimation of school enrollment\label{tab FD enroll original HH}}\\
\setlength{\tabcolsep}{1pt}
\setlength{\baselineskip}{8pt}
\renewcommand{\arraystretch}{.48}
\hfil\begin{tikzpicture}
\node (tbl) {\input{c:/data/GUK/analysis/save/membership1or4/SchoolingOriginalHHsFDEstimationResults.tex}};
%\input{c:/dropbox/data/ramadan/save/tablecolortemplate.tex}
\end{tikzpicture}\\
\renewcommand{\arraystretch}{.8}
\setlength{\tabcolsep}{1pt}
\begin{tabular}{>{\hfill\scriptsize}p{1cm}<{}>{\hfill\scriptsize}p{.25cm}<{}>{\scriptsize}p{12cm}<{\hfill}}
Source:& \multicolumn{2}{l}{\scriptsize Estimated with GUK administrative and survey data.}\\
Notes: & 1. & First-difference estimates. A first-difference is defined as $\Delta x_{t+1}\equiv x_{t+1} - x_{t}$. First-differenced regressands are regressed on categorical and time-variant covariates. Net saving is taken from administrative data and merged with survey data at Year-Month of survey interviews. Head age and literacy are from baseline data. Intercept terms are omitted in estimating equations. Net saving is saving - withdrawal. \\
& 2. & ${}^{***}$, ${}^{**}$, ${}^{*}$ indicate statistical significance at 1\%, 5\%, 10\%, respetively. Standard errors are clustered at group (village) level.
\end{tabular}
\end{minipage}

\hspace{-1cm}\begin{minipage}[t]{14cm}
\hfil\textsc{\normalsize Table \refstepcounter{table}\thetable: FD estimation of net school enrollment, ultra poor vs. moderately poor\label{tab FD enroll2 original HH}}\\
\setlength{\tabcolsep}{1pt}
\setlength{\baselineskip}{8pt}
\renewcommand{\arraystretch}{.55}
\hfil\begin{tikzpicture}
\node (tbl) {\input{c:/data/GUK/analysis/save/membership1or4/SchoolingPovertystatusOriginalHHsFDEstimationResults.tex}};
%\input{c:/dropbox/data/ramadan/save/tablecolortemplate.tex}
\end{tikzpicture}\\
\renewcommand{\arraystretch}{.8}
\setlength{\tabcolsep}{1pt}
\begin{tabular}{>{\hfill\scriptsize}p{1cm}<{}>{\hfill\scriptsize}p{.25cm}<{}>{\scriptsize}p{12cm}<{\hfill}}
Source:& \multicolumn{2}{l}{\scriptsize Estimated with GUK administrative and survey data.}\\
Notes: & 1. & First-difference estimates. A first-difference is defined as $\Delta x_{t+1}\equiv x_{t+1} - x_{t}$. First-differenced regressands are regressed on categorical and time-variant covariates. Net saving is taken from administrative data and merged with survey data at Year-Month of survey interviews. Head age and literacy are from baseline data. Intercept terms are omitted in estimating equations. Net saving is saving - withdrawal. \\
& 2. & ${}^{***}$, ${}^{**}$, ${}^{*}$ indicate statistical significance at 1\%, 5\%, 10\%, respetively. Standard errors are clustered at group (village) level.
\end{tabular}
\end{minipage}

\hspace{-1cm}\begin{minipage}[t]{14cm}
\hfil\textsc{\normalsize Table \refstepcounter{table}\thetable: FD estimation of school enrollment, with vs. without a grace period\label{tab FD enroll3 original HH}}\\
\setlength{\tabcolsep}{1pt}
\setlength{\baselineskip}{8pt}
\renewcommand{\arraystretch}{.55}
\hfil\begin{tikzpicture}
\node (tbl) {\input{c:/data/GUK/analysis/save/membership1or4/SchoolingGraceOriginalHHsFDEstimationResults.tex}};
%\input{c:/dropbox/data/ramadan/save/tablecolortemplate.tex}
\end{tikzpicture}\\
\renewcommand{\arraystretch}{.8}
\setlength{\tabcolsep}{1pt}
\begin{tabular}{>{\hfill\scriptsize}p{1cm}<{}>{\hfill\scriptsize}p{.25cm}<{}>{\scriptsize}p{12cm}<{\hfill}}
Source:& \multicolumn{2}{l}{\scriptsize Estimated with GUK administrative and survey data.}\\
Notes: & 1. & First-difference estimates. A first-difference is defined as $\Delta x_{t+1}\equiv x_{t+1} - x_{t}$. First-differenced regressands are regressed on categorical and time-variant covariates. Net saving is taken from administrative data and merged with survey data at Year-Month of survey interviews. Head age and literacy are from baseline data. Intercept terms are omitted in estimating equations. Net saving is saving - withdrawal. All dummy interaction terms are first demeaned and then interacted.\\
& 2. & ${}^{***}$, ${}^{**}$, ${}^{*}$ indicate statistical significance at 1\%, 5\%, 10\%, respetively. Standard errors are clustered at group (village) level.
\end{tabular}
\end{minipage}

\hspace{-1cm}\begin{minipage}[t]{14cm}
\hfil\textsc{\normalsize Table \refstepcounter{table}\thetable: FD estimation of school enrollment, small size vs. large size\label{tab FD enroll4 original HH}}\\
\setlength{\tabcolsep}{1pt}
\setlength{\baselineskip}{8pt}
\renewcommand{\arraystretch}{.55}
\hfil\begin{tikzpicture}
\node (tbl) {\input{c:/data/GUK/analysis/save/membership1or4/SchoolingSizeOriginalHHsFDEstimationResults.tex}};
%\input{c:/dropbox/data/ramadan/save/tablecolortemplate.tex}
\end{tikzpicture}\\
\renewcommand{\arraystretch}{.8}
\setlength{\tabcolsep}{1pt}
\begin{tabular}{>{\hfill\scriptsize}p{1cm}<{}>{\hfill\scriptsize}p{.25cm}<{}>{\scriptsize}p{12cm}<{\hfill}}
Source:& \multicolumn{2}{l}{\scriptsize Estimated with GUK administrative and survey data.}\\
Notes: & 1. & First-difference estimates. A first-difference is defined as $\Delta x_{t+1}\equiv x_{t+1} - x_{t}$. First-differenced regressands are regressed on categorical and time-variant covariates. Net saving is taken from administrative data and merged with survey data at Year-Month of survey interviews. Head age and literacy are from baseline data. Intercept terms are omitted in estimating equations. Net saving is saving - withdrawal. All dummy interaction terms are first demeaned and then interacted.\\
& 2. & ${}^{***}$, ${}^{**}$, ${}^{*}$ indicate statistical significance at 1\%, 5\%, 10\%, respetively. Standard errors are clustered at group (village) level.
\end{tabular}
\end{minipage}


\hspace{-1cm}\begin{minipage}[t]{14cm}
\hfil\textsc{\normalsize Table \refstepcounter{table}\thetable: FD estimation of school enrollment, round 1 vs. round 4 differences\label{tab FD enroll5 original HH}}\\
\setlength{\tabcolsep}{1pt}
\setlength{\baselineskip}{8pt}
\renewcommand{\arraystretch}{.55}
\hfil\begin{tikzpicture}
\node (tbl) {\input{c:/data/GUK/analysis/save/membership1or4/SchoolingRd14DiffOriginalHHsFDEstimationResults.tex}};
%\input{c:/dropbox/data/ramadan/save/tablecolortemplate.tex}
\end{tikzpicture}\\
\renewcommand{\arraystretch}{.8}
\setlength{\tabcolsep}{1pt}
\begin{tabular}{>{\hfill\scriptsize}p{1cm}<{}>{\hfill\scriptsize}p{.25cm}<{}>{\scriptsize}p{12cm}<{\hfill}}
Source:& \multicolumn{2}{l}{\scriptsize Estimated with GUK administrative and survey data.}\\
Notes: & 1. & First-difference estimates. A first-difference is defined as $\Delta x_{t+1}\equiv x_{t+1} - x_{t}$. First-differenced regressands are regressed on categorical and time-variant covariates. Net saving is taken from administrative data and merged with survey data at Year-Month of survey interviews. Head age and literacy are from baseline data. Intercept terms are omitted in estimating equations. Net saving is saving - withdrawal. All dummy interaction terms are first demeaned and then interacted.\\
& 2. & ${}^{***}$, ${}^{**}$, ${}^{*}$ indicate statistical significance at 1\%, 5\%, 10\%, respetively. Standard errors are clustered at group (village) level.
\end{tabular}
\end{minipage}


\hspace{-1cm}\begin{minipage}[t]{14cm}
\hfil\textsc{\normalsize Table \refstepcounter{table}\thetable: FD estimation of school enrollment, round 1 vs. round 4 differences, grace period\label{tab FD enroll6 14  original HH}}\\
\setlength{\tabcolsep}{1pt}
\setlength{\baselineskip}{8pt}
\renewcommand{\arraystretch}{.55}
\hfil\begin{tikzpicture}
\node (tbl) {\input{c:/data/GUK/analysis/save/membership1or4/SchoolingRd14DiffGraceOriginalHHsFDEstimationResults.tex}};
%\input{c:/dropbox/data/ramadan/save/tablecolortemplate.tex}
\end{tikzpicture}\\
\renewcommand{\arraystretch}{.8}
\setlength{\tabcolsep}{1pt}
\begin{tabular}{>{\hfill\scriptsize}p{1cm}<{}>{\hfill\scriptsize}p{.25cm}<{}>{\scriptsize}p{12cm}<{\hfill}}
Source:& \multicolumn{2}{l}{\scriptsize Estimated with GUK administrative and survey data.}\\
Notes: & 1. & First-difference estimates. A first-difference is defined as $\Delta x_{t+1}\equiv x_{t+1} - x_{t}$. First-differenced regressands are regressed on categorical and time-variant covariates. Net saving is taken from administrative data and merged with survey data at Year-Month of survey interviews. Head age and literacy are from baseline data. Intercept terms are omitted in estimating equations. Net saving is saving - withdrawal. All dummy interaction terms are first demeaned and then interacted.\\
& 2. & ${}^{***}$, ${}^{**}$, ${}^{*}$ indicate statistical significance at 1\%, 5\%, 10\%, respetively. Standard errors are clustered at group (village) level.
\end{tabular}
\end{minipage}


\hspace{-1cm}\begin{minipage}[t]{14cm}
\hfil\textsc{\normalsize Table \refstepcounter{table}\thetable: FD estimation of school enrollment, period interactions\label{tab FD enroll6 original HH}}\\
\setlength{\tabcolsep}{1pt}
\setlength{\baselineskip}{8pt}
\renewcommand{\arraystretch}{.45}
\hfil\begin{tikzpicture}
\node (tbl) {\input{c:/data/GUK/analysis/save/membership1or4/SchoolingTIntOriginalHHsFDEstimationResults1.tex}};
\end{tikzpicture}\\
\renewcommand{\arraystretch}{.8}
\setlength{\tabcolsep}{1pt}
\begin{tabular}{>{\hfill\scriptsize}p{1cm}<{}>{\hfill\scriptsize}p{.25cm}<{}>{\scriptsize}p{12cm}<{\hfill}}
Source:& \multicolumn{2}{l}{\scriptsize Estimated with GUK administrative and survey data.}\\
Notes: & 1. & First-difference estimates. A first-difference is defined as $\Delta x_{t+1}\equiv x_{t+1} - x_{t}$. First-differenced regressands are regressed on categorical and time-variant covariates. Net saving is taken from administrative data and merged with survey data at Year-Month of survey interviews. Head age and literacy are from baseline data. Intercept terms are omitted in estimating equations. Net saving is saving - withdrawal. All dummy interaction terms are first demeaned and then interacted.\\
& 2. & ${}^{***}$, ${}^{**}$, ${}^{*}$ indicate statistical significance at 1\%, 5\%, 10\%, respetively. Standard errors are clustered at group (village) level.
\end{tabular}
\end{minipage}

\hspace{-1cm}\begin{minipage}[t]{14cm}
\hfil\textsc{\normalsize Table \ref{tab FD enroll6}: FD estimation of school enrollment, period interactions, continued \label{tab FD enroll6c original HH}}\\
\setlength{\tabcolsep}{1pt}
\setlength{\baselineskip}{8pt}
\renewcommand{\arraystretch}{.45}
\hfil\begin{tikzpicture}
\node (tbl) {\input{c:/data/GUK/analysis/save/membership1or4/SchoolingTIntOriginalHHsFDEstimationResults2.tex}};
\end{tikzpicture}\\
\renewcommand{\arraystretch}{.8}
\setlength{\tabcolsep}{1pt}
\begin{tabular}{>{\hfill\scriptsize}p{1cm}<{}>{\hfill\scriptsize}p{.25cm}<{}>{\scriptsize}p{12cm}<{\hfill}}
Source:& \multicolumn{2}{l}{\scriptsize Estimated with GUK administrative and survey data.}\\
Notes: & 1. & First-difference estimates. A first-difference is defined as $\Delta x_{t+1}\equiv x_{t+1} - x_{t}$. First-differenced regressands are regressed on categorical and time-variant covariates. Net saving is taken from administrative data and merged with survey data at Year-Month of survey interviews. Head age and literacy are from baseline data. Intercept terms are omitted in estimating equations. Net saving is saving - withdrawal. All dummy interaction terms are first demeaned and then interacted.\\
& 2. & ${}^{***}$, ${}^{**}$, ${}^{*}$ indicate statistical significance at 1\%, 5\%, 10\%, respetively. Standard errors are clustered at group (village) level.
\end{tabular}
\end{minipage}

\hspace{-1cm}\begin{minipage}[t]{14cm}
\hfil\textsc{\normalsize Table \refstepcounter{table}\thetable: FD estimation of school enrollment, period interactions, grace period\label{tab FD enroll7 original HH}}\\
\setlength{\tabcolsep}{1pt}
\setlength{\baselineskip}{8pt}
\renewcommand{\arraystretch}{.5}
\hfil\begin{tikzpicture}
\node (tbl) {\input{c:/data/GUK/analysis/save/membership1or4/SchoolingTIntGraceOriginalHHsFDEstimationResults.tex}};
%\input{c:/dropbox/data/ramadan/save/tablecolortemplate.tex}
\end{tikzpicture}\\
\renewcommand{\arraystretch}{.8}
\setlength{\tabcolsep}{1pt}
\begin{tabular}{>{\hfill\scriptsize}p{1cm}<{}>{\hfill\scriptsize}p{.25cm}<{}>{\scriptsize}p{12cm}<{\hfill}}
Source:& \multicolumn{2}{l}{\scriptsize Estimated with GUK administrative and survey data.}\\
Notes: & 1. & First-difference estimates. A first-difference is defined as $\Delta x_{t+1}\equiv x_{t+1} - x_{t}$. First-differenced regressands are regressed on categorical and time-variant covariates. Net saving is taken from administrative data and merged with survey data at Year-Month of survey interviews. Head age and literacy are from baseline data. Intercept terms are omitted in estimating equations. Net saving is saving - withdrawal. All dummy interaction terms are first demeaned and then interacted.\\
& 2. & ${}^{***}$, ${}^{**}$, ${}^{*}$ indicate statistical significance at 1\%, 5\%, 10\%, respetively. Standard errors are clustered at group (village) level.
\end{tabular}
\end{minipage}


\hspace{-1cm}\begin{minipage}[t]{14cm}
\hfil\textsc{\normalsize Table \refstepcounter{table}\thetable: FD estimation of school enrollment, period interactions, small vs. large sized loans\label{tab FD enroll8 original HH}}\\
\setlength{\tabcolsep}{1pt}
\setlength{\baselineskip}{8pt}
\renewcommand{\arraystretch}{.5}
\hfil\begin{tikzpicture}
\node (tbl) {\input{c:/data/GUK/analysis/save/membership1or4/SchoolingTIntSizeOriginalHHsFDEstimationResults.tex}};
%\input{c:/dropbox/data/ramadan/save/tablecolortemplate.tex}
\end{tikzpicture}\\
\renewcommand{\arraystretch}{.8}
\setlength{\tabcolsep}{1pt}
\begin{tabular}{>{\hfill\scriptsize}p{1cm}<{}>{\hfill\scriptsize}p{.25cm}<{}>{\scriptsize}p{12cm}<{\hfill}}
Source:& \multicolumn{2}{l}{\scriptsize Estimated with GUK administrative and survey data.}\\
Notes: & 1. & First-difference estimates. A first-difference is defined as $\Delta x_{t+1}\equiv x_{t+1} - x_{t}$. First-differenced regressands are regressed on categorical and time-variant covariates. Net saving is taken from administrative data and merged with survey data at Year-Month of survey interviews. Head age and literacy are from baseline data. Intercept terms are omitted in estimating equations. Net saving is saving - withdrawal. All dummy interaction terms are first demeaned and then interacted.\\
& 2. & ${}^{***}$, ${}^{**}$, ${}^{*}$ indicate statistical significance at 1\%, 5\%, 10\%, respetively. Standard errors are clustered at group (village) level.
\end{tabular}
\end{minipage}


\subsection{Repayment and net saving}


\begin{Schunk}
\begin{Soutput}
Warning in Ops.factor(X[[FDThese[i]]], X[[paste0("L", FDThese[i])]]): '-' not meaningful for factors
\end{Soutput}
\begin{Soutput}
Dropped 87 obs due to T<2.
Dropped 5854 obs due to NA.
\end{Soutput}
\begin{Soutput}
Warning in Ops.factor(X[[FDThese[i]]], X[[paste0("L", FDThese[i])]]): '-' not meaningful for factors
\end{Soutput}
\begin{Soutput}
Dropped 87 obs due to T<2.
Dropped 5856 obs due to NA.
\end{Soutput}
\begin{Soutput}
< table of extent 0 x 0 >
\end{Soutput}
\end{Schunk}
Repayment started in round 2. So taking a first-difference leaves us with period 2-3 and period 3-4. Drop 0 observations in \textsf{ar} that have round 1 data (for unknown reasons). After first-differencing, \textsf{ar} has 0 rows with  individuals with repeatedly observed for  times, respectively. NA individuals observed for 4 times started repayment even before official disbursement date, so its round 1 will be dropped.

Note all binary interaction terms are demeaned and then interacted.
\begin{Schunk}
\begin{Soutput}
   RArm
tee traditional large large grace cow
  1         400   400         400 400
  2         280   384         342 366
  3         277   386         348 366
  4         240   382         343 342
\end{Soutput}
\end{Schunk}
NAs in \textsf{CumRepaid}.
\begin{Schunk}
\begin{Soutput}
   Arm
tee traditional large large grace cow
  1         398   400         398 400
  2         113    41           0  20
  3         110    39           0  19
  4          75    39           0   0
\end{Soutput}
\end{Schunk}
Tabulation at rd 1:
\begin{Schunk}
\begin{Soutput}
              RArm
Mstatus        traditional large large grace cow
  gErosion              40     0          20  20
  gRejection            80    40          20   0
  iRejection            54    12          22  72
  iReplacement           0     0           0   0
  newGroup               0     0           0   0
  oldMember            226   348         338 308
\end{Soutput}
\end{Schunk}
\begin{Schunk}
\begin{figure}

{\centering \includegraphics[width=\maxwidth]{figure/ImpactEstimationMembership1or4/Cumulative_net_saving_original_HHs-1} 

}

\caption[Cumulative weekly net saving]{Cumulative weekly net saving}\label{Figure Cumulative net saving original HHs}
\end{figure}
\end{Schunk}


\begin{Schunk}
\begin{Sinput}
source(paste0(pathprogram, "RepaymentCovariateSelection.R"))
\end{Sinput}
\end{Schunk}







\hspace{-1cm}\begin{minipage}[t]{14cm}
\hfil\textsc{\normalsize Table \refstepcounter{table}\thetable: FD estimation of cumulative net saving and repayment\label{tab FD saving original HH}}\\
\setlength{\tabcolsep}{1pt}
\setlength{\baselineskip}{8pt}
\renewcommand{\arraystretch}{.55}
\hfil\begin{tikzpicture}
\node (tbl) {\input{c:/data/GUK/analysis/save/membership1or4/SavingOriginalHHsFDEstimationResults.tex}};
%\input{c:/dropbox/data/ramadan/save/tablecolortemplate.tex}
\end{tikzpicture}\\
\renewcommand{\arraystretch}{.8}
\setlength{\tabcolsep}{1pt}
\begin{tabular}{>{\hfill\scriptsize}p{1cm}<{}>{\hfill\scriptsize}p{.25cm}<{}>{\scriptsize}p{12cm}<{\hfill}}
Source:& \multicolumn{2}{l}{\scriptsize Estimated with GUK administrative and survey data.}\\
Notes: & 1. & First-difference estimates using rd 2 - rd 4 data. First-differenced ($\Delta x_{t+1}\equiv x_{t+1} - x_{t}$) regressands are regressed on categorical and time-variant covariates. Net saving is taken from administrative data and merged with survey data at Year-Month of survey interviews. Head age and literacy are from baseline data. Intercept terms are omitted in estimating equations. Net saving is saving - withdrawal. \\
& 2. & ${}^{***}$, ${}^{**}$, ${}^{*}$ indicate statistical significance at 1\%, 5\%, 10\%, respetively. Standard errors are clustered at group (village) level.
\end{tabular}
\end{minipage}

\hspace{-1cm}\begin{minipage}[t]{14cm}
\hfil\textsc{\normalsize Table \refstepcounter{table}\thetable: FD estimation of net cumulative saving and repayment, ultra poor vs. moderately poor\label{tab FD saving2 original HH}}\\
\setlength{\tabcolsep}{1pt}
\setlength{\baselineskip}{8pt}
\renewcommand{\arraystretch}{.55}
\hfil\begin{tikzpicture}
\node (tbl) {\input{c:/data/GUK/analysis/save/membership1or4/SavingPovertystatusOriginalHHsFDEstimationResults.tex}};
%\input{c:/dropbox/data/ramadan/save/tablecolortemplate.tex}
\end{tikzpicture}\\
\renewcommand{\arraystretch}{.8}
\setlength{\tabcolsep}{1pt}
\begin{tabular}{>{\hfill\scriptsize}p{1cm}<{}>{\hfill\scriptsize}p{.25cm}<{}>{\scriptsize}p{12cm}<{\hfill}}
Source:& \multicolumn{2}{l}{\scriptsize Estimated with GUK administrative and survey data.}\\
Notes: & 1. & First-difference estimates using rd 2 - rd 4 data. First-differenced ($\Delta x_{t+1}\equiv x_{t+1} - x_{t}$) regressands are regressed on categorical and time-variant covariates. Net saving is taken from administrative data and merged with survey data at Year-Month of survey interviews. Head age and literacy are from baseline data. Intercept terms are omitted in estimating equations. Net saving is saving - withdrawal. \\
& 2. & ${}^{***}$, ${}^{**}$, ${}^{*}$ indicate statistical significance at 1\%, 5\%, 10\%, respetively. Standard errors are clustered at group (village) level.
\end{tabular}
\end{minipage}

\hspace{-1cm}\begin{minipage}[t]{14cm}
\hfil\textsc{\normalsize Table \refstepcounter{table}\thetable: FD estimation of net cumulative saving and repayment, with vs. without a grace period\label{tab FD saving3 original HH}}\\
\setlength{\tabcolsep}{1pt}
\setlength{\baselineskip}{8pt}
\renewcommand{\arraystretch}{.55}
\hfil\begin{tikzpicture}
\node (tbl) {\input{c:/data/GUK/analysis/save/membership1or4/SavingGraceOriginalHHsFDEstimationResults.tex}};
%\input{c:/dropbox/data/ramadan/save/tablecolortemplate.tex}
\end{tikzpicture}\\
\renewcommand{\arraystretch}{.8}
\setlength{\tabcolsep}{1pt}
\begin{tabular}{>{\hfill\scriptsize}p{1cm}<{}>{\hfill\scriptsize}p{.25cm}<{}>{\scriptsize}p{12cm}<{\hfill}}
Source:& \multicolumn{2}{l}{\scriptsize Estimated with GUK administrative and survey data.}\\
Notes: & 1. & First-difference estimates using rd 2 - rd 4 data. First-differenced ($\Delta x_{t+1}\equiv x_{t+1} - x_{t}$) regressands are regressed on categorical and time-variant covariates. Net saving is taken from administrative data and merged with survey data at Year-Month of survey interviews. Head age and literacy are from baseline data. Intercept terms are omitted in estimating equations. Net saving is saving - withdrawal. All dummy interaction terms are first demeaned and then interacted.\\
& 2. & ${}^{***}$, ${}^{**}$, ${}^{*}$ indicate statistical significance at 1\%, 5\%, 10\%, respetively. Standard errors are clustered at group (village) level.
\end{tabular}
\end{minipage}

\hspace{-1cm}\begin{minipage}[t]{14cm}
\hfil\textsc{\normalsize Table \refstepcounter{table}\thetable: FD estimation of net cumulative saving and repayment, small size vs. large size\label{tab FD saving4 original HH}}\\
\setlength{\tabcolsep}{1pt}
\setlength{\baselineskip}{8pt}
\renewcommand{\arraystretch}{.55}
\hfil\begin{tikzpicture}
\node (tbl) {\input{c:/data/GUK/analysis/save/membership1or4/SavingSizeOriginalHHsFDEstimationResults.tex}};
%\input{c:/dropbox/data/ramadan/save/tablecolortemplate.tex}
\end{tikzpicture}\\
\renewcommand{\arraystretch}{.8}
\setlength{\tabcolsep}{1pt}
\begin{tabular}{>{\hfill\scriptsize}p{1cm}<{}>{\hfill\scriptsize}p{.25cm}<{}>{\scriptsize}p{12cm}<{\hfill}}
Source:& \multicolumn{2}{l}{\scriptsize Estimated with GUK administrative and survey data.}\\
Notes: & 1. & First-difference estimates using rd 2 - rd 4 data. First-differenced ($\Delta x_{t+1}\equiv x_{t+1} - x_{t}$) regressands are regressed on categorical and time-variant covariates. Net saving is taken from administrative data and merged with survey data at Year-Month of survey interviews. Head age and literacy are from baseline data. Intercept terms are omitted in estimating equations. Net saving is saving - withdrawal. All dummy interaction terms are first demeaned and then interacted.\\
& 2. & ${}^{***}$, ${}^{**}$, ${}^{*}$ indicate statistical significance at 1\%, 5\%, 10\%, respetively. Standard errors are clustered at group (village) level.
\end{tabular}
\end{minipage}


\begin{palepinkleftbar}
\begin{finding}
\textsc{\small Table \ref{tab FD saving original HH}} (1) shows net saving increases, (2) shows that initially a larger then a smaller extent in the later rounds. This reduction may reflect the use of saving for repayment. \textsf{traditional} arm has the lowest repayment rates. Ultra poor and moderately poor have similar repayment rates as indicated in \textsc{\small Table \ref{tab FD saving2 original HH}}. \textsc{Table \ref{tab FD saving3 original HH}} (2) shows having a grace period increases the repayment amount while reduces net saving in later rounds. (4) and (5) show cumulative repayment is greater for with grace because each installment is larger. These are all by design that they do not repay in rd 1 so saving increases then they tap in these saving for repayment. 
\end{finding}
\end{palepinkleftbar}


\subsection{Assets}

Assets reportd in rd 1 is too small, indicating possible errors or different way of reporting only in rd 1. So we also examine rd 2 vs. rd 4 differences (\textsf{as3, as4}).


\begin{Schunk}
\begin{Soutput}
Dropped 2804 obs due to NA.
Dropped 2804 obs due to NA.
Dropped 2804 obs due to NA.
Dropped 2804 obs due to NA.
Dropped 2039 obs due to NA.
Dropped 2039 obs due to NA.
Dropped 2039 obs due to NA.
Dropped 2039 obs due to NA.
\end{Soutput}
\end{Schunk}

Main assets are household assets (\textsf{HAssetAmount}) and production assets (\textsf{PAssetAmount}) both with 4973 observations. After first-differencing, they become 3595 observations, with 21, 94, 3480 households observed for 2, 3, 4 times. We also examine rd 2 vs. rd 4 differences, which has 2389 observations. After first-differencing, they become 1161 observations.



\begin{Schunk}
\begin{Sinput}
source(paste0(pathprogram, "AssetCovariateSelection.R"))
\end{Sinput}
\end{Schunk}





\hspace{-1cm}\begin{minipage}[t]{14cm}
\hfil\textsc{\normalsize Table \refstepcounter{table}\thetable: FD estimation of assets\label{tab FD assets original HH}}\\
\setlength{\tabcolsep}{1pt}
\setlength{\baselineskip}{8pt}
\renewcommand{\arraystretch}{.55}
\hfil\begin{tikzpicture}
\node (tbl) {\input{c:/data/GUK/analysis/save/membership1or4/AssetOriginalHHsFDEstimationResults.tex}};
%\input{c:/dropbox/data/ramadan/save/tablecolortemplate.tex}
\end{tikzpicture}\\
\renewcommand{\arraystretch}{.8}
\setlength{\tabcolsep}{1pt}
\begin{tabular}{>{\hfill\scriptsize}p{1cm}<{}>{\hfill\scriptsize}p{.25cm}<{}>{\scriptsize}p{12cm}<{\hfill}}
Source:& \multicolumn{2}{l}{\scriptsize Estimated with GUK administrative and survey data.}\\
Notes: & 1. & First-difference estimates. A first-difference is defined as $\Delta x_{t+k}\equiv x_{t+k} - x_{t}$  for $k=1, 2, \dots$. Saving and repayment misses are taken from administrative data and merged with survey data at Year-Month of survey interviews. Intercept terms are omitted in estimating equations. Sample is continuing members and replacing members of early rejecters and received loans prior to 2015 Janunary. Household assets do not include livestock. Regressions (1)-(3), (5)-(6) use only arm and calendar information. (4) and (7) use previous six month repayment and saving information which is lacking in rd 1, hence starts from rd 2.\\
& 2. & ${}^{***}$, ${}^{**}$, ${}^{*}$ indicate statistical significance at 1\%, 5\%, 10\%, respetively. Standard errors are clustered at group (village) level.
\end{tabular}
\end{minipage}

\hspace{-1cm}\begin{minipage}[t]{14cm}
\hfil\textsc{\normalsize Table \refstepcounter{table}\thetable: FD estimation of assets, moderately poor vs. ultra poor\label{tab FD assets2 original HH}}\\
\setlength{\tabcolsep}{1pt}
\setlength{\baselineskip}{8pt}
\renewcommand{\arraystretch}{.55}
\hfil\begin{tikzpicture}
\node (tbl) {\input{c:/data/GUK/analysis/save/membership1or4/AssetPovertyStatusOriginalHHsFDEstimationResults.tex}};
%\input{c:/dropbox/data/ramadan/save/tablecolortemplate.tex}
\end{tikzpicture}\\
\renewcommand{\arraystretch}{.8}
\setlength{\tabcolsep}{1pt}
\begin{tabular}{>{\hfill\scriptsize}p{1cm}<{}>{\hfill\scriptsize}p{.25cm}<{}>{\scriptsize}p{12cm}<{\hfill}}
Source:& \multicolumn{2}{l}{\scriptsize Estimated with GUK administrative and survey data.}\\
Notes: & 1. & First-difference estimates. A first-difference is defined as $\Delta x_{t+k}\equiv x_{t+k} - x_{t}$  for $k=1, 2, \dots$. Saving and repayment misses are taken from administrative data and merged with survey data at Year-Month of survey interviews. Intercept terms are omitted in estimating equations. Sample is continuing members and replacing members of early rejecters and received loans prior to 2015 Janunary. Household assets do not include livestock. \\
& 2. & ${}^{***}$, ${}^{**}$, ${}^{*}$ indicate statistical significance at 1\%, 5\%, 10\%, respetively. Standard errors are clustered at group (village) level.
\end{tabular}
\end{minipage}

\hspace{-1cm}\begin{minipage}[t]{14cm}
\hfil\textsc{\normalsize Table \refstepcounter{table}\thetable: FD estimation of assets, small vs. large size loans\label{tab FD assets3 original HH}}\\
\setlength{\tabcolsep}{1pt}
\setlength{\baselineskip}{8pt}
\renewcommand{\arraystretch}{.55}
\hfil\begin{tikzpicture}
\node (tbl) {\input{c:/data/GUK/analysis/save/membership1or4/AssetGraceOriginalHHsFDEstimationResults.tex}};
%\input{c:/dropbox/data/ramadan/save/tablecolortemplate.tex}
\end{tikzpicture}\\
\renewcommand{\arraystretch}{.8}
\setlength{\tabcolsep}{1pt}
\begin{tabular}{>{\hfill\scriptsize}p{1cm}<{}>{\hfill\scriptsize}p{.25cm}<{}>{\scriptsize}p{12cm}<{\hfill}}
Source:& \multicolumn{2}{l}{\scriptsize Estimated with GUK administrative and survey data.}\\
Notes: & 1. & First-difference estimates. A first-difference is defined as $\Delta x_{t+k}\equiv x_{t+k} - x_{t}$  for $k=1, 2, \dots$. Saving and repayment misses are taken from administrative data and merged with survey data at Year-Month of survey interviews. Intercept terms are omitted in estimating equations. Sample is continuing members and replacing members of early rejecters and received loans prior to 2015 Janunary. Household assets do not include livestock. \\
& 2. & ${}^{***}$, ${}^{**}$, ${}^{*}$ indicate statistical significance at 1\%, 5\%, 10\%, respetively. Standard errors are clustered at group (village) level.
\end{tabular}
\end{minipage}

\hspace{-1cm}\begin{minipage}[t]{14cm}
\hfil\textsc{\normalsize Table \refstepcounter{table}\thetable: FD estimation of assets, with vs. without a grace period\label{tab FD assets4 original HH}}\\
\setlength{\tabcolsep}{1pt}
\setlength{\baselineskip}{8pt}
\renewcommand{\arraystretch}{.55}
\hfil\begin{tikzpicture}
\node (tbl) {\input{c:/data/GUK/analysis/save/membership1or4/AssetSizeOriginalHHsFDEstimationResults.tex}};
%\input{c:/dropbox/data/ramadan/save/tablecolortemplate.tex}
\end{tikzpicture}\\
\renewcommand{\arraystretch}{.8}
\setlength{\tabcolsep}{1pt}
\begin{tabular}{>{\hfill\scriptsize}p{1cm}<{}>{\hfill\scriptsize}p{.25cm}<{}>{\scriptsize}p{12cm}<{\hfill}}
Source:& \multicolumn{2}{l}{\scriptsize Estimated with GUK administrative and survey data.}\\
Notes: & 1. & First-difference estimates. A first-difference is defined as $\Delta x_{t+k}\equiv x_{t+k} - x_{t}$  for $k=1, 2, \dots$. Saving and repayment misses are taken from administrative data and merged with survey data at Year-Month of survey interviews. Intercept terms are omitted in estimating equations. Sample is continuing members and replacing members of early rejecters and received loans prior to 2015 Janunary. Household assets do not include livestock. \\
& 2. & ${}^{***}$, ${}^{**}$, ${}^{*}$ indicate statistical significance at 1\%, 5\%, 10\%, respetively. Standard errors are clustered at group (village) level.
\end{tabular}
\end{minipage}

\hspace{-1cm}\begin{minipage}[t]{14cm}
\hfil\textsc{\normalsize Table \refstepcounter{table}\thetable: FD estimation of assets, round 2 and 4 comparison\label{tab FD assets rd24 original HH}}\\
\setlength{\tabcolsep}{1pt}
\setlength{\baselineskip}{8pt}
\renewcommand{\arraystretch}{.55}
\hfil\begin{tikzpicture}
\node (tbl) {\input{c:/data/GUK/analysis/save/membership1or4/AssetRd24DiffOriginalHHsFDEstimationResults.tex}};
%\input{c:/dropbox/data/ramadan/save/tablecolortemplate.tex}
\end{tikzpicture}\\
\renewcommand{\arraystretch}{.8}
\setlength{\tabcolsep}{1pt}
\begin{tabular}{>{\hfill\scriptsize}p{1cm}<{}>{\hfill\scriptsize}p{.25cm}<{}>{\scriptsize}p{12cm}<{\hfill}}
Source:& \multicolumn{2}{l}{\scriptsize Estimated with GUK administrative and survey data.}\\
Notes: & 1. & First-difference estimates between round 2 and 4. A first-difference is defined as $\Delta x_{t+k}\equiv x_{t+k} - x_{t}$ for $k=1, 2, \dots$. Saving and repayment misses are taken from administrative data and merged with survey data at Year-Month of survey interviews. Intercept terms are omitted in estimating equations. Sample is continuing members and replacing members of early rejecters and received loans prior to 2015 Janunary. Household assets do not include livestock. Regressions (1)-(3), (5)-(6) use only arm and calendar information. (4) and (7) use previous six month repayment and saving information which is lacking in rd 1, hence starts from rd 2.\\
& 2. & ${}^{***}$, ${}^{**}$, ${}^{*}$ indicate statistical significance at 1\%, 5\%, 10\%, respetively. Standard errors are clustered at group (village) level.
\end{tabular}
\end{minipage}

\hspace{-1cm}\begin{minipage}[t]{14cm}
\hfil\textsc{\normalsize Table \refstepcounter{table}\thetable: FD estimation of assets, round 2 and 4 comparison, grace period\label{tab FD assets rd24 grace original HH}}\\
\setlength{\tabcolsep}{1pt}
\setlength{\baselineskip}{8pt}
\renewcommand{\arraystretch}{.55}
\hfil\begin{tikzpicture}
\node (tbl) {\input{c:/data/GUK/analysis/save/membership1or4/AssetRd24GraceOriginalHHsFDEstimationResults.tex}};
%\input{c:/dropbox/data/ramadan/save/tablecolortemplate.tex}
\end{tikzpicture}\\
\renewcommand{\arraystretch}{.8}
\setlength{\tabcolsep}{1pt}
\begin{tabular}{>{\hfill\scriptsize}p{1cm}<{}>{\hfill\scriptsize}p{.25cm}<{}>{\scriptsize}p{12cm}<{\hfill}}
Source:& \multicolumn{2}{l}{\scriptsize Estimated with GUK administrative and survey data.}\\
Notes: & 1. & First-difference estimates between round 2 and 4. A first-difference is defined as $\Delta x_{t+k}\equiv x_{t+k} - x_{t}$ for $k=1, 2, \dots$. Saving and repayment misses are taken from administrative data and merged with survey data at Year-Month of survey interviews. Intercept terms are omitted in estimating equations. Sample is continuing members and replacing members of early rejecters and received loans prior to 2015 Janunary. Household assets do not include livestock. Regressions (1)-(3), (5)-(6) use only arm and calendar information. (4) and (7) use previous six month repayment and saving information which is lacking in rd 1, hence starts from rd 2.\\
& 2. & ${}^{***}$, ${}^{**}$, ${}^{*}$ indicate statistical significance at 1\%, 5\%, 10\%, respetively. Standard errors are clustered at group (village) level.
\end{tabular}
\end{minipage}


\begin{palepinkleftbar}
\begin{finding}
\textsc{\small Table \ref{tab FD assets original HH}} (1) shows household assets increase after receiving the loans in all arms. Total incremant is largest among the \textsf{large grace} arm as indicated in (2). In (3), increments are positive in rd 2 - 3, suggesting substantial purchase after receiving a loan. Significant decreases in rd 3 - 4 for all arms indicate liquidation of assets for repayment. Productive assets of large size loan arms decrease in rd 3 - 4 in \textsc{\small Table \ref{tab FD assets3 original HH}} (6). These may indicate forced liquidation for repayment, which can entail efficiency losses.
\end{finding}
\end{palepinkleftbar}


\subsection{Livestock}


\begin{Schunk}
\begin{Soutput}
Dropped 2807 obs due to NA.
Dropped 2807 obs due to NA.
Dropped 2041 obs due to NA.
Dropped 2041 obs due to NA.
\end{Soutput}
\end{Schunk}

\begin{Schunk}
\begin{Sinput}
source(paste0(pathprogram, "LivestockCovariateSelection.R"))
\end{Sinput}
\end{Schunk}




\hspace{-1cm}\begin{minipage}[t]{14cm}
\hfil\textsc{\normalsize Table \refstepcounter{table}\thetable: FD estimation of livestock holding values\label{tab FD livestock original HH}}\\
\setlength{\tabcolsep}{1pt}
\setlength{\baselineskip}{8pt}
\renewcommand{\arraystretch}{.55}
\hfil\begin{tikzpicture}
\node (tbl) {\input{c:/data/GUK/analysis/save/membership1or4/LivestockOriginalHHsFDEstimationResults.tex}};
%\input{c:/dropbox/data/ramadan/save/tablecolortemplate.tex}
\end{tikzpicture}\\
\renewcommand{\arraystretch}{.8}
\setlength{\tabcolsep}{1pt}
\begin{tabular}{>{\hfill\scriptsize}p{1cm}<{}>{\hfill\scriptsize}p{.25cm}<{}>{\scriptsize}p{12cm}<{\hfill}}
Source:& \multicolumn{2}{l}{\scriptsize Estimated with GUK administrative and survey data.}\\
Notes: & 1. & First-difference estimates. A first-difference is defined as $\Delta x_{t+1}\equiv x_{t+1} - x_{t}$. Saving and repayment misses are taken from administrative data and merged with survey data at Year-Month of survey interviews. Intercept terms are omitted in estimating equations. Sample is continuing members and replacing members of early rejecters and received loans prior to 2015 Janunary. Regressand is \textsf{TotalImputedValue}, a sum of all livestock holding values evaluated at respective median market prices in the same year. \\
& 2. & ${}^{***}$, ${}^{**}$, ${}^{*}$ indicate statistical significance at 1\%, 5\%, 10\%, respetively. Standard errors are clustered at group (village) level.
\end{tabular}
\end{minipage}

\hspace{-1cm}\begin{minipage}[t]{14cm}
\hfil\textsc{\normalsize Table \refstepcounter{table}\thetable: FD estimation of livestock holding values, with vs. without a grace period\label{tab FD livestock2 original HH}}\\
\setlength{\tabcolsep}{1pt}
\setlength{\baselineskip}{8pt}
\renewcommand{\arraystretch}{.55}
\hfil\begin{tikzpicture}
\node (tbl) {\input{c:/data/GUK/analysis/save/membership1or4/LivestockPovertyStatusOriginalHHsFDEstimationResults.tex}};
%\input{c:/dropbox/data/ramadan/save/tablecolortemplate.tex}
\end{tikzpicture}\\
\renewcommand{\arraystretch}{.8}
\setlength{\tabcolsep}{1pt}
\begin{tabular}{>{\hfill\scriptsize}p{1cm}<{}>{\hfill\scriptsize}p{.25cm}<{}>{\scriptsize}p{12cm}<{\hfill}}
Source:& \multicolumn{2}{l}{\scriptsize Estimated with GUK administrative and survey data.}\\
Notes: & 1. & First-difference estimates. A first-difference is defined as $\Delta x_{t+1}\equiv x_{t+1} - x_{t}$. Saving and repayment misses are taken from administrative data and merged with survey data at Year-Month of survey interviews. Intercept terms are omitted in estimating equations. Sample is continuing members and replacing members of early rejecters and received loans prior to 2015 Janunary. Regressand is \textsf{TotalImputedValue}, a sum of all livestock holding values evaluated at respective median market prices in the same year. \\
& 2. & ${}^{***}$, ${}^{**}$, ${}^{*}$ indicate statistical significance at 1\%, 5\%, 10\%, respetively. Standard errors are clustered at group (village) level.
\end{tabular}
\end{minipage}

\hspace{-1cm}\begin{minipage}[t]{14cm}
\hfil\textsc{\normalsize Table \refstepcounter{table}\thetable: FD estimation of livestock holding values, rd 1 vs. rd 4 comparison\label{tab FD livestock3 original HH}}\\
\setlength{\tabcolsep}{1pt}
\setlength{\baselineskip}{8pt}
\renewcommand{\arraystretch}{.55}
\hfil\begin{tikzpicture}
\node (tbl) {\input{c:/data/GUK/analysis/save/membership1or4/LivestockRd14DiffOriginalHHsFDEstimationResults.tex}};
%\input{c:/dropbox/data/ramadan/save/tablecolortemplate.tex}
\end{tikzpicture}\\
\renewcommand{\arraystretch}{.8}
\setlength{\tabcolsep}{1pt}
\begin{tabular}{>{\hfill\scriptsize}p{1cm}<{}>{\hfill\scriptsize}p{.25cm}<{}>{\scriptsize}p{12cm}<{\hfill}}
Source:& \multicolumn{2}{l}{\scriptsize Estimated with GUK administrative and survey data.}\\
Notes: & 1. & First-difference estimates. A first-difference is defined as $\Delta x_{t+1}\equiv x_{t+1} - x_{t}$. Saving and repayment misses are taken from administrative data and merged with survey data at Year-Month of survey interviews. Intercept terms are omitted in estimating equations. Sample is continuing members and replacing members of early rejecters and received loans prior to 2015 Janunary. Regressand is \textsf{TotalImputedValue}, a sum of all livestock holding values evaluated at respective median market prices in the same year. \\
& 2. & ${}^{***}$, ${}^{**}$, ${}^{*}$ indicate statistical significance at 1\%, 5\%, 10\%, respetively. Standard errors are clustered at group (village) level.
\end{tabular}
\end{minipage}

Check quickly if the estimated results make sense.
\begin{Schunk}
\begin{Soutput}
       hhid         Arm Year LivestockCode number_owned mrkt_value
 1: 7020319       large 2012        cow/ox            2          0
 2: 7020319       large 2014        cow/ox            5      18000
 3: 7020319       large 2015        cow/ox            6      19000
 4: 7020319       large 2017        cow/ox            5      25000
 5: 7020614 large grace 2012                          0          0
 6: 7020614 large grace 2014        cow/ox            2      16000
 7: 7020614 large grace 2015        cow/ox            5      16000
 8: 7020614 large grace 2017        cow/ox            6      24000
 9: 7021003 large grace 2012        cow/ox            1          0
10: 7021003 large grace 2014        cow/ox            8      18000
11: 7021003 large grace 2015        cow/ox            4      20000
12: 7021003 large grace 2017        cow/ox            4      23000
13: 7021012 large grace 2012  Chicken/duck            4          0
14: 7021012 large grace 2014        cow/ox            2      24000
15: 7021012 large grace 2015        cow/ox            3      19000
16: 7021012 large grace 2017        cow/ox            8      25000
17: 7021216         cow 2012        cow/ox            6          0
18: 7021216         cow 2014        cow/ox            5      20000
19: 7021216         cow 2015        cow/ox            3      18000
20: 7021216         cow 2017        cow/ox            3      30000
21: 7031706       large 2012        cow/ox            1          0
22: 7031706       large 2014        cow/ox            7      12000
23: 7031706       large 2015        cow/ox            3      15000
24: 7031706       large 2017        cow/ox            3      38000
25: 7031715       large 2012        cow/ox            2          0
26: 7031715       large 2014        cow/ox            9      15000
27: 7031715       large 2015        cow/ox            8      16000
28: 7031715       large 2017        cow/ox            1      30000
29: 7031716       large 2012        cow/ox            1          0
30: 7031716       large 2014        cow/ox            6      16000
31: 7031716       large 2015        cow/ox            5      17000
32: 7031716       large 2017        cow/ox            2      42000
33: 7031905       large 2012        cow/ox            4          0
34: 7031905       large 2014        cow/ox            5      16000
35: 7031905       large 2015        cow/ox            7      20000
36: 7031905       large 2017        cow/ox            7      20000
37: 7042017       large 2012                          0          0
38: 7042017       large 2014        cow/ox            3      18000
39: 7042017       large 2015        cow/ox            6      20000
40: 7042017       large 2017        cow/ox            4      20000
41: 7054005 large grace 2012                          0          0
42: 7054005 large grace 2014        cow/ox            4      18000
43: 7054005 large grace 2015        cow/ox            2      16000
44: 7054005 large grace 2017        cow/ox           10      15000
45: 7054012 large grace 2012        cow/ox            4          0
46: 7054012 large grace 2014        cow/ox           15      20000
47: 7054012 large grace 2015        cow/ox           12      16000
48: 7054012 large grace 2017        cow/ox           10      22000
49: 7085916         cow 2012                          0          0
50: 7085916         cow 2014                         NA         NA
51: 7085916         cow 2015        cow/ox            2      18000
52: 7085916         cow 2017        cow/ox            6      20000
53: 7096202       large 2012        cow/ox            4          0
54: 7096202       large 2014        cow/ox            2      10000
55: 7096202       large 2015        cow/ox            8      14000
56: 7096202       large 2017        cow/ox            9      20000
57: 7096207       large 2012        cow/ox            1          0
58: 7096207       large 2014        cow/ox            6      12000
59: 7096207       large 2015        cow/ox            7      22000
60: 7096207       large 2017        cow/ox            6      16000
61: 7096218       large 2012        cow/ox            1          0
62: 7096218       large 2014        cow/ox            9      16000
63: 7096218       large 2015        cow/ox            7      16000
64: 7096218       large 2017        cow/ox            6      20000
65: 7106408         cow 2012        cow/ox            2          0
66: 7106408         cow 2014        cow/ox            3      15000
67: 7106408         cow 2016        cow/ox            7      14500
68: 7137207 traditional 2012                          0          0
69: 7137207 traditional 2014        cow/ox            1      16000
70: 7137207 traditional 2015        cow/ox            6      14000
71: 7137207 traditional 2017        cow/ox            6      16000
72: 8169519 large grace 2012  Chicken/duck            4          0
73: 8169519 large grace 2014        cow/ox            1      20000
74: 8169519 large grace 2015        cow/ox            6      25000
75: 8169519 large grace 2017        cow/ox            3      40000
76: 8169619       large 2012  Chicken/duck            4          0
77: 8169619       large 2014        cow/ox            3      16000
78: 8169619       large 2016        cow/ox            6      18000
79: 8169619       large 2017        cow/ox            6      38000
       hhid         Arm Year LivestockCode number_owned mrkt_value
    TotalImputedValue
 1:             40000
 2:            100000
 3:            120000
 4:            100000
 5:                 0
 6:             40000
 7:            100000
 8:            120000
 9:             20000
10:            160000
11:             80000
12:             80000
13:                 0
14:             40000
15:             60000
16:            160000
17:            120000
18:            100000
19:             60000
20:             60000
21:             20000
22:            140000
23:             60000
24:             60000
25:             40000
26:            180000
27:            160000
28:             20000
29:             20000
30:            120000
31:            100000
32:             40000
33:             80000
34:            100000
35:            140000
36:            140000
37:                 0
38:             60000
39:            120000
40:             80000
41:                 0
42:             80000
43:             40000
44:            200000
45:             80000
46:            300000
47:            240000
48:            200000
49:                 0
50:                 0
51:             40000
52:            120000
53:             80000
54:             40000
55:            160000
56:            180000
57:             20000
58:            120000
59:            140000
60:            120000
61:             20000
62:            180000
63:            140000
64:            120000
65:             40000
66:             60000
67:            140000
68:                 0
69:             20000
70:            120000
71:            120000
72:                 0
73:             20000
74:            120000
75:             60000
76:                 0
77:             60000
78:            120000
79:            120000
    TotalImputedValue
\end{Soutput}
\end{Schunk}
\begin{Schunk}
\begin{figure}

{\centering \includegraphics[width=\maxwidth]{figure/ImpactEstimationMembership1or4/Total_imputed_value_histogram_original_HHs-1} 

}

\caption{Total imputed value of livestock holding\\ {\footnotesize Livestock holding values are computed by using respective median prices of each year.\setlength{\baselineskip}{8pt}}}\label{Figure Total imputed value histogram original HHs}
\end{figure}
\end{Schunk}
\begin{Schunk}
\begin{figure}

{\centering \includegraphics[width=\maxwidth]{figure/ImpactEstimationMembership1or4/Histogram_of_livestock_holding_classes_original_HHs-1} 

}

\caption{Histogram of livestock holding classes\\ {\footnotesize Livestock holding values are computed by using respective median prices of each year.\setlength{\baselineskip}{8pt}}}\label{Figure Histogram of livestock holding classes original HHs}
\end{figure}
\end{Schunk}
\begin{Schunk}
\begin{figure}

{\centering \includegraphics[width=\maxwidth]{figure/ImpactEstimationMembership1or4/Histogram_of_livestock_holding_classes_by_year_original_HHs-1} 

}

\caption{Histogram of livestock holding classes by year\\ {\footnotesize Livestock holding values are computed by using respective median prices of each year.\setlength{\baselineskip}{8pt}}}\label{Figure Histogram of livestock holding classes by year original HHs}
\end{figure}
\end{Schunk}
\begin{itemize}
\vspace{1.0ex}\setlength{\itemsep}{1.0ex}\setlength{\baselineskip}{12pt}
\item	Why does \textsf{cow} report below 1000 holding in rds 2-4?
\end{itemize}
\begin{Schunk}
\begin{Soutput}
            Arm survey MeanImputedVal MeanNumCows   N
 1: traditional      1        5065.33    0.233668 398
 2: traditional      2       15854.00    0.817844 280
 3: traditional      3       20179.62    1.022059 277
 4: traditional      4       21233.75    1.050000 240
 5:       large      1        6092.42    0.275689 399
 6:       large      3       31056.41    1.625000 386
 7:       large      2       24992.86    1.278820 383
 8:       large      4       32686.07    1.630890 382
 9: large grace      1        7392.54    0.333333 399
10: large grace      2       21510.32    1.150943 341
11: large grace      3       27565.65    1.422619 347
12: large grace      4       30276.97    1.528024 343
13:         cow      1        4997.68    0.218045 399
14:         cow      2       20550.29    1.078035 364
15:         cow      3       25399.62    1.300562 365
16:         cow      4       28700.23    1.436950 342
\end{Soutput}
\end{Schunk}
\begin{Schunk}
\begin{figure}

{\centering \includegraphics[width=\maxwidth]{figure/ImpactEstimationMembership1or4/Number_of_cows_by_year_original_HHs-1} 

}

\caption{Number of cows/oxen by year\\ {\footnotesize Means are mean holding among the owners. Totals are total number of cows/oxen owned. Mean and total number of cows/oxen may diverge because the number of owners differ across round.\setlength{\baselineskip}{8pt}}}\label{Figure Number of cows by year original HHs}
\end{figure}
\end{Schunk}
\begin{palepinkleftbar}
\begin{finding}
\textsc{\small Figure \ref{Figure Total imputed value histogram}} shows general increase in upper holding classes round 3 and further upper holding classes in round 4. \textsc{\small Figure \ref{Figure Number of cows by year}} shows livestock type is not entered (yet collected) in rd3. At this moment, one needs to omit rd 3. All estimation results by far are subject to this omission.
\end{finding}
\end{palepinkleftbar}


\subsection{Assets+Livestock}




\begin{Schunk}
\begin{Soutput}
                 creditstatus
BorrowerStatus     Yes   No
  borrower        1192  157
  pure saver         0   26
  quit membership    0  220
\end{Soutput}
\begin{Soutput}
              creditstatus
Mstatus         Yes   No
  gErosion        0   80
  gRejection      0  140
  iRejection      1  157
  iReplacement    0    0
  newGroup        0    0
  oldMember    1191   26
\end{Soutput}
\begin{Soutput}
Dropped 196 obs due to T<2.
Dropped 1402 obs due to NA.
Dropped 196 obs due to T<2.
Dropped 1402 obs due to NA.
Dropped 196 obs due to T<2.
Dropped 1402 obs due to NA.
Dropped 196 obs due to T<2.
Dropped 1402 obs due to NA.
Dropped 130 obs due to T<2.
Dropped 1274 obs due to NA.
Dropped 130 obs due to T<2.
Dropped 1274 obs due to NA.
Dropped 130 obs due to T<2.
Dropped 1274 obs due to NA.
Dropped 130 obs due to T<2.
Dropped 1274 obs due to NA.
\end{Soutput}
\end{Schunk}

\begin{Schunk}
\begin{Soutput}
   Arm
tee traditional large large grace cow
  1         398   399         379 398
  2         283   390         373 379
  3         276   384         348 365
  4         238   377         330 328
\end{Soutput}
\begin{Soutput}
   Arm
tee traditional large large grace cow
  1          66    78          81  63
  2         151   254         258 283
  3         189   348         323 324
  4         156   328         291 287
\end{Soutput}
\begin{Soutput}
Dropped 196 obs due to T<2.
Dropped 1402 obs due to NA.
Dropped 196 obs due to T<2.
Dropped 1402 obs due to NA.
Dropped 154 obs due to T<2.
Dropped 1272 obs due to NA.
Dropped 154 obs due to T<2.
Dropped 1272 obs due to NA.
\end{Soutput}
\end{Schunk}

\begin{Schunk}
\begin{Soutput}
Warning in `[.data.table`(AL2R, , `:=`(grepout("Time", colnames(AL2)), NULL)): length(LHS)==0; no columns to delete or assign RHS to.
\end{Soutput}
\end{Schunk}



\begin{Schunk}
\begin{Sinput}
source(paste0(pathprogram, "AssetLivestockCovariateSelection.R"))
\end{Sinput}
\end{Schunk}

\begin{Schunk}
\begin{figure}

{\centering \includegraphics[width=\maxwidth]{figure/ImpactEstimationMembership1or4/Total_values_original_HHs-1} 

}

\caption{Total asset values\\ {\footnotesize Sum of assets and livestock holding values. Original 1600 HHs.\setlength{\baselineskip}{8pt}}}\label{Figure Total values original HHs}
\end{figure}
\end{Schunk}


\hspace{-1cm}\begin{minipage}[t]{14cm}
\hfil\textsc{\normalsize Table \refstepcounter{table}\thetable: FD estimation of total assets, original HHs\label{tab FD total assets original HHs}}\\
\setlength{\tabcolsep}{1pt}
\setlength{\baselineskip}{8pt}
\renewcommand{\arraystretch}{.55}
\hfil\begin{tikzpicture}
\node (tbl) {\input{c:/data/GUK/analysis/save/membership1or4/AssetLivestockOriginalHHsFDEstimationResults.tex}};
%\input{c:/dropbox/data/ramadan/save/tablecolortemplate.tex}
\end{tikzpicture}\\
\renewcommand{\arraystretch}{.8}
\setlength{\tabcolsep}{1pt}
\begin{tabular}{>{\hfill\scriptsize}p{1cm}<{}>{\hfill\scriptsize}p{.25cm}<{}>{\scriptsize}p{12cm}<{\hfill}}
Source:& \multicolumn{2}{l}{\scriptsize Estimated with GUK administrative and survey data.}\\
Notes: & 1. & First-difference estimates. A first-difference is defined as $\Delta x_{t+k}\equiv x_{t+k} - x_{t}$  for $k=1, 2, \dots$. Saving and repayment misses are taken from administrative data and merged with survey data at Year-Month of survey interviews. Intercept terms are omitted in estimating equations. Sample is continuing members and replacing members of early rejecters and received loans prior to 2015 Janunary. Household assets do not include livestock. Regressions (1)-(3), (5)-(6) use only arm and calendar information. (4) and (7) use previous six month repayment and saving information which is lacking in rd 1, hence starts from rd 2.\\
& 2. & ${}^{***}$, ${}^{**}$, ${}^{*}$ indicate statistical significance at 1\%, 5\%, 10\%, respetively. Standard errors are clustered at group (village) level.
\end{tabular}
\end{minipage}

\hspace{-1cm}\begin{minipage}[t]{14cm}
\hfil\textsc{\normalsize Table \refstepcounter{table}\thetable: FD estimation of total assets, moderately poor vs. ultra poor, original HHs \label{tab FD total assets2 original HH}}\\
\setlength{\tabcolsep}{1pt}
\setlength{\baselineskip}{8pt}
\renewcommand{\arraystretch}{.55}
\hfil\begin{tikzpicture}
\node (tbl) {\input{c:/data/GUK/analysis/save/membership1or4/AssetLivestockPovertyStatusOriginalHHsFDEstimationResults.tex}};
%\input{c:/dropbox/data/ramadan/save/tablecolortemplate.tex}
\end{tikzpicture}\\
\renewcommand{\arraystretch}{.8}
\setlength{\tabcolsep}{1pt}
\begin{tabular}{>{\hfill\scriptsize}p{1cm}<{}>{\hfill\scriptsize}p{.25cm}<{}>{\scriptsize}p{12cm}<{\hfill}}
Source:& \multicolumn{2}{l}{\scriptsize Estimated with GUK administrative and survey data.}\\
Notes: & 1. & First-difference estimates. A first-difference is defined as $\Delta x_{t+k}\equiv x_{t+k} - x_{t}$  for $k=1, 2, \dots$. Saving and repayment misses are taken from administrative data and merged with survey data at Year-Month of survey interviews. Intercept terms are omitted in estimating equations. Sample is continuing members and replacing members of early rejecters and received loans prior to 2015 Janunary. Household assets do not include livestock. \\
& 2. & ${}^{***}$, ${}^{**}$, ${}^{*}$ indicate statistical significance at 1\%, 5\%, 10\%, respetively. Standard errors are clustered at group (village) level.
\end{tabular}
\end{minipage}

\hspace{-1cm}\begin{minipage}[t]{14cm}
\hfil\textsc{\normalsize Table \refstepcounter{table}\thetable: FD estimation of total assets, small vs. large size loans, original HHs \label{tab FD total assets3 original HHs}}\\
\setlength{\tabcolsep}{1pt}
\setlength{\baselineskip}{8pt}
\renewcommand{\arraystretch}{.55}
\hfil\begin{tikzpicture}
\node (tbl) {\input{c:/data/GUK/analysis/save/membership1or4/AssetLivestockSizeOriginalHHsFDEstimationResults.tex}};
%\input{c:/dropbox/data/ramadan/save/tablecolortemplate.tex}
\end{tikzpicture}\\
\renewcommand{\arraystretch}{.8}
\setlength{\tabcolsep}{1pt}
\begin{tabular}{>{\hfill\scriptsize}p{1cm}<{}>{\hfill\scriptsize}p{.25cm}<{}>{\scriptsize}p{12cm}<{\hfill}}
Source:& \multicolumn{2}{l}{\scriptsize Estimated with GUK administrative and survey data.}\\
Notes: & 1. & First-difference estimates. A first-difference is defined as $\Delta x_{t+k}\equiv x_{t+k} - x_{t}$  for $k=1, 2, \dots$. Saving and repayment misses are taken from administrative data and merged with survey data at Year-Month of survey interviews. Intercept terms are omitted in estimating equations. Sample is continuing members and replacing members of early rejecters and received loans prior to 2015 Janunary. Household assets do not include livestock. \\
& 2. & ${}^{***}$, ${}^{**}$, ${}^{*}$ indicate statistical significance at 1\%, 5\%, 10\%, respetively. Standard errors are clustered at group (village) level.
\end{tabular}
\end{minipage}

\hspace{-1cm}\begin{minipage}[t]{14cm}
\hfil\textsc{\normalsize Table \refstepcounter{table}\thetable: FD estimation of total assets, with vs. without a grace period, original HHs \label{tab FD total assets4 original HHs}}\\
\setlength{\tabcolsep}{1pt}
\setlength{\baselineskip}{8pt}
\renewcommand{\arraystretch}{.55}
\hfil\begin{tikzpicture}
\node (tbl) {\input{c:/data/GUK/analysis/save/membership1or4/AssetLivestockGraceOriginalHHsFDEstimationResults.tex}};
%\input{c:/dropbox/data/ramadan/save/tablecolortemplate.tex}
\end{tikzpicture}\\
\renewcommand{\arraystretch}{.8}
\setlength{\tabcolsep}{1pt}
\begin{tabular}{>{\hfill\scriptsize}p{1cm}<{}>{\hfill\scriptsize}p{.25cm}<{}>{\scriptsize}p{12cm}<{\hfill}}
Source:& \multicolumn{2}{l}{\scriptsize Estimated with GUK administrative and survey data.}\\
Notes: & 1. & First-difference estimates. A first-difference is defined as $\Delta x_{t+k}\equiv x_{t+k} - x_{t}$  for $k=1, 2, \dots$. Saving and repayment misses are taken from administrative data and merged with survey data at Year-Month of survey interviews. Intercept terms are omitted in estimating equations. Sample is continuing members and replacing members of early rejecters and received loans prior to 2015 Janunary. Household assets do not include livestock. \\
& 2. & ${}^{***}$, ${}^{**}$, ${}^{*}$ indicate statistical significance at 1\%, 5\%, 10\%, respetively. Standard errors are clustered at group (village) level.
\end{tabular}
\end{minipage}

\hspace{-1cm}\begin{minipage}[t]{14cm}
\hfil\textsc{\normalsize Table \refstepcounter{table}\thetable: FD estimation of total assets, round 2 and 4 comparison, original HHs \label{tab FD total assets rd24, original HHs }}\\
\setlength{\tabcolsep}{1pt}
\setlength{\baselineskip}{8pt}
\renewcommand{\arraystretch}{.55}
\hfil\begin{tikzpicture}
\node (tbl) {\input{c:/data/GUK/analysis/save/membership1or4/AssetLivestockRd24DiffOriginalHHsFDEstimationResults.tex}};
%\input{c:/dropbox/data/ramadan/save/tablecolortemplate.tex}
\end{tikzpicture}\\
\renewcommand{\arraystretch}{.8}
\setlength{\tabcolsep}{1pt}
\begin{tabular}{>{\hfill\scriptsize}p{1cm}<{}>{\hfill\scriptsize}p{.25cm}<{}>{\scriptsize}p{12cm}<{\hfill}}
Source:& \multicolumn{2}{l}{\scriptsize Estimated with GUK administrative and survey data.}\\
Notes: & 1. & First-difference estimates between round 2 and 4. A first-difference is defined as $\Delta x_{t+k}\equiv x_{t+k} - x_{t}$ for $k=1, 2, \dots$. Saving and repayment misses are taken from administrative data and merged with survey data at Year-Month of survey interviews. Intercept terms are omitted in estimating equations. Sample is continuing members and replacing members of early rejecters and received loans prior to 2015 Janunary. Household assets do not include livestock. Regressions (1)-(3), (5)-(6) use only arm and calendar information. (4) and (7) use previous six month repayment and saving information which is lacking in rd 1, hence starts from rd 2.\\
& 2. & ${}^{***}$, ${}^{**}$, ${}^{*}$ indicate statistical significance at 1\%, 5\%, 10\%, respetively. Standard errors are clustered at group (village) level.
\end{tabular}
\end{minipage}

\hspace{-1cm}\begin{minipage}[t]{14cm}
\hfil\textsc{\normalsize Table \refstepcounter{table}\thetable: FD estimation of total assets, round 2 and 4 comparison, grace period, original HHs \label{tab FD total assets rd24 grace original HHs}}\\
\setlength{\tabcolsep}{1pt}
\setlength{\baselineskip}{8pt}
\renewcommand{\arraystretch}{.55}
\hfil\begin{tikzpicture}
\node (tbl) {\input{c:/data/GUK/analysis/save/membership1or4/AssetLivestockRd24DiffGraceOriginalHHsFDEstimationResults.tex}};
%\input{c:/dropbox/data/ramadan/save/tablecolortemplate.tex}
\end{tikzpicture}\\
\renewcommand{\arraystretch}{.8}
\setlength{\tabcolsep}{1pt}
\begin{tabular}{>{\hfill\scriptsize}p{1cm}<{}>{\hfill\scriptsize}p{.25cm}<{}>{\scriptsize}p{12cm}<{\hfill}}
Source:& \multicolumn{2}{l}{\scriptsize Estimated with GUK administrative and survey data.}\\
Notes: & 1. & First-difference estimates between round 2 and 4. A first-difference is defined as $\Delta x_{t+k}\equiv x_{t+k} - x_{t}$ for $k=1, 2, \dots$. Saving and repayment misses are taken from administrative data and merged with survey data at Year-Month of survey interviews. Intercept terms are omitted in estimating equations. Sample is continuing members and replacing members of early rejecters and received loans prior to 2015 Janunary. Household assets do not include livestock. Regressions (1)-(3), (5)-(6) use only arm and calendar information. (4) and (7) use previous six month repayment and saving information which is lacking in rd 1, hence starts from rd 2.\\
& 2. & ${}^{***}$, ${}^{**}$, ${}^{*}$ indicate statistical significance at 1\%, 5\%, 10\%, respetively. Standard errors are clustered at group (village) level.
\end{tabular}
\end{minipage}

\hspace{-1cm}\begin{minipage}[t]{14cm}
\hfil\textsc{\normalsize Table \refstepcounter{table}\thetable: FD estimation of total assets, round 2 and 4 comparison, ultra poor vs. moderately poor, original HHs \label{tab FD total assets rd24 poor original HHs}}\\
\setlength{\tabcolsep}{1pt}
\setlength{\baselineskip}{8pt}
\renewcommand{\arraystretch}{.55}
\hfil\begin{tikzpicture}
\node (tbl) {\input{c:/data/GUK/analysis/save/membership1or4/AssetLivestockRd24DiffPovertyStatusOriginalHHsFDEstimationResults.tex}};
%\input{c:/dropbox/data/ramadan/save/tablecolortemplate.tex}
\end{tikzpicture}\\
\renewcommand{\arraystretch}{.8}
\setlength{\tabcolsep}{1pt}
\begin{tabular}{>{\hfill\scriptsize}p{1cm}<{}>{\hfill\scriptsize}p{.25cm}<{}>{\scriptsize}p{12cm}<{\hfill}}
Source:& \multicolumn{2}{l}{\scriptsize Estimated with GUK administrative and survey data.}\\
Notes: & 1. & First-difference estimates between round 2 and 4. A first-difference is defined as $\Delta x_{t+k}\equiv x_{t+k} - x_{t}$ for $k=1, 2, \dots$. Saving and repayment misses are taken from administrative data and merged with survey data at Year-Month of survey interviews. Intercept terms are omitted in estimating equations. Sample is continuing members and replacing members of early rejecters and received loans prior to 2015 Janunary. Household assets do not include livestock. Regressions (1)-(3), (5)-(6) use only arm and calendar information. (4) and (7) use previous six month repayment and saving information which is lacking in rd 1, hence starts from rd 2.\\
& 2. & ${}^{***}$, ${}^{**}$, ${}^{*}$ indicate statistical significance at 1\%, 5\%, 10\%, respetively. Standard errors are clustered at group (village) level.
\end{tabular}
\end{minipage}

\clearpage
\subsection{Incomes}



\begin{Schunk}
\begin{Soutput}
Warning in `[.data.table`(lab, , `:=`(grepout("RM", colnames(lab)), NULL)): length(LHS)==0; no columns to delete or assign RHS to.
\end{Soutput}
\begin{Soutput}
Warning in `[.data.table`(far, , `:=`(grepout("RM", colnames(far)), NULL)): length(LHS)==0; no columns to delete or assign RHS to.
\end{Soutput}
\begin{Soutput}
Dropped 4546 obs due to T<2.
Dropped 1133 obs due to NA.
Dropped 4546 obs due to T<2.
Dropped 1133 obs due to NA.
Dropped 6242 obs due to NA.
Dropped 6242 obs due to NA.
\end{Soutput}
\end{Schunk}

Income sources are mainly labour incomes (\textsf{lab}) and farm revenues (\textsf{far}) with 6165 and 6400 observations, respectively. After first-differencing, they become 486 and 150 observations, with 486 households observed for 487 times. 


Obs for survey labour income.
\begin{Schunk}
\begin{Soutput}

  1   2   3   4 
  1 311 128  46 
\end{Soutput}
\end{Schunk}
Obs for survey labour income and admin repayment data.
\begin{Schunk}
\begin{Soutput}

  3   4 
128  46 
\end{Soutput}
\begin{Soutput}

 3  4 
79 71 
\end{Soutput}
\end{Schunk}
Obs for survey farm revenue.
\begin{Schunk}
\begin{Soutput}

 3  4 
79 71 
\end{Soutput}
\end{Schunk}
Obs for survey farm revenue and admin repayment data.
\begin{Schunk}
\begin{Soutput}

 3  4 
79 71 
\end{Soutput}
\end{Schunk}

\begin{Schunk}
\begin{Sinput}
source(paste0(pathprogram, "IncomeCovariateSelection.R"))
\end{Sinput}
\end{Schunk}





\hspace{-1cm}\begin{minipage}[t]{14cm}
\hfil\textsc{\normalsize Table \refstepcounter{table}\thetable: FD estimation of incomes\label{tab FD incomes original HH}}\\
\setlength{\tabcolsep}{1pt}
\setlength{\baselineskip}{8pt}
\renewcommand{\arraystretch}{.55}
\hfil\begin{tikzpicture}
\node (tbl) {\input{c:/data/GUK/analysis/save/membership1or4/IncomesOriginalHHsFDEstimationResults.tex}};
%\input{c:/dropbox/data/ramadan/save/tablecolortemplate.tex}
\end{tikzpicture}\\
\renewcommand{\arraystretch}{.8}
\setlength{\tabcolsep}{1pt}
\begin{tabular}{>{\hfill\scriptsize}p{1cm}<{}>{\hfill\scriptsize}p{.25cm}<{}>{\scriptsize}p{12cm}<{\hfill}}
Source:& \multicolumn{2}{l}{\scriptsize Estimated with GUK administrative and survey data.}\\
Notes: & 1. & First-difference estimates. A first-difference is defined as $\Delta x_{t+1}\equiv x_{t+1} - x_{t}$. Saving and repayment misses are taken from administrative data and merged with survey data at Year-Month of survey interviews. Intercept terms are omitted in estimating equations. Sample is continuing members and replacing members of early rejecters and received loans prior to 2015 Janunary. Labour income is in 1000 Tk unit andis sum of all earned labour incomes. Farm revenue is total of agricultural produce sales. \\
& 2. & ${}^{***}$, ${}^{**}$, ${}^{*}$ indicate statistical significance at 1\%, 5\%, 10\%, respetively. Standard errors are clustered at group (village) level.
\end{tabular}
\end{minipage}

\hspace{-1cm}\begin{minipage}[t]{14cm}
\hfil\textsc{\normalsize Table \refstepcounter{table}\thetable: FD estimation of incomes, moderately poor vs. ultra poor\label{tab FD incomes2 original HH}}\\
\setlength{\tabcolsep}{1pt}
\setlength{\baselineskip}{8pt}
\renewcommand{\arraystretch}{.55}
\hfil\begin{tikzpicture}
\node (tbl) {\input{c:/data/GUK/analysis/save/membership1or4/IncomesPovertyStatusOriginalHHsFDEstimationResults.tex}};
%\input{c:/dropbox/data/ramadan/save/tablecolortemplate.tex}
\end{tikzpicture}\\
\renewcommand{\arraystretch}{.8}
\setlength{\tabcolsep}{1pt}
\begin{tabular}{>{\hfill\scriptsize}p{1cm}<{}>{\hfill\scriptsize}p{.25cm}<{}>{\scriptsize}p{12cm}<{\hfill}}
Source:& \multicolumn{2}{l}{\scriptsize Estimated with GUK administrative and survey data.}\\
Notes: & 1. & First-difference estimates. A first-difference is defined as $\Delta x_{t+1}\equiv x_{t+1} - x_{t}$. Saving and repayment misses are taken from administrative data and merged with survey data at Year-Month of survey interviews. Intercept terms are omitted in estimating equations. Sample is continuing members and replacing members of early rejecters and received loans prior to 2015 Janunary. Labour income is in 1000 Tk unit andis sum of all earned labour incomes. Farm revenue is total of agricultural produce sales. \\
& 2. & ${}^{***}$, ${}^{**}$, ${}^{*}$ indicate statistical significance at 1\%, 5\%, 10\%, respetively. Standard errors are clustered at group (village) level.
\end{tabular}
\end{minipage}

\hspace{-1cm}\begin{minipage}[t]{14cm}
\hfil\textsc{\normalsize Table \refstepcounter{table}\thetable: FD estimation of incomes, loan size\label{tab FD incomes3 original HH}}\\
\setlength{\tabcolsep}{1pt}
\setlength{\baselineskip}{8pt}
\renewcommand{\arraystretch}{.55}
\hfil\begin{tikzpicture}
\node (tbl) {\input{c:/data/GUK/analysis/save/membership1or4/IncomesSizeOriginalHHsFDEstimationResults.tex}};
%\input{c:/dropbox/data/ramadan/save/tablecolortemplate.tex}
\end{tikzpicture}\\
\renewcommand{\arraystretch}{.8}
\setlength{\tabcolsep}{1pt}
\begin{tabular}{>{\hfill\scriptsize}p{1cm}<{}>{\hfill\scriptsize}p{.25cm}<{}>{\scriptsize}p{12cm}<{\hfill}}
Source:& \multicolumn{2}{l}{\scriptsize Estimated with GUK administrative and survey data.}\\
Notes: & 1. & First-difference estimates. A first-difference is defined as $\Delta x_{t+1}\equiv x_{t+1} - x_{t}$. Saving and repayment misses are taken from administrative data and merged with survey data at Year-Month of survey interviews. Intercept terms are omitted in estimating equations. Sample is continuing members and replacing members of early rejecters and received loans prior to 2015 Janunary. Labour income is in 1000 Tk unit andis sum of all earned labour incomes. Farm revenue is total of agricultural produce sales. \\
& 2. & ${}^{***}$, ${}^{**}$, ${}^{*}$ indicate statistical significance at 1\%, 5\%, 10\%, respetively. Standard errors are clustered at group (village) level.
\end{tabular}
\end{minipage}

\hspace{-1cm}\begin{minipage}[t]{14cm}
\hfil\textsc{\normalsize Table \refstepcounter{table}\thetable: FD estimation of incomes, with vs. without a grace period\label{tab FD incomes4 original HH}}\\
\setlength{\tabcolsep}{1pt}
\setlength{\baselineskip}{8pt}
\renewcommand{\arraystretch}{.55}
\hfil\begin{tikzpicture}
\node (tbl) {\input{c:/data/GUK/analysis/save/membership1or4/IncomesGraceOriginalHHsFDEstimationResults.tex}};
%\input{c:/dropbox/data/ramadan/save/tablecolortemplate.tex}
\end{tikzpicture}\\
\renewcommand{\arraystretch}{.8}
\setlength{\tabcolsep}{1pt}
\begin{tabular}{>{\hfill\scriptsize}p{1cm}<{}>{\hfill\scriptsize}p{.25cm}<{}>{\scriptsize}p{12cm}<{\hfill}}
Source:& \multicolumn{2}{l}{\scriptsize Estimated with GUK administrative and survey data.}\\
Notes: & 1. & First-difference estimates. A first-difference is defined as $\Delta x_{t+1}\equiv x_{t+1} - x_{t}$. Saving and repayment misses are taken from administrative data and merged with survey data at Year-Month of survey interviews. Intercept terms are omitted in estimating equations. Sample is continuing members and replacing members of early rejecters and received loans prior to 2015 Janunary. Labour income is in 1000 Tk unit andis sum of all earned labour incomes. Farm revenue is total of agricultural produce sales. \\
& 2. & ${}^{***}$, ${}^{**}$, ${}^{*}$ indicate statistical significance at 1\%, 5\%, 10\%, respetively. Standard errors are clustered at group (village) level.
\end{tabular}
\end{minipage}


\begin{palepinkleftbar}
\begin{finding}
\textsc{\small Table \ref{tab FD incomes original HH}} (1) and (3) show a general decrease in rd 1 - 2 period and a general increase in rd 2 - 4 periods for labour incomes. (2) and (4) suggest \textsf{Large grace} arm saw a greater swing (decrease and increases) which resulted in overall significant mean increase of -5.55 (at $p$ value of 21.66\%), yet not statistically different from \textsf{traditional}, while other arms have estimates closer to \textsf{traditional}. This labour income response can be due to the flood in rd 1 which reduced the labour incomes while repayment burden in later rounds prompted households to earn more labour incomes. Strong positive correlation with other members' previous 6 month repayment in (4) may be due to concerted peer efforts in repayment. Farm revenues do not show any systematic trend.
\end{finding}
\end{palepinkleftbar}


\subsection{Consumption}



%Number of HHs with consumption before the loan is disbursed (\textsf{ConsumptionBaseline} == 1) is small.
\begin{Schunk}
\begin{Soutput}
             ConsumptionBaseline
Arm              0    1
  traditional  513  284
  large        146 1002
  large grace   51  981
  cow          200  874
\end{Soutput}
\begin{Soutput}
Warning in `[.data.table`(con, , `:=`(grepout("RM", colnames(con)), NULL)): length(LHS)==0; no columns to delete or assign RHS to.
\end{Soutput}
\begin{Soutput}
Dropped 4028 obs due to NA.
Dropped 4028 obs due to NA.
\end{Soutput}
\begin{Soutput}
Warning in `[.data.table`(dat, , `:=`(grepout("Time.?2", colnames(dat)), : length(LHS)==0; no columns to delete or assign RHS to.
\end{Soutput}
\end{Schunk}

Consumption is observed in rd 2-4. There are 6400 observations, with first-differencing, it becomes 2372 observations with 42, 2330 households observed for 2, 3 times. 



\begin{Schunk}
\begin{Sinput}
source(paste0(pathprogram, "ConsumptionCovariateSelection.R"))
\end{Sinput}
\end{Schunk}





\hspace{-1cm}\begin{minipage}[t]{14cm}
\hfil\textsc{\normalsize Table \refstepcounter{table}\thetable: FD estimation of consumption\label{tab FD consumption original HH}}\\
\setlength{\tabcolsep}{1pt}
\setlength{\baselineskip}{8pt}
\renewcommand{\arraystretch}{.55}
\hfil\begin{tikzpicture}
\node (tbl) {\input{c:/data/GUK/analysis/save/membership1or4/ConsumptionOriginalHHsFDEstimationResults.tex}};
%\input{c:/dropbox/data/ramadan/save/tablecolortemplate.tex}
\end{tikzpicture}\\
\renewcommand{\arraystretch}{.8}
\setlength{\tabcolsep}{1pt}
\begin{tabular}{>{\hfill\scriptsize}p{1cm}<{}>{\hfill\scriptsize}p{.25cm}<{}>{\scriptsize}p{12cm}<{\hfill}}
Source:& \multicolumn{2}{l}{\scriptsize Estimated with GUK administrative and survey data.}\\
Notes: & 1. & First-difference estimates. A first-difference is defined as $\Delta x_{t+1}\equiv x_{t+1} - x_{t}$. Saving and repayment misses are taken from administrative data and merged with survey data at Year-Month of survey interviews. Intercept terms are omitted in estimating equations. Sample is continuing members and replacing members of early rejecters and received loans prior to 2015 Janunary. Consumption is annualised values. \\
& 2. & ${}^{***}$, ${}^{**}$, ${}^{*}$ indicate statistical significance at 1\%, 5\%, 10\%, respetively. Standard errors are clustered at group (village) level.
\end{tabular}
\end{minipage}


\hspace{-1cm}\begin{minipage}[t]{14cm}
\hfil\textsc{\normalsize Table \refstepcounter{table}\thetable: FD estimation of consumption, moderately poor vs. ultra poor\label{tab FD consumption2 original HH}}\\
\setlength{\tabcolsep}{1pt}
\setlength{\baselineskip}{8pt}
\renewcommand{\arraystretch}{.55}
\hfil\begin{tikzpicture}
\node (tbl) {\input{c:/data/GUK/analysis/save/membership1or4/ConsumptionPovertyStatusOriginalHHsFDEstimationResults.tex}};
%\input{c:/dropbox/data/ramadan/save/tablecolortemplate.tex}
\end{tikzpicture}\\
\renewcommand{\arraystretch}{.8}
\setlength{\tabcolsep}{1pt}
\begin{tabular}{>{\hfill\scriptsize}p{1cm}<{}>{\hfill\scriptsize}p{.25cm}<{}>{\scriptsize}p{12cm}<{\hfill}}
Source:& \multicolumn{2}{l}{\scriptsize Estimated with GUK administrative and survey data.}\\
Notes: & 1. & First-difference estimates. A first-difference is defined as $\Delta x_{t+1}\equiv x_{t+1} - x_{t}$. Saving and repayment misses are taken from administrative data and merged with survey data at Year-Month of survey interviews. Intercept terms are omitted in estimating equations. Sample is continuing members and replacing members of early rejecters and received loans prior to 2015 Janunary. Consumption is annualised values. \\
& 2. & ${}^{***}$, ${}^{**}$, ${}^{*}$ indicate statistical significance at 1\%, 5\%, 10\%, respetively. Standard errors are clustered at group (village) level.
\end{tabular}
\end{minipage}


\hspace{-1cm}\begin{minipage}[t]{14cm}
\hfil\textsc{\normalsize Table \refstepcounter{table}\thetable: FD estimation of consumption, large vs. small size loans\label{tab FD consumption3 original HH}}\\
\setlength{\tabcolsep}{1pt}
\setlength{\baselineskip}{8pt}
\renewcommand{\arraystretch}{.55}
\hfil\begin{tikzpicture}
\node (tbl) {\input{c:/data/GUK/analysis/save/membership1or4/ConsumptionSizeOriginalHHsFDEstimationResults.tex}};
%\input{c:/dropbox/data/ramadan/save/tablecolortemplate.tex}
\end{tikzpicture}\\
\renewcommand{\arraystretch}{.8}
\setlength{\tabcolsep}{1pt}
\begin{tabular}{>{\hfill\scriptsize}p{1cm}<{}>{\hfill\scriptsize}p{.25cm}<{}>{\scriptsize}p{12cm}<{\hfill}}
Source:& \multicolumn{2}{l}{\scriptsize Estimated with GUK administrative and survey data.}\\
Notes: & 1. & First-difference estimates. A first-difference is defined as $\Delta x_{t+1}\equiv x_{t+1} - x_{t}$. Saving and repayment misses are taken from administrative data and merged with survey data at Year-Month of survey interviews. Intercept terms are omitted in estimating equations. Sample is continuing members and replacing members of early rejecters and received loans prior to 2015 Janunary. Consumption is annualised values. \\
& 2. & ${}^{***}$, ${}^{**}$, ${}^{*}$ indicate statistical significance at 1\%, 5\%, 10\%, respetively. Standard errors are clustered at group (village) level.
\end{tabular}
\end{minipage}

\hspace{-1cm}\begin{minipage}[t]{14cm}
\hfil\textsc{\normalsize Table \refstepcounter{table}\thetable: FD estimation of consumption, with vs. without a grace period\label{tab FD consumption4 original HH}}\\
\setlength{\tabcolsep}{1pt}
\setlength{\baselineskip}{8pt}
\renewcommand{\arraystretch}{.55}
\hfil\begin{tikzpicture}
\node (tbl) {\input{c:/data/GUK/analysis/save/membership1or4/ConsumptionGraceOriginalHHsFDEstimationResults.tex}};
%\input{c:/dropbox/data/ramadan/save/tablecolortemplate.tex}
\end{tikzpicture}\\
\renewcommand{\arraystretch}{.8}
\setlength{\tabcolsep}{1pt}
\begin{tabular}{>{\hfill\scriptsize}p{1cm}<{}>{\hfill\scriptsize}p{.25cm}<{}>{\scriptsize}p{12cm}<{\hfill}}
Source:& \multicolumn{2}{l}{\scriptsize Estimated with GUK administrative and survey data.}\\
Notes: & 1. & First-difference estimates. A first-difference is defined as $\Delta x_{t+1}\equiv x_{t+1} - x_{t}$. Saving and repayment misses are taken from administrative data and merged with survey data at Year-Month of survey interviews. Intercept terms are omitted in estimating equations. Sample is continuing members and replacing members of early rejecters and received loans prior to 2015 Janunary. Consumption is annualised values. \\
& 2. & ${}^{***}$, ${}^{**}$, ${}^{*}$ indicate statistical significance at 1\%, 5\%, 10\%, respetively. Standard errors are clustered at group (village) level.
\end{tabular}
\end{minipage}


\begin{palepinkleftbar}
\begin{finding}
\textsc{\small Table \ref{tab FD consumption  original HH}} uses rd 2 - 4 data and shows an increase in per member consumption in rd 2 - 3 period. The estimates are imprecise for all interaction terms. Continued increases in consumption hints welfare gains, but do not differ by arms. Per member food consumption increases in rd 2- 3 period but decreases in rd 3 - 4 period.
\end{finding}
\end{palepinkleftbar}


\subsection{IGA}


\begin{Schunk}
\begin{figure}

{\centering \includegraphics[width=\maxwidth]{figure/ImpactEstimationMembership1or4/IGA_choices_original_HHs-1} 

}

\caption{Income generatng activity choices\\ {\footnotesize The first income generating activity choices are plotted.\setlength{\baselineskip}{8pt}}}\label{Figure IGA choices original HHs}
\end{figure}
\end{Schunk}
\begin{Schunk}
\begin{figure}

{\centering \includegraphics[width=\maxwidth]{figure/ImpactEstimationMembership1or4/All_IGA_choices_original_HHs-1} 

}

\caption{All income generatng activity choices\\ {\footnotesize All of multiple investment choices are summed by arms and the number of IGAs and plotted as bars. \setlength{\baselineskip}{8pt}}}\label{Figure All IGA choices original HHs}
\end{figure}
\end{Schunk}
\begin{Schunk}
\begin{figure}

{\centering \includegraphics[width=\maxwidth]{figure/ImpactEstimationMembership1or4/All_IGA_choices_collapsed_original_HHs-1} 

}

\caption{All income generatng activity choices collapsed over different number of IGAs\\ {\footnotesize All of multiple investment choices are summed by arms and plotted as bars. \setlength{\baselineskip}{8pt}}}\label{Figure All IGA choices collapsed original HHs}
\end{figure}
\end{Schunk}

\begin{palepinkleftbar}
\begin{finding}
\textsc{\small Figure \ref{Figure IGA choices}, \ref{Figure All IGA choices}} show that there are very few members who chose to invest in more than one project for the ``large'' arms, while in the \textsf{traditional} arm, almost no one invested only in one project. Goat/sheep and small trades are the top choices for the first IGA in \textsf{traditional}. This indicates the exitence of both a liquidity constraint and convexity in the production technology of large domestic animals. This also validates our supposition that dairy livestock production is the most preferred and probably the only economically viable investment choice. It reduces a concern that the \textsf{cow} arm may have imposed an unnecessary restriction in an investment choice by forcing to receive a cow. \textsc{\small Figure \ref{Figure All IGA choices collapsed}} shows there are a significant number of cases in the \textsf{traditional} arm that members reportedly raise cows, yet they are also accompanied by pararell projects in smaller livestock production and small trades. Contrasting \textsf{large}, \textsf{large grace} with \textsf{cow} arms, it suggests that entrepreneurship (to the extent that is necessary for dairy livestock production) may not be an impediment for a microfinance loan uptake among members.
\end{finding}
\end{palepinkleftbar}

Together with \textsc{\small Table \ref{tab FD saving}} showing smaller net saving and repayment among \textsf{traditional}, the restriction on a project choice induced by a smaller loaned sum resulted in smaller returns. Between with or no grace period loans, cumulative net saving and repayment are both larger with loans with a grace period. No such difference is found between \textsf{cow} and other arms.


\end{document}
