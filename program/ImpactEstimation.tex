% path0 <- "c:/data/GUK/"; path <- paste0(path0, "analysis/"); setwd(pathprogram <- paste0(path, "program/")); pathsource.mar <- paste0(path, "source/mar/"); pathreceived.mar <- paste0(path0, "received/mar/")
%  path0 <- "c:/data/GUK/"; path <- paste0(path0, "analysis/"); setwd(pathprogram <- paste0(path, "program/")); pathsource.mar <- paste0(path, "source/mar/"); pathreceived.mar <- paste0(path0, "received/mar/"); library(knitr); knit("ImpactEstimation.rnw", "ImpactEstimation.tex"); system("platex ImpactEstimation"); system("dvipdfmx ImpactEstimation")
%  path0 <- "c:/data/GUK/"; path <- paste0(path0, "analysis/"); setwd(pathprogram <- paste0(path, "program/")); system("recycle c:/data/GUK/analysis/program/cache/ImpactEstimation/"); library(knitr); knit("ImpactEstimation.rnw", "ImpactEstimation.tex"); system("platex ImpactEstimation"); system("dvipdfmx ImpactEstimation")

\input{c:/data/knitr_preamble.rnw}
\renewcommand\Routcolor{\color{gray30}}
\newtheorem{finding}{Finding}[section]
\makeatletter
\g@addto@macro{\UrlBreaks}{\UrlOrds}
\newcommand\gobblepars{%
    \@ifnextchar\par%
        {\expandafter\gobblepars\@gobble}%
        {}}
\newenvironment{lightgrayleftbar}{%
  \def\FrameCommand{\textcolor{lightgray}{\vrule width 1zw} \hspace{10pt}}% 
  \MakeFramed {\advance\hsize-\width \FrameRestore}}%
{\endMakeFramed}
\newenvironment{palepinkleftbar}{%
  \def\FrameCommand{\textcolor{palepink}{\vrule width 1zw} \hspace{10pt}}% 
  \MakeFramed {\advance\hsize-\width \FrameRestore}}%
{\endMakeFramed}
\makeatother
\usepackage{caption}
\usepackage{setspace}
\usepackage{framed}
\captionsetup[figure]{font={stretch=.6}} 
\def\pgfsysdriver{pgfsys-dvipdfm.def}
\usepackage{tikz}
\usetikzlibrary{calc, arrows, decorations, decorations.pathreplacing, backgrounds}
\usepackage{adjustbox}
\tikzstyle{toprow} =
[
top color = gray!20, bottom color = gray!50, thick
]
\tikzstyle{maintable} =
[
top color = blue!1, bottom color = blue!20, draw = white
%top color = green!1, bottom color = green!20, draw = white
]
\tikzset{
%Define standard arrow tip
>=stealth',
%Define style for different line styles
help lines/.style={dashed, thick},
axis/.style={<->},
important line/.style={thick},
connection/.style={thick, dotted},
}


\begin{document}
\setlength{\baselineskip}{12pt}





\hfil Estimating lending impacts with survey and admin data (after a meeting on May 10, 2018)\\

\hfil\MonthDY\\
\hfil{\footnotesize\currenttime}\\

\hfil Seiro Ito

\setcounter{tocdepth}{3}
\tableofcontents
\newpage

\setlength{\parindent}{1em}
\vspace{2ex}



\section{Summary}

\begin{description}
\vspace{1.0ex}\setlength{\itemsep}{1.0ex}\setlength{\baselineskip}{12pt}
\item[Low repayment rates]	Repayment was poor. Net saving was forfeit for repayment. Mean raw loan recovery rate (counting only repayments) measured at the end of third year was 0.67 overall, and was lowest for \textsf{traditional} at 0.48. Counting also net saving, these numbers change to 0.85, 0.59, respectively.
\item[No difference in repayment risk by poverty status] Raw loan recovery rates are 0.67, 0.67, respectively, for ultra poor and moderately poor. Also no statitically meaningful difference is found for cumulative repayment plus cumulative net saving.
\item[Traditional chose multiple small projects]	IGA is more diversified in the \textsf{traditional} than in other arms. With almost all the members in large-sized loan arms choosing cows, it suggests the presence of a poverty trap induced by a liquidity constraint and convexity in livestock production technology.
\item[Large-sized or grace period loans resulted in higher repayment rates]	Controlling for the loan size, larger initial lending resulted in larger repayment and net saving. As opposed to GUK's anxiety, lending was relatively less risky with large loans and loans with a grace period.
\item[Boys' junior high schooling suffered, but not girls']	(Using original panel) Schooling was negatively affected for boys attending a junior high school, but such an effect was mitigated for girls under arms with a grace period.  A weaker but similar pattern is also found for high school aged children. It hints increased labour demand for boys but the mechanism is unclear.
\item[No concern for entrepreneurship]	No difference in project choices between \textsf{cow} and \textsf{large, large grace}. Members who participated do not seem to show concerns for (lack of) entrepreneurship.
\item[No difference in household assets]	Household assets increased in rd 1 - 3, then reduced in rd 4 (possibly liquidating for repayment purpose), with the overall impact of increased household asset values yet no statistically significant difference between arms. 
\item[No difference in labour incomes, per member consumption, marriage rates]	Per member consumption increased in all arms with no difference between arms. Marriage rates do not difffer between arms. A greater swing in labour incomes for \textsf{large}.
\end{description}

\section{Read files}





Description of data:
\begin{description}
\vspace{1.0ex}\setlength{\itemsep}{1.0ex}\setlength{\baselineskip}{12pt}
\item[ad]	Administrative data: Up to [-24, 48] months after first loan disbursement.
\item[X1]	Schooling panel with attrition. Aged 6-18 in rd1. \textsf{Enrolled=\{0,1\}} is defined for children aged 6-18 in rd1 by referencing to \textsf{currently\_enrolled} and age information.
\item[X2]	Schooling panel after augmenting attrited children to \textsf{X1}. Attrited children are augmented by assuming to be out of school. \textsf{AssignRegression} is group classification: Number of observation is \textsf{618, 633, 594, 593, 363, 100} for \textsf{traditional, large, largeGrace, cow, dropOuts, forcedDropOuts}, respectively.
\item[ros]	 \textsf{roster} to condition the initial status prior to participation.
\item[ass]	 Assets. Household assets (houses, durables) and productive assets (machines, tools). 
\item[lvo]	Livestock holding. Rd 3 data is not entered yet.
\item[lab]	Labour incomes.
\item[far]	Farming revenues (no costs reported).
\item[con]	Household consumption. Food expenditure asks both bought and consumed volumes and prices. We impute consumption values by using median prices. All quantity is set to annualised quantity.
\end{description}

How I combined between pages: First, merge admin data \textsf{ad} with roster data \textsf{ros} with \textsf{hhid, Year, Month} as keys. Keep only at dates when survey data match. Second, merge \textsf{ad+ros} with other data \textsf{X1}, \textsf{X2}, \textsf{ass}, ... 

There are 4029 non-matching cases if we merge using \textsf{Year, Month} of \textsf{IntDate} in survey data and of \textsf{Date} in admin data. This is inevitable because the admin data starts from 2013-05-01 while survey data starts from 2011-10-09 and rd 1 ends at 2013-10-12 for \textsf{oldMember}s with the median date 2012-10-18. Below gives \textsf{Year, Month} in roster data with no match in admin data.
\begin{Schunk}
\begin{Soutput}
                YearMonthOfIntDate
AssignRegression 2011-October 2011-November 2012-January 2012-October
  traditional               0             0            0          108
  large                     0             1            0          222
  largeGrace                1             0           17          216
  cow                       4             0            0          248
  dropOuts                  1             0            0          173
  forcedDropOuts            0             0            0           35
                YearMonthOfIntDate
AssignRegression 2012-November 2012-December 2013-September 2013-October
  traditional               79            12              6           13
  large                     72             7              0            0
  largeGrace                36            35              0            0
  cow                       19             7              0            0
  dropOuts                  43            12              0            2
  forcedDropOuts            34             0              0            0
                YearMonthOfIntDate
AssignRegression 2014-January 2014-October 2014-November 2014-December
  traditional               5           26             2             8
  large                     0            0             0             0
  largeGrace                0            0             0             0
  cow                       0            0             0             0
  dropOuts                  6           40            35            22
  forcedDropOuts            0            0             2             0
                YearMonthOfIntDate
AssignRegression 2015-November 2015-December 2016-January 2017-January
  traditional               28             9            5           16
  large                      0             0            0            0
  largeGrace                 0             0            0            0
  cow                        0             0            0            0
  dropOuts                  65            23           17           20
  forcedDropOuts             1             0            0            0
                YearMonthOfIntDate
AssignRegression 2017-February 2017-March 2017-April NA-NA
  traditional               19          0          5     1
  large                      0          0          0     0
  largeGrace                 0          0          0     0
  cow                        0          0          0     0
  dropOuts                  61         14          8    18
  forcedDropOuts             0          0          0     0
\end{Soutput}
\begin{Soutput}
               FALSE  TRUE
YearMonthMatch  4029 12396
YearMatch       4029 12396
\end{Soutput}
\end{Schunk}

%I will use one month lag in admin data as a match with roster data because it retains MatchRecord[1, 2, with = F] roster entries. \gobblepars



In \textsf{roster + admin} (base: roster): Tabulate \textsf{hhid} observations by \textsf{survey} round and \textsf{Arm}.
\begin{Schunk}
\begin{Soutput}
      Arm
survey traditional large large grace cow forcedDropOuts dropOuts
     1         419   405         411 423             69      142
     2         419   408         402 408             48      109
     3         422   411         410 412             47      103
     4         408   403         403 400              0      103
\end{Soutput}
\end{Schunk}
In \textsf{roster + admin}: Tabulate observations after keeping only observations used in estimation: Keep if \textsf{creditstatuts} = yes, \& \textsf{Mstatus} includes strings old, iRep, newG, \& \textsf{DisDate1} is before 2015-01-01, \& \textsf{TradGroup} does not include strings tw or dou.
\begin{Schunk}
\begin{Soutput}
      Arm
survey traditional large large grace cow
     1         166   333         317 290
     2         103   342         318 292
     3         103   343         320 292
     4         102   338         315 283
\end{Soutput}
\end{Schunk}
In \textsf{roster + admin}: If we keep \textsf{creditstatuts} = yes, \& \textsf{Mstatus} includes strings old, iRep, newG, \& \textsf{TradGroup} does not include strings tw or dou (relaxing  \textsf{DisDate1} is before 2015-01-01). \gobblepars
\begin{Schunk}
\begin{Soutput}
      Arm
survey traditional large large grace cow
     1         192   387         394 367
     2         103   401         392 365
     3         103   402         397 368
     4         102   394         390 356
\end{Soutput}
\end{Schunk}
This shows addition is mostly in round 1 for \textsf{traditional} but in all rds for other arms. \textsf{FirstDisPeriod} gives the period of first disbursement, and all credit reeceivers received loans by the end of 2015.
\begin{Schunk}
\begin{Soutput}
        creditstatus
DisDate1  No <NA>
    <NA> 146  235
\end{Soutput}
\end{Schunk}
See the breakdown of first disbursement by Arm at rd 1.
\begin{Schunk}
\begin{Soutput}
               Arm
FirstDisPeriod  traditional large large grace cow
  BeforeJan2015         166   333         317 290
  Year2015               26    54          77  77
  Year2016                0     0           0   0
  AfterJan2017            0     0           0   0
\end{Soutput}
\end{Schunk}
Schooling pattern in X1.
\begin{Schunk}
\begin{Soutput}

0000 0001 000n 0011 001n 00nn 0100 0101 010n 0111 011n 01nn 0nnn 1000 1001 100n 
 208   36  216  152   33  192   16    4    9  840  105   70  316   64    8   45 
1011 101n 10nn 1100 1101 110n 1110 1111 111n 11nn 1nnn 
  56   24   86   48   16   84   28 5172  654  326  199 
\end{Soutput}
\end{Schunk}
Save roster-admin data.
\begin{Schunk}
\begin{Sinput}
saveRDS(ar, paste0(path1234, "RosterAdminData.rds"))
fwrite(ar, paste0(path1234, "RosterAdminData.prn"), sep = "\t", quote = F)
\end{Sinput}
\end{Schunk}
A snippet of \textsf{admin + roster} data:
\begin{Schunk}
\begin{Soutput}
      Arm    hhid mid survey    IntDate       Date CumRepaid AgeComputed
 1: large 7010101   3      1 2011-11-06       <NA>        NA          15
 2: large 7010101   3      2 2014-10-11 2014-10-01      7000          17
 3: large 7010101   3      3 2015-11-21 2015-11-01      9500          18
 4: large 7010101   3      4 2017-02-14 2017-02-01     15970          20
 5: large 7010102   5      1 2012-11-06       <NA>        NA           1
 6: large 7010102   5      2 2014-10-11 2014-10-01      8000           3
 7: large 7010102   5      3 2015-11-22 2015-11-01     12225           4
 8: large 7010102   5      4 2017-02-14 2017-02-01     16000           6
 9: large 7010105   3      1 2012-11-07       <NA>        NA           8
10: large 7010105   3      2 2014-10-11 2014-10-01      4925          10
11: large 7010105   3      3 2015-11-19 2015-11-01      8050          11
12: large 7010105   3      4 2017-02-14 2017-02-01     10050          13
\end{Soutput}
\end{Schunk}
In \textsf{X1}: Number of unique \textsf{hhid}s by \textsf{year} (original entry) or \textsf{Year} (extracted from \textsf{IntDate}).
\begin{Schunk}
\begin{Soutput}
             year
NumberOfHHids 2012 2013 2014 2015 2017
         1542 2098  806 2282 2024 1797
\end{Soutput}
\begin{Soutput}
             Year
NumberOfHHids 2011 2012 2013 2014 2015 2016 2017 <NA>
         1542    7 2030  691 2182 1366  575 1695  461
\end{Soutput}
\end{Schunk}
In \textsf{X1}: Number of observations tabulated by \textsf{year} (original entry) and round (\textsf{survey}).
\begin{Schunk}
\begin{Soutput}
      survey
year      1    2    3    4
  2012 2071    0    0    0
  2013  689    0    0    0
  2014    0 2179    0    0
  2015    0    0 1943    0
  2017    0    0    0 1697
\end{Soutput}
\end{Schunk}
In \textsf{X1}: RoundOrder is 1 if individual is observed for the first time in data, 2 if for the second time, ...
\begin{Schunk}
\begin{Soutput}
      RoundOrder
year      1    2    3    4
  2012 2098    0    0    0
  2013  806    0    0    0
  2014    0 2282    0    0
  2015    0   79 1945    0
  2017    0   28  107 1662
\end{Soutput}
\end{Schunk}
In \textsf{X2}: Number of observations tabulated by year and round (\textsf{survey}).
\begin{Schunk}
\begin{Soutput}
      survey
year      1    2    3    4
  2012 2071    0    0    0
  2013  689    0    0    0
  2014    0 2598    0    0
  2015    0    0 2451    0
  2017    0    0    0 2203
\end{Soutput}
\end{Schunk}
In \textsf{X2}: RoundOrder.
\begin{Schunk}
\begin{Soutput}
      RoundOrder
year      1    2    3    4    5
  2012 2901    0    0    0    0
  2013    0 2901    0    0    0
  2014    0    0 2901    0    0
  2015    0    0    0 2901    0
  2017    0    0    0    0 2901
\end{Soutput}
\end{Schunk}
In \textsf{X1}: Number of observations tabulated by year and age (\textsf{AgeComputed}).
\begin{Schunk}
\begin{Soutput}
      AgeComputed
year     6   7   8   9  10  11  12  13  14  15  16  17  18  19  20  21  22  23
  2012 168 264 279 114 333  77 237 109 104 173 103  43  94   0   0   0   0   0
  2013  48  93  90  61 118  60  79  55  46  58  46  14  38   0   0   0   0   0
  2014   0  43 222 317 298 211 346 131 234 121 124 152  62  15   6   0   0   0
  2015   0   0  42 225 311 291 198 302 118 192 100  93  95  38  11   8   0   0
  2017   0   0   0   0  40 218 289 279 186 272 110 171  90  64  51  22   4   1
\end{Soutput}
\end{Schunk}
In \textsf{X2}: Number of observations tabulated by year and age (\textsf{AgeComputed}).
\begin{Schunk}
\begin{Soutput}
      AgeComputed
year     5   6   7   8   9  10  11  12  13  14  15  16  17  18  19  20  21  22
  2012  48 261 354 340 232 393 156 291 155 161 219 116  81  94   0   0   0   0
  2013   0  48 261 354 340 232 393 156 291 155 161 219 116  81  94   0   0   0
  2014   0   0  48 261 354 340 232 393 156 291 155 161 219 116  81  94   0   0
  2015   0   0   0  48 261 354 340 232 393 156 291 155 161 219 116  81  94   0
  2017   0   0   0   0   0  48 261 354 340 232 393 156 291 155 161 219 116  81
      AgeComputed
year    23
  2012   0
  2013   0
  2014   0
  2015   0
  2017  94
\end{Soutput}
\end{Schunk}
\begin{Schunk}
\begin{Soutput}
                MonthsElapsedNA
Arm              FALSE TRUE
  traditional     2676 1127
  large           3015  778
  large grace     2884  813
  cow             2739 1071
  forcedDropOuts     0  359
  dropOuts           0  963
\end{Soutput}
\end{Schunk}

Save all data.

\begin{Schunk}
\begin{Sinput}
fwrite(x1, paste0(path1234, "RosterAdminSchoolingData.prn"), sep = "\t", quote = F)
fwrite(x2, paste0(path1234, "RosterAdminSchoolingAugmentedData.prn"), sep = "\t", quote = F)
fwrite(ass, paste0(path1234, "AssetAdminData.prn"), sep = "\t", quote = F)
fwrite(lvo, paste0(path1234, "LivestockAdminData.prn"), sep = "\t", quote = F)
fwrite(lab, paste0(path1234, "LabourIncomeAdminData.prn"), sep = "\t", quote = F)
fwrite(far, paste0(path1234, "FarmRevenueAdminData.prn"), sep = "\t", quote = F)
fwrite(con, paste0(path1234, "ConsumptionAdminData.prn"), sep = "\t", quote = F)
\end{Sinput}
\end{Schunk}

Further data preparations (trimming, round numbering, creating dummy vectors, interaction terms) for estimation. Produces files: \textsf{\footnotesize RosterAdminDataUsedForEstimation.prn, AssetAdminDataUsedForEstimation.prn, LivestockAdminDataUsedForEstimation.prn, LabourIncomeAdminDataUsedForEstimation.prn, FarmRevenueAdminDataUsedForEstimation.prn, ConsumptionAdminDataUsedForEstimation.prn}.



\section{Estimation}


\subsection{Schooling}


If using \textsf{x1}, retain only the complete portion of panel.



\hspace{-1cm}\begin{minipage}[t]{14cm}
\hfil\textsc{\normalsize Table \refstepcounter{table}\thetable: OLS estimation of school enrollment\label{tab ols school}}\\
\setlength{\tabcolsep}{1pt}
\setlength{\baselineskip}{8pt}
\renewcommand{\arraystretch}{.55}
\hfil\begin{tikzpicture}
\node (tbl) {\input{c:/data/GUK/analysis/save/SchoolingOLSEstimationResults.tex}};
%\input{c:/dropbox/data/ramadan/save/tablecolortemplate.tex}
\end{tikzpicture}\\
\renewcommand{\arraystretch}{.8}
\setlength{\tabcolsep}{1pt}
\begin{tabular}{>{\hfill\scriptsize}p{1cm}<{}>{\hfill\scriptsize}p{.25cm}<{}>{\scriptsize}p{12cm}<{\hfill}}
Source:& \multicolumn{2}{l}{\scriptsize Estimated with GUK administrative and survey data.}\\
Notes: & 1. & Intercept terms are omitted in estimating equations. Year effects are included in estimation (not shown). \textsf{x1} is complete portion of panel. \textsf{x2} is a panel data augmenting attrited members in \textsf{x1} with an assumption that they are out of school unless it is explicitly stated as attending school by family members. \\
& 2. & ${}^{***}$, ${}^{**}$, ${}^{*}$ indicate statistical significance at 1\%, 5\%, 10\%, respetively. Standard errors are clustered at group (village) level.
\end{tabular}
\end{minipage}

Check number of observations in each cell:





\hspace{-1cm}\begin{minipage}[t]{14cm}
\hfil\textsc{\normalsize Table \refstepcounter{table}\thetable: Number of observations in each cells of schooling regression in Table \ref{tab ols school}\label{tab num obs ols school}}\\
\setlength{\tabcolsep}{1pt}
\setlength{\baselineskip}{8pt}
\renewcommand{\arraystretch}{.45}
\hfil\begin{tikzpicture}
\node (tbl) {\input{c:/data/GUK/analysis/save/NumObsSchoolingOLS.tex}};
%\input{c:/dropbox/data/ramadan/save/tablecolortemplate.tex}
\end{tikzpicture}\\
\renewcommand{\arraystretch}{.8}
\setlength{\tabcolsep}{1pt}
\begin{tabular}{>{\hfill\scriptsize}p{1cm}<{}>{\hfill\scriptsize}p{.25cm}<{}>{\scriptsize}p{12cm}<{\hfill}}
Source:& \multicolumn{2}{l}{\scriptsize GUK administrative and survey data.}\\
Notes: & 1. &  \\
& 2. & %${}^{***}$, ${}^{**}$, ${}^{*}$ indicate statistical significance at 1\%, 5\%, 10\%, respetively. Standard errors are clustered at group (village) level.
\end{tabular}
\end{minipage}

\hspace{-1cm}\begin{minipage}[t]{14cm}
\hfil\textsc{\normalsize Table \refstepcounter{table}\thetable: OLS estimation of school enrollment, different grouping\label{tab ols school2}}\\
\setlength{\tabcolsep}{1pt}
\setlength{\baselineskip}{8pt}
\renewcommand{\arraystretch}{.55}
\hfil\begin{tikzpicture}
\node (tbl) {\input{c:/data/GUK/analysis/save/SchoolingOLSEstimationResults2.tex}};
%\input{c:/dropbox/data/ramadan/save/tablecolortemplate.tex}
\end{tikzpicture}\\
\renewcommand{\arraystretch}{.8}
\setlength{\tabcolsep}{1pt}
\begin{tabular}{>{\hfill\scriptsize}p{1cm}<{}>{\hfill\scriptsize}p{.25cm}<{}>{\scriptsize}p{12cm}<{\hfill}}
Source:& \multicolumn{2}{l}{\scriptsize Estimated with GUK administrative and survey data.}\\
Notes: & 1. & Intercept terms are omitted in estimating equations. Year effects are included in estimation (not shown). \textsf{x1} is complete portion of panel. \textsf{x2} is a panel data augmenting attrited members in \textsf{x1} with an assumption that they are out of school unless it is explicitly stated as attending school by family members. \textsf{SmallSize} includes \textsf{Traditional}, \textsf{LargeSize} includes \textsf{Large, Large grace, Cow}. \textsf{WithoutGrace} includes \textsf{Traditional, Large}, \textsf{WithGrace} includes \textsf{Large grace, cow}.\\
& 2. & ${}^{***}$, ${}^{**}$, ${}^{*}$ indicate statistical significance at 1\%, 5\%, 10\%, respetively. Standard errors are clustered at group (village) level.
\end{tabular}
\end{minipage}

\hspace{-1cm}\begin{minipage}[t]{14cm}
\hfil\textsc{\normalsize Table \refstepcounter{table}\thetable: Number of observations in each cells of schooling regression in Table \ref{tab ols school2}\label{tab num obs ols school2}}\\
\setlength{\tabcolsep}{1pt}
\setlength{\baselineskip}{8pt}
\renewcommand{\arraystretch}{.45}
\hfil\begin{tikzpicture}
\node (tbl) {
\input{c:/data/GUK/analysis/save/NumObsSchoolingOLS21.tex}
\input{c:/data/GUK/analysis/save/NumObsSchoolingOLS22.tex}};
%\input{c:/dropbox/data/ramadan/save/tablecolortemplate.tex}
\end{tikzpicture}\\
\renewcommand{\arraystretch}{.8}
\setlength{\tabcolsep}{1pt}
\begin{tabular}{>{\hfill\scriptsize}p{1cm}<{}>{\hfill\scriptsize}p{.25cm}<{}>{\scriptsize}p{12cm}<{\hfill}}
Source:& \multicolumn{2}{l}{\scriptsize GUK administrative and survey data.}\\
Notes: & 1. &  \\
& 2. & %${}^{***}$, ${}^{**}$, ${}^{*}$ indicate statistical significance at 1\%, 5\%, 10\%, respetively. Standard errors are clustered at group (village) level.
\end{tabular}
\end{minipage}


\hspace{-1cm}\begin{minipage}[t]{14cm}
\hfil\textsc{\normalsize Table \refstepcounter{table}\thetable: OLS estimation of school enrollment, ultra poor vs. moderately poor\label{tab ols school3}}\\
\setlength{\tabcolsep}{1pt}
\setlength{\baselineskip}{8pt}
\renewcommand{\arraystretch}{.55}
\hfil\begin{tikzpicture}
\node (tbl) {\input{c:/data/GUK/analysis/save/SchoolingOLSEstimationPovertystatusResults.tex}};
%\input{c:/dropbox/data/ramadan/save/tablecolortemplate.tex}
\end{tikzpicture}\\
\renewcommand{\arraystretch}{.8}
\setlength{\tabcolsep}{1pt}
\begin{tabular}{>{\hfill\scriptsize}p{1cm}<{}>{\hfill\scriptsize}p{.25cm}<{}>{\scriptsize}p{12cm}<{\hfill}}
Source:& \multicolumn{2}{l}{\scriptsize Estimated with GUK administrative and survey data.}\\
Notes: & 1. & Intercept terms are omitted in estimating equations. Year effects are included in estimation (not shown). \textsf{x1} is complete portion of panel. \textsf{x2} is a panel data augmenting attrited members in \textsf{x1} with an assumption that they are out of school unless it is explicitly stated as attending school by family members. \\
& 2. & ${}^{***}$, ${}^{**}$, ${}^{*}$ indicate statistical significance at 1\%, 5\%, 10\%, respetively. Standard errors are clustered at group (village) level.
\end{tabular}
\end{minipage}

\begin{palepinkleftbar}
\begin{finding}
\textsc{\small Table \ref{tab ols school}} shows school enrollment is higher for \textsf{x1} than \textsf{x2}, indicating nonattriting members are school goers. When using \textsf{x1} data, \textsf{cow} and \textsf{large grace} show negative impacts for older children, yet not for girls in junior high schools. In fact, (1) shows that girles in high school have higher enrollment in \textsf{x1} and both junior and high schools for \textsf{x2} in (4). This may be due to increased labour demand within a family for boys. Similar patterns are found in \textsf{x2} data, yet not statistically significant, probably because data augmentation introduces more school dropouts among older girls. In \textsc{\small Table \ref{tab ols school2}} when using with grace/without grace grouping, the pattern becomes statistically significant for both \textsf{x1} and \textsf{x2}. Large size vs. small size contrast has smaller statistical power that more subtle outcomes cannot be detected. No difference between ultra and moderately poor is found in \textsc{\small Table \ref{tab ols school3}}.
\end{finding}
\end{palepinkleftbar}


\subsection{Repayment and net saving}

Note all binary interaction terms are demeaned and then interacted.

\begin{Schunk}
\begin{figure}

{\centering \includegraphics[width=\maxwidth]{figure/ImpactEstimation/Cumulative_net_saving-1} 

}

\caption[Cumulative weekly net saving]{Cumulative weekly net saving}\label{Figure Cumulative net saving}
\end{figure}
\end{Schunk}





\hspace{-1cm}\begin{minipage}[t]{14cm}
\hfil\textsc{\normalsize Table \refstepcounter{table}\thetable: FD estimation of cumulative net saving and repayment\label{tab FD saving}}\\
\setlength{\tabcolsep}{1pt}
\setlength{\baselineskip}{8pt}
\renewcommand{\arraystretch}{.55}
\hfil\begin{tikzpicture}
\node (tbl) {\input{c:/data/GUK/analysis/save/SavingFDEstimationResults.tex}};
%\input{c:/dropbox/data/ramadan/save/tablecolortemplate.tex}
\end{tikzpicture}\\
\renewcommand{\arraystretch}{.8}
\setlength{\tabcolsep}{1pt}
\begin{tabular}{>{\hfill\scriptsize}p{1cm}<{}>{\hfill\scriptsize}p{.25cm}<{}>{\scriptsize}p{12cm}<{\hfill}}
Source:& \multicolumn{2}{l}{\scriptsize Estimated with GUK administrative and survey data.}\\
Notes: & 1. & First-difference estimates. First-differenced regressands are regressed on categorical and time-variant covariates. Net saving is taken from administrative data and merged with survey data at Year-Month of survey interviews. Head age and literacy are from baseline data. Intercept terms are omitted in estimating equations. Net saving is saving - withdrawal. \\
& 2. & ${}^{***}$, ${}^{**}$, ${}^{*}$ indicate statistical significance at 1\%, 5\%, 10\%, respetively. Standard errors are clustered at group (village) level.
\end{tabular}
\end{minipage}

\hspace{-1cm}\begin{minipage}[t]{14cm}
\hfil\textsc{\normalsize Table \refstepcounter{table}\thetable: FD estimation of net cumulative saving and repayment, ultra poor vs. moderately poor\label{tab FD saving2}}\\
\setlength{\tabcolsep}{1pt}
\setlength{\baselineskip}{8pt}
\renewcommand{\arraystretch}{.55}
\hfil\begin{tikzpicture}
\node (tbl) {\input{c:/data/GUK/analysis/save/SavingPovertystatusFDEstimationResults.tex}};
%\input{c:/dropbox/data/ramadan/save/tablecolortemplate.tex}
\end{tikzpicture}\\
\renewcommand{\arraystretch}{.8}
\setlength{\tabcolsep}{1pt}
\begin{tabular}{>{\hfill\scriptsize}p{1cm}<{}>{\hfill\scriptsize}p{.25cm}<{}>{\scriptsize}p{12cm}<{\hfill}}
Source:& \multicolumn{2}{l}{\scriptsize Estimated with GUK administrative and survey data.}\\
Notes: & 1. & First-difference estimates. First-differenced regressands are regressed on categorical and time-variant covariates. Net saving is taken from administrative data and merged with survey data at Year-Month of survey interviews. Head age and literacy are from baseline data. Intercept terms are omitted in estimating equations. Net saving is saving - withdrawal. \\
& 2. & ${}^{***}$, ${}^{**}$, ${}^{*}$ indicate statistical significance at 1\%, 5\%, 10\%, respetively. Standard errors are clustered at group (village) level.
\end{tabular}
\end{minipage}


\begin{palepinkleftbar}
\begin{finding}
\textsc{\small Table \ref{tab FD saving}} shows net saving decreases, mostly in the later rounds. This reflects the use of saving for repayment. Ultra poor had a setback in repayment during rd 2-3 as indicated in \textsc{\small Table \ref{tab FD saving2}}, but the the repayment rates at the end of third year do not differ as we have seen at the beginning of this note.
\end{finding}
\end{palepinkleftbar}


\subsection{Assets}



\hspace{-1cm}\begin{minipage}[t]{14cm}
\hfil\textsc{\normalsize Table \refstepcounter{table}\thetable: FD estimation of assets\label{tab FD assets}}\\
\setlength{\tabcolsep}{1pt}
\setlength{\baselineskip}{8pt}
\renewcommand{\arraystretch}{.55}
\hfil\begin{tikzpicture}
\node (tbl) {\input{c:/data/GUK/analysis/save/AssetFDEstimationResults.tex}};
%\input{c:/dropbox/data/ramadan/save/tablecolortemplate.tex}
\end{tikzpicture}\\
\renewcommand{\arraystretch}{.8}
\setlength{\tabcolsep}{1pt}
\begin{tabular}{>{\hfill\scriptsize}p{1cm}<{}>{\hfill\scriptsize}p{.25cm}<{}>{\scriptsize}p{12cm}<{\hfill}}
Source:& \multicolumn{2}{l}{\scriptsize Estimated with GUK administrative and survey data.}\\
Notes: & 1. & First-difference estimates. Saving and repayment misses are taken from administrative data and merged with survey data at Year-Month of survey interviews. Intercept terms are omitted in estimating equations. Sample is continuing members and replacing members of early rejecters and received loans prior to 2015 Janunary. Household assets do not include livestock. \\
& 2. & ${}^{***}$, ${}^{**}$, ${}^{*}$ indicate statistical significance at 1\%, 5\%, 10\%, respetively. Standard errors are clustered at group (village) level.
\end{tabular}
\end{minipage}

\begin{palepinkleftbar}
\begin{finding}
\textsc{\small Table \ref{tab FD assets}} shows household assets increase after receiving the loans. Total incremant is largest among the \textsf{large} arm. Increments are positive in rd 1 - 3, suggesting substantial purchase after receiving a loan. Significant decreases in rd 3 - 4 for large loan arms indicate liquidation of assets for repayment.
\end{finding}
\end{palepinkleftbar}


\subsection{Livestock}



\hspace{-1cm}\begin{minipage}[t]{14cm}
\hfil\textsc{\normalsize Table \refstepcounter{table}\thetable: FD estimation of livestock holding values\label{tab FD livestock}}\\
\setlength{\tabcolsep}{1pt}
\setlength{\baselineskip}{8pt}
\renewcommand{\arraystretch}{.55}
\hfil\begin{tikzpicture}
\node (tbl) {\input{c:/data/GUK/analysis/save/LivestockFDEstimationResults.tex}};
%\input{c:/dropbox/data/ramadan/save/tablecolortemplate.tex}
\end{tikzpicture}\\
\renewcommand{\arraystretch}{.8}
\setlength{\tabcolsep}{1pt}
\begin{tabular}{>{\hfill\scriptsize}p{1cm}<{}>{\hfill\scriptsize}p{.25cm}<{}>{\scriptsize}p{12cm}<{\hfill}}
Source:& \multicolumn{2}{l}{\scriptsize Estimated with GUK administrative and survey data.}\\
Notes: & 1. & First-difference estimates. Saving and repayment misses are taken from administrative data and merged with survey data at Year-Month of survey interviews. Intercept terms are omitted in estimating equations. Sample is continuing members and replacing members of early rejecters and received loans prior to 2015 Janunary. Regressand is \textsf{TotalImputedValue}, a sum of all livestock holding values evaluated at respective median market prices in the same year. \\
& 2. & ${}^{***}$, ${}^{**}$, ${}^{*}$ indicate statistical significance at 1\%, 5\%, 10\%, respetively. Standard errors are clustered at group (village) level.
\end{tabular}
\end{minipage}

Check quickly if the estimated results make sense.
\begin{Schunk}
\begin{Soutput}
           hhid         Arm Year livestock_code number_owned mrkt_value
 1:     7020601 large grace 2012            Cow            7          0
 2:     7020601 large grace 2014            Hen            2        150
 3:     7020601 large grace 2015             NA            1      15000
 4:     7020601 large grace 2017            Hen            4        250
 5:     7020606 large grace 2012            Cow            7          0
 6:     7020606 large grace 2014            Cow            1      25000
 7:     7020606 large grace 2015             NA           NA         NA
 8:     7020606 large grace 2017            Cow            1      30000
 9:     7020614 large grace 2012             NA            0          0
10:     7020614 large grace 2014            Cow            2      16000
11:     7020614 large grace 2015             NA            5      16000
12:     7020614 large grace 2017            Cow            6      24000
13:     7020918       large 2012            Cow            7          0
14:     7020918       large 2014          Sheep            1       1800
15:     7020918       large 2015             NA            4       2000
16:     7020918       large 2017            Cow            1      30000
17:     7021004 large grace 2012            Cow            7          0
18:     7021004 large grace 2014            Cow            4      24000
19:     7021004 large grace 2016             NA            2      25000
20:     7021004 large grace 2017           Goat            6       4000
21:     7021216         cow 2012            Cow            6          0
22:     7021216         cow 2014           Goat            4       1500
23:     7021216         cow 2015             NA            3      18000
24:     7021216         cow 2017            Cow            3      30000
25:     7021307       large 2012            Cow            7          0
26:     7021307       large 2014            Hen            5        150
27:     7021307       large 2015             NA            1        200
28:     7021307       large 2017            Cow            3      38000
29:     7054012 large grace 2012           Goat            8          0
30:     7054012 large grace 2014            Cow           15      20000
31:     7054012 large grace 2015             NA           12      16000
32:     7054012 large grace 2017           Goat            5       2800
33:     7096202       large 2012            Cow            8          0
34:     7096202       large 2014            Hen            4        150
35:     7096202       large 2015             NA            4        200
36:     7096202       large 2017            Cow            9      20000
37:     7096207       large 2012            Cow            4          0
38:     7096207       large 2014            Hen           12        100
39:     7096207       large 2015             NA            7      22000
40:     7096207       large 2017            Cow            6      16000
41:     7096218       large 2012            Cow            1          0
42:     7096218       large 2014            Cow            9      16000
43:     7096218       large 2015             NA            7      16000
44:     7096218       large 2017            Cow            6      20000
45:     8169619       large 2012             NA            0          0
46:     8169619       large 2014          Sheep            2       1400
47:     8169619       large 2016             NA            2       1800
48:     8169619       large 2017            Cow            6      38000
49:  9907031414 large grace 2013            Cow            6          0
50:  9907031414 large grace 2014            Cow            2      19000
51:  9907031414 large grace 2015             NA            2       1800
52:  9907031414 large grace 2017            Cow            2      30000
53: 99070211804       large 2013            Cow            6          0
54: 99070211804       large 2014           Goat            2       1600
55: 99070211804       large 2015             NA            2        200
56: 99070211804       large 2017            Hen            2        250
57: 99070211805       large 2013            Cow            8          0
58: 99070211805       large 2014            Cow            3      18000
59: 99070211805       large 2015             NA            1        200
60: 99070211805       large 2017            Cow            2      39000
61: 99070211810       large 2013            Cow            6          0
62: 99070211810       large 2014            Hen            1        200
63: 99070211810       large 2015             NA            4       3400
64: 99070211810       large 2017           Goat            4       3300
65: 99070511013       large 2013            Cow            3          0
66: 99070511013       large 2014            Cow            6      20000
67: 99070511017       large 2013            Cow            7          0
68: 99070511017       large 2014           Goat            1       2000
69: 99070511017       large 2015             NA           NA         NA
70: 99070511017       large 2017           Goat            1       2000
71: 99070911605         cow 2013            Cow            8          0
72: 99070911605         cow 2014           Goat            1       3000
73: 99070911605         cow 2015             NA            3      25000
74: 99070911605         cow 2017            Cow            4      28000
75: 99070911613         cow 2013            Cow            7      13000
76: 99070911613         cow 2014            Cow            1      15000
77: 99070911613         cow 2015             NA            4        100
78: 99070911613         cow 2017           Goat            6       4000
           hhid         Arm Year livestock_code number_owned mrkt_value
    TotalImputedValue
 1:            140000
 2:               150
 3:             15000
 4:               250
 5:            140000
 6:             20000
 7:                 0
 8:             20000
 9:                 0
10:             40000
11:             16000
12:            120000
13:            140000
14:              1800
15:              2000
16:             20000
17:            140000
18:             80000
19:             25000
20:              8400
21:            120000
22:              5600
23:             18000
24:             60000
25:            140000
26:               150
27:               200
28:             60000
29:             11200
30:            300000
31:             16000
32:              7000
33:            160000
34:               150
35:               200
36:            180000
37:             80000
38:               100
39:             22000
40:            120000
41:             20000
42:            180000
43:             16000
44:            120000
45:                 0
46:              1400
47:              1800
48:            120000
49:            120000
50:             40000
51:              1800
52:             40000
53:            120000
54:              2800
55:               200
56:               250
57:            160000
58:             60000
59:               200
60:             40000
61:            120000
62:               200
63:              3400
64:              5600
65:             60000
66:            120000
67:            140000
68:              1400
69:                 0
70:              1400
71:            160000
72:              1400
73:             25000
74:             80000
75:            140000
76:             20000
77:               100
78:              8400
    TotalImputedValue
\end{Soutput}
\end{Schunk}
\begin{Schunk}
\begin{figure}

{\centering \includegraphics[width=\maxwidth]{figure/ImpactEstimation/Total_imputed_value_histogram-1} 

}

\caption{Total imputed value of livestock holding\\ {\footnotesize Livestock holding values are computed by using respective median prices of each year.\setlength{\baselineskip}{8pt}}}\label{Figure Total imputed value histogram}
\end{figure}
\end{Schunk}
\begin{Schunk}
\begin{figure}

{\centering \includegraphics[width=\maxwidth]{figure/ImpactEstimation/Histogram_of_livestock_holding_classes-1} 

}

\caption{Histogram of livestock holding classes\\ {\footnotesize Livestock holding values are computed by using respective median prices of each year.\setlength{\baselineskip}{8pt}}}\label{Figure Histogram of livestock holding classes}
\end{figure}
\end{Schunk}
\begin{Schunk}
\begin{figure}

{\centering \includegraphics[width=\maxwidth]{figure/ImpactEstimation/Histogram_of_livestock_holding_classes_by_year-1} 

}

\caption{Histogram of livestock holding classes by year\\ {\footnotesize Livestock holding values are computed by using respective median prices of each year.\setlength{\baselineskip}{8pt}}}\label{Figure Histogram of livestock holding classes by year}
\end{figure}
\end{Schunk}
\begin{itemize}
\vspace{1.0ex}\setlength{\itemsep}{1.0ex}\setlength{\baselineskip}{12pt}
\item	Why does \textsf{cow} report below 1000 holding in rds 2-4?
\end{itemize}
\begin{Schunk}
\begin{Soutput}
            Arm survey MeanImputedVal MeanNumCows   N
 1: traditional      1       16583.42    0.736559 186
 2: traditional      2        9508.64    0.436893 103
 3: traditional      3        6403.81    0.000000 103
 4: traditional      4       11844.02    0.549020 102
 5:       large      1       22288.39    1.038860 386
 6:       large      2       11530.35    0.561224 400
 7:       large      3        9759.15    0.000000 402
 8:       large      4       20170.58    0.969543 394
 9: large grace      1       18154.40    0.805128 390
10: large grace      2       10583.45    0.521053 391
11: large grace      3       10197.16    0.000000 397
12: large grace      4       15674.29    0.748705 389
13:         cow      1       14750.19    0.657303 356
14:         cow      2       11996.71    0.599432 363
15:         cow      3        9497.28    0.000000 367
16:         cow      4       18386.08    0.889831 355
\end{Soutput}
\end{Schunk}
\begin{Schunk}
\begin{figure}

{\centering \includegraphics[width=\maxwidth]{figure/ImpactEstimation/Number_of_cows_by_year-1} 

}

\caption{Number of cows/oxen by year\\ {\footnotesize Means are mean holding among the owners. Totals are total number of cows/oxen owned. Mean and total number of cows/oxen may diverge because the number of owners differ across round.\setlength{\baselineskip}{8pt}}}\label{Figure Number of cows by year}
\end{figure}
\end{Schunk}
\begin{palepinkleftbar}
\begin{finding}
\textsc{\small Figure \ref{Figure Total imputed value histogram}} shows general increase in upper holding classes round 3 and further upper holding classes in round 4. \textsc{\small Figure \ref{Figure Number of cows by year}} shows livestock type is not entered (yet collected) in rd3. At this moment, one needs to omit rd 3. All estimation results by far are subject to this omission.
\end{finding}
\end{palepinkleftbar}

\clearpage
\subsection{Incomes}



\hspace{-1cm}\begin{minipage}[t]{14cm}
\hfil\textsc{\normalsize Table \refstepcounter{table}\thetable: FD estimation of incomes\label{tab FD incomes}}\\
\setlength{\tabcolsep}{1pt}
\setlength{\baselineskip}{8pt}
\renewcommand{\arraystretch}{.55}
\hfil\begin{tikzpicture}
\node (tbl) {\input{c:/data/GUK/analysis/save/IncomesFDEstimationResults.tex}};
%\input{c:/dropbox/data/ramadan/save/tablecolortemplate.tex}
\end{tikzpicture}\\
\renewcommand{\arraystretch}{.8}
\setlength{\tabcolsep}{1pt}
\begin{tabular}{>{\hfill\scriptsize}p{1cm}<{}>{\hfill\scriptsize}p{.25cm}<{}>{\scriptsize}p{12cm}<{\hfill}}
Source:& \multicolumn{2}{l}{\scriptsize Estimated with GUK administrative and survey data.}\\
Notes: & 1. & First-difference estimates. Saving and repayment misses are taken from administrative data and merged with survey data at Year-Month of survey interviews. Intercept terms are omitted in estimating equations. Sample is continuing members and replacing members of early rejecters and received loans prior to 2015 Janunary. Labour income is sum of all earned labour incomes. Farm revenue is total of agricultural produce sales. \\
& 2. & ${}^{***}$, ${}^{**}$, ${}^{*}$ indicate statistical significance at 1\%, 5\%, 10\%, respetively. Standard errors are clustered at group (village) level.
\end{tabular}
\end{minipage}

\begin{palepinkleftbar}
\begin{finding}
\textsc{\small Table \ref{tab FD incomes}} shows a general decrease in rd 1 - 2 period and a general increase in rd 2 - 4 periods for labour incomes. \textsf{Large} arm saw a greater swing (decrease and increases) which resulted in overall significant mean increase of 10536.62 (at $p$ value of 0.42\%), yet not statistically different from \textsf{traditional}, while other arms are similar to \textsf{traditional}. Farm revenues do not show any systematic trend.
\end{finding}
\end{palepinkleftbar}


\subsection{Consumption}



Number of HHs with consumption before the loan is disbursed (\textsf{ConsumptionBaseline} == 1) is small.
\begin{Schunk}
\begin{Soutput}
                ConsumptionBaseline
Arm                0   1
  traditional    307   0
  large          876 321
  large grace    852 326
  cow            756 333
  forcedDropOuts   0   0
  dropOuts         0   0
\end{Soutput}
\end{Schunk}

\hspace{-1cm}\begin{minipage}[t]{14cm}
\hfil\textsc{\normalsize Table \refstepcounter{table}\thetable: FD estimation of consumption\label{tab FD consumption}}\\
\setlength{\tabcolsep}{1pt}
\setlength{\baselineskip}{8pt}
\renewcommand{\arraystretch}{.55}
\hfil\begin{tikzpicture}
\node (tbl) {\input{c:/data/GUK/analysis/save/consumptionFDEstimationResults.tex}};
%\input{c:/dropbox/data/ramadan/save/tablecolortemplate.tex}
\end{tikzpicture}\\
\renewcommand{\arraystretch}{.8}
\setlength{\tabcolsep}{1pt}
\begin{tabular}{>{\hfill\scriptsize}p{1cm}<{}>{\hfill\scriptsize}p{.25cm}<{}>{\scriptsize}p{12cm}<{\hfill}}
Source:& \multicolumn{2}{l}{\scriptsize Estimated with GUK administrative and survey data.}\\
Notes: & 1. & First-difference estimates. Saving and repayment misses are taken from administrative data and merged with survey data at Year-Month of survey interviews. Intercept terms are omitted in estimating equations. Sample is continuing members and replacing members of early rejecters and received loans prior to 2015 Janunary. Consumption is annualised values. \\
& 2. & ${}^{***}$, ${}^{**}$, ${}^{*}$ indicate statistical significance at 1\%, 5\%, 10\%, respetively. Standard errors are clustered at group (village) level.
\end{tabular}
\end{minipage}

\begin{palepinkleftbar}
\begin{finding}
\textsc{\small Table \ref{tab FD consumption}} uses rd 2 - 4 data and shows an increase in per member consumption in rd 2 - 3 period. The estimates are imprecise for all interaction terms. Per member food consumption increases in rd 2- 3 period but decreases in rd 3 - 4 period.
\end{finding}
\end{palepinkleftbar}


\subsection{IGA}


\begin{Schunk}
\begin{figure}

{\centering \includegraphics[width=\maxwidth]{figure/ImpactEstimation/IGA_choices-1} 

}

\caption{Income generatng activity choices\\ {\footnotesize The first income generating activity choices are plotted.\setlength{\baselineskip}{8pt}}}\label{Figure IGA choices}
\end{figure}
\end{Schunk}
\begin{Schunk}
\begin{figure}

{\centering \includegraphics[width=\maxwidth]{figure/ImpactEstimation/All_IGA_choices-1} 

}

\caption{All income generatng activity choices\\ {\footnotesize All of multiple investment choices are summed by arms and the number of IGAs and plotted as bars. \setlength{\baselineskip}{8pt}}}\label{Figure All IGA choices}
\end{figure}
\end{Schunk}
\begin{Schunk}
\begin{figure}

{\centering \includegraphics[width=\maxwidth]{figure/ImpactEstimation/All_IGA_choices_collapsed-1} 

}

\caption{All income generatng activity choices collapsed over different number of IGAs\\ {\footnotesize All of multiple investment choices are summed by arms and plotted as bars. \setlength{\baselineskip}{8pt}}}\label{Figure All IGA choices collapsed}
\end{figure}
\end{Schunk}

\begin{palepinkleftbar}
\begin{finding}
\textsc{\small Figure \ref{Figure IGA choices}, \ref{Figure All IGA choices}} show that there are very few members who chose to invest in more than one project for the ``large'' arms, while in the \textsf{traditional} arm, almost no one invested only in one project. Goat/sheep and small trades are the top choices for the first IGA in \textsf{traditional}. This indicates the exitence of both a liquidity constraint and convexity in the production technology of large domestic animals. This also validates our supposition that dairy livestock production is the most preferred and probably the only economically viable investment choice. It reduces a concern that the \textsf{cow} arm may have imposed an unnecessary restriction in an investment choice by forcing to receive a cow. \textsc{\small Figure \ref{Figure All IGA choices collapsed}} shows there are a significant number of cases in the \textsf{traditional} arm that members reporting to raise cows, yet they are also accompanied by pararell projects in smaller livestock production and small trades. Contrasting \textsf{large}, \textsf{large grace} with \textsf{cow} arms, it suggests that entrepreneurship (to the extent that is necessary for dairy livestock production) may not be an impediment for a microfinance loan uptake among members.
\end{finding}
\end{palepinkleftbar}

Together with \textsc{\small Table \ref{tab FD saving}} showing smaller net saving and repayment among \textsf{traditional}, the restriction on a project choice induced by a smaller loaned sum resulted in smaller returns. Between with or no grace period loans, cumulative net saving and repayment are both larger with loans with a grace period. No such difference is found between \textsf{cow} and other arms.

\subsection{Marriage}

\begin{Schunk}
\begin{Soutput}
                 TradGroup
creditstatus      planned twice double <NA>
  Yes                  29   197    133 2957
  No                    0     0      0  353
  Replaced Member       0     0      0    0
  <NA>                  0     0      0  493
\end{Soutput}
\begin{Soutput}
              Arm NumEligible.1 NumEligible.2 NumEligible.3 NumEligible.4
1:    traditional           125             0             0           211
2:          large           146             2             0           219
3:    large grace           163             1             2           243
4:            cow           161             0             0           251
5: forcedDropOuts            22             0             0            NA
6:       dropOuts            55             1             0            57
\end{Soutput}
\end{Schunk}
Tabulate marriage for \textsf{sex} == ``Female" \& \textsf{ReadyToMarry}, where the latter is unmarried females with ages between 10 and 40.

When we compare the marriage rates, we need to define the denominator sensibly. It should be all relevant aged females that are present in baseline. As we do not want to include marriages immediately after receiving loans, we need to take off some period to count the marriage cases. We will consider 1 year, 2 years, and 3 years. At the same time, there are househods who chose not to receive a loan. 
Then, we need the denominator to be relevant aged females who do not attrit by:
\begin{itemize}
\vspace{1.0ex}\setlength{\itemsep}{1.0ex}\setlength{\baselineskip}{12pt}
\item	1 year (678 individuals), or,
\item	2 years (453 individuals), or,
\item	3 years (314 individuals).
\end{itemize}
\begin{Schunk}
\begin{Soutput}
               Arm AttritedBefore NumEligible Married MarriageRate
 1:    traditional         year 1          72       7         0.10
 2:    traditional         year 2          26       3         0.12
 3:    traditional         year 3          86      12         0.14
 4:    traditional          never         152      16         0.11
 5:          large         year 1          14       4         0.29
 6:          large         year 2          31       3         0.10
 7:          large         year 3          79      11         0.14
 8:          large          never         243      26         0.11
 9:    large grace         year 1          21       6         0.29
10:    large grace         year 2          81      10         0.12
11:    large grace         year 3          90      11         0.12
12:    large grace          never         217      20         0.09
13:            cow         year 1          45       5         0.11
14:            cow         year 2          67       8         0.12
15:            cow         year 3         105      15         0.14
16:            cow          never         195      22         0.11
17: forcedDropOuts         year 1          22       4         0.18
18:       dropOuts         year 1         113      15         0.13
\end{Soutput}
\end{Schunk}
\begin{palepinkleftbar}
\begin{finding}
There is very small difference in marriage rates between arms with grace and without grace.
\end{finding}
\end{palepinkleftbar}

\end{document}
