%  path0 <- "c:/data/GUK/"; path <- paste0(path0, "analysis/"); setwd(pathprogram <- paste0(path, "program/")); system("recycle c:/data/GUK/analysis/program/cache/ImpactEstimation2020August21/"); library(knitr); knit("ImpactEstimation2020August21.rnw", "ImpactEstimation2020August21.tex"); system("platex ImpactEstimation2020August21"); system("pbibtex ImpactEstimation2020August21"); system("dvipdfmx ImpactEstimation2020August21")




\input{c:/seiro/settings/Rsetting/knitrPreamble/knitr_preamble.rnw}
\newcounter{armindex}
\renewcommand\Routcolor{\color{gray30}}
\newcommand{\SetLengthSkip}[1]{\setlength{\baselineskip}{#1}}
\newtheorem{finding}{Finding}[section]
\makeatletter
\g@addto@macro{\UrlBreaks}{\UrlOrds}
\newcommand\gobblepars{%
    \@ifnextchar\par%
        {\expandafter\gobblepars\@gobble}%
        {}}
\newenvironment{lightgrayleftbar}{%
  \def\FrameCommand{\textcolor{lightgray}{\vrule width 1zw} \hspace{10pt}}% 
  \MakeFramed {\advance\hsize-\width \FrameRestore}}%
{\endMakeFramed}
\newenvironment{palepinkleftbar}{%
  \def\FrameCommand{\textcolor{palepink}{\vrule width 1zw} \hspace{10pt}}% 
  \MakeFramed {\advance\hsize-\width \FrameRestore}}%
{\endMakeFramed}
\newcommand{\gettikzxy}[3]{%
  \tikz@scan@one@point\pgfutil@firstofone#1\relax
  \edef#2{\the\pgf@x}%
  \edef#3{\the\pgf@y}%
}
\def\BibTeX{{\textrm{B}\kern-.05em{\textsc{i}\kern-.025em \textsc{b}}\kern-.08em
    \textrm{T}\kern-.1667em\lower.7ex\hbox{\textrm{E}}\kern-.125em\textrm{X} }}
\def\BGColour{green!10}
\makeatother
\AtBeginDvi{\special{pdf:tounicode 90ms-RKSJ-UCS2}}
\special{papersize= 209.9mm, 297.04mm}
\usepackage{caption}
\usepackage{setspace}
\usepackage{framed}
\usepackage[framemethod=TikZ]{mdframed}
\captionsetup[figure]{font={stretch=.6}} 
\def\pgfsysdriver{pgfsys-dvipdfm.def}
\usepackage{tikz}
\usetikzlibrary{intersections, calc, arrows, decorations, decorations.pathreplacing, backgrounds, positioning, fit, shadows}
\usepackage{pgfplots, pgfplotstable}
\usepgfplotslibrary{fillbetween}
\pgfplotsset{compat=1.3}
\usepackage{adjustbox}
\tikzstyle{toprow} =
[
top color = gray!20, bottom color = gray!50, thick
]
\tikzstyle{maintable} =
[
top color = blue!1, bottom color = blue!20, draw = white
%top color = green!1, bottom color = green!20, draw = white
]
\tikzset{
%Define standard arrow tip
>=stealth',
%Define style for different line styles
help lines/.style={dashed, thick},
axis/.style={<->},
important line/.style={thick},
connection/.style={thick, dotted},
}
\mdfsetup{
linecolor=black!40,
outerlinewidth=1pt,
roundcorner=.5em,
innertopmargin=1ex,
innerbottommargin=.5\baselineskip,
innerrightmargin=1em,
innerleftmargin=1em,
backgroundcolor=blue!10,
%userdefinedwidth=1\textwidth,
shadow=true,
shadowsize=6,
shadowcolor=black!20,
frametitlebackgroundcolor=cyan!40,
frametitlerulewidth=10pt,
splittopskip=2\topsep
}
\global\mdfdefinestyle{SecItemize}{%
linecolor=black!40,
outerlinewidth=1pt,
roundcorner=1em,
innertopmargin=1ex,
innerbottommargin=.5\baselineskip,
innerrightmargin=1em,
innerleftmargin=1em,
backgroundcolor=blue!10,
%userdefinedwidth=1\textwidth,
shadow=true,
shadowsize=2,
shadowcolor=black!20,
frametitlebackgroundcolor=cyan!10,
frametitlerulewidth=10pt
}

\newcommand{\myquote}[1]{\begin{quotation}#1\end{quotation}}
\newcommand{\BGbox}[1]{\par\noindent\colorbox{lightblue}
{\parbox{\dimexpr\textwidth-2\fboxsep\relax}{#1}}}


\begin{document}
\setlength{\baselineskip}{12pt}

\hfil An escape from a poverty trap and the role of entrepreneurship:\\
\hfil Microfinance lending to the ultra poor in the Northern Bangladesh\\

\hfil\MonthDY\\
\hfil{\footnotesize\currenttime}\\

\hfil Seiro Ito\footnote{Corresponding author. IDE-JETRO. seiroi@gmail.com}, Takashi Kurosaki\footnote{Hitotsubashi University}, Abu Shonchoy\footnote{Florida International University}, Kazushi Takahashi\footnote{GRIPS}\\

\hfil\mpage{10cm}{\footnotesize
\textsc{\textbf{Abstract}} \hspace{1em} The existing microcredit programs rarely lend to the ultra poor people. With a randomised controlled trial in a rural, low income setting of northern Bangladesh, we assess the creditworthiness of the ultra poor and suitable debt contract design to help them invest and escape from poverty. We use a stepped-wedge design over the key features of loans, i.e., small-scale sequential disbursement vs. lumpy upfront disbursement, with vs. without a grace period, and cash vs. in-kind loan with a managerial support program. Compared with the traditional, Grameen-style microcredit, the provision of large, upfront liquidity increases both repayment rates and net asset levels. This is consistent with the existence of an asset-based poverty trap which can be overcome by increasing the loan size. Provision of a grace period does not change the repayment rates or asset levels. It is also shown that managerial supports induce participation of less experienced and poorer households to microfinance without affecting the repayment and asset accumulation. For all households, labour incomes become larger towards the end of loan cycle, which we interpret as evidence of repayment discipline. %Given the lack of alternative lenders in the area, we argue that the high repayment rates need not generalise to other contexts. 
Our main findings, a large, upfront disbursement results in faster asset accumulation that is suggestive of an escape from a poverty trap and managerial support programs induce the participation of the ultra poor, are generalisable to other rural areas that are suited to livestock production.
}

\newpage

\textbf{\textbf{Revisions}}

\vspace{2ex}
Title and abstract:
\begin{enumerate}
\vspace{1.0ex}\setlength{\itemsep}{1.0ex}\setlength{\baselineskip}{12pt}
\item	Dropped: Given the lack of alternative lenders in the area, we argue that the high repayment rates need not generalise to other contexts. 
\end{enumerate}

Introduction:
\begin{enumerate}
\vspace{1.0ex}\setlength{\itemsep}{1.0ex}\setlength{\baselineskip}{12pt}
\item	Still in need of a reference for the Prof Yunus' claim.
\end{enumerate}
Existing studies:
\begin{enumerate}
\vspace{1.0ex}\setlength{\itemsep}{1.0ex}\setlength{\baselineskip}{12pt}
\item	Dropped: Fortunately, our program ended with relatively low delinquency rates compared to other programs, and there is even suggestive evidence of repayment discipline among the borrowers. 
\item	Dropped more detailed description of results and changed to: \myquote{By complementing this result with the fact that borrowers purchase cattle only when large upfront liquidity is provided, we conclude that there is a poverty trap. }
\end{enumerate}
Study sample:
\begin{enumerate}
\vspace{1.0ex}\setlength{\itemsep}{1.0ex}\setlength{\baselineskip}{12pt}
\item	Added the whole section. Moved rejection contents from Experimental design section. Added attrition description.
\item	Describes sampling and summary statistics. 
\item	Needed: Statistical, not descriptional as in Background section, information about chars.
\item	Added: \myquote{Using a Landsat imagery, we identified 128 \textit{char}s within a day boat ride from the Gaibandha peer and collected information by filed visits. From this list of \textit{char}s, we randomly selected 80 \textit{char}s.}
\item	Clarified the population of our study:
\myquote{In each village, we created a census of household wealth ranking by a participatory ranking process. Following a process similar to the paired ranking as in \citet[][p.1212]{Alatas2012} and the Peruvian ultra poor case of \citet[][p.66]{KarlanThuysbaert2019}, we asked the least wealthy households in terms of asset ownership. We then asked to form a member committee of 10 households, of which 6 are ultra poor and 4 are moderately poor. As we admitted households on a first come, first served basis, these 10 households are the first to join the membership of microfinance in respective poverty classes.}
At the last paragraph of the section:
\myquote{As we track all --- barring the flood victims whose villages were washed away and other attriters --- the potential borrowers including who eventually opted out the borrowing, we are able to estimate the intention-to-treat effects of offering loans with different feartures on the population who showed interests in joining microfinance membership. }
\item	Added: \textsc{\small Table \ref{tab DestatMainByArm}} (descriptive statistics of sample households) and its description. \myquote{\textsc{\small Table \ref{tab DestatMainByArm}} shows descriptive statistics of sample households. As we randomly allocate them into four different arms, \textsf{traditional}, \textsf{large}, \textsf{large grace}, and \textsf{cattle}, summary is shown by the arms. As shown in the Appendix, these baseline household characteristics does not differ statistically between the arms. Our sample is characteretised by relatively low literacy rate (\textsf{HeadLiteracy}) and relatively young age (\textsf{HeadAge}) of the household heads. Literacy rate is lower than the national average of adult males at 61.54\% in 2012 (UNESCO). Household size (\textsf{HHsize}) is not large, around 4.1-4.3 members, due probably to the constant flood threats, as indicated by above 40\% exposure at the baseline (\textsf{FloodInRd1}), that do not easily allow a large household formation. Cattle holding per household (\textsf{NumCows}) shows cattle rearing is not common and the mean herd size is between .2 to .3. Mean net asset values per housheold (\textsf{NetValue}) differ between the arms, but they mostly reflect sampling errors as indicated by the large standard deviations. \textsf{Attrited} indicates attrition rates in the household survey, and \textsf{GRejected} and \textsf{IRejected} show group rejection rates and individual rejection rates to the lending program. We will analyse attrition and rejection later in Section \ref{ResultsSectionParticipation}, but at this point, we just note that the attrition rates are not statistically different between the arms. \textsf{Non-attriting borrowers} indicates the non-attrited borrower ratios. Because there are more rejecters in the \textsf{traditional} arm, this ratio is smaller than in other arms. }
\item	Added to clarify the difference between rejection and attrition: \myquote{While loan rejecters remained in our sample, we lost four groups to floods in 2013. As they relocated, we had no choice but to drop them from the study. In our study, attrition refers to a household drop out from our survey. Rejection refers to a loan rejection in our intervention, and majority of rejecters (81.25\%) did not attrit from our household survey.} Also in the \textsc{\normalsize Table \ref{tab DestatMainByArm}} footnote: \myquote{Because attrition and rejection are separate events, a household can reject and attrit, so active members $\geqslant$ total - (rejected members + attrited members).}
\end{enumerate}
Experimental design:
\begin{enumerate}
\vspace{1.0ex}\setlength{\itemsep}{1.0ex}\setlength{\baselineskip}{12pt}
\item	Moved to Study sample section: Rejection contents.
\end{enumerate} 
Results: 
\begin{enumerate}
\vspace{1.0ex}\setlength{\itemsep}{1.0ex}\setlength{\baselineskip}{12pt}
\item	Participation and attrition are separated as two subsections.
\item	Added to clarify the difference between rejection and attrition: \myquote{We analyse nonparticipation in relation to the debt contract design that they were randomly allocated to. }
\item	Added: \myquote{Group rejecters of non-\textsf{traditional} arms did not participate because of negative asset shocks. }
\item	Consolidated 3 tables in the appendix to \textsc{\normalsize Table \ref{tab MainTextIRjecters}}, consolidated 2 tables in the appendix to \textsc{\normalsize Table \ref{tab main cownoncow}}. In doing so, changed to:
\myquote{	In \textsc{\small Table \ref{tab main cownoncow}}, we compare if the \textsf{cattle} arm participants differ from participants in other arms at the baseline. In the left panel, we compare participants (borrowers). It is worth noting that partcipants of \textsf{cattle} arm differ from other arms in having less cattle rearing experience as observed in smaller initial cattle holding ($p$ value = .156) and in having lower asset values ($p$ value = .058). These features that are plausibly disadvantageous in rearing a heifer notwithstanding, the \textsf{cattle} arm with training induced partcipation. As we will see in Section \ref{Sec Impacts}, the choice of lending vehicle (cash or in-kind) does not matter in investments. So it is natural to infer that the training component has induced the members with less experiences and fewer assets to take up loans. In the right panel, we compare \textsf{cattle} arm borrowers who did not attrit by the end of final survey round with those in the other arms. At the baseline, \textsf{cattle} arm non-attriting borrowers have smaller baseline livestock holding ($p$ value = .016) and smaller baseline net asset holding ($p$ value = .007) than other arms' non-attriting borrowers. These hint that more disadvantaged borrowers participated and managed to stay on the survey until the end of the study in the \textsf{cattle} arm with a help of managerial supports.}
\item	Attrition subsection only refers to the appendix, not to any of its tables.
\item	Changed to: \myquote{These features that are plausibly disadvantageous in rearing a heifer notwithstanding, the \textsf{cattle} arm with training induced partcipation. }
\item	Added: \myquote{Contents of IGAs are cattle, goat/sheep, growing cereals (paddy, corn) and nuts, small trades, and house and land leasing.}
\item	Reorganised: Previously, Figure 7 was missing. In addition to impact estimates in error bars (\textsc{\footnotesize Figure \ref{fig NetAssetsLivestockEffects}}, \textsc{\footnotesize Figure \ref{fig IncomeConsumptionEffects}}, \textsc{\footnotesize Figure \ref{fig SchoolingEffects}}), we show IGA contents with \textsc{\footnotesize Cattle holding by arm (Figure \ref{fig CattleHoldingArm})}, \textsc{\footnotesize All IGAs (Figure \ref{fig AllIGAChoices})}, \textsc{\footnotesize Contents of first IGA (Figure \ref{fig FirstIGAChoicesCattleVsOther})}. 
\item	Changed cattle holding description: 
\myquote{	\textsc{\footnotesize Figure \ref{fig CattleHoldingArm}} shows more detailed changes in cattle ownership by arm. Holder rates are the number of cattle owners per arm size, holding size is average holding per owner, initial owner holding are herd size for owners who held cattle at baseline, and per capita holding is mean cattle holding in each arm. Initial owner holding and holder rates show impacts on the intensive and extensive margins, respectively. Per capita holding tracks impacts on both the intensive margins (growth of initial owners) and the extensive margins (growth of new owners). All the indicators are similar across arms at the baseline. }
\myquote{We see that the holder rates increased in all arms, but the increase was smallest for the \textsf{traditional}. This shows that, even the small upfront lending of \textsf{traditional} arm helped increase catte ownership but to a lesser degree. Without equally large upfront liquidity and with the repayment installments that began immediately, a smaller fraction of borrowers could purchase their first cattle. Holding size increased in all non-\textsf{traditional} arms, while the \textsf{traditional} arm remained stagnant. It is also the \textsf{traditional} arm that has the smallest, or negligible, impacts on the initial owners. \textcolor{red}{[These initial owners, overall, diversified their portfolio rather than increasing the cattle investments. ]} Initial owner holding size is larger than the average holding size per owner for the non-\textsf{traditional} arm, indicating the higher returns to members with experiences. The per capita holding growth was smallest in the \textsf{traditional} arm. This is due to smaller impacts on the extensive margins (fewer new ownership, smaller growth by new owners) and little impacts on the intensive margins (negligible growth by initial owners). 
}
\item	Added: \myquote{	\textsc{\footnotesize Figure \ref{fig FirstIGAChoicesCattleVsOther}} shows the first IGAs of members. The first IGA is defined as the oldest IGA for the household. For most of the households, the oldest IGA had started after the baseline, and it is the IGA with the largest cash flow. Of course, there are a small percentage of households with an existing IGA before the baseline, but, with randomisation, the fraction of such households are similar across arms. Therefore, the between arm comparison of the first IGA gives us an idea about how the households had chosen the initial investments. In the \textsf{traditional} arm, there are 33 borrowing members who report cattle as their first IGA, and 76 borrowing members (69.72\%) who report other than cattle as their first IGA. This contrasts with the non-\textsf{traditional} arms that 466 borrowing members who report cattle as their first IGA and 25 borrowing members (5.09\%) other than cattle as their first IGA. As can be seen from \textsc{\small Figure \ref{fig AllIGAChoices}}, for the 2nd and 3rd IGAs, a diversified IGA portfolio is continued to be held by all the \textsf{traditional} arm borrowers, and only the minority of non-\textsf{traditional} arm borrowers has a diversified portfolio.}
\item	Added: Figure of mean consumption and labour incomes for all arms by period in the appendix and description in the main text. Given labour incomes and consumption are similar across arms, even the traditional arm borrowers seem to have tried to repay, only that they managed to do so at a lower rate.
\myquote{		In the appendix, \textsc{\footnotesize Figure \ref{fig MeanOutcomes}} shows mean outcomes for per capita consumption and labour incomes. Labour income is increasing from period 3. Despite growths in labour incomes, in periods 3 and 4, per capita consumption did not change. As labour income growths and steady consumption are uniformly observed, it is accrued to the loan and its repayment, not to the functional attributes. The households seem to have put asset accumulation and repayment a priority before consumption growths. It indicates that the borrowers did not choose to strategically default but tried to repay. 
}
\item	Added: \myquote{The low repayment rates among \textsf{traditional} arm borrowers may be due to our experimental design that a new loan is disbursed unconditionally up to three cycles, lacking the dynamic incentives to repay, or due to the fact that they had lower returns on their investments. Our finding of labour income growths and the steady consumption indicates the latter possibility. } 
\item	Added: \myquote{We also observe that impacts on all outcome measures are not statistically different between the poverty classes (see Appendix \ref{AppendixEstimationTables}). }
\item	Added: \myquote{In summary, we found that our managerial support programs induce the members of disadvantaged background to participate in microfinance, achieving the further outreach with the impacts that are no different with other borrowers. This is consistent with the finding of the previous studies that a certain level of skills is necessary for participation, and our managerial support progams supplemented the lack thereof. We observed that the upfront loan disbursements have allowed borrowers to invest in cattle while members with sequential disbursements mostly opted for smaller livestock and small business trades. In combination with a greater return to cattle on net asset accumulation and a greater rate of loan repayment, we consider it as evidence of a poverty trap and an effective measure to break it. We also found the impacts and repayment rates are indistinguishable between the moderately poor and the ultra poor.}
\end{enumerate}
Conclusion:
\begin{enumerate}
\vspace{1.0ex}\setlength{\itemsep}{1.0ex}\setlength{\baselineskip}{12pt}
\item	Added on external validity and a lesson on nonconvexity:
\myquote{	We note that our study site is rich in rainfall, giving more advantages to cattle production over sheep/goat production. In contrast, if the climate is more arid, sheep and goats are better suited because of less water logging and their greater viability in relying on natural grass. This raises a concern that our results may not directly transferrable to more arid areas. However, the greater point of the lesson from the study is the presence of fixed inputs in scaling the herd size. While sheep/goats are easier to scale than cattle, it will require larger land and roofed facilities at some point. This can effectively form nonconvexity in the production set, and large enough lending may allow herders to go pass the threshold. 
}
\end{enumerate}
Appendix:
\begin{enumerate}
\vspace{1.0ex}\setlength{\itemsep}{1.0ex}\setlength{\baselineskip}{12pt}
\item	Keep only impact estimation tables referred in the main text.
\end{enumerate}

\textbf{\textbf{Contrasts with \citet{Balboni2020}}}
\setlength{\parindent}{1em}

	\citet{Balboni2020} collect data from transfer recipients and control group of BRAC's TUP. Using the recipient data, they estimate the equation of motion $K_{t+1}=\phi(K_{t})$, show the S shape, and compute the threshold asset level $\hat{k}$ that separates the low and high equilibria. They then show that individuals who are above $\hat{k}$ increase the assets while individuals below it decrease them. The variations of initial asset level allows the identification of bifurcation as these variations effectively allocate inidividuals to below and above the threshold. Anticipating the endogeneity of initial asset levels to asset growths, they show that initial asset levels have no correlation with post intervention asset growths after conditioning on the above-threshold dummy. This is suggestive evidence that the unobservables that correlate with initial asset levels are exogenous to post-transfer asset increases.

	In the poverty trap dynamics, the key is the low returns among the low $k$ holders. The returns to high $k$ holders are qualitatively similar in the covex and concave production functions so long as they are above the 45 degree line. Why are they low? Authors show the baseline vehicle ownership is statistically smaller by 4\% (but not for other assets, total assets are not tested) for the below threshold households, and the differences relative to the above threshold households grew after the transfer receipt. They conclude that the complimentary assets serve as the fixed inputs of production, and the lack thereof withholds households from escaping the poverty trap.

\begin{description}
\vspace{1.0ex}\setlength{\itemsep}{1.0ex}\setlength{\baselineskip}{12pt}
\item[their strength]	Large sample size, precise asset and labour data, direct estimation of equation of motion, and associated tests of multiple equilibria.
\item[our strength]	Experimental variations in contract design (\textsf{Upfront}, support programs), use of IGA information that allows the (poverty trap) interpretations without structural estimation, reference to market costs/prices. 
\end{description}

\noindent
\textbf{\textbf{Contrasts with \citet{Banerjee2019MFPovertyTrap}}}

	\citet{Banerjee2019MFPovertyTrap} use regionally matched-pair data under staggered branch opening of an urban MFI. They divide the sample into borrowers with a prior business experience (GE) and others (non-GE), and show that impacts are persistently positive for GE borrowers but not for non-GE borrowers. With structural estimation, given a talent distribution, they interpret this as evidence of a poverty trap through a liquidity constraint. They also note the impact heterogeneity is due to MFI selection but not talent heterogeneity, as pre-MFI entry businesses are more profitable than post-MFI entry businesses of the same firm age. 

\begin{description}
\vspace{1.0ex}\setlength{\itemsep}{1.0ex}\setlength{\baselineskip}{12pt}
\item[their strength]	Urban setting, contrast of long-run versus short-run impacts, data on business outcomes, gross substitute/compliment with demand for informal loans, explaining the lack of average impacts by finding the subpopulation with superior talents and contrasting with other subpopulation.
\item[our strength]	Ultra poor population, rural and fragile setting, selection on entrepreneurship without affecting outcomes, \textsf{Upfront} leads to faster asset accumulation and higher repayment rates.
\end{description}

\noindent
\textbf{What the three papers agree}

\begin{itemize}
\vspace{1.0ex}\setlength{\itemsep}{1.0ex}\setlength{\baselineskip}{12pt}
\item	A need for larger lending than regular MF.
\item	Existence of a poverty trap.
\item	Evidence of a nonconvex production set as a source of poverty trap.
\end{itemize}

\newpage
\setcounter{tocdepth}{3}
\tableofcontents
\newpage

\setlength{\parindent}{1em}
\vspace{2ex}









\renewcommand{\thefootnote}{*\arabic{footnote}}
\setcounter{footnote}{0}

%\section{Introduction 1}

% \begin{mdframed}[style={SecItemize}, frametitle={Introduction}]
% \begin{itemize}
% \vspace{1.0ex}\setlength{\itemsep}{1.0ex}\setlength{\baselineskip}{12pt}
% \item	Credit outreach to the ultra poor is slower than the moderately poor.
% 	\begin{itemize}
% 	\vspace{1.0ex}\setlength{\itemsep}{1.0ex}\setlength{\baselineskip}{12pt}
% 	\item	Demand side: The ultra poor lack entrepreneurship, access to more efficient production possibility given the small credit size. 
% 	\item	Supply side: The ultra poor may be riskier, its loan size is too small to justify the fixed costs.
% 	\end{itemize}
% \item	We supply credits to test the demand side constraints.
% \item	Test if entrepreneurship is a constraint.
% 	\begin{itemize}
% 	\vspace{1.0ex}\setlength{\itemsep}{1.0ex}\setlength{\baselineskip}{12pt}
% 	\item	Entrepreneurship: Packaged loan vs. cash loan.
% 	\item	In comparing with classic Grameen style loans, Large and LargeGrace loans are also offered, forming a stepped wedge design in \textsf{Upfront}, \textsf{WithGrace}, and \textsf{InKind} attributes.
% 	\end{itemize}
% \item	Test if nonconvexity is a constraint.
% 	\begin{itemize}
% 	\vspace{1.0ex}\setlength{\itemsep}{1.0ex}\setlength{\baselineskip}{12pt}
% 	\item	Test the existence of a poverty trap: Exists if the upfront attribute exhibits a faster asset accumulation at a no smaller repayment rate. A poverty trap is casued, in theory, production set nonconvexity, in practice, heifer-cattle production set nonconvexity.
% 	\end{itemize}
% \end{itemize}
% \end{mdframed}

%knit_child(paste0(pathprogram, "Intro1.rnw"))

%	In the following section, we summarise the existing literature. Section \ref{SecExistingStudies} gives the link to the previous literature. Section \ref{SecBackground} gives the brief account of background of study site. Section \ref{SecExperimentalDesign} lays out the details of experimental design. Section \ref{SecEmpiricalStrategy} explains the estimation strategy. In section \ref{SecResults}, we provide a brief overview of the experimental results. Section \ref{SecTheory} shows a possible mechanism of poverty trap that our target population is under. Section \ref{SecConclusion} discusses the interpretation of results.

\section{Introduction}


	Since the microcredit became popular in Bangladesh in the late 1980's, the number of borrowers increased rapidly throughout the world. According to over 3700 microfinance institutions (MFIs), there are estimated 204 million borrowers around the world in 2013, of which 110 million are the poor borrowers whose incomes are below the national poverty line \citep{MFGateway2015}. The outreach to the extremely poor population or the \textit{ultra poor}, however, is arguably slow in comparison.\footnote{MF is not successful in reaching out to the poorest of the poor, or the ultra poor \citep[][]{Scully2004}. Empirical evidence in \citet{Yaron1994, Navajas2000, RahmanRazzaque2000, AghionMorduch2007} supports this claim. Some authors discuss the tradeoff between sustainability and outreach for microfinance institutions (MFIs) \citet{HermesLensink2011, HermesLensinkMeesters2011, Cull2011}. } 

	There are demand and supply side reasons behind the slow outreach to the ultra poor. On the demand side, the ultra poor borrowers may not be entrepreneurial enough to demand credits for production, or may face an inferior production possibility than the wealthier borrowers. On the supply side, MFIs may perceive the ultra poor as riskier than the moderately poor, or their loan size may be too small to justify the fixed transaction costs while the lender is constrained to keep the interest rate low to avoid adverse selection and moral hazard. 

	As the rigorous evaluations of microfinance progress, it has become clear that the impacts are not uniformly positive. A group of influential research has shown that only a subgroup of borrowers, those with prior experiences or high ability, have positive returns from borrowing \citep{Banerjee2015Miracle, Mckenzie2017Spurring, Buera2017, Banerjee2019MFPovertyTrap}. This is in a stark contrast to the popular belief in microfinance that anyone can become a successful borrower.\footnote{A leading proponent is the nobel laureate Professor Mohammad Yunus who claims that ``we are all entrepreneurs.'' \textcolor{red}{[Abu-san: Can you get a reference for this in the \BibTeX format? I have a 2017 Guardian article quoting \href{we are all entrepreneurs}{https://www.theguardian.com/sustainable-business/2017/mar/29/we-are-all-entrepreneurs-muhammad-yunus-on-changing-the-world-one-microloan-at-a-time}]}} Logically, there must be some minimal level of entrepreneurship to participate and continue as a borrower in any form of finance. Then, the question is, what sort or how much of entrepreneurship is required in microfinance? 
	
	To shed light on the required level of entrepreneurship, we took this question to the Northern Bangladesh where a flood threat limits the leading production process to be least complex: livestock rearing. There are residents who own livestock so its know-how is semi-public knowledge. The required entrepreneurship, then, is to gather all the pieces of information, decide to raise livestock, form a production and a sales plan, and implement. This is the definition of entrepreneurship we use in our paper. In our intervention, we provided a heifer to a selected subgroup of participants as an in-kind loan and bundled it with training and consultation services to make sure the borrower has the right cookbook to follow. Under this treatment, the entrepreneurship to decide what to invest and how to come up with a solid plan is no longer a necessity.

	In our study, we compare the borrowers who were provided with such knowledge and the managerial supports with the borrowers who were not. By randomising the offers, we identify the causal impacts of not requiring the entrepreneurship on both the participation and the outcomes. We find that our managerial support program, that renders entrepreneurship redundant, induces residents with fewer experiences in livestock production and a lower asset level to participate while keeping the mean outcomes the same as in the comparison group.%, despite the lower qualification. 
	
	There is yet another motivation for our study. In bridging these two arms that are different in multiple aspects, we introduced intermediate arms. At the end, we were left with an arm of conventional microcredit that disburses small upfront liquidity for three times, and several arms with large upfront liquidity that disburse the equivalent total amount once under three period maturity. This gives an opportunity to test if frontloading the liquidity, while keeping the total loan size and maturity equivalent, matters in the future asset levels. If the production technology is nonconvex and if there is a credit constraint, it gives rise to a poverty trap which can result in larger asset accumulation when provided with large upfront liquidity. While we do not directly test for a poverty trap, the investment choices strongly indicate its existence: Only borrowers without a frontloaded loan opted for smaller, multiple investments. Our experimental design tests if frontloading the liquidity breaks a poverty trap, under the assumption that there is one, and found that upfront liquidity results in larger asset accumulation without affecting the repayment rates.

	Our study follows the literature of microfinance debt contract design as hallmarked in \citet{Field2013} who found a grace period induces more risk taking and subsequent loan delinquency. Under our setting of limited production choices, it is irrational to invest in riskier assets, such as goats, when the designed grace period suits the heifer cash flow and a heifer's risk-return profile is considered to be Pareto-dominating. A strategic default is also more difficult in our setting because the number of formal credit suppliers is limited, which is probably zero,\footnote{As we surveyed the area before the study, we note several NGOs provide a relief credit to flood victims, but not regular finance. In selecting the study site, we purposefully chose the population without access to any financial institution.\textcolor{red}{[Abu-san: A better description for this?]} } and relocation requires a costly boat ride and financially reliable mainland contacts. The repayment rates in our study turned out to be no lower than the comparable microfinance schemes \citep{BanerjeeKarlanZinman2015}.

	Our study is closely related to a large scale cattle transfer study conducted in the neighbouring area \citep{BandieraBRAC2017, Balboni2020}. The targeted population of their study is similar to ours, yet our study population resides on less stable terrain, are more exposed to flood and water logging, are considered to be less well connected to the market, are equally less trained, and are probably poorer. The chance of survival for each investment project is expected to be no higher. More prominently, our study is commercially oriented: It uses a loan than a transfer, and charge market level fees to all the services provided. 
	
	We consider our finding is generalisable to rural areas where small scale livestock production is prevalent. While there is a caveat that the domain of our results is a low level herd size and the entrepreneurial capacity to hold a larger herd size can be different from what our study suggests,\footnote{Our study matches the scale of lower equilibrium of \citet{Lybbertetal2004} which is much smaller than the scale of the high equilibrium of around 50 herd size. } the successful livestock transfer program in the neighbouring areas \citep{BandieraBRAC2017, Balboni2020} and our results indicate that supporting asset accumulation through large livestock has wide applicability in assisting the rural ultra poor to escape from poverty.


	In the following section, we summarise the existing literature. Section \ref{SecExistingStudies} gives the link to the previous literature. Section \ref{SecBackground} gives the brief account of background of study site. Section \ref{SecExperimentalDesign} lays out the details of experimental design. Section \ref{SecEmpiricalStrategy} explains the estimation strategy. In section \ref{SecResults}, we provide the experimental results and contents of income generating activities (IGAs). Section \ref{SecTheory} shows a possible mechanism of poverty trap that our target population is under. Section \ref{SecConclusion} discusses the interpretation of results.

\section{A brief review of existing studies}
\label{SecExistingStudies}

% \begin{mdframed}[style={SecItemize}, frametitle={Existing studies}]
% \begin{itemize}
% \vspace{1.0ex}\setlength{\itemsep}{1.0ex}\setlength{\baselineskip}{12pt}
% \item	A relatively high uptake rate (among members) of our study poses less of the statistical power issue that plagues the benchmark study of \citet{BanerjeeKarlanZinman2015}
% \item	Heterogenous impacts of microcredits: Experiences/skills matter. Our study shows that skills do not matter for the impacts on the extensive margins.
% \item	Mixed and weak impacts of MFI training programs: Entrepreneurial skills are not trained easily, implying entrepreneurial skills, if required, may have to be outsourced in production processes. Our study shows entrepreneurial skills may not be required for production at this micro level of production.
% \item	Grace period: Our study is marked to actual cash flow profile, thereby easing the term mismaych, which explains the reduction of defaults
% \item	Lending suffices: We also observe sustained asset level increase as in asset transfer programs
% \end{itemize}
% \end{mdframed}

	There are four aspects in our study that relate to the existing literature: The role of entrepreneurship in microfinance impacts, variations in debt contract design, empirical assessment of a poverty trap, and targeting the ultra poor. We will discuss these in turn.

	Much has been discussed about the poverty reduction impacts of microfinance in the early days of microfinance studies \citep{Morduch1999}. Recently, doubts are cast on the magnitude of microfinance impacts \citep{BanerjeeKarlanZinman2015, DuvendackMader2019, Meager2019} while asset grants (capital injection) remain to show high returns \citep{deMel2008, DeMel2014, FafchampsFlypaper2014, BandieraBRAC2017}. \footnote{This is due partly to insufficient statistical power \citep{MckenzieWoodruff2013}. \citet{BanerjeeKarlanZinman2015} collects five studies of microfinance lending impacts. They also point the lack of statistical power due to low take up while noting more able and experienced borrowers saw larger ``transformative effects.'' %However, one must be careful with a low statistical power study as it naturally gives a way to erroneously large impacts. 
	In the current study, in contrast, the take up rate is relatively high at 74.32\%, of which 5.16\% is lost to the flood.  } Lack of mean impacts led researchers to look for a particular subgroup which shows impacts, or impact heterogeneity \citep{Banerjee2017HyderabadFollowup}: Borrowers with prior experiences or high ability are shown to have higher returns \citep{Banerjee2015Miracle, Mckenzie2017Spurring, Buera2017, Banerjee2019MFPovertyTrap}. The studies with a focus on experienced members or existing firms can be considered as looking at impacts on the intensive margins. In contrast, our study is focused on an isolated greenfield population, or poverty impacts on the extensive margins, which are relatively less studied.

	The fact that experienced members gain larger benefits from microcredit is consistent with the positive impacts of capital grant programs on existing firm owners. Whether such experience is trainable for novice entrepreneurs remains unsettled. A recent microfinance study indicates that there is an advantageous selection through talents in the existing firm owners, so trainability is called into a question \citep{Banerjee2019MFPovertyTrap}. A growing body of management capital literature in developing countries is insightful yet most of the research is necessarily geared to existing firms, so it does not inform much on how one can assist novice entrepreneurs.\footnote{\citet{BruhnKarlanSchoar2018} shows intensive management consulting services to the small scale firms in Mexico resulted in sustained improvements in management practices which led to higher TFP and larger employment. Others also show effectiveness \citep{Calderon2011, Berge2012, Bloometal2013} while others do not \citep{Bruhn2012, KarlanKnightUdry2015}. \citet{MckenzieWoodruff2013} put them as: These managerial impacts studies are too different to compare, in terms of population, interventions, measurement (variables, timing), and most importantly, implied statistical power in the design. } \citet{KarlanValvidia2011, BruhnZia2011, Argent2014} are the exceptions, but results and quality of evidence are mixed and inconclusive. The current study explicitly tests if the entrepreneurship matters in microfinance by using a heifer lending with a manegerial support program. We also examine the self-selection on entrepreneurship into microcredit, which we find to exist.%Entrepreneurship and training components in the current study are to provide basic knowledge of dairy cattle production which can easily be written down. They, the cristalised intelligence, are outsourceable in nature. We consider it is the skills to deploy services in a timely manner, rather than the knowledge contents \textit{per se}, that we provide to help borrowers in increasing efficiency.

	The corporate finance devotes a substantial part of its field in understanding the consequences of contract designs on entrepreneur's incentives. \citet{Field2013} was the first to examine if the traditional lending style of microfinance inhibits the spawning of entrepreneurship by experimentally allocating different types of debt contracts.  %shows that a two-month grace period increases the investment size, raises profitability, but also increases the default rates. They discuss how it influences the investment riskiness that varies along risk preference heterogeneity. 	Compared to them, the experimental setting of the current study has a smaller investment opportunities that limits the scope of risk taking. 
	As we will disscuss in the Section \ref{SecExperimentalDesign}, our study follows the similar strategy. In an attempt to tease out the impacts of entrepreneurship, we introduced longer maturity and a grace period in other arms. While there was a strong concern among practitioners that a grace period induces untruthful borrowing, there was no alternative in borrowing other than relatives and money lenders due to ruralness and isolation. This gave us flexibility in designing the debt contracts. Similar to \citet{Beaman2015} who redesigned the repayment schedule to adapt the borrower's cash flow profile (repay after harvest), we designed the debt contract to best suit the cash flow profile of the most popular investment project in the area, rearing a heifer. %Fortunately, our program ended with relatively low delinquency rates compared to other programs, and there is even suggestive evidence of repayment discipline among the borrowers. 
	Our study exemplifies the economic gains from designing the debt contract to match the presumed investment choices in microfinance.

	Another strand of the literature related to our study links capital grant effectiveness with the production set nonconvexity. Theories base lumpiness and credit market imperfection as keys to a povety trap \citep[e.g., ][]{GalorZeira1993}. When the production set is convex, a small scale transfer may not lead to a sustained increase in income, as it can be either consumed or invested to a technology with decreasing marginal returns that brings back to the original income level (i.e., the lower equilibrium of a poverty trap). 
	
	Despite its popularity as a theory, the empirical evidence of a poverty trap is mixed. \citet{KraayMcKenzie2014} note that a poverty trap finding is rare, while \citet{BarrettGargMcBride2016} state the otherwise and there is overwhelming evidence.\footnote{\citet{KraayMcKenzie2014} also note that upward transition from one poverty trap to another may negate the notion of a trap, while \citet{BarrettGargMcBride2016} base their affirmation by counting both the direct asset dynamics and the indirect inference that tests the behavioral responses that are consistent with poverty traps. See also \citet{CarterBarrett2006, BarrettCarter2013} for earlier evidence and discussions. } Interestingly, however, they both agree that, when there is a range of assets and production opportunities, it is inherently difficult to emipirically single out a particular poverty trap, and that existing evidence comes mostly from remote and isolated areas with a single primary production opportunity and an associated asset. Our study is no exception. It comes from a remote and isolated area of the Northern Bangladesh where the single most important production opportunity is livestock production.  
	
	An earlier finding of a poverty trap includes the cattle herd size dynamics of Southern Ethiopian pastoralists that indicates existence of a poverty trap over a 17 year recall period \citep{Lybbertetal2004}. More recently, \citet{Balboni2020} estimates the equation of motion for assets and show the direct evidence of a poverty trap among the recipients of a large scale transfer program targeted in the neighbouring areas of our study site. The source of nonconvexity is cattle and the complimentary assets (vehicles) at the baseline which serve as a fixed input that the ultra poor cannot afford. Similar to these studies, our study examines the nonconvexity of a higher-return production set. Our study regresses the future asset values on the current asset values and intervention dummies, thereby adding evidence, in the \citet{BarrettGargMcBride2016}'s terminology, using the direct method. %The estimation shows elevated returns to frontloading, which is consistent with a poverty trap but does not constitute as direct evidence. So we turn to investment choices and find that the large upfront liquidity is mostly associated with a heifer purchase while the small upfront liquidity is associated with smaller livestock purchase and other small business trading which are known to have lower returns and higher risks, and only at a later point in time they acquire a heifer. 
	By complementing this estimated result with the fact that borrowers purchase cattle only when large upfront liquidity is provided, we conclude that there is a poverty trap. In our study, the source of nonconvexity is the price of a heifer that is about three times the price of a goat. We also show that frontloading the liquidity in lending is effective in escaping the poverty trap.

	Lastly, selecting the ultra poor as the population to provide supports have often involved free consultation/training and transfers in the past. A few studies of ultra poor transfer programs report sustained increase in assets and incomes \citep{Blattman2014, BanerjeeetalScience2015, Blattmanetal2016, HaushoferShapiro2016}. A transfer program in the Northern Bangladesh shows an occupational change and an income increase \citep{BandieraBRAC2017} and long-run asset accumulation \citep{Balboni2020}. %\citet{Kaboski2018Indivisibility} uses a lab-in-the-field experiment to show the link between investment indivisibility, saving, and patience. 
	In an attempt to test commecial viability, our experimental design makes a reference to markets. It uses loans rather than transfers, and any training and consulting components charge a fee for services. The resulting repayment rates are not lower than the majority of representative microfinance programs, and we also find significant accumulation of assets. Our study can be considered as an example of market based interventions that can play a role in ultra poor programs.


\section{Background}
\label{SecBackground}

% \begin{mdframed}[style={SecItemize}, frametitle={Background}]
% \begin{itemize}
% \vspace{1.0ex}\setlength{\itemsep}{1.0ex}\setlength{\baselineskip}{12pt}
% \item	Lowest income area with high annual flood risks
% \item	No NGO/MFI presence
% \item	Argue: Cattle $\succcurlyeq$ goat in risk-return if invested
% \item	But: higher inputs and upfront fixed costs
% \item	Goats: Can have higher returns (in developed countries, smaller inputs and higher fertility), but worse in mortality/morbidity risks, particularly to water logging
% \item	Goat cash flow: Meat demand or kid sales require relatively high incomes, is infrequent
% \item	Cows inputs: Vaccination, fodder
% \item	Cow cash flow: Milk sales is more frequent, a calf sales requires even higher incomes
% \end{itemize}
% \end{mdframed}

	The study area is in the river island, known as \textit{chars} in Bengali, of northern Bangladesh in Gaibandha and Kurigram districts. Chars are formed by sediments and silt depositions and are prone to cyclical river erosions and floods. Chars are not stable in size and even in existence, and episodes of their partial or complete erosion or submerging are quite common. Chars accommodate ultra-poor inhabitants who are forced, as a desperate attempt for survival, to relocate across islands due to river erosion and floods.

	In the study area, the asset, a heifer, is the prime investment choice. A heifer needs to be at least two years old to start lactation.\footnote{They typically need to be about 15 months old to be ready for insemination and takes about 9.5 months to deliver a calf as it starts lactation, or the total of about 2 years. \label{heifer2years}} Rearing costs are higher for cattle as it requires fodder while a goat will eat the bushes. Cattle requires vaccination shots when a goat is usually left unvaccinated. Reproductive capacity of goats are high.\footnote{Parity size approaches to 2 at the third birth, and the birth interval is about 200 days \citep{Hasan2014goat}. An indigenous cow has a birth interval of 375 to 458 days \citep{Hasan2018}, resulting in about 2 years for gestation and calving interval \citep{Habib2012} with the mean lifetime births of 4 \citep[][Table 1]{Hasan2018}. } However, in comparison with cattle, their higher reproductive capacity and lower rearing costs are more than offset by the elevated morbidity and mortality risks,\footnote{Indeed, morbidity of goat kids ranges from 12\% \citep{Mahmud2015} to more than 50\% in some diseases \citep[][Table 5]{Nandi2011}, while cattle morbidity is around 22\% \citep{Bangar2013}. Goat kid mortality ranges from 6\% \citep{Mahmud2015} to 30\% \citep[][Table 5]{Paul2014} \citep{Ershaduzzaman2007}. Heifer mortality is between 5\% \citep[][p.332R]{Hossain2014} to 10\% \citep{Alauddin2018}. Higher morbidity of goat kids partly reflects their eating style that uses lips rather than tongues (as cattles do) and vulnerability to logging water. } and a less frequent cash flow.\footnote{Produce of goats are mostly meat. Cow's lactation length is 227 days and milk yield is 2.2 kg per day \citep{Rokonuzzaman2009} while goat milk is seldom marketed.  A meat market requires a cluster of relatively high income earners, which takes some efforts to get to from the river islands. Goat meat sales is seasonal and it does not provide a frequent cash flow. } In comparison with smaller livestock such as goats, cattle is more versatile in flood-prone areas. Residents also report that a goat herd is less mobile than single cattle when they are forced to evacuate during the flood. All of these considerations prompt residents to opt for cattle when they can afford it, and do not expand the herd size of goats, which are both confirmed in our data.

\section{Study sample}
\label{SecStudySample}

	Our sample is drawn from the population of river island villages in Northern Bangladesh. 
	\textcolor{red}{[Can someone provide the regional characteristics of the area, esp. poverty, using statistical information?]}
	
	In the \textit{char} region, the majority of \textit{char}s have only one village. The majority of \textit{char}s have no MFI activity, and we delisted the \textit{char}s if any of MFI, NGO, or \textit{Char Livelihood Program (CLP)} is active. Using a Landsat imagery, we identified 128 \textit{char}s within a day boat ride from the Gaibandha peer and collected information by field visits. From this list of \textit{char}s, we randomly selected 80 \textit{char}s. In each village, we created a census of household wealth ranking by a participatory ranking process. Following a process similar to the paired ranking as in \citet[][p.1212]{Alatas2012} and the Peruvian ultra poor case of \citet[][p.66]{KarlanThuysbaert2019}, we asked the least wealthy households in terms of asset ownership. We then asked to form a member committee of 10 households, of which 6 are ultra poor and 4 are moderately poor.  As we admitted households on a first come, first served basis, these 10 households are the first to join the membership of microfinance in respective poverty classes. %We randomised the loan arms at the village level. 
	
	After receiving acceptance for study participation (`pre-acceptance' in \textsc{\normalsize Figure \ref{fig experimantal design}}), baseline data was collected in 2012 prior to the debt contract type randomisation. After offering the each type of debt contract, three groups opted out as a group. In addition to the group level rejection, we had 89 individual loan rejectors. This happened despite we had explained about the debt contract types, the random assignment process, various other group based obligations, and had obtained everyone's consent to participate before randomisation. Although both type of rejecters refused to receive a loan, they gave a consent to be surveyed so we tracked them in subsequent survey rounds. 
	
	While loan rejecters remained in our sample, we lost four groups to floods in 2013. As they relocated, we had no choice but to drop them from the study. In our study, attrition refers to a drop out from our household survey. Rejection refers to a loan rejection in our intervention, and majority of rejecters (81.25\%) did not attrit from our household survey. Counting all other individual attriters, we have a total of 116 subjects (14.9\%) who attrited by the final round. 
	
	As a result, in the baseline survey sample, there are flood victims whom we do not track, group rejectors, individual rejectors and borrowers that we track. See \citet{GUK2016} for more details on the randomisation and acceptance process. As we track all --- barring the flood victims whose villages were washed away and other attriters --- the potential borrowers including who eventually opted out the borrowing, we are able to estimate the intention-to-treat effects of offering loans with different feartures on the population who showed interests in joining microfinance membership. 

	%Using the survey baseline data of 776 households, we confirm the randomisation balance was reasonably achieved as there is no household characteristics whose $p$ value for the mean difference to exceed 10\% between intervention arms (See Appendix \ref{AppSecRandomisation}). %We also check livestock holding in all rounds. We see that it is indistinguishable between arms at the baseline (\textsc{Table \ref{table anova CattleHoldingArm}}).

\hspace{-1cm}\begin{minipage}[t]{14cm}
\hfil\textsc{\normalsize Table \refstepcounter{table}\thetable: Descriptive statistics by arm for all households including nonparticipants\label{tab DestatMainByArm}}\\
\setlength{\tabcolsep}{1pt}
\setlength{\baselineskip}{8pt}
\renewcommand{\arraystretch}{.55}
\hfil\begin{tikzpicture}
\node (tbl) {\input{c:/data/GUK/analysis/save/EstimationMemo/DestatMainByArm.tex}};
\end{tikzpicture}\\
\renewcommand{\arraystretch}{.8}
\setlength{\tabcolsep}{1pt}
\begin{tabular}{>{\hfill\scriptsize}p{1cm}<{}>{\hfill\scriptsize}p{.25cm}<{}>{\scriptsize}p{12cm}<{\hfill}}
Source:& \multicolumn{2}{l}{\mpage{12.25cm}{\scriptsize Information of 776 households in GUK administrative data and household survey data at the baseline. Survey respondents include nonparticipants to the experiments.}}\\
Notes: & 1. & Values are means, values in brackets are standard deviations. \\
& 2. & \textsf{HeadLiteracy} is an indicator variable of household head literacy. \textsf{HeadAge} is age of household head. \textsf{HHsize} is total number of household members. \textsf{FloodInRd1} is an indicator variable of flood exposure. \textsf{HAssetAmount} and \textsf{PAssetAmount} are amount of household and productive assets, respectively, in Tk. \textsf{NumCows} is cattle holding per household. \textsf{NetValue} is net asset values per housheold. \textsf{Attrited} indicates attrition rates in the household survey, and \textsf{GRejected} and \textsf{IRejected} show group rejection rates and individual rejection rates to the lending program. \textsf{Non-attriting borrowers} indicates the ratio of non-attriting borrowers to all borrowers. Because attrition and rejection are separate events, a household can reject and attrit, so non-attrited borrowers $\geqslant$ total - (rejected members + attrited members).
\end{tabular}
\end{minipage}

\vspace{2ex}
	\textsc{\small Table \ref{tab DestatMainByArm}} shows descriptive statistics of sample households. As we randomly allocate them into four different arms named as \textsf{traditional}, \textsf{large}, \textsf{large grace}, and \textsf{cattle}, summary is shown by the arms and the overall. As shown in the Appendix \ref{AppSecRandomisation} \textsc{\normalsize Table \ref{tab perm}}, these baseline household characteristics does not differ statistically between the arms. Our sample is characterised by relatively low literacy rate (\textsf{HeadLiteracy}) and relatively young age (\textsf{HeadAge}) of the household heads. Literacy rate is lower than the national average of adult males at 61.54\% in 2012 (UNESCO). Household size (\textsf{HHsize}) is not large, 4.189 members overall, due probably to the constant flood threats, as indicated by above 40\% exposure at the baseline (\textsf{FloodInRd1}), that do not easily allow a large household formation. Cattle holding per household (\textsf{NumCows}) shows cattle rearing is not common and the mean herd size is between .2 to .4.\footnote{ \textsc{Table \ref{table anova CattleHoldingArm}} in Appendix \ref{AppSecRandomisation} shows the test results that \textsf{NumCows} do not differ across arms at the baseline. } Mean net asset values per household (\textsf{NetValue}) and its components, household asset values per household (\textsf{HAssetAmount}) and productive asset values per household (\textsf{PAssetAmount}), differ between the arms, but they mostly reflect sampling errors as indicated by the large standard deviations.\footnote{There is an alternative measure for net assets, which we call narrow net assets: Narrow net assets = Narrow assets + net saving - debt to GUK - debts to relatives and money lenders. Narrow assets use only items observed for all 4 rounds for household assets. All estimation results hold with narrow net assets with narrower confidence intervals. See \textsc{\footnotesize Figure \ref{fig NarrowNetAssetsLivestockEffects}} for details. } \textsf{Attrited} indicates attrition rates in the household survey, and \textsf{GRejected} and \textsf{IRejected} show group rejection rates and individual rejection rates to the lending program. We will analyse attrition and rejection later in Section \ref{ResultsSectionParticipation}, \ref{SecAttrition}, but at this point, we just note that the attrition rates are not statistically different between the arms. \textsf{Non-attriting borrowers} indicates the ratio of non-attriting borrowers to all borrowers. Because there are more rejecters in the \textsf{traditional} arm, this ratio is smaller than in other arms. 



\section{Experimental design}
\label{SecExperimentalDesign}


% \begin{mdframed}[style={SecItemize}, frametitle={Experimental design}]
% \begin{itemize}
% \vspace{1.0ex}\setlength{\itemsep}{1.0ex}\setlength{\baselineskip}{12pt}
% \item	Stepped-wedge design allows us to test a series of contraints.
% 	\begin{itemize}
% 	\vspace{1.0ex}\setlength{\itemsep}{1.0ex}\setlength{\baselineskip}{12pt}
% 	\item	Cow vs. Large grace: Entrepreneurship constraint (\textsf{InKind})
% 	\item	Large grace vs. Large: Saving constraint (\textsf{WithGrace})
% 	\item	Large vs. Traditional: Liquidity constraint (\textsf{Upfront})
% 	\end{itemize}
% \item	Interpretation of entrepreneurship contsraint: Access to (textual) knowledge does not increase profits
% \item	We track everyone except flood victims. This allows us to consistently estimate the ITT effects.
% \end{itemize}
% \end{mdframed}

	To investigate the detailed demand-side constraints and suitable credit scheme for the ultra poor, we implemented the village-level clustered randomization across the four treatment arms as follows (see \textsc{\normalsize Figure \ref{fig DestatMainByArm}}):

\begin{figure}
\hfil\mpage{12cm}{\footnotesize
\hfil\textsc{\normalsize Figure \refstepcounter{figure}\thefigure: Description of experimental arms\label{fig DestatMainByArm}}\\
\hfil\BGbox{
	\begin{description}
	\vspace{1.0ex}\setlength{\itemsep}{1.0ex}\setlength{\baselineskip}{12pt}
	\item[T1]	Traditional microcredit. 
		\begin{description}
		\vspace{1ex}\setlength{\itemsep}{.5ex}\setlength{\baselineskip}{8pt}
		\item[Credit]	5600 taka (approximately USD 50).
		\item[Installments]	Repay with weekly installments of 125 taka (approximately USD 1.1).
		\item[Maturity]	Total installments of 50 or a loan maturity of one year. Take another two loan contracts of equivalent amounts over the next consecutive years.
		\item[Repayment start]	Two weeks after the disbursement.
		\item[Weekly obligations]	Attend a meeting and deposit an amount decided jointly with group members.
		\end{description}
	\item[T2]	Upfront lumpy credit. Following conditions in black colours differ from \textsf{T1}:
		\begin{description}
		\vspace{1ex}\setlength{\itemsep}{.5ex}\setlength{\baselineskip}{8pt}
		\item[Credit]	16,800 taka (approximately USD ).
		\textcolor{gray}{\item[Installments]	Repay with weekly installments of 125 taka (approximately USD 1.1).}
		\item[Maturity]	Total installments of 150 or a loan maturity of three years.
		\textcolor{gray}{\item[Repayment start]	Two weeks after the disbursement.
		\item[Weekly obligations]	Attend a meeting and deposit an amount decided jointly with group members.}
		\end{description}
	\item[T3]	Upfront lumpy credit with a grace period. Following conditions in black colour differ from \textsf{T2}:
		\begin{description}
		\vspace{1ex}\setlength{\itemsep}{.5ex}\setlength{\baselineskip}{8pt}
		\item[\textcolor{gray}{Credit}]	\textcolor{gray}{16,800 taka (approximately USD ).}
		\item[Installments]	Repay with weekly installments of 190 taka (approximately USD 1.7).
		\item[Maturity]	Total installments of 100 or two years. 
		\item[Repayment start]	One year after the disbursement.
		\item[\textcolor{gray}{Weekly obligations}]	\textcolor{gray}{Attend a meeting and deposit an amount decided jointly with group members.}
		\end{description}
	\item[T4]	In-kind credit with a one-year grace period and managerial support programs. Following conditions in black colours differ from \textsf{T3}:
		\begin{description}
		\vspace{1ex}\setlength{\itemsep}{.5ex}\setlength{\baselineskip}{8pt}
		\item[Credit]	Receive a credit in the form of a one-year old heifer with the price of 16,000 taka (approximately USD 145).
		\item[\textcolor{gray}{Installments}]	\textcolor{gray}{Repay with weekly installments of 190 taka (approximately USD 1.7).}
		\item[\textcolor{gray}{Maturity}]	\textcolor{gray}{Total installments of 100 or two years. }
		\item[\textcolor{gray}{Repayment start}]	\textcolor{gray}{One year after the disbursement.}
		\item[\textcolor{gray}{Weekly obligations}]	\textcolor{gray}{Attend a meeting and deposit an amount decided jointly with group members.}
		\item[Support program] 	Receive fodder, training on cow rearing, regular veterinary and vaccination services, and marketing consultancy services from the local NGO, at the total fee of 800 taka (approximately USD 7.2) charged over the three years.
		\end{description}
	\end{description}
	}
}
\end{figure}

	\begin{description}
	\vspace{1.0ex}\setlength{\itemsep}{1.0ex}\setlength{\baselineskip}{12pt}
	\item[T1]	Traditional microcredit. Members of the group receive 5600 taka (approximately USD 50) credit, and the loan repayment begins two weeks after the disbursement. Members repay with weekly installments and are required to attend weekly meetings as well as to regularly save an amount decided jointly by the group members. The loan maturity is one year, and borrowers are allowed to take another two loan contracts of equivalent amounts over the next consecutive years. The weekly repayment is 125 taka (approximately USD 1.1) payable in 50 installments.
	\item[T2]	Upfront lumpy credit. Members receive 16,800 taka credit with a longer loan maturity, and the loan repayments begin two weeks after the disbursement. The weekly repayment and the design of compulsory saving are exactly the same as in \textsf{T1} arm. The loan maturity is three years. The required weekly repayment is 125 taka payable in 150 weekly instalments (for three years). 
	\item[T3]	Upfront lumpy credit with a grace period. Members receive 16,800 taka credit with loan repayments begin one year after the disbursement. During the first year grace period, members are required to meet weekly and follow group activities such as compulsory savings just as in other arms. The design of compulsory saving is the same as in the \textsf{T1}, \textsf{T2} arms. The loan maturity is three years. The required weekly repayment is 190 taka (approximately USD 1.7) payable in 100 weekly installments, starting after one year.
	\item[T4]	In-kind credit with a one-year grace period and managerial support programs.\footnote{It is almost the same as the finance lease, but it is difficult to distinguish it from a debt with the purchased asset set as a collateral. Under a finance lease, asset ownership belongs to the lessor, while under a collateralised debt, the asset ownership moves to the borrower. Heifer ownership was never explicitly agreed upon, and it is generally understood by the borrowers that they own the heifer, which gives similarity to a collateralised debt. } Members receive in-kind credit in the form of a one-year old heifer with the price of 16,000 taka (approximately USD 145), and the loan repayment begin one year after the disbursement. The grace period length is equal to the one provided under \textsf{T3} and \textsf{T4} arms. %As stated in footnote \ref{heifer2years} in Section \ref{SecBackground} and given that we acquire one year old heifers, a one year grace period results in two year old heifers, which is the age they start lactation. 
	In addition, the members receive fodder, training on cow rearing, regular veterinary and vaccination services, and marketing consultancy services from the local NGO, at the total fee of 800 taka (approximately USD 7.2) charged over the three years. 
	\end{description}

	One of the aims of the study is to assess if the entrepreneurship matters in microfinance lending outcomes. Assuming that, below 17000 Taka, the productive asset with the highest return is a heifer, we bundle training and consultation with a heifer lending. At the start of a loan, our expert procures a heifer from the local market, so the borrower does not have to have the knowledge required for the quality purchase. By providing the knowledge to a group of borrowers through training and disallowing an investment choice with a in-kind, heifer lending, some aspects of entrepreneurship will no longer be a prerequisite. It can be seen that we are offering a capacity to use the best practice or the \textit{cristalised intelligence} related to cattle production \citep{Cattell1963}. This is only a part of entrepreneurial skills. The remainder, a capacity to apply a suitable action to unforeseen events or the \textit{fluid intelligence} related to cattle production, and other inter-personal skills, are left unchanged. If the entrepreneurship raises productivity, borrowers of other arms who are not provided the knowledge are expected to opt out the loan more frequently or perform worse. One can measure effects of the entrepreneurship on participation and outcomes by comparing these two groups, in-kind credit with training vs. cash credit.
	%So the estimated impacts may reflect a hightened awareness to the production knowledge that can be manipulated by outsiders. This characterisation suggests that the entrepreneurial skills we provide overlaps with what the professional consultants advise in the management capital literature. %If there is an impact of cristalised intelligence, then one can compute its net returns. We note that, in an attempt not to give any monetary subsidy, we are charging fees at market prices to all the services we provide at request: Fodder supply, milk marketing, veterinary, and insemination. What we do not charge is a form of consulting services that answer to the questions from members who may lack practical knowledge of cattle production. We also provided compulsory training. \footnote{This could have served as a levy on the members of in-kind arm if they already had the relevant knowledge, as we did not give a choice of nonparticipation had they agreed to receive an in-kind loan. Majority of households had no experience in dairy cattle production at the baseline, and we expect them to lack the practical knowledge. }  
	
	As a natural reference, we want to compare the training cum in-kind loan \textsf{T4} with the traditional regular microcredit \textsf{T1}, a classic Grameen style loan that is about a third in loan size and maturity with no grace period. In order to make comparison feasible, we added two intermediate treatment arms to bridge them: Two arms with upfront lumpy lending that is equivalent of a heifer price, one with a grace period \textsf{T3} and another without a grace period \textsf{T2}. With the loan sizes that are three times the traditional microfinance loans, we extended the maturity to three years to even out the repayment burden. The comparison arm, the traditional regular microcredit, has only one year maturity.\footnote{Each arms have pure control groups who did not receive a loan until 1, 2 years later into the program. Due to a concern for within group spill overs, we do not use them here. } We therefore provided the total of three loans in three loan cycles in \textsf{T1} which are unconditionally disbursed annually, so the total loaned amount will be aligned and there is no exit-selection due to delinquency before three cycles are complete. %Consequently, except for the traditional Grameen-style credit arm \textsf{T1}, the amount of upfront liquidity is set at the level just affordable to purchase a one-year old heifer. 
	
	Under this setting, frontloading liquidity without changing the total loan size eases a liquidity constraint, attaching a grace period under the same loan size and disbursement timing eases a saving constraint prior to a loan receipt, and offering an in-kind loan with a managerial support without changing other features eases an entrepreneurship constraint. In effect, we constructed a stepped-wedge design over these key features of loans, namely, upfront liquidity (\textsf{Upfront}), a grace period (\textsf{WithGrace}), and in-kind with managerial supports (\textsf{InKind}), to assess the impacts of respective constraints on participation and outcomes as indicated in \textsc{\normalsize Table \ref{tab factorial design}}. 

	An in-kind offer in treatment \textsf{T4} is generally thought to be less efficient than a cash offer as it takes away an investment choice from the borrower. However, the local microfinance practitioners widely agree that other production opportunities are limited, so not much is lost in terms of the choice set, under our setting of island location and occasional floods.\footnote{A closely related project in the neighbouring areas transfers an asset in the form of a cow \citep{BandieraBRAC2017}. }  Given the small set of the productive investment choices, our experiment gives a unique chance to compare cash lending against in-kind lending, even without controlling for a potentially wider choice set of cash lending. 
	%It is generally thought in practice that an in-kind offer, with only a single asset to lease out, is less efficient than a cash offer as it takes away a choice from the borrower. However, the local microfinance practitioners widely agree that little is lost in a production opportunity even when the loan takes an in-kind form in a heifer, because a heifer is almost the only investment choice in our study area.\footnote{I is also notable that a closely related project in the neighbouring areas transfers an asset in the form of a cow\citep{BandieraBRAC2017}. } If this presumption is correct, it gives a unique chance to compare cash lending with in-kind lending, even without controlling for the different choice set of projects. In the later section examining the income generating activities, we show that this is actually the case.
	Indeed, we found in our data that most of \textsf{T2} and \textsf{T3} cash borrowers started to invest in cattle after receiving a loan. Consequently, in our study, the cash-grace-period and in-kind-grace-period lending differ effectively only in the managerial support services bundled in the latter. 

	All loan products are of individual liability and the committee was intended to serve as an activity platform for microfinance operations. Among the \textsf{traditional} members, there were 24 members who received disbursements twice, not three times. We drop them from the analysis and use 776 members in the below. \textcolor{red}{[Abu-san: Do you know why these 24 households received the loans twice, not three times?]}



\begin{table}
\hspace{-1cm}\begin{minipage}[t]{14cm}
\hfil\textsc{\normalsize Table \refstepcounter{table}\thetable: A 4$\times$4 factorial, stepped wedge design\label{tab factorial design}}\\
\setlength{\tabcolsep}{1pt}
\setlength{\baselineskip}{8pt}
\renewcommand{\arraystretch}{.55}
\vspace{2ex}
\hfil\begin{tabular}{>{\footnotesize\hfill}p{2cm}<{}
>{\footnotesize\hfil}p{2.5cm}<{}
>{\footnotesize\hfil}p{2.5cm}<{}
>{\footnotesize\hfil}p{2.5cm}<{}}
					& \cellcolor{paleblue}\textcolor{black}{large, grace} 			& \cellcolor{paleblue}\textcolor{black}{large} & \cellcolor{paleblue}\textcolor{black}{traditional} \\\cellcolor{paleblue}
\textcolor{black}{cattle} 				& \mpage{2.5cm}{\hfil entrepreneurship\\\hfil constraint\\\hfil (\textsf{InKind})} &\cellcolor{gray80}\mpage{2.5cm}{\textcolor{gray}{\hfil saving\\\hfil constraint\\\hfil (\textsf{WithGrace})}} &\cellcolor{gray80}\mpage{2.5cm}{\textcolor{gray}{\hfil liquidity\\\hfil constraint\\\hfil (\textsf{Upfront})}}\\\cellcolor{paleblue}
\textcolor{black}{large, grace} &\cellcolor{gray20} 	&  \mpage{2.5cm}{\hfil saving\\\hfil constraint\\\hfil (\textsf{WithGrace})} & \cellcolor{gray80}\mpage{2.5cm}{\textcolor{gray}{\hfil liquidity\\\hfil constraint\\\hfil (\textsf{Upfront})}}\\\cellcolor{paleblue}
\textcolor{black}{large} 			&\cellcolor{gray20} 	&\cellcolor{gray20}& \mpage{2.5cm}{\hfil liquidity\\\hfil constraint\\\hfil (\textsf{Upfront})}%\\\cellcolor{pink}
%\textcolor{black}{control} & \multicolumn{3}{c}{\cellcolor{green}\textcolor{black}{level \hspace{1em} impacts}}
\end{tabular}
\end{minipage}

\footnotesize Note: \mpage{12cm}{\footnotesize Cell contents are hypothesised constraints on investments that exists in the column arm but are eased in the row arm. Contents in brackets are variable names of respective attributes.}
\end{table}


\begin{figure}
\mpage{\linewidth}{
\vspace{2ex}

\hfil\textsc{\normalsize Figure \refstepcounter{figure}\thefigure: Sampling framework, rejection, and attrition\label{fig experimantal design}}\\[2ex]
\hfil\begin{adjustbox}{max size={.9\textwidth}{.8\textheight}}
\hfil\tikzstyle{attritionbox} = 
[rectangle, rounded corners, text width = 2cm, minimum height=1cm, text centered, fill=blue!10,
 drop shadow={opacity=.5, shadow scale=1.05}]
\tikzstyle{dummybox} = 
[draw = none, fill = {\BGColour}, text width = 0cm, drop shadow={opacity=1, fill = {\BGColour}, shadow scale=1.01}]
\begin{tikzpicture}[
    every node/.style={
        font=\sffamily,
        drop shadow,
        fill=red!10,
        text width=5cm,
        align=center},
        >=latex, %Make the arrow tips latex
        myline/.style={ultra thick, black!50},
        shorter/.style={shorten <=1mm, shorten >=0.5mm}]
% start node [for some reasons, I cannot place a node by writing (5, 0)]
%\node[fill = white, text width = 0cm, drop shadow={fill=white, opacity=1, shadow scale=1.01}] (L0) at (0, 0) {};
\node (startnode) {\mpage{5cm}{\hfil Pre-acceptance ($g=80$, $n=800$)\\\\[-1ex]\mpage{5cm}{\footnotesize\hfill moderately poor (240)\\\hfill utra poor (560)\SetLengthSkip{10pt}}\SetLengthSkip{10pt}}};
\node[below = .75cm of startnode] (baseline) {Baseline survey};
\node[below = .75cm of baseline] (random) {Cluster randomisation};

% randomisaion
\node[below = 2.5cm of random, dummybox] (dummy) {};
\node[left = 4.75cm of dummy, text width = 3cm] 
  (trad) {\mpage{3cm}{\hfil T1 Traditional\\\hfil ($n=176$)\\
  \mpage{3cm}{\footnotesize\hfill accepted (105)\\
  \hfill group rejection ($40$)\\
  \hfill individual rejection ($31$)%\\
  %\hfill two disbursements ($24$)
  \SetLengthSkip{10pt}}\SetLengthSkip{10pt}}};
\node[right = 1cm of trad, text width = 3cm] 
  (large) {\mpage{3cm}{\hfil T2 Large\\\hfil ($n=200$)\\
  \mpage{3cm}{\footnotesize\hfill accepted (171)\\
  \hfill group rejection ($20$)\\
  \hfill individual rejection ($9$)%\\
  \SetLengthSkip{10pt}}\SetLengthSkip{10pt}}};
\node[right = 1cm of large, text width = 3cm] 
  (grace) {\mpage{3cm}{\hfil T3 Large grace\\\hfil ($n=200$)\\
  \mpage{3cm}{\footnotesize\hfill accepted (177)\\
  \hfill group rejection ($10$)\\
  \hfill individual rejection ($13$)%\\
  \SetLengthSkip{10pt}}\SetLengthSkip{10pt}}};
\node[right = 1cm of grace, text width = 3cm] 
  (cattle) {\mpage{3cm}{\hfil T4 Cattle\\\hfil ($n=200$)\\
  \mpage{3cm}{\footnotesize\hfill accepted (163)\\
  \hfill group rejection ($0$)\\
  \hfill individual rejection ($37$)%\\
  \SetLengthSkip{10pt}}\SetLengthSkip{10pt}}};
% add arrows from start to each arms
\draw[myline, ->, shorter] (startnode) -- (baseline);
\draw[myline, ->, shorter] (baseline) -- (random);
\draw[myline, ->, shorter, draw = none] 
  (random) -- (trad.north) node[midway, dummybox, yshift = .25cm] (rthalf) {};
\draw[myline, ->, shorter] (random) -- (large.north);
\draw[myline, ->, shorter] (random) -- (grace.north);
\draw[myline, ->, shorter] (random) -- (cattle.north);
% draw an invisible horizontal line at rthalf 
\draw[name path=tradArrow, myline, ->, shorter, draw = none] (random) -- (trad.north);
\draw[name path=invisH, draw = none] (rthalf -| trad.west) -- (rthalf -| cattle.west);
% find an intersection of invisH and tradArrow
\path [name intersections={of=invisH and tradArrow, by={tradMidArrow}}];
\draw[myline, ->, shorter] ([xshift = .1cm] tradMidArrow) -- ($(tradMidArrow) + (-2.5, 0)$) 
  node[attritionbox, anchor = east] (DoubleTrad) 
  {\mpage{2cm}{\hfil 2 loans\\\hfil ($n=24$)\SetLengthSkip{10pt}}};
\draw[myline, ->, shorter] (random) -- (trad.north);

% trad to attriters
\node[below = 1cm of trad.south west, xshift = 1.5cm, anchor = west, attritionbox]  
  (tradFlood) {\mpage{2cm}{\hfil flood victims\\\hfil ($n=20$)\SetLengthSkip{10pt}}};
\node[below = .25cm of tradFlood, attritionbox]  
  (tradAttrit2) {\mpage{2cm}{\hfil attrited in 2\\\hfil ($n=6$)\SetLengthSkip{10pt}}};
\node[below = .25cm of tradAttrit2, attritionbox]  
  (tradAttrit3) {\mpage{2cm}{\hfil attrited in 3\\\hfil ($n=4$)\SetLengthSkip{10pt}}};
\node[below = .25cm of tradAttrit3, attritionbox]  
  (tradAttrit4) {\mpage{2cm}{\hfil attrited in 4\\\hfil ($n=2$)\SetLengthSkip{10pt}}};
\node[below = 6cm of trad, text width = 3cm]
  (tradLast) {\mpage{3cm}{\hfil Traditional\\\hfil ($n=144$)\\
  \mpage{3cm}{\footnotesize\hfill accepted (83)\\
  \hfill group rejection ($36$)\\
  \hfill individual rejection ($25$)
  \SetLengthSkip{10pt}}\SetLengthSkip{10pt}}};
  % draw kinked arrows
	\draw[myline, shorter, ->] ([xshift = 1cm] trad.south west) |- (tradFlood.west);
	\draw[myline, shorter, ->] ([xshift = 1cm] trad.south west) |- (tradAttrit2.west);
	\draw[myline, shorter, ->] ([xshift = 1cm] trad.south west) |- (tradAttrit3.west);
	\draw[myline, shorter, ->] ([xshift = 1cm] trad.south west) |- (tradAttrit4.west);
	%\draw[myline, shorter, ->] ([xshift = 1cm] trad.south west) --++ (0, -5cm);
	\draw[myline, shorter, ->] ([xshift = 1cm] trad.south west) -- ([xshift = 1cm] tradLast.north west);
% large to attriters
\node[right = 2cm of tradAttrit2, attritionbox]  
  (largeAttrit2) {\mpage{2cm}{\hfil attrited in 2\\\hfil ($n=5$)\SetLengthSkip{10pt}}};
\node[below = .25cm of largeAttrit2, attritionbox]  
  (largeAttrit3) {\mpage{2cm}{\hfil attrited in 3\\\hfil ($n=2$)\SetLengthSkip{10pt}}};
\node[below = .25cm of largeAttrit3, attritionbox]  
  (largeAttrit4) {\mpage{2cm}{\hfil attrited in 4\\\hfil ($n=1$)\SetLengthSkip{10pt}}};
\node[below = 6cm of large, text width = 3cm]  
  (largeLast) {\mpage{3cm}{\hfil Large\\\hfil ($n=192$)\\
  \mpage{3cm}{\footnotesize\hfill accepted (164)\\
  \hfill group rejection ($19$)\\
  \hfill individual rejection ($9$)
  \SetLengthSkip{10pt}}\SetLengthSkip{10pt}}};
  % draw kinked arrows
%	\draw[myline, shorter, ->] ([xshift = 1cm] large.south west) |- (largeFlood.west);
	\draw[myline, shorter, ->] ([xshift = 1cm] large.south west) |- (largeAttrit2.west);
	\draw[myline, shorter, ->] ([xshift = 1cm] large.south west) |- (largeAttrit3.west);
	\draw[myline, shorter, ->] ([xshift = 1cm] large.south west) |- (largeAttrit4.west);
	\draw[myline, shorter, ->] ([xshift = 1cm] large.south west) -- ([xshift = 1cm] largeLast.north west);
% grace to attriters
\node[below = 1cm of grace.south west, xshift = 1.5cm, anchor = west, attritionbox]  
  (graceFlood) {\mpage{2cm}{\hfil flood victims\\\hfil ($n=10$)\SetLengthSkip{10pt}}};
\node[below = .25cm of graceFlood, attritionbox]  
  (graceAttrit2) {\mpage{2cm}{\hfil attrited in 2\\\hfil ($n=13$)\SetLengthSkip{10pt}}};
\node[below = .25cm of graceAttrit2, attritionbox]  
  (graceAttrit3) {\mpage{2cm}{\hfil attrited in 3\\\hfil ($n=3$)\SetLengthSkip{10pt}}};
\node[below = .25cm of graceAttrit3, attritionbox]  
  (graceAttrit4) {\mpage{2cm}{\hfil attrited in 4\\\hfil ($n=3$)\SetLengthSkip{10pt}}};
\node[below = 6cm of grace, text width = 3cm]  
  (graceLast) {\mpage{3cm}{\hfil Large grace\\\hfil ($n=171$)\\
  \mpage{3cm}{\footnotesize\hfill accepted (160)\\
  \hfill group rejection ($0$)\\
  \hfill individual rejection ($11$)
  \SetLengthSkip{10pt}}\SetLengthSkip{10pt}}};
  % draw kinked arrows
	\draw[myline, shorter, ->] ([xshift = 1cm] grace.south west) |- (graceFlood.west);
	\draw[myline, shorter, ->] ([xshift = 1cm] grace.south west) |- (graceAttrit2.west);
	\draw[myline, shorter, ->] ([xshift = 1cm] grace.south west) |- (graceAttrit3.west);
	\draw[myline, shorter, ->] ([xshift = 1cm] grace.south west) |- (graceAttrit4.west);
	\draw[myline, shorter, ->] ([xshift = 1cm] grace.south west) -- ([xshift = 1cm] graceLast.north west);
% cattle to attriters
\node[below = 1cm of cattle.south west, xshift = 1.5cm, anchor = west, attritionbox]  
  (cattleFlood) {\mpage{2cm}{\hfil flood victims\\\hfil ($n=10$)\SetLengthSkip{10pt}}};
\node[below = .25cm of cattleFlood, attritionbox]  
  (cattleAttrit2) {\mpage{2cm}{\hfil attrited in 2\\\hfil ($n=5$)\SetLengthSkip{10pt}}};
\node[below = .25cm of cattleAttrit2, attritionbox]  
  (cattleAttrit3) {\mpage{2cm}{\hfil attrited in 3\\\hfil ($n=5$)\SetLengthSkip{10pt}}};
\node[below = .25cm of cattleAttrit3, attritionbox]  
  (cattleAttrit4) {\mpage{2cm}{\hfil attrited in 4\\\hfil ($n=3$)\SetLengthSkip{10pt}}};
\node[below = 6cm of cattle, text width = 3cm]  
  (cattleLast) {\mpage{3cm}{\hfil Cattle\\\hfil ($n=177$)\\
  \mpage{3cm}{\footnotesize\hfill accepted (147)\\
  \hfill group rejection ($0$)\\
  \hfill individual rejection ($30$)
  \SetLengthSkip{10pt}}\SetLengthSkip{10pt}}};
  % draw kinked arrows
	\draw[myline, shorter, ->] ([xshift = 1cm] cattle.south west) |- (cattleFlood.west);
	\draw[myline, shorter, ->] ([xshift = 1cm] cattle.south west) |- (cattleAttrit2.west);
	\draw[myline, shorter, ->] ([xshift = 1cm] cattle.south west) |- (cattleAttrit3.west);
	\draw[myline, shorter, ->] ([xshift = 1cm] cattle.south west) |- (cattleAttrit4.west);
	\draw[myline, shorter, ->] ([xshift = 1cm] cattle.south west) -- ([xshift = 1cm] cattleLast.north west);

% Draw the background
% \begin{pgfonlayer}{background}
% \path (tradLast.west |- startnode.north)+(-.5cm, .5cm) 
% 	node (topleft1) {};
% \path (cattleLast.south east)+(1cm, -.5cm) 
% 	node (bottomright1) {};
% \path[fill = {\BGColour}, rounded corners]
% 	(topleft1) rectangle (bottomright1);
% \end{pgfonlayer}

\begin{pgfonlayer}{background}
\node [fill={\BGColour}, fit=(trad) (startnode) (cattleAttrit4) (cattleLast)] {};
\end{pgfonlayer}

\end{tikzpicture}

\end{adjustbox}

\vspace{2ex}
}
\end{figure}

\section{Empirical strategy}
\label{SecEmpiricalStrategy}

% \begin{mdframed}[style={SecItemize}, frametitle={Empirical strategy}]
% \begin{itemize}
% \vspace{1.0ex}\setlength{\itemsep}{1.0ex}\setlength{\baselineskip}{12pt}
% \item	We use ANCOVA estimates.
% \item	We estimate both arm wise impacts and attribute wise impacts.
% \end{itemize}
% \end{mdframed}

	We collected data at one baseline survey and three annual follow up surveys. With successful randomisation (see Section \ref{ResultsSectionParticipation} and Appendix \ref{AppSecRandomisation}), we use ANCOVA estimators to measure impacts of each experimental arms and loan attributes. ANCOVA estimators are more efficient than DID estimators \citep{FrisonPocock1992, McKenzie2012}. As we include loan rejecters, what we are estimating is intention-to-treat effects. For an ease of interpretation, we sometimes use indicator variables of each attributes, \textsf{Upfront, WithGrace, InKind} in place of arms in several estimating equations. Numerically, both are equivalent. In what follows, we will refer to these attributes as \textit{functional attributes}.
	
	The basic estimating equation for our intention-to-treat effects is:
	\begin{equation}
	y_{it}=b_{10}+\bfb'_{1}\bfdee_{i}+b_{2}y_{i0}+e_{it},
	\end{equation}
	where, for member $i$ in period $t$, $y_{it}$ is an outcome measure, $\bfdee_{i}$ is a vector of indicator variables in arms or functional attributes that $i$ receives, $e_{it}$ is an error term. For the \textsf{traditional} arm, the conditional mean of outcome given covariates and baseline outcome variable is given by $b_{10}$. For an arm or a functional attribute $a$, the impact relative to the traditional arm is measured with $b_{1a}$. As we are interested in the time course of impacts, we allow for time-varying impacts as:
	\begin{equation}
	y_{it}=b_{10}+\bfb'_{1}\bfdee_{i}+b_{t0}c_{t}+\bfb'_{t}c_{t}\bfdee_{i}+b_{2}y_{i0}+e_{it},
	\label{EstimatingEqTimeVarying}
	\end{equation}
	where $c_{t}$ is a period indicator variable for $t>1$ that takes the value of 1 at $t$, 0 otherwise. We use the second period (period 2 in most cases) as the reference for time dummies. $b_{t0}$ measures the period $t$ deviation from $b_{10}$ for the \textsf{traditional} arm, $\bfb'_{t}$ measures the period $t$ deviation from the concurrent \textsf{traditional} arm for non-\textsf{traditional} arms or functional attributes. For the \textsf{traditional} arm, the conditional mean of outcome given covariates and baseline outcome variable is provided by $b_{10}+b_{t0}$. For the non-\textsf{traditional} arms, the deviation of conditional mean, given covariates and the baseline outcome variable, from \textsf{traditional} arm outcome in period $t$ is provided by $\bfb_{1}+\bfb_{t}$ with $\bfb_{t}=\0$ for $t=2$. Cumulative impacts are time-series sums of each impacts. In the Section \ref{SecResults}, we will plot and focus on the cumulative conditional mean deviations of each non-\textsf{traditional} arms in each period. All the standard errors are clustered at the group (char) level as suggested by \citet{AbadieAtheyImbensWooldridge2017}.%\footnote{To aid the understanding if the data is more suited to the assumption of first-difference (FD) rather than fixed-effects (FE), we use a check suggested by \citet[][10.71]{Wooldridge2010}. It is an AR(1) regression using FD residuals. Most of results show low autocorrelations in FD residuals which is consistent with the assumption of FD estimator. The issue of choice between FD or FE is not of primary importance, as the use of cluster-robust standard errors gives consistent estimates of SEs in both estimators, and it boils down to efficiency.  }

% \hfil\begin{adjustbox}{max size={1.02\textwidth}{1.21\textheight}}
% \hspace{-1cm}\input{paste0(pathprogram, "GUK_RCTDesignDiagram.tex")}
% \end{adjustbox}


\section{Results}
\label{SecResults}







% \begin{mdframed}[style={SecItemize}, frametitle={Results}]
% \begin{itemize}
% \vspace{1.0ex}\setlength{\itemsep}{1.0ex}\setlength{\baselineskip}{12pt}
% \item	Randomisation went well at the group level.
% \item	Group loan rejecters of \textsf{traditional} and non-\textsf{traditional} differ.
% 	\begin{itemize}
% 	\vspace{1.0ex}\setlength{\itemsep}{1.0ex}\setlength{\baselineskip}{12pt}
% 	\item	\textsf{traditional}: Lower livestock values, smaller cattle holding, and smaller net asset values. Non-\textsf{traditional}: A higher baseline flood exposure rate, a younger household head, and higher cattle holding.
% 	\item	Traditional arm rejecters are relatively less wealthy than non-traditional rejecters, consistent with a binding liquidity constraint which prevented them from participation, and rejecters might have participated to large sized lending if offered.
% 	\end{itemize}
% \item	Individual rejecters are similar between \textsf{traditional} and non-\textsf{traditional} arms. 
% 	\begin{itemize}
% 	\vspace{1.0ex}\setlength{\itemsep}{1.0ex}\setlength{\baselineskip}{12pt}
% 	\item	Common factors relative to nonrejecters: Smaller household size and smaller livestock holding.
% 	\item	These hint that it may take a larger household size to raise cattle, and (conditional on household size) households who have more livestock may have the capacity to raise more. These are consistent with a domestic capacity (labour and/or space) contsraint and a liquidity constraint.
% 	\end{itemize}
% \item	Overall, attrition is not correlated with household characteristics.
% 	\begin{itemize}
% 	\vspace{1.0ex}\setlength{\itemsep}{1.0ex}\setlength{\baselineskip}{12pt}
% 	\item	Less educated members attrited in traditional arm indicates there may be underestimation, if there is an attrition bias at all (so, no need to use Lee bounds, I think).
% 	\end{itemize}
% \item	Greater accumulation of assets (livestock, productive assets, land holding, household assets) for \textsf{Upfront} attribute
% 	\begin{itemize}
% 	\vspace{1.0ex}\setlength{\itemsep}{1.0ex}\setlength{\baselineskip}{12pt}
% 	\item	More diverse and smaller scale investment portfolio among \textsf{traditional}. 
% 	\end{itemize}
% \item	Lower repayment rates for \textsf{traditional}. 
% 	\begin{itemize}
% 	\vspace{1.0ex}\setlength{\itemsep}{1.0ex}\setlength{\baselineskip}{12pt}
% 	\item	This also is at odds with a popular belief `start small and grow' is more prudent.
% 	\end{itemize}
% \item	Greater asset accumulation and higher repayment rates for \textsf{Upfront} is consistent with nonconvex production, or a poverty trap.
% \item	No impacts of \textsf{InKind} on asset accumulation.
% 	\begin{itemize}
% 	\vspace{1.0ex}\setlength{\itemsep}{1.0ex}\setlength{\baselineskip}{12pt}
% 	\item	Negates the necessity of entrepreneurship, 
% 	\item	This is in contrast to the finding of existing studies that impacts are larger for the experienced borrowers ... everyone can be an entrepreneur at this level of skills and production possibilities? 
% 	\end{itemize}
% \item	No impact on consumption. 
% \item	Larger increase in labour incomes in period 3, probably induced by a repayment burden and borrower's compliance.
% \item	Schooling of children is generally not affected.  
% 	\begin{itemize}
% 	\vspace{1.0ex}\setlength{\itemsep}{1.0ex}\setlength{\baselineskip}{12pt}
% 	\item	While this is reassuring, there is a weak indication that female schooling at college level may have been adversely affected in \textsf{Large} arm and positively affected by \textsf{WithGrace} attribute in period 4.
% 	\end{itemize}
% \end{itemize}
% \end{mdframed}

	The reasons behind nonparticipation are fundamental in understanding the outreach. We analyse nonparticipation in relation to the debt contract design that they were randomly allocated to. In addition, selective attrition from the household surveys, if any, biases the estimates so we need to compare the attriter's characteristics with the nonattriters. 	In this section, we check how participation and attrition are different between the arms by using permutation tests. We use the \textsf{coin} package of \textsf{R} with 100000 random draws from all admissible permutations. 


\subsection{Participation}
\label{ResultsSectionParticipation}


	%We examined the difference between various groups defined by rejections and attrition in Appendix \ref{AppSecAttritionRejection}. 
	%In examining nonparticipation, we test if the household characteristics are different between participants and nonparticipants (rejecters, hereafter).  %Holm's step-down method is used to adjust $p$ values for multiple testing of multi-factor grouping variables.
	As noted in Section \ref{SecExperimentalDesign}, there are two kinds of rejecters in participation. One is group rejecters who jointly turned down the offer as a group, and another is individual rejecters who decided not to participate while fellow members of the group participated. %We will examine the characteristics of both type of rejecters and compare them between \textsf{traditional} and non-\textsf{traditional} arms. 

\begin{table}
\hfil\begin{minipage}[t]{14cm}
\hfil\textsc{\normalsize Table \refstepcounter{table}\thetable: Individual rejecters vs. non-rejecters\label{tab MainTextIRjecters}}\\
\setlength{\tabcolsep}{1pt}
\setlength{\baselineskip}{8pt}
\renewcommand{\arraystretch}{.55}
\hfil\input{c:/data/GUK/analysis/save/EstimationMemo/IndividualRejectionTestResults.tex}
\vspace{2ex}
\end{minipage}

\hfil{\footnotesize Note:} \mpage{12cm}{\footnotesize Individual rejecters are the members who did not accept a loan based on an individual decision after the period when group rejection was decided. After 70 people group-rejected, the total number of individuals who was in a position to individually reject the loan was 706 people, of which 90 individually rejected. Non-\textsf{traditional} arms are \textsf{large}, \textsf{large grace} and \textsf{cattle} arms. The variable \textsf{Arm} is the ratio of \textsf{traditional} arm members in individual nonrejecters and individual rejecters. Respective rejection rates are given in the brackets in the row \textsf{n}. \setlength{\baselineskip}{7pt}}

\vspace{3ex}
\hfil\begin{minipage}[t]{14cm}
\hfil\textsc{\normalsize Table \refstepcounter{table}\thetable: Contrasting \textsf{cattle} arm and other arms, borrowers and non-attriting borrowers\label{tab main cownoncow}}\\
\setlength{\tabcolsep}{1pt}
\setlength{\baselineskip}{8pt}
\renewcommand{\arraystretch}{.55}
\hfil\input{c:/data/GUK/analysis/save/EstimationMemo/CowVsNonCowTestResults.tex}
\vspace{2ex}
\end{minipage}

\hfil{\footnotesize Note:} \mpage{12cm}{\footnotesize Borrowers are members who accepted a loan, non-attriting borrowers are borrowers who stayed in the household survey until the final round. Both borrower panel and non-attriting borrower panel show the contrasts between the \textsf{cattle} arm and all other arms. Borrower panel compares the difference in participant characteristics between \textsf{cattle} and other arms. Non-attriting borrower panel compares the difference in non-attriting participant characteristics between \textsf{cattle} and other arms. Both show \textsf{cattle} arm induced participation of asset-poor households at the beginning and until the end of the project. Respective rejection rates are given in the brackets in the row \textsf{n}.\setlength{\baselineskip}{7pt}}
\end{table}


	Group rejecters of \textsf{traditional} and non-\textsf{traditional} arms differ in household characteristics. In the Appendix \ref{AppSecRejection}, it is shown that the asset-poor households did not participate in the \textsf{traditional} arm, while it is recent flood victims who did not participate in the non-\textsf{traditional} arms. We conjecture that it is lack of \textsf{Upfront} liquidity that prevented asset-poor households of \textsf{traditional} arm from participating because they cannot purchase cattle due to insufficient net asset values or an insufficient resale value of owned livestock, when members of similar characteristics partcipated in non-\textsf{traditional} arms. Group rejecters of non-\textsf{traditional} arms did not participate because of negative asset shocks. %This is a real resource constraint that binds the households. This is different from a psychological constraint that, so long as there is a cost or a payment involved, albeit at a minimal level, there remains a group of households who would not take up the investment \citep{Ashraf2010, CohenDupas2010}. 

	\textsc{\normalsize Table \ref{tab MainTextIRjecters}} compares individual rejecters and non individual rejecters for \textsf{traditional} arm and non-\textsf{traditional} arms. Rejecters of both arms share similar characteristics. %This is consistent with the conjecture that, had the \textsf{traditional} arm group rejecters been offered any of the non-\textsf{traditional} arms, they, as a group, may have accepted it.  %It shows the latter is more exposed to flood in baseline and has larger livestock values. This implies that, once large enough sum of loan is disbursed, %there is no minimum livestock and asset holding level to partake in the larger loans, and 
	%despite a negative asset shock in flood and a poverty trap at this level may be overcome once household size and negative asset shocks are accounted for.
	In the panel comparing individual rejecters and non individual rejecters of all arms, the common factors associated with nonparticipation are a smaller household size and smaller livestock holding%(\textsc{\normalsize Table \ref{tab1 Permutation test results of individual rejection among traditional arm}} and \textsc{\normalsize Table \ref{tab1 Permutation test results of individual rejection among non-traditional arm}})
	,\footnote{\textsf{NetValue} also shows a difference but this is due mostly to a difference in livestock holding. } although the $p$ value for livestock holding difference between individual rejecters and non individual rejecters is .084.
	% (\textsc{\normalsize Table \ref{TabLabel1[grep("l rejection$", TabLabel1)]}})
	%In non-\textsf{traditional} arms, the individual rejecters have only marginally different mean values relative to individual nonrejecters (\textsc{\normalsize Table \ref{tab Ireject nontrad perm}}). 

	These hint that it may take a larger household size to raise cattle. %, and the households who have more livestock may have the capacity to raise more. To interpret this, 
	It is possible that smaller households may be facing a domestic labour constraint or a space limitation to accommodate cattle under the roof. These constraints are expected to be absent in asset transfer programs where targeted residents can sell the asset if either of constraints binds. We conjecture that the households under a binding liquidity constraint and/or a binding domestic capacity constraint did not meet the conditions to raise cattle, and have withheld themselves from the program with an individual rejection. This self-selection may have caused the repayment rates to be higher than when everyone participated. 

	A strong correlation between baseline flood exposure and individual rejection among the non-\textsf{traditional} arm members suggests that a population prone to natural calamity and associated asset shocks have voluntarily opted out the borrowing. This partly explains the lack of commercial and even noncommercial/NGO lenders in the flood prone area. 
	
	In \textsc{\small Table \ref{tab main cownoncow}}, we compare if the \textsf{cattle} arm participants (borrowers) differ from participants in other arms at the baseline. In the left panel, we compare participants. It is worth noting that partcipants of \textsf{cattle} arm differ from other arms in having less cattle rearing experience as observed in smaller initial cattle holding ($p$ value = .156) and in having lower net asset values ($p$ value = .058). 
	%\textsc{\normalsize Table \ref{TabLabel1[grep("bo.*non-ca", TabLabel1)]}}
	These features that are plausibly disadvantageous in rearing a heifer notwithstanding, the \textsf{cattle} arm with training induced partcipation. As we will see in Section \ref{Sec Impacts}, the choice of lending vehicle (cash or in-kind) does not matter in investments. So it is natural to infer that the training component has induced the members with less experiences and fewer assets to take up loans. In the right panel, we compare the borrowers who did not attrit by the end of final survey round between \textsf{cattle} arm with other arms. At the baseline, \textsf{cattle} arm non-attriting borrowers have smaller baseline livestock holding ($p$ value = .016) and smaller baseline net asset holding ($p$ value = .007) than other arms' non-attriting borrowers. %(\textsc{\normalsize Table \ref{tab1 Permutation test results of surviving members of cattle and all other arms}})
	These hint that more disadvantaged borrowers participated and managed to stay on the survey until the end of the study in the \textsf{cattle} arm with a help of managerial supports.
	
	%	Group level rejection to participate is negatively correlated with literacy of household head (\textsc{\normalsize Table \ref{tab Greject perm MainText}}). Acknowledging the reasons for rejection can be different for individuals, we also tested the independence of each characteristics for individual rejecters (vs. non-individual rejecters) in \textsc{\normalsize Table \ref{tab Ireject perm MainText}}. One sees that smaller \textsf{HHsize}, being affected with \textsf{FloodInRd1}, and smaller \textsf{LivestockValues} and \textsf{NumCows} are associated with individual rejecters. We conjecture that individual decisions not to participate may be understood as: Smaller household size leaves a smaller capacity for cattle production labour in a household, and being hit with a flood may have resulted in lower livestock levels that would prompt them to reconsider partaking in another livestock project. 
%
%	A closer look at the nonparticipation correlates among \textsf{traditional} arm members in \textsc{\normalsize Table \ref{tab reject trad perm MainText}} and non-\textsf{traditional} arm members in \textsc{\normalsize Table \ref{tab reject nontrad perm MainText}} reveals possible differences in the causes. Rejection among \textsf{traditional} members tend to be associated with lower livestock holding but not with higher flood exposure nor smaller household size, while rejecters among non-\textsf{traditional} members are more likely to have suffered from flood at the baseline and have smaller household size. \textsc{\normalsize Table \ref{tab reject trad nontrad perm MainText}} shows rejecters of \textsf{traditional} have less flood exposure, smaller livestock and cattle holding, but not necessarily poorer as indicated by head literacy and asset holding than non-\textsf{traditional} counterpart. Given \textsf{traditional} rejecters at the mean have smaller livestock while household size is similar, it hints some capacity to supply labour for cattle production if an opportunity arises.

%	Since the offered arms were randomised, individual rejecters of \textsf{traditional} arm, who are similar in characteristcs to individual rejecters of non-\textsf{traditional} arm whose impediments are baseline flood and small household size, may have accepted the offer had their household size is larger and had they been offered non-\textsf{traditional} lending. Henceforth, we conjecture that flood exposure and household size are the potential impediments for uptake in larger size loans. 


	%Even we are targeting the ultra poor and designed the loan products to help them rise above the poverty trap, we still find lacking minimum level of assets, despite at a very low level, had kept the ultra poor from participating in microfinance. In the results of lending we consider in the below, the bottom class of the ultra poor had not lept its benefits.

\subsection{Attrition}
\label{SecAttrition}

	The survey resulted in the attrition of a moderate rate, 14.9\%. We checked for systematic differences between attriters and nonattriters 
	%\textsc{\normalsize Table \ref{TabLabel1[1]}}.
	and found the attrition is not correlated with any household level characteristics (see more detailed attrition examination in Appendix \ref{AppSecAttrition}). %As attrition rates differ between \textsf{traditional} and non-\textsf{traditional} arms, we %compare them %in \textsc{\normalsize Table \ref{tab1 Permutation test results of attriters of traditional and non-traditional arms}}. It shows 
	We also found that \textsf{traditional} arm attriters have a lower rate of head literacy while non-\textsf{traditional} arm attriters are more exposed to the flood and have a larger household size. One can argue that, with attrition, the estimated impacts of borrowing could have increased for the \textsf{traditional} arm while not for the non-\textsf{traditional} arms. Such a conjecture hints there may be underestimation, if any, but it is unlikely to inflate the impact estimates.

\subsection{Impacts}
\label{Sec Impacts}

\begin{figure}
\mpage{12cm}{
\hfil\textsc{\footnotesize Figure \refstepcounter{figure}\thefigure: Cumulative effects on livestock and net assets\label{fig NetAssetsLivestockEffects}}\\

\vspace{2ex}
\hfil\includegraphics[height = 8cm, width = 12cm]{c:/data/GUK/analysis/program/figure/EstimationMemo/NetAssetsNumCowsCumRelativeToConcurrentTradEffects.eps}\\
\renewcommand{\arraystretch}{1}
\setlength{\tabcolsep}{1pt}
\hfil\begin{tabular}{>{\hfill\scriptsize}p{1cm}<{}>{\scriptsize}p{12.5cm}<{\hfill}}
Source: & Constructed from ANCOVA estimation results \textsc{Table \ref{tab ANCOVA net assets timevarying}}, \textsc{Table \ref{tab ANCOVA net assets timevarying attributes}}, \textsc{Table \ref{tab ANCOVA cow time varying}}, \textsc{Table \ref{tab ANCOVA cow time varying attributes}}. \textsf{NetAssets} has 5 specifications (2-6), \textsf{NumCows} have 4 specifications (2-5).\\
Note:& Panels show cumulative impacts of respective arm or attributes \textsf{X} relative to \textsf{tradiotional} arm which are obtained by $\Delta\mbox{2nd period}=\mbox{\textsf{intercept}}+\mbox{\textsf{X}}, \Delta\mbox{3rd period}=\Delta\mbox{2nd period}+\mbox{\textsf{Period3}}+\mbox{\textsf{X*Period3}}$, $\Delta\mbox{4th period}=\Delta\mbox{2nd period}+\Delta\mbox{3rd period}+\mbox{\textsf{Period4}}+\mbox{\textsf{X*Period4}}$. Bars show 95\% confidence intervals using cluster robust standard errors.\\[1ex]
\end{tabular}
}
\end{figure}

	\textsc{\footnotesize Figure \ref{fig NetAssetsLivestockEffects}} summarises the cumulative impact estimates in time-varying specification of \eqref{EstimatingEqTimeVarying}.  See Appendix \ref{AppendixEstimationTables} for full estimation results. There are two stock outcome variables,  number of cattle and net asset values, where net assets are defined as total assets less debt outstanding to all sources. For each outcome, there are five panels of arms and functional attributes. Since \textsf{large} arm and \textsf{Upfront} functional attribute are numerically same in \eqref{EstimatingEqTimeVarying}, they are put in one panel column. In all panels, points show the estimates of cumulative deviation from concurrent \textsf{traditional} arm values, or how much the impacts relative to \textsf{traditional} arm have evolved. Vertical bars indicate 95\% confidence intervals. 
	For all panels, in each period, there are several estimation specifications bunched side-by-side. \footnote{Specification 1 is omitted from the plot, because it is an OLS regression without the baseline outcome that is intended to provide a reference for ANCOVA estimates. } This is intended to show robustness to specification changes at a glance.\footnote{As multiple specifications are estimated to show uniformity of results, not to pick one specific estimate, inference corrections for multiple testing are unncessary. } One sees that there is little variation across specifications. %Cattle and net assets have more regression specifications due to their possible dependence on previous cattle ownership and its inclusion as a covariate.
	
	There are notable tendencies in the figure. First, in both cattle holding and net asset panels, point estimates show there is a one time increase at period 2 in the \textsf{Large/Upfront} column. %while the conditional mean values are relatively unchanged for the \textsf{traditional} arm. 
	The non-\textsf{traditional} arms have increased cattle holding and net assets once and stayed increased relative to the \textsf{traditional} arm. As time passes, standard errors get necessarily magnified because borrowers get exposed to more random variations, so the bars grow longer, making the estimates noisier and error bars crossing the zero line in round 4. %Estimates for cattle holding of \textsf{traditional} arm remain relatively unchanged in all periods, so a one time increase implies a gap in cattle holding was created in period 2 and the gap stayed unchanged. 

	Secondly, it is the \textsf{Upfront} functinal attribute that shows positive impacts in both outcomes. %Estimates for net assets of \textsf{traditional} arm show an upward trend. On top of this underlying trend, all non-\textsf{traditional} arms show a one-time increase, or a gap relative to \textsf{traditional} due to the \textsf{Upfront} aspect of lending. 
	This is consistent with the nonconvex production technology of a larger investment under a liquidity constraint, coupled with an inferior, smaller investment technology.

	Thirdly, comparing the impacts of the \textsf{InKind} attribute on both stock outcomes against \textsf{Upfront} and \textsf{WithGrace}, we see statistically zero differences. In light of the fact that individuals with less cattle rearing experiences and lower asset values participated and survived in the \textsf{cattle} arm, the finding that their outcomes are statistically indistinguishable from other non-\textsf{traditional} arms implies the treatment arm facilitated the returns to cattle rearing at no lower level. The managerial supports of \textsf{cattle} arm have induced participation and achieved the same level of impacts among the members who would otherwise not take a loan, probably out of their relatively disadvantaged background characteristics. The reason can either the managerial support program complimented the necessary codifiable knowledge, or these participants had the same level of knowledge as other participants but noticed the managerial support program as useful. Either possibility is consisitent with the finding by previous studies that only the experienced or skilled members could reap the benefits of microfinance. Previous studies have targeted the population with a richer set of investment possibilities in a more urbanised setting under which the experience may have a positive return. In the current study, the population resides in a remote, rural area. Even the simpler production process of cattle farming that consists of procuring feeding, grazing, insemination and calving turns out to demand unignorable codifiable skills, or the crystalised intelligence, to participate and sustain in microfinance.
	
	The \textsf{NumCows} row shows the number of cattle owned and it also serves as a check that non-\textsf{traditional} members actually own cattle once the loan/lease is made. The ANCOVA estimates plotted in the figure are net of baseline cattle holding, so even the non-traditional holding estimates sometimes add up to less than 1. %As shown in column (1) of \textsc{\normalsize Table \ref{tab ANCOVA cow time varying}}, both \textsf{traditional} arm and the non-\textsf{traditional} arms increase the cattle ownership. 
	The figure shows that, on average, the non-\textsf{traditional} arms continue to own about .4 more cattle than the \textsf{traditional} arm members, conditional on the initial cattle holding. 
	
	\textsc{\footnotesize Figure \ref{fig CattleHoldingArm}} shows more detailed changes in cattle ownership by arm. Holder rates are the number of cattle owners per arm size, holding size is average holding per owner, initial owner holding are herd size for owners who held cattle at baseline, and per capita holding is mean cattle holding in each arm. Initial owner holding and holder rates show impacts on the intensive and extensive margins, respectively. Per capita holding tracks impacts on both the intensive margins (growth of initial owners) and the extensive margins (growth of new owners). All the indicators are similar across arms at the baseline. 
	
	We see that the holder rates increased in all arms, although the increase was smallest for the \textsf{traditional}. This shows that, even the small upfront lending of \textsf{traditional} arm helped increase catte ownership but to a lesser degree. Without equally large upfront liquidity and with the repayment installments that began immediately, a smaller fraction of borrowers could purchase their first cattle. Holding size increased in all non-\textsf{traditional} arms, while the \textsf{traditional} arm remained stagnant. It is also the \textsf{traditional} arm that has the smallest, or negligible, impacts on the initial owners. \textcolor{red}{[These initial owners, overall, diversified their portfolio rather than increasing the cattle investments. ]} For the non-\textsf{traditional} arm, Initial owner holding size is larger than the average holding size per owner, indicating the higher returns to members with experiences. The per capita holding growth was smallest in the \textsf{traditional} arm. This is due to smaller impacts on the extensive margins (fewer new ownership, smaller growth by new owners) and little impacts on the intensive margins (negligible growth by initial owners). 

	%One also sees that about 79\% of \textsf{traditional} arm members own cattle in period 2. This indicates that even a small loan helped some borrower to increase cattle ownership, but to a smaller magnitude than in the non-\textsf{traditional} arm.

\begin{figure}
\mpage{\linewidth}{
\hfil\textsc{\footnotesize Figure \refstepcounter{figure}\thefigure: Cattle holding by arm\label{fig CattleHoldingArm}}\\
\hfil\includegraphics{c:/data/GUK/analysis/program/figure/EstimationMemo/CattleHoldingByArm.png}\\
\renewcommand{\arraystretch}{1}
\hfil\begin{tabular}{>{\hfill\scriptsize}p{1cm}<{}>{\scriptsize}p{12cm}<{\hfill}}
Source: & Survey data.\\
Note:& \textsf{HolderRates} are the number of cattle owners per arm size, \textsf{HoldingSize} is average holding per owner, \textsf{InitialOownerHolding} are average holding per owner who held cattle at baseline, and \textsf{PerCapitaHolding} is cattle owned per arm member. \textsf{InitialOownerHolding} and \textsf{HolderRates} show impacts on the intensive and extensive margins, respectively. \textsf{PerCapitaHolding} shows the time trend in mean cattle holding.\\[1ex]
\end{tabular}
}
\end{figure}


\begin{figure}
\mpage{\linewidth}{
\hfil\textsc{\footnotesize Figure \refstepcounter{figure}\thefigure: All IGAs\label{fig AllIGAChoices}}\\

\vspace{1ex}
\hfil\includegraphics[height = 6cm, width = 12cm]{c:/data/GUK/analysis/program/figure/EstimationMemo/AllIGAChoicesByNumIGA.pdf}\\
\renewcommand{\arraystretch}{1}
\hfil\begin{tabular}{>{\hfill\scriptsize}p{1cm}<{}>{\scriptsize}p{12cm}<{\hfill}}
Source: & Administrative data, based on the information reported at the weekly meeting. Only borrowing member data are shown.\\
Note:& Contents of IGAs are cattle, goat/sheep, growing cereals (paddy, corn) and nuts, small trades, and house and land leasing. Row panels indicate the total number of IGAs that borrowers own. For example, the row panel under the number `1' indicates the distribution of projects owned by single project members. There is no borrower with only one project in the \textsf{traditional} arm. 
\end{tabular}
}
\end{figure}

\begin{figure}
\mpage{\linewidth}{
\hfil\textsc{\footnotesize Figure \refstepcounter{figure}\thefigure: Contents of first IGA \label{fig FirstIGAChoicesCattleVsOther}}\\
\hfil\includegraphics[height = 4cm, width = 8cm]{c:/data/GUK/analysis/program/figure/EstimationMemo/FirstIGAChoicesCattleVsOther.pdf}\\
\renewcommand{\arraystretch}{1}
\hfil\begin{tabular}{>{\hfill\scriptsize}p{1cm}<{}>{\scriptsize}p{12cm}<{\hfill}}
Source: & Administrative data, based on the information reported at the weekly meeting. Only borrowing member data are shown.\\
Note:&  The first IGA is defined as the oldest IGA for the household. Blue bars are the cattle rearing, white bars are the sum of all other projects listed in \textsc{\footnotesize Figure \ref{fig AllIGAChoices}}. \\[1ex]
\end{tabular}
}
\end{figure}


% \begin{figure}
% \mpage{\linewidth}{
% \hfil\textsc{\footnotesize Figure \refstepcounter{figure}\thefigure: All IGA choices\label{fig AllIGAChoicesCollapsed}}\\
% 
% \vspace{1ex}
% \hfil\includegraphics[height = 4cm, width = 12cm]{    paste0(pathprogram, "figure/ImpactEstimationOriginal1600Memo3/AllIGAChoicesCollapsed.eps")}\\
% \renewcommand{\arraystretch}{1}
% \hfil\begin{tabular}{>{\hfill\scriptsize}p{1cm}<{}>{\scriptsize}p{12cm}<{\hfill}}
% Source: & Administrative data. Only borrowing member data are shown.\\
% Note:& Based on information reported at the weekly meeting. The figure shows the sum collapsed over the total number of projects in each arms of \textsc{\footnotesize Figure \ref{fig AllIGAChoices}}. 
% \end{tabular}
% }
% \end{figure}
% 
	To understand the reasons behind the slower pace of asset accumulation of \textsf{traditional} arm, in \textsc{\small Figure \ref{fig AllIGAChoices}}, we plot borrower's reported income generating activities (IGAs) separately by the total number of projects that the borrowers report. Contents of IGAs are cattle, goat/sheep, growing cereals (paddy, corn) and nuts, small trades, and house and land leasing. The row panel headed by the number `1' indicates the distribution of projects among single-project owners, `2' indicates the distribution among double-project owners, and so on. This shows that almost no one of the \textsf{traditional} arm invested only in one project while only few members did so in the non-\textsf{traditional} arms. We also note that there are a significant number of cases in the \textsf{traditional} arm that members reportedly raise cattle, yet they are also accompanied by pararell projects in smaller livestock production and small trades. Popularity of small trades and smaller livestock for the \textsf{traditional} arm members is consistent with convexity in the production technology of large domestic animals under a liquidity constraint. This also validates our supposition in experimental design that cattle production is the most preferred and probably the only economically viable investment choice. It eases a concern that the \textsf{cattle} arm may have imposed an unnecessary restriction in an investment choice by forcing to receive a heifer. 



	\textsc{\footnotesize Figure \ref{fig FirstIGAChoicesCattleVsOther}} shows the first IGAs of members. The first IGA is defined as the oldest IGA for the household. For most of the households, the oldest IGA had started after the baseline, and it is the IGA with the largest cash flow. Of course, there are a small percentage of households with an existing IGA before the baseline, but, with randomisation, the fraction of such households are similar across arms. Therefore, the between arm comparison of the first IGA gives us an idea about how the households had chosen the initial investments. In the \textsf{traditional} arm, there are 33 borrowing members who report cattle as their first IGA, and 76 borrowing members (69.72\%) who report other than cattle as their first IGA. This contrasts with the non-\textsf{traditional} arms that 466 borrowing members who report cattle as their first IGA and 25 borrowing members (5.09\%) other than cattle as their first IGA. As can be seen from \textsc{\small Figure \ref{fig AllIGAChoices}}, for the 2nd and 3rd IGAs, a diversified IGA portfolio is continued to be held by all the \textsf{traditional} arm borrowers, and only the minority of non-\textsf{traditional} arm borrowers has a diversified portfolio.

	%\textsc{\small Figure \ref{fig AllIGAChoicesCollapsed}} collapses the all reported projects over borrowers and shows the total number of IGAs in each arms. 
	
	
\begin{figure}
\mpage{12cm}{
\hfil\textsc{\footnotesize Figure \refstepcounter{figure}\thefigure: Cumulative effects on income and consumption\label{fig IncomeConsumptionEffects}}\\

\vspace{2ex}
\hspace{-2em}\includegraphics[height = 8cm, width = 14cm]{c:/data/GUK/analysis/program/figure/EstimationMemo/ConsumptionIncomeCumRelativeToConcurrentTradEffects.pdf}\\
\renewcommand{\arraystretch}{1}
\setlength{\tabcolsep}{1pt}
\hfil\begin{tabular}{>{\hfill\scriptsize}p{1cm}<{}>{\scriptsize}p{12.5cm}<{\hfill}}
Source: & Constructed from ANCOVA estimation results 
\textsc{Table \ref{tab ANCOVA consumption timevarying}}, \textsc{Table \ref{tab ANCOVA consumption timevarying attributes original HH}}, \textsc{Table \ref{tab ANCOVA labour incomes timevarying}}, \textsc{Table \ref{tab ANCOVA labour incomes timevarying attributes}}.\\
Note:& Panels show cumulative impacts of respective arm or attributes \textsf{X} relative to \textsf{tradiotional} arm which are obtained by $\Delta\mbox{2nd period}=\mbox{\textsf{intercept}}+\mbox{\textsf{X}}, \Delta\mbox{3rd period}=\Delta\mbox{2nd period}+\mbox{\textsf{Period3}}+\mbox{\textsf{X*Period3}}$, $\Delta\mbox{4th period}=\Delta\mbox{2nd period}+\Delta\mbox{3rd period}+\mbox{\textsf{Period4}}+\mbox{\textsf{X*Period4}}$. Bars show 95\% confidence intervals using cluster robust standard errors. \textsf{Consumption} is annualised per capita consumption in Taka. Per capita consumption is a total of food, hygiene, social, and energy expenditure divided by the number of household members. In-kind consumption of home made products is imputed at median prices. \textsf{Incomes} is labour incomes of household in 1000 Taka units. \\[1ex]
\end{tabular}
}
\end{figure}

	\textsc{\small Figure \ref{fig IncomeConsumptionEffects}} shows impacts on consumption and labour incomes. Style and placement of panels follow the \textsc{\footnotesize Figure \ref{fig NetAssetsLivestockEffects}}. Consumption is not measured at the baseline, so we do not use it to understand the welfare impacts but to understand how the members have dealt with the loan repayment through consumption choices. Given randomisation, one can still identify impacts on repayment efforts in terms of consumption suppression relative to the \textsf{traditional} arm. In obtaining ANCOVA estimates, we condition on period 2 consumption. \textcolor{red}{[This can be problematic as period 2 consumption is correlated with arm assignment. But the results do not change if we estimate without period 2 consumption as a covariate in specification 1.]} Consumption is per capita expenditure of the household. Labour income is a household level variable and measures earnings from casual jobs. Both consumption and labour incomes do not show any impact by the arms or functional attributes.

\begin{figure}
\renewcommand{\arraystretch}{.6}
\mpage{\linewidth}{
\hfil\textsc{\footnotesize Figure \refstepcounter{figure}\thefigure: Mean consumption and labour incomes by arm and period\label{fig MeanOutcomes}}\\
%\hfil\includegraphics[width = 12cm]{    paste0(pathprogram, "figure/EstimationMemo/MeanOutcomesByArmAndPeriod.pdf")}\\
\hfil\includegraphics[width = 12cm, height = 6cm]{c:/data/GUK/analysis/program/figure/EstimationMemo/MeanConsumptionIncomeByArmAndPeriod.pdf}\\
\renewcommand{\arraystretch}{1}
\hfil\begin{tabular}{>{\hfill\scriptsize}p{1cm}<{}>{\scriptsize}p{12cm}<{\hfill}}
Source: & Survey data.\\
Note:& Points indicate means, vertical bars indicate 95\% confidence intervals. %\textsf{NumCows} is number of cattle owned. \textsf{NetAssets} is total assets less debt outstanding to all sources. 
\textsf{Consumption} is annualised per capita consumption in Taka. Per capita consumption is a total of food, hygiene, social, and energy expenditure divided by the number of household members. In-kind consumption of home made products is imputed at median prices. \textsf{Incomes} is labour incomes of household in 1000 Taka units. %\textsf{Sch0512}, \textsf{Sch1315}, \textsf{Sch1618} are enrollment at primary, secondary, and tertiary levels. \textsf{F} and \textsf{M} are female and male enrollment, respectively. 
\end{tabular}
}
\end{figure}

	 %In consumption estimation, the estimates of \textsf{traditional} in specification 2 and 3 differ significantly as the latter involves baseline household size. The impacts of non-\textsf{traditional} arms are almost zero in all panels. This is in contrast to assets where we saw an increase in cattle, land holding and net assets. 
	In \textsc{\footnotesize Figure \ref{fig MeanOutcomes}}, we see that, across arms, the labour income is increasing from period 3, and per capita consumption did not change between periods 3 and 4 despite growths in labour incomes.\footnote{One notes that the labour income is lowest in period 2 for all non-\textsf{traditional} arms, second lowest for the \textsf{traditional} arm, and start increasing from period 3. The fall in period 2 is due to the floods. Period 2 consumption is reportedly lower than period 3 and 4 because of flood damages. } %As labour income growths and steady consumption are uniformly observed, it is accrued to the loan and its repayment, not to the functional attributes. 
	The households seem to have put asset accumulation and repayment a priority before consumption growths. It indicates that the borrowers did not choose to strategically default but tried to repay. 


\begin{figure}
\hfil\textsc{\footnotesize Figure \refstepcounter{figure}\thefigure: Cumulative weekly repayment rates\label{fig weeklysavingrepayrate}}\\
\hfil\includegraphics{c:/data/GUK/analysis/program/figure/ImpactEstimationOriginal1600Memo2/CumulativeWeeklyRepaymentRateByPovertystatus.png}\\
\renewcommand{\arraystretch}{1}
\hfil\begin{tabular}{>{\hfill\scriptsize}p{1cm}<{}>{\scriptsize}p{12cm}<{\hfill}}
Note:& Each dot represents weekly observations. Only members who received loans are shown. Each panel shows ratio of cumulative repayment sum to cumulative due amount sum, ratio of sum of cumulative repayment and cumulative net saving (saving - withdrawal) sum to cumulative due amount sum, both are plotted against weeks after first disbursement. Value of 1 indicates the member is at per with repayment schedule. Horizontal lines has a $Y$ intercept at 1. Lines are smoothed lines with a penalized cubic regression spline in \textsf{ggplot2::geom\_smooth} function, originally from \textsf{mgcv::gam} with \textsf{bs=`cs'}. \\[-1ex]
\end{tabular}
\end{figure}


	\textsc{\footnotesize Figure \ref{fig weeklysavingrepayrate}} shows the repayment results. Top panel shows the ratios of cumulative repayment to cumulative planned installment, the bottom panel shows the ratios of sum of cumulative repayment and cumulative net saving (saving - withdrawal) to cumulative planned installment. Both are plotted against weeks after first disbursement. Each dot represents a member at each time point. Value of 1, which is given by a horizontal line, indicates the member is at per with repayment schedule. Some members saved more than the required repayment at each time points that go beyond 1 in the figure. One sees that repayment rates are above 1 at the beginning but stay below 1 for most of the time. The majority of borrowing members did not repay the loan by the 48th month with prespecified installments. One notes the \textsf{traditional} arm has more of lower repayment rates among all arms. When a member does not reach the due amount with installments, they had to repay from the (net) saving, an arrangement to which the lender and the borrowers made at the loan contract signment. Repayment rates after using net saving are 44.71, 93.57, 97.01, 95.42\%, respectively, for \textsf{traditional, large, large grace, cow} arms and 87.85\% for overall (from \textsf{\footnotesize AllMeetingsRepaymentInitialSample.rds}). \textcolor{red}{[Abu-san: Why does the admin data continue up to the 48th month, not 36th?]} The low repayment rates among \textsf{traditional} arm borrowers may be due to our experimental design that a new loan is disbursed unconditionally up to three cycles, lacking the dynamic incentives to repay, or due to the fact that they had lower returns on their investments. Our finding of labour income growths and the steady consumption indicates the latter possibility is more likely. 

	There is little difference in repayment rates by poverty classes. \textsc{\footnotesize Figure \ref{fig weeklysavingrepayrate}} depicts both moderately poor and ultra poor in different colours. It is impossible to distinguish between them with eyeballs, and ANCOVA estimates also confirm this  (see Appendix \ref{AppendixEstimationTables}, \textsc{\small Table \ref{tab shortfall indiv}, \ref{tab shortfall indiv2}} for details). We also observe that impacts on all outcome measures are not statistically different between the poverty classes (see Appendix \ref{AppendixEstimationTables}). This is in contrast to a popular belief that the ultra poor are the riskiest among all income classes. %Poverty gradation through a participatory process, however, does not distinguish the moderately poor and the ultra poor on the observables. \textsc{\footnotesize Figure \ref{fig NetAssetValuesAtRd1}} shows net asset values at baseline by poverty class, and \textsc{\footnotesize Figure \ref{fig LivestockValuesAtRd1}} shows initial livestock values at baseline by poverty class. Both show little difference in these observable characteristics. \textcolor{red}{[According to Abu-san, participatory poverty gradation may have been imprecise.]}

	Smaller cumulative impacts and lower repayment rates of \textsf{traditional} arm members stand out once we acknowledge that they are receiving an equivalent amount and their contract differs with other arms only in the attributes we focus. These differences arose partly from the different investment choices observed in \textsc{\small Figure \ref{fig AllIGAChoices}, \ref{fig FirstIGAChoicesCattleVsOther}}, which were induced by the lack of \textsf{Upfront} functional attribute in lending.
	
% \mpage{12cm}{
% \hfil\textsc{\footnotesize Figure \refstepcounter{figure}\thefigure: Repayments\label{fig Repayments}}\\
% 
% \vspace{2ex}
% \hspace{-2em}\includegraphics[height = 10cm, width = 12cm]{       paste0(pathprogram, "figure/ImpactEstimationOriginal1600Memo3/Repayments.eps") }\\
% \renewcommand{\arraystretch}{1}
% \hfil\begin{tabular}{>{\hfill\scriptsize}p{1cm}<{}>{\scriptsize}p{11cm}<{\hfill}}
% Source: & Constructed from FD estimation results.\\
% Note:& CumImpactText3 \\[1ex]
% \end{tabular}
% }
%
%	Annualised repayment is depicted in \hfil\textsc{\footnotesize Figure \ref{fig Repayments}}. The top three panels show net saving. As in \textsc{\footnotesize Figure \ref{fig LivestockCumulativeEffects}}, each subpanel shows cumulative changes, per period changes, and differences in changes relative to concurrent changes of \textsf{traditional} arm. 
%
% \textsf{InKind} attribute does not increase repayment by itself. Repayment is highest with the \textsf{Upfront} attribute. It is driven by the initial year repayment and subsequent repayment is smaller than \textsf{traditional}. With \textsf{WithGrace} attribute, repayment is larger by more than Tk. 15000 in total during period 2 and 4 due to the accumulated saving in period 1 or the grace period. For net saving, there is a steady increase in all arms. \textsf{Upfront} and \textsf{WithGrace} attributes see a large boost in period 1 and the growth relative to \textsf{traditional} becomes negative subsequently.



\begin{figure}
\hfil\mpage{12cm}{
\hfil\textsc{\footnotesize Figure \refstepcounter{figure}\thefigure: Period wise effects on schooling\label{fig SchoolingEffects}}\\

\vspace{2ex}
\hfil\includegraphics[height = 12cm, width = 12cm]{c:/data/GUK/analysis/program/figure/EstimationMemo/SchoolingEffects.png}\\
\renewcommand{\arraystretch}{1}
\hfil\begin{tabular}{>{\hfill\scriptsize}p{1cm}<{}>{\scriptsize}p{11cm}<{\hfill}}
Source: & Constructed from ANCOVA estimation results
\textsc{Table \ref{tab ANCOVA enroll time varying1}}, \textsc{Table \ref{tab ANCOVA enroll time varying attributes}}.\\
Note:&  Left most column panel shows the conditional means of \textsf{traditional} arm which serves as a benchmark in estimating impacts. In other column panels, all points show the relative difference from concurrent \textsf{traditional} levels depicted in the left most column. \textsf{Large} and \textsf{Upfront} are the same values. Other column panels are grouped either by arm or by attribute. Row panels show different outcomes. Bars show 95\% confidence intervals using cluster robust standard errors.  \\[1ex]
\end{tabular}
}
\end{figure}

	In \textsc{\footnotesize Figure \ref{fig SchoolingEffects}}, the effects on child school enrollment are displayed. Unlike the previous figures, \textsf{traditional} column shows the conditional mean values and other non-\textsf{traditional} columns show per period impacts relative to the concurrent \textsf{traditional} arm values. What we display in the non-\textsf{traditional} columns are per period impacts, not the cumulative impacts. We chose to show per period impacts because annual enrollment status matters in schooling. In general, there is no detectable impact of the intervention, except for a negative impact for women at the college level for \textsf{Upfront} in period 4 and a positive impact for women at the college level for \textsf{WithGrace} in period 4. Women at the college level are about 5.9\% of sample, so the effective sample size of each cell is about 11-12 (=800*.056/4), and it is difficult to interpret the results on these small samples. If anything, negative impacts of elder girl's schooling may be due to stronger demand for cattle production in a household. This is in line with the finding in rejection that the limited household size can be a constraint on participation, especially when there is no grace period. Cattle ownership naturally shifts the relative shadow prices in a household against child schooling, especially for the elder girls as their returns on human capital are considered to be lower than younger girls, and the task contents of cattle rearing labour are less brawn intensive yet requires to be above the primary school ages. This may be a potential downside of having greater cattle production in a household.

	In summary, we found that our managerial support programs induce the members of disadvantaged background to participate in microfinance, achieving the further outreach, and achieve the results that are no different with other borrowers. This is consistent with the finding of the previous studies that a certain level of skills is necessary for participation, and our managerial support progams supplemented the lack thereof. 	We found that the large upfront disbursements allow borrowers to invest in cattle while members with sequential disbursements mostly opted for smaller livestock and small trades. In combination with a greater return to cattle on net asset accumulation and a greater rate of loan repayment, we consider it as evidence of a poverty trap and an effective measure to break it. We also found the impacts and repayment rates are indistinguishable between the moderately poor and the ultra poor.




\section{Theory}
\label{SecTheory}

% \begin{mdframed}[style={SecItemize}, frametitle={Theory}]
% \begin{itemize}
% \vspace{1.0ex}\setlength{\itemsep}{1.0ex}\setlength{\baselineskip}{12pt}
% \item	Contour of two production functions, a nonconvex production set, gives rise to a poverty trap
% \item	Goats relative to cows as an investment: Infrequent income stream, limited local consumption, vulnerability to logging water, a herd is less mobile
% \item	Goat returns net of mortality are lower (not generally, only in this area) and one cannot scale up goats: Takes long to switch to cow ownership
% %\item	No saving constraint required, saving = depreciation at equilibria
% \item	The entire region depicted in the diagram represents poverty, so it shows a poverty trap within poverty (i.e., ultra poor and moderately poor)
% \item	We are not going to show the production nonconvexity, instead we show lower repayment rates and smaller cattle holding for a smaller loan size, just as \citet{BandieraBRAC2017} did
% \end{itemize}
% \end{mdframed}

\begin{figure}
\hspace{-3em}\mpage{\linewidth}{
\hfil\textsc{\normalsize Figure \refstepcounter{figure}\thefigure: A poverty trap with goats and cattle\label{fig poverty trap}}\\

\noindent\mpage{\linewidth}{
\input{GoatCowProdFunctions.tkz}\hspace{-2em}
\input{poverty_trapCompoundScurve.tkz}}

\noindent\hfil\mpage{12cm}{\footnotesize Note: The current period per capita herd size $k_{t}$ is on the horizontal axis, the next period per capita herd size $k_{t+1}$ is on the vertical axis. The production function for goats $g(k)$ is multiplied with a fixed saving rate $s$ and is added current herd size net of mortality $(1-\delta)k_{t}$ that is passed on to the next period per capita herd size. Depreciation is zero for the fixed cost segment. Similar description applies to the cow production. The left figure shows each production sets, the right figure shows the contour of two production sets. Red points are stable equilibria, blue points are unstable equilibria. \setlength{\baselineskip}{8pt} }
}
\end{figure}

\vspace{2ex}
	In this section, we use a simplified version of \citet{GalorZeira1993} to illustrate a theoretical framework to aid the interpretation of the empirical finding that asset accumulation is faster while the repayment rate is higher for upfront lending. Let us consider that there are two production sets called `goats' and 'cattle.' Both sets are nonconvex with fixed inputs as shown in \textsc{\small Figure \ref{fig poverty trap} (left panel)}, with period $t$ per capita herd size in monetary units $k_{t}$ on the horizontal axis and the period $t+1$ per capita herd size in monetary units $k_{t+1}$ on the vertical axis. The next period net per capita herd size is given by carry over herd size net of mortality and its addition with saving. When a production set has a fixed input portion, carry over herd size is zero over that portion and becomes non-negative once production becomes positive after $\underline{k}\in\mathbb R_{++}$. This is given as the flat segment next to the origin of $sg(k_{t})+(1-\delta)k_{t}$ for goats and $sf(k_{t})+(1-\delta)k_{t}$ for cattle. For $k_{t}>\underline{k}$, taking the cattle as an example, next period net per capita herd size traces each production sets after rescaling with saving rate $s$, or $sf(k)$, and linearly deducting the depreciation with $(1-\delta)k_{t}$, so nonconvex production sets are shown in the figure. We assume the population size, saving rate $s$, and depreciation rate $\delta$ are fixed. We note from the previous section that the returns to goats net of mortality are lower in the region depicted in \textsc{\small Figure \ref{fig poverty trap}}, and the steady state goat herd size is small in their livestock values. We also note that a goat investment, when compared to a heifer investment, requires smaller upfront costs but has an infrequent income stream, faces a more limited local demand, shows vulnerability to logging water, all pointing to smaller investments and their returns. We will use these points to assume that the fixed costs and steady state production level are smaller for goats than cattle.

	When there is only a goat production technology, individuals eventually reaches the point $G$, a steady state where the per capita herd size is constant, or $k_{t+1}=k_{t}$. When the cattle production technology is added to the picture, there is no change in the equlibrium for individuals whose initial assets are in $[k_{1}, k_{2})$. For individuals with initial assets in $[k_{2}, \infty)$, one chooses cattle, because the resulting income level is higher, and eventually arrive at the steady state $C$.\footnote{$k_{2}$ is an unstable equilibrium that no individual would deviate from, but we include this point to the region of attraction of $C$ for the sake of simplicity. } 

	Over the domain of $k_{t}\in[0, \infty)$, the production possibility frontier, or the contour of the union of two production sets, becomes M-shaped (\textsc{right panel}). Under the configuration depicted in the figure, there will be five equilibria of which three are stable. Ruling out the zero equilbrium as irrelevant, one is left with two stable equilibria, named as goats and cattle in the figure.\footnote{A similar diagram is found in \citet[][Figure 3, with $k-y$ space]{KraayMcKenzie2014}. }

	Formally, one requires the production set $j=\{\mbox{goat, cattle}\}$ to satisfy: there exists $\underline{k}_{j}>0$ that the production is zero for input $k<\underline{k}_{j}$ and is strictly positive for $k\geqslant\underline{k}_{j}$. We assume the production set exhibits decreasing returns to scale for $k\geqslant\underline{k}_{j}$. Let the contour of the production set be $f_{j}(k)$. Assume for expositional simplcity that there is a fixed saving rate $s$. Further assume that there exists $k_{2}>\underline{k}_{j}$ such that $sf_{j}(k)+(1-\delta)k>k$ for $k\in(k_{2}, k^{*})$, with $k^{*}>k_{2}$ is a fixed point $k^{*}=sf_{j}(k^{*})+(1-\delta)k^{*}$. Under these assumptions, decreasing returns ensure there exists two intersections between the steady state line, one unstable and one stable equilibria.\footnote{In \textsc{Figure \ref{fig poverty trap}}, depreciation below $\underline{k}$ is not accounted as capital cannot be negative. Once the production starts for $k>\underline{k}$, the contour shows net of depreciation so $sf(k)+(1-\delta)k$. } 

	In light of this argument, a loan that is larger than $k_{2}$ allows individuals in the goat equilibrium to transition to cattle production and arrive at the cattle equilibrium. If the lending market is competitive, the interest rate is the same as the return on capital and thus lending, not a transfer, suffices for the transition, so long as the upperbound of the loan size is no smaller than $k_{2}$. The entire region depicted in the diagram is considered as in the realm of poverty, so it shows a poverty trap within poverty (i.e., goat as ultra poor and cattle as moderately poor). 

	In the empirical section, we followed \citet{BandieraBRAC2017} and took the production nonconvexity as given and interpreted the lower repayment rates and smaller cattle holding for a smaller upfront loan size as evidence consistent with a poverty trap. 

\section{Conclusion}
\label{SecConclusion}

\begin{mdframed}[style={SecItemize}, frametitle={Conclusion}]
\begin{itemize}
\vspace{1.0ex}\setlength{\itemsep}{1.0ex}\setlength{\baselineskip}{12pt}
\item	Entrepreneurship is necessary for project success, even with a simpler production process.
\item	Upfront liquidity increases asset holding and repayment rates.
\item	Cattle has higher returns and lower risks, resulting in higher repayment rates, but also has larger initial fixed costs, possibly generating a poverty trap.
\item	Lending uptake is impeded by small household size, asset shocks, and a lack of supports for managerial capacity.
\item	If these are relaxed, a poverty trap may be overcome.
\item	In the remote rural setting, larger upfront loan suited to the project cash flow is shown to be Pareto improving, despite widely believed fears of inefficiency due to information asymmetry.
\item	Consumption and labour incomes were not affected in non-\textsf{traditional} arms. Labour incomes increased toward the end of repayment for all arms which can be a repayment effort.
\item	Schooling was not affected in general. It finds a sign of a loss to college level women, hinting a domestic labour constraint in cattle production. But there was also a positive impact for women at the college level in \textsf{WithGrace} arm. While these are possibilities, cell sample sizes are too small to draw anything conclusive.
\end{itemize}
\end{mdframed}

	The poverty reduction impacts of microfinance was a firm belief in the early days of microfinance. Yet it suffered from a puzzling weak spot that microfinance is slow to reach the ultra poor, which is still debated today. Recently, even the poverty reduction impacts are subject to doubts, and it has been shown that the only borrowers with experience or skills are able to leap benefits. In this study, we examined the role of entrepreneurship in leaping benefits. We showed, under the rural setting, experiences or entrepreneurship seem to matter for participation and resulting impacts. We note the usefulness of having consulting services available for the prospective clients of MFIs when expanding the credit to the ultra poor. 

	This study employs a stepped-wedge design of multiple arms to isolate different functional attributes of loan contract: Frontloading, a grace period, and in-kind loan with management supports. These map to a liquidity constraint, a saving constraint, and an entrepreneurship constraint. Only frontloading the disbursement matters in all outcomes, which signifies the importance of a liquidity constraint. With evidence that borrowers with frontloaded arms invested in cattle while the borrowers under incremental lending invested in multiple, smaller projects, and the repayment rates are higher for the frontloaded arms, we conclude that there is a poverty trap which cannot be overcome by the traditional approach of microfinance. Under the study's setting, escaping from the poverty trap requires frontloading the lending, not lending incrementally as practiced by the majority of microfinance institutions. In addition, lending rather than a transfer may suffice to support the transition. 
	
	While we did not observe impacts of managerial supports, we found that more members with disadvantaged background participated. This implies that managerial supports can invite more disadvantaged prospective borrowers without adversely affecting the outcomes. To expand the coverage to the ultra poor, it may be useful to have consulting services.
	
	We have witnessed that a binding domestic capacity constraint may impede potential borrowers from participation. This limits the potential benefit of lending a larger amount from the start of the program. While it in unclear why the outsourced labour cannot substitute the domestic labour, one can consider organising an arrangement in each group, tended by the group members, to collectively graze the cattle during the daytime. This partly eases the domestic labour and/or space constraints faced by small households. 

	We note that our study site is rich in rainfall, giving more advantages to cattle production over sheep/goat production. In contrast, if the climate is more arid, sheep and goats are better suited because of less water logging and their greater viability in relying on natural grass. This raises a concern that our results may not directly transferrable to more arid areas. However, the greater point of the lesson from the study is the presence of fixed inputs in scaling the herd size. While sheep/goats are easier to scale than cattle, it will require larger land and roofed facilities at some point. This can effectively form nonconvexity in the production set, and large enough lending may allow herders to go pass the threshold. 

	We have seen that borrowers accumulated assets, increased labour supplies, but not increasing the consumption. This is consistent with a high morale of repayment, which can partly be explained by the lack of alternative lenders in the study area. With stronger incentives to repay, the evidence on stronger repayment discipline of large sized arm members need not generalise in the areas outside the study site. On the other hand, the necessity of codifiable knowledge in participation even for a simple production process and the scope for escaping the poverty trap with large, frontloaded lending may be more generalisable to other rural areas that are suited to livestock production.

{\footnotesize\bibliographystyle{aer}
\setlength{\baselineskip}{8pt}
\bibliography{c:/seiro/settings/TeX/seiro}
}

\appendix
\setcounter{section}{0}
\setcounter{figure}{0}
\setcounter{table}{0}
\renewcommand{\thefigure}{\Alph{section}\arabic{figure}}
\renewcommand{\thetable}{\Alph{section}\arabic{table}}
\renewcommand{\thesection}{\Alph{section}}




\section{Randomisation checks}
\label{AppSecRandomisation}
\setcounter{table}{0}

\hspace{-1.5cm}\begin{minipage}[t]{14cm}
\hfil\textsc{\normalsize Table \refstepcounter{table}\thetable: Permutation test results\label{tab perm}}\\
\setlength{\tabcolsep}{.5pt}
\setlength{\baselineskip}{8pt}
\renewcommand{\arraystretch}{.50}
\hfil\begin{tikzpicture}
\node (tbl) {\input{c:/data/GUK/analysis/save/EstimationMemo/PermutationTestResults.tex}};
\end{tikzpicture}\\
\renewcommand{\arraystretch}{.8}
\setlength{\tabcolsep}{1pt}
\begin{tabular}{>{\hfill\scriptsize}p{1cm}<{}>{\hfill\scriptsize}p{.25cm}<{}>{\scriptsize}p{12cm}<{\hfill}}
Source:& \multicolumn{2}{l}{\scriptsize Estimated with GUK administrative and survey data.}\\
Notes: & 1. & \textsf{R}'s package \textsf{coin} is used for baseline group mean covariates to conduct approximate permutation tests. Number of repetition is set to 100000. Number of groups is 72. %Step-down method is used to adjust for multiple testing of a multi-factor grouping variable.
\\
& 2. & ${}^{***}$, ${}^{**}$, ${}^{*}$ indicate statistical significance at 1\%, 5\%, 10\%, respetively. Standard errors are clustered at group (village) level.
\end{tabular}
\end{minipage}

\vspace{2ex}
\mpage{\linewidth}{
\renewcommand{\arraystretch}{.6}
\hfil\textsc{\normalsize Table \refstepcounter{table}\thetable: Anova results for cattle holding equality by arm\label{table anova CattleHoldingArm}}\\
\hfil\input{c:/data/GUK/analysis/program/table/EstimationMemo/anovaCowResults.tex}\\
\renewcommand{\arraystretch}{1}
\hfil\begin{tabular}{>{\hfill\scriptsize}p{1cm}<{}>{\scriptsize}p{12cm}<{\hfill}}
Source: & Survey data.\\
Note:& Each column uses respective year cattle ownership information. Columns (1) to (5) tests cattle holding equality for each survey rounds. In column (2), we edited the data by assigning 1 to members of \textsf{cow} arm who report holding is NA or zero. For ANOVA and Kruskal-Wallis, each entry indicates $p$ values. ANOVA tests for the null of equality of all means under normality. Kruskal-Wallis tests for the null of no stochastic dominance among samples without using the normality assumption. Tukey's honest significant tests show difference in means and $p$ values in parenthesis that account for multiple testing under the normality assumption.  \\[1ex]
\end{tabular}}


\section{Rejection}
\label{AppSecRejection}
\setcounter{table}{0}



Among 800 observations, there are 4 whose villages are washd away and 70 who by group rejected the assigned arms which are traditional, large, large grace with 40, 20, 10, 0 individuals, respectively. There are 31, 9, 13, 37 individuals who individually rejected traditional, large, large grace, cow, respectively. %Among attrited HHs, when were they lost?

%Reasons for attrition and relation to flood damage.

Use \textsf{coin} package's \textsf{independence\_test}: Approximate permutation tests by randomly resampling 100000 times.

% below form permutation tables for all groups examined in AttritionTestContents2.rnw






\hfil\begin{minipage}[t]{14cm}\hfil\textsc{\normalsize Table \refstepcounter{table}\thetable: Permutation test results of rejection\label{tab1 Permutation test results of rejection}}\\\setlength{\tabcolsep}{.5pt}\setlength{\baselineskip}{8pt}\renewcommand{\arraystretch}{.50}\hfil\begin{tikzpicture}\node (tbl) {\input{c:/data/GUK/analysis/save/EstimationMemo/RejectedPermutationTestResultso800.tex}};\end{tikzpicture}\\\begin{tabular}{>{\hfill\scriptsize}p{1cm}<{}>{\hfill\scriptsize}p{.25cm}<{}>{\scriptsize}p{12cm}<{\hfill}}Source:& \multicolumn{2}{l}{\scriptsize Estimated with GUK administrative and survey data.}\\Notes: & 1. & \textsf{R}'s package \textsf{coin} is used for baseline mean covariates to conduct approximate permutation tests. Number of repetition is set to 100000. Step-down method is used to adjust for multiple testing of a multi-factor grouping variable. \textsf{Attrited} and \textsf{Nonattrited} columns show means of each group. For \textsf{Arm}, proportions of non-traditional arm are given. \\& 2. & ${}^{***}$, ${}^{**}$, ${}^{*}$ indicate statistical significance at 1\%, 5\%, 10\%, respetively. Standard errors are clustered at group (village) level.\end{tabular}\end{minipage}\\\vspace{2ex}\hfil\begin{minipage}[t]{14cm}\hfil\textsc{\normalsize Table \refstepcounter{table}\thetable: Permutation test results of rejection among traditional arm\label{tab1 Permutation test results of rejection among traditional arm}}\\\setlength{\tabcolsep}{.5pt}\setlength{\baselineskip}{8pt}\renewcommand{\arraystretch}{.50}\hfil\begin{tikzpicture}\node (tbl) {\input{c:/data/GUK/analysis/save/EstimationMemo/RejectedInTradPermutationTestResultso800.tex}};\end{tikzpicture}\\\begin{tabular}{>{\hfill\scriptsize}p{1cm}<{}>{\hfill\scriptsize}p{.25cm}<{}>{\scriptsize}p{12cm}<{\hfill}}Source:& \multicolumn{2}{l}{\scriptsize Estimated with GUK administrative and survey data.}\\Notes: & 1. & \textsf{R}'s package \textsf{coin} is used for baseline mean covariates to conduct approximate permutation tests. Number of repetition is set to 100000. Step-down method is used to adjust for multiple testing of a multi-factor grouping variable. \textsf{Attrited} and \textsf{Nonattrited} columns show means of each group. For \textsf{Arm}, proportions of non-traditional arm are given. \\& 2. & ${}^{***}$, ${}^{**}$, ${}^{*}$ indicate statistical significance at 1\%, 5\%, 10\%, respetively. Standard errors are clustered at group (village) level.\end{tabular}\end{minipage}\\\vspace{2ex}\hfil\begin{minipage}[t]{14cm}\hfil\textsc{\normalsize Table \refstepcounter{table}\thetable: Permutation test results of rejection among non-traditional arm\label{tab1 Permutation test results of rejection among non-traditional arm}}\\\setlength{\tabcolsep}{.5pt}\setlength{\baselineskip}{8pt}\renewcommand{\arraystretch}{.50}\hfil\begin{tikzpicture}\node (tbl) {\input{c:/data/GUK/analysis/save/EstimationMemo/RejectedInNonTradPermutationTestResultso800.tex}};\end{tikzpicture}\\\begin{tabular}{>{\hfill\scriptsize}p{1cm}<{}>{\hfill\scriptsize}p{.25cm}<{}>{\scriptsize}p{12cm}<{\hfill}}Source:& \multicolumn{2}{l}{\scriptsize Estimated with GUK administrative and survey data.}\\Notes: & 1. & \textsf{R}'s package \textsf{coin} is used for baseline mean covariates to conduct approximate permutation tests. Number of repetition is set to 100000. Step-down method is used to adjust for multiple testing of a multi-factor grouping variable. \textsf{Attrited} and \textsf{Nonattrited} columns show means of each group. For \textsf{Arm}, proportions of non-traditional arm are given. \\& 2. & ${}^{***}$, ${}^{**}$, ${}^{*}$ indicate statistical significance at 1\%, 5\%, 10\%, respetively. Standard errors are clustered at group (village) level.\end{tabular}\end{minipage}\\\vspace{2ex}\hfil\begin{minipage}[t]{14cm}\hfil\textsc{\normalsize Table \refstepcounter{table}\thetable: Permutation test results of rejecters, traditional vs. non-traditional arm\label{tab1 Permutation test results of rejecters, traditional vs. non-traditional arm}}\\\setlength{\tabcolsep}{.5pt}\setlength{\baselineskip}{8pt}\renewcommand{\arraystretch}{.50}\hfil\begin{tikzpicture}\node (tbl) {\input{c:/data/GUK/analysis/save/EstimationMemo/TradNonTradRejectedPermutationTestResultso800.tex}};\end{tikzpicture}\\\begin{tabular}{>{\hfill\scriptsize}p{1cm}<{}>{\hfill\scriptsize}p{.25cm}<{}>{\scriptsize}p{12cm}<{\hfill}}Source:& \multicolumn{2}{l}{\scriptsize Estimated with GUK administrative and survey data.}\\Notes: & 1. & \textsf{R}'s package \textsf{coin} is used for baseline mean covariates to conduct approximate permutation tests. Number of repetition is set to 100000. Step-down method is used to adjust for multiple testing of a multi-factor grouping variable. \textsf{Attrited} and \textsf{Nonattrited} columns show means of each group. For \textsf{Arm}, proportions of non-traditional arm are given. \\& 2. & ${}^{***}$, ${}^{**}$, ${}^{*}$ indicate statistical significance at 1\%, 5\%, 10\%, respetively. Standard errors are clustered at group (village) level.\end{tabular}\end{minipage}\\\vspace{2ex}\hfil\begin{minipage}[t]{14cm}\hfil\textsc{\normalsize Table \refstepcounter{table}\thetable: Permutation test results of group rejection\label{tab1 Permutation test results of group rejection}}\\\setlength{\tabcolsep}{.5pt}\setlength{\baselineskip}{8pt}\renewcommand{\arraystretch}{.50}\hfil\begin{tikzpicture}\node (tbl) {\input{c:/data/GUK/analysis/save/EstimationMemo/GRejectedPermutationTestResultso800.tex}};\end{tikzpicture}\\\begin{tabular}{>{\hfill\scriptsize}p{1cm}<{}>{\hfill\scriptsize}p{.25cm}<{}>{\scriptsize}p{12cm}<{\hfill}}Source:& \multicolumn{2}{l}{\scriptsize Estimated with GUK administrative and survey data.}\\Notes: & 1. & \textsf{R}'s package \textsf{coin} is used for baseline mean covariates to conduct approximate permutation tests. Number of repetition is set to 100000. Step-down method is used to adjust for multiple testing of a multi-factor grouping variable. \textsf{Attrited} and \textsf{Nonattrited} columns show means of each group. For \textsf{Arm}, proportions of non-traditional arm are given. \\& 2. & ${}^{***}$, ${}^{**}$, ${}^{*}$ indicate statistical significance at 1\%, 5\%, 10\%, respetively. Standard errors are clustered at group (village) level.\end{tabular}\end{minipage}\\\vspace{2ex}\hfil\begin{minipage}[t]{14cm}\hfil\textsc{\normalsize Table \refstepcounter{table}\thetable: Permutation test results of group rejection among traditional arm\label{tab1 Permutation test results of group rejection among traditional arm}}\\\setlength{\tabcolsep}{.5pt}\setlength{\baselineskip}{8pt}\renewcommand{\arraystretch}{.50}\hfil\begin{tikzpicture}\node (tbl) {\input{c:/data/GUK/analysis/save/EstimationMemo/GRejectedInTradPermutationTestResultso800.tex}};\end{tikzpicture}\\\begin{tabular}{>{\hfill\scriptsize}p{1cm}<{}>{\hfill\scriptsize}p{.25cm}<{}>{\scriptsize}p{12cm}<{\hfill}}Source:& \multicolumn{2}{l}{\scriptsize Estimated with GUK administrative and survey data.}\\Notes: & 1. & \textsf{R}'s package \textsf{coin} is used for baseline mean covariates to conduct approximate permutation tests. Number of repetition is set to 100000. Step-down method is used to adjust for multiple testing of a multi-factor grouping variable. \textsf{Attrited} and \textsf{Nonattrited} columns show means of each group. For \textsf{Arm}, proportions of non-traditional arm are given. \\& 2. & ${}^{***}$, ${}^{**}$, ${}^{*}$ indicate statistical significance at 1\%, 5\%, 10\%, respetively. Standard errors are clustered at group (village) level.\end{tabular}\end{minipage}\\\vspace{2ex}\hfil\begin{minipage}[t]{14cm}\hfil\textsc{\normalsize Table \refstepcounter{table}\thetable: Permutation test results of group rejection among non-traditional arm\label{tab1 Permutation test results of group rejection among non-traditional arm}}\\\setlength{\tabcolsep}{.5pt}\setlength{\baselineskip}{8pt}\renewcommand{\arraystretch}{.50}\hfil\begin{tikzpicture}\node (tbl) {\input{c:/data/GUK/analysis/save/EstimationMemo/GRejectedInNonTradPermutationTestResultso800.tex}};\end{tikzpicture}\\\begin{tabular}{>{\hfill\scriptsize}p{1cm}<{}>{\hfill\scriptsize}p{.25cm}<{}>{\scriptsize}p{12cm}<{\hfill}}Source:& \multicolumn{2}{l}{\scriptsize Estimated with GUK administrative and survey data.}\\Notes: & 1. & \textsf{R}'s package \textsf{coin} is used for baseline mean covariates to conduct approximate permutation tests. Number of repetition is set to 100000. Step-down method is used to adjust for multiple testing of a multi-factor grouping variable. \textsf{Attrited} and \textsf{Nonattrited} columns show means of each group. For \textsf{Arm}, proportions of non-traditional arm are given. \\& 2. & ${}^{***}$, ${}^{**}$, ${}^{*}$ indicate statistical significance at 1\%, 5\%, 10\%, respetively. Standard errors are clustered at group (village) level.\end{tabular}\end{minipage}\\\vspace{2ex}\hfil\begin{minipage}[t]{14cm}\hfil\textsc{\normalsize Table \refstepcounter{table}\thetable: Permutation test results of group rejecters, traditional vs. non-traditional arm\label{tab1 Permutation test results of group rejecters, traditional vs. non-traditional arm}}\\\setlength{\tabcolsep}{.5pt}\setlength{\baselineskip}{8pt}\renewcommand{\arraystretch}{.50}\hfil\begin{tikzpicture}\node (tbl) {\input{c:/data/GUK/analysis/save/EstimationMemo/TradNonTradGRejectedPermutationTestResultso800.tex}};\end{tikzpicture}\\\begin{tabular}{>{\hfill\scriptsize}p{1cm}<{}>{\hfill\scriptsize}p{.25cm}<{}>{\scriptsize}p{12cm}<{\hfill}}Source:& \multicolumn{2}{l}{\scriptsize Estimated with GUK administrative and survey data.}\\Notes: & 1. & \textsf{R}'s package \textsf{coin} is used for baseline mean covariates to conduct approximate permutation tests. Number of repetition is set to 100000. Step-down method is used to adjust for multiple testing of a multi-factor grouping variable. \textsf{Attrited} and \textsf{Nonattrited} columns show means of each group. For \textsf{Arm}, proportions of non-traditional arm are given. \\& 2. & ${}^{***}$, ${}^{**}$, ${}^{*}$ indicate statistical significance at 1\%, 5\%, 10\%, respetively. Standard errors are clustered at group (village) level.\end{tabular}\end{minipage}\\\vspace{2ex}\hfil\begin{minipage}[t]{14cm}\hfil\textsc{\normalsize Table \refstepcounter{table}\thetable: Permutation test results of individual rejection\label{tab1 Permutation test results of individual rejection}}\\\setlength{\tabcolsep}{.5pt}\setlength{\baselineskip}{8pt}\renewcommand{\arraystretch}{.50}\hfil\begin{tikzpicture}\node (tbl) {\input{c:/data/GUK/analysis/save/EstimationMemo/IRejectedPermutationTestResultso800.tex}};\end{tikzpicture}\\\begin{tabular}{>{\hfill\scriptsize}p{1cm}<{}>{\hfill\scriptsize}p{.25cm}<{}>{\scriptsize}p{12cm}<{\hfill}}Source:& \multicolumn{2}{l}{\scriptsize Estimated with GUK administrative and survey data.}\\Notes: & 1. & \textsf{R}'s package \textsf{coin} is used for baseline mean covariates to conduct approximate permutation tests. Number of repetition is set to 100000. Step-down method is used to adjust for multiple testing of a multi-factor grouping variable. \textsf{Attrited} and \textsf{Nonattrited} columns show means of each group. For \textsf{Arm}, proportions of non-traditional arm are given. \\& 2. & ${}^{***}$, ${}^{**}$, ${}^{*}$ indicate statistical significance at 1\%, 5\%, 10\%, respetively. Standard errors are clustered at group (village) level.\end{tabular}\end{minipage}\\\vspace{2ex}\hfil\begin{minipage}[t]{14cm}\hfil\textsc{\normalsize Table \refstepcounter{table}\thetable: Permutation test results of individual rejection among traditional arm\label{tab1 Permutation test results of individual rejection among traditional arm}}\\\setlength{\tabcolsep}{.5pt}\setlength{\baselineskip}{8pt}\renewcommand{\arraystretch}{.50}\hfil\begin{tikzpicture}\node (tbl) {\input{c:/data/GUK/analysis/save/EstimationMemo/IRejectedInTradPermutationTestResultso800.tex}};\end{tikzpicture}\\\begin{tabular}{>{\hfill\scriptsize}p{1cm}<{}>{\hfill\scriptsize}p{.25cm}<{}>{\scriptsize}p{12cm}<{\hfill}}Source:& \multicolumn{2}{l}{\scriptsize Estimated with GUK administrative and survey data.}\\Notes: & 1. & \textsf{R}'s package \textsf{coin} is used for baseline mean covariates to conduct approximate permutation tests. Number of repetition is set to 100000. Step-down method is used to adjust for multiple testing of a multi-factor grouping variable. \textsf{Attrited} and \textsf{Nonattrited} columns show means of each group. For \textsf{Arm}, proportions of non-traditional arm are given. \\& 2. & ${}^{***}$, ${}^{**}$, ${}^{*}$ indicate statistical significance at 1\%, 5\%, 10\%, respetively. Standard errors are clustered at group (village) level.\end{tabular}\end{minipage}\\\vspace{2ex}\hfil\begin{minipage}[t]{14cm}\hfil\textsc{\normalsize Table \refstepcounter{table}\thetable: Permutation test results of individual rejection among non-traditional arm\label{tab1 Permutation test results of individual rejection among non-traditional arm}}\\\setlength{\tabcolsep}{.5pt}\setlength{\baselineskip}{8pt}\renewcommand{\arraystretch}{.50}\hfil\begin{tikzpicture}\node (tbl) {\input{c:/data/GUK/analysis/save/EstimationMemo/IRejectedInNonTradPermutationTestResultso800.tex}};\end{tikzpicture}\\\begin{tabular}{>{\hfill\scriptsize}p{1cm}<{}>{\hfill\scriptsize}p{.25cm}<{}>{\scriptsize}p{12cm}<{\hfill}}Source:& \multicolumn{2}{l}{\scriptsize Estimated with GUK administrative and survey data.}\\Notes: & 1. & \textsf{R}'s package \textsf{coin} is used for baseline mean covariates to conduct approximate permutation tests. Number of repetition is set to 100000. Step-down method is used to adjust for multiple testing of a multi-factor grouping variable. \textsf{Attrited} and \textsf{Nonattrited} columns show means of each group. For \textsf{Arm}, proportions of non-traditional arm are given. \\& 2. & ${}^{***}$, ${}^{**}$, ${}^{*}$ indicate statistical significance at 1\%, 5\%, 10\%, respetively. Standard errors are clustered at group (village) level.\end{tabular}\end{minipage}\\\vspace{2ex}\hfil\begin{minipage}[t]{14cm}\hfil\textsc{\normalsize Table \refstepcounter{table}\thetable: Permutation test results of individual rejecters, traditional vs. non-traditional arm\label{tab1 Permutation test results of individual rejecters, traditional vs. non-traditional arm}}\\\setlength{\tabcolsep}{.5pt}\setlength{\baselineskip}{8pt}\renewcommand{\arraystretch}{.50}\hfil\begin{tikzpicture}\node (tbl) {\input{c:/data/GUK/analysis/save/EstimationMemo/TradNonTradIRejectedPermutationTestResultso800.tex}};\end{tikzpicture}\\\begin{tabular}{>{\hfill\scriptsize}p{1cm}<{}>{\hfill\scriptsize}p{.25cm}<{}>{\scriptsize}p{12cm}<{\hfill}}Source:& \multicolumn{2}{l}{\scriptsize Estimated with GUK administrative and survey data.}\\Notes: & 1. & \textsf{R}'s package \textsf{coin} is used for baseline mean covariates to conduct approximate permutation tests. Number of repetition is set to 100000. Step-down method is used to adjust for multiple testing of a multi-factor grouping variable. \textsf{Attrited} and \textsf{Nonattrited} columns show means of each group. For \textsf{Arm}, proportions of non-traditional arm are given. \\& 2. & ${}^{***}$, ${}^{**}$, ${}^{*}$ indicate statistical significance at 1\%, 5\%, 10\%, respetively. Standard errors are clustered at group (village) level.\end{tabular}\end{minipage}\\\vspace{2ex}\hfil\begin{minipage}[t]{14cm}\hfil\textsc{\normalsize Table \refstepcounter{table}\thetable: Permutation test results of borrowers, cattle vs. non-cattle arms\label{tab1 Permutation test results of borrowers, cattle vs. non-cattle arms}}\\\setlength{\tabcolsep}{.5pt}\setlength{\baselineskip}{8pt}\renewcommand{\arraystretch}{.50}\hfil\begin{tikzpicture}\node (tbl) {\input{c:/data/GUK/analysis/save/EstimationMemo/AcceptedCowNonCowPermutationTestResultso800.tex}};\end{tikzpicture}\\\begin{tabular}{>{\hfill\scriptsize}p{1cm}<{}>{\hfill\scriptsize}p{.25cm}<{}>{\scriptsize}p{12cm}<{\hfill}}Source:& \multicolumn{2}{l}{\scriptsize Estimated with GUK administrative and survey data.}\\Notes: & 1. & \textsf{R}'s package \textsf{coin} is used for baseline mean covariates to conduct approximate permutation tests. Number of repetition is set to 100000. Step-down method is used to adjust for multiple testing of a multi-factor grouping variable. \textsf{Attrited} and \textsf{Nonattrited} columns show means of each group. For \textsf{Arm}, proportions of non-traditional arm are given. \\& 2. & ${}^{***}$, ${}^{**}$, ${}^{*}$ indicate statistical significance at 1\%, 5\%, 10\%, respetively. Standard errors are clustered at group (village) level.\end{tabular}\end{minipage}\\\vspace{2ex}


%	\noindent\textsc{\normalsize Table \ref{grep(TabLabelStrings[1], TabLabel1)}} to \textsc{\normalsize Table \ref{grep(TabLabelStrings[13], TabLabel1)}}: Trimmed sample.\\

	\textsc{\normalsize Table \ref{tab1 Permutation test results of rejection}} shows test results of independence between loan receivers and nonreceivers (group, individual rejecters) on the analysis sample of 776 members. It shows that lower head literacy, smaller household size, being affected by flood at the baseline, smaller livestock holding, and smaller net assets are correlated with opting out the offered type of lending. \textsc{\normalsize Table \ref{tab1 Permutation test results of rejection among traditional arm}} indicates that lower asset and livestock holding is more pronounced among \textsf{traditional} rejecters relative to loan receivers. It also shows that flood exposure is less frequent, contrary to \textsc{\normalsize Table \ref{tab1 Permutation test results of rejection}}, among the rejecters. \textsc{\normalsize Table \ref{tab1 Permutation test results of rejection among non-traditional arm}} indicates that lower head literacy, smaller household size, higher flood exposure, are more pronounced among non-\textsf{traditional} rejecters relative to loan receivers. It also shows that asset and livestock holding is no different relative to the receivers. Comparing rejecters of \textsf{traditional} arm, lower flood exposure may be the only stark difference against non-\textsf{traditional} arm members, and smaller asset and livestock holding is merely suggestive (\textsc{\normalsize Table \ref{tab1 Permutation test results of rejecters, traditional vs. non-traditional arm}}). 
	
	Group rejecters and non-group rejecters are compared in \textsc{\normalsize Table \ref{tab1 Permutation test results of group rejection}}. Marked differences are found in arm (\textsf{traditional} vs. non-\textsf{traditional}) and net asset values and head literacy are noted. \textsc{\normalsize Table \ref{tab1 Permutation test results of group rejection among traditional arm}} compares group rejecters in \textsf{traditional} arm and finds smaller flood exposure and lower livestock and net asset holding are associated with group rejection. Group rejecters in non-\textsf{traditional} arm are examined in \textsc{\normalsize Table \ref{tab1 Permutation test results of group rejection among non-traditional arm}} and younger head age, flood at baseline, and smaller household asset holding are correlated with rejection. Comparing group rejecters between \textsf{traditional} and non-\textsf{traditional} arms in \textsc{\normalsize Table \ref{tab1 Permutation test results of group rejecters, traditional vs. non-traditional arm}}, younger head age, higher flood exposure, larger net asset values and livestock holding are noted among the non-\textsf{traditional} group rejecters. These hint that for non-\textsf{traditional} arm group rejecters, it is the smaller household size and the baseline flood that may have constrained them from participation, and for \textsf{traditional} group rejecters, it is the low asset levels.

	Acknowledging the reasons for rejection can be different, we tested the independence of each characteristics for individual rejecters (vs. non-individual rejeceters) in \textsc{\normalsize Table \ref{tab1 Permutation test results of individual rejection}}. Smaller \textsf{HHsize}, being affected with \textsf{FloodInRd1}, and smaller \textsf{LivestockValue}, \textsf{NumCows}, and \textsf{NetValue} are associated with individual rejecters. Individual decisions not to participate may be more straightforward: Smaller household size may indicate difficulty in securing the cattle production labour in a household, being hit with a flood may have resulted in lower livestock levels that would prompt them to reconsider partaking in another livestock project. 

	\textsc{\normalsize Table \ref{tab1 Permutation test results of individual rejection among traditional arm}} and \textsc{\normalsize Table \ref{tab1 Permutation test results of individual rejection among non-traditional arm}} compare individual rejecters and nonrejecters in \textsf{traditional} arm and non-\textsf{traditional} arms, respectively. For \textsf{traditional} rejecters, livestock and other asset values are not correlated with rejection, but the values are similar to non-\textsf{traditional} and higher $p$ values may be due to smaller sample size. For non-\textsf{traditional} arm rejecters, household size and flood exposure are correlated. Comparison of individual rejecters between \textsf{traditional} and non-\textsf{traditional} arms show no detectable difference (\textsc{\normalsize Table \ref{tab1 Permutation test results of individual rejecters, traditional vs. non-traditional arm}}). This suggests that indvidual rejecters in all arms were constrained with small household size and small asset holding.

	%	A closer look at the nonparticipation correlates among \textsf{traditional} arm mebers in \textsc{\normalsize Table \ref{tab reject trad perm}} and non-\textsf{traditional} arm members in \textsc{\normalsize Table \ref{tab reject nontrad perm}} reveal possible differences in the causes. Rejection among \textsf{traditional} members tend to be associated with lower livestock holding but not with higher flood exposure nor smaller household size, while rejecters among non-\textsf{traditional} members are more likely to have suffered from flood in baseline and have smaller household size. Since the offered arms were randomised, rejecters of \textsf{traditional} arm, who were not more exposed to flood and have similar household size at the mean, may have accepted the offer had they been offered non- \textsf{traditional} lending. Henceforth, we conjecture that flood exposure and household size are the potential impediments in larger size loans. This implies that there may not be minimum livestock and asset holding levels to partake the larger loans, and a poverty trap at this level may be overcome.



\section{Attrition}
\label{AppSecAttrition}
\setcounter{table}{0}


\hfil\begin{minipage}[t]{14cm}\hfil\textsc{\normalsize Table \refstepcounter{table}\thetable: Permutation test results of attrition\label{tab1 Permutation test results of attrition}}\\\setlength{\tabcolsep}{.5pt}\setlength{\baselineskip}{8pt}\renewcommand{\arraystretch}{.50}\hfil\begin{tikzpicture}\node (tbl) {\input{c:/data/GUK/analysis/save/EstimationMemo/AttritedPermutationTestResultso800.tex}};\end{tikzpicture}\\\begin{tabular}{>{\hfill\scriptsize}p{1cm}<{}>{\hfill\scriptsize}p{.25cm}<{}>{\scriptsize}p{12cm}<{\hfill}}Source:& \multicolumn{2}{l}{\scriptsize Estimated with GUK administrative and survey data.}\\Notes: & 1. & \textsf{R}'s package \textsf{coin} is used for baseline mean covariates to conduct approximate permutation tests. Number of repetition is set to 100000. Step-down method is used to adjust for multiple testing of a multi-factor grouping variable. \textsf{Attrited} and \textsf{Nonattrited} columns show means of each group. For \textsf{Arm}, proportions of non-traditional arm are given. \\& 2. & ${}^{***}$, ${}^{**}$, ${}^{*}$ indicate statistical significance at 1\%, 5\%, 10\%, respetively. Standard errors are clustered at group (village) level.\end{tabular}\end{minipage}\\\vspace{2ex}\hfil\begin{minipage}[t]{14cm}\hfil\textsc{\normalsize Table \refstepcounter{table}\thetable: Permutation test results of attrition among traditional arm\label{tab1 Permutation test results of attrition among traditional arm}}\\\setlength{\tabcolsep}{.5pt}\setlength{\baselineskip}{8pt}\renewcommand{\arraystretch}{.50}\hfil\begin{tikzpicture}\node (tbl) {\input{c:/data/GUK/analysis/save/EstimationMemo/AttritedInTradPermutationTestResultso800.tex}};\end{tikzpicture}\\\begin{tabular}{>{\hfill\scriptsize}p{1cm}<{}>{\hfill\scriptsize}p{.25cm}<{}>{\scriptsize}p{12cm}<{\hfill}}Source:& \multicolumn{2}{l}{\scriptsize Estimated with GUK administrative and survey data.}\\Notes: & 1. & \textsf{R}'s package \textsf{coin} is used for baseline mean covariates to conduct approximate permutation tests. Number of repetition is set to 100000. Step-down method is used to adjust for multiple testing of a multi-factor grouping variable. \textsf{Attrited} and \textsf{Nonattrited} columns show means of each group. For \textsf{Arm}, proportions of non-traditional arm are given. \\& 2. & ${}^{***}$, ${}^{**}$, ${}^{*}$ indicate statistical significance at 1\%, 5\%, 10\%, respetively. Standard errors are clustered at group (village) level.\end{tabular}\end{minipage}\\\vspace{2ex}\hfil\begin{minipage}[t]{14cm}\hfil\textsc{\normalsize Table \refstepcounter{table}\thetable: Permutation test results of attrition among non-traditional arm\label{tab1 Permutation test results of attrition among non-traditional arm}}\\\setlength{\tabcolsep}{.5pt}\setlength{\baselineskip}{8pt}\renewcommand{\arraystretch}{.50}\hfil\begin{tikzpicture}\node (tbl) {\input{c:/data/GUK/analysis/save/EstimationMemo/AttritedInNonTradPermutationTestResultso800.tex}};\end{tikzpicture}\\\begin{tabular}{>{\hfill\scriptsize}p{1cm}<{}>{\hfill\scriptsize}p{.25cm}<{}>{\scriptsize}p{12cm}<{\hfill}}Source:& \multicolumn{2}{l}{\scriptsize Estimated with GUK administrative and survey data.}\\Notes: & 1. & \textsf{R}'s package \textsf{coin} is used for baseline mean covariates to conduct approximate permutation tests. Number of repetition is set to 100000. Step-down method is used to adjust for multiple testing of a multi-factor grouping variable. \textsf{Attrited} and \textsf{Nonattrited} columns show means of each group. For \textsf{Arm}, proportions of non-traditional arm are given. \\& 2. & ${}^{***}$, ${}^{**}$, ${}^{*}$ indicate statistical significance at 1\%, 5\%, 10\%, respetively. Standard errors are clustered at group (village) level.\end{tabular}\end{minipage}\\\vspace{2ex}\hfil\begin{minipage}[t]{14cm}\hfil\textsc{\normalsize Table \refstepcounter{table}\thetable: Permutation test results of attriters of traditional and non-traditional arms\label{tab1 Permutation test results of attriters of traditional and non-traditional arms}}\\\setlength{\tabcolsep}{.5pt}\setlength{\baselineskip}{8pt}\renewcommand{\arraystretch}{.50}\hfil\begin{tikzpicture}\node (tbl) {\input{c:/data/GUK/analysis/save/EstimationMemo/TradNonTradAttritedPermutationTestResultso800.tex}};\end{tikzpicture}\\\begin{tabular}{>{\hfill\scriptsize}p{1cm}<{}>{\hfill\scriptsize}p{.25cm}<{}>{\scriptsize}p{12cm}<{\hfill}}Source:& \multicolumn{2}{l}{\scriptsize Estimated with GUK administrative and survey data.}\\Notes: & 1. & \textsf{R}'s package \textsf{coin} is used for baseline mean covariates to conduct approximate permutation tests. Number of repetition is set to 100000. Step-down method is used to adjust for multiple testing of a multi-factor grouping variable. \textsf{Attrited} and \textsf{Nonattrited} columns show means of each group. For \textsf{Arm}, proportions of non-traditional arm are given. \\& 2. & ${}^{***}$, ${}^{**}$, ${}^{*}$ indicate statistical significance at 1\%, 5\%, 10\%, respetively. Standard errors are clustered at group (village) level.\end{tabular}\end{minipage}\\\vspace{2ex}\hfil\begin{minipage}[t]{14cm}\hfil\textsc{\normalsize Table \refstepcounter{table}\thetable: Permutation test results of survival\label{tab1 Permutation test results of survival}}\\\setlength{\tabcolsep}{.5pt}\setlength{\baselineskip}{8pt}\renewcommand{\arraystretch}{.50}\hfil\begin{tikzpicture}\node (tbl) {\input{c:/data/GUK/analysis/save/EstimationMemo/SurvivedPermutationTestResultso800.tex}};\end{tikzpicture}\\\begin{tabular}{>{\hfill\scriptsize}p{1cm}<{}>{\hfill\scriptsize}p{.25cm}<{}>{\scriptsize}p{12cm}<{\hfill}}Source:& \multicolumn{2}{l}{\scriptsize Estimated with GUK administrative and survey data.}\\Notes: & 1. & \textsf{R}'s package \textsf{coin} is used for baseline mean covariates to conduct approximate permutation tests. Number of repetition is set to 100000. Step-down method is used to adjust for multiple testing of a multi-factor grouping variable. \textsf{Attrited} and \textsf{Nonattrited} columns show means of each group. For \textsf{Arm}, proportions of non-traditional arm are given. \\& 2. & ${}^{***}$, ${}^{**}$, ${}^{*}$ indicate statistical significance at 1\%, 5\%, 10\%, respetively. Standard errors are clustered at group (village) level.\end{tabular}\end{minipage}\\\vspace{2ex}\hfil\begin{minipage}[t]{14cm}\hfil\textsc{\normalsize Table \refstepcounter{table}\thetable: Permutation test results of surviving members of cattle and large grace\label{tab1 Permutation test results of surviving members of cattle and large grace}}\\\setlength{\tabcolsep}{.5pt}\setlength{\baselineskip}{8pt}\renewcommand{\arraystretch}{.50}\hfil\begin{tikzpicture}\node (tbl) {\input{c:/data/GUK/analysis/save/EstimationMemo/SurvivingCowLargeGracePermutationTestResultso800.tex}};\end{tikzpicture}\\\begin{tabular}{>{\hfill\scriptsize}p{1cm}<{}>{\hfill\scriptsize}p{.25cm}<{}>{\scriptsize}p{12cm}<{\hfill}}Source:& \multicolumn{2}{l}{\scriptsize Estimated with GUK administrative and survey data.}\\Notes: & 1. & \textsf{R}'s package \textsf{coin} is used for baseline mean covariates to conduct approximate permutation tests. Number of repetition is set to 100000. Step-down method is used to adjust for multiple testing of a multi-factor grouping variable. \textsf{Attrited} and \textsf{Nonattrited} columns show means of each group. For \textsf{Arm}, proportions of non-traditional arm are given. \\& 2. & ${}^{***}$, ${}^{**}$, ${}^{*}$ indicate statistical significance at 1\%, 5\%, 10\%, respetively. Standard errors are clustered at group (village) level.\end{tabular}\end{minipage}\\\vspace{2ex}\hfil\begin{minipage}[t]{14cm}\hfil\textsc{\normalsize Table \refstepcounter{table}\thetable: Permutation test results of surviving members of cattle and all other arms\label{tab1 Permutation test results of surviving members of cattle and all other arms}}\\\setlength{\tabcolsep}{.5pt}\setlength{\baselineskip}{8pt}\renewcommand{\arraystretch}{.50}\hfil\begin{tikzpicture}\node (tbl) {\input{c:/data/GUK/analysis/save/EstimationMemo/SurvivingCowNonCowPermutationTestResultso800.tex}};\end{tikzpicture}\\\begin{tabular}{>{\hfill\scriptsize}p{1cm}<{}>{\hfill\scriptsize}p{.25cm}<{}>{\scriptsize}p{12cm}<{\hfill}}Source:& \multicolumn{2}{l}{\scriptsize Estimated with GUK administrative and survey data.}\\Notes: & 1. & \textsf{R}'s package \textsf{coin} is used for baseline mean covariates to conduct approximate permutation tests. Number of repetition is set to 100000. Step-down method is used to adjust for multiple testing of a multi-factor grouping variable. \textsf{Attrited} and \textsf{Nonattrited} columns show means of each group. For \textsf{Arm}, proportions of non-traditional arm are given. \\& 2. & ${}^{***}$, ${}^{**}$, ${}^{*}$ indicate statistical significance at 1\%, 5\%, 10\%, respetively. Standard errors are clustered at group (village) level.\end{tabular}\end{minipage}\\\vspace{2ex}


	\textsc{\normalsize Table \ref{tab1 Permutation test results of attrition}} shows results from tests of independence between attriters and non-attriters. We see the moderate rate of attrition is not correlated with household level characteristics at the conventional $p$ value level. Productive asset amounts seem to differ between attriters and non-attriters, with the former being larger than the latter. This positive attrition selection can cause underestimation of impacts, if the asset values are positively correlated with entrepreneurial capacity. \textsc{\normalsize Table \ref{tab1 Permutation test results of attrition among traditional arm}} shows attrition in the \textsf{traditional} arm. Household heads of attriters are relatively less literate than non-attriters. \textsc{\normalsize Table \ref{tab1 Permutation test results of attrition among non-traditional arm}} compares attriters and non-attriters in the non-\textsf{traditional} arm. Unlike \textsf{traditional} arm attriters, non-\textsf{traditional} arm attriters have more literate household heads, have a larger household size, are more exposed to floods, and have larger productive assets. The \textsf{traditional} arm attriters may be less entrepreneurial, if anything, so their attrition may upwardly bias the positive gains of the arm, hence understate the impacts of non-\textsf{traditional} arm. These are explicitly shown in \textsc{\normalsize Table \ref{tab1 Permutation test results of attriters of traditional and non-traditional arms}} where we compare attriters of \textsf{traditional} and non-\textsf{traditional} arms. Overall, attrition may have attenuated the impacts but is not likely to have inflated them.\footnote{So one can employ the Lee bounds for stronger results, but doing so will give us less precision and require more assumptions. We will not use the Lee bounds \textcolor{red}{[we can show them if necessary]}. }


	\textsc{\normalsize Table \ref{tab1 Permutation test results of survival}} picks up only program surviving members (nonattrited and loan recepients) have greater asset values than non-survivors. Comparing the surviving members, characteristics are similar except that the \textsf{traditional} members are more exposed to the flood than the non-\textsf{traditional} members. Comparing against the large grace arm, survivors in the cattle arm are more exposed to the flood, have fewer productive assets, and have less livestock with $p$ value at .124 (\textsc{\normalsize Table \ref{tab1 Permutation test results of surviving members of cattle and large grace}}). This shows that the smaller livestock holders are encouraged to participate and continue to operate in the cattle arm that has a managerial support program with all other features being equal. This underscores our intrepretation that the current impact estimates may be downward biased, if any, as people who would otherwise attrit or reject in cattle arm stayed on. In \textsc{\normalsize Table \ref{tab1 Permutation test results of surviving members of cattle and all other arms}}, we compare the borrowers who did not attrit by the end of final survey round between \textsf{cattle} arm with other arms. At the baseline, \textsf{cattle} arm non-attriting borrowers have smaller baseline livestock holding ($p$ value = .016) and smaller baseline net asset holding ($p$ value = .007) than other arms' non-attriting borrowers.

%	Moved from main text.
	
%	Group rejecters of \textsf{traditional} and non-\textsf{traditional} arms differ in household characteristics. Lower livestock values, smaller cattle holding, and smaller net asset values are associated with group rejection for \textsf{traditional} arm (\textsc{\normalsize Table \ref{TabLabel1[grep("p rej.* traditional arm$", TabLabel1)]}}), while higher baseline flood exposure rates and younger household heads are associated with group rejection for non-\textsf{traditional} arms (\textsc{\normalsize Table \ref{TabLabel1[grep("p rej.*g non-traditional arm$", TabLabel1)]}}). Given randomisation, we conjecture that it is lack of \textsf{Upfront} liquidity that prevented smaller livestock holders of \textsf{traditional} arm from participating because they cannot purchase cattle due to insufficient net asset values or an insufficient resale value of owned livestock, when members of similar characteristics partcipated in non-\textsf{traditional} arms. %This is a real resource constraint that binds the households. This is different from a psychological constraint that, so long as there is a cost or a payment involved, albeit at a minimal level, there remains a group of households who would not take up the investment \citep{Ashraf2010, CohenDupas2010}. 
%	For non-\textsf{traditional} arm rejecters, it is the past flood that kept members from participating, even they are younger and have similar cattle holding as the non group-rejecters. 

%	Individual rejecters of \textsf{traditional} arm and non-\textsf{traditional} arms share similar characteristics (\textsc{\normalsize Table \ref{TabLabel1[grep("l rej.*vs", TabLabel1)]}}). In fact, they are not very different in all the variables considered. %This is consistent with the conjecture that, had the \textsf{traditional} arm group rejecters been offered any of the non-\textsf{traditional} arms, they, as a group, may have accepted it.  %It shows the latter is more exposed to flood in baseline and has larger livestock values. This implies that, once large enough sum of loan is disbursed, %there is no minimum livestock and asset holding level to partake in the larger loans, and 
	%despite a negative asset shock in flood and a poverty trap at this level may be overcome once household size and negative asset shocks are accounted for.
%	The common factors associated with nonparticipation are a smaller household size and smaller livestock holding (\textsc{\normalsize Table \ref{TabLabel1[grep("l rej.*g tra", TabLabel1)]}} and \textsc{\normalsize Table \ref{TabLabel1[grep("l rej.*g non-", TabLabel1)]}}), although the $p$ values for livestock holding difference between individual rejecters and non individual rejecters are around 7\% (\textsc{\normalsize Table \ref{TabLabel1[grep("l rejection$", TabLabel1)]}}). %In non-\textsf{traditional} arms, the individual rejecters have only marginally different mean values relative to individual nonrejecters (\textsc{\normalsize Table \ref{tab Ireject nontrad perm}}). 



\section{Estimated results}
\label{AppendixEstimationTables}
\setcounter{table}{0}




\hspace{-1cm}\begin{minipage}[t]{14cm} \hfil\textsc{\normalsize Table \refstepcounter{table}\thetable: ANCOVA estimation of net assets by period\label{tab ANCOVA net assets timevarying}}\\ \setlength{\tabcolsep}{1pt}
  \setlength{\baselineskip}{8pt}
  \renewcommand{\arraystretch}{.55}
  \hfil\begin{tikzpicture}
  \node (tbl) {\input{ c:/data/GUK/analysis/save/EstimationMemo/NetAssetsTimeVaryingANCOVAEstimationResults.tex }};
%\input{c:/dropbox/data/ramadan/save/tablecolortemplate.tex}
\end{tikzpicture}\\
\renewcommand{\arraystretch}{.8}
\setlength{\tabcolsep}{1pt} \begin{tabular}{>{\hfill\scriptsize}p{1cm}<{}>{\hfill\scriptsize}p{.25cm}<{}>{\scriptsize}p{12cm}<{\hfill}} 
Source:& \multicolumn{2}{l}{\scriptsize Estimated with GUK administrative and survey data.}\\
Notes: & 1. & ANCOVA estimates using administrative and survey data. Post treatment regressands are regressed on categorical variables, pre-treatment regressand and other covariates. Time invariant household characteristics (head age and head literacy) are taken from baseline housheold survey data. Sample comprises of (1) continuing members, and (2) replacing members of early rejecters who received a loan prior to Janunary, 2015.  \textsf{Large}, \textsf{LargeGrace}, \textsf{Cattle} are indicator variables of the \textsf{large}, \textsf{large grace}, and \textsf{cattle} arms, respectively. The default arm category is \textsf{traditional} arm. Household assets do not include livestock. Regressions (1)-(3), (5)-(6) use only arm and calendar information. (4) and (7) use previous six month repayment and saving information which is lacking in rd 1, hence starts from rd 2.\\
& 2. &  $P$ values in percentages in parenthesises. Standard errors are clustered at group (village) level. %${}^{***}$, ${}^{**}$, ${}^{*}$ indicate statistical significance at 1\%, 5\%, 10\%, respetively. Standard errors are clustered at group (village) level.
 \end{tabular}
\end{minipage} \\\\\hspace{-1cm}\begin{minipage}[t]{14cm} \hfil\textsc{\normalsize Table \refstepcounter{table}\thetable: ANCOVA estimation of net assets by attrbutes and period\label{tab ANCOVA net assets timevarying attributes}}\\ \setlength{\tabcolsep}{1pt}
  \setlength{\baselineskip}{8pt}
  \renewcommand{\arraystretch}{.55}
  \hfil\begin{tikzpicture}
  \node (tbl) {\input{ c:/data/GUK/analysis/save/EstimationMemo/NetAssetsTimeVaryingAttributesANCOVAEstimationResults.tex }};
%\input{c:/dropbox/data/ramadan/save/tablecolortemplate.tex}
\end{tikzpicture}\\
\renewcommand{\arraystretch}{.8}
\setlength{\tabcolsep}{1pt} \begin{tabular}{>{\hfill\scriptsize}p{1cm}<{}>{\hfill\scriptsize}p{.25cm}<{}>{\scriptsize}p{12cm}<{\hfill}} 
Source:& \multicolumn{2}{l}{\scriptsize Estimated with GUK administrative and survey data.}\\
Notes: & 1. & ANCOVA estimates using administrative and survey data. Post treatment regressands are regressed on categorical variables, pre-treatment regressand and other covariates. Time invariant household characteristics (head age and head literacy) are taken from baseline housheold survey data. Sample comprises of (1) continuing members, and (2) replacing members of early rejecters who received a loan prior to Janunary, 2015.  \textsf{Upfront} is an indicator variable of the arm with an upfront large disbursement, \textsf{WithGrace} is an indicator variable of the arm with a grace period, \textsf{InKind} is an indicator variable of the arm which lends a heifer. Household assets do not include livestock. Regressions (1)-(3), (5)-(6) use only arm and calendar information. (4) and (7) use previous six month repayment and saving information which is lacking in rd 1, hence starts from rd 2.\\
& 2. &  $P$ values in percentages in parenthesises. Standard errors are clustered at group (village) level. %${}^{***}$, ${}^{**}$, ${}^{*}$ indicate statistical significance at 1\%, 5\%, 10\%, respetively. Standard errors are clustered at group (village) level.
 \end{tabular}
\end{minipage} \\\\\hspace{-1cm}\begin{minipage}[t]{14cm} \hfil\textsc{\normalsize Table \refstepcounter{table}\thetable: ANCOVA estimation of cattle holding by arm and period\label{tab ANCOVA cow time varying}}\\ \setlength{\tabcolsep}{1pt}
  \setlength{\baselineskip}{8pt}
  \renewcommand{\arraystretch}{.55}
  \hfil\begin{tikzpicture}
  \node (tbl) {\input{ c:/data/GUK/analysis/save/EstimationMemo/NumCowsTimeVaryingANCOVAEstimationResults.tex }};
%\input{c:/dropbox/data/ramadan/save/tablecolortemplate.tex}
\end{tikzpicture}\\
\renewcommand{\arraystretch}{.8}
\setlength{\tabcolsep}{1pt} \begin{tabular}{>{\hfill\scriptsize}p{1cm}<{}>{\hfill\scriptsize}p{.25cm}<{}>{\scriptsize}p{12cm}<{\hfill}} 
Source:& \multicolumn{2}{l}{\scriptsize Estimated with GUK administrative and survey data.}\\
Notes: & 1. & ANCOVA estimates using administrative and survey data. Post treatment regressands are regressed on categorical variables, pre-treatment regressand and other covariates. Time invariant household characteristics (head age and head literacy) are taken from baseline housheold survey data. Sample comprises of (1) continuing members, and (2) replacing members of early rejecters who received a loan prior to Janunary, 2015.  \textsf{Large}, \textsf{LargeGrace}, \textsf{Cattle} are indicator variables of the \textsf{large}, \textsf{large grace}, and \textsf{cattle} arms, respectively. The default arm category is \textsf{traditional} arm. \textsf{rd2, rd3, rd4} are dummy variables for second, third, and fourth round of survey. Sample is continuing members and replacing members of early rejecters and received loans prior to 2015 Janunary. Regressand is \textsf{NumCows}, number of cattle holding. \\
& 2. & $P$ values in percentages in parenthesises. Standard errors are clustered at group (village) level. %${}^{***}$, ${}^{**}$, ${}^{*}$ indicate statistical significance at 1\%, 5\%, 10\%, respetively. Standard errors are clustered at group (village) level.
 \end{tabular}
\end{minipage} \\\\\hspace{-1cm}\begin{minipage}[t]{14cm} \hfil\textsc{\normalsize Table \refstepcounter{table}\thetable: ANCOVA estimation of cattle holding by attributes and period\label{tab ANCOVA cow time varying attributes}}\\ \setlength{\tabcolsep}{1pt}
  \setlength{\baselineskip}{8pt}
  \renewcommand{\arraystretch}{.55}
  \hfil\begin{tikzpicture}
  \node (tbl) {\input{ c:/data/GUK/analysis/save/EstimationMemo/NumCowsTimeVaryingAttributesANCOVAEstimationResults.tex }};
%\input{c:/dropbox/data/ramadan/save/tablecolortemplate.tex}
\end{tikzpicture}\\
\renewcommand{\arraystretch}{.8}
\setlength{\tabcolsep}{1pt} \begin{tabular}{>{\hfill\scriptsize}p{1cm}<{}>{\hfill\scriptsize}p{.25cm}<{}>{\scriptsize}p{12cm}<{\hfill}} 
Source:& \multicolumn{2}{l}{\scriptsize Estimated with GUK administrative and survey data.}\\
Notes: & 1. & ANCOVA estimates using administrative and survey data. Post treatment regressands are regressed on categorical variables, pre-treatment regressand and other covariates. Time invariant household characteristics (head age and head literacy) are taken from baseline housheold survey data. Sample comprises of (1) continuing members, and (2) replacing members of early rejecters who received a loan prior to Janunary, 2015.  \textsf{Upfront} is an indicator variable of the arm with an upfront large disbursement, \textsf{WithGrace} is an indicator variable of the arm with a grace period, \textsf{InKind} is an indicator variable of the arm which lends a heifer. \textsf{rd2, rd3, rd4} are dummy variables for second, third, and fourth round of survey. Sample is continuing members and replacing members of early rejecters and received loans prior to 2015 Janunary. Regressand is \textsf{NumCows}, number of cattle holding. \\
& 2. & $P$ values in percentages in parenthesises. Standard errors are clustered at group (village) level. %${}^{***}$, ${}^{**}$, ${}^{*}$ indicate statistical significance at 1\%, 5\%, 10\%, respetively. Standard errors are clustered at group (village) level.
 \end{tabular}
\end{minipage} \\\\\hspace{-1cm}\begin{minipage}[t]{14cm} \hfil\textsc{\normalsize Table \refstepcounter{table}\thetable: ANCOVA estimation of consumption by period\label{tab ANCOVA consumption timevarying}}\\ \setlength{\tabcolsep}{1pt}
  \setlength{\baselineskip}{8pt}
  \renewcommand{\arraystretch}{.55}
  \hfil\begin{tikzpicture}
  \node (tbl) {\input{ c:/data/GUK/analysis/save/EstimationMemo/ConsumptionTimeVaryingANCOVAEstimationResults.tex }};
%\input{c:/dropbox/data/ramadan/save/tablecolortemplate.tex}
\end{tikzpicture}\\
\renewcommand{\arraystretch}{.8}
\setlength{\tabcolsep}{1pt} \begin{tabular}{>{\hfill\scriptsize}p{1cm}<{}>{\hfill\scriptsize}p{.25cm}<{}>{\scriptsize}p{12cm}<{\hfill}} 
Source:& \multicolumn{2}{l}{\scriptsize Estimated with GUK administrative and survey data.}\\
Notes: & 1. & ANCOVA estimates using administrative and survey data. Post treatment regressands are regressed on categorical variables, pre-treatment regressand and other covariates. Time invariant household characteristics (head age and head literacy) are taken from baseline housheold survey data. Sample comprises of (1) continuing members, and (2) replacing members of early rejecters who received a loan prior to Janunary, 2015.  \textsf{UltraPoor} is an indicator variable if the household is classified as the ultra poor. \textsf{Large}, \textsf{LargeGrace}, \textsf{Cattle} are indicator variables of the \textsf{large}, \textsf{large grace}, and \textsf{cattle} arms, respectively. The default arm category is \textsf{traditional} arm. Consumption is annualised values. \\
& 2. & $P$ values in percentages in parenthesises. Standard errors are clustered at group (village) level. %${}^{***}$, ${}^{**}$, ${}^{*}$ indicate statistical significance at 1\%, 5\%, 10\%, respetively. Standard errors are clustered at group (village) level.
 \end{tabular}
\end{minipage} \\\\\hspace{-1cm}\begin{minipage}[t]{14cm} \hfil\textsc{\normalsize Table \refstepcounter{table}\thetable: ANCOVA estimation of consumption by attributes and period\label{tab ANCOVA consumption timevarying attributes original HH}}\\ \setlength{\tabcolsep}{1pt}
  \setlength{\baselineskip}{8pt}
  \renewcommand{\arraystretch}{.55}
  \hfil\begin{tikzpicture}
  \node (tbl) {\input{ c:/data/GUK/analysis/save/EstimationMemo/ConsumptionTimeVaryingAttributesANCOVAEstimationResults.tex }};
%\input{c:/dropbox/data/ramadan/save/tablecolortemplate.tex}
\end{tikzpicture}\\
\renewcommand{\arraystretch}{.8}
\setlength{\tabcolsep}{1pt} \begin{tabular}{>{\hfill\scriptsize}p{1cm}<{}>{\hfill\scriptsize}p{.25cm}<{}>{\scriptsize}p{12cm}<{\hfill}} 
Source:& \multicolumn{2}{l}{\scriptsize Estimated with GUK administrative and survey data.}\\
Notes: & 1. & ANCOVA estimates using administrative and survey data. Post treatment regressands are regressed on categorical variables, pre-treatment regressand and other covariates. Time invariant household characteristics (head age and head literacy) are taken from baseline housheold survey data. Sample comprises of (1) continuing members, and (2) replacing members of early rejecters who received a loan prior to Janunary, 2015.  \textsf{UltraPoor} is an indicator variable if the household is classified as the ultra poor. \textsf{Upfront} is an indicator variable of the arm with an upfront large disbursement, \textsf{WithGrace} is an indicator variable of the arm with a grace period, \textsf{InKind} is an indicator variable of the arm which lends a heifer. Consumption is annualised values. \\
& 2. & $P$ values in percentages in parenthesises. Standard errors are clustered at group (village) level. %${}^{***}$, ${}^{**}$, ${}^{*}$ indicate statistical significance at 1\%, 5\%, 10\%, respetively. Standard errors are clustered at group (village) level.
 \end{tabular}
\end{minipage} \\\\\hspace{-1cm}\begin{minipage}[t]{14cm} \hfil\textsc{\normalsize Table \refstepcounter{table}\thetable: ANCOVA estimation of household labour incomes and farm incomes by period\label{tab ANCOVA labour incomes timevarying}}\\ \setlength{\tabcolsep}{1pt}
  \setlength{\baselineskip}{8pt}
  \renewcommand{\arraystretch}{.55}
  \hfil\begin{tikzpicture}
  \node (tbl) {\input{ c:/data/GUK/analysis/save/EstimationMemo/LabourIncomeTimeVaryingANCOVAEstimationResults.tex }};
%\input{c:/dropbox/data/ramadan/save/tablecolortemplate.tex}
\end{tikzpicture}\\
\renewcommand{\arraystretch}{.8}
\setlength{\tabcolsep}{1pt} \begin{tabular}{>{\hfill\scriptsize}p{1cm}<{}>{\hfill\scriptsize}p{.25cm}<{}>{\scriptsize}p{12cm}<{\hfill}} 
Source:& \multicolumn{2}{l}{\scriptsize Estimated with GUK administrative and survey data.}\\
Notes: & 1. & ANCOVA estimates using administrative and survey data. Post treatment regressands are regressed on categorical variables, pre-treatment regressand and other covariates. Time invariant household characteristics (head age and head literacy) are taken from baseline housheold survey data. Sample comprises of (1) continuing members, and (2) replacing members of early rejecters who received a loan prior to Janunary, 2015.  \textsf{Large}, \textsf{LargeGrace}, \textsf{Cattle} are indicator variables of the \textsf{large}, \textsf{large grace}, and \textsf{cattle} arms, respectively. The default arm category is \textsf{traditional} arm. \textsf{rd2, rd3, rd4} are dummy variables for second, third, and fourth round of survey. Labour incomes are in 1000 Tk units and are a sum of all earned labour incomes of household members. Farm revenues are in 1000 Tk units and are a total of agricultural produce sales. \\
& 2. & $P$ values in percentages in parenthesises. Standard errors are clustered at group (village) level. %${}^{***}$, ${}^{**}$, ${}^{*}$ indicate statistical significance at 1\%, 5\%, 10\%, respetively. Standard errors are clustered at group (village) level.
 \end{tabular}
\end{minipage} \\\\\hspace{-1cm}\begin{minipage}[t]{14cm} \hfil\textsc{\normalsize Table \refstepcounter{table}\thetable: ANCOVA estimation of household labour incomes and farm incomes by attributes and period\label{tab ANCOVA labour incomes timevarying attributes}}\\ \setlength{\tabcolsep}{1pt}
  \setlength{\baselineskip}{8pt}
  \renewcommand{\arraystretch}{.55}
  \hfil\begin{tikzpicture}
  \node (tbl) {\input{ c:/data/GUK/analysis/save/EstimationMemo/LabourIncomeTimeVaryingAttributesANCOVAEstimationResults.tex }};
%\input{c:/dropbox/data/ramadan/save/tablecolortemplate.tex}
\end{tikzpicture}\\
\renewcommand{\arraystretch}{.8}
\setlength{\tabcolsep}{1pt} \begin{tabular}{>{\hfill\scriptsize}p{1cm}<{}>{\hfill\scriptsize}p{.25cm}<{}>{\scriptsize}p{12cm}<{\hfill}} 
Source:& \multicolumn{2}{l}{\scriptsize Estimated with GUK administrative and survey data.}\\
Notes: & 1. & ANCOVA estimates using administrative and survey data. Post treatment regressands are regressed on categorical variables, pre-treatment regressand and other covariates. Time invariant household characteristics (head age and head literacy) are taken from baseline housheold survey data. Sample comprises of (1) continuing members, and (2) replacing members of early rejecters who received a loan prior to Janunary, 2015.  \textsf{Upfront} is an indicator variable of the arm with an upfront large disbursement, \textsf{WithGrace} is an indicator variable of the arm with a grace period, \textsf{InKind} is an indicator variable of the arm which lends a heifer. \textsf{rd2, rd3, rd4} are dummy variables for second, third, and fourth round of survey. Labour incomes are in 1000 Tk units and are a sum of all earned labour incomes of household members. Farm revenues are in 1000 Tk units and are a total of agricultural produce sales. \\
& 2. & $P$ values in percentages in parenthesises. Standard errors are clustered at group (village) level. %${}^{***}$, ${}^{**}$, ${}^{*}$ indicate statistical significance at 1\%, 5\%, 10\%, respetively. Standard errors are clustered at group (village) level.
 \end{tabular}
\end{minipage} \\\\\hspace{-1cm}\begin{minipage}[t]{14cm} \hfil\textsc{\normalsize Table \refstepcounter{table}\thetable: ANCOVA estimation of school enrollment by time\label{tab ANCOVA enroll time varying1}}\\ \setlength{\tabcolsep}{1pt}
  \setlength{\baselineskip}{8pt}
  \renewcommand{\arraystretch}{.55}
  \hfil\begin{tikzpicture}
  \node (tbl) {\input{ c:/data/GUK/analysis/save/EstimationMemo/SchoolingTimeVaryingANCOVAEstimationResults_1.tex }};
%\input{c:/dropbox/data/ramadan/save/tablecolortemplate.tex}
\end{tikzpicture}\\
\renewcommand{\arraystretch}{.8}
\setlength{\tabcolsep}{1pt} \begin{tabular}{>{\hfill\scriptsize}p{1cm}<{}>{\hfill\scriptsize}p{.25cm}<{}>{\scriptsize}p{12cm}<{\hfill}} 
Source:& \multicolumn{2}{l}{\scriptsize Estimated with GUK administrative and survey data.}\\
Notes: & 1. & ANCOVA estimates using administrative and survey data. Post treatment regressands are regressed on categorical variables, pre-treatment regressand and other covariates. Time invariant household characteristics (head age and head literacy) are taken from baseline housheold survey data. Sample comprises of (1) continuing members, and (2) replacing members of early rejecters who received a loan prior to Janunary, 2015.  \textsf{Large}, \textsf{LargeGrace}, \textsf{Cattle} are indicator variables of the \textsf{large}, \textsf{large grace}, and \textsf{cattle} arms, respectively. The default arm category is \textsf{traditional} arm. \textsf{Secondary} and \textsf{College} are indicator variables of secondary schooling (ages 13-15) and tertiary schooling (ages 16-18), both at the time of baseline. Default category is primary (ages 05-12). \textsf{rd2, rd3, rd4} are dummy variables for second, third, and fourth round of survey. Interaction terms of dummy variables are demeaned before interacting. The first column gives mean and standard deviation (in parenthesises) of each covariates before demeaning. \\
& 2. & $P$ values in percentages in parenthesises. Standard errors are clustered at group (village) level. %${}^{***}$, ${}^{**}$, ${}^{*}$ indicate statistical significance at 1\%, 5\%, 10\%, respetively. Standard errors are clustered at group (village) level.
 \end{tabular}
\end{minipage} \\\\\hspace{-1cm}\begin{minipage}[t]{14cm} \hfil\textsc{\normalsize Table \refstepcounter{table}\thetable: ANCOVA estimation of school enrollment by time (continued)\label{tab ANCOVA enroll time varying2}}\\ \setlength{\tabcolsep}{1pt}
  \setlength{\baselineskip}{8pt}
  \renewcommand{\arraystretch}{.55}
  \hfil\begin{tikzpicture}
  \node (tbl) {\input{ c:/data/GUK/analysis/save/EstimationMemo/SchoolingTimeVaryingANCOVAEstimationResults_2.tex }};
%\input{c:/dropbox/data/ramadan/save/tablecolortemplate.tex}
\end{tikzpicture}\\
\renewcommand{\arraystretch}{.8}
\setlength{\tabcolsep}{1pt} \begin{tabular}{>{\hfill\scriptsize}p{1cm}<{}>{\hfill\scriptsize}p{.25cm}<{}>{\scriptsize}p{12cm}<{\hfill}} 
Source:& \multicolumn{2}{l}{\scriptsize Estimated with GUK administrative and survey data.}\\
Notes: & 1. & ANCOVA estimates using administrative and survey data. Post treatment regressands are regressed on categorical variables, pre-treatment regressand and other covariates. Time invariant household characteristics (head age and head literacy) are taken from baseline housheold survey data. Sample comprises of (1) continuing members, and (2) replacing members of early rejecters who received a loan prior to Janunary, 2015.  \textsf{Large}, \textsf{LargeGrace}, \textsf{Cattle} are indicator variables of the \textsf{large}, \textsf{large grace}, and \textsf{cattle} arms, respectively. The default arm category is \textsf{traditional} arm. \textsf{Secondary} and \textsf{College} are indicator variables of secondary schooling (ages 13-15) and tertiary schooling (ages 16-18), both at the time of baseline. Default category is primary (ages 05-12). \textsf{rd2, rd3, rd4} are dummy variables for second, third, and fourth round of survey. Interaction terms of dummy variables are demeaned before interacting. The first column gives mean and standard deviation (in parenthesises) of each covariates before demeaning. \\
& 2. & $P$ values in percentages in parenthesises. Standard errors are clustered at group (village) level. %${}^{***}$, ${}^{**}$, ${}^{*}$ indicate statistical significance at 1\%, 5\%, 10\%, respetively. Standard errors are clustered at group (village) level.
 \end{tabular}
\end{minipage} \\\\\hspace{-1cm}\begin{minipage}[t]{14cm} \hfil\textsc{\normalsize Table \refstepcounter{table}\thetable: ANCOVA estimation of school enrollment by time (continued 2)\label{tab ANCOVA enroll time varying3}}\\ \setlength{\tabcolsep}{1pt}
  \setlength{\baselineskip}{8pt}
  \renewcommand{\arraystretch}{.55}
  \hfil\begin{tikzpicture}
  \node (tbl) {\input{ c:/data/GUK/analysis/save/EstimationMemo/SchoolingTimeVaryingAttributesANCOVAEstimationResults_3.tex }};
%\input{c:/dropbox/data/ramadan/save/tablecolortemplate.tex}
\end{tikzpicture}\\
\renewcommand{\arraystretch}{.8}
\setlength{\tabcolsep}{1pt} \begin{tabular}{>{\hfill\scriptsize}p{1cm}<{}>{\hfill\scriptsize}p{.25cm}<{}>{\scriptsize}p{12cm}<{\hfill}} 
Source:& \multicolumn{2}{l}{\scriptsize Estimated with GUK administrative and survey data.}\\
Notes: & 1. & ANCOVA estimates using administrative and survey data. Post treatment regressands are regressed on categorical variables, pre-treatment regressand and other covariates. Time invariant household characteristics (head age and head literacy) are taken from baseline housheold survey data. Sample comprises of (1) continuing members, and (2) replacing members of early rejecters who received a loan prior to Janunary, 2015.  \textsf{Large}, \textsf{LargeGrace}, \textsf{Cattle} are indicator variables of the \textsf{large}, \textsf{large grace}, and \textsf{cattle} arms, respectively. The default arm category is \textsf{traditional} arm. \textsf{Secondary} and \textsf{College} are indicator variables of secondary schooling (ages 13-15) and tertiary schooling (ages 16-18), both at the time of baseline. Default category is primary (ages 05-12). \textsf{rd2, rd3, rd4} are dummy variables for second, third, and fourth round of survey. Interaction terms of dummy variables are demeaned before interacting. The first column gives mean and standard deviation (in parenthesises) of each covariates before demeaning. \\
& 2. & $P$ values in percentages in parenthesises. Standard errors are clustered at group (village) level. %${}^{***}$, ${}^{**}$, ${}^{*}$ indicate statistical significance at 1\%, 5\%, 10\%, respetively. Standard errors are clustered at group (village) level.
 \end{tabular}
\end{minipage} \\\\\hspace{-1cm}\begin{minipage}[t]{14cm} \hfil\textsc{\normalsize Table \refstepcounter{table}\thetable: ANCOVA estimation of school enrollment by attributes and time\label{tab ANCOVA enroll time varying attributes}}\\ \setlength{\tabcolsep}{1pt}
  \setlength{\baselineskip}{8pt}
  \renewcommand{\arraystretch}{.55}
  \hfil\begin{tikzpicture}
  \node (tbl) {\input{ c:/data/GUK/analysis/save/EstimationMemo/SchoolingTimeVaryingAttributesANCOVAEstimationResults_1.tex }};
%\input{c:/dropbox/data/ramadan/save/tablecolortemplate.tex}
\end{tikzpicture}\\
\renewcommand{\arraystretch}{.8}
\setlength{\tabcolsep}{1pt} \begin{tabular}{>{\hfill\scriptsize}p{1cm}<{}>{\hfill\scriptsize}p{.25cm}<{}>{\scriptsize}p{12cm}<{\hfill}} 
Source:& \multicolumn{2}{l}{\scriptsize Estimated with GUK administrative and survey data.}\\
Notes: & 1. & ANCOVA estimates using administrative and survey data. Post treatment regressands are regressed on categorical variables, pre-treatment regressand and other covariates. Time invariant household characteristics (head age and head literacy) are taken from baseline housheold survey data. Sample comprises of (1) continuing members, and (2) replacing members of early rejecters who received a loan prior to Janunary, 2015.  \textsf{Upfront} is an indicator variable of the arm with an upfront large disbursement, \textsf{WithGrace} is an indicator variable of the arm with a grace period, \textsf{InKind} is an indicator variable of the arm which lends a heifer. \textsf{Secondary} and \textsf{College} are indicator variables of secondary schooling (ages 13-15) and tertiary schooling (ages 16-18), both at the time of baseline. Default category is primary (ages 05-12). \textsf{rd2, rd3, rd4} are dummy variables for second, third, and fourth round of survey. Interaction terms of dummy variables are demeaned before interacting. The first column gives mean and standard deviation (in parenthesises) of each covariates before demeaning. \\
& 2. & $P$ values in percentages in parenthesises. Standard errors are clustered at group (village) level. %${}^{***}$, ${}^{**}$, ${}^{*}$ indicate statistical significance at 1\%, 5\%, 10\%, respetively. Standard errors are clustered at group (village) level.
 \end{tabular}
\end{minipage} \\\\\hspace{-1cm}\begin{minipage}[t]{14cm} \hfil\textsc{\normalsize Table \refstepcounter{table}\thetable: ANCOVA estimation of school enrollment by attributes and time (continued)\label{tab ANCOVA enroll time varying attributes2}}\\ \setlength{\tabcolsep}{1pt}
  \setlength{\baselineskip}{8pt}
  \renewcommand{\arraystretch}{.55}
  \hfil\begin{tikzpicture}
  \node (tbl) {\input{ c:/data/GUK/analysis/save/EstimationMemo/SchoolingTimeVaryingAttributesANCOVAEstimationResults_2.tex }};
%\input{c:/dropbox/data/ramadan/save/tablecolortemplate.tex}
\end{tikzpicture}\\
\renewcommand{\arraystretch}{.8}
\setlength{\tabcolsep}{1pt} \begin{tabular}{>{\hfill\scriptsize}p{1cm}<{}>{\hfill\scriptsize}p{.25cm}<{}>{\scriptsize}p{12cm}<{\hfill}} 
Source:& \multicolumn{2}{l}{\scriptsize Estimated with GUK administrative and survey data.}\\
Notes: & 1. & ANCOVA estimates using administrative and survey data. Post treatment regressands are regressed on categorical variables, pre-treatment regressand and other covariates. Time invariant household characteristics (head age and head literacy) are taken from baseline housheold survey data. Sample comprises of (1) continuing members, and (2) replacing members of early rejecters who received a loan prior to Janunary, 2015.  \textsf{Upfront} is an indicator variable of the arm with an upfront large disbursement, \textsf{WithGrace} is an indicator variable of the arm with a grace period, \textsf{InKind} is an indicator variable of the arm which lends a heifer. \textsf{Secondary} and \textsf{College} are indicator variables of secondary schooling (ages 13-15) and tertiary schooling (ages 16-18), both at the time of baseline. Default category is primary (ages 05-12). \textsf{rd2, rd3, rd4} are dummy variables for second, third, and fourth round of survey. Interaction terms of dummy variables are demeaned before interacting. The first column gives mean and standard deviation (in parenthesises) of each covariates before demeaning. \\
& 2. & $P$ values in percentages in parenthesises. Standard errors are clustered at group (village) level. %${}^{***}$, ${}^{**}$, ${}^{*}$ indicate statistical significance at 1\%, 5\%, 10\%, respetively. Standard errors are clustered at group (village) level.
 \end{tabular}
\end{minipage} \\\\\hspace{-1cm}\begin{minipage}[t]{14cm} \hfil\textsc{\normalsize Table \refstepcounter{table}\thetable: ANCOVA estimation of school enrollment by attributes and time (continued 2)\label{tab ANCOVA enroll time varying attributes3}}\\ \setlength{\tabcolsep}{1pt}
  \setlength{\baselineskip}{8pt}
  \renewcommand{\arraystretch}{.55}
  \hfil\begin{tikzpicture}
  \node (tbl) {\input{ c:/data/GUK/analysis/save/EstimationMemo/SchoolingTimeVaryingAttributesANCOVAEstimationResults_3.tex }};
%\input{c:/dropbox/data/ramadan/save/tablecolortemplate.tex}
\end{tikzpicture}\\
\renewcommand{\arraystretch}{.8}
\setlength{\tabcolsep}{1pt} \begin{tabular}{>{\hfill\scriptsize}p{1cm}<{}>{\hfill\scriptsize}p{.25cm}<{}>{\scriptsize}p{12cm}<{\hfill}} 
Source:& \multicolumn{2}{l}{\scriptsize Estimated with GUK administrative and survey data.}\\
Notes: & 1. & ANCOVA estimates using administrative and survey data. Post treatment regressands are regressed on categorical variables, pre-treatment regressand and other covariates. Time invariant household characteristics (head age and head literacy) are taken from baseline housheold survey data. Sample comprises of (1) continuing members, and (2) replacing members of early rejecters who received a loan prior to Janunary, 2015.  \textsf{Upfront} is an indicator variable of the arm with an upfront large disbursement, \textsf{WithGrace} is an indicator variable of the arm with a grace period, \textsf{InKind} is an indicator variable of the arm which lends a heifer. \textsf{Secondary} and \textsf{College} are indicator variables of secondary schooling (ages 13-15) and tertiary schooling (ages 16-18), both at the time of baseline. Default category is primary (ages 05-12). \textsf{rd2, rd3, rd4} are dummy variables for second, third, and fourth round of survey. Interaction terms of dummy variables are demeaned before interacting. The first column gives mean and standard deviation (in parenthesises) of each covariates before demeaning. \\
& 2. & $P$ values in percentages in parenthesises. Standard errors are clustered at group (village) level. %${}^{***}$, ${}^{**}$, ${}^{*}$ indicate statistical significance at 1\%, 5\%, 10\%, respetively. Standard errors are clustered at group (village) level.
 \end{tabular}
\end{minipage} \\\\\hspace{-1cm}\begin{minipage}[t]{14cm} \hfil\textsc{\normalsize Table \refstepcounter{table}\thetable: Individual level effects of repayment shortfall\label{tab shortfall indiv}}\\ \setlength{\tabcolsep}{1pt}
  \setlength{\baselineskip}{8pt}
  \renewcommand{\arraystretch}{.55}
  \hfil\begin{tikzpicture}
  \node (tbl) {\input{ c:/data/GUK/analysis/save/EstimationMemo/ShortfallIndividualEstimationResults1.tex }};
%\input{c:/dropbox/data/ramadan/save/tablecolortemplate.tex}
\end{tikzpicture}\\
\renewcommand{\arraystretch}{.8}
\setlength{\tabcolsep}{1pt} \begin{tabular}{>{\hfill\scriptsize}p{1cm}<{}>{\hfill\scriptsize}p{.25cm}<{}>{\scriptsize}p{12cm}<{\hfill}} 
Source:& \multicolumn{2}{l}{\scriptsize Estimated with GUK administrative data.}\\
Notes: & 1. & Group fixed effects estimates of repayment shortfall. Group fixed effects are controlled by differncing out respecive means from the data matrix. Shortfall is (planned installment) - (actual repayment). OtherShortfall indicates mean shortfall of other members in a group. Group repayment shortfall rates (GRSR) is (shortfall)/(planned installment). GRSR is defined as high if the first six months' repayment shortfall rate is above median, low if otherwise. Median GRSR is -1.42. \\
& 2. &  $P$ values in percentages in parenthesises. Standard errors are clustered at group (village) level. %${}^{***}$, ${}^{**}$, ${}^{*}$ indicate statistical significance at 1\%, 5\%, 10\%, respetively. Standard errors are clustered at group (village) level.
 \end{tabular}
\end{minipage} \\\\\hspace{-1cm}\begin{minipage}[t]{14cm} \hfil\textsc{\normalsize Table \refstepcounter{table}\thetable: Individual level effects of repayment shortfall (continued)\label{tab shortfall indiv2}}\\ \setlength{\tabcolsep}{1pt}
  \setlength{\baselineskip}{8pt}
  \renewcommand{\arraystretch}{.55}
  \hfil\begin{tikzpicture}
  \node (tbl) {\input{ c:/data/GUK/analysis/save/EstimationMemo/ShortfallIndividualEstimationResults2.tex }};
%\input{c:/dropbox/data/ramadan/save/tablecolortemplate.tex}
\end{tikzpicture}\\
\renewcommand{\arraystretch}{.8}
\setlength{\tabcolsep}{1pt} \begin{tabular}{>{\hfill\scriptsize}p{1cm}<{}>{\hfill\scriptsize}p{.25cm}<{}>{\scriptsize}p{12cm}<{\hfill}} 
Source:& \multicolumn{2}{l}{\scriptsize Estimated with GUK administrative data.}\\
Notes: & 1. & Group fixed effects estimates of repayment shortfall. Group fixed effects are controlled by differncing out respecive means from the data matrix. Shortfall is (planned installment) - (actual repayment). OtherShortfall indicates mean shortfall of other members in a group. Group repayment shortfall rates (GRSR) is (shortfall)/(planned installment). GRSR is defined as high if the first six months' repayment shortfall rate is above median, low if otherwise. Median GRSR is -1.42. \\
& 2. &  $P$ values in percentages in parenthesises. Standard errors are clustered at group (village) level. %${}^{***}$, ${}^{**}$, ${}^{*}$ indicate statistical significance at 1\%, 5\%, 10\%, respetively. Standard errors are clustered at group (village) level.
 \end{tabular}
\end{minipage} \\\\


\begin{figure}
\mpage{12cm}{
\hfil\textsc{\footnotesize Figure \refstepcounter{figure}\thefigure: Cumulative effects on livestock, net assets, and narrow net assets\label{fig NarrowNetAssetsLivestockEffects}}\\

\vspace{2ex}
\hfil\includegraphics[height = 8cm, width = 12cm]{c:/data/GUK/analysis/program/figure/EstimationMemo/NarrowNetAssetsNumCowsCumRelativeToConcurrentTradEffects.eps}\\
\renewcommand{\arraystretch}{1}
\setlength{\tabcolsep}{1pt}
\hfil\begin{tabular}{>{\hfill\scriptsize}p{1cm}<{}>{\scriptsize}p{12.5cm}<{\hfill}}
Source: & Constructed from ANCOVA estimation results.\\
Note:& Left most column panel shows the conditional means of \textsf{traditional} arm which serves as a benchmark in estimating impacts. In other column panels, all points show the relative difference from concurrent \textsf{traditional} levels depicted in the left most column. \textsf{Large} and \textsf{Upfront} are the same values. Other column panels are grouped either by arm or by attribute. Row panels show different outcomes. Bars show 95\% confidence intervals using cluster robust standard errors. Narrow net assets = Narrow assets + net saving - debt to GUK - debts to relatives and money lenders. Narrow assets use only items observed for all 4 rounds for household assets. \\[1ex]
\end{tabular}
}
\end{figure}




\begin{figure}
\mpage{12cm}{
\hfil\textsc{\footnotesize Figure \refstepcounter{figure}\thefigure: Effects on land, livestock, and net assets\label{fig LivestockEffects}}\\

\vspace{2ex}
\hspace{-2em}\includegraphics[height = 12cm, width = 14cm]{c:/data/GUK/analysis/program/figure/EstimationMemo/AssetRelativeToConcurrentTradEffects.eps}\\
\renewcommand{\arraystretch}{1}
\setlength{\tabcolsep}{1pt}
\hfil\begin{tabular}{>{\hfill\scriptsize}p{1cm}<{}>{\scriptsize}p{12.5cm}<{\hfill}}
Source: & Constructed from ANCOVA estimation results.\\
Note:& Left most column panel shows the conditional means of \textsf{traditional} arm which serves as a benchmark in estimating impacts. In other column panels, all points show the relative difference from concurrent \textsf{traditional} levels depicted in the left most column. \textsf{Large} and \textsf{Upfront} are the same values. Other column panels are grouped either by arm or by attribute. Row panels show different outcomes. Bars show 95\% confidence intervals using cluster robust standard errors.\\[1ex]
\end{tabular}
}
\end{figure}




\end{document}
