%  path0 <- "c:/data/GUK/"; path <- paste0(path0, "analysis/"); setwd(pathprogram <- paste0(path, "program/")); system("recycle c:/data/GUK/analysis/program/cache/ImpactEstimation2020Apr27/"); library(knitr); knit("ImpactEstimation2020Apr27.rnw", "ImpactEstimation2020Apr27.tex"); system("platex ImpactEstimation2020Apr27"); system("pbibtex ImpactEstimation2020Apr27"); system("dvipdfmx ImpactEstimation2020Apr27")

\input{c:/migrate/R/knitrPreamble/knitr_preamble.rnw}
\renewcommand\Routcolor{\color{gray30}}
\newtheorem{finding}{Finding}[section]
\makeatletter
\g@addto@macro{\UrlBreaks}{\UrlOrds}
\newcommand\gobblepars{%
    \@ifnextchar\par%
        {\expandafter\gobblepars\@gobble}%
        {}}
\newenvironment{lightgrayleftbar}{%
  \def\FrameCommand{\textcolor{lightgray}{\vrule width 1zw} \hspace{10pt}}% 
  \MakeFramed {\advance\hsize-\width \FrameRestore}}%
{\endMakeFramed}
\newenvironment{palepinkleftbar}{%
  \def\FrameCommand{\textcolor{palepink}{\vrule width 1zw} \hspace{10pt}}% 
  \MakeFramed {\advance\hsize-\width \FrameRestore}}%
{\endMakeFramed}
\newcommand{\gettikzxy}[3]{%
  \tikz@scan@one@point\pgfutil@firstofone#1\relax
  \edef#2{\the\pgf@x}%
  \edef#3{\the\pgf@y}%
}
\def\BibTeX{{\textrm{B}\kern-.05em{\textsc{i}\kern-.025em \textsc{b}}\kern-.08em
    \textrm{T}\kern-.1667em\lower.7ex\hbox{\textrm{E}}\kern-.125em\textrm{X} }}
\makeatother
\AtBeginDvi{\special{pdf:tounicode 90ms-RKSJ-UCS2}}
\special{papersize= 209.9mm, 297.04mm}
\usepackage{caption}
\usepackage{setspace}
\usepackage{framed}
\usepackage[framemethod=TikZ]{mdframed}
\captionsetup[figure]{font={stretch=.6}} 
\def\pgfsysdriver{pgfsys-dvipdfm.def}
\usepackage{tikz}
\usetikzlibrary{intersections, calc, arrows, decorations, decorations.pathreplacing, backgrounds, shadows}
\usepackage{pgfplots, pgfplotstable}
\usepgfplotslibrary{fillbetween}
\pgfplotsset{compat=1.3}
\usepackage{adjustbox}
\tikzstyle{toprow} =
[
top color = gray!20, bottom color = gray!50, thick
]
\tikzstyle{maintable} =
[
top color = blue!1, bottom color = blue!20, draw = white
%top color = green!1, bottom color = green!20, draw = white
]
\tikzset{
%Define standard arrow tip
>=stealth',
%Define style for different line styles
help lines/.style={dashed, thick},
axis/.style={<->},
important line/.style={thick},
connection/.style={thick, dotted},
}
\mdfsetup{
linecolor=black!40,
outerlinewidth=1pt,
roundcorner=.5em,
innertopmargin=1ex,
innerbottommargin=.5\baselineskip,
innerrightmargin=1em,
innerleftmargin=1em,
backgroundcolor=blue!10,
%userdefinedwidth=1\textwidth,
shadow=true,
shadowsize=6,
shadowcolor=black!20,
frametitlebackgroundcolor=cyan!40,
frametitlerulewidth=10pt,
splittopskip=2\topsep
}
\global\mdfdefinestyle{SecItemize}{%
linecolor=black!40,
outerlinewidth=1pt,
roundcorner=1em,
innertopmargin=1ex,
innerbottommargin=.5\baselineskip,
innerrightmargin=1em,
innerleftmargin=1em,
backgroundcolor=blue!10,
%userdefinedwidth=1\textwidth,
shadow=true,
shadowsize=2,
shadowcolor=black!20,
frametitlebackgroundcolor=cyan!10,
frametitlerulewidth=10pt
}


\begin{document}
\setlength{\baselineskip}{12pt}

\hfil An empirical investigation into a financial poverty trap\\

\hfil\MonthDY\\
\hfil{\footnotesize\currenttime}\\


\hfil\mpage{10cm}{\footnotesize
\textsc{\textbf{Abstract}} \hspace{1em} Despite the claimed target to the poor, the existing microcredits rarely cover the ultra poor people. Using a randomised controlled trial in a rural, low income setting of northern Bangladesh, we assess the creditworthiness of the ultra poor and a suitable microcredit scheme to help them escape from the poverty trap. We use a stepped-wedge design over the key features of loans, i.e., small-scale sequential disbursement vs lumpy upfront disbursement, with vs without grace period, and cash vs in-kind loans with managerial training programs. Compared with the traditional, Grameen-style microcredit, the provision of upfront liquidity increases both repayment rates and asset levels. This is consistent with the existence of asset-based poverty traps which can be overcome by increasing the loan size. It is also shown that entrepreneurship supports induce participation to microfinance of less experienced and poorer households, but a grace period does not change asset levels. These are accompanied with increased labour income growth towards the end of loan cycle. We interpret this as evidence of repayment discipline. Given the lack of alternative lenders in the area, we argue that high repayment rates need not generalise to other contexts. Our main findings, managerial support programs induces the participation of the ultra poor and upfront liquidity with a large sum results in faster asset accumulation that is suggestive of an escape from a poverty trap, are generalisable to other rural areas that are suited to cattle and goat production.%We estimate the impacts of entrepreneurship in microfinance under a rural, low income setting of Northern Bangladesh using a randomised controlled trial. We provided a packaged loan that bundles an asset lending with managerial support programs which is intended to render entrepreneurship unnecessary. Following the cash flow of the asset, the packaged loan has a loan maturity of three years with one year of grace period. In comparing with a classic Grameen style loan that is a third in amount and has no grace period, we add two more treatment arms which jointly serve as a bridge between the two, large loan size arms with and without a grace period. For the Grameen style loans which serves as a control group, we repeat loan disbursement twice so the total loan size becomes equivalent. We thereby obtain a stepped-wedge design over the key features of loans, i.e., upfront liquidity, grace period, and in-kind loans with support programs. It is shown that entrepreneurship supports and a grace period do not change asset levels. It is also found that upfront liquidity increases both repayment rates and asset levels. We take these results as evidence of a poverty trap which can be overcome by increasing the loan size. These are accompanied with increased labour income growth towards the end of loan cycle. We interpret this as evidence of repayment discipline. Given the relative simplicity and lack of alternative lenders in the area, we argue that irrelevance of entrepreneurial skills and high repayment rates need not generalise in other contexts. Our main finding, upfront liquidity with a large sum results in faster asset accumulation that is suggestive of an escape from a poverty trap, remains generalisable to other rural areas that are suited to cattle and goat production.

}

\newpage
\setcounter{tocdepth}{3}
\tableofcontents
\newpage

\setlength{\parindent}{1em}
\vspace{2ex}








\renewcommand{\thefootnote}{*\arabic{footnote}}
\setcounter{footnote}{0}

\section{Introduction}

% \begin{mdframed}[style={SecItemize}, frametitle={Introduction}]
% \begin{itemize}
% \vspace{1.0ex}\setlength{\itemsep}{1.0ex}\setlength{\baselineskip}{12pt}
% \item	Credit outreach to the ultra poor is slower than the moderately poor.
% 	\begin{itemize}
% 	\vspace{1.0ex}\setlength{\itemsep}{1.0ex}\setlength{\baselineskip}{12pt}
% 	\item	Demand side: The ultra poor lack entrepreneurship, access to more efficient production possibility given the small credit size. 
% 	\item	Supply side: The ultra poor may be riskier, its loan size is too small to justify the fixed costs.
% 	\end{itemize}
% \item	We supply credits to test the demand side constraints.
% \item	Test if entrepreneurship is a constraint.
% 	\begin{itemize}
% 	\vspace{1.0ex}\setlength{\itemsep}{1.0ex}\setlength{\baselineskip}{12pt}
% 	\item	Entrepreneurship: Packaged loan vs. cash loan.
% 	\item	In comparing with classic Grameen style loans, Large and LargeGrace loans are also offered, forming a stepped wedge design in \textsf{Upfront}, \textsf{WithGrace}, and \textsf{InKind} attributes.
% 	\end{itemize}
% \item	Test if nonconvexity is a constraint.
% 	\begin{itemize}
% 	\vspace{1.0ex}\setlength{\itemsep}{1.0ex}\setlength{\baselineskip}{12pt}
% 	\item	Test the existence of a poverty trap: Exists if the upfront attribute exhibits a faster asset accumulation at a no smaller repayment rate. A poverty trap is casued, in theory, production set nonconvexity, in practice, heifer-cattle production set nonconvexity.
% 	\end{itemize}
% \end{itemize}
% \end{mdframed}

	According to over 3700 microfinance institutions (MFIs), there are estimated 204 million borrowers around the world in 2013, of which 110 million are the poor borrowers whose incomes are below the national poverty line \citep{MFGateway2015}. The outreach to the extremely poor population or the \textit{ultra poor}, however, is arguably slow in comparison.\footnote{MF is not successful in reaching out to the poorest of the poor, or the ultra poor \citep[][]{Scully2004}. Empirical evidence in \citet{Yaron1994, Navajas2000, RahmanRazzaque2000, AghionMorduch2007} supports this claim. Some authors discuss the tradeoff between sustainability and outreach for microfinance institutions (MFIs) \citet{HermesLensink2011, HermesLensinkMeesters2011, Cull2011}. } This is in contrast with the idea that ``everyone is an entrepreneur'' where MFIs provide credits to the people of any income levels.\textcolor{red}{[Abu-san: Can you get a reference for this in the \BibTeX format? I have a 2017 Guardian article quoting \href{we are all entrepreneurs}{https://www.theguardian.com/sustainable-business/2017/mar/29/we-are-all-entrepreneurs-muhammad-yunus-on-changing-the-world-one-microloan-at-a-time}]}

	The potential reasons behind the slow outreach to the ultra poor can be classified into demand and supply sides. On the demand side, the ultra poor borrowers may not be entrepreneurial enough to demand credits for production, or may face an inferior production possibility than wealthier borrowers. On the supply side, MFIs may perceive the ultra poor as riskier than the moderately poor, or the loan size may be too small to justify the fixed transaction costs while the lender is constrained to keep the interest rate low to avoid adverse selection and moral hazard. However, the creditworthiness of the ultra poor and the resultant welfare dynamics have been rarely examined empirically. The existing program targeting the ultra poor, exemplified by BRAC graduation program, is based on grants, rather than loans, and the sustainability of such a program may be questionable given the limited governmental budgets.

	In assessing the creditworthiness and suitable credit scheme for the ultra poor on a quasi-commercial basis, we ran a randomised controlled trial on the poorest population in the river island areas of northern Bangladesh. By changing the conditions on the credit supply, we estimate and test the effects of demand side constraints. Specifically, by collaborating with a local NGO operating microcredit,  we assign the following four arms that have the equivalent loan size with different characteristics in frontloaded liquidity, grace period, the lending vehicle, and bundling of support programs: (1) A traditional Grameen-style microcredit programme with a small loan amount, which requires clients to start repayment within two weeks after receiving the loan, with one-year maturity. Loan disbursement is repeated twice so the total loan size becomes the same with other schemes; (2) A loan that is three times larger than the regular programme with three-year maturity; (3) A loan that equivalent to the second one, except that it comes with a one-year grace period before they start to repay; (4) An in-kind loan equivalent to the third one bundled with services to implement a microenterprise project. Hence, we take a stepped-wedge design over the key attributes of loans: small-scale credit vs upfront lumpy credit; with vs without a grace period; and cash vs in-kind loans with support programs. By comparing the additional traits assigned to each treatment arm, this study aims to identify the demand-side constraints as well as welfare and asset dynamics of alternative credit schemes. 
	%we test the necessity of entrepreneurial skills in successfully completing a loan cycle. To do so, we offered a packaged loan that bundles an asset lease with managerial support programs which is intended to render entrepreneurship unnecessary. Provided that our managerial support program covers a sufficiently wide range of issues, the package is expected to achieve a return that is no smaller than a regular credit, even when the entrepreneurial skills are essential. As we track all --- barring the flood victims whose villages were washed away --- the potential borrowers including who eventually opted out the borrowing, we are able to estimate the intention-to-treat effects of offering loans and their implied necessity for entrepreneurial skills. 
	
	The leased out asset, a heifer, is a prime investment choice in the studied area. where periodic floods and land erosion frequently threaten the livelihoods of its dwellers. A heifer needs to be at least two years old to start lactation. The maximum amount households can borrow is set at the affordable level to purchase a one-year old heifer except for the traditional Grameen-style credit arm. We explore the investment decisions and welfare consequences induced by relaxing liquidity constraints in comparison between (1) and others.  Since  households with a one-year old heifer must wait for another one year before it starts to lactate, we give one year grace period to match the cash flow profile of presumed (dairy cattle) production in treatment (3) and (4).  

	An in-kind offer in treatment (4) is generally thought to be less efficient than a cash offer as it takes away a choice from the borrower. In addition, only the borrowers of cash lending have a chance to misuse the cash unproductively. However, the local microfinance practitioners widely agree that other production opportunities are limited in our setting given environmental constraints.\footnote{I is also notable that a closely related project in the neighbouring areas transfers an asset in the form of a cow\citep{BandieraBRAC2017}. }  
	%It is generally thought in practice that an in-kind offer, with only a single asset to lease out, is less efficient than a cash offer as it takes away a choice from the borrower. However, the local microfinance practitioners widely agree that little is lost in a production opportunity even when the loan takes an in-kind form in a heifer, because a heifer is almost the only investment choice in our study area.\footnote{I is also notable that a closely related project in the neighbouring areas transfers an asset in the form of a cow\citep{BandieraBRAC2017}. } If this presumption is correct, it gives a unique chance to compare cash lending with in-kind lending, even without controlling for the different choice set of projects. In the later section examining the income generating activities, we show that this is actually the case.
	We also found in our data that most of cash borrowers started to invest in cattle, hence we find no evidence of misuse. Consequently, in our study, the cash-grace period and in-kind-grace-period lending differ only in the bundled services provided in the latter. Given the small set of the productive investment choices in our setting, our experiment gives a unique chance to compare cash lending with in-kind lending, even without controlling for the different choice set of projects. Moreover, a packaged loan that bundles an asset lease (one-year old heifer) with managerial support programs is intended to render entrepreneurship unnecessary, which is thought to be one of the major constraints for the ultra poor to access the microcredit. Provided that our managerial support program offers a sufficiently wide range of services, the package is expected to achieve a return that is no smaller than a regular credit, even when the entrepreneurial skills are essential. 
	As we track all --- barring the flood victims whose villages were washed away --- the potential borrowers including who eventually opted out the borrowing, we are able to estimate the intention-to-treat effects of offering loans and their implied necessity for entrepreneurial skills.
	%A heifer needs to be at least 2 years old to start lactation. As the packaged loan provides a heifer of one year old, we give one year of grace period. In comparing with the classic Grameen style loan that is smaller in amount and has no grace period, we add two more treatment arms which jointly serve as a bridge between the two, large loan arms with and without a grace period. For the Grameen style loans which serves as a control group, we repeat loan disbursement twice so the total loan size becomes the same for all arms. We thereby obtain a stepped-wedge design over the key attributes of loans, i.e., frontloaded liquidity, a grace period, and in-kind loans with support programs.

	Our study closely follows the literature of microfinance design as hallmarked in \citet{Field2013} who found a grace period induces more risk taking and subsequent loan delinquency. Similar to their study, we allow selected borrowers a grace period in repayment. Under our setting, however, it is irrational to invest in riskier assets when the designed grace period suits the actual cash flow, provided that a heifer has a Pareto-dominant risk-return investment profile. A strategic default is also more difficult in our setting because the number of alternative credit suppliers is limited, which is probably zero.\footnote{As we surveyed the area before the study, we note several NGOs provide a relief credit to flood victims, not regular finance. We also choose population without access to any financial institution.\textcolor{red}{[Abu-san: A better description for this?]} } We therefore expect a larger loan size, longer maturity, and a grace period would not directly result in moral hazard both in \textit{ex ante} and \textit{ex post} sense. 

	Our study is closely related to a large scale cattle transfer study conducted in the neighbouring area \citep{BandieraBRAC2017, Balboni2020}. The targeted population of their study is similar to ours, yet our study population resides on less stable terrain, are more exposed to flood and water logging, are considered to be less well connected to the market, are equally less trained, and are probably poorer. The chance of survival for each investment is expected to be no higher. The difference in experimental design is that they use a transfer while we use loans and leases, and charge a market-rate fee to everything we provide. Our experiment is designed to be financially viable if the repayment is made. 

	Our study is also related to the poverty trap literature. With the stepped-wedge design, the difference in the loan size allows us to test if there is an increasing returns to scale, or nonconvexity in the production set. A nonconvex production set is considered as the leading cause of a poverty trap in development economics \citep{GalorZeira1993}. In our study area, a regular, small scale loan is just the size of acquiring a goat or a sheep, while a larger sized arm allows a purchase of a heifer. When the return to a heifer is higher than to a goat/sheep, there is a scope of poverty trap because a heifer cannot be acquired in parts and the borrower is facing a binding credit constraint. Our study therefore serves to provide micro-level evidence of a poverty trap that is frequently studied in macroeconomics.
	
	We found that having large liquidity in upfront leads to faster asset accumulation and higher loan repayment rates. We consider this as evidence of a poverty trap which is formed by the nonconvex production set of a heifer, consistent with the recent study by \citet{Balboni2020}. There is little difference in the outcomes between cash and in-kind lending as well as with and without a grace period. This implies they have no ITT effects. We interpret this as due to a more homogenous investment opportunity in the area compared with the urban setting of \citet{Field2013}. Looking closely at the participant characteristics, however, we found that in-kind lending attracted the borrowers with less cattle rearing experiences and lower asset values. In interpretation, people who are not entrepreneurial enough to overcome less cattle rearing experiences and not entrepreneurial enough to take risks at a lower asset level did not participate in the cash lending arms, while did so in the in-kind lending arm. This hints some entrepreneurship is required for microfinance participation and repayments, and the managerial support program might have closed such gaps in entrepreneurial skills. 
	
	%We found that entrepreneurship is not a prerequisite for microfinance lending and repayments. There is little difference in the outcomes between the in-kind lending and cash lending. We intepret this as due to a more homogenous investment opportunity in the area compared with the urban setting of \citet{Field2013}. We found that having upfront liquidity is the key to faster asset accumulation and higher loan repayment rates. We consider this as evidence of a poverty trap which is formed by the nonconvex production set of heifer. 

	In the following section, we summarise the existing literature. Section \ref{SecBackground} gives the brief account of background of study site. Section \ref{SecTheory} shows a possible mechanism of poverty trap that our target population is under. Section \ref{SecExperimentalDesign} lays out the details of experimental design. Section \ref{SecEmpiricalStrategy} explains the estimation strategy. In section \ref{SecResults}, we provide a brief overview of the experimental results. Section \ref{SecConclusion} discusses the interpretation of results.

\section{A brief review of existing studies}
\label{SecExistingStudies}

% \begin{mdframed}[style={SecItemize}, frametitle={Existing studies}]
% \begin{itemize}
% \vspace{1.0ex}\setlength{\itemsep}{1.0ex}\setlength{\baselineskip}{12pt}
% \item	A relatively high uptake rate (among members) of our study poses less of the statistical power issue that plagues the benchmark study of \citet{BanerjeeKarlanZinman2015}
% \item	Heterogenous impacts of microcredits: Experiences/skills matter. Our study shows that skills do not matter for the impacts on the extensive margins.
% \item	Mixed and weak impacts of MFI training programs: Entrepreneurial skills are not trained easily, implying entrepreneurial skills, if required, may have to be outsourced in production processes. Our study shows entrepreneurial skills may not be required for production at this micro level of production.
% \item	Grace period: Our study is marked to actual cash flow profile, thereby easing the term mismaych, which explains the reduction of defaults
% \item	Lending suffices: We also observe sustained asset level increase as in asset transfer programs
% \end{itemize}
% \end{mdframed}

	Much has been discussed about the poverty reduction impacts of microfinance in the early days of microfinance studies \citep{Morduch1999}. Recently, doubts are cast on the magnitude of microfinance impacts \citep{BanerjeeKarlanZinman2015, DuvendackMader2019, Meager2019} while asset grants (capital injection) remain to show high returns \citep{deMel2008, DeMel2014, FafchampsFlypaper2014, BandieraBRAC2017}. \footnote{This is due partly to insufficient statistical power \citep{MckenzieWoodruff2013}. \citet{BanerjeeKarlanZinman2015} collects five studies of microfinance lending impacts. They raise lack of statistical power due to low take up. This naturally gives way to erroneously large impacts. \citet{BanerjeeKarlanZinman2015} point that more able and experienced borrowers saw larger, ``transformative effects.'' In the current study, in contrast, the up take rate is relatively high at 75\%, of which 5\% is lost to flood.  } Lack of mean impacts led researchers to look for a particular subgroup which shows impacts, or impact heterogeneity \citep{Banerjee2017HyderabadFollowup}: Borrowers with prior experiences or high ability are shown to have higher returns \citep{Banerjee2015Miracle, Mckenzie2017Spurring, Buera2017}. The studies with a focus on experienced members or existing firms can be considered as looking at impacts on the intensive margins. In contrast, our study is targeted to an isolated greenfield population. We look at impacts on the extensive margins which are relatively less studied.

	The fact that experienced members gain larger benefits from microcredit is consistent with the positive impacts of capital grant programs on existing firm owners. Whether such experience is trainable for novice entrepreneurs remains unsettled. A growing body of management capital literature in developing countries is insightful yet most of the research is necessarily geared to existing firms, so it does not inform much on how one can assist novice entrepreneurs.\footnote{\citet{BruhnKarlanSchoar2018} shows intensive management consulting services to the small scale firms in Mexico resulted in sustained improvements in management practices which led to higher TFP and larger employment. Others also show effectiveness \citep{Calderon2011, Berge2012, Bloometal2013} while others do not \citep{Bruhn2012, KarlanKnightUdry2015}. \citet{MckenzieWoodruff2013} put them as: These managerial impacts studies are too different to compare, in terms of population, interventions, measurement (variables, timing), and most importantly, implied statistical power in the design. } \citet{KarlanValvidia2011, BruhnZia2011, Argent2014} are the exceptions, but results and quality of evidence are mixed and inconclusive. The current study explicitly tests if the entrepreneurship matters in microfinance outcomes for an isolated greenfield population, which is still relatively less studied. %Entrepreneurship and training components in the current study are to provide basic knowledge of dairy cattle production which can easily be written down. They, the cristalised intelligence, are outsourceable in nature. We consider it is the skills to deploy services in a timely manner, rather than the knowledge contents \textit{per se}, that we provide to help borrowers in increasing efficiency.

	Another strand of the literature links capital grant effectiveness with production set nonconvexity. Theories base lumpiness and credit market imperfection as keys to a povety trap \citep[e.g., ][]{GalorZeira1993} but its empirical application is scant. When the production set is convex, a small scale transfer may not lead to a sustained increase in income, as it can be either consumed or invested to a technology with decreasing marginal returns that brings back to the original income level (i.e., the lower equilibrium of poverty trap). A few studies of transfer programs report sustained increase in assets and incomes. A transfer program in northern Bangladesh that is closely related to this study shows an occupational change and an income increase \citep{BandieraBRAC2017}. Other transfer programs to the ultra poor also show increases in incomes and assets \citep{Blattman2014, BanerjeeetalScience2015, Blattmanetal2016, HaushoferShapiro2016}. \citet{BanerjeeetalScience2015} reports increased consumption, asset levels, saving, various incomes of the ultra poor after receiving a large transfer. %\citet{Kaboski2018Indivisibility} uses a lab-in-the-field experiment to show the link between investment indivisibility, saving, and patience. 
	Similar to these studies, our study our study examines the nonconvexity of higher-return production set.
	finds evidence consistent with the nonconvexity of higher-return production set. Our study incorporates a heifer lease. As a heifer requires a lead time before producing milk, we introduced a grace period in the cash lending to make the comparison with the heifer lease more straightforward. Previous research  in the urban setting has shown that a grace period induces more aggressive risk-taking \citep{Field2013}. %shows that a two-month grace period increases the investment size, raises profitability, but also increases the default rates. They discuss how it influences the investment riskiness that varies along risk preference heterogeneity. 
	The experimental setting of the current study has much a smaller choice set that limits the scope of risk taking. The design of this study is in line with \citet{Beaman2015} who redesigned the repayment schedule adopted to the borrower's cash flow profile (repay after harvest), thus, on a good faith, a grace period is expected to reduce delinquency because the term mismatch is eased. 


\section{Background}
\label{SecBackground}

% \begin{mdframed}[style={SecItemize}, frametitle={Background}]
% \begin{itemize}
% \vspace{1.0ex}\setlength{\itemsep}{1.0ex}\setlength{\baselineskip}{12pt}
% \item	Lowest income area with high annual flood risks
% \item	No NGO/MFI presence
% \item	Argue: Cattle $\succcurlyeq$ goat in risk-return if invested
% \item	But: higher inputs and upfront fixed costs
% \item	Goats: Can have higher returns (in developed countries, smaller inputs and higher fertility), but worse in mortality/morbidity risks, particularly to water logging
% \item	Goat cash flow: Meat demand or kid sales require relatively high incomes, is infrequent
% \item	Cows inputs: Vaccination, fodder
% \item	Cow cash flow: Milk sales is more frequent, a calf sales requires even higher incomes
% \end{itemize}
% \end{mdframed}

	he study area is in the river island, known as Chars in Bengali, of northern Bangladesh in Gaibandha and Kurigram districts. Chars are formed by sediments and silt depositions and are prone to cyclical river erosions and floods. Chars are not stable in size and even in existence, and episodes of their partial or complete erosion or submerging are quite common. Chars accommodate ultra-poor inhabitants who are forced, as a desperate attempt for survival, to relocate across islands due to river erosion and floods.

	In the study area, cattle and goats/sheep are the main livestock that residents own. Rearing costs are higher for cattle as it requires fodder while a goat will eat the bushes. Cattle requires vaccination shots when a goat is usually left unvaccinated. Reproductive capacity of goats are high that parity size approaches to 2 at the third birth, and the birth interval is about 200 days \citep{Hasan2014goat}. An indigenous cow has a birth interval of 375 to 458 days \citep{Hasan2018}, resulting in about 2 years for gestation and calving interval \citep{Habib2012} with the mean lifetime births of 4 \citep[][Table 1]{Hasan2018}. 
	
	Goat kid's potentially higher reproductive capacity and lower rearing costs are, however, more than offset by the elevated morbidity and mortality risks, and a less frequent cash flow. Indeed, morbidity of goat kids ranges from 12\% \citep{Mahmud2015} to more than 50\% in some diseases \citep[][Table 5]{Nandi2011}, while cattle morbidity is around 22\% \citep{Bangar2013}. Goat kid mortality ranges from 6\% \citep{Mahmud2015} to 30\% \citep[][Table 5]{Paul2014} \citep{Ershaduzzaman2007}. Heifer mortality is between 5\% \citep[][p.332R]{Hossain2014} to 10\% \citep{Alauddin2018}. Higher morbidity of goat kids partly reflects their eating style that uses lips rather than tongues (as cattles do) and vulnerability to logging water. 

	Lactation length is 227 days and milk yield is 2.2 kg per day \citep{Rokonuzzaman2009} while goat milk is seldom marketed. It is also worth noting that a meat market requires a cluster of relatively high income earners, which takes some efforts to get to from the river islands. Goat meat sales is seasonal and it does not provide a frequent cash flow. Residents also report that a goat herd is less mobile than single cattle when they are forced to evacuate during the flood. All of these considerations prompt residents to opt for cattle when they can afford it, and do not expand the herd size of goats, which are both confirmed in our data.

\section{Experimental design}
\label{SecExperimentalDesign}


% \begin{mdframed}[style={SecItemize}, frametitle={Experimental design}]
% \begin{itemize}
% \vspace{1.0ex}\setlength{\itemsep}{1.0ex}\setlength{\baselineskip}{12pt}
% \item	Stepped-wedge design allows us to test a series of contraints.
% 	\begin{itemize}
% 	\vspace{1.0ex}\setlength{\itemsep}{1.0ex}\setlength{\baselineskip}{12pt}
% 	\item	Cow vs. Large grace: Entrepreneurship constraint (\textsf{InKind})
% 	\item	Large grace vs. Large: Saving constraint (\textsf{WithGrace})
% 	\item	Large vs. Traditional: Liquidity constraint (\textsf{Upfront})
% 	\end{itemize}
% \item	Interpretation of entrepreneurship contsraint: Access to (textual) knowledge does not increase profits
% \item	We track everyone except flood victims. This allows us to consistently estimate the ITT effects.
% \end{itemize}
% \end{mdframed}

	To investigate the detailed demand-side constraints and suitable credit scheme for the ultra poor, we implemented the village-level clustered randomization across the four treatment arms as follows:

	\begin{description}
	\vspace{1.0ex}\setlength{\itemsep}{1.0ex}\setlength{\baselineskip}{12pt}
	\item[T1]	Traditional microcredit. The design of this treatment arm is similar to that of the flagship Grameen-style microcredit lending, which is widespread in Bangladesh. Under this treatment arm, members of the group will receive 5600 taka credit, with loan repayment beginning two weeks after disbursement. Members repay with weekly installments and are required to attend weekly meetings as well as to regularly save an amoun	t decided jointly by the group members. The contract maturity is one year, and borrowers are allowed to make another two loan contracts of equivalent amounts over the next consecutive years. The weekly repayment is 125 taka payable in 50 installments.
	\item[T2]	Upfront lumpy credit without a grace period. Under this treatment arm, group members receive 16,800 taka credit with a longer period of loan maturity, with loan repayments beginning two weeks after disbursement. The weekly repayment is the same amount as in the T1 arm. The compulsory saving applies as in the T1 arm. The loan maturity is three years. The required weekly repayment is 125 taka payable in 150 weekly instalments (for three years). 
	\item[T3]	Upfront lumpy credit without a grace period. Under this treatment arm, group members receive 16,800 taka credit with loan repayments beginning one year after disbursement. During the first year grace period, members are required to meet weekly and follow group activities such as compulsory savings just as in other arms. The compulsory saving is the same as in the T1, T2 arms. The loan maturity is three years. The required weekly repayment is 190 taka payable in 100 weekly installments, starting after one year.
	\item[T4]	In-kind credit with a one-year grace period and managerial support programs. Under this treatment arm, group members receive in-kind credit in the form of a one-year old heifer, within the price range of 16,000 taka with loan repayment beginning one year after disbursement. In comparison with smaller livestock such as goats, cows are more versatile in flood-prone areas. They typically need to be about 15 months old to be ready for insemination and takes about 9.5 months to deliver a calf as it starts lactation, or the total of about 2 years. This corresponds to the grace period length provided under the T3 and T4 arms when we acquire one year old heifers. In addition, the members receive fodder, training on cow rearing, regular veterinary and vaccination services, and marketing consultancy services from the local NGO, at the total fee of 800 taka charged over the three years. 
	\end{description}

	One of the aims of the study is to assess if the entrepreneurship matter in microfinance lending outcomes. Assuming that the economically most lucrative asset is a heifer, we bundle training with a heifer lease. At the start of a lease, our expert procures a heifer from the local market, so the leasee does not have to have the knowledge required for the quality purchase. We provide knowledge to a group of borrowers through training and disallow an investment choice by leasing out an asset, so some part of entrepreneurship will no longer be a prerequisite. It can be seen that we are offering a capacity to use the best practice or the \textit{cristalised intelligence} related to cattle production \citep{Cattell1963}. This is only a part of entrepreneurial skills. The remainder, a capacity to apply a suitable action to unforeseen events or the \textit{fluid intelligence} related to cattle production is left unchanged. Borrowers of other arms who are not provided the knowledge may opt out the loan or perform worse, if entrepreneurship raises productivity. One can measure impacts of entrepreneurship by comparing these two groups, in-kind credit vs. cash credit.
	%So the estimated impacts may reflect a hightened awareness to the production knowledge that can be manipulated by outsiders. This characterisation suggests that the entrepreneurial skills we provide overlaps with what the professional consultants advise in the management capital literature. %If there is an impact of cristalised intelligence, then one can compute its net returns. We note that, in an attempt not to give any monetary subsidy, we are charging fees at market prices to all the services we provide at request: Fodder supply, milk marketing, veterinary, and insemination. What we do not charge is a form of consulting services that answer to the questions from members who may lack practical knowledge of cattle production. We also provided compulsory training. \footnote{This could have served as a levy on the members of in-kind arm if they already had the relevant knowledge, as we did not give a choice of nonparticipation had they agreed to receive an in-kind loan. Majority of households had no experience in dairy cattle production at the baseline, and we expect them to lack the practical knowledge. }  
	
	As a natural reference, we compare the in-kind packaged loan (T4) with the traditional regular microcredit, a classic Grameen style loan that is about a third in loan size and maturity with no grace period (T1). In order to make comparison feasible, we added two intermediate treatment arms to bridge them: Two arms with a large amount of cash lending that is equivalent of heifer price, one with a grace period (T3) and another without a grace period (T2). With the loan sizes that are three times the traditional microfinance loans, we extended the maturity to three years. The comparison arm, the traditional regular microcredit, has only one year maturity. We therefore provided the total of three loans in three loan cycles which are unconditionally disbursed annually so the total loaned amount will be aligned and there is no selection before three cycles are complete due to delinquency. 
	
	In these settings, frontloading liquidity without changing total loan size eases a liquidity constraint, attaching a grace period under the same loan size and disbursement timing eases a saving constraint prior to loan receipt, and offering an in-kind lease with managerial support without changing loan size, disbursement timing, and a grace period eases an entrepreneurship constraint. In effect, we constructed a stepped-wedge design over these key features of loans, namely, upfront liquidity (\textsf{Upfront}), a grace period (\textsf{WithGrace}), and in-kind with managerial supports (\textsf{InKind}), to assess respective impacts on the outcomes as indicated in \textsc{\normalsize Table \ref{tab factorial design}}.

	Our sample is drawn from the population of river island villages in Northern Bangladesh. We selected the areas of no NGO/MFI activity. 80 villages are randomly chosen and we formed a member committee of 10 households, of which 6 are ultra poor and 4 are moderately poor. The poverty status was determined by a participatory ranking process. We randomised the loan arms at the village level. All loan products are of individual liability and the committee was intended to serve as an activity platform for MFI operations.

	Baseline data was collected in 2012 prior to the loan type randomisation. After offering the arms, three groups opted out as a group. This was unexpected as we have explained the loan types, the random assignment process, and have obtained everyone’s consent to participate before randomisation. Although they refused to receive a loan, they gave a consent to be surveyed so we track them in subsequent survey rounds. We further lost four groups to the flood in 2013. As they relocated, we had no choice but to drop them from the study. Counting other individual attriters, we have 116 subjects (14.5\%) who attrited by the final round. We find that attrition as random (\textsc{\small Table \ref{tab attrit perm MainText}}). In addition to group level rejection/attrition, we had 90 individual loan rejectors. They agreed to receive a loan before we offered it, and they changed their minds. We retain them in the study as they agreed to be surveyed even in the absence of loans. As a result, we have flood victims whom we do not track, group rejectors, individual rejectors and borrowers that we track. See \citet{GUK2016} for more details on the randomisation and acceptance process.



\begin{table}
\hspace{-1cm}\begin{minipage}[t]{14cm}
\hfil\textsc{\normalsize Table \refstepcounter{table}\thetable: A 4$\times$4 factorial, stepped wedge design\label{tab factorial design}}\\
\setlength{\tabcolsep}{1pt}
\setlength{\baselineskip}{8pt}
\renewcommand{\arraystretch}{.55}
\vspace{2ex}
\hfil\begin{tabular}{>{\footnotesize\hfill}p{2cm}<{}
>{\footnotesize\hfil}p{2.5cm}<{}
>{\footnotesize\hfil}p{2.5cm}<{}
>{\footnotesize\hfil}p{2.5cm}<{}}
					& \cellcolor{paleblue}\textcolor{black}{large, grace} 			& \cellcolor{paleblue}\textcolor{black}{large} & \cellcolor{paleblue}\textcolor{black}{traditional} \\\cellcolor{paleblue}
\textcolor{black}{cow} 				& \mpage{2.5cm}{\hfil entrepreneurship\\\hfil constraint\\\hfil (\textsf{InKind})} &\cellcolor{gray80}\mpage{2.5cm}{\textcolor{gray}{\hfil saving\\\hfil constraint\\\hfil (\textsf{WithGrace})}} &\cellcolor{gray80}\mpage{2.5cm}{\textcolor{gray}{\hfil liquidity\\\hfil constraint\\\hfil (\textsf{Upfront})}}\\\cellcolor{paleblue}
\textcolor{black}{large, grace} &\cellcolor{gray20} 	&  \mpage{2.5cm}{\hfil saving\\\hfil constraint\\\hfil (\textsf{WithGrace})} & \cellcolor{gray80}\mpage{2.5cm}{\textcolor{gray}{\hfil liquidity\\\hfil constraint\\\hfil (\textsf{Upfront})}}\\\cellcolor{paleblue}
\textcolor{black}{large} 			&\cellcolor{gray20} 	&\cellcolor{gray20}& \mpage{2.5cm}{\hfil liquidity\\\hfil constraint\\\hfil (\textsf{Upfront})}%\\\cellcolor{pink}
%\textcolor{black}{control} & \multicolumn{3}{c}{\cellcolor{green}\textcolor{black}{level \hspace{1em} impacts}}
\end{tabular}
\end{minipage}

\footnotesize Note: \mpage{12cm}{\footnotesize Cell contents are hypothesised constraints on investments that exists in the column arm but are eased in the row arm. Contents in brackets are variable names of respective attributes.}
\end{table}


\section{Empirical strategy}
\label{SecEmpiricalStrategy}

% \begin{mdframed}[style={SecItemize}, frametitle={Empirical strategy}]
% \begin{itemize}
% \vspace{1.0ex}\setlength{\itemsep}{1.0ex}\setlength{\baselineskip}{12pt}
% \item	We use ANCOVA estimates.
% \item	We estimate both arm wise impacts and attribute wise impacts.
% \end{itemize}
% \end{mdframed}

	We collect data in one baseline and three annual follow up surveys. With successful randomisation (see Section \ref{ResultsSectionParticipation} and Appendix \ref{AppSecRandomisation}), we use ANCOVA estimators to measure impacts of each experimental arms and loan attributes. ANCOVA estimators are more efficient than DID estimators \citep{FrisonPocock1992, McKenzie2012}. As we include loan rejecters, what we are estimating is intention-to-treat effects. For an ease of interpretation, we will also use indicator variables for each attribute, \textsf{Upfront, WithGrace, InKind}. Numerically, both are equivalent. Arms and attributes are just two ways of labeling the same data, so, in what follows, we will jointly refer to them as attributes for simplicity.
	
	The basic estimating equation for our intention-to-treat effects is:
	\begin{equation}
	y_{it}=b_{10}+\bfb'_{1}\bfdee_{i}+b_{2}y_{i0}+e_{it},
	\end{equation}
	where, for member $i$ in period $t$, $y_{it}$ is an outcome measure, $\bfdee_{i}$ is a vector of indicator variables in loan attributes that $i$ receives, $e_{it}$ is an error term. For the \textsf{traditional} arm, the conditional mean of outcome given covariates and baseline outcome variable is given by $b_{10}$. For an attribute $a$, the impact relative to the traditional arm is measured with $b_{1a}$. As we are interested in the time course of impacts, we allow for time-varying impacts as:
	\begin{equation}
	y_{it}=b_{10}+\bfb'_{1}\bfdee_{i}+b_{t0}c_{t}+\bfb'_{t}c_{t}\bfdee_{i}+b_{2}y_{i0}+e_{it},
	\label{EstimatingEqTimeVarying}
	\end{equation}
	where $c_{t}$ is a period indicator variable for $t>1$ that takes the value of 1 at $t$, 0 otherwise. We use the second period (period 2 in most cases) as the reference for time dummies. $b_{t0}$ measures the period $t$ deviation from $b_{10}$ for the \textsf{traditional} arm, $\bfb'_{t}$ measures the period $t$ deviation from the concurrent \textsf{traditional} arm for each attribute. For the \textsf{traditional} arm, the conditional mean of outcome given covariates and baseline outcome variable is provided by $b_{10}+b_{t0}$. All the standard errors are clustered at the group (char) level as suggested by \citet{AbadieAtheyImbensWooldridge2017}.%\footnote{To aid the understanding if the data is more suited to the assumption of first-difference (FD) rather than fixed-effects (FE), we use a check suggested by \citet[][10.71]{Wooldridge2010}. It is an AR(1) regression using FD residuals. Most of results show low autocorrelations in FD residuals which is consistent with the assumption of FD estimator. The issue of choice between FD or FE is not of primary importance, as the use of cluster-robust standard errors gives consistent estimates of SEs in both estimators, and it boils down to efficiency.  }


\section{Results}
\label{SecResults}






% \begin{mdframed}[style={SecItemize}, frametitle={Results}]
% \begin{itemize}
% \vspace{1.0ex}\setlength{\itemsep}{1.0ex}\setlength{\baselineskip}{12pt}
% \item	Randomisation went well at the group level.
% \item	Group loan rejecters of \textsf{traditional} and non-\textsf{traditional} differ.
% 	\begin{itemize}
% 	\vspace{1.0ex}\setlength{\itemsep}{1.0ex}\setlength{\baselineskip}{12pt}
% 	\item	\textsf{traditional}: Lower livestock values, smaller cattle holding, and smaller net asset values. Non-\textsf{traditional}: A higher baseline flood exposure rate, a younger household head, and higher cattle holding.
% 	\item	Traditional arm rejecters are relatively less wealthy than non-traditional rejecters, consistent with a binding liquidity constraint which prevented them from participation, and rejecters might have participated to large sized lending if offered.
% 	\end{itemize}
% \item	Individual rejecters are similar between \textsf{traditional} and non-\textsf{traditional} arms. 
% 	\begin{itemize}
% 	\vspace{1.0ex}\setlength{\itemsep}{1.0ex}\setlength{\baselineskip}{12pt}
% 	\item	Common factors relative to nonrejecters: Smaller household size and smaller livestock holding.
% 	\item	These hint that it may take a larger household size to raise cattle, and (conditional on household size) households who have more livestock may have the capacity to raise more. These are consistent with a domestic capacity (labour and/or space) contsraint and a liquidity constraint.
% 	\end{itemize}
% \item	Overall, attrition is not correlated with household characteristics.
% 	\begin{itemize}
% 	\vspace{1.0ex}\setlength{\itemsep}{1.0ex}\setlength{\baselineskip}{12pt}
% 	\item	Less educated members attrited in traditional arm indicates there may be underestimation, if there is an attrition bias at all (so, no need to use Lee bounds, I think).
% 	\end{itemize}
% \item	Greater accumulation of assets (livestock, productive assets, land holding, household assets) for \textsf{Upfront} attribute
% 	\begin{itemize}
% 	\vspace{1.0ex}\setlength{\itemsep}{1.0ex}\setlength{\baselineskip}{12pt}
% 	\item	More diverse and smaller scale investment portfolio among \textsf{traditional}. 
% 	\end{itemize}
% \item	Lower repayment rates for \textsf{traditional}. 
% 	\begin{itemize}
% 	\vspace{1.0ex}\setlength{\itemsep}{1.0ex}\setlength{\baselineskip}{12pt}
% 	\item	This also is at odds with a popular belief `start small and grow' is more prudent.
% 	\end{itemize}
% \item	Greater asset accumulation and higher repayment rates for \textsf{Upfront} is consistent with nonconvex production, or a poverty trap.
% \item	No impacts of \textsf{InKind} on asset accumulation.
% 	\begin{itemize}
% 	\vspace{1.0ex}\setlength{\itemsep}{1.0ex}\setlength{\baselineskip}{12pt}
% 	\item	Negates the necessity of entrepreneurship, 
% 	\item	This is in contrast to the finding of existing studies that impacts are larger for the experienced borrowers ... everyone can be an entrepreneur at this level of skills and production possibilities? 
% 	\end{itemize}
% \item	No impact on consumption. 
% \item	Larger increase in labour incomes in period 3, probably induced by a repayment burden and borrower's compliance.
% \item	Schooling of children is generally not affected.  
% 	\begin{itemize}
% 	\vspace{1.0ex}\setlength{\itemsep}{1.0ex}\setlength{\baselineskip}{12pt}
% 	\item	While this is reassuring, there is a weak indication that female schooling at college level may have been adversely affected in \textsf{Large} arm and positively affected by \textsf{WithGrace} attribute in period 4.
% 	\end{itemize}
% \end{itemize}
% \end{mdframed}


\subsection{Participation}
\label{ResultsSectionParticipation}

	The reasons behind nonparticipation are fundamental in understanding the outreach. In addition, selective attrition from the sample, if any, biases the estimates so we need to compare the attriter's characteristics with the nonattriters. In this section, we check how participation and attrition are different between the arms. To do so, we test if the household characteristics are different between participants and rejecters, or attriters and nonattriters. We use permutation tests to examine if there is a difference in mean characteristics between any two groups. We use 100000 random draws from all admissible permutations. Holm's step-down method is used to adjust $p$ values for multiple testing of multi-factor grouping variables.

	Before examining participation decisions, we confirm the randomisation balance. Despite there were rejections to participate at the group level, we see randomisation balance was reasonably achieved as there is no household characteristics whose $p$ value for the mean difference to exceed 10\% between intervention arms (\textsc{\normalsize Table \ref{tab perm}} in Appendix \ref{AppSecRandomisation}). %We also check livestock holding in all rounds. We see that it is indistinguishable between arms at the baseline (\textsc{Table \ref{table anova CattleHoldingArm}}).

	We examined the difference between various groups defined by rejections and attrition in Appendix \ref{AppSecAttritionRejection}. In summary, group rejecters of \textsf{traditional} and non-\textsf{traditional} differ. Lower livestock values, smaller cattle holding, and smaller net asset values are associated with group rejection for \textsf{traditional} arm (\textsc{\normalsize Table \ref{tab Greject trad perm}}), while a higher baseline flood exposure rate, a younger household head, and higher cattle holding are associated with group rejection for non-\textsf{traditional} arms (\textsc{\normalsize Table \ref{tab Greject nontrad perm}}). Given randomisation, we conjecture that it is lack of liquidity, or lack of \textsf{Upfront} attribute, prevented smaller livestock holders of \textsf{traditional} arm from participating because they cannot purchase cattle due to insufficient net asset values or an insufficient resale value of livestock, when members of similar characteristics partcipated in non-\textsf{traditional} arms. %This is a real resource constraint that binds the households. This is different from a psychological constraint that, so long as there is a cost or a payment involved, albeit at a minimal level, there remains a group of households who would not take up the investment \citep{Ashraf2010, CohenDupas2010}. 
	It is reasonable to see that flood victims did not participate in the non-\textsf{traditional} arm, when they are younger and have already relatively more cattle than average. 

	While group rejecters have different characteristics between \textsf{traditional} and non-\textsf{traditional} arms, we observe the similarity between individual rejecters of \textsf{traditional} arm and non-\textsf{traditional} arms (\textsc{\normalsize Table \ref{tab Ireject trad nontrad perm}}). In fact, they are not very different in all the variables considered. %This is consistent with the conjecture that, had the \textsf{traditional} arm group rejecters been offered any of the non-\textsf{traditional} arms, they, as a group, may have accepted it.  %It shows the latter is more exposed to flood in baseline and has larger livestock values. This implies that, once large enough sum of loan is disbursed, %there is no minimum livestock and asset holding level to partake in the larger loans, and 
	%despite a negative asset shock in flood and a poverty trap at this level may be overcome once household size and negative asset shocks are accounted for.
	The common factors associated with nonparticipation are a smaller household size and smaller livestock holding (\textsc{\normalsize Table \ref{tab Ireject perm}}, \textsc{\normalsize Table \ref{tab Ireject trad perm}}), although, in non-\textsf{traditional} arms, the individual rejecters have only marginally different mean values relative to individual nonrejecters (\textsc{\normalsize Table \ref{tab Ireject nontrad perm}}). 
	
	\textcolor{red}{[I added below but deleted by Takahashi-san, any reason why we should?]
	\begin{itemize}
	\vspace{1.0ex}\setlength{\itemsep}{1.0ex}\setlength{\baselineskip}{12pt}
	\item	These hint that it may take a larger household size to raise cattle, and the households who have more livestock may have the capacity to raise more. To interpret this, it is possible that smaller households may be facing a domestic labour constraint in raising cattle. There is yet another possibility that a smaller household size reflects the space limitation to accommodate cattle under the roof. These constraints are expected to be absent in asset transfer programs where targeted residents can sell the asset if the constraint binds. In either case, it is a binding domestic physical capacity constraint that withholds participation.	 We conjecture that the households under a binding liquidity constraint and/or a binding domestic capacity constraint did not meet the conditions to raise cattle, thereby have withheld themselves from the program. This self selection may have caused the repayment rates to be higher than when everyone participated. 
	\end{itemize}
	}
	For the non-\textsf{traditional} arm members, baseline flood exposure is strongly correlated with individual rejection. This suggests that a population prone to natural calamity and associated asset shocks may voluntarily opt out the borrowing, which explains the lack of commercial and even noncommercial/NGO lenders in the flood prone area. 

	It is worth noting that partcipants in in-kind arm differ from other arms in having less cattle rearing experience seen in initial cattle holding, in having higher flood exposure rate, and in having lower asset values. These may be seen as disadvantages in rearing a heifer, and in-kind arm induced partcipation despite them.
	
	%	Group level rejection to participate is negatively correlated with literacy of household head (\textsc{\normalsize Table \ref{tab Greject perm MainText}}). Acknowledging the reasons for rejection can be different for individuals, we also tested the independence of each characteristics for individual rejecters (vs. non-individual rejecters) in \textsc{\normalsize Table \ref{tab Ireject perm MainText}}. One sees that smaller \textsf{HHsize}, being affected with \textsf{FloodInRd1}, and smaller \textsf{LivestockValues} and \textsf{NumCows} are associated with individual rejecters. We conjecture that individual decisions not to participate may be understood as: Smaller household size leaves a smaller capacity for cattle production labour in a household, and being hit with a flood may have resulted in lower livestock levels that would prompt them to reconsider partaking in another livestock project. 
%
%	A closer look at the nonparticipation correlates among \textsf{traditional} arm members in \textsc{\normalsize Table \ref{tab reject trad perm MainText}} and non-\textsf{traditional} arm members in \textsc{\normalsize Table \ref{tab reject nontrad perm MainText}} reveals possible differences in the causes. Rejection among \textsf{traditional} members tend to be associated with lower livestock holding but not with higher flood exposure nor smaller household size, while rejecters among non-\textsf{traditional} members are more likely to have suffered from flood at the baseline and have smaller household size. \textsc{\normalsize Table \ref{tab reject trad nontrad perm MainText}} shows rejecters of \textsf{traditional} have less flood exposure, smaller livestock and cattle holding, but not necessarily poorer as indicated by head literacy and asset holding than non-\textsf{traditional} counterpart. Given \textsf{traditional} rejecters at the mean have smaller livestock while household size is similar, it hints some capacity to supply labour for cattle production if an opportunity arises.

%	Since the offered arms were randomised, individual rejecters of \textsf{traditional} arm, who are similar in characteristcs to individual rejecters of non-\textsf{traditional} arm whose impediments are baseline flood and small household size, may have accepted the offer had their household size is larger and had they been offered non-\textsf{traditional} lending. Henceforth, we conjecture that flood exposure and household size are the potential impediments for uptake in larger size loans. 


	%Even we are targeting the ultra poor and designed the loan products to help them rise above the poverty trap, we still find lacking minimum level of assets, despite at a very low level, had kept the ultra poor from participating in microfinance. In the results of lending we consider in the below, the bottom class of the ultra poor had not lept its benefits.

\begin{table}
\hfil\begin{minipage}[t]{14cm}
\hfil\textsc{\normalsize Table \refstepcounter{table}\thetable: Permutation test results of attrition\label{tab attrit perm MainText}}\\
\setlength{\tabcolsep}{.5pt}
\setlength{\baselineskip}{8pt}
\renewcommand{\arraystretch}{.50}
\hfil\begin{tikzpicture}
\node (tbl) {\input{c:/data/GUK/analysis/save/EstimationMemo/AttritedPermutationTestResultso800.tex}};
%\input{c:/dropbox/data/ramadan/save/tablecolortemplate.tex}
\end{tikzpicture}\\
\begin{tabular}{>{\hfill\scriptsize}p{1cm}<{}>{\hfill\scriptsize}p{.25cm}<{}>{\scriptsize}p{12cm}<{\hfill}}
Source:& \multicolumn{2}{l}{\scriptsize Estimated with GUK administrative and survey data.}\\
Notes: & 1. & \textsf{R}'s package \textsf{coin} is used for baseline mean covariates to conduct approximate permutation tests. Number of repetition is set to 100000. Step-down method is used to adjust for multiple testing of a multi-factor grouping variable. \textsf{Attrited} and \textsf{Nonattrited} columns show means of each group. For \textsf{Arm}, proportions of non-traditional arm are given. \\
& 2. & ${}^{***}$, ${}^{**}$, ${}^{*}$ indicate statistical significance at 1\%, 5\%, 10\%, respetively. Standard errors are clustered at group (village) level.
\end{tabular}
\end{minipage}

\hfil\begin{minipage}[t]{14cm}
\hfil\textsc{\normalsize Table \refstepcounter{table}\thetable: Permutation test results of attriters between traditional and non-traditional arms\label{tab attrit TNT perm MainText}}\\
\setlength{\tabcolsep}{.5pt}
\setlength{\baselineskip}{8pt}
\renewcommand{\arraystretch}{.50}
\hfil\begin{tikzpicture}
\node (tbl) {\input{c:/data/GUK/analysis/save/EstimationMemo/TradNonTradAttritedPermutationTestResultso800.tex}};
%\input{c:/dropbox/data/ramadan/save/tablecolortemplate.tex}
\end{tikzpicture}\\
\begin{tabular}{>{\hfill\scriptsize}p{1cm}<{}>{\hfill\scriptsize}p{.25cm}<{}>{\scriptsize}p{12cm}<{\hfill}}
Source:& \multicolumn{2}{l}{\scriptsize Estimated with GUK administrative and survey data.}\\
Notes: & 1. & \textsf{R}'s package \textsf{coin} is used for baseline mean covariates to conduct approximate permutation tests. Number of repetition is set to 100000. Step-down method is used to adjust for multiple testing of a multi-factor grouping variable. \textsf{NonTradArm} and \textsf{TradArm} columns show means of each group. Attrition due to flood is dropped. \\
& 2. & ${}^{***}$, ${}^{**}$, ${}^{*}$ indicate statistical significance at 1\%, 5\%, 10\%, respetively. Standard errors are clustered at group (village) level.
\end{tabular}
\end{minipage}
\end{table}

	The survey resulted in a moderate rate of attrition. We checked for systematic differences between attriters and nonattriters in \textsc{\normalsize Table \ref{tab attrit perm MainText}} (see more detailed attrition examination in Appendix \ref{AppSecAttritionRejection}). The attrition is not correlated with a household level characteristics. As attrition rates differ between \textsf{traditional} and non-\textsf{traditional} arms, we compare them in \textsc{\normalsize Table \ref{tab attrit TNT perm MainText}}. It shows that \textsf{traditional} arm attriters have a lower rate of head literacy while non-\textsf{traditional} arm attriters are more exposed to the flood. The \textsf{traditional} arm attriters may be less entrepreneurial, if anything, so their attrition can upwardly bias the positive gains of the arm, hence understate the relative impacts of non-\textsf{traditional} arm. Attriters of non-\textsf{traditional} arms have similar literacy as non-attriters but have more exposure to flood. Attrited members of non-\textsf{traditional} arms do not show indication of being different in terms of productivity, thus is expected not to cause a bias in a predictable way. Overall, attrition may have attenuated the impacts but is not likely to inflate them.\footnote{So one can employ the Lee bounds for stronger results, but doing so will give us less precision and require more assumptions. We will not use the Lee bounds \textcolor{red}{[we can show them if necessary]}. }


\subsection{Impacts}

\begin{figure}
\mpage{12cm}{
\hfil\textsc{\footnotesize Figure \refstepcounter{figure}\thefigure: Effects on land, livestock, and net assets\label{fig LivestockEffects}}\\

\vspace{2ex}
\hspace{-2em}\includegraphics[height = 12cm, width = 14cm]{c:/data/GUK/analysis/program/figure/EstimationMemo/AssetRelativeToConcurrentTradEffects.eps}\\
\renewcommand{\arraystretch}{1}
\setlength{\tabcolsep}{1pt}
\hfil\begin{tabular}{>{\hfill\scriptsize}p{1cm}<{}>{\scriptsize}p{12.5cm}<{\hfill}}
Source: & Constructed from ANCOVA estimation results.\\
Note:& Left most column panel shows the conditional means of \textsf{traditional} arm which serves as a benchmark in estimating impacts. In other column panels, all points show the relative difference from concurrent \textsf{traditional} levels depicted in the left most column. \textsf{Large} and \textsf{Upfront} are the same values. Other column panels are grouped either by arm or by attribute. Row panels show different outcomes. Bars show 95\% confidence intervals using cluster robust standard errors.\\[1ex]
\end{tabular}
}
\end{figure}

	\textsc{\footnotesize Figure \ref{fig LivestockEffects}} summarises the main impact estimation results in time-varying specification of \eqref{EstimatingEqTimeVarying}.  See Appendix \ref{AppendixEstimationTables} for full estimation results. There are three stock outcome variables, land holding values, number of cattle, and net asset values, where net assets are defined as total assets less debt outstanding. For each outcome, there are six panels. The left most column panel shows stock evolution for the \textsf{traditional} arm. The \textsf{traditional} panels are intended to indicate the underlying trend conditional on covariates and baseline outcomes. In all other panel columns show the deviation from concurrent \textsf{traditional} arm values. The non-\textsf{traditional} panels give the time-varying impact estimates (relative to \textsf{traditional} arm) of each attribute and their 95\% confidence intervals. In each period, there are several estimation specifications which are bunched side-by-side. \textcolor{red}{[Following may be in footnote] Specification 1, which is omitted from the plot, is an OLS regression without the baseline outcome provides a reference for ANCOVA estimates.} This is intended to show robustness to specification changes at a glance.\footnote{As multiple tests are conducted to show uniformity across specifications, not to pick one specific estimate, inference corrections for multiple testing are unncessary. } One sees that there is little variation across specifications. Cattle and net assets have more regression specifications due to their possible dependence on previous cattle ownership and its inclusion as a covariate.
	
	There are notable tendencies in the figure. First, in cattle holding and net asset panels, there is a one time increase at period 2 in all non-\textsf{traditional} arms while the conditional mean values are relatively unchanged for the \textsf{traditional} arm. This means that the non-\textsf{traditional} arms have increased cattle holding once and stayed increased relative to the \textsf{traditional} arm. Estimates for cattle holding of \textsf{traditional} arm remain relatively unchanged in all periods, so a one time increase implies a gap in cattle holding was created in period 2 and the gap stayed unchanged. 

	Secondly, it is the \textsf{Upfront} attribute that shows positive impacts in all outcomes. Estimates for net assets of \textsf{traditional} arm show an upward trend. On top of this underlying trend, all non-\textsf{traditional} arms show a one-time increase, or a gap relative to \textsf{traditional} due to the \textsf{Upfront} aspect of lending. Results of land holding is similar to net assets, as it is a part of net assets, but the gap widens as period progresses. This is seen in the point estimates of non-\textsf{traditional} arms that are positive, yet most of estimates are imprecise and have their 95\% confidence intervals crossing zero. This is consistent with the nonconvex production technology for cattle under a liquidity constraint coupled with an inferior, goat production technology.

	Third, comparing the impacts of the \textsf{InKind} attribute on all stock outcomes against \textsf{Upfront} and \textsf{WithGrace}, we see statistically zero differences. In light of the fact that individuals with less cattle rearing experiences, lower asset values, and higher rate of flood exposure participated in the in-kind arm, the finding that their outcomes are statistically indistinguishable from other arms implies the treatment arm facilitated the average returns to cattle rearing. This implies that the entrepreneurial skills increase the participation to microfinance. This is consisitent with the finding by Banerjee et al that only the experienced or skilled members could reap the benefits. It can be due either to the managerial support program that complimented the necessary codifiable knowledge, or these participants had the same level of knowledge as other participants but found the utility of the managerial support program at their participation decisions. Previous studies targeted the population with a richer set of investment possibilities in a more urbanised setting under which the experience may have a positive return. In the current study, the population resides in a remote area. Even the simpler production process of dairy cattle farming that consists of feeding, grazing, insemination and calving turn out to demand some codifiable skills, or the crystalised intelligence, for participation in microfinance.
	
	Among all three assets, land holding may be most reliable indicator of wealth for fewer missingness. Net assets are defined as total assets less debt outstanding, yet we have smaller coverage of asset items in the first period which inflates the increasing trend.\footnote{This change in coverage is common to all arms, and given randomisation, this should not affect identification of imapcts by ANCOVA estimator as it is captured in the estimates of \textsf{traditional} arm, although it adds an extra noise. } Cattle shows the number of cattle owned and it also serves as a check that non-\textsf{traditional} members actually own cattle once the loan/lease is made. \textcolor{red}{[Move this to footnote: The ANCOVA estimates plotted in the figure are net of baseline cattle holding, so even the non-traditional holding estimates sometimes add up to less than 1.]} As shown in column (1) of \textsc{\normalsize Table \ref{tab ANCOVA cow time varying}}, \textsc{\normalsize Table \ref{tab ANCOVA cow time varying attributes}}, in \textsf{traditional} arm, about 79\% of members own cattle in period 2. This indicates that even a small loan helped some borrower to increase cattle ownership. On the other hand, non-\textsf{traditional} arms show a larger increase in cattle ownership. 

\begin{figure}
\mpage{\linewidth}{
\hfil\textsc{\footnotesize Figure \refstepcounter{figure}\thefigure: All IGA choices\label{fig AllIGAChoices}}\\

\vspace{1ex}
\hfil\includegraphics[height = 4cm, width = 12cm]{c:/data/GUK/analysis/program/figure/ImpactEstimationOriginal1600Memo3/AllIGAChoices.eps}\\
\renewcommand{\arraystretch}{1}
\hfil\begin{tabular}{>{\hfill\scriptsize}p{1cm}<{}>{\scriptsize}p{12cm}<{\hfill}}
Source: & Administrative data.\\
Note:& Based on information reported at the weekly meeting. Row panels indicate the total number of IGAs that borrowers own. For example, the row panel under the number `1' indicates the distribution of projects owned by single project members. There is no borrower with only one project in the \textsf{traditional} arm. 
\end{tabular}
}
\end{figure}

\begin{figure}
\mpage{\linewidth}{
\hfil\textsc{\footnotesize Figure \refstepcounter{figure}\thefigure: All IGA choices\label{fig AllIGAChoicesCollapsed}}\\

\vspace{1ex}
\hfil\includegraphics[height = 4cm, width = 12cm]{c:/data/GUK/analysis/program/figure/ImpactEstimationOriginal1600Memo3/AllIGAChoicesCollapsed.eps}\\
\renewcommand{\arraystretch}{1}
\hfil\begin{tabular}{>{\hfill\scriptsize}p{1cm}<{}>{\scriptsize}p{12cm}<{\hfill}}
Source: & Administrative data.\\
Note:& Based on information reported at the weekly meeting. The figure shows the sum collapsed over the total number of projects in each arms of \textsc{\footnotesize Figure \ref{fig AllIGAChoices}}. 
\end{tabular}
}
\end{figure}

	To understand the reasons behind a slower pace of asset accumulation of \textsf{traditional} arm, we plot borrower's reported income generating activities (IGAs)in \textsc{\small Figure \ref{fig AllIGAChoices}} shown by the total number of reported projects that the borrowers have. Row panel under the number `1' indicates the distribution of projects owned by single project members, and so on. This shows that almost no one of the \textsf{traditional} arm invested only in one project while only few members did so with the \textsf{Upfront} attribute. Goat/sheep and small trades are the top choices for the first income generating activities in \textsf{traditional}. This is consistent with convexity in the production technology of large domestic animals under a liquidity constraint. This also validates our supposition in experimental design that cattle production is the most preferred and probably the only economically viable investment choice. It eases a concern that the \textsf{cow} arm may have imposed an unnecessary restriction in an investment choice by forcing to receive cattle. \textsc{\small Figure \ref{fig AllIGAChoicesCollapsed}} collapses the reported projects over borrowers and shows the total number of IGAs in each arms. There are a significant number of cases in the \textsf{traditional} arm that members reportedly raise cows, yet they are also accompanied by pararell projects in smaller livestock production and small trades. 

\begin{figure}
\mpage{12cm}{
\hfil\textsc{\footnotesize Figure \refstepcounter{figure}\thefigure: Effects on income and consumption\label{fig IncomeConsumptionEffects}}\\

\vspace{2ex}
\hspace{-2em}\includegraphics[height = 12cm, width = 14cm]{c:/data/GUK/analysis/program/figure/EstimationMemo/IncomeConsumptionRelativeToConcurrentTradEffects.eps}\\
\renewcommand{\arraystretch}{1}
\setlength{\tabcolsep}{1pt}
\hfil\begin{tabular}{>{\hfill\scriptsize}p{1cm}<{}>{\scriptsize}p{12.5cm}<{\hfill}}
Source: & Constructed from ANCOVA estimation results.\\
Note:& Left most column panel shows the conditional means of \textsf{traditional} arm which serves as a benchmark in estimating impacts. In other column panels, all points show the relative difference from concurrent \textsf{traditional} levels depicted in the left most column. \textsf{Large} and \textsf{Upfront} are the same values. Other column panels are grouped either by arm or by attribute. Row panels show different outcomes. Bars show 95\% confidence intervals using cluster robust standard errors.\\[1ex]
\end{tabular}
}
\end{figure}

	\textsc{\small Figure \ref{fig IncomeConsumptionEffects}} shows impacts on consumption and labour incomes. Style and placement of panels follow the \textsc{\footnotesize Figure \ref{fig LivestockEffects}}. Consumption is not measured at the baseline, so we do not use it to understand the welfare impacts but to understand how the members have dealt with the loan repayment through consumption choices. Given randomisation, one can still identify impacts on repayment efforts in terms of consumption suppression relative to the \textsf{traditional} arm. In obtaining ANCOVA estimates, we condition on period 2 consumption. \textcolor{red}{[This can be problematic as period 2 consumption is correlated with arm assignment. But the results do not change if we estimate without period 2 consumption as a covariate in specification 1.]} Consumption is per capita consumption of the household. Labour incomes is a household level variable and measures earnings from day-to-day casual jobs. 

	In consumption estimation, the estimates of \textsf{traditional} in specification 2 and 3 differ significantly as the latter involves baseline household size. The impacts of non-\textsf{traditional} arms are almost zero in all panels. This is in contrast to assets where we saw an increase in cattle, land holding and net assets. Borrowing members seem to have put asset accumulation a priority before consumption. Labour incomes are volatile across periods in \textsf{traditional} arm. Just like the consumption, we see no impact in all non-\textsf{traditional} arms. One also notes that labour income is highest in period 2 and is rising from period 3 onwards. Former is due to the flood in period 2 when members were trying to make up for the losses with an increased labour supply. The latter rising trend is consistent with the repayment burden, and is further consistent with the view that the borrowers did not choose to strategically default but tried to repay. 
	

\begin{figure}
\hfil\textsc{\footnotesize Figure \refstepcounter{figure}\thefigure: Cumulative weekly repayment rates\label{fig weeklysavingrepayrate}}\\
\hfil\includegraphics{c:/data/GUK/analysis/program/figure/ImpactEstimationOriginal1600Memo2/CumulativeWeeklyRepaymentRateByPovertystatus.png}\\
\renewcommand{\arraystretch}{1}
\hfil\begin{tabular}{>{\hfill\scriptsize}p{1cm}<{}>{\scriptsize}p{12cm}<{\hfill}}
Note:& Each dot represents weekly observations. Only members who received loans are shown. Each panel shows ratio of cumulative repayment sum to cumulative due amount sum, ratio of sum of cumulative repayment and cumulative net saving (saving - withdrawal) sum to cumulative due amount sum, both are plotted against weeks after first disbursement. Value of 1 indicates the member is at per with repayment schedule. Horizontal lines has a $Y$ intercept at 1. Lines are smoothed lines with a penalized cubic regression spline in \textsf{ggplot2::geom\_smooth} function, originally from \textsf{mgcv::gam} with \textsf{bs=`cs'}. \\[-1ex]
\end{tabular}
\end{figure}


	\textsc{\footnotesize Figure \ref{fig weeklysavingrepayrate}} shows the repayment results. Top panel shows the ratios of cumulative repayment to cumulative planned installment, the bottom panel shows the ratios of sum of cumulative repayment and cumulative net saving (saving - withdrawal) to cumulative planned installment. Both are plotted against weeks after first disbursement. Each dot represents a member at each time point. Value of 1, which is given by a horizontal line, indicates the member is at per with repayment schedule. Some members saved more than the required repayment at each time points that go beyond 1 in the figure. One sees that repayment rates are above 1 at the beginning but stay below 1 for most of the time. The majority of borrowing members did not repay the loan by the 48th month with prespecified installments. One notes the \textsf{traditional} arm has more of lower repayment rates among all arms. When a member does not reach the due amount with installments, they had to repay from the (net) saving, an arrangement to which the lender and the borrowers agreed at the loan contract signment. Repayment rates after using net saving are 44.71, 93.57, 97.01, 95.42\%, respectively, for \textsf{traditional, large, large grace, cow} arms and 87.85\% for overall (from \textsf{\footnotesize AllMeetingsRepaymentInitialSample.rds}). \textcolor{red}{[Abu-san: Why does the admin data continue up to the 48th month, not 36th?]}

	There is little difference in repayment rates by poverty classes. \textsc{\footnotesize Figure \ref{fig weeklysavingrepayrate}} depicts both moderately poor and ultra poor in different colours. It is impossible to distinguish between them with eyeballs, and ANCOVA estimates also confirm this (see Appendix \ref{AppendixShortfall} for details). This is in contrast to a popular belief that the ultra poor are the riskiest among all income classes. Poverty gradation through a participatory process, however, does not distinguish the moderately poor and the ultra poor on the observables. \textsc{\footnotesize Figure \ref{fig NetAssetValuesAtRd1}} shows net asset values at baseline by poverty class, and \textsc{\footnotesize Figure \ref{fig LivestockValuesAtRd1}} shows initial livestock values at baseline by poverty class. Both show little difference in these observable characteristics. \textcolor{red}{[According to Abu-san, participatory poverty gradation may have been imprecise.]}

	Smaller cumulative impacts and lower repayment rates of \textsf{traditional} arm members stand out once we acknowledge that they are receiving an equivalent amount and their contract differs with other arms only in the attributes we focus. These differences arise partly from the difference in investment choices observed in \textsc{\small Figure \ref{fig AllIGAChoices}, \ref{fig AllIGAChoicesCollapsed}} that were induced by availability of \textsf{Upfront} attribute in lending.
	
% \mpage{12cm}{
% \hfil\textsc{\footnotesize Figure \refstepcounter{figure}\thefigure: Repayments\label{fig Repayments}}\\
% 
% \vspace{2ex}
% \hspace{-2em}\includegraphics[height = 10cm, width = 12cm]{       paste0(pathprogram, "figure/ImpactEstimationOriginal1600Memo3/Repayments.eps") }\\
% \renewcommand{\arraystretch}{1}
% \hfil\begin{tabular}{>{\hfill\scriptsize}p{1cm}<{}>{\scriptsize}p{11cm}<{\hfill}}
% Source: & Constructed from FD estimation results.\\
% Note:& CumImpactText2 \\[1ex]
% \end{tabular}
% }
%
%	Annualised repayment is depicted in \hfil\textsc{\footnotesize Figure \ref{fig Repayments}}. The top three panels show net saving. As in \textsc{\footnotesize Figure \ref{fig LivestockCumulativeEffects}}, each subpanel shows cumulative changes, per period changes, and differences in changes relative to concurrent changes of \textsf{traditional} arm. 
%
% \textsf{InKind} attribute does not increase repayment by itself. Repayment is highest with the \textsf{Upfront} attribute. It is driven by the initial year repayment and subsequent repayment is smaller than \textsf{traditional}. With \textsf{WithGrace} attribute, repayment is larger by more than Tk. 15000 in total during period 2 and 4 due to the accumulated saving in period 1 or the grace period. For net saving, there is a steady increase in all arms. \textsf{Upfront} and \textsf{WithGrace} attributes see a large boost in period 1 and the growth relative to \textsf{traditional} becomes negative subsequently.



\begin{figure}
\hfil\mpage{12cm}{
\hfil\textsc{\footnotesize Figure \refstepcounter{figure}\thefigure: Effects on schooling\label{fig SchoolingEffects}}\\

\vspace{2ex}
\hfil\includegraphics[height = 12cm, width = 12cm]{c:/data/GUK/analysis/program/figure/EstimationMemo/SchoolingEffects.png}\\
\renewcommand{\arraystretch}{1}
\hfil\begin{tabular}{>{\hfill\scriptsize}p{1cm}<{}>{\scriptsize}p{11cm}<{\hfill}}
Source: & Constructed from ANCOVA estimation results.\\
Note:&  See footnotes of \textsc{Figure \ref{fig LivestockEffects}}. \\[1ex]
\end{tabular}
}
\end{figure}

	In \textsc{\footnotesize Figure \ref{fig SchoolingEffects}}, effects on child school enrollment are plotted. As in the previous figure, \textsf{traditional} column shows the conditional mean values and other non-\textsf{traditional} columns show impacts relative to concurrent \textsf{traditional} arm values. In general, there is no detectable impact of the intervention, except for a negative impact for women at the college level for \textsf{Upfront} in period 4 and a positive impact for women at the college level for \textsf{WithGrace} in period 4. Women at the college level are about 5.9\% of sample, so the effective sample size of each cell is about 1-3, and it is difficult to interpret the results on these small samples. If anything, negative impacts of elder girl's schooling may be due to stronger demand for cattle production in a household. This is in line with the finding in rejection that the limited household size can be a constraint on participation, especially when there is no grace period. Cattle ownership naturally shifts the relative prices in a household against child schooling, especially for the elder girls as their returns on human capital are considered to be lower than younger girls, and task contents of cattle labour are less brawn intensive yet requires to be above the primary school ages. This may be a downside of having more household production with cattle.



\section{Theory}
\label{SecTheory}

% \begin{mdframed}[style={SecItemize}, frametitle={Theory}]
% \begin{itemize}
% \vspace{1.0ex}\setlength{\itemsep}{1.0ex}\setlength{\baselineskip}{12pt}
% \item	Contour of two production functions, a nonconvex production set, gives rise to a poverty trap
% \item	Goats relative to cows as an investment: Infrequent income stream, limited local consumption, vulnerability to logging water, a herd is less mobile
% \item	Goat returns net of mortality are lower (not generally, only in this area) and one cannot scale up goats: Takes long to switch to cow ownership
% %\item	No saving constraint required, saving = depreciation at equilibria
% \item	The entire region depicted in the diagram represents poverty, so it shows a poverty trap within poverty (i.e., ultra poor and moderately poor)
% \item	We are not going to show the production nonconvexity, instead we show lower repayment rates and smaller cattle holding for a smaller loan size, just as \citet{BandieraBRAC2017} did
% \end{itemize}
% \end{mdframed}

\hspace{-3em}\mpage{\linewidth}{
\hfil\textsc{\normalsize Figure \refstepcounter{figure}\thefigure: A poverty trap with goats and cows\label{fig poverty trap}}\\

\noindent\mpage{\linewidth}{\input{GoatCowProdFunctions.tkz}\input{poverty_trapCompoundScurve.tkz}}}

	In this section, we use a simplified version of \citet{GalorZeira1993} to illustrate a theoretical framework to aid the interpretation of the empirical finding that asset accumulation is faster while the repayment rate is higher for upfront lending. Let us consider that there are two production sets called `goat' and 'cow.' Both sets are nonconvex with fixed inputs. We note from the previous section that returns to goat net of mortality are lower in this area, and one cannot scale up goats as it takes a long period to reproduce to the herd size that is large enough to switch to cow ownership. We also note that a goat investment relative to a heifer has an infrequent income stream, limited local consumption, vulnerability to logging water, all pointing to lower returns. We will use these points to assume that the fixed costs and steady state production level are smaller for goats than cows. 
	
	When there is only a goat production technology, individuals eventually reaches the point $G$, a steady state where the capital-labour ratio is constant, or $k_{t+1}=k_{t}$. When the cow production technology is added to the picture, there is no change in the equlibrium for individuals whose initial assets are in $[k_{1}, k_{2})$. For individuals with initial assets in $[k_{2}, \infty)$, one chooses a cow, because the resulting income level is higher, and eventually arrive at the steady state $C$. 

	For the economy as a whole, the production possibility frontier, or the contour of the union of two production sets, becomes M-shaped. Under the configuration depicted in the figure, there will be five equilibria of which three are stable. Ruling out the zero equilbrium as irrelevant, one is left with two stable equilibria, named as goats and cows in the figure. 
	
	Formally, one requires the production set $j=\{\mbox{goat, cow}\}$ to satisfy: there exists $\underline{k}_{j}>0$ that the production is zero for input $k<\underline{k}_{j}$ and strictly positive for $k\geqslant\underline{k}_{j}$. We assume the production set exhibits decreasing returns to scale for $k\geqslant\underline{k}_{j}$. Let the contour of the production set be $f(k)$. Assume for expositional simplcity that a fixed saving rate $s$ is such that the steady state saving $sf(k^{*})$ is net of capital depreciation. Further assume that there exists $k_{2}>\underline{k}_{j}$ such that $sf(k)>k$ for $k\in(k_{2}, k^{*})$, with $k^{*}>k_{2}$ is a fixed point $k^{*}=sf(k^{*})$. Under these assumptions, decreasing returns ensure there exists two intersections between the steady state line, one unstable and one stable equilibria.\footnote{In \textsc{Figure \ref{fig poverty trap}}, depreciation below $\underline{k}$ is not accounted as capital cannot be negative. Once the production starts for $k>\underline{k}$, the contour shows net of depreciation so $sf(k)-\delta k$. } 

	In light of this argument, a loan that is larger than $\underline{k}$ allows individuals in the goat equilibrium to transition to cow production and arrive at the cow equilibrium. If the lending market is competitive, the interest rate is the same as the return on capital and thus lending, not a transfer, suffices for the transition. The entire region depicted in the diagram is considered as in the realm of poverty, so it shows a poverty trap within poverty (i.e., goat as ultra poor and cow as moderately poor). 

	In the empirical section, we follow \citet{BandieraBRAC2017} and take the production nonconvexity as given and examine lower repayment rates and smaller cattle holding for a smaller loan size as evidence consistent with a poverty trap. 

\section{Conclusion}
\label{SecConclusion}

\begin{mdframed}[style={SecItemize}, frametitle={Conclusion}]
\begin{itemize}
\vspace{1.0ex}\setlength{\itemsep}{1.0ex}\setlength{\baselineskip}{12pt}
\item	No entrepreneurship is necessary for project success, due probably to a simpler production process.
\item	Upfront liquidity increases asset holding and repayment rates, not the loan size \textit{per se}.
\item	Cattle has higher returns and lower risks, resulting in higher repayment rates, but also has larger initial fixed costs, possibly generating a poverty trap.
\item	Lending uptake is impeded by small household size and asset shocks.
\item	If these are relaxed, a poverty trap may be overcome.
\item	In the remote rural setting, larger upfront loan suited to project needs is shown to be Pareto improving, despite widely believed fears of inefficiency due to information asymmetry.
\item	In the remote rural setting, slow pace of outreach may be explained by not sufficiently cracking the liquidity constraint.
\item	Consumption and labour incomes were not affected in non-\textsf{traditional} arms. Labour income increased toward the end of repayment which can be a repayment effort.
\item	Schooling was not affected in general. It finds a sign of a loss to college level women, hinting a domestic labour constraint in cattle production. But there was also a positive impact for women at the college level in \textsf{WithGrace} arm. While these are possibilities, cell sample sizes are too small to draw anything conclusive.
\end{itemize}
\end{mdframed}

	The poverty reduction impacts of microfinance was a firm belief in the early days of microfinance. Yet it suffered from a puzzling weak spot that microfinance is slow to reach the ultra poor, which is still debated today. Recently, even the poverty reduction impacts are subject to doubts, and it has been shown that the only borrowers with experience or skills are able to leap benefits. In this study, we examined the role of entrepreneurship in leaping benefits. We showed, under the rural setting, experiences or entrepreneurship seem to matter for participation and resulting impacts. note the usefulness of having consulting services available for the prospective clients of MFIs when expanding the credit to the ultra poor. 

	This study employs a stepped-wedge design of multiple arms to isolate different attributes of loan contract: Frontloading, a grace period, and in-kind lease with management supports. These map to a liquidity constraint, a saving constraint, and an entrepreneurship constraint. Only frontloading the disbursement matters in all outcomes, which signifies the importance of a liquidity constraint. With evidence that borrowers with frontloaded arms invested in cattle while the borrowers under incremental lending invested in multiple, smaller livestock, and the repayment rates are higher for the frontloaded arms, we conclude that there is a poverty trap which cannot be overcome by the traditional approach of microfinance. Under the study's setting, escaping from the poverty trap only requires frontloading the lending, not lending incrementally as practiced by the majority of microfinance institutions. In addition, lending rather than a transfer may suffice to support the transition. To expand the coverage to the ultra poor, it may be useful to have consulting services.
	
	We have witnessed that a binding domestic capacity constraint may impede potential borrowers from participation. This limits the potential benefit of lending a larger amount from the start of the program. While it in unclear why the outsourced labour cannot substitute the domestic labour, one can consider organising a day service run in each group, which can be tended by the group members by taking turns, to collectively graze the cattle during the daytime. This partly eases the domestic labour and/or space constraints faced by small households. 

	We have seen that borrowers accumulated assets, increased labour supplies, but not increasing the consumption. This is consistent with high morale of repayment, which can be explained by the lack of alternative lenders in the study area. With stronger incentives to repay, the evidence on stronger repayment discipline of large sized arm members need not generalise in the areas outside the study site. On the other hand, the necessity of codifiable knowledge in participation even for a simple production process and the scope for escaping the poverty trap with larger lending may be more generalisable to other rural areas that are suited to cow and goat production.



{\footnotesize\bibliographystyle{aer}
\setlength{\baselineskip}{8pt}
\bibliography{c:/docs/migrate/TeX/seiro}
}

\appendix
\setcounter{section}{0}
\setcounter{figure}{0}
\setcounter{table}{0}
\renewcommand{\thefigure}{\Alph{section}\arabic{figure}}
\renewcommand{\thetable}{\Alph{section}\arabic{table}}
\renewcommand{\thesection}{\Alph{section}}



\section{Data description}

\hspace{-1cm}\begin{minipage}[t]{14cm}
\hfil\textsc{\normalsize Table \refstepcounter{table}\thetable: Descriptive statistics by arm in administrative data\label{tab DestatByArm}}\\
\setlength{\tabcolsep}{1pt}
\setlength{\baselineskip}{8pt}
\renewcommand{\arraystretch}{.55}
\hfil\begin{tikzpicture}
\node (tbl) {\input{c:/data/GUK/analysis/save/EstimationMemo/DestatByArm.tex}};
\end{tikzpicture}\\
\renewcommand{\arraystretch}{.8}
\setlength{\tabcolsep}{1pt}
\begin{tabular}{>{\hfill\scriptsize}p{1cm}<{}>{\hfill\scriptsize}p{.25cm}<{}>{\scriptsize}p{12cm}<{\hfill}}
Source:& \multicolumn{2}{l}{\scriptsize Estimated with GUK administrative and survey data.}\\
Notes: & 1. & Information of original 800 households. Net saving as percentage of loan amount is a mean over loan recipients whose first disbursement is in 2013. Effective repayment is a sum of repayment and net saving. \\
& 2. & \textsf{Loan year} -1 is preparation period for loan disbursement when only saving is allowed. \\
\end{tabular}
\end{minipage}

\mpage{\linewidth}{
\hfil\textsc{\footnotesize Figure \refstepcounter{figure}\thefigure: Net asset values at baseline\label{fig NetAssetValuesAtRd1}}\\
\hfil\includegraphics[width = 10cm]{c:/data/GUK/analysis/program/figure/EstimationMemo/NetAssetsAtRd1.pdf}\\
\renewcommand{\arraystretch}{1}
\hfil\begin{tabular}{>{\hfill\scriptsize}p{1cm}<{}>{\scriptsize}p{12cm}<{\hfill}}
Source: & Survey data.\\
Note:& Net asset values = total gross asset values - debt outstanding. Debt outstanding takes the value of the month immediately after the respective survey round interview. \\[1ex]
\end{tabular}
}

\mpage{\linewidth}{
\hfil\textsc{\footnotesize Figure \refstepcounter{figure}\thefigure: Lvestock holding at baseline\label{fig LivestockValuesAtRd1}}\\
\hfil\includegraphics[width = 8cm, height = 5cm]{c:/data/GUK/analysis/program/figure/EstimationMemo/LivestockValuesAtRd1.eps}\\
\renewcommand{\arraystretch}{1}
\hfil\begin{tabular}{>{\hfill\scriptsize}p{1cm}<{}>{\scriptsize}p{12cm}<{\hfill}}
Source: & Survey data.\\
Note:& Livestock holding at baseline. Median market price is used to convert holding to values. \\[1ex]
\end{tabular}
}

\renewcommand{\arraystretch}{.6}
\mpage{\linewidth}{
\hfil\textsc{\footnotesize Table \refstepcounter{table}\thetable: Number of observations by borrower status and arm\label{tab NumObsByBStatusArmFile}}\\
\hfil\input{c:/data/GUK/analysis/program/table/EstimationMemo/NumObsByBStatusArmFile.tex}\\
\renewcommand{\arraystretch}{1}
\hfil\begin{tabular}{>{\hfill\scriptsize}p{1cm}<{}>{\scriptsize}p{12cm}<{\hfill}}
Source: & Survey data.\\
Note:&  \\[1ex]
\end{tabular}
}

\renewcommand{\arraystretch}{.6}
\mpage{\linewidth}{
\hfil\textsc{\footnotesize Table \refstepcounter{table}\thetable: Number of observations used in estimation by borrower status and arm at period 1\label{tab NumObsByBStatusArmRegUsedMin}}\\
\hfil\input{c:/data/GUK/analysis/program/table/EstimationMemo/NumObsByBStatusArmRegUsedMin.tex}\\
\renewcommand{\arraystretch}{1}
\hfil\begin{tabular}{>{\hfill\scriptsize}p{1cm}<{}>{\scriptsize}p{12cm}<{\hfill}}
Source: & Survey data.\\
Note:&  \\[1ex]
\end{tabular}
}

\renewcommand{\arraystretch}{.6}
\mpage{\linewidth}{
\hfil\textsc{\footnotesize Table \refstepcounter{table}\thetable: Number of observations used in estimation by borrower status and arm at last period\label{tab NumObsByBStatusArmRegUsedMax}}\\
\hfil\input{c:/data/GUK/analysis/program/table/EstimationMemo/NumObsByBStatusArmRegUsedMax.tex}\\
\renewcommand{\arraystretch}{1}
\hfil\begin{tabular}{>{\hfill\scriptsize}p{1cm}<{}>{\scriptsize}p{12cm}<{\hfill}}
Source: & Survey data.\\
Note:&  \\[1ex]
\end{tabular}
}

\section{Randomisation checks}
\label{AppSecRandomisation}
\setcounter{table}{0}

\hspace{-1.5cm}\begin{minipage}[t]{14cm}
\hfil\textsc{\normalsize Table \refstepcounter{table}\thetable: Permutation test results\label{tab perm}}\\
\setlength{\tabcolsep}{.5pt}
\setlength{\baselineskip}{8pt}
\renewcommand{\arraystretch}{.50}
\hfil\begin{tikzpicture}
\node (tbl) {\input{c:/data/GUK/analysis/save/EstimationMemo/PermutationTestResults.tex}};
\end{tikzpicture}\\
\renewcommand{\arraystretch}{.8}
\setlength{\tabcolsep}{1pt}
\begin{tabular}{>{\hfill\scriptsize}p{1cm}<{}>{\hfill\scriptsize}p{.25cm}<{}>{\scriptsize}p{12cm}<{\hfill}}
Source:& \multicolumn{2}{l}{\scriptsize Estimated with GUK administrative and survey data.}\\
Notes: & 1. & \textsf{R}'s package \textsf{coin} is used for baseline group mean covariates to conduct approximate permutation tests. Number of repetition is set to 100000. Number of groups is 72. Step-down method is used to adjust for multiple testing of a multi-factor grouping variable.\\
& 2. & ${}^{***}$, ${}^{**}$, ${}^{*}$ indicate statistical significance at 1\%, 5\%, 10\%, respetively. Standard errors are clustered at group (village) level.
\end{tabular}
\end{minipage}

\mpage{\linewidth}{
\renewcommand{\arraystretch}{.6}
\hfil\textsc{\footnotesize Table \refstepcounter{table}\thetable: Anova results for cattle holding equality by arm\label{table anova CattleHoldingArm}}\\
\hfil\input{c:/data/GUK/analysis/program/table/EstimationMemo/anovaCowResults.tex}\\
\renewcommand{\arraystretch}{1}
\hfil\begin{tabular}{>{\hfill\scriptsize}p{1cm}<{}>{\scriptsize}p{12cm}<{\hfill}}
Source: & Survey data.\\
Note:& Each column uses respective year cattle ownership information. For ANOVA and Kruskal-Wallis, each entry indicates $p$ values. ANOVA tests for the null of equality of means under normality. Kruskal-Wallis tests for the null of no stochastic dominance among samples without using the normality assumption. Tukey's honest significant tests show difference in means and $p$ values in parenthesis that account for multiple testing under normality. In column 2, we edited data by assigning 1 to members of \textsf{cow} arm at dates after disbursement if reported holding is NA or zero. \\[1ex]
\end{tabular}}


\section{Attrition and rejection}
\label{AppSecAttritionRejection}
\setcounter{table}{0}



Among 800 observations, there are 4 whose villages are washd away and 70 who by group rejected the assigned arms which are traditional, large, large grace with 40, 20, 10, 0 individuals, respectively. There are 31, 9, 13, 37 individuals who individually rejected traditional, large, large grace, cow, respectively. Among attrited HHs, when were they lost?
\begin{Schunk}
\begin{Soutput}

  1 
116 
\end{Soutput}
\end{Schunk}
Reasons for attrition and relation to flood damage.
\begin{Schunk}
\begin{Soutput}
          BStatus
FloodInRd1 borrower individual rejection group rejection rejection by flood
      0          26                    7               2                 23
      1          20                    7              13                 17
      <NA>        0                    1               0                  0
\end{Soutput}
\begin{Soutput}
              BStatus
AssignOriginal borrower individual rejection group rejection rejection by flood
   traditional       26                    6               0                  0
   large              7                    0               0                  0
   large grace        7                    2               0                  0
   cow                6                    7               0                  0
   <NA>               0                    0              15                 40
\end{Soutput}
\end{Schunk}
Use \textsf{coin} package's \textsf{independence\_test}: Approximate permutation tests by randomly resampling 100000 times.


\hspace{-1.5cm}\begin{minipage}[t]{14cm}
\hfil\textsc{\normalsize Table \refstepcounter{table}\thetable: Permutation test results of attrition\label{tab attrit perm}}\\
\setlength{\tabcolsep}{.5pt}
\setlength{\baselineskip}{8pt}
\renewcommand{\arraystretch}{.50}
\hfil\begin{tikzpicture}
\node (tbl) {\input{c:/data/GUK/analysis/save/EstimationMemo/AttritedPermutationTestResultso800.tex}};
%\input{c:/dropbox/data/ramadan/save/tablecolortemplate.tex}
\end{tikzpicture}\\
\begin{tabular}{>{\hfill\scriptsize}p{1cm}<{}>{\hfill\scriptsize}p{.25cm}<{}>{\scriptsize}p{12cm}<{\hfill}}
Source:& \multicolumn{2}{l}{\scriptsize Estimated with GUK administrative and survey data.}\\
Notes: & 1. & \textsf{R}'s package \textsf{coin} is used for baseline mean covariates to conduct approximate permutation tests. Number of repetition is set to 100000. Step-down method is used to adjust for multiple testing of a multi-factor grouping variable. \textsf{Attrited} and \textsf{Nonattrited} columns show means of each group. For \textsf{Arm}, proportions of non-traditional arm are given. \\
& 2. & ${}^{***}$, ${}^{**}$, ${}^{*}$ indicate statistical significance at 1\%, 5\%, 10\%, respetively. Standard errors are clustered at group (village) level.
\end{tabular}
\end{minipage}

\hspace{-1.5cm}\begin{minipage}[t]{14cm}
\hfil\textsc{\normalsize Table \refstepcounter{table}\thetable: Permutation test results of attrition among traditional arm\label{tab attrit Trad perm}}\\
\setlength{\tabcolsep}{.5pt}
\setlength{\baselineskip}{8pt}
\renewcommand{\arraystretch}{.50}
\hfil\begin{tikzpicture}
\node (tbl) {\input{c:/data/GUK/analysis/save/EstimationMemo/AttritedInTradPermutationTestResultso800.tex}};
%\input{c:/dropbox/data/ramadan/save/tablecolortemplate.tex}
\end{tikzpicture}\\
\begin{tabular}{>{\hfill\scriptsize}p{1cm}<{}>{\hfill\scriptsize}p{.25cm}<{}>{\scriptsize}p{12cm}<{\hfill}}
Source:& \multicolumn{2}{l}{\scriptsize Estimated with GUK administrative and survey data.}\\
Notes: & 1. & \textsf{R}'s package \textsf{coin} is used for baseline mean covariates to conduct approximate permutation tests. Number of repetition is set to 100000. Step-down method is used to adjust for multiple testing of a multi-factor grouping variable. \textsf{Attrited} and \textsf{Nonattrited} columns show means of each group. For \textsf{Arm}, proportions of non-traditional arm are given. \\
& 2. & ${}^{***}$, ${}^{**}$, ${}^{*}$ indicate statistical significance at 1\%, 5\%, 10\%, respetively. Standard errors are clustered at group (village) level.
\end{tabular}
\end{minipage}

\hspace{-1.5cm}\begin{minipage}[t]{14cm}
\hfil\textsc{\normalsize Table \refstepcounter{table}\thetable: Permutation test results of attriters between traditional and non-traditional arms\label{tab attrit TNT perm}}\\
\setlength{\tabcolsep}{.5pt}
\setlength{\baselineskip}{8pt}
\renewcommand{\arraystretch}{.50}
\hfil\begin{tikzpicture}
\node (tbl) {\input{c:/data/GUK/analysis/save/EstimationMemo/TradNonTradAttritedPermutationTestResultso800.tex}};
%\input{c:/dropbox/data/ramadan/save/tablecolortemplate.tex}
\end{tikzpicture}\\
\begin{tabular}{>{\hfill\scriptsize}p{1cm}<{}>{\hfill\scriptsize}p{.25cm}<{}>{\scriptsize}p{12cm}<{\hfill}}
Source:& \multicolumn{2}{l}{\scriptsize Estimated with GUK administrative and survey data.}\\
Notes: & 1. & \textsf{R}'s package \textsf{coin} is used for baseline mean covariates to conduct approximate permutation tests. Number of repetition is set to 100000. Step-down method is used to adjust for multiple testing of a multi-factor grouping variable. \textsf{NonTradArm} and \textsf{TradArm} columns show means of each group. Attrition due to flood is dropped. \\
& 2. & ${}^{***}$, ${}^{**}$, ${}^{*}$ indicate statistical significance at 1\%, 5\%, 10\%, respetively. Standard errors are clustered at group (village) level.
\end{tabular}
\end{minipage}


\hspace{-1.5cm}\begin{minipage}[t]{14cm}
\hfil\textsc{\normalsize Table \refstepcounter{table}\thetable: Permutation test results of rejection\label{tab reject perm}}\\
\setlength{\tabcolsep}{.5pt}
\setlength{\baselineskip}{8pt}
\renewcommand{\arraystretch}{.50}
\hfil\begin{tikzpicture}
\node (tbl) {\input{c:/data/GUK/analysis/save/EstimationMemo/RejectedPermutationTestResultso800.tex}};
%\input{c:/dropbox/data/ramadan/save/tablecolortemplate.tex}
\end{tikzpicture}\\

\hfil\textsc{\normalsize Table \refstepcounter{table}\thetable: Permutation test results of rejection among traditional arm\label{tab reject trad perm}}\\
\setlength{\tabcolsep}{.5pt}
\setlength{\baselineskip}{8pt}
\renewcommand{\arraystretch}{.50}
\hfil\begin{tikzpicture}
\node (tbl) {\input{c:/data/GUK/analysis/save/EstimationMemo/RejectedInTradPermutationTestResultso800.tex}};
%\input{c:/dropbox/data/ramadan/save/tablecolortemplate.tex}
\end{tikzpicture}\\

\hfil\textsc{\normalsize Table \refstepcounter{table}\thetable: Permutation test results of rejection among non-traditional arm\label{tab reject nontrad perm}}\\
\setlength{\tabcolsep}{.5pt}
\setlength{\baselineskip}{8pt}
\renewcommand{\arraystretch}{.50}
\hfil\begin{tikzpicture}
\node (tbl) {\input{c:/data/GUK/analysis/save/EstimationMemo/RejectedInNonTradPermutationTestResultso800.tex}};
%\input{c:/dropbox/data/ramadan/save/tablecolortemplate.tex}
\end{tikzpicture}\\

\hfil\textsc{\normalsize Table \refstepcounter{table}\thetable: Permutation test results of rejecters, traditional vs. non-traditional arm\label{tab reject trad nontrad perm}}\\
\setlength{\tabcolsep}{.5pt}
\setlength{\baselineskip}{8pt}
\renewcommand{\arraystretch}{.50}
\hfil\begin{tikzpicture}
\node (tbl) {\input{c:/data/GUK/analysis/save/EstimationMemo/TradNonTradRejectedPermutationTestResultso800.tex}};
%\input{c:/dropbox/data/ramadan/save/tablecolortemplate.tex}
\end{tikzpicture}\\
\begin{tabular}{>{\hfill\scriptsize}p{1cm}<{}>{\hfill\scriptsize}p{.25cm}<{}>{\scriptsize}p{12cm}<{\hfill}}
Source:& \multicolumn{2}{l}{\scriptsize Estimated with GUK administrative and survey data.}\\
Notes: & 1. & \textsf{R}'s package \textsf{coin} is used for baseline group mean covariates to conduct approximate permutation tests. Number of repetition is set to 100000. Step-down method is used to adjust for multiple testing of a multi-factor grouping variable. Rejection is either group-rejection or individual-rejection. \textsf{TradArm} and \textsf{NonTradArm} columns show means of each group. \\
& 2. & ${}^{***}$, ${}^{**}$, ${}^{*}$ indicate statistical significance at 1\%, 5\%, 10\%, respetively. Standard errors are clustered at group (village) level.
\end{tabular}
\end{minipage}

	\textsc{\normalsize Table \ref{tab attrit perm}} shows results from tests of independence between attriters and non-attriters. We see a moderate rate of attrition is not correlated with household level characteristics. \textsc{\normalsize Table \ref{tab attrit Trad perm}} shows attrition in the \textsf{traditional} arm. Household heads of attriters are relatively less literate than non-attriters. \textsc{\normalsize Table \ref{tab attrit TNT perm}} compares attriters of \textsf{traditional} arm and non-\textsf{traditional} arms. It shows that \textsf{traditional} arm attriters have a (marginally) lower rate of head literacy while non-\textsf{traditional} arm attriters are more exposed to the flood. The \textsf{traditional} arm attriters may be less entrepreneurial, if anything, so their attrition may upwardly bias the positive gains of the arm, hence understate the impacts of non-\textsf{traditional} arm. So one can employ Lee bounds for stronger results, but doing so will give us less precision and require more assumptions.

	\textsc{\normalsize Table \ref{tab reject perm}} shows test results of independence between loan receivers and nonreceivers (group, individual rejecters) on 760 members whose residence was not washed away by flood. It shows that smaller household size, being affected by flood at the baseline, smaller livestock holding, smaller net assets, and less exposue to cattle growing are correlated with opting out the offered type of lending. 
	
	Group rejecters and non-group rejecters are compared in \textsc{\normalsize Table \ref{tab Greject perm}}. Marked differences are found in arm (\textsf{traditional} vs. non-\textsf{traditional}) and net asset values. \textsc{\normalsize Table \ref{tab Greject trad perm}} compares group rejecters in \textsf{traditional} arm and finds lower livestock values, smaller cattle holding, smaller net asset values, and smaller flood exposure are associated with group rejection for \textsf{traditional} arm (\textsc{\normalsize Table \ref{tab Greject trad perm}}). Group rejecters in non-\textsf{traditional} arm are examined in \textsc{\normalsize Table \ref{tab Greject nontrad perm}} and flood at baseline, younger head age, and higher cattle holding are correlated with rejection. Comparing group rejecters between \textsf{traditional} and non-\textsf{traditional} arms, flood at baseline, net asset values, and livestock holding are different (\textsc{\normalsize Table \ref{tab Greject trad nontrad perm}}). These hint that for non-\textsf{traditional} arm group rejecters, it is baseline flood that may have constrained them from participation, and asset levels for \textsf{traditional} group rejecters.
	
	Acknowledging the reasons for rejection can be different, we tested the independence of each characteristics for individual rejecters (vs. non-individual rejeceters) in \textsc{\normalsize Table \ref{tab Ireject perm}}. Smaller \textsf{HHsize}, being affected with \textsf{FloodInRd1}, and smaller \textsf{NumCows} are associated with individual rejecters. Individual decisions not to participate may be more straightforward: Smaller household size may indicate difficulty in securing the cattle production labour in a household, being hit with a flood may have resulted in lower livestock levels that would prompt them to reconsider partaking in another livestock project. 

	\textsc{\normalsize Table \ref{tab Ireject trad perm}} and \textsc{\normalsize Table \ref{tab Ireject nontrad perm}} compare individual rejecters and nonrejecters in \textsf{traditional} arm and non-\textsf{traditional} arms, respectively. Somewhat surprisingly, smaller household size is found to be correlated with rejection in all arms but more pronounced among \textsf{traditional} members. This hints that \textsf{traditional} arm borrowers may have been looking into cattle production but were held back by lack of household labour. Livestock and other asset values are not correlated with rejection, only cattle holding is smaller for \textsf{traditional} rejecters. Comparison of individual rejecters between \textsf{traditional} and non-\textsf{traditional} arms show no detectable difference (\textsc{\normalsize Table \ref{tab Ireject trad nontrad perm}}). This suggests that indvidual rejecters in all arms were constrained with small household size.

	In summary, group level rejecters between \textsf{traditional} and non-\textsf{traditional} differ that smaller household size and baseline flood withheld participation for non-\textsf{traditional} while low livestock values withheld participation for \textsf{traditional}. Individual rejecters have similar characteristics between two groups.

\hfil\begin{minipage}[t]{14cm}
\hfil\textsc{\normalsize Table \refstepcounter{table}\thetable: Permutation test results of group rejection\label{tab Greject perm}}\\
\setlength{\tabcolsep}{.5pt}
\setlength{\baselineskip}{8pt}
\renewcommand{\arraystretch}{.50}
\hfil\begin{tikzpicture}
\node (tbl) {\input{c:/data/GUK/analysis/save/EstimationMemo/GRejectedPermutationTestResultso800.tex}};
%\input{c:/dropbox/data/ramadan/save/tablecolortemplate.tex}
\end{tikzpicture}\\
\end{minipage}

\hfil\begin{minipage}[t]{14cm}
\hfil\textsc{\normalsize Table \refstepcounter{table}\thetable: Permutation test results of group rejection among traditional arm\label{tab Greject trad perm}}\\
\setlength{\tabcolsep}{.5pt}
\setlength{\baselineskip}{8pt}
\renewcommand{\arraystretch}{.50}
\hfil\begin{tikzpicture}
\node (tbl) {\input{c:/data/GUK/analysis/save/EstimationMemo/GRejectedInTradPermutationTestResultso800.tex}};
%\input{c:/dropbox/data/ramadan/save/tablecolortemplate.tex}
\end{tikzpicture}\\
\end{minipage}

\hfil\begin{minipage}[t]{14cm}
\hfil\textsc{\normalsize Table \refstepcounter{table}\thetable: Permutation test results of group rejection among non-traditional arm\label{tab Greject nontrad perm}}\\
\setlength{\tabcolsep}{.5pt}
\setlength{\baselineskip}{8pt}
\renewcommand{\arraystretch}{.50}
\hfil\begin{tikzpicture}
\node (tbl) {\input{c:/data/GUK/analysis/save/EstimationMemo/GRejectedInNonTradPermutationTestResultso800.tex}};
%\input{c:/dropbox/data/ramadan/save/tablecolortemplate.tex}
\end{tikzpicture}\\
\end{minipage}

\hfil\begin{minipage}[t]{14cm}
\hfil\textsc{\normalsize Table \refstepcounter{table}\thetable: Permutation test results of group rejecters, traditional vs. non-traditional arm\label{tab Greject trad nontrad perm}}\\
\setlength{\tabcolsep}{.5pt}
\setlength{\baselineskip}{8pt}
\renewcommand{\arraystretch}{.50}
\hfil\begin{tikzpicture}
\node (tbl) {\input{c:/data/GUK/analysis/save/EstimationMemo/TradNonTradGRejectedPermutationTestResultso800.tex}};
%\input{c:/dropbox/data/ramadan/save/tablecolortemplate.tex}
\end{tikzpicture}\\
\begin{tabular}{>{\hfill\scriptsize}p{1cm}<{}>{\hfill\scriptsize}p{.25cm}<{}>{\scriptsize}p{12cm}<{\hfill}}
Source:& \multicolumn{2}{l}{\scriptsize Estimated with GUK administrative and survey data.}\\
Notes: & 1. & \textsf{R}'s package \textsf{coin} is used for baseline group mean covariates to conduct approximate permutation tests. Number of repetition is set to 100000. Step-down method is used to adjust for multiple testing of a multi-factor grouping variable. Rejection is individual-rejection. \textsf{TradArm} and \textsf{NonTradArm} columns show means of each group. \\
& 2. & ${}^{***}$, ${}^{**}$, ${}^{*}$ indicate statistical significance at 1\%, 5\%, 10\%, respetively. Standard errors are clustered at group (village) level.
\end{tabular}
\end{minipage}

\hfil\begin{minipage}[t]{14cm}
\hfil\textsc{\normalsize Table \refstepcounter{table}\thetable: Permutation test results of group rejection in traditional arm vs. participants in non-traditional arm\label{tab trad Greject nontrad participate perm}}\\
\setlength{\tabcolsep}{.5pt}
\setlength{\baselineskip}{8pt}
\renewcommand{\arraystretch}{.50}
\hfil\begin{tikzpicture}
\node (tbl) {\input{c:/data/GUK/analysis/save/EstimationMemo/GRejectedTradParticipatedNonTradPermutationTestResultso800.tex}};
%\input{c:/dropbox/data/ramadan/save/tablecolortemplate.tex}
\end{tikzpicture}\\
\begin{tabular}{>{\hfill\scriptsize}p{1cm}<{}>{\hfill\scriptsize}p{.25cm}<{}>{\scriptsize}p{12cm}<{\hfill}}
Source:& \multicolumn{2}{l}{\scriptsize Estimated with GUK administrative and survey data.}\\
Notes: & 1. & \textsf{R}'s package \textsf{coin} is used for baseline group mean covariates to conduct approximate permutation tests. Number of repetition is set to 100000. Step-down method is used to adjust for multiple testing of a multi-factor grouping variable. Rejection is group-rejection. \textsf{TradArm} and \textsf{NonTradArm} columns show means of each group. \\
& 2. & ${}^{***}$, ${}^{**}$, ${}^{*}$ indicate statistical significance at 1\%, 5\%, 10\%, respetively. Standard errors are clustered at group (village) level.
\end{tabular}\\
\end{minipage}

\hfil\begin{minipage}[t]{14cm}
\hfil\textsc{\normalsize Table \refstepcounter{table}\thetable: Permutation test results of individual rejecters, traditional vs. non-traditional arm\label{tab Ireject trad nontrad perm}}\\
\setlength{\tabcolsep}{.5pt}
\setlength{\baselineskip}{8pt}
\renewcommand{\arraystretch}{.50}
\hfil\begin{tikzpicture}
\node (tbl) {\input{c:/data/GUK/analysis/save/EstimationMemo/TradNonTradIRejectedPermutationTestResultso800.tex}};
%\input{c:/dropbox/data/ramadan/save/tablecolortemplate.tex}
\end{tikzpicture}\\
\begin{tabular}{>{\hfill\scriptsize}p{1cm}<{}>{\hfill\scriptsize}p{.25cm}<{}>{\scriptsize}p{12cm}<{\hfill}}
Source:& \multicolumn{2}{l}{\scriptsize Estimated with GUK administrative and survey data.}\\
Notes: & 1. & \textsf{R}'s package \textsf{coin} is used for baseline group mean covariates to conduct approximate permutation tests. Number of repetition is set to 100000. Step-down method is used to adjust for multiple testing of a multi-factor grouping variable. Rejection is individual-rejection. \textsf{TradArm} and \textsf{NonTradArm} columns show means of each group. \\
& 2. & ${}^{***}$, ${}^{**}$, ${}^{*}$ indicate statistical significance at 1\%, 5\%, 10\%, respetively. Standard errors are clustered at group (village) level.
\end{tabular}
\end{minipage}

\hfil\begin{minipage}[t]{14cm}
\hfil\textsc{\normalsize Table \refstepcounter{table}\thetable: Permutation test results of individual rejection\label{tab Ireject perm}}\\
\setlength{\tabcolsep}{.5pt}
\setlength{\baselineskip}{8pt}
\renewcommand{\arraystretch}{.50}
\hfil\begin{tikzpicture}
\node (tbl) {\input{c:/data/GUK/analysis/save/EstimationMemo/IRejectedPermutationTestResultso800.tex}};
%\input{c:/dropbox/data/ramadan/save/tablecolortemplate.tex}
\end{tikzpicture}\\
\begin{tabular}{>{\hfill\scriptsize}p{1cm}<{}>{\hfill\scriptsize}p{.25cm}<{}>{\scriptsize}p{12cm}<{\hfill}}
Source:& \multicolumn{2}{l}{\scriptsize Estimated with GUK administrative and survey data.}\\
Notes: & 1. & \textsf{R}'s package \textsf{coin} is used for baseline group mean covariates to conduct approximate permutation tests. Number of repetition is set to 100000. Step-down method is used to adjust for multiple testing of a multi-factor grouping variable. Rejection is either group-rejection or individual-rejection. \textsf{Rejected} and \textsf{Nonrejected} columns show means of each group. For \textsf{Arm}, proportions of non-traditional arm are given. Individual rejection is observed only for non group rejecters. Sample size is smaller in \textsc{Table \ref{tab Ireject perm}} as 70 observations are dropped. \\
& 2. & ${}^{***}$, ${}^{**}$, ${}^{*}$ indicate statistical significance at 1\%, 5\%, 10\%, respetively. Standard errors are clustered at group (village) level.
\end{tabular}\\
\end{minipage}

\hfil\begin{minipage}[t]{14cm}
\hfil\textsc{\normalsize Table \refstepcounter{table}\thetable: Permutation test results of individual rejection among traditional arm\label{tab Ireject trad perm}}\\
\setlength{\tabcolsep}{.5pt}
\setlength{\baselineskip}{8pt}
\renewcommand{\arraystretch}{.50}
\hfil\begin{tikzpicture}
\node (tbl) {\input{c:/data/GUK/analysis/save/EstimationMemo/IRejectedInTradPermutationTestResultso800.tex}};
%\input{c:/dropbox/data/ramadan/save/tablecolortemplate.tex}
\end{tikzpicture}\\
\end{minipage}

\hfil\begin{minipage}[t]{14cm}
\hfil\textsc{\normalsize Table \refstepcounter{table}\thetable: Permutation test results of individual rejection among non-traditional arm\label{tab Ireject nontrad perm}}\\
\setlength{\tabcolsep}{.5pt}
\setlength{\baselineskip}{8pt}
\renewcommand{\arraystretch}{.50}
\hfil\begin{tikzpicture}
\node (tbl) {\input{c:/data/GUK/analysis/save/EstimationMemo/IRejectedInNonTradPermutationTestResultso800.tex}};
%\input{c:/dropbox/data/ramadan/save/tablecolortemplate.tex}
\end{tikzpicture}\\
\end{minipage}


%	A closer look at the nonparticipation correlates among \textsf{traditional} arm mebers in \textsc{\normalsize Table \ref{tab reject trad perm}} and non-\textsf{traditional} arm members in \textsc{\normalsize Table \ref{tab reject nontrad perm}} reveal possible differences in the causes. Rejection among \textsf{traditional} members tend to be associated with lower livestock holding but not with higher flood exposure nor smaller household size, while rejecters among non-\textsf{traditional} members are more likely to have suffered from flood in baseline and have smaller household size. Since the offered arms were randomised, rejecters of \textsf{traditional} arm, who were not more exposed to flood and have similar household size at the mean, may have accepted the offer had they been offered non- \textsf{traditional} lending. Henceforth, we conjecture that flood exposure and household size are the potential impediments in larger size loans. This implies that there may not be minimum livestock and asset holding levels to partake the larger loans, and a poverty trap at this level may be overcome.



\hfil\begin{minipage}[t]{14cm}
\hfil\textsc{\normalsize Table \refstepcounter{table}\thetable: Permutation test results of borrowers, cow vs. non-cow arms\label{tab accept cow noncow perm}}\\
\setlength{\tabcolsep}{.5pt}
\setlength{\baselineskip}{8pt}
\renewcommand{\arraystretch}{.50}
\hfil\begin{tikzpicture}
\node (tbl) {\input{c:/data/GUK/analysis/save/EstimationMemo/AcceptedCowNonCowPermutationTestResultso800.tex}};
%\input{c:/dropbox/data/ramadan/save/tablecolortemplate.tex}
\end{tikzpicture}\\
\begin{tabular}{>{\hfill\scriptsize}p{1cm}<{}>{\hfill\scriptsize}p{.25cm}<{}>{\scriptsize}p{12cm}<{\hfill}}
Source:& \multicolumn{2}{l}{\scriptsize Estimated with GUK administrative and survey data.}\\
Notes: & 1. & \textsf{R}'s package \textsf{coin} is used for baseline group mean covariates to conduct approximate permutation tests. Number of repetition is set to 100000. Step-down method is used to adjust for multiple testing of a multi-factor grouping variable. Rejection is group-rejection. \textsf{CowArm} and \textsf{NonCowArm} columns show means of each group. \\
& 2. & ${}^{***}$, ${}^{**}$, ${}^{*}$ indicate statistical significance at 1\%, 5\%, 10\%, respetively. Standard errors are clustered at group (village) level.
\end{tabular}
\end{minipage}

\hfil\begin{minipage}[t]{14cm}
\hfil\textsc{\normalsize Table \refstepcounter{table}\thetable: Permutation test results of individual rejecters, traditional vs. non-traditional arm\label{tab indiv reject trad nontrad}}\\
\setlength{\tabcolsep}{.5pt}
\setlength{\baselineskip}{8pt}
\renewcommand{\arraystretch}{.50}
\hfil\begin{tikzpicture}
\node (tbl) {\input{c:/data/GUK/analysis/save/EstimationMemo/TradNonTradIRejectedPermutationTestResultso800.tex}};
%\input{c:/dropbox/data/ramadan/save/tablecolortemplate.tex}
\end{tikzpicture}\\
\begin{tabular}{>{\hfill\scriptsize}p{1cm}<{}>{\hfill\scriptsize}p{.25cm}<{}>{\scriptsize}p{12cm}<{\hfill}}
Source:& \multicolumn{2}{l}{\scriptsize Estimated with GUK administrative and survey data.}\\
Notes: & 1. & \textsf{R}'s package \textsf{coin} is used for baseline group mean covariates to conduct approximate permutation tests. Number of repetition is set to 100000. Step-down method is used to adjust for multiple testing of a multi-factor grouping variable. \textsf{TradArm} is group-rejecters in \textsf{traditional} arm \textsf{NonTradArm} is borrowers in non-\textsf{traditional} arms. Both columns show means of each group. \\
& 2. & ${}^{***}$, ${}^{**}$, ${}^{*}$ indicate statistical significance at 1\%, 5\%, 10\%, respetively. Standard errors are clustered at group (village) level.
\end{tabular}
\end{minipage}

\hfil\begin{minipage}[t]{14cm}
\hfil\textsc{\normalsize Table \refstepcounter{table}\thetable: Permutation test results of borowers, cow vs. large grace arms\label{tab accept cow large grace perm}}\\
\setlength{\tabcolsep}{.5pt}
\setlength{\baselineskip}{8pt}
\renewcommand{\arraystretch}{.50}
\hfil\begin{tikzpicture}
\node (tbl) {\input{c:/data/GUK/analysis/save/EstimationMemo/AcceptedCowLargeGracePermutationTestResultso800.tex}};
%\input{c:/dropbox/data/ramadan/save/tablecolortemplate.tex}
\end{tikzpicture}\\
\begin{tabular}{>{\hfill\scriptsize}p{1cm}<{}>{\hfill\scriptsize}p{.25cm}<{}>{\scriptsize}p{12cm}<{\hfill}}
Source:& \multicolumn{2}{l}{\scriptsize Estimated with GUK administrative and survey data.}\\
Notes: & 1. & \textsf{R}'s package \textsf{coin} is used for baseline group mean covariates to conduct approximate permutation tests. Number of repetition is set to 100000. Step-down method is used to adjust for multiple testing of a multi-factor grouping variable. Rejection is group-rejection. \textsf{CowArm} and \textsf{LargeGraceArm} columns show means of each group. \\
& 2. & ${}^{***}$, ${}^{**}$, ${}^{*}$ indicate statistical significance at 1\%, 5\%, 10\%, respetively. Standard errors are clustered at group (village) level.
\end{tabular}\\
\end{minipage}

\hfil\begin{minipage}[t]{14cm}
\hfil\textsc{\normalsize Table \refstepcounter{table}\thetable: Permutation test results of arm assignment, traditional vs. non-traditional arms\label{tab trad nontrad random assignment perm}}\\
\setlength{\tabcolsep}{.5pt}
\setlength{\baselineskip}{8pt}
\renewcommand{\arraystretch}{.50}
\hfil\begin{tikzpicture}
\node (tbl) {\input{c:/data/GUK/analysis/save/EstimationMemo/RandomAssignmentTradNonTradPermutationTestResultso800.tex}};
%\input{c:/dropbox/data/ramadan/save/tablecolortemplate.tex}
\end{tikzpicture}\\
\begin{tabular}{>{\hfill\scriptsize}p{1cm}<{}>{\hfill\scriptsize}p{.25cm}<{}>{\scriptsize}p{12cm}<{\hfill}}
Source:& \multicolumn{2}{l}{\scriptsize Estimated with GUK administrative and survey data.}\\
Notes: & 1. & \textsf{R}'s package \textsf{coin} is used for baseline group mean covariates to conduct approximate permutation tests. Number of repetition is set to 100000. Step-down method is used to adjust for multiple testing of a multi-factor grouping variable. Rejection is group-rejection. \textsf{CowArm} and \textsf{LargeGraceArm} columns show means of each group. \\
& 2. & 40 are lost to flood before arm assignment. Standard errors are clustered at group (village) level.
\end{tabular}\\
\end{minipage}


\section{Estimated results}
\label{AppendixEstimationTables}
\setcounter{table}{0}



\subsection{Repayment}
\subsubsection{Saving and repayment}

\hspace{-1cm}\begin{minipage}[t]{14cm}
\hfil\textsc{\normalsize Table \refstepcounter{table}\thetable: ANCOVA estimation of net saving and repayment\label{tab ANCOVA Repayment}}\\
\setlength{\tabcolsep}{1pt}
\setlength{\baselineskip}{8pt}
\renewcommand{\arraystretch}{.55}
\hspace{-.75cm}\begin{tikzpicture}
\node (tbl) {\input{c:/data/GUK/analysis/save/EstimationMemo/RepaymentANCOVAEstimationResults.tex}};
%\input{c:/dropbox/data/ramadan/save/tablecolortemplate.tex}
\end{tikzpicture}\\
\renewcommand{\arraystretch}{.8}
\setlength{\tabcolsep}{1pt}
\begin{tabular}{>{\hfill\scriptsize}p{1cm}<{}>{\hfill\scriptsize}p{.25cm}<{}>{\scriptsize}p{12cm}<{\hfill}}
Source:& \multicolumn{2}{l}{\scriptsize Estimated with GUK administrative and survey data.}\\
Notes: & 1. & ANCOVA estimates using administrative and survey data. Post treatment regressands are regressed on categorical variables, pre-treatment regressand and other covariates. Head age and literacy are from baseline survey data.  Saving and repayment information is taken from administrative data. Time invariant household characteristics are taken from household survey data. Administrative data are merged with survey data by the dating the survey rounds in administrative data. Net saving is saving - withdrawal. Excess repayment is repayment - due amount. \textsf{LY2, LY3, LY4} are dummy variables for second, third, and 	fourth year into borrowing.\\
& 2. & $P$ values in percentages in parenthesises. Standard errors are clustered at group (village) level.
%${}^{***}$, ${}^{**}$, ${}^{*}$ indicate statistical significance at 1\%, 5\%, 10\%, respetively. Standard errors are clustered at group (village) level.
\end{tabular}
\end{minipage}


\hspace{-1cm}\begin{minipage}[t]{14cm}
\hfil\textsc{\normalsize Table \refstepcounter{table}\thetable: ANCOVA estimation of net saving and repayment by attributes\label{tab ANCOVA Repayment attributes}}\\
\setlength{\tabcolsep}{1pt}
\setlength{\baselineskip}{8pt}
\renewcommand{\arraystretch}{.55}
\hspace{-.75cm}\begin{tikzpicture}
\node (tbl) {\input{c:/data/GUK/analysis/save/EstimationMemo/RepaymentAttributesANCOVAEstimationResults.tex}};
%\input{c:/dropbox/data/ramadan/save/tablecolortemplate.tex}
\end{tikzpicture}\\
\renewcommand{\arraystretch}{.8}
\setlength{\tabcolsep}{1pt}
\begin{tabular}{>{\hfill\scriptsize}p{1cm}<{}>{\hfill\scriptsize}p{.25cm}<{}>{\scriptsize}p{12cm}<{\hfill}}
Source:& \multicolumn{2}{l}{\scriptsize Estimated with GUK administrative and survey data.}\\
Notes: & 1. & ANCOVA estimates using administrative and survey data. Post treatment regressands are regressed on categorical variables, pre-treatment regressand and other covariates. Head age and literacy are from baseline survey data.  \textsf{LargeSize} is an indicator function if the arm is of large size, \textsf{WithGrace} is an indicator function if the arm is with a grace period, \textsf{InKind} is an indicator function if the arm provides a cow. Saving and repayment information is taken from administrative data. Time invariant household characteristics are taken from household survey data. Administrative data are merged with survey data by the dating the survey rounds in administrative data. Net saving is saving - withdrawal. Excess repayment is repayment - due amount. \textsf{LY2, LY3, LY4} are dummy variables for second, third, and 	fourth year into borrowing.\\
& 2. & $P$ values in percentages in parenthesises. Standard errors are clustered at group (village) level.
%${}^{***}$, ${}^{**}$, ${}^{*}$ indicate statistical significance at 1\%, 5\%, 10\%, respetively. Standard errors are clustered at group (village) level.
\end{tabular}
\end{minipage}

\hspace{-1cm}\begin{minipage}[t]{14cm}
\hfil\textsc{\normalsize Table \refstepcounter{table}\thetable: ANCOVA estimation of net saving and repayment, ultra poor vs. moderately poor\label{tab ANCOVA Repayment time varying poverty}}\\
\setlength{\tabcolsep}{1pt}
\setlength{\baselineskip}{8pt}
\renewcommand{\arraystretch}{.55}
\hspace{-.75cm}\begin{tikzpicture}
\node (tbl) {\input{c:/data/GUK/analysis/save/EstimationMemo/RepaymentTimeVaryingPovertyStatusANCOVAEstimationResults.tex}};
%\input{c:/dropbox/data/ramadan/save/tablecolortemplate.tex}
\end{tikzpicture}\\
\renewcommand{\arraystretch}{.8}
\setlength{\tabcolsep}{1pt}
\begin{tabular}{>{\hfill\scriptsize}p{1cm}<{}>{\hfill\scriptsize}p{.25cm}<{}>{\scriptsize}p{12cm}<{\hfill}}
Source:& \multicolumn{2}{l}{\scriptsize Estimated with GUK administrative and survey data.}\\
Notes: & 1. & ANCOVA estimates using administrative and survey data. Post treatment regressands are regressed on categorical variables, pre-treatment regressand and other covariates. Head age and literacy are from baseline survey data.  \textsf{UltraPoor} is an indicator function if the household is classified as the ultra poor. Saving and repayment information is taken from administrative data. Time invariant household characteristics are taken from household survey data. Administrative data are merged with survey data by the dating the survey rounds in administrative data. Net saving is saving - withdrawal. Excess repayment is repayment - due amount. \textsf{LY2, LY3, LY4} are dummy variables for second, third, and 	fourth year into borrowing.\\
& 2. & $P$ values in percentages in parenthesises. Standard errors are clustered at group (village) level.
%${}^{***}$, ${}^{**}$, ${}^{*}$ indicate statistical significance at 1\%, 5\%, 10\%, respetively. Standard errors are clustered at group (village) level.
\end{tabular}
\end{minipage}


\begin{palepinkleftbar}
\begin{finding}
\textsc{\small Table \ref{tab ANCOVA Repayment}} shows regression results for net saving, repayment, and effective repayment (net saving + repayment) using monthly administrative data. Monthly mean repayment is given by 48 times the estimated values in colum (5). One sees that \textsf{traditional} has the lowest mean repayment. It is shown that they repaid loan year 2 and 3 

\end{finding}
\end{palepinkleftbar}

\subsubsection{Shortfall}
\label{AppendixShortfall}


\hspace{-1cm}\begin{minipage}[t]{14cm}
\hfil\textsc{\normalsize Table \refstepcounter{table}\thetable: Group level effects of repayment shortfall\label{tab shortfall group}}\\
\setlength{\tabcolsep}{1pt}
\setlength{\baselineskip}{8pt}
\renewcommand{\arraystretch}{.6}
\hfil\begin{tikzpicture}
\node (tbl) {\input{c:/data/GUK/analysis/save/EstimationMemo/ShortfallGroupEstimationResults1.tex}};
%\input{c:/dropbox/data/ramadan/save/tablecolortemplate.tex}
\end{tikzpicture}
\end{minipage}

\addtocounter{table}{-1}
\hspace{-1cm}\begin{minipage}[t]{14cm}
\hfil\textsc{\normalsize Table \refstepcounter{table}\thetable: Group level effects of repayment shortfall (continued)\label{tab shortfall group2}}\\
\setlength{\tabcolsep}{1pt}
\setlength{\baselineskip}{8pt}
\renewcommand{\arraystretch}{.6}
\hfil\begin{tikzpicture}
\node (tbl) {\input{c:/data/GUK/analysis/save/EstimationMemo/ShortfallGroupEstimationResults2.tex}};
%\input{c:/dropbox/data/ramadan/save/tablecolortemplate.tex}
\end{tikzpicture}\\
\renewcommand{\arraystretch}{.8}
\setlength{\tabcolsep}{1pt}
\begin{tabular}{>{\hfill\scriptsize}p{1cm}<{}>{\hfill\scriptsize}p{.25cm}<{}>{\scriptsize}p{12cm}<{\hfill}}
Source:& \multicolumn{2}{l}{\scriptsize Estimated with GUK administrative data.}\\
Notes: & 1. & Group fixed effects estimates of repayment shortfall. Group fixed effects are controlled by differncing out respecive means from the data matrix. Intercept terms are omitted in estimating equations. Shortfall is (planned installment) - (actual repayment). OtherShortfall indicates mean shortfall of other members in a group. Group repayment shortfall rates (GRSR) is (shortfall)/(planned installment). GRSR is defined as high if the first six months' repayment shortfall rate is above median, low if otherwise. Median GRSR is -1.42.\\
& 2. & ${}^{***}$, ${}^{**}$, ${}^{*}$ indicate statistical significance at 1\%, 5\%, 10\%, respetively. Standard errors are clustered at group (village) level.
\end{tabular}
\end{minipage}

\hspace{-1cm}\begin{minipage}[t]{14cm}
\hfil\textsc{\normalsize Table \refstepcounter{table}\thetable: Individual level effects of repayment shortfall\label{tab shortfall indiv}}\\
\setlength{\tabcolsep}{1pt}
\setlength{\baselineskip}{8pt}
\renewcommand{\arraystretch}{.6}
\hfil\begin{tikzpicture}
\node (tbl) {\input{c:/data/GUK/analysis/save/EstimationMemo/ShortfallIndividualEstimationResults1.tex}};
%\input{c:/dropbox/data/ramadan/save/tablecolortemplate.tex}
\end{tikzpicture}
\end{minipage}

\addtocounter{table}{-1}
\hspace{-1cm}\begin{minipage}[t]{14cm}
\hfil\textsc{\normalsize Table \refstepcounter{table}\thetable: Individual level effects of repayment shortfall (continued)\label{tab shortfall indiv2}}\\
\setlength{\tabcolsep}{1pt}
\setlength{\baselineskip}{8pt}
\renewcommand{\arraystretch}{.6}
\hfil\begin{tikzpicture}
\node (tbl) {\input{c:/data/GUK/analysis/save/EstimationMemo/ShortfallIndividualEstimationResults2.tex}};
%\input{c:/dropbox/data/ramadan/save/tablecolortemplate.tex}
\end{tikzpicture}\\
\renewcommand{\arraystretch}{.8}
\setlength{\tabcolsep}{1pt}
\begin{tabular}{>{\hfill\scriptsize}p{1cm}<{}>{\hfill\scriptsize}p{.25cm}<{}>{\scriptsize}p{12cm}<{\hfill}}
Source:& \multicolumn{2}{l}{\scriptsize Estimated with GUK administrative data.}\\
Notes: & 1. & Group fixed effects estimates of repayment shortfall. Group fixed effects are controlled by differncing out respecive means from the data matrix. Intercept terms are omitted in estimating equations. Shortfall is (planned installment) - (actual repayment). OtherShortfall indicates mean shortfall of other members in a group. Group repayment shortfall rates (GRSR) is (shortfall)/(planned installment). GRSR is defined as high if the first six months' repayment shortfall rate is above median, low if otherwise. Median GRSR is -1.42.\\
& 2. & ${}^{***}$, ${}^{**}$, ${}^{*}$ indicate statistical significance at 1\%, 5\%, 10\%, respetively. Standard errors are clustered at group (village) level.
\end{tabular}
\end{minipage}

\begin{palepinkleftbar}
\begin{finding}
\textsc{\small Table \ref{tab shortfall group}} shows group level repayment shortfall has a positive autocorrelation hence is persistent. In (1), the coefficient is smaller in groups with high shortfall rates, hinting loan repayment discipline as a group at some intermediate level. In (2) and (3), group level shortfall gets smaller in the third year than in the second year for all arms, indicating stronger efforts in repayment in the final loan year. In (4) and (5), the \textsf{UltraPoor} is found to have no larger repayment shortfall than the moderately poor, except for the \textsf{Large} arm or \textsf{Upfront} attribute in the second loan year. \textsc{\small Table \ref{tab shortfall indiv}} (1), (4) and (5) also show persistence for individuals, although the magnitude is much smaller. In (1), lagged shortfall of others decreases with own shortfall only in high GRSR group. This confirms the group level repayment discipline that is consistent with a steady state short fall rate at an intermediate level as a group.  In (2), shortfall is larger in the second and third year for the arms with a grace period. This reflects that a grace period does not necessarily help the borrowers to prepare repayments, which is against the intention to match the repayment with the cash flow. The ultra poor has smaller shortfall in all arms in year 2 except in the large grace arm in year 3. The results on the ultra poor may indicate the difference with the moderately poor is nominal.
\end{finding}
\end{palepinkleftbar}

\subsection{Schooling}

\hspace{-1cm}\begin{minipage}[t]{14cm}
\hfil\textsc{\normalsize Table \refstepcounter{table}\thetable: ANCOVA estimation of school enrollment\label{tab ANCOVA enroll}}\\
\setlength{\tabcolsep}{1pt}
\setlength{\baselineskip}{8pt}
\renewcommand{\arraystretch}{.55}
\hfil\begin{tikzpicture}
\node (tbl) {\input{c:/data/GUK/analysis/save/EstimationMemo/SchoolingANCOVAEstimationResults.tex}};
%\input{c:/dropbox/data/ramadan/save/tablecolortemplate.tex}
\end{tikzpicture}\\
\renewcommand{\arraystretch}{.8}
\setlength{\tabcolsep}{1pt}
\begin{tabular}{>{\hfill\scriptsize}p{1cm}<{}>{\hfill\scriptsize}p{.25cm}<{}>{\scriptsize}p{12cm}<{\hfill}}
Source:& \multicolumn{2}{l}{\scriptsize Estimated with GUK administrative and survey data.}\\
Notes: & 1. & ANCOVA estimates using administrative and survey data. Post treatment regressands are regressed on categorical variables, pre-treatment regressand and other covariates. Head age and literacy are from baseline survey data. Interaction terms of dummy variables are demeaned before interacting. The first column gives mean and standard deviation (in parenthesises) of each covariates before demeaning.\\
& 2. & $P$ values in percentages in parenthesises. Standard errors are clustered at group (village) level.%
%${}^{***}$, ${}^{**}$, ${}^{*}$ indicate statistical significance at 1\%, 5\%, 10\%, respetively. Standard errors are clustered at group (village) level.
\end{tabular}
\end{minipage}

$ $\\

\vspace{-1cm}
\hspace{-1cm}\begin{minipage}[t]{14cm}
\hfil\textsc{\normalsize Table \refstepcounter{table}\thetable: ANCOVA estimation of school enrollment by attributes\label{tab ANCOVA enroll attributes}}\\
\setlength{\tabcolsep}{1pt}
\setlength{\baselineskip}{8pt}
\renewcommand{\arraystretch}{.55}
\hfil\begin{tikzpicture}
\node (tbl) {\input{c:/data/GUK/analysis/save/EstimationMemo/SchoolingAttributesANCOVAEstimationResults.tex}};
%\input{c:/dropbox/data/ramadan/save/tablecolortemplate.tex}
\end{tikzpicture}\\
\renewcommand{\arraystretch}{.8}
\setlength{\tabcolsep}{1pt}
\begin{tabular}{>{\hfill\scriptsize}p{1cm}<{}>{\hfill\scriptsize}p{.25cm}<{}>{\scriptsize}p{12cm}<{\hfill}}
Source:& \multicolumn{2}{l}{\scriptsize Estimated with GUK administrative and survey data.}\\
Notes: & 1. & ANCOVA estimates using administrative and survey data. Post treatment regressands are regressed on categorical variables, pre-treatment regressand and other covariates. Head age and literacy are from baseline survey data.  Interaction terms of dummy variables are demeaned before interacting. The first column gives mean and standard deviation (in parenthesises) of each covariates before demeaning.\\
& 2. & $P$ values in percentages in parenthesises. Standard errors are clustered at group (village) level.%
%${}^{***}$, ${}^{**}$, ${}^{*}$ indicate statistical significance at 1\%, 5\%, 10\%, respetively. Standard errors are clustered at group (village) level.
\end{tabular}
\end{minipage}

\newpage
$ $ 
\vspace{-1cm}
\hspace{-1cm}\begin{minipage}[t]{14cm}
\hfil\textsc{\normalsize Table \refstepcounter{table}\thetable: ANCOVA estimation of school enrollment by poverty status\label{tab ANCOVA enroll poverty}}\\
\setlength{\tabcolsep}{1pt}
\setlength{\baselineskip}{8pt}
\renewcommand{\arraystretch}{.55}
\hfil\begin{tikzpicture}
\node (tbl) {\input{c:/data/GUK/analysis/save/EstimationMemo/SchoolingPovertyStatusANCOVAEstimationResults.tex}};
%\input{c:/dropbox/data/ramadan/save/tablecolortemplate.tex}
\end{tikzpicture}\\
\renewcommand{\arraystretch}{.8}
\setlength{\tabcolsep}{1pt}
\begin{tabular}{>{\hfill\scriptsize}p{1cm}<{}>{\hfill\scriptsize}p{.25cm}<{}>{\scriptsize}p{12cm}<{\hfill}}
Source:& \multicolumn{2}{l}{\scriptsize Estimated with GUK administrative and survey data.}\\
Notes: & 1. & ANCOVA estimates using administrative and survey data. Post treatment regressands are regressed on categorical variables, pre-treatment regressand and other covariates. Head age and literacy are from baseline survey data.  Interaction terms of dummy variables are demeaned before interacting. The first column gives mean and standard deviation (in parenthesises) of each covariates before demeaning.\\
& 2. & $P$ values in percentages in parenthesises. Standard errors are clustered at group (village) level.%
%${}^{***}$, ${}^{**}$, ${}^{*}$ indicate statistical significance at 1\%, 5\%, 10\%, respetively. Standard errors are clustered at group (village) level.
\end{tabular}
\end{minipage}

\hspace{-1cm}\begin{minipage}[t]{14cm}
\hfil\textsc{\normalsize Table \refstepcounter{table}\thetable: ANCOVA estimation of school enrollment by attributes and time\label{tab ANCOVA enroll time varying attributes}}\\
\setlength{\tabcolsep}{1pt}
\setlength{\baselineskip}{8pt}
\renewcommand{\arraystretch}{.55}
\hfil\begin{tikzpicture}
\node (tbl) {\input{c:/data/GUK/analysis/save/EstimationMemo/SchoolingTimeVaryingAttributesANCOVAEstimationResults_1.tex}};
%\input{c:/dropbox/data/ramadan/save/tablecolortemplate.tex}
\end{tikzpicture}
\end{minipage}

\hspace{-1cm}\begin{minipage}[t]{14cm}
\hfil\textsc{\normalsize Table \refstepcounter{table}\thetable: ANCOVA estimation of school enrollment by attributes and time (continued)\label{tab ANCOVA enroll time varying attributes2}}\\
\setlength{\tabcolsep}{1pt}
\setlength{\baselineskip}{8pt}
\renewcommand{\arraystretch}{.55}
\hfil\begin{tikzpicture}
\node (tbl) {\input{c:/data/GUK/analysis/save/EstimationMemo/SchoolingTimeVaryingAttributesANCOVAEstimationResults_2.tex}};
%\input{c:/dropbox/data/ramadan/save/tablecolortemplate.tex}
\end{tikzpicture}\\
\renewcommand{\arraystretch}{.8}
\setlength{\tabcolsep}{1pt}
\begin{tabular}{>{\hfill\scriptsize}p{1cm}<{}>{\hfill\scriptsize}p{.25cm}<{}>{\scriptsize}p{12cm}<{\hfill}}
Source:& \multicolumn{2}{l}{\scriptsize Estimated with GUK administrative and survey data.}\\
Notes: & 1. & ANCOVA estimates using administrative and survey data. Post treatment regressands are regressed on categorical variables, pre-treatment regressand and other covariates. Head age and literacy are from baseline survey data.  Interaction terms of dummy variables are demeaned before interacting. The first column gives mean and standard deviation (in parenthesises) of each covariates before demeaning.\\
& 2. & $P$ values in percentages in parenthesises. Standard errors are clustered at group (village) level.%
%${}^{***}$, ${}^{**}$, ${}^{*}$ indicate statistical significance at 1\%, 5\%, 10\%, respetively. Standard errors are clustered at group (village) level.
\end{tabular}
\end{minipage}

\hspace{-1cm}\begin{minipage}[t]{14cm}
\hfil\textsc{\normalsize Table \refstepcounter{table}\thetable: ANCOVA estimation of school enrollment by attributes and time (continued 2)\label{tab ANCOVA enroll time varying attributes3}}\\
\setlength{\tabcolsep}{1pt}
\setlength{\baselineskip}{8pt}
\renewcommand{\arraystretch}{.55}
\hfil\begin{tikzpicture}
\node (tbl) {\input{c:/data/GUK/analysis/save/EstimationMemo/SchoolingTimeVaryingAttributesANCOVAEstimationResults_3.tex}};
%\input{c:/dropbox/data/ramadan/save/tablecolortemplate.tex}
\end{tikzpicture}\\
\renewcommand{\arraystretch}{.8}
\setlength{\tabcolsep}{1pt}
\begin{tabular}{>{\hfill\scriptsize}p{1cm}<{}>{\hfill\scriptsize}p{.25cm}<{}>{\scriptsize}p{12cm}<{\hfill}}
Source:& \multicolumn{2}{l}{\scriptsize Estimated with GUK administrative and survey data.}\\
Notes: & 1. & ANCOVA estimates using administrative and survey data. Post treatment regressands are regressed on categorical variables, pre-treatment regressand and other covariates. Head age and literacy are from baseline survey data.  Interaction terms of dummy variables are demeaned before interacting. The first column gives mean and standard deviation (in parenthesises) of each covariates before demeaning.\\
& 2. & $P$ values in percentages in parenthesises. Standard errors are clustered at group (village) level.%
%${}^{***}$, ${}^{**}$, ${}^{*}$ indicate statistical significance at 1\%, 5\%, 10\%, respetively. Standard errors are clustered at group (village) level.
\end{tabular}
\end{minipage}

\subsection{Assets}


\hspace{-1cm}\begin{minipage}[t]{14cm}
\hfil\textsc{\normalsize Table \refstepcounter{table}\thetable: ANCOVA estimation of assets\label{tab ANCOVA assets}}\\
\setlength{\tabcolsep}{1pt}
\setlength{\baselineskip}{8pt}
\renewcommand{\arraystretch}{.55}
\hfil\begin{tikzpicture}
\node (tbl) {\input{c:/data/GUK/analysis/save/EstimationMemo/AssetANCOVAEstimationResults.tex}};
%\input{c:/dropbox/data/ramadan/save/tablecolortemplate.tex}
\end{tikzpicture}\\
\renewcommand{\arraystretch}{.8}
\setlength{\tabcolsep}{1pt}
\begin{tabular}{>{\hfill\scriptsize}p{1cm}<{}>{\hfill\scriptsize}p{.25cm}<{}>{\scriptsize}p{12cm}<{\hfill}}
Source:& \multicolumn{2}{l}{\scriptsize Estimated with GUK administrative and survey data.}\\
Notes: & 1. & ANCOVA estimates using administrative and survey data. Post treatment regressands are regressed on categorical variables, pre-treatment regressand and other covariates. Head age and literacy are from baseline survey data.  Interaction terms of dummy variables are demeaned before interacting. The first column gives mean and standard deviation (in parenthesises) of each covariates before demeaning.\\
& 2. & $P$ values in percentages in parenthesises. Standard errors are clustered at group (village) level.
%${}^{***}$, ${}^{**}$, ${}^{*}$ indicate statistical significance at 1\%, 5\%, 10\%, respetively. Standard errors are clustered at group (village) level.
\end{tabular}
\end{minipage}

\hspace{-1cm}\begin{minipage}[t]{14cm}
\hfil\textsc{\normalsize Table \refstepcounter{table}\thetable: ANCOVA estimation of assets by poverty status \label{tab ANCOVA assets poverty}}\\
\setlength{\tabcolsep}{1pt}
\setlength{\baselineskip}{8pt}
\renewcommand{\arraystretch}{.55}
\hfil\begin{tikzpicture}
\node (tbl) {\input{c:/data/GUK/analysis/save/EstimationMemo/AssetPovertyStatusANCOVAEstimationResults.tex}};
%\input{c:/dropbox/data/ramadan/save/tablecolortemplate.tex}
\end{tikzpicture}\\
\renewcommand{\arraystretch}{.8}
\setlength{\tabcolsep}{1pt}
\begin{tabular}{>{\hfill\scriptsize}p{1cm}<{}>{\hfill\scriptsize}p{.25cm}<{}>{\scriptsize}p{12cm}<{\hfill}}
Source:& \multicolumn{2}{l}{\scriptsize Estimated with GUK administrative and survey data.}\\
Notes: & 1. & ANCOVA estimates using administrative and survey data. Post treatment regressands are regressed on categorical variables, pre-treatment regressand and other covariates. Head age and literacy are from baseline survey data.  Interaction terms of dummy variables are demeaned before interacting. The first column gives mean and standard deviation (in parenthesises) of each covariates before demeaning.\\
& 2. & $P$ values in percentages in parenthesises. Standard errors are clustered at group (village) level.
%${}^{***}$, ${}^{**}$, ${}^{*}$ indicate statistical significance at 1\%, 5\%, 10\%, respetively. Standard errors are clustered at group (village) level.
\end{tabular}
\end{minipage}

\hspace{-1cm}\begin{minipage}[t]{14cm}
\hfil\textsc{\normalsize Table \refstepcounter{table}\thetable: ANCOVA estimation of assets by attributes \label{tab ANCOVA assets attributes}}\\
\setlength{\tabcolsep}{1pt}
\setlength{\baselineskip}{8pt}
\renewcommand{\arraystretch}{.55}
\hfil\begin{tikzpicture}
\node (tbl) {\input{c:/data/GUK/analysis/save/EstimationMemo/AssetAttributesANCOVAEstimationResults.tex}};
%\input{c:/dropbox/data/ramadan/save/tablecolortemplate.tex}
\end{tikzpicture}\\
\renewcommand{\arraystretch}{.8}
\setlength{\tabcolsep}{1pt}
\begin{tabular}{>{\hfill\scriptsize}p{1cm}<{}>{\hfill\scriptsize}p{.25cm}<{}>{\scriptsize}p{12cm}<{\hfill}}
Source:& \multicolumn{2}{l}{\scriptsize Estimated with GUK administrative and survey data.}\\
Notes: & 1. & ANCOVA estimates using administrative and survey data. Post treatment regressands are regressed on categorical variables, pre-treatment regressand and other covariates. Head age and literacy are from baseline survey data.  Interaction terms of dummy variables are demeaned before interacting. The first column gives mean and standard deviation (in parenthesises) of each covariates before demeaning.\\
& 2. & $P$ values in percentages in parenthesises. Standard errors are clustered at group (village) level.
%${}^{***}$, ${}^{**}$, ${}^{*}$ indicate statistical significance at 1\%, 5\%, 10\%, respetively. Standard errors are clustered at group (village) level.
\end{tabular}
\end{minipage}


\hspace{-1cm}\begin{minipage}[t]{14cm}
\hfil\textsc{\normalsize Table \refstepcounter{table}\thetable: ANCOVA estimation of assets by period\label{tab ANCOVA assets period}}\\
\setlength{\tabcolsep}{1pt}
\setlength{\baselineskip}{8pt}
\renewcommand{\arraystretch}{.55}
\hfil\begin{tikzpicture}
\node (tbl) {\input{c:/data/GUK/analysis/save/EstimationMemo/AssetTimeVaryingANCOVAEstimationResults.tex}};
%\input{c:/dropbox/data/ramadan/save/tablecolortemplate.tex}
\end{tikzpicture}\\
\renewcommand{\arraystretch}{.8}
\setlength{\tabcolsep}{1pt}
\begin{tabular}{>{\hfill\scriptsize}p{1cm}<{}>{\hfill\scriptsize}p{.25cm}<{}>{\scriptsize}p{12cm}<{\hfill}}
Source:& \multicolumn{2}{l}{\scriptsize Estimated with GUK administrative and survey data.}\\
Notes: & 1. & ANCOVA estimates using administrative and survey data. Post treatment regressands are regressed on categorical variables, pre-treatment regressand and other covariates. Head age and literacy are from baseline survey data.  \textsf{LargeSize} is an indicator function if the arm is of large size, \textsf{WithGrace} is an indicator function if the arm is with a grace period, \textsf{InKind} is an indicator function if the arm provides a cow. Interaction terms of dummy variables are demeaned before interacting. The first column gives mean and standard deviation (in parenthesises) of each covariates before demeaning.\\
& 2. & $P$ values in percentages in parenthesises. Standard errors are clustered at group (village) level.
%${}^{***}$, ${}^{**}$, ${}^{*}$ indicate statistical significance at 1\%, 5\%, 10\%, respetively. Standard errors are clustered at group (village) level.
\end{tabular}
\end{minipage}


\hspace{-1cm}\begin{minipage}[t]{14cm}
\hfil\textsc{\normalsize Table \refstepcounter{table}\thetable: ANCOVA estimation of assets by period and attributes\label{tab ANCOVA assets period attributes}}\\
\setlength{\tabcolsep}{1pt}
\setlength{\baselineskip}{8pt}
\renewcommand{\arraystretch}{.55}
\hfil\begin{tikzpicture}
\node (tbl) {\input{c:/data/GUK/analysis/save/EstimationMemo/AssetTimeVaryingAttributesANCOVAEstimationResults.tex}};
%\input{c:/dropbox/data/ramadan/save/tablecolortemplate.tex}
\end{tikzpicture}\\
\renewcommand{\arraystretch}{.8}
\setlength{\tabcolsep}{1pt}
\begin{tabular}{>{\hfill\scriptsize}p{1cm}<{}>{\hfill\scriptsize}p{.25cm}<{}>{\scriptsize}p{12cm}<{\hfill}}
Source:& \multicolumn{2}{l}{\scriptsize Estimated with GUK administrative and survey data.}\\
Notes: & 1. & ANCOVA estimates using administrative and survey data. Post treatment regressands are regressed on categorical variables, pre-treatment regressand and other covariates. Head age and literacy are from baseline survey data.  \textsf{LargeSize} is an indicator function if the arm is of large size, \textsf{WithGrace} is an indicator function if the arm is with a grace period, \textsf{InKind} is an indicator function if the arm provides a cow. Interaction terms of dummy variables are demeaned before interacting. The first column gives mean and standard deviation (in parenthesises) of each covariates before demeaning.\\
& 2. & $P$ values in percentages in parenthesises. Standard errors are clustered at group (village) level.
%${}^{***}$, ${}^{**}$, ${}^{*}$ indicate statistical significance at 1\%, 5\%, 10\%, respetively. Standard errors are clustered at group (village) level.
\end{tabular}
\end{minipage}

\hspace{-1cm}\begin{minipage}[t]{14cm}
\hfil\textsc{\normalsize Table \refstepcounter{table}\thetable: ANCOVA estimation of assets by period, arm, and poverty status\label{tab ANCOVA assets period poverty}}\\
\setlength{\tabcolsep}{1pt}
\setlength{\baselineskip}{8pt}
\renewcommand{\arraystretch}{.55}
\hfil\begin{tikzpicture}
\node (tbl) {\input{c:/data/GUK/analysis/save/EstimationMemo/AssetTimeVaryingPovertyStatusANCOVAEstimationResults.tex}};
%\input{c:/dropbox/data/ramadan/save/tablecolortemplate.tex}
\end{tikzpicture}\\
\renewcommand{\arraystretch}{.8}
\setlength{\tabcolsep}{1pt}
\begin{tabular}{>{\hfill\scriptsize}p{1cm}<{}>{\hfill\scriptsize}p{.25cm}<{}>{\scriptsize}p{12cm}<{\hfill}}
Source:& \multicolumn{2}{l}{\scriptsize Estimated with GUK administrative and survey data.}\\
Notes: & 1. & ANCOVA estimates using administrative and survey data. Post treatment regressands are regressed on categorical variables, pre-treatment regressand and other covariates. Head age and literacy are from baseline survey data.  \textsf{LargeSize} is an indicator function if the arm is of large size, \textsf{WithGrace} is an indicator function if the arm is with a grace period, \textsf{InKind} is an indicator function if the arm provides a cow. Interaction terms of dummy variables are demeaned before interacting. The first column gives mean and standard deviation (in parenthesises) of each covariates before demeaning.\\
& 2. & $P$ values in percentages in parenthesises. Standard errors are clustered at group (village) level.
%${}^{***}$, ${}^{**}$, ${}^{*}$ indicate statistical significance at 1\%, 5\%, 10\%, respetively. Standard errors are clustered at group (village) level.
\end{tabular}
\end{minipage}

\hspace{-1cm}\begin{minipage}[t]{14cm}
\hfil\textsc{\normalsize Table \refstepcounter{table}\thetable: ANCOVA estimation of assets by period, attributes, and poverty status\label{tab ANCOVA assets period poverty attributes}}\\
\setlength{\tabcolsep}{1pt}
\setlength{\baselineskip}{8pt}
\renewcommand{\arraystretch}{.55}
\hfil\begin{tikzpicture}
\node (tbl) {\input{c:/data/GUK/analysis/save/EstimationMemo/AssetTimeVaryingPovertyStatusAttributesANCOVAEstimationResults.tex}};
%\input{c:/dropbox/data/ramadan/save/tablecolortemplate.tex}
\end{tikzpicture}\\
\renewcommand{\arraystretch}{.8}
\setlength{\tabcolsep}{1pt}
\begin{tabular}{>{\hfill\scriptsize}p{1cm}<{}>{\hfill\scriptsize}p{.25cm}<{}>{\scriptsize}p{12cm}<{\hfill}}
Source:& \multicolumn{2}{l}{\scriptsize Estimated with GUK administrative and survey data.}\\
Notes: & 1. & ANCOVA estimates using administrative and survey data. Post treatment regressands are regressed on categorical variables, pre-treatment regressand and other covariates. Head age and literacy are from baseline survey data.  \textsf{LargeSize} is an indicator function if the arm is of large size, \textsf{WithGrace} is an indicator function if the arm is with a grace period, \textsf{InKind} is an indicator function if the arm provides a cow. Interaction terms of dummy variables are demeaned before interacting. The first column gives mean and standard deviation (in parenthesises) of each covariates before demeaning.\\
& 2. & $P$ values in percentages in parenthesises. Standard errors are clustered at group (village) level.
%${}^{***}$, ${}^{**}$, ${}^{*}$ indicate statistical significance at 1\%, 5\%, 10\%, respetively. Standard errors are clustered at group (village) level.
\end{tabular}
\end{minipage}


\hspace{-1cm}\begin{minipage}[t]{14cm}
\hfil\textsc{\normalsize Table \refstepcounter{table}\thetable: ANCOVA estimation of assets, loan recipients vs. pure control\label{tab ANCOVA assets pure control time varying}}\\
\setlength{\tabcolsep}{1pt}
\setlength{\baselineskip}{8pt}
\renewcommand{\arraystretch}{.55}
\hfil\begin{tikzpicture}
\node (tbl) {\input{c:/data/GUK/analysis/save/EstimationMemo/AssetRobustnessANCOVAEstimationResults.tex}};
%\input{c:/dropbox/data/ramadan/save/tablecolortemplate.tex}
\end{tikzpicture}\\
\renewcommand{\arraystretch}{.8}
\setlength{\tabcolsep}{1pt}
\begin{tabular}{>{\hfill\scriptsize}p{1cm}<{}>{\hfill\scriptsize}p{.25cm}<{}>{\scriptsize}p{12cm}<{\hfill}}
Source:& \multicolumn{2}{l}{\scriptsize Estimated with GUK administrative and survey data.}\\
Notes: & 1. & ANCOVA estimates. Pure control is members not receiving loans while they were put on a wait list. 
Sample is continuing members and replacing members of early rejecters. Household assets do not include livestock. Regressions (1)-(2), (4)-(5) use only arm and calendar information. (3) and (6) information if the household was exposed to the flood in round 1. Pure controls are households who rejected to receive a loan.\\
& 2. & $P$ values in percentages in parenthesises. Standard errors are clustered at group (village) level.
%${}^{***}$, ${}^{**}$, ${}^{*}$ indicate statistical significance at 1\%, 5\%, 10\%, respetively. Standard errors are clustered at group (village) level.
\end{tabular}
\end{minipage}

\hspace{-1cm}\begin{minipage}[t]{14cm}
\hfil\textsc{\normalsize Table \refstepcounter{table}\thetable: ANCOVA estimation of assets, loan recipients vs. pure control by attributes \label{tab ANCOVA assets pure control attributes}}\\
\setlength{\tabcolsep}{1pt}
\setlength{\baselineskip}{8pt}
\renewcommand{\arraystretch}{.55}
\hfil\begin{tikzpicture}
\node (tbl) {\input{c:/data/GUK/analysis/save/EstimationMemo/AssetRobustnessAttributesANCOVAEstimationResults.tex}};
%\input{c:/dropbox/data/ramadan/save/tablecolortemplate.tex}
\end{tikzpicture}\\
\renewcommand{\arraystretch}{.8}
\setlength{\tabcolsep}{1pt}
\begin{tabular}{>{\hfill\scriptsize}p{1cm}<{}>{\hfill\scriptsize}p{.25cm}<{}>{\scriptsize}p{12cm}<{\hfill}}
Source:& \multicolumn{2}{l}{\scriptsize Estimated with GUK administrative and survey data.}\\
Notes: & 1. & ANCOVA estimates. Pure control is members not receiving loans while they were put on a wait list. 
Sample is continuing members and replacing members of early rejecters. Household assets do not include livestock. Regressions (1)-(2), (4)-(5) use only arm and calendar information. (3) and (6) information if the household was exposed to the flood in round 1. Pure controls are households who rejected to receive a loan.\\
& 2. & $P$ values in percentages in parenthesises. Standard errors are clustered at group (village) level.
%${}^{***}$, ${}^{**}$, ${}^{*}$ indicate statistical significance at 1\%, 5\%, 10\%, respetively. Standard errors are clustered at group (village) level.
\end{tabular}
\end{minipage}

\hspace{-1cm}\begin{minipage}[t]{14cm}
\hfil\textsc{\normalsize Table \refstepcounter{table}\thetable: ANCOVA estimation of assets, loan recipients vs. pure control\label{tab ANCOVA assets pure control time varying attributes}}\\
\setlength{\tabcolsep}{1pt}
\setlength{\baselineskip}{8pt}
\renewcommand{\arraystretch}{.55}
\hfil\begin{tikzpicture}
\node (tbl) {\input{c:/data/GUK/analysis/save/EstimationMemo/AssetRobustnessTimeVaryingAttributesANCOVAEstimationResults.tex}};
%\input{c:/dropbox/data/ramadan/save/tablecolortemplate.tex}
\end{tikzpicture}\\
\renewcommand{\arraystretch}{.8}
\setlength{\tabcolsep}{1pt}
\begin{tabular}{>{\hfill\scriptsize}p{1cm}<{}>{\hfill\scriptsize}p{.25cm}<{}>{\scriptsize}p{12cm}<{\hfill}}
Source:& \multicolumn{2}{l}{\scriptsize Estimated with GUK administrative and survey data.}\\
Notes: & 1. & ANCOVA estimates. Pure control is members not receiving loans while they were put on a wait list. 
Sample is continuing members and replacing members of early rejecters. Household assets do not include livestock. Regressions (1)-(2), (4)-(5) use only arm and calendar information. (3) and (6) information if the household was exposed to the flood in round 1. Pure controls are households who rejected to receive a loan.\\
& 2. & $P$ values in percentages in parenthesises. Standard errors are clustered at group (village) level.
%${}^{***}$, ${}^{**}$, ${}^{*}$ indicate statistical significance at 1\%, 5\%, 10\%, respetively. Standard errors are clustered at group (village) level.
\end{tabular}
\end{minipage}

\subsection{Land}


\hspace{-1cm}\begin{minipage}[t]{14cm}
\hfil\textsc{\normalsize Table \refstepcounter{table}\thetable: ANCOVA estimation of land holding\label{tab ANCOVA land}}\\
\setlength{\tabcolsep}{1pt}
\setlength{\baselineskip}{8pt}
\renewcommand{\arraystretch}{.55}
\hfil\begin{tikzpicture}
\node (tbl) {\input{c:/data/GUK/analysis/save/EstimationMemo/LandANCOVAEstimationResults.tex}};
%\input{c:/dropbox/data/ramadan/save/tablecolortemplate.tex}
\end{tikzpicture}\\
\renewcommand{\arraystretch}{.8}
\setlength{\tabcolsep}{1pt}
\begin{tabular}{>{\hfill\scriptsize}p{1cm}<{}>{\hfill\scriptsize}p{.25cm}<{}>{\scriptsize}p{12cm}<{\hfill}}
Source:& \multicolumn{2}{l}{\scriptsize Estimated with GUK administrative and survey data.}\\
Notes: & 1. & ANCOVA estimates using administrative and survey data. Post treatment regressands are regressed on categorical variables, pre-treatment regressand and other covariates. Head age and literacy are from baseline survey data.  Interaction terms of dummy variables are demeaned before interacting. The first column gives mean and standard deviation (in parenthesises) of each covariates before demeaning.\\
& 2. & $P$ values in percentages in parenthesises. Standard errors are clustered at group (village) level.
%${}^{***}$, ${}^{**}$, ${}^{*}$ indicate statistical significance at 1\%, 5\%, 10\%, respetively. Standard errors are clustered at group (village) level.
\end{tabular}
\end{minipage}

\hspace{-1cm}\begin{minipage}[t]{14cm}
\hfil\textsc{\normalsize Table \refstepcounter{table}\thetable: ANCOVA estimation of land holding by attributes\label{tab ANCOVA land attributes}}\\
\setlength{\tabcolsep}{1pt}
\setlength{\baselineskip}{8pt}
\renewcommand{\arraystretch}{.55}
\hfil\begin{tikzpicture}
\node (tbl) {\input{c:/data/GUK/analysis/save/EstimationMemo/LandAttributesANCOVAEstimationResults.tex}};
%\input{c:/dropbox/data/ramadan/save/tablecolortemplate.tex}
\end{tikzpicture}\\
\renewcommand{\arraystretch}{.8}
\setlength{\tabcolsep}{1pt}
\begin{tabular}{>{\hfill\scriptsize}p{1cm}<{}>{\hfill\scriptsize}p{.25cm}<{}>{\scriptsize}p{12cm}<{\hfill}}
Source:& \multicolumn{2}{l}{\scriptsize Estimated with GUK administrative and survey data.}\\
Notes: & 1. & ANCOVA estimates using administrative and survey data. Post treatment regressands are regressed on categorical variables, pre-treatment regressand and other covariates. Head age and literacy are from baseline survey data.  Interaction terms of dummy variables are demeaned before interacting. The first column gives mean and standard deviation (in parenthesises) of each covariates before demeaning.\\
& 2. & $P$ values in percentages in parenthesises. Standard errors are clustered at group (village) level.
%${}^{***}$, ${}^{**}$, ${}^{*}$ indicate statistical significance at 1\%, 5\%, 10\%, respetively. Standard errors are clustered at group (village) level.
\end{tabular}
\end{minipage}

\hspace{-1cm}\begin{minipage}[t]{14cm}
\hfil\textsc{\normalsize Table \refstepcounter{table}\thetable: ANCOVA estimation of land holding by period, arm\label{tab ANCOVA land period}}\\
\setlength{\tabcolsep}{1pt}
\setlength{\baselineskip}{8pt}
\renewcommand{\arraystretch}{.55}
\hfil\begin{tikzpicture}
\node (tbl) {\input{c:/data/GUK/analysis/save/EstimationMemo/LandTimeVaryingANCOVAEstimationResults.tex}};
%\input{c:/dropbox/data/ramadan/save/tablecolortemplate.tex}
\end{tikzpicture}\\
\renewcommand{\arraystretch}{.8}
\setlength{\tabcolsep}{1pt}
\begin{tabular}{>{\hfill\scriptsize}p{1cm}<{}>{\hfill\scriptsize}p{.25cm}<{}>{\scriptsize}p{12cm}<{\hfill}}
Source:& \multicolumn{2}{l}{\scriptsize Estimated with GUK administrative and survey data.}\\
Notes: & 1. & ANCOVA estimates using administrative and survey data. Post treatment regressands are regressed on categorical variables, pre-treatment regressand and other covariates. Head age and literacy are from baseline survey data.  \textsf{LargeSize} is an indicator function if the arm is of large size, \textsf{WithGrace} is an indicator function if the arm is with a grace period, \textsf{InKind} is an indicator function if the arm provides a cow. Interaction terms of dummy variables are demeaned before interacting. The first column gives mean and standard deviation (in parenthesises) of each covariates before demeaning.\\
& 2. & $P$ values in percentages in parenthesises. Standard errors are clustered at group (village) level.
%${}^{***}$, ${}^{**}$, ${}^{*}$ indicate statistical significance at 1\%, 5\%, 10\%, respetively. Standard errors are clustered at group (village) level.
\end{tabular}
\end{minipage}

\hspace{-1cm}\begin{minipage}[t]{14cm}
\hfil\textsc{\normalsize Table \refstepcounter{table}\thetable: ANCOVA estimation of land holding by period, arm, and poverty status\label{tab ANCOVA land period poverty}}\\
\setlength{\tabcolsep}{1pt}
\setlength{\baselineskip}{8pt}
\renewcommand{\arraystretch}{.55}
\hfil\begin{tikzpicture}
\node (tbl) {\input{c:/data/GUK/analysis/save/EstimationMemo/LandTimeVaryingPovertyStatusANCOVAEstimationResults.tex}};
%\input{c:/dropbox/data/ramadan/save/tablecolortemplate.tex}
\end{tikzpicture}\\
\renewcommand{\arraystretch}{.8}
\setlength{\tabcolsep}{1pt}
\begin{tabular}{>{\hfill\scriptsize}p{1cm}<{}>{\hfill\scriptsize}p{.25cm}<{}>{\scriptsize}p{12cm}<{\hfill}}
Source:& \multicolumn{2}{l}{\scriptsize Estimated with GUK administrative and survey data.}\\
Notes: & 1. & ANCOVA estimates using administrative and survey data. Post treatment regressands are regressed on categorical variables, pre-treatment regressand and other covariates. Head age and literacy are from baseline survey data.  \textsf{LargeSize} is an indicator function if the arm is of large size, \textsf{WithGrace} is an indicator function if the arm is with a grace period, \textsf{InKind} is an indicator function if the arm provides a cow. Interaction terms of dummy variables are demeaned before interacting. The first column gives mean and standard deviation (in parenthesises) of each covariates before demeaning.\\
& 2. & $P$ values in percentages in parenthesises. Standard errors are clustered at group (village) level.
%${}^{***}$, ${}^{**}$, ${}^{*}$ indicate statistical significance at 1\%, 5\%, 10\%, respetively. Standard errors are clustered at group (village) level.
\end{tabular}
\end{minipage}

\hspace{-1cm}\begin{minipage}[t]{14cm}
\hfil\textsc{\normalsize Table \refstepcounter{table}\thetable: ANCOVA estimation of land holding by period and attributes\label{tab ANCOVA land period attributes}}\\
\setlength{\tabcolsep}{1pt}
\setlength{\baselineskip}{8pt}
\renewcommand{\arraystretch}{.55}
\hfil\begin{tikzpicture}
\node (tbl) {\input{c:/data/GUK/analysis/save/EstimationMemo/LandTimeVaryingAttributesANCOVAEstimationResults.tex}};
%\input{c:/dropbox/data/ramadan/save/tablecolortemplate.tex}
\end{tikzpicture}\\
\renewcommand{\arraystretch}{.8}
\setlength{\tabcolsep}{1pt}
\begin{tabular}{>{\hfill\scriptsize}p{1cm}<{}>{\hfill\scriptsize}p{.25cm}<{}>{\scriptsize}p{12cm}<{\hfill}}
Source:& \multicolumn{2}{l}{\scriptsize Estimated with GUK administrative and survey data.}\\
Notes: & 1. & ANCOVA estimates using administrative and survey data. Post treatment regressands are regressed on categorical variables, pre-treatment regressand and other covariates. Head age and literacy are from baseline survey data.  \textsf{LargeSize} is an indicator function if the arm is of large size, \textsf{WithGrace} is an indicator function if the arm is with a grace period, \textsf{InKind} is an indicator function if the arm provides a cow. Interaction terms of dummy variables are demeaned before interacting. The first column gives mean and standard deviation (in parenthesises) of each covariates before demeaning.\\
& 2. & $P$ values in percentages in parenthesises. Standard errors are clustered at group (village) level.
%${}^{***}$, ${}^{**}$, ${}^{*}$ indicate statistical significance at 1\%, 5\%, 10\%, respetively. Standard errors are clustered at group (village) level.
\end{tabular}
\end{minipage}

\hspace{-1cm}\begin{minipage}[t]{14cm}
\hfil\textsc{\normalsize Table \refstepcounter{table}\thetable: ANCOVA estimation of land holding by period, attributes, and poverty status\label{tab ANCOVA land period poverty attributes}}\\
\setlength{\tabcolsep}{1pt}
\setlength{\baselineskip}{8pt}
\renewcommand{\arraystretch}{.55}
\hfil\begin{tikzpicture}
\node (tbl) {\input{c:/data/GUK/analysis/save/EstimationMemo/LandTimeVaryingPovertyStatusAttributesANCOVAEstimationResults.tex}};
%\input{c:/dropbox/data/ramadan/save/tablecolortemplate.tex}
\end{tikzpicture}\\
\renewcommand{\arraystretch}{.8}
\setlength{\tabcolsep}{1pt}
\begin{tabular}{>{\hfill\scriptsize}p{1cm}<{}>{\hfill\scriptsize}p{.25cm}<{}>{\scriptsize}p{12cm}<{\hfill}}
Source:& \multicolumn{2}{l}{\scriptsize Estimated with GUK administrative and survey data.}\\
Notes: & 1. & ANCOVA estimates using administrative and survey data. Post treatment regressands are regressed on categorical variables, pre-treatment regressand and other covariates. Head age and literacy are from baseline survey data.  \textsf{LargeSize} is an indicator function if the arm is of large size, \textsf{WithGrace} is an indicator function if the arm is with a grace period, \textsf{InKind} is an indicator function if the arm provides a cow. Interaction terms of dummy variables are demeaned before interacting. The first column gives mean and standard deviation (in parenthesises) of each covariates before demeaning.\\
& 2. & $P$ values in percentages in parenthesises. Standard errors are clustered at group (village) level.
%${}^{***}$, ${}^{**}$, ${}^{*}$ indicate statistical significance at 1\%, 5\%, 10\%, respetively. Standard errors are clustered at group (village) level.
\end{tabular}
\end{minipage}

\subsection{Livestock}


\hspace{-1cm}\begin{minipage}[t]{14cm}
\hfil\textsc{\normalsize Table \refstepcounter{table}\thetable: ANCOVA estimation of livestock holding values\label{tab ANCOVA livestock}}\\
\setlength{\tabcolsep}{1pt}
\setlength{\baselineskip}{8pt}
\renewcommand{\arraystretch}{.55}
\hfil\begin{tikzpicture}
\node (tbl) {\input{c:/data/GUK/analysis/save/EstimationMemo/LivestockANCOVAEstimationResults.tex}};
%\input{c:/dropbox/data/ramadan/save/tablecolortemplate.tex}
\end{tikzpicture}\\
\renewcommand{\arraystretch}{.8}
\setlength{\tabcolsep}{1pt}
\begin{tabular}{>{\hfill\scriptsize}p{1cm}<{}>{\hfill\scriptsize}p{.25cm}<{}>{\scriptsize}p{12cm}<{\hfill}}
Source:& \multicolumn{2}{l}{\scriptsize Estimated with GUK administrative and survey data.}\\
Notes: & 1. & ANCOVA estimates using administrative and survey data. Post treatment regressands are regressed on categorical variables, pre-treatment regressand and other covariates. Head age and literacy are from baseline survey data.  Saving and repayment information is taken from administrative data. Time invariant household characteristics are taken from household survey data. Administrative data are merged with survey data by the dating the survey rounds in administrative data. Net saving is saving - withdrawal. Excess repayment is repayment - due amount. \textsf{LY2, LY3, LY4} are dummy variables for second, third, and 	fourth year into borrowing. Sample is continuing members and replacing members of early rejecters and received loans prior to 2015 Janunary. Regressand is \textsf{TotalImputedValue}, a sum of all livestock holding values evaluated at respective median market prices in the same year. \\
& 2. & $P$ values in percentages in parenthesises. Standard errors are clustered at group (village) level.%
%${}^{***}$, ${}^{**}$, ${}^{*}$ indicate statistical significance at 1\%, 5\%, 10\%, respetively. 
$P$ values in parenthesises. Standard errors are clustered at group (village) level.
\end{tabular}
\end{minipage}

\hspace{-1cm}\begin{minipage}[t]{14cm}
\hfil\textsc{\normalsize Table \refstepcounter{table}\thetable: ANCOVA estimation of livestock holding values by attributes\label{tab ANCOVA livestock attributes}}\\
\setlength{\tabcolsep}{1pt}
\setlength{\baselineskip}{8pt}
\renewcommand{\arraystretch}{.55}
\hfil\begin{tikzpicture}
\node (tbl) {\input{c:/data/GUK/analysis/save/EstimationMemo/LivestockAttributesANCOVAEstimationResults.tex}};
%\input{c:/dropbox/data/ramadan/save/tablecolortemplate.tex}
\end{tikzpicture}\\
\renewcommand{\arraystretch}{.8}
\setlength{\tabcolsep}{1pt}
\begin{tabular}{>{\hfill\scriptsize}p{1cm}<{}>{\hfill\scriptsize}p{.25cm}<{}>{\scriptsize}p{12cm}<{\hfill}}
Source:& \multicolumn{2}{l}{\scriptsize Estimated with GUK administrative and survey data.}\\
Notes: & 1. & ANCOVA estimates using administrative and survey data. Post treatment regressands are regressed on categorical variables, pre-treatment regressand and other covariates. Head age and literacy are from baseline survey data.  \textsf{LargeSize} is an indicator function if the arm is of large size, \textsf{WithGrace} is an indicator function if the arm is with a grace period, \textsf{InKind} is an indicator function if the arm provides a cow. Saving and repayment information is taken from administrative data. Time invariant household characteristics are taken from household survey data. Administrative data are merged with survey data by the dating the survey rounds in administrative data. Net saving is saving - withdrawal. Excess repayment is repayment - due amount. \textsf{LY2, LY3, LY4} are dummy variables for second, third, and 	fourth year into borrowing. Sample is continuing members and replacing members of early rejecters and received loans prior to 2015 Janunary. Regressand is \textsf{TotalImputedValue}, a sum of all livestock holding values evaluated at respective median market prices in the same year. \\
& 2. & $P$ values in percentages in parenthesises. Standard errors are clustered at group (village) level.
%${}^{***}$, ${}^{**}$, ${}^{*}$ indicate statistical significance at 1\%, 5\%, 10\%, respetively. Standard errors are clustered at group (village) level.
\end{tabular}
\end{minipage}



\hspace{-1cm}\begin{minipage}[t]{14cm}
\hfil\textsc{\normalsize Table \refstepcounter{table}\thetable: ANCOVA estimation of livestock holding values, ultra vs. moderately poor\label{tab ANCOVA livestock poor}}\\
\setlength{\tabcolsep}{1pt}
\setlength{\baselineskip}{8pt}
\renewcommand{\arraystretch}{.55}
\hfil\begin{tikzpicture}
\node (tbl) {\input{c:/data/GUK/analysis/save/EstimationMemo/LivestockPovertyStatusANCOVAEstimationResults.tex}};
%\input{c:/dropbox/data/ramadan/save/tablecolortemplate.tex}
\end{tikzpicture}\\
\renewcommand{\arraystretch}{.8}
\setlength{\tabcolsep}{1pt}
\begin{tabular}{>{\hfill\scriptsize}p{1cm}<{}>{\hfill\scriptsize}p{.25cm}<{}>{\scriptsize}p{12cm}<{\hfill}}
Source:& \multicolumn{2}{l}{\scriptsize Estimated with GUK administrative and survey data.}\\
Notes: & 1. & ANCOVA estimates using administrative and survey data. Post treatment regressands are regressed on categorical variables, pre-treatment regressand and other covariates. Head age and literacy are from baseline survey data.  \textsf{UltraPoor} is an indicator function if the household is classified as the ultra poor. Sample is continuing members and replacing members of early rejecters and received loans prior to 2015 Janunary. Regressand is \textsf{TotalImputedValue}, a sum of all livestock holding values evaluated at respective median market prices in the same year. \\
& 2. & $P$ values in percentages in parenthesises. Standard errors are clustered at group (village) level.
%${}^{***}$, ${}^{**}$, ${}^{*}$ indicate statistical significance at 1\%, 5\%, 10\%, respetively. Standard errors are clustered at group (village) level.
\end{tabular}
\end{minipage}

\hspace{-1cm}\begin{minipage}[t]{14cm}
\hfil\textsc{\normalsize Table \refstepcounter{table}\thetable: ANCOVA estimation of livestock holding values by attributes and period\label{tab ANCOVA livestock timevarying attributes}}\\
\setlength{\tabcolsep}{1pt}
\setlength{\baselineskip}{8pt}
\renewcommand{\arraystretch}{.55}
\hfil\begin{tikzpicture}
\node (tbl) {\input{c:/data/GUK/analysis/save/EstimationMemo/LivestockTimeVaryingAttributesANCOVAEstimationResults.tex}};
%\input{c:/dropbox/data/ramadan/save/tablecolortemplate.tex}
\end{tikzpicture}\\
\renewcommand{\arraystretch}{.8}
\setlength{\tabcolsep}{1pt}
\begin{tabular}{>{\hfill\scriptsize}p{1cm}<{}>{\hfill\scriptsize}p{.25cm}<{}>{\scriptsize}p{12cm}<{\hfill}}
Source:& \multicolumn{2}{l}{\scriptsize Estimated with GUK administrative and survey data.}\\
Notes: & 1. & ANCOVA estimates using administrative and survey data. Post treatment regressands are regressed on categorical variables, pre-treatment regressand and other covariates. Head age and literacy are from baseline survey data.  \textsf{LargeSize} is an indicator function if the arm is of large size, \textsf{WithGrace} is an indicator function if the arm is with a grace period, \textsf{InKind} is an indicator function if the arm provides a cow. Saving and repayment information is taken from administrative data. Time invariant household characteristics are taken from household survey data. Administrative data are merged with survey data by the dating the survey rounds in administrative data. Net saving is saving - withdrawal. Excess repayment is repayment - due amount. \textsf{LY2, LY3, LY4} are dummy variables for second, third, and 	fourth year into borrowing. Sample is continuing members and replacing members of early rejecters and received loans prior to 2015 Janunary. Regressand is \textsf{TotalImputedValue}, a sum of all livestock holding values evaluated at respective median market prices in the same year. \\
& 2. & $P$ values in percentages in parenthesises. Standard errors are clustered at group (village) level.
%${}^{***}$, ${}^{**}$, ${}^{*}$ indicate statistical significance at 1\%, 5\%, 10\%, respetively. Standard errors are clustered at group (village) level.
\end{tabular}
\end{minipage}


\subsection{Cattle holding}


\hspace{-1cm}\begin{minipage}[t]{14cm}
\hfil\textsc{\normalsize Table \refstepcounter{table}\thetable: ANCOVA estimation of cattle holding by attributes\label{tab ANCOVA cow attributes}}\\
\setlength{\tabcolsep}{1pt}
\setlength{\baselineskip}{8pt}
\renewcommand{\arraystretch}{.55}
\hfil\begin{tikzpicture}
\node (tbl) {\input{c:/data/GUK/analysis/save/EstimationMemo/NumCowsAttributesANCOVAEstimationResults.tex}};
%\input{c:/dropbox/data/ramadan/save/tablecolortemplate.tex}
\end{tikzpicture}\\
\renewcommand{\arraystretch}{.8}
\setlength{\tabcolsep}{1pt}
\begin{tabular}{>{\hfill\scriptsize}p{1cm}<{}>{\hfill\scriptsize}p{.25cm}<{}>{\scriptsize}p{12cm}<{\hfill}}
Source:& \multicolumn{2}{l}{\scriptsize Estimated with GUK administrative and survey data.}\\
Notes: & 1. & ANCOVA estimates using administrative and survey data. Post treatment regressands are regressed on categorical variables, pre-treatment regressand and other covariates. Head age and literacy are from baseline survey data.  \textsf{LargeSize} is an indicator function if the arm is of large size, \textsf{WithGrace} is an indicator function if the arm is with a grace period, \textsf{InKind} is an indicator function if the arm provides a cow. Saving and repayment information is taken from administrative data. Time invariant household characteristics are taken from household survey data. Administrative data are merged with survey data by the dating the survey rounds in administrative data. Net saving is saving - withdrawal. Excess repayment is repayment - due amount. \textsf{LY2, LY3, LY4} are dummy variables for second, third, and 	fourth year into borrowing. Sample is continuing members and replacing members of early rejecters and received loans prior to 2015 Janunary. Regressand is \textsf{NumCows}, number of cattle holding. \\
& 2. & $P$ values in percentages in parenthesises. Standard errors are clustered at group (village) level.
%${}^{***}$, ${}^{**}$, ${}^{*}$ indicate statistical significance at 1\%, 5\%, 10\%, respetively. Standard errors are clustered at group (village) level.
\end{tabular}
\end{minipage}

\hspace{-1cm}\begin{minipage}[t]{14cm}
\hfil\textsc{\normalsize Table \refstepcounter{table}\thetable: ANCOVA estimation of cattle holding, ultra vs. moderately poor\label{tab ANCOVA NumCows poor}}\\
\setlength{\tabcolsep}{1pt}
\setlength{\baselineskip}{8pt}
\renewcommand{\arraystretch}{.55}
\hfil\begin{tikzpicture}
\node (tbl) {\input{c:/data/GUK/analysis/save/EstimationMemo/NumCowsPovertyStatusANCOVAEstimationResults.tex}};
\end{tikzpicture}\\
\renewcommand{\arraystretch}{.8}
\setlength{\tabcolsep}{1pt}
\begin{tabular}{>{\hfill\scriptsize}p{1cm}<{}>{\hfill\scriptsize}p{.25cm}<{}>{\scriptsize}p{12cm}<{\hfill}}
Source:& \multicolumn{2}{l}{\scriptsize Estimated with GUK administrative and survey data.}\\
Notes: & 1. & ANCOVA estimates using administrative and survey data. Post treatment regressands are regressed on categorical variables, pre-treatment regressand and other covariates. Head age and literacy are from baseline survey data.  \textsf{LargeSize} is an indicator function if the arm is of large size, \textsf{WithGrace} is an indicator function if the arm is with a grace period, \textsf{InKind} is an indicator function if the arm provides a cow. Saving and repayment information is taken from administrative data. Time invariant household characteristics are taken from household survey data. Administrative data are merged with survey data by the dating the survey rounds in administrative data. Net saving is saving - withdrawal. Excess repayment is repayment - due amount. \textsf{LY2, LY3, LY4} are dummy variables for second, third, and 	fourth year into borrowing. Sample is continuing members and replacing members of early rejecters and received loans prior to 2015 Janunary. Regressand is \textsf{NumCows}, number of cattle holding. \\
& 2. & $P$ values in percentages in parenthesises. Standard errors are clustered at group (village) level.
%${}^{***}$, ${}^{**}$, ${}^{*}$ indicate statistical significance at 1\%, 5\%, 10\%, respetively. Standard errors are clustered at group (village) level.
\end{tabular}
\end{minipage}

\hspace{-1cm}\begin{minipage}[t]{14cm}
\hfil\textsc{\normalsize Table \refstepcounter{table}\thetable: ANCOVA estimation of cattle holding by arm and period\label{tab ANCOVA cow time varying}}\\
\setlength{\tabcolsep}{1pt}
\setlength{\baselineskip}{8pt}
\renewcommand{\arraystretch}{.55}
\hfil\begin{tikzpicture}
\node (tbl) {\input{c:/data/GUK/analysis/save/EstimationMemo/NumCowsTimeVaryingANCOVAEstimationResults.tex}};
\end{tikzpicture}\\
\renewcommand{\arraystretch}{.8}
\setlength{\tabcolsep}{1pt}
\begin{tabular}{>{\hfill\scriptsize}p{1cm}<{}>{\hfill\scriptsize}p{.25cm}<{}>{\scriptsize}p{12cm}<{\hfill}}
Source:& \multicolumn{2}{l}{\scriptsize Estimated with GUK administrative and survey data.}\\
Notes: & 1. & ANCOVA estimates using administrative and survey data. Post treatment regressands are regressed on categorical variables, pre-treatment regressand and other covariates. Head age and literacy are from baseline survey data.  \textsf{LargeSize} is an indicator function if the arm is of large size, \textsf{WithGrace} is an indicator function if the arm is with a grace period, \textsf{InKind} is an indicator function if the arm provides a cow. Saving and repayment information is taken from administrative data. Time invariant household characteristics are taken from household survey data. Administrative data are merged with survey data by the dating the survey rounds in administrative data. Net saving is saving - withdrawal. Excess repayment is repayment - due amount. \textsf{LY2, LY3, LY4} are dummy variables for second, third, and 	fourth year into borrowing. Sample is continuing members and replacing members of early rejecters and received loans prior to 2015 Janunary. Regressand is \textsf{NumCows}, number of cattle holding. \\
& 2. & $P$ values in percentages in parenthesises. Standard errors are clustered at group (village) level.
%${}^{***}$, ${}^{**}$, ${}^{*}$ indicate statistical significance at 1\%, 5\%, 10\%, respetively. Standard errors are clustered at group (village) level.
\end{tabular}
\end{minipage}

\hspace{-1cm}\begin{minipage}[t]{14cm}
\hfil\textsc{\normalsize Table \refstepcounter{table}\thetable: ANCOVA estimation of cattle holding by attributes and period\label{tab ANCOVA cow time varying attributes}}\\
\setlength{\tabcolsep}{1pt}
\setlength{\baselineskip}{8pt}
\renewcommand{\arraystretch}{.55}
\hfil\begin{tikzpicture}
\node (tbl) {\input{c:/data/GUK/analysis/save/EstimationMemo/NumCowsTimeVaryingAttributesANCOVAEstimationResults.tex}};
\end{tikzpicture}\\
\renewcommand{\arraystretch}{.8}
\setlength{\tabcolsep}{1pt}
\begin{tabular}{>{\hfill\scriptsize}p{1cm}<{}>{\hfill\scriptsize}p{.25cm}<{}>{\scriptsize}p{12cm}<{\hfill}}
Source:& \multicolumn{2}{l}{\scriptsize Estimated with GUK administrative and survey data.}\\
Notes: & 1. & ANCOVA estimates using administrative and survey data. Post treatment regressands are regressed on categorical variables, pre-treatment regressand and other covariates. Head age and literacy are from baseline survey data.  \textsf{LargeSize} is an indicator function if the arm is of large size, \textsf{WithGrace} is an indicator function if the arm is with a grace period, \textsf{InKind} is an indicator function if the arm provides a cow. Saving and repayment information is taken from administrative data. Time invariant household characteristics are taken from household survey data. Administrative data are merged with survey data by the dating the survey rounds in administrative data. Net saving is saving - withdrawal. Excess repayment is repayment - due amount. \textsf{LY2, LY3, LY4} are dummy variables for second, third, and 	fourth year into borrowing. Sample is continuing members and replacing members of early rejecters and received loans prior to 2015 Janunary. Regressand is \textsf{NumCows}, number of cattle holding. \\
& 2. & $P$ values in percentages in parenthesises. Standard errors are clustered at group (village) level.
%${}^{***}$, ${}^{**}$, ${}^{*}$ indicate statistical significance at 1\%, 5\%, 10\%, respetively. Standard errors are clustered at group (village) level.
\end{tabular}
\end{minipage}

\subsection{Net assets}


\hspace{-1cm}\begin{minipage}[t]{14cm}
\hfil\textsc{\normalsize Table \refstepcounter{table}\thetable: ANCOVA estimation of net assets\label{tab ANCOVA net assets}}\\
\setlength{\tabcolsep}{1pt}
\setlength{\baselineskip}{8pt}
\renewcommand{\arraystretch}{.55}
\hfil\begin{tikzpicture}
\node (tbl) {\input{c:/data/GUK/analysis/save/EstimationMemo/NetAssetANCOVAEstimationResults.tex}};
%\input{c:/dropbox/data/ramadan/save/tablecolortemplate.tex}
\end{tikzpicture}\\
\renewcommand{\arraystretch}{.8}
\setlength{\tabcolsep}{1pt}
\begin{tabular}{>{\hfill\scriptsize}p{1cm}<{}>{\hfill\scriptsize}p{.25cm}<{}>{\scriptsize}p{12cm}<{\hfill}}
Source:& \multicolumn{2}{l}{\scriptsize Estimated with GUK administrative and survey data.}\\
Notes: & 1. & ANCOVA estimates using administrative and survey data. Post treatment regressands are regressed on categorical variables, pre-treatment regressand and other covariates. Head age and literacy are from baseline survey data.  Sample is continuing members and replacing members of early rejecters and received loans prior to 2015 Janunary. Household assets do not include livestock. Regressions (1)-(3), (5)-(6) use only arm and calendar information. (4) and (7) use previous six month repayment and saving information which is lacking in rd 1, hence starts from rd 2.\\
& 2. & $P$ values in percentages in parenthesises. Standard errors are clustered at group (village) level.
%${}^{***}$, ${}^{**}$, ${}^{*}$ indicate statistical significance at 1\%, 5\%, 10\%, respetively. Standard errors are clustered at group (village) level.
\end{tabular}
\end{minipage}

\hspace{-1cm}\begin{minipage}[t]{14cm}
\hfil\textsc{\normalsize Table \refstepcounter{table}\thetable: ANCOVA estimation of net assets by attributes\label{tab ANCOVA net assets attributes}}\\
\setlength{\tabcolsep}{1pt}
\setlength{\baselineskip}{8pt}
\renewcommand{\arraystretch}{.55}
\hfil\begin{tikzpicture}
\node (tbl) {\input{c:/data/GUK/analysis/save/EstimationMemo/NetAssetAttributesANCOVAEstimationResults.tex}};
%\input{c:/dropbox/data/ramadan/save/tablecolortemplate.tex}
\end{tikzpicture}\\
\renewcommand{\arraystretch}{.8}
\setlength{\tabcolsep}{1pt}
\begin{tabular}{>{\hfill\scriptsize}p{1cm}<{}>{\hfill\scriptsize}p{.25cm}<{}>{\scriptsize}p{12cm}<{\hfill}}
Source:& \multicolumn{2}{l}{\scriptsize Estimated with GUK administrative and survey data.}\\
Notes: & 1. & ANCOVA estimates using administrative and survey data. Post treatment regressands are regressed on categorical variables, pre-treatment regressand and other covariates. Head age and literacy are from baseline survey data.  \textsf{LargeSize} is an indicator function if the arm is of large size, \textsf{WithGrace} is an indicator function if the arm is with a grace period, \textsf{InKind} is an indicator function if the arm provides a cow. Sample is continuing members and replacing members of early rejecters and received loans prior to 2015 Janunary. Household assets do not include livestock. Regressions (1)-(3), (5)-(6) use only arm and calendar information. (4) and (7) use previous six month repayment and saving information which is lacking in rd 1, hence starts from rd 2.\\
& 2. & $P$ values in percentages in parenthesises. Standard errors are clustered at group (village) level.
%${}^{***}$, ${}^{**}$, ${}^{*}$ indicate statistical significance at 1\%, 5\%, 10\%, respetively. Standard errors are clustered at group (village) level.
\end{tabular}
\end{minipage}

\hspace{-1cm}\begin{minipage}[t]{14cm}
\hfil\textsc{\normalsize Table \refstepcounter{table}\thetable: ANCOVA estimation of net assets by period\label{tab ANCOVA net assets timevarying}}\\
\setlength{\tabcolsep}{1pt}
\setlength{\baselineskip}{8pt}
\renewcommand{\arraystretch}{.55}
\hfil\begin{tikzpicture}
\node (tbl) {\input{c:/data/GUK/analysis/save/EstimationMemo/NetAssetTimeVaryingANCOVAEstimationResults.tex}};
%\input{c:/dropbox/data/ramadan/save/tablecolortemplate.tex}
\end{tikzpicture}\\
\renewcommand{\arraystretch}{.8}
\setlength{\tabcolsep}{1pt}
\begin{tabular}{>{\hfill\scriptsize}p{1cm}<{}>{\hfill\scriptsize}p{.25cm}<{}>{\scriptsize}p{12cm}<{\hfill}}
Source:& \multicolumn{2}{l}{\scriptsize Estimated with GUK administrative and survey data.}\\
Notes: & 1. & First-difference estimates between round 2 and 4. A first-difference is defined as $\Delta x_{t+k}\equiv x_{t+k} - x_{t}$ for $k=1, 2, \dots$. Saving and repayment misses are taken from administrative data and merged with survey data at Year-Month of survey interviews. Intercept terms are omitted in estimating equations. Sample is continuing members and replacing members of early rejecters and received loans prior to 2015 Janunary. Household assets do not include livestock. Regressions (1)-(3), (5)-(6) use only arm and calendar information. (4) and (7) use previous six month repayment and saving information which is lacking in rd 1, hence starts from rd 2.\\
& 2. & $P$ values in percentages in parenthesises. Standard errors are clustered at group (village) level.
%${}^{***}$, ${}^{**}$, ${}^{*}$ indicate statistical significance at 1\%, 5\%, 10\%, respetively. Standard errors are clustered at group (village) level.
\end{tabular}
\end{minipage}

\hspace{-1cm}\begin{minipage}[t]{14cm}
\hfil\textsc{\normalsize Table \refstepcounter{table}\thetable: ANCOVA estimation of net assets by attrbutes and period\label{tab ANCOVA net assets timevarying attributes}}\\
\setlength{\tabcolsep}{1pt}
\setlength{\baselineskip}{8pt}
\renewcommand{\arraystretch}{.55}
\hfil\begin{tikzpicture}
\node (tbl) {\input{c:/data/GUK/analysis/save/EstimationMemo/NetAssetTimeVaryingAttributesANCOVAEstimationResults.tex}};
%\input{c:/dropbox/data/ramadan/save/tablecolortemplate.tex}
\end{tikzpicture}\\
\renewcommand{\arraystretch}{.8}
\setlength{\tabcolsep}{1pt}
\begin{tabular}{>{\hfill\scriptsize}p{1cm}<{}>{\hfill\scriptsize}p{.25cm}<{}>{\scriptsize}p{12cm}<{\hfill}}
Source:& \multicolumn{2}{l}{\scriptsize Estimated with GUK administrative and survey data.}\\
Notes: & 1. & First-difference estimates between round 2 and 4. A first-difference is defined as $\Delta x_{t+k}\equiv x_{t+k} - x_{t}$ for $k=1, 2, \dots$. Saving and repayment misses are taken from administrative data and merged with survey data at Year-Month of survey interviews. Intercept terms are omitted in estimating equations. Sample is continuing members and replacing members of early rejecters and received loans prior to 2015 Janunary. Household assets do not include livestock. Regressions (1)-(3), (5)-(6) use only arm and calendar information. (4) and (7) use previous six month repayment and saving information which is lacking in rd 1, hence starts from rd 2.\\
& 2. & $P$ values in percentages in parenthesises. Standard errors are clustered at group (village) level.
%${}^{***}$, ${}^{**}$, ${}^{*}$ indicate statistical significance at 1\%, 5\%, 10\%, respetively. Standard errors are clustered at group (village) level.
\end{tabular}
\end{minipage}


\hspace{-1cm}\begin{minipage}[t]{14cm}
\hfil\textsc{\normalsize Table \refstepcounter{table}\thetable: ANCOVA estimation of net assets by arm, poverty status, and period\label{tab ANCOVA net assets timevarying poverty status}}\\
\setlength{\tabcolsep}{1pt}
\setlength{\baselineskip}{8pt}
\renewcommand{\arraystretch}{.55}
\hfil\begin{tikzpicture}
\node (tbl) {\input{c:/data/GUK/analysis/save/EstimationMemo/NetAssetTimeVaryingPovertyStatusANCOVAEstimationResults.tex}};
%\input{c:/dropbox/data/ramadan/save/tablecolortemplate.tex}
\end{tikzpicture}\\
\renewcommand{\arraystretch}{.8}
\setlength{\tabcolsep}{1pt}
\begin{tabular}{>{\hfill\scriptsize}p{1cm}<{}>{\hfill\scriptsize}p{.25cm}<{}>{\scriptsize}p{12cm}<{\hfill}}
Source:& \multicolumn{2}{l}{\scriptsize Estimated with GUK administrative and survey data.}\\
Notes: & 1. & First-difference estimates between round 2 and 4. A first-difference is defined as $\Delta x_{t+k}\equiv x_{t+k} - x_{t}$ for $k=1, 2, \dots$. Saving and repayment misses are taken from administrative data and merged with survey data at Year-Month of survey interviews. Intercept terms are omitted in estimating equations. Sample is continuing members and replacing members of early rejecters and received loans prior to 2015 Janunary. Household assets do not include livestock. Regressions (1)-(3), (5)-(6) use only arm and calendar information. (4) and (7) use previous six month repayment and saving information which is lacking in rd 1, hence starts from rd 2.\\
& 2. & $P$ values in percentages in parenthesises. Standard errors are clustered at group (village) level.
%${}^{***}$, ${}^{**}$, ${}^{*}$ indicate statistical significance at 1\%, 5\%, 10\%, respetively. Standard errors are clustered at group (village) level.
\end{tabular}
\end{minipage}

\hspace{-1cm}\begin{minipage}[t]{14cm}
\hfil\textsc{\normalsize Table \refstepcounter{table}\thetable: ANCOVA estimation of net assets by attrbutes, poverty status, and period\label{tab ANCOVA net assets timevarying poverty status attributes}}\\
\setlength{\tabcolsep}{1pt}
\setlength{\baselineskip}{8pt}
\renewcommand{\arraystretch}{.55}
\hfil\begin{tikzpicture}
\node (tbl) {\input{c:/data/GUK/analysis/save/EstimationMemo/NetAssetTimeVaryingPovertyStatusAttributesANCOVAEstimationResults.tex}};
%\input{c:/dropbox/data/ramadan/save/tablecolortemplate.tex}
\end{tikzpicture}\\
\renewcommand{\arraystretch}{.8}
\setlength{\tabcolsep}{1pt}
\begin{tabular}{>{\hfill\scriptsize}p{1cm}<{}>{\hfill\scriptsize}p{.25cm}<{}>{\scriptsize}p{12cm}<{\hfill}}
Source:& \multicolumn{2}{l}{\scriptsize Estimated with GUK administrative and survey data.}\\
Notes: & 1. & First-difference estimates between round 2 and 4. A first-difference is defined as $\Delta x_{t+k}\equiv x_{t+k} - x_{t}$ for $k=1, 2, \dots$. Saving and repayment misses are taken from administrative data and merged with survey data at Year-Month of survey interviews. Intercept terms are omitted in estimating equations. Sample is continuing members and replacing members of early rejecters and received loans prior to 2015 Janunary. Household assets do not include livestock. Regressions (1)-(3), (5)-(6) use only arm and calendar information. (4) and (7) use previous six month repayment and saving information which is lacking in rd 1, hence starts from rd 2.\\
& 2. & $P$ values in percentages in parenthesises. Standard errors are clustered at group (village) level.
%${}^{***}$, ${}^{**}$, ${}^{*}$ indicate statistical significance at 1\%, 5\%, 10\%, respetively. Standard errors are clustered at group (village) level.
\end{tabular}
\end{minipage}

\subsection{Consumption}


\hspace{-1cm}\begin{minipage}[t]{14cm}
\hfil\textsc{\normalsize Table \refstepcounter{table}\thetable: ANCOVA estimation of consumption\label{tab ANCOVA consumption}}\\
\setlength{\tabcolsep}{1pt}
\setlength{\baselineskip}{8pt}
\renewcommand{\arraystretch}{.55}
\hfil\begin{tikzpicture}
\node (tbl) {\input{c:/data/GUK/analysis/save/EstimationMemo/ConsumptionANCOVAEstimationResults.tex}};
%\input{c:/dropbox/data/ramadan/save/tablecolortemplate.tex}
\end{tikzpicture}\\
\renewcommand{\arraystretch}{.8}
\setlength{\tabcolsep}{1pt}
\begin{tabular}{>{\hfill\scriptsize}p{1cm}<{}>{\hfill\scriptsize}p{.25cm}<{}>{\scriptsize}p{12cm}<{\hfill}}
Source:& \multicolumn{2}{l}{\scriptsize Estimated with GUK administrative and survey data of round 2 - 4.}\\
Notes: & 1. & ANCOVA estimates using administrative and survey data. Post treatment regressands are regressed on categorical variables, pre-treatment regressand and other covariates. Head age and literacy are from baseline survey data.  Sample is continuing members and replacing members of early rejecters and received loans prior to 2015 Janunary. Consumption is annualised values. \\
& 2. & $P$ values in percentages in parenthesises. Standard errors are clustered at group (village) level.
%${}^{***}$, ${}^{**}$, ${}^{*}$ indicate statistical significance at 1\%, 5\%, 10\%, respetively. Standard errors are clustered at group (village) level.
\end{tabular}
\end{minipage}

\hspace{-1cm}\begin{minipage}[t]{14cm}
\hfil\textsc{\normalsize Table \refstepcounter{table}\thetable: ANCOVA estimation of consumption by attributes \label{tab ANCOVA consumption attributes original HH}}\\
\setlength{\tabcolsep}{1pt}
\setlength{\baselineskip}{8pt}
\renewcommand{\arraystretch}{.55}
\hfil\begin{tikzpicture}
\node (tbl) {\input{c:/data/GUK/analysis/save/EstimationMemo/ConsumptionAttributesANCOVAEstimationResults.tex}};
%\input{c:/dropbox/data/ramadan/save/tablecolortemplate.tex}
\end{tikzpicture}\\
\renewcommand{\arraystretch}{.8}
\setlength{\tabcolsep}{1pt}
\begin{tabular}{>{\hfill\scriptsize}p{1cm}<{}>{\hfill\scriptsize}p{.25cm}<{}>{\scriptsize}p{12cm}<{\hfill}}
Source:& \multicolumn{2}{l}{\scriptsize Estimated with GUK administrative and survey data.}\\
Notes: & 1. & ANCOVA estimates using administrative and survey data. Post treatment regressands are regressed on categorical variables, pre-treatment regressand and other covariates. Head age and literacy are from baseline survey data.  \textsf{LargeSize} is an indicator function if the arm is of large size, \textsf{WithGrace} is an indicator function if the arm is with a grace period, \textsf{InKind} is an indicator function if the arm provides a cow. Sample is continuing members and replacing members of early rejecters and received loans prior to 2015 Janunary. Consumption is annualised values. \\
& 2. & $P$ values in percentages in parenthesises. Standard errors are clustered at group (village) level.
%${}^{***}$, ${}^{**}$, ${}^{*}$ indicate statistical significance at 1\%, 5\%, 10\%, respetively. Standard errors are clustered at group (village) level.
\end{tabular}
\end{minipage}


\hspace{-1cm}\begin{minipage}[t]{14cm}
\hfil\textsc{\normalsize Table \refstepcounter{table}\thetable: ANCOVA estimation of consumption, moderately poor vs. ultra poor\label{tab ANCOVA consumption2 original HH}}\\
\setlength{\tabcolsep}{1pt}
\setlength{\baselineskip}{8pt}
\renewcommand{\arraystretch}{.55}
\hfil\begin{tikzpicture}
\node (tbl) {\input{c:/data/GUK/analysis/save/EstimationMemo/ConsumptionPovertyStatusANCOVAEstimationResults.tex}};
%\input{c:/dropbox/data/ramadan/save/tablecolortemplate.tex}
\end{tikzpicture}\\
\renewcommand{\arraystretch}{.8}
\setlength{\tabcolsep}{1pt}
\begin{tabular}{>{\hfill\scriptsize}p{1cm}<{}>{\hfill\scriptsize}p{.25cm}<{}>{\scriptsize}p{12cm}<{\hfill}}
Source:& \multicolumn{2}{l}{\scriptsize Estimated with GUK administrative and survey data.}\\
Notes: & 1. & ANCOVA estimates using administrative and survey data. Post treatment regressands are regressed on categorical variables, pre-treatment regressand and other covariates. Head age and literacy are from baseline survey data.  \textsf{UltraPoor} is an indicator function if the household is classified as the ultra poor. Sample is continuing members and replacing members of early rejecters and received loans prior to 2015 Janunary. Consumption is annualised values. \\
& 2. & $P$ values in percentages in parenthesises. Standard errors are clustered at group (village) level.
%${}^{***}$, ${}^{**}$, ${}^{*}$ indicate statistical significance at 1\%, 5\%, 10\%, respetively. Standard errors are clustered at group (village) level.
\end{tabular}
\end{minipage}


\hspace{-1cm}\begin{minipage}[t]{14cm}
\hfil\textsc{\normalsize Table \refstepcounter{table}\thetable: ANCOVA estimation of consumption by attributes and period\label{tab ANCOVA consumption timevarying attributes original HH}}\\
\setlength{\tabcolsep}{1pt}
\setlength{\baselineskip}{8pt}
\renewcommand{\arraystretch}{.55}
\hfil\begin{tikzpicture}
\node (tbl) {\input{c:/data/GUK/analysis/save/EstimationMemo/ConsumptionTimeVaryingAttributesANCOVAEstimationResults.tex}};
%\input{c:/dropbox/data/ramadan/save/tablecolortemplate.tex}
\end{tikzpicture}\\
\renewcommand{\arraystretch}{.8}
\setlength{\tabcolsep}{1pt}
\begin{tabular}{>{\hfill\scriptsize}p{1cm}<{}>{\hfill\scriptsize}p{.25cm}<{}>{\scriptsize}p{12cm}<{\hfill}}
Source:& \multicolumn{2}{l}{\scriptsize Estimated with GUK administrative and survey data.}\\
Notes: & 1. & ANCOVA estimates using administrative and survey data. Post treatment regressands are regressed on categorical variables, pre-treatment regressand and other covariates. Head age and literacy are from baseline survey data.  \textsf{LargeSize} is an indicator function if the arm is of large size, \textsf{WithGrace} is an indicator function if the arm is with a grace period, \textsf{InKind} is an indicator function if the arm provides a cow. Sample is continuing members and replacing members of early rejecters and received loans prior to 2015 Janunary. Consumption is annualised values. \\
& 2. & $P$ values in percentages in parenthesises. Standard errors are clustered at group (village) level.
%${}^{***}$, ${}^{**}$, ${}^{*}$ indicate statistical significance at 1\%, 5\%, 10\%, respetively. Standard errors are clustered at group (village) level.
\end{tabular}
\end{minipage}



\subsection{Income}


\hspace{-1cm}\begin{minipage}[t]{14cm}
\hfil\textsc{\normalsize Table \refstepcounter{table}\thetable: ANCOVA estimation of labour incomes\label{tab ANCOVA LabourIncomes}}\\
\setlength{\tabcolsep}{1pt}
\setlength{\baselineskip}{8pt}
\renewcommand{\arraystretch}{.55}
\hfil\begin{tikzpicture}
\node (tbl) {\input{c:/data/GUK/analysis/save/EstimationMemo/LabourIncomeANCOVAEstimationResults.tex}};
%\input{c:/dropbox/data/ramadan/save/tablecolortemplate.tex}
\end{tikzpicture}
\hfil\begin{tikzpicture}
\node (tbl) {\input{c:/data/GUK/analysis/save/EstimationMemo/FarmIncomeANCOVAEstimationResults.tex}};
%\input{c:/dropbox/data/ramadan/save/tablecolortemplate.tex}
\end{tikzpicture}\\
\renewcommand{\arraystretch}{.8}
\setlength{\tabcolsep}{1pt}
\begin{tabular}{>{\hfill\scriptsize}p{1cm}<{}>{\hfill\scriptsize}p{.25cm}<{}>{\scriptsize}p{12cm}<{\hfill}}
Source:& \multicolumn{2}{l}{\scriptsize Estimated with GUK administrative and survey data.}\\
Notes: & 1. & ANCOVA estimates using administrative and survey data. Post treatment regressands are regressed on categorical variables, pre-treatment regressand and other covariates. Head age and literacy are from baseline survey data.  Sample is continuing members and replacing members of early rejecters and received loans prior to 2015 Janunary. Labour income is in 1000 Tk unit andis sum of all earned labour incomes. Farm revenue is total of agricultural produce sales. \\
& 2. & $P$ values in percentages in parenthesises. Standard errors are clustered at group (village) level.
%${}^{***}$, ${}^{**}$, ${}^{*}$ indicate statistical significance at 1\%, 5\%, 10\%, respetively. Standard errors are clustered at group (village) level.
\end{tabular}
\end{minipage}

\hspace{-1cm}\begin{minipage}[t]{14cm}
\hfil\textsc{\normalsize Table \refstepcounter{table}\thetable: ANCOVA estimation of labour incomes by attributes \label{tab ANCOVA labour incomes attributes}}\\
\setlength{\tabcolsep}{1pt}
\setlength{\baselineskip}{8pt}
\renewcommand{\arraystretch}{.55}
\hfil\begin{tikzpicture}
\node (tbl) {\input{c:/data/GUK/analysis/save/EstimationMemo/LabourIncomeAttributesANCOVAEstimationResults.tex}};
%\input{c:/dropbox/data/ramadan/save/tablecolortemplate.tex}
\end{tikzpicture}\hfil\begin{tikzpicture}
\node (tbl) {\input{c:/data/GUK/analysis/save/EstimationMemo/FarmIncomeAttributesANCOVAEstimationResults.tex}};
%\input{c:/dropbox/data/ramadan/save/tablecolortemplate.tex}
\end{tikzpicture}\\
\renewcommand{\arraystretch}{.8}
\setlength{\tabcolsep}{1pt}
\begin{tabular}{>{\hfill\scriptsize}p{1cm}<{}>{\hfill\scriptsize}p{.25cm}<{}>{\scriptsize}p{12cm}<{\hfill}}
Source:& \multicolumn{2}{l}{\scriptsize Estimated with GUK administrative and survey data.}\\
Notes: & 1. & ANCOVA estimates using administrative and survey data. Post treatment regressands are regressed on categorical variables, pre-treatment regressand and other covariates. Head age and literacy are from baseline survey data.  \textsf{LargeSize} is an indicator function if the arm is of large size, \textsf{WithGrace} is an indicator function if the arm is with a grace period, \textsf{InKind} is an indicator function if the arm provides a cow. Sample is continuing members and replacing members of early rejecters and received loans prior to 2015 Janunary. Labour income is in 1000 Tk unit andis sum of all earned labour incomes. Farm revenue is total of agricultural produce sales. \\
& 2. & $P$ values in percentages in parenthesises. Standard errors are clustered at group (village) level.
%${}^{***}$, ${}^{**}$, ${}^{*}$ indicate statistical significance at 1\%, 5\%, 10\%, respetively. Standard errors are clustered at group (village) level.
\end{tabular}
\end{minipage}

\hspace{-1cm}\begin{minipage}[t]{14cm}
\hfil\textsc{\normalsize Table \refstepcounter{table}\thetable: ANCOVA estimation of labour incomes by period\label{tab ANCOVA labour incomes timevarying}}\\
\setlength{\tabcolsep}{1pt}
\setlength{\baselineskip}{8pt}
\renewcommand{\arraystretch}{.55}
\hfil\begin{tikzpicture}
\node (tbl) {\input{c:/data/GUK/analysis/save/EstimationMemo/LabourIncomeTimeVaryingANCOVAEstimationResults.tex}};
%\input{c:/dropbox/data/ramadan/save/tablecolortemplate.tex}
\end{tikzpicture}\hfil\begin{tikzpicture}
\node (tbl) {\input{c:/data/GUK/analysis/save/EstimationMemo/FarmIncomeTimeVaryingANCOVAEstimationResults.tex}};
%\input{c:/dropbox/data/ramadan/save/tablecolortemplate.tex}
\end{tikzpicture}\\
\renewcommand{\arraystretch}{.8}
\setlength{\tabcolsep}{1pt}
\begin{tabular}{>{\hfill\scriptsize}p{1cm}<{}>{\hfill\scriptsize}p{.25cm}<{}>{\scriptsize}p{12cm}<{\hfill}}
Source:& \multicolumn{2}{l}{\scriptsize Estimated with GUK administrative and survey data.}\\
Notes: & 1. & ANCOVA estimates using administrative and survey data. Post treatment regressands are regressed on categorical variables, pre-treatment regressand and other covariates. Head age and literacy are from baseline survey data.  \textsf{LargeSize} is an indicator function if the arm is of large size, \textsf{WithGrace} is an indicator function if the arm is with a grace period, \textsf{InKind} is an indicator function if the arm provides a cow. Sample is continuing members and replacing members of early rejecters and received loans prior to 2015 Janunary. Labour income is in 1000 Tk unit andis sum of all earned labour incomes. Farm revenue is total of agricultural produce sales. \\
& 2. & $P$ values in percentages in parenthesises. Standard errors are clustered at group (village) level.
%${}^{***}$, ${}^{**}$, ${}^{*}$ indicate statistical significance at 1\%, 5\%, 10\%, respetively. Standard errors are clustered at group (village) level.
\end{tabular}
\end{minipage}

\hspace{-1cm}\begin{minipage}[t]{14cm}
\hfil\textsc{\normalsize Table \refstepcounter{table}\thetable: ANCOVA estimation of labour incomes by attributes and period\label{tab ANCOVA labour incomes timevarying attributes}}\\
\setlength{\tabcolsep}{1pt}
\setlength{\baselineskip}{8pt}
\renewcommand{\arraystretch}{.55}
\hfil\begin{tikzpicture}
\node (tbl) {\input{c:/data/GUK/analysis/save/EstimationMemo/LabourIncomeTimeVaryingAttributesANCOVAEstimationResults.tex}};
%\input{c:/dropbox/data/ramadan/save/tablecolortemplate.tex}
\end{tikzpicture}\hfil\begin{tikzpicture}
\node (tbl) {\input{c:/data/GUK/analysis/save/EstimationMemo/FarmIncomeTimeVaryingAttributesANCOVAEstimationResults.tex}};
%\input{c:/dropbox/data/ramadan/save/tablecolortemplate.tex}
\end{tikzpicture}\\
\renewcommand{\arraystretch}{.8}
\setlength{\tabcolsep}{1pt}
\begin{tabular}{>{\hfill\scriptsize}p{1cm}<{}>{\hfill\scriptsize}p{.25cm}<{}>{\scriptsize}p{12cm}<{\hfill}}
Source:& \multicolumn{2}{l}{\scriptsize Estimated with GUK administrative and survey data.}\\
Notes: & 1. & ANCOVA estimates using administrative and survey data. Post treatment regressands are regressed on categorical variables, pre-treatment regressand and other covariates. Head age and literacy are from baseline survey data.  \textsf{LargeSize} is an indicator function if the arm is of large size, \textsf{WithGrace} is an indicator function if the arm is with a grace period, \textsf{InKind} is an indicator function if the arm provides a cow. Sample is continuing members and replacing members of early rejecters and received loans prior to 2015 Janunary. Labour income is in 1000 Tk unit andis sum of all earned labour incomes. Farm revenue is total of agricultural produce sales. \\
& 2. & $P$ values in percentages in parenthesises. Standard errors are clustered at group (village) level.
%${}^{***}$, ${}^{**}$, ${}^{*}$ indicate statistical significance at 1\%, 5\%, 10\%, respetively. Standard errors are clustered at group (village) level.
\end{tabular}
\end{minipage}





\end{document}
