%  path0 <- "c:/data/GUK/"; path <- paste0(path0, "analysis/"); setwd(pathprogram <- paste0(path, "program/")); system("recycle c:/data/GUK/analysis/program/cache/EstimationMemo/"); library(knitr); knit("EstimationMemo.rnw", "EstimationMemo.tex"); system("platex EstimationMemo"); system("pbibtex EstimationMemo"); system("dvipdfmx EstimationMemo")

\input{c:/seiro/settings/Rsetting/knitrPreamble/knitr_preamble.rnw}
\renewcommand\Routcolor{\color{gray30}}
\newtheorem{finding}{Finding}[section]
\makeatletter
\g@addto@macro{\UrlBreaks}{\UrlOrds}
\newcommand\gobblepars{%
    \@ifnextchar\par%
        {\expandafter\gobblepars\@gobble}%
        {}}
\newenvironment{lightgrayleftbar}{%
  \def\FrameCommand{\textcolor{lightgray}{\vrule width 1zw} \hspace{10pt}}% 
  \MakeFramed {\advance\hsize-\width \FrameRestore}}%
{\endMakeFramed}
\newenvironment{palepinkleftbar}{%
  \def\FrameCommand{\textcolor{palepink}{\vrule width 1zw} \hspace{10pt}}% 
  \MakeFramed {\advance\hsize-\width \FrameRestore}}%
{\endMakeFramed}
\newcommand*\justify{%
  \fontdimen2\font=0.4em% interword space
  \fontdimen3\font=0.2em% interword stretch
  \fontdimen4\font=0.1em% interword shrink
  \fontdimen7\font=0.1em% extra space
  \hyphenchar\font=`\-% allowing hyphenation
}
\makeatother
\AtBeginDvi{\special{pdf:tounicode 90ms-RKSJ-UCS2}}
\special{papersize= 209.9mm, 297.04mm}
\usepackage{caption}
\usepackage{setspace}
\usepackage{framed}
\usepackage{ascmac}
\captionsetup[figure]{font={stretch=.6}} 
\def\pgfsysdriver{pgfsys-dvipdfm.def}
\usepackage{tikz}
\usetikzlibrary{calc, arrows, decorations, decorations.pathreplacing, backgrounds}
\usepackage{adjustbox}
\tikzstyle{toprow} =
[
top color = gray!20, bottom color = gray!50, thick
]
\tikzstyle{maintable} =
[
top color = blue!1, bottom color = blue!20, draw = white
%top color = green!1, bottom color = green!20, draw = white
]
\tikzset{
%Define standard arrow tip
>=stealth',
help lines/.style={dashed, thick},
axis/.style={<->},
important line/.style={thick},
connection/.style={thick, dotted},
}


\begin{document}
\setlength{\baselineskip}{12pt}

\hfil ANCOVA estimation of lending impacts\\

\hfil\MonthDY\\
\hfil{\footnotesize\currenttime}\\

\hfil Seiro Ito

\setcounter{tocdepth}{3}
\tableofcontents
\newpage

\setlength{\parindent}{1em}
\vspace{2ex}

Need: packages \textsf{lmtest, sandwich}.


To reach to this file:
\begin{enumerate}
\vspace{1.0ex}\setlength{\itemsep}{1.0ex}\setlength{\baselineskip}{12pt}
\item	read\_cleaned\_data: This reads survey files. Corrects errors.
\item	read\_admin\_data: This reads administrative file. Corrects errors, define \textsf{TradGroup2}  ``NotReceivedLoan''.
\item	ReadFilesMergeAdminRoster: This merges survey files with admin file (e.g., \textsf{AssetAdminData.rds}). Create \textsf{ar, arA} and attach \textsf{o800, o1600}. Attrition in \textsf{o800} is 92. Define \textsf{BStatus}.
\end{enumerate}

In what follows, 
\begin{enumerate}
\vspace{1.0ex}\setlength{\itemsep}{1.0ex}\setlength{\baselineskip}{12pt}
\item	Read RosterAdminData.rds, etc., create dummy interactions and trim observations if \verb+grepl("tw|dou", TradGroup)+ is true. 
\item	Summarise descriptive statistics, estimate ANCOVA, graph estimates and IGAs.
\end{enumerate}





\begin{Schunk}
\begin{Soutput}
qs 0.26.3
\end{Soutput}
\end{Schunk}



	This note uses ANCOVA as the estimator of choice. ANCOVA assumes the initial value of outcome variable is a pure nusissance that it only adds a noise and is uncorrelated with the main regressor of interest, hence uses it as a covariate. Under such assumptions, it is shown that ANCOVA is more efficient than DID as it renders data to control for baseline differences in outcomes \citep{FrisonPocock1992}. ANCOVA become numerically the same as DID if the estimated coefficient on the covariate is unity. As shown in the results, we see that it smaller and the claim that DID overcorrects for the initial values applies to our data.

 \textcolor{red}{Read: c:/data/GUK/analysis/save/EstimationMemo/AllMeetingsRosterAdminData.rds}.

Original data files (incl. Abu-san's correction files) $\rightarrow$ (read\_cleaned\_data.rnw) $\rightarrow$  
\verb+data_read_in_a_list_with_baseline_patched.rds+ $\rightarrow$ 
add admin data in ReadFilesMergeAdminRoster.rnw $\rightarrow$ 
Individual data files (RosterAdminSchoolingData.rds, RosterAdminData.rds, AllMeetingsRosterAdminData.rds, AssetAdminData.rds, LivestockAdminData.rds, LivestockLongAdminData.rds, LivestockProductsAdminData.rds, LabourIncomeAdminData.rds, FarmRevenueAdminData.rds, ConsumptionAdminData.rds, OtherBorrowingAdminData.rds, Shocks.rds) in c:/data/GUK/analysis/save/EstimationMemo/.

Further data preparations (trimming, adding shocks, round numbering, creating dummy vectors, interaction terms) are done in this file for estimation. Produces files: \textsf{\footnotesize SchoolingAdminDataUsedForEstimation.prn, AllMeetingsRepaymentAdminDataUsedForEstimation.prn, RepaymentAdminDataUsedForEstimation.prn, AssetAdminDataUsedForEstimation.prn, LivestockAdminDataUsedForEstimation.prn, LivestockLongAdminDataUsedForEstimation.prn, LivestockProductsAdminDataUsedForEstimation.prn, LabourIncomeAdminDataUsedForEstimation.prn, FarmIncomeAdminDataUsedForEstimation.prn, ConsumptionAdminDataUsedForEstimation.prn, OtherBorrowingAdminDataUsedForEstimation.prn}.
\begin{Schunk}
\begin{Soutput}
 [1] "s1"   "arA"  "ar"   "ass"  "lvo"  "lvoL" "lvp"  "lab"  "far"  "con" 
[11] "obr" 
\end{Soutput}
\end{Schunk}


Create \textsf{RepaymentTrimmed.rds} sample from \textsf{RosterAdminData.rds}.


Original HHs are 800 HHs at the baseline, whose size shrinks by attrition: 743, 745, 708. They are used for attrition and randomisation tests. 

Add \textsf{BStatus} etc. to all files and save as: 

\textcolor{red}{c:/data/GUK/analysis/save/EstimationMemo/RepaymentTrimmed.rds}

for example. It is trimmed as we keep only \verb+grepl("old|iRej|^g", Mstatus)+ (old members, individual rejection, group rejection, group erosion). Here, we have not yet dropped \verb+grepl("tw|dou", TradGroup)+ (twice received, double received in tradional arm). 


This is done in \textsf{EstimationMemo\_ANCOVA3.rnw}: Dropping members who received only 2 loans (twice and double in \textsf{TradGroup}) and save as an estimating sample, for example:

\textcolor{red}{Save: c:/data/GUK/analysis/save/EstimationMemo/RepaymentInitialSample.rds}

As we use 
\textcolor{red}{RepaymentInitialSample.rds} as our base to merge other files, files ending with \textsf{InitialSample.rds} are the data we use.


Check asset file entries. There are 797 households who respond at least once to asset questionnaire, but only
741 households respond at baseline (out of 800, response rate of 92.62\%). Below displays the timing of first reponse on assets by HHs. 
\begin{Schunk}
\begin{Sinput}
addmargins(table0(ass[o800 == 1L, .(Arm, survey, tee = 1:.N), 
  by = .(hhid)][tee == 1, .(survey, Arm)]))
\end{Sinput}
\begin{Soutput}
      Arm
survey traditional large large grace cattle Sum
   1           184   189         189    179 741
   2            14    10          10     19  53
   3             0     1           0      2   3
   Sum         198   200         199    200 797
\end{Soutput}
\end{Schunk}
Among which, if we drop the sample in \textsf{Traditional} arm who receive the loan only twice, we have:
\begin{Schunk}
\begin{Sinput}
addmargins(table0(ass[o800 == 1L & !grepl("tw|dou", TradGroup) & 
  hhid %in% hhid[survey==1], 
  .(Arm, tee = 1:.N), by = .(survey, hhid)][tee == 1, .(survey, Arm)]))
\end{Sinput}
\begin{Soutput}
      Arm
survey traditional large large grace cattle  Sum
   1           162   189         189    179  719
   2           153   181         161    169  664
   3           154   182         164    169  669
   4           135   182         161    156  634
   Sum         604   734         675    673 2686
\end{Soutput}
\end{Schunk}




\hspace{-1cm}\begin{minipage}[t]{14cm}
\hfil\textsc{\normalsize Table \refstepcounter{table}\thetable: Data trimming results\label{tab trim}}\\
\setlength{\tabcolsep}{1pt}
\setlength{\baselineskip}{8pt}
\renewcommand{\arraystretch}{.48}
\hfil\begin{tikzpicture}
\node (tbl) {\input{c:/data/GUK/analysis/save/EstimationMemo/TrimmingNumObsTable.tex}};
%\input{c:/dropbox/data/ramadan/save/tablecolortemplate.tex}
\end{tikzpicture}\\
\renewcommand{\arraystretch}{.8}
\setlength{\tabcolsep}{1pt}
\begin{tabular}{>{\hfill\scriptsize}p{1cm}<{}>{\hfill\scriptsize}p{.25cm}<{}>{\scriptsize}p{12cm}<{\hfill}}
Source:& \multicolumn{2}{l}{\scriptsize GUK survey data.}\\
Notes: & 1. & Top panel is observations for all rounds. Middle panel is observations for round 1 only. Bottom panel is observations for original 800 households at round 1. old$|$iRej$|$\^{}g in \textsf{Mstatus} are strings for old members, individual rejecters, group rejecters, group erosion. con$|$\^{}dro$|$\^{}rep in \textsf{Mgroup} indicates continuing, dropouts, replacing members. tw$|$dou in \textsf{TradGroup} are members who received loans twice and double amount in the 2nd loans. They are omitted from analysis because they are under a different treatment arm. \\
& 2. & \textsf{ar} lists all survey respondents, \textsf{arA} lists all loan recipients. There are 0 members in \textsf{traditional} arm who received loans twice, not three times. They will be omitted from ITT effects estimation. Consumption is sampled in round 2.
\end{tabular}
\end{minipage}

%Number of observations after trimming: 1. Keep only membership = 1 or 4, which corresponds to Mstatus old, iRej, gR, gE; 2. Keep only continuing, dropouts members in Mgroup.

Tabulate number of obs in each files. Read: c:/data/GUK/analysis/save/EstimationMemo/AllMeetingsRepaymentAdminData.rds.

Tabulate number of obs in each files for original 800 households (before dropping 24 HHs in trad). 
\begin{Schunk}
\begin{Soutput}
x
traditional       large large grace      cattle         Sum 
        168         192         171         177         708 
x
traditional       large large grace      cattle         Sum 
        168         192         171         177         708 
\end{Soutput}
\end{Schunk}

\hfil\begin{minipage}[t]{12cm}
\hfil\textsc{\normalsize Table \refstepcounter{table}\thetable: Number of observations  in each file at round 1 from HHs with single treatment\label{tab NObsOHall}}\\
\setlength{\tabcolsep}{.5pt}
\setlength{\baselineskip}{10pt}
\renewcommand{\arraystretch}{.7}
\hfil\begin{tikzpicture}
\node (tbl) {\input{c:/data/GUK/analysis/save/EstimationMemo/NumObsOriginalHHs_all.tex}};
%\input{c:/dropbox/data/ramadan/save/tablecolortemplate.tex}
\end{tikzpicture}\\
\renewcommand{\arraystretch}{.8}
\setlength{\tabcolsep}{1pt}
\begin{tabular}{>{\hfill\scriptsize}p{1cm}<{}>{\hfill\scriptsize}p{.25cm}<{}>{\scriptsize}p{10cm}<{\hfill}}
Source:& \multicolumn{2}{l}{\scriptsize Estimated with GUK administrative and survey data.}\\
Notes: & 1. & Sample is all households: Original 1600 and added households through new groups and individuals replacing opt-out members. All households in traditional arm who received more than one loan are excluded.\\
& 2. &  
\end{tabular}
\end{minipage}

\hfil\begin{minipage}[t]{12cm}
\hfil\textsc{\normalsize Table \refstepcounter{table}\thetable: Number of observations in each file at round 1 from original 1600 HHs\label{tab NObsOHo1600}}\\
\setlength{\tabcolsep}{.5pt}
\setlength{\baselineskip}{10pt}
\renewcommand{\arraystretch}{.7}
\hfil\begin{tikzpicture}
\node (tbl) {\input{c:/data/GUK/analysis/save/EstimationMemo/NumObsOriginalHHs_o1600.tex}};
%\input{c:/dropbox/data/ramadan/save/tablecolortemplate.tex}
\end{tikzpicture}\\
\renewcommand{\arraystretch}{.8}
\setlength{\tabcolsep}{1pt}
\begin{tabular}{>{\hfill\scriptsize}p{1cm}<{}>{\hfill\scriptsize}p{.25cm}<{}>{\scriptsize}p{10cm}<{\hfill}}
Source:& \multicolumn{2}{l}{\scriptsize Estimated with GUK administrative and survey data.}\\
Notes: & 1. & Sample is original 1600 households who agree to join the group. This includes households who later dropped out due to flood, group rejections, and individual rejections. All original 1600 households are tracked but some attrit from the sample.\\
& 2. &  
\end{tabular}
\end{minipage}

\hfil\begin{minipage}[t]{12cm}
\hfil\textsc{\normalsize Table \refstepcounter{table}\thetable: Number of observations in each file at round 1 from original 800 HHs admin data before trimming\label{tab NObsOHo800 AdminData}}\\
\setlength{\tabcolsep}{.5pt}
\setlength{\baselineskip}{10pt}
\renewcommand{\arraystretch}{.7}
\hfil\begin{tikzpicture}
\node (tbl) {\input{c:/data/GUK/analysis/save/EstimationMemo/NumObsOriginalHHsAdminData_o800.tex}};
%\input{c:/dropbox/data/ramadan/save/tablecolortemplate.tex}
\end{tikzpicture}\\
\renewcommand{\arraystretch}{.8}
\setlength{\tabcolsep}{1pt}
\begin{tabular}{>{\hfill\scriptsize}p{1cm}<{}>{\hfill\scriptsize}p{.25cm}<{}>{\scriptsize}p{10cm}<{\hfill}}
Source:& \multicolumn{2}{l}{\scriptsize Estimated with GUK administrative and survey data.}\\
Notes: & 1. & Sample is original 800 households who agree to join the group in \textsf{RosterAdmin.rds}. This includes households who later dropped out due to flood, group rejections, and individual rejections. All original 800 households are tracked but some attrit from the sample.\\
& 2. &  
\end{tabular}
\end{minipage}

\hfil\begin{minipage}[t]{12cm}
\hfil\textsc{\normalsize Table \refstepcounter{table}\thetable: Number of observations in each file at round 1 from original 800 HHs trimmed for Mstatus\label{tab NObsOHo800 Trimmed}}\\
\setlength{\tabcolsep}{.5pt}
\setlength{\baselineskip}{10pt}
\renewcommand{\arraystretch}{.7}
\hfil\begin{tikzpicture}
\node (tbl) {\input{c:/data/GUK/analysis/save/EstimationMemo/NumObsOriginalHHsTrimmed_o800.tex}};
%\input{c:/dropbox/data/ramadan/save/tablecolortemplate.tex}
\end{tikzpicture}\\
\renewcommand{\arraystretch}{.8}
\setlength{\tabcolsep}{1pt}
\begin{tabular}{>{\hfill\scriptsize}p{1cm}<{}>{\hfill\scriptsize}p{.25cm}<{}>{\scriptsize}p{10cm}<{\hfill}}
Source:& \multicolumn{2}{l}{\scriptsize Estimated with GUK administrative and survey data.}\\
Notes: & 1. & Sample is based on original 800 households who agree to join the group in \textsf{RosterAdmin.rds}, and keeping old member, individual and group rejecters, and flood eroded households. Some households later drop out due to flood, group rejections, and individual rejections. \\
& 2. &  
\end{tabular}
\end{minipage}



\textcolor{blue}{This file reads data from a list \textsf{data\_read\_in\_a\_list\_with\_baseline\_patched.rds}, merge all non-roster files with admin-roster, and saves in c:/data/GUK/analysis/save/EstimationMemo/.}




\section{Summary}

\subsection{Definitions}

\subsubsection{Arms}
(125*45*3) or, CumRepaid/(190*45*2)
\begin{description}
\vspace{1.0ex}\setlength{\itemsep}{1.0ex}\setlength{\baselineskip}{12pt}
\item[Traditional]	A cash loan of Tk. 5600 with one year maturity. Repay Tk 125 * 45 weeks = 5625 each year for 3 years.
\item[Large]	A cash loan of Tk. 16800 with three year maturity. Repay Tk 125 * 45 weeks * 3 years = 16875
\item[Large Grace]	A cash loan of Tk. 16800 with a one year grace period and three year maturity. Repay Tk 190 * 45 weeks * 2 years = 17100.
\item[Cow]	An in-kind loan of a cow worth Tk. 16800 with a one year grace period and three year maturity. Repay Tk 190 * 45 weeks * 2 years = 17100.
\item[LargeSize]	An indicator variable takes the value of 1 if the arm is Large, Large Grace, or Cow.
\item[WithGrace]	An indicator variable takes the value of 1 if the arm is Large Grace or Cow.
\item[InKind]	Same as Cow.
\end{description}
When one uses covariates \textsf{Large, Large Grace, Cow} in estimation, their estimates represent each arm's characteristics relative to \textsf{Traditional}. When one uses covariates \textsf{LargeSize, WithGrace, InKind}, their estimates represent their labeled names.

\subsubsection{Assets}


Net assets, net broad assets, and livestock values are computed/defined in \textsf{MergeAllNarrowNetAssetsANCOVA.R}. All other asset components are summed/defined in \textsf{read\_cleaned\_data.rnw}. 
\begin{description}
\vspace{1.0ex}\setlength{\itemsep}{1.0ex}\setlength{\baselineskip}{12pt}
\item[Household assets]	Non-livestock asset items, excluding land holding, reported in all rounds. Tubewell, mobile phone, bicycle, wrist watch, sewing machine, rickshaw/van, wall clock, radio/tv, solar, electric fan, cassette player.
\item[Broad household assets]	All non-livestock asset items (some are reported only in some rounds). Land holding at baseline is asked as a recall in round 2. When land is added, we assume its baseline value to be zero. This may inflate the asset growth.
\item[Total imputed value of livestock]	Livestock holding. Median sales price through out the survey rounds are used to impute values.
\item[Total imputed 2 value of livestock]	Livestock holding. Median annual sales price are used to impute values.
\item[Productive assets]	Tractor, thresher, power tiller, power pump, hand pump, deep tube-well, shallow tube-well, treddle pump, rower pump, done/swing basket, plough and yoke, spray, husking machine, ginning machine, country boat, engine boat, fishing net, cage incubator, brooder, bees-box, weeder, ladder (moi), sickle/dao/axe/spade, gola (grain storage), dheki, jata, sewing machine, rickshaw, etc.
\item[Net assets]	TotalImputedValue + NLAssetAmount - DebtOutstanding.before - NonNGOBal. I.e., household assets + productive assets + livestock holding + net saving - debt to GUK - debts to relatives and money lenders. Assets use only items observed for all 4 rounds for household assets \textit{including} radios and cassette players (which have possibly large errors) but exclusing land. 
\item[Net broad assets]	TotalImputedValue + BroadNLAssetAmount - DebtOutstanding.before - NonNGOBal. I.e., broad household assets + productive assets + livestock holding + net saving - debt to GUK - debts to relatives and money lenders. Assets use all items observed for household assets, including land. 
\item[Net non-livestock assets]	 NLAssetAmount - DebtOutstanding.before - NonNGOBal. I.e., household assets + productive assets + net saving - debt to GUK - debts to relatives and money lenders. Assets use only items observed for all 4 rounds for household assets \textit{including} radios and cassette players (which have possibly large errors) but exclusing land. 
\item[Cattle holding]	Number of cattlle holding in counts, not in monetary units. 
\end{description}


\subsection{Inference}

\begin{itemize}
\vspace{1.0ex}\setlength{\itemsep}{1.0ex}\setlength{\baselineskip}{12pt}
\item	First-difference estimators are used. This can be seen as an extension of DID to multi-periods (although historically the latter precedes the former). FD is used also for a binary indicator such as schooling.
\item	All the standard errors are clustered at the group (char) level.
\item	To aid the understanding if the data is more suited to the assumption of first-difference rather than fixed-effects, I used a check suggested by \citet[][10.71]{Wooldridge2010}. It is an AR(1) regression of FD residuals. Most of results show low autocorrelations which is consistent with the assumption of FD estimator. The use of cluster-robust standard errors gives consistent estimates of SEs, so it boils down to efficiency. 
\item	I rely more on the formulation using \textsf{LargeSize, WithGrace, InKind} than \textsf{Large, LargeGrace, Cow} due to an ease in interpretation. Numerically, both are equivalent.
\item	A caution on reading the estimates: All are estimates on increments. If \textsf{LargeSize} has an estimate of 10, then it is a 10 unit larger change than the baseline (traditional). If the interaction of \textsf{LargeSize} with rd 2-3 is 10, then it is a 10 unit larger change than rd 2-3 change of baseline. If the estimated value of intercept is 10 and rd 2-3 is 10, then rd 2-3 change is 20 for baseline, 30 for \textsf{LargeSize}. 
\end{itemize}

\subsection{Findings}

Overall, the intervention reveals that larger sized loans accerelate the timing of becoming an owner of large livestock without adversely affecting the repayments. This applies to both the ultra poor and the moderately poor. A loan amount seems to have convex returns at a low level of assets. Higher growths come at a cost of slower school progression of older girls and smaller increases in consumption for the arm of in-kind, so the welfare implication is mixed. In addition, given that the number of cows per owner remains the similar after 2 years, it does not provide evidence for accelerated growth of livestock after becoming an owner in this short window. Another note is that the loan repayment was poor for unknown reasons so, in the hindsight, the risks required a higher margin for this type of lending to the target population, which could have reduced participation.

\begin{description}
\vspace{1.0ex}\setlength{\itemsep}{1.0ex}\setlength{\baselineskip}{12pt}
\item[Net saving and repayments]	 Sample uses administrative records of \textcolor{red}{all borrowers in the original 800 households}. Smaller net saving for \textsf{traditional} arm. Period of rds 2, 3 saw a positive net saving, then became negative in rd 4 for \textsf{LargeGrace, Cow}. Repayment is greater for \textsf{Large, LargeGrace, Cow} in rds 2, 3. In rd 4, repayment of \textsf{Large} becomes statistically the same with \textsf{Traditional} while \textsf{LargeGrace, Cow} are greater (\textsc{Table \ref{tab FD saving original HH}}). \textsc{\normalsize Table \ref{tab FD saving attributes original HH}} (1) reveals \textsf{LargeSize} have larger net saving while (2) shows \textsf{WithGrace} has a faster decline in rds 2, 3, 4. Repayment is larger with \textsf{LargeSize} but smaller with \textsf{WithGrace} in (3). (4) shows rd 2-3 have larger repayment for \textsf{WithGrace}, which is by design. Repayment is positively autocorrelated and is negatively correlated with previous net saving. The ultra poor repaid just as much as the moderately poor, (\textsc{Table \ref{tab FD saving2 original HH}}). This is evidence against the popular belief that the ultra poor are riskier.  
\item[Schooling]	Enrollment changes are larger for primary school girls in \textsf{Large} and \textsf{Cow} arms for primary but smaller for junior in rd 1 vs rd 4 comparisons (\textsc{Table \ref{tab FD enroll5 original HH}}). When seen by attributes in \textsc{\normalsize Table \ref{tab FD enroll5 attributes original HH}}, \textsf{LargeSize} shows smaller changes especially for primary school boys. Primary school girls in \textsf{LargeSize} and \textsf{InKind} show larger changes, while junior and high school girls in \textsf{LargeSize} show smaller changes than boys. This indicates that large sized arms have detrimetal impacts on older girls' schooling but promotional impacts on primary school aged girls. No decline in enrollment changes when repaying for the arms of \textsf{WithGrace}, despite the larger installments.
\item[Assets]	Household assets increased in all arms. Asset values initially increased then decreased, but do not fully cancel out and remain increased. There might have been liquidation of assets to repay the loans. Productive assets declined consecutively. Flood in rd 1 makes the increase in household assets smaller. Productive assets see a major decline among \textsf{Large} during rd 3-4 period (\textsc{\normalsize Table \ref{tab FD assets original HH}}). Comparison by attributes (\textsc{\normalsize Table \ref{tab FD assets attributes original HH}}) or of rd 2 and rd 4 gives the same picture (\textsc{\normalsize Table \ref{tab FD assets rd24 original HH}}). Comparison against the loan non-recipients shows that they also experience a similar, increase-increase-decrease pattern. This indicates that the pattern observed among the loan recipients may be a systemic pattern of the area, not necessarily reflecting the repayment burdern (\textsc{\normalsize Table \ref{tab FD assets pure control original HHs}}). Comparison of productive asset holding of loan recipients (\textsc{\footnotesize Figure \ref{fig PAssets}}) and loan nonrecipients (\textsc{\footnotesize Figure \ref{fig PAssetsLoanNonrecipients}}) reveals that productive asset holding declined at the top end of loan nonrecipients in all arms (they only save or left the program). This indicates that the decline in productive asset holding among the loan recipients are not due to the repayment burden but a general pattern of the area.
\item[Livestock]	Larger increases in holding values in rd 1-2, smaller increases in rd 2-3, no change in rd 3-4. Previous cow owners show a smaller increase in rd 1-2 while not rd 3-4 or rd 2-3 in the \textsf{Cow} arm (\textsc{\normalsize Table \ref{tab FD livestock original HH}}). Figures show that cow ownership increased for all arms but the \textsf{traditional} arm (see \textsc{\normalsize Figure \ref{Figure Number of cows by year original HHs}}). \textsc{\normalsize Table \ref{tab FD livestock attributes original HH}} shows baseline trend is a large increse in rd 1-2, a small increase in rd 2-3, a small decline in rd 3-4, while \textsf{LargeSize} sees an even larger increase in rd 1-2 and similar trend as baseline afterwards. This shows that member who received a larger sized disbursement could hold on to its level of livestock accumulation. \textsc{\normalsize Table \ref{tab FD livestock poor original HH}} shows, albeit at $p$ values around 10\%, the ultra poor has a larger increase relative to the moderately poor, which is another manifestation against the popular notion that the ultra poor are riskier.
\item[Total asset values]	Similar resulsts as assets.
\item[Labour incomes]	Small sample. Increased during rd 2-3 in all arms (\textsc{\normalsize Table \ref{tab FD incomes original HH}}). 
\item[Consumption]	Increased during rd 2-3 in all arms, a decrese in rd 3-4 (\textsc{\normalsize Table \ref{tab FD consumption original HH}}). Another notable result is that \textsf{InKind} reduced the consumption in rd 3-4 even further than the baseline loan (\textsc{\normalsize Table \ref{tab FD consumption attributes original HH}}).
\item[IGAs]	Multiple IGAs for \textsf{Tradtional} arm. Everyone else chose to invest in cows, suggesting entrepreneurship does not seem to matter in the uptake of loans. It is consistent with the presence of a poverty trap induced by a liquidity constraint and convexity in livestock production technology.
\item[Project choice]	\textsf{Traditional} arm has a smaller rate of second investments, and second investment amounts are generally smaller (\textsc{\footnotesize Figure \ref{fig first2ndFixedInvest}}). This confirms that most of \textsf{Traditional} arm members do not use own fund to increase the size of investments even after a few years into the program.
\end{description}

One sees changes in investment choices when one compares \textsf{traditional} and all other arms. However, consumption does not seem to differ. Repayments and asset holding are greater in all other arms. These are consistent with households are enforcing the repayment disciplines and reinvesting the proceeds rather than increasing consumption. 






\begin{Schunk}
\begin{Soutput}
Warning in `[.data.table`(s1xR, , `:=`(c("Age_1", grepout("Primary", colnames(s1xR))), : Column 'Age_1' does not exist to remove
\end{Soutput}
\begin{Soutput}
Dropped 902 obs due to NA.
Dropped 902 obs due to NA.
Dropped 223 obs due to T<2.
Dropped 574 obs due to NA.
\end{Soutput}
\end{Schunk}
Enrollment pattern in original schooling panel. `n' indicates NA (either attrition or not reported).
\begin{Schunk}
\begin{Soutput}
         SchPattern
ObPattern 0000 0001 000n 0011 001n 00nn 010n 0111 011n 01nn 0nnn 1000 1001 100n
     0111    0    0    0    0    0    0    0    0    2    1    2    0    0    0
     1000    0    0    0    0    0    0    0    0    0    0   32    0    0    0
     1010    0    0    0    0    0    1    0    0    0    0    2    0    0    0
     1011    0    0    0    0    0    0    0    0    0    0    0    0    0    0
     1100    0    0    0    0    0    0    0    0    0    3    1    0    0    0
     1110    0    0    5    0    2    1    0    0    3    0    2    0    0    1
     1111   21    2   21   12    2   28    1   83    8    6   68    5    1    4
         SchPattern
ObPattern 1011 101n 10nn 1100 1101 110n 1110 1111 111n 11n1 11nn 1nnn
     0111    0    0    0    0    0    0    0    0    6    0    0    1
     1000    0    0    0    0    0    0    0    0    0    0    0   22
     1010    0    0    0    0    0    0    0    0    0    0    1    2
     1011    0    0    0    0    0    0    0    0    0    0    0    0
     1100    0    0    0    0    0    0    0    0    0    0    6    0
     1110    0    1    0    0    0    0    0    0   25    0    3    0
     1111    6    3   10    8    1    9    3  397   41    1   29   42
\end{Soutput}
\end{Schunk}
Left panel is before dropping \textsf{nnn}, right panel is after: Original panel.
\begin{Schunk}
\begin{Soutput}
  traditional large large grace cattle traditional large large grace cattle
1         205   246         251    235         166   208         186    203
2         166   208         186    203         166   208         186    203
3         148   184         168    173         148   184         168    173
4         113   161         135    131         113   161         135    131
\end{Soutput}
\end{Schunk}
\textsf{sch} has 2913 rows. Drop 174 observations in \textsf{sch} with nnn in \textsf{SchPattern}. 
%and nrow(s.1x[!grepl("nnn", Spattern) & grepl("1001", EnrollPattern), ]) observations with 1001 in \textsf{EnrollPattern} because they are likely to be errors. This leaves us with nrow(s1x) rows. 

With OLS,  89, 135, 539 individuals are repeatedly observed for 2, 3, 4 times, respectively. With FD, \textsf{sch} is reduced to 1837 rows after first-differencing with 75, 129, 499 individuals with repeatedly observed for 1, 2, 3 times, respectively.
Individuals with NAs in \textsf{Enrolled}: 0 obs for \textsf{sch}. 
%Mostly older children (round(mean(s.1[is.na(Enrolled), Age_1], na.rm = T), 1) in \textsf{s1x}, round(mean(s.2[is.na(Enrolled), Age_1], na.rm = T), 1) in \textsf{s.2}) but with a high reported enrollment rate (round(mean(s.1[is.na(Enrolled) & tee == 4, Enrolled]), 1) for \textsf{s1x}, round(mean(s.2[is.na(Enrolled) & tee == 4, Enrolled]), 1) for \textsf{s.2}) at rd 4. We will substitute relevant schooling levels to \textsf{Enrolled}.
Check missingness in schooling level information.
\begin{Schunk}
\begin{Soutput}
x
   0    1 
1575 1164 
\end{Soutput}
\end{Schunk}
Drop 1575 obs without school level information.

%Check missingness in arm information.

%An example of dummy interactions: \textsf{\footnotesize tobeintSchdumTimdum}.
%Obs for \textsf{sch} after FD.

%Obs for \textsf{s1x} and admin repayment data.

\begin{Schunk}
\begin{Soutput}
Dropped 902 obs due to NA.
Dropped 902 obs due to NA.
Dropped 223 obs due to T<2.
Dropped 574 obs due to NA.
\end{Soutput}
\end{Schunk}


\begin{Schunk}
\begin{Soutput}
要求されたパッケージ sandwich をロード中です
\end{Soutput}
\begin{Soutput}
要求されたパッケージ lmtest をロード中です
\end{Soutput}
\begin{Soutput}
要求されたパッケージ zoo をロード中です
\end{Soutput}
\begin{Soutput}

次のパッケージを付け加えます: 'zoo'
\end{Soutput}
\begin{Soutput}
以下のオブジェクトは 'package:data.table' からマスクされています:

    yearmon, yearqtr
\end{Soutput}
\begin{Soutput}
以下のオブジェクトは 'package:base' からマスクされています:

    as.Date, as.Date.numeric
\end{Soutput}
\end{Schunk}




\newpage


\section{Data preparation}

\subsection{Define initial sample}


\textcolor{red}{c:/data/GUK/analysis/save/EstimationMemo/RosterAdmin.rds} keeps all 800 members which will be used in attrition and randomisation tests. They are maked as \textsf{o800==1L}. 

\textcolor{red}{c:/data/GUK/analysis/save/EstimationMemo/RepaymentTrimmed.rds} keeps 798 members after keeping only old members, individual rejecters, and group rejecters. Trimmed sample is produced in \textsf{EstimationMemo\_ChildFile1.rnw}. 

\textsf{InitialSample} is produced by dropping 24 HHs of traditional arm (who received only 2 loans [twice and double in \textsf{TradGroup}]) from Trimmed Sample . \textcolor{red}{c:/data/GUK/analysis/save/EstimationMemo/RepaymentInitialSample.rds}. 



\begin{Schunk}
\begin{Soutput}
             Only2Loans
Arm           FALSE TRUE Sum
  traditional   176   24 200
  large         200    0 200
  large grace   200    0 200
  cattle        200    0 200
  Sum           776   24 800
\end{Soutput}
\begin{Soutput}
x
traditional       large large grace      cattle         Sum 
        200         200         200         200         800 
\end{Soutput}
\begin{Soutput}
Empty data.table (0 rows and 5 cols): hhid,VArm,Mstatus,BStatus,ObPattern
\end{Soutput}
\begin{Soutput}
     AttritIn
Tee     2   3   4   9 Sum
  1    41   0   0   0  41
  2     0  14   0   0  14
  3     0   0  37   0  37
  4     0   0   0 684 684
  Sum  41  14  37 684 776
\end{Soutput}
\end{Schunk}









The study followed the stepped wedge design within each group due to administrative and budgetary constraints. Our initial identification strategy was comaprison between arms and did not use the stepped wedge design to estimate impacts because of possible spillovers within a group and a relatively short period for outcomes to change before the control gets treated [We can estimate within-group, we may just have underestimated impacts]. A half of members in a group, approximately 800 in total, are assigned initially as the treated and then the rest was treated in the following months. So the number of the treated increased as time passes. 

We restrict ourselves to this initial 800 members in estimating the impacts. We do so because of possible spill overs within groups. We compare between arms, not individuals in a group. One can see how impacts may differ if we compare between-group and within-group estimates. Such comparison is left as future exercises.

	 %To define our initially-treated sample, we need to use a cutoff month up to which we will include members in the intial-treated. We do not have the information of the dates we offered the members the loans [Were there members who hesitated and waited some months to get a loan?]. To rule out possible endogeneity of loan receipt timing, we decide the cutoff month gap that gives about 800, a half of 1600 households, in the resulting sample size. We compute \textsf{MonthGap} which is the number of months between the disbursement of first loan and own loan for each group. There are sum(is.na(ar[tee == 1, MonthGap])) NAs, which is the total of members who stayed as a pure saver or who quit the group. We track all the individuals who were offered a loan, including pure savers and people who left the group. 



We will add a binary indicator function \textsf{o800} to indicate the initial sample. In below, we first use the roster-administrative data to choose the households of \textsf{o800}, because it has the most complete record. Then, I look for these households in other files and create \textsf{o800} variable in them.

Correct NAs in \textsf{LoanYear} to -1 when members start repayment before disbursement.




\subsection{Descriptive statistics}


% For mixed knitr and plain latex file, need to use knitr_child function. A chunk child-file style gives a latex compilation error.

The majority of descriptive statistics are related to assets. We base our descriptive statistics on the asset data.

\begin{Schunk}
\begin{Soutput}


Number of obs by Arm and attrition
             AttritIn
Arm             2   3   4   9 Sum
  traditional   6   4  20 144 174
  large         5   2   1 192 200
  large grace  22   3   3 171 199
  cattle        5   5  13 177 200
  Sum          38  14  37 684 773


Number of obs by membership status and attrition
                      AttritIn
BStatus                  2   3   4   9 Sum
  borrower               8   6   8 578 600
  pure saver             0   0   0   0   0
  individual rejection   9   4   1  75  89
  group rejection        9   4   0  55  68
  rejection by flood    12   0  28   0  40
  Sum                   38  14  37 708 797
\end{Soutput}
\end{Schunk}
\begin{Schunk}
\begin{Soutput}
Arm
traditional       large large grace      cattle         Sum 
        200         200         200         200         800 
\end{Soutput}
\end{Schunk}
There are 24 members with TradGroup = twice, double. They were dropped from estimation sample. If \textsf{UseTrimmedSample==T}, attrition is based on all 800 members, if \textsf{F}, attrition is analysed using 776 members. We use the `initial' sample (has only 776 members after dropping members who received loans only twice), not the `trimmed' sample (has all 800 members). \gobblepars

\begin{Schunk}
\begin{Sinput}
cat("UseTrimmedSample is", UseTrimmedSample, "\n")
\end{Sinput}
\begin{Soutput}
UseTrimmedSample is FALSE 
\end{Soutput}
\begin{Sinput}
if (!UseTrimmedSample) 
  ar <- ar[!grepl("tw|dou", TradGroup), ]
addmargins(table0(ar[o800 == 1L & tee == 1, .(Tee, AttritIn)]))
\end{Sinput}
\begin{Soutput}
     AttritIn
Tee     2   3   4   9 Sum
  1    41   0   0   0  41
  2     0  14   0   0  14
  3     0   0  37   0  37
  4     0   0   0 684 684
  Sum  41  14  37 684 776
\end{Soutput}
\end{Schunk}
Out of 776 members, there are 92 members who attrited.
\begin{Schunk}
\begin{Soutput}
                      AttritIn
BStatus                 2  3  4 Sum
  borrower              8  6  8  22
  pure saver            0  0  0   0
  individual rejection 10  4  1  15
  group rejection      11  4  0  15
  rejection by flood   12  0 28  40
  Sum                  41 14 37  92
\end{Soutput}
\end{Schunk}

%Correct \textsf{AttritIn} for these nrow(ar[grepl("tw|dou", TradGroup) & tee == 1, ]) members. Keep only the 1st obs for all members.


\begin{Schunk}
\begin{Soutput}
             AttritIn
Arm             2   3   4   9 Sum
  traditional   8   4  20 144 176
  large         5   2   1 192 200
  large grace  23   3   3 171 200
  cattle        5   5  13 177 200
  Sum          41  14  37 684 776
\end{Soutput}
\end{Schunk}
\begin{Schunk}
\begin{Soutput}
             Attrited
Arm             0   1 Sum
  traditional 144  32 176
  large       192   8 200
  large grace 171  29 200
  cattle      177  23 200
  Sum         684  92 776
\end{Soutput}
\end{Schunk}




\hspace{-1cm}\begin{minipage}[t]{14cm}
\hfil\textsc{\normalsize Table \refstepcounter{table}\thetable: Baseline descriptive statistics by arm for all households including nonparticipants\label{tab DestatMainByArm}}\\
\setlength{\tabcolsep}{1pt}
\setlength{\baselineskip}{8pt}
\renewcommand{\arraystretch}{.55}
\hfil\begin{tikzpicture}
\node (tbl) {\input{c:/data/GUK/analysis/save/EstimationMemo/DestatMainByArm.tex}};
\end{tikzpicture}\\
\renewcommand{\arraystretch}{.8}
\setlength{\tabcolsep}{1pt}
\begin{tabular}{>{\hfill\scriptsize}p{1cm}<{}>{\hfill\scriptsize}p{.25cm}<{}>{\scriptsize}p{12cm}<{\hfill}}
Source:& \multicolumn{2}{l}{\scriptsize Estimated with GUK administrative and survey data at the period 2. Survey respondents include nonparticipants to the experiments.}\\
Notes: & 1. & Information of original 800 households. Values are means, values in brackets are standard deviations. \\
& 2 & \textsf{HeadLiteracy}, \textsf{HeadAge} are literacy and ages of household heads. \textsf{HHsize} is total number of household members. \textsf{FloodInRd1} is flood exposure at period 2. \textsf{NLHAssetAmount} is non-land household asset holding value, \textsf{PAssetAmount} is productive asset holding value, \textsf{TotalImputedAmount} is imputed value of livestock holding. 
\textsf{NumCows} is cattle holding per household. \textsf{NetValue} is net asset values per housheold for asset items observed in all 4 rounds given by \textsf{NLHAssetAmount}+\textsf{PAssetAmount}+\textsf{TotalImputedAmount} - total debt. %\textsf{NarrowNetValue} is net asset values per housheold for asset items observed in all 4 rounds excluding cassette players, radios. 
\textsf{NetBroadValue} is net asset values per housheold for all asset items. %\textsf{RNetValue} is net asset values per housheold for asset items observed in all 4 rounds with cassette players, radios winsorised at BDT 20000 and bicycles at BDT 2000. \textsf{RNarrowNetValue} and \textsf{RNetBroadValue} are winsorised equivalents to \textsf{NarrowNetValue} and \textsf{NetBroadValue}. 
All asset values are expressed in BDT. \textsf{Attrited} indicates attrition rates in the household survey, and \textsf{GRejected} and \textsf{IRejected} show group rejection rates and individual rejection rates to the lending program. \textsf{Active} indicates the nonattrited borrower ratios. Because attrition and rejection are separate events, a household can reject and attrit, so active members $\geqslant$ total - (rejected members + attrited members). \textsf{Risk preference} is the respondent's choice of the acceptable minimum excess monetary value of the risky option over a certainty option. Lower values indicate a greater risk tolerance. \textsf{Time preference 1} is the respondent's choice of the acceptable minimum excess monetary value in 3 months that is no smaller than present monetary benefit, and \textsf{Time preference 2} is the the minimum excess value in 1 year and 3 months that is no smaller than monetary benefits of 1 year from now. Lower values indicate a greater patience. If a respondent's \textsf{Time preference 1} is greater than \textsf{Time preference 2}, the respondent is considered to be present-biased. \textsf{Present bias} is an indicator function that takes the value of 1 if the respondent is considered to be present-biased, 0 otherwise.
\end{tabular}
\end{minipage}



\subsection{Changes in assets}


After winsorising cassette players, radios, and bicycles, there is no HH with anomalous asset values (changes in narrow net asset values $<-50000$).


%There are members who report sharp decline in net assets. This is mostly due to radio and casette player entries. There are length(da50K) households among \textsf{o800==1L} whose changes in net total asset $< -50000$ and have assets of values greater than 50000.

%When radios and casset players are excluded from/winsorised in total, there are length(dna50K) households among \textsf{o800==1L} whose changes in net total asset $< -10000$ and have assets of values greater than 10000. Note that residential land is only included in BroadHAssetAmount. 
\begin{Schunk}
\begin{Soutput}
Key: <hhid>
            Arm    hhid     t             type amount     H     BH NLHAssetNum
         <fctr>   <num> <num>           <char>  <int> <int>  <int>       <int>
 1: traditional 8169717     1         tubewell   1500  1500   1500           1
 2: traditional 8169717     2         tubewell   1600  1600   7600           1
 3: traditional 8169717     2 residential land   6000  1600   7600           1
 4: traditional 8169717     3         tubewell   1200 82600 121600           3
 5: traditional 8169717     3  almirah/cabinet   2600 82600 121600           3
 6: traditional 8169717     3          jewelry    400 82600 121600           3
 7: traditional 8169717     3     mobile phone   1400 82600 121600           3
 8: traditional 8169717     3 residential land  36000 82600 121600           3
 9: traditional 8169717     3     rickshaw/van  80000 82600 121600           3
10: traditional 8169717     4         tubewell    400  3300  46400           3
11: traditional 8169717     4  almirah/cabinet   2500  3300  46400           3
12: traditional 8169717     4          jewelry    600  3300  46400           3
13: traditional 8169717     4 residential land  40000  3300  46400           3
14: traditional 8169717     4          bicycle   2500  3300  46400           3
15: traditional 8169717     4     mobile phone    400  3300  46400           3
\end{Soutput}
\end{Schunk}


\subsection{Error bar graphs of outcomes}







\renewcommand{\arraystretch}{.6}
\mpage{\linewidth}{
\hfil\textsc{\footnotesize Figure \refstepcounter{figure}\thefigure: Mean outcomes by arm and period\label{fig MeanOutcomes}}\\
\hfil\includegraphics[width = 11cm]{c:/data/GUK/analysis/program/figure/EstimationMemo/MeanOutcomesByArmAndPeriod.pdf}\\
\renewcommand{\arraystretch}{1}
\hfil\begin{tabular}{>{\hfill\scriptsize}p{.75cm}<{}>{\scriptsize}p{12cm}<{\hfill}}
Source: & Survey data.\\
Note:& Points indicate means, vertical bars indicate 95\% confidence intervals. 
\textsf{NumCows} is number of cattle owned. \textsf{NetValue} is net asset values per housheold for asset items observed in all 4 rounds. %\textsf{NarrowNetValue} is net asset values per housheold for asset items observed in all 4 rounds excluding cassette players, radios. %\textsf{NetBroadValue} is net asset values per housheold for all asset items. \textsf{RNetValue} and \textsf{RNarrowNetValue} are winsorised equivalents to \textsf{RNetValue} and \textsf{NarrowNetValue} with cassette players, radios winsorised at BDT 20000 and bicycles at BDT 2000. %and \textsf{RNetBroadValue} are and \textsf{NetBroadValue}.  
\textsf{Consumption} is annualised per capita consumption in Taka. Per capita consumption is a total of food, hygiene, social, and energy expenditure divided by the number of household members. In-kind consumption of home made products is imputed at median prices. \textsf{HHIncomes} is labour incomes of household, \textsf{pcHHIncomes} is per capita housheold labour incomes. \textsf{Sch0512}, \textsf{Sch1315}, \textsf{Sch1618} are enrollment at primary, secondary, and tertiary levels. \textsf{Female} and \textsf{Male} are female and male enrollment, respectively. \\[1ex]
\end{tabular}
}

\renewcommand{\arraystretch}{.6}
\mpage{\linewidth}{
\hfil\textsc{\footnotesize Figure \refstepcounter{figure}\thefigure: Mean asset outcomes by arm and period\label{fig MeanAssetOutcomes}}\\
\hfil\includegraphics[width = 11cm]{c:/data/GUK/analysis/program/figure/EstimationMemo/MeanAssetOutcomesByArmAndPeriod.pdf}\\
\renewcommand{\arraystretch}{1}
\hfil\begin{tabular}{>{\hfill\scriptsize}p{1cm}<{}>{\scriptsize}p{12cm}<{\hfill}}
Source: & Survey data.\\
Note:& Points indicate means, vertical bars indicate 95\% confidence intervals. \textsf{NetAssets} is total assets less debt outstanding to all sources. 
\textsf{Livestock and productive assets} is total assets less household assets and debt outstanding to all sources. \\[1ex]
\end{tabular}
}

\mpage{\linewidth}{
\hfil\textsc{\footnotesize Figure \refstepcounter{figure}\thefigure: Mean income and consumption outcomes by arm and period\label{fig MeanICOutcomes}}\\
\hfil\includegraphics[width = 6cm]{c:/data/GUK/analysis/program/figure/EstimationMemo/MeanConsumptionIncomeDodge.pdf}\\
\renewcommand{\arraystretch}{1}
\hfil\begin{tabular}{>{\hfill\scriptsize}p{1cm}<{}>{\scriptsize}p{12cm}<{\hfill}}
Source: & Survey data.\\
Note:& Points indicate means, vertical bars indicate 95\% confidence intervals. 
\textsf{Consumption} is annualised per capita consumption in Taka. Per capita consumption is a total of food, hygiene, social, and energy expenditure divided by the number of household members. In-kind consumption of home made products is imputed at median prices. \textsf{Incomes} is labour incomes of household in 1000 Taka units. \\[1ex]
\end{tabular}
}


\subsection{Graphs of repayments}

In \textsc{\normalsize Table \ref{tab DestatByArm using both}}, one sees that later receivers of \textsf{large grace} and \textsf{cattle} arm members could prepare better by saving before disbursement. 
\begin{Schunk}
\begin{Soutput}


Number of obs by Arm and attrition
             AttritIn
Arm             2   3   4   9 Sum
  traditional   6   4  20 144 174
  large         5   2   1 192 200
  large grace  22   3   3 171 199
  cattle        5   5  13 177 200
  Sum          38  14  37 684 773


Number of obs by membership status and attrition
                      AttritIn
BStatus                  2   3   4   9 Sum
  borrower               8   6   8 578 600
  pure saver             0   0   0   0   0
  individual rejection   9   4   1  75  89
  group rejection        9   4   0  55  68
  rejection by flood    12   0  28   0  40
  Sum                   38  14  37 708 797
UseTrimmedSample is FALSE 
\end{Soutput}
\end{Schunk}

One also sees that \textsf{traditional} has lower repayment rates in the 2nd and 3rd loan years. This can be due to lower returns on small assets, or, moral hazard that they get new disbursements irrespective of loan delinquency. 
\begin{Schunk}
\begin{Soutput}
ar : Number of member entries are less than 12 per year (good).
[1] "Year"     "LoanYear" "MtgYear"  "LYear"   
arA : Number of member entries are less than 12 per year (good).
[1] "Year"     "LoanYear" "MtgYear"  "LYear"   
arACompletePanel : Number of member entries are less than 12 per year (good).
[1] "Year"     "LoanYear" "MtgYear"  "LYear"   
\end{Soutput}
\end{Schunk}
One may worry if flood affected repayments. Split sample into flood affected and unaffected. Affected by flood does not seem to change the repayment numbers.
\begin{Schunk}
\begin{Soutput}
ar 
arA 
Flood dummy = 0
              variables traditional    large large grace   cattle   stat
                 <char>      <char>   <char>      <char>   <char> <char>
1: repay in Loan Year-1       56.47    35.57        0.00     0.00    sum
2:  repay in Loan Year1     3238.29  4253.51      566.28   597.21    sum
3:  repay in Loan Year2     2218.53  3924.16     4998.00  4973.81    sum
4:  repay in Loan Year3     2046.90  3836.48     5403.50  4679.49    sum
5:  repay in Loan Year4     3046.93  2820.97     3031.19  2764.97    sum
6:      Total repayment    10607.12 14870.69    13998.96 13015.47    sum
Flood dummy = 1
              variables traditional    large large grace   cattle   stat
                 <char>      <char>   <char>      <char>   <char> <char>
1: repay in Loan Year-1       41.30    50.65        0.00     0.00    sum
2:  repay in Loan Year1     3244.31  4355.89      528.25   497.85    sum
3:  repay in Loan Year2     2052.53  3716.43     4879.90  4303.48    sum
4:  repay in Loan Year3     1920.05  3813.12     5007.63  4362.31    sum
5:  repay in Loan Year4     3190.27  3259.28     2787.62  4714.35    sum
6:      Total repayment    10448.48 15195.37    13203.41 13878.00    sum
arACompletePanel 
Flood dummy = 0
              variables traditional    large large grace   cattle   stat
                 <char>      <char>   <char>      <char>   <char> <char>
1: repay in Loan Year-1       55.32    40.54        0.00     0.00    sum
2:  repay in Loan Year1     2941.92  4347.92      590.32   494.34    sum
3:  repay in Loan Year2     2104.38  3927.98     5139.65  5210.10    sum
4:  repay in Loan Year3     2004.32  3972.16     5451.62  5197.27    sum
5:  repay in Loan Year4     3126.27  2625.90     3052.12  2553.86    sum
6:      Total repayment    10232.21 14914.50    14233.71 13455.58    sum
Flood dummy = 1
              variables traditional    large large grace   cattle   stat
                 <char>      <char>   <char>      <char>   <char> <char>
1: repay in Loan Year-1       12.50    55.75        0.00     0.00    sum
2:  repay in Loan Year1     3399.77  4511.78      526.88   572.72    sum
3:  repay in Loan Year2     2580.30  3827.31     4804.22  4219.91    sum
4:  repay in Loan Year3     2220.68  3858.15     4630.91  3776.60    sum
5:  repay in Loan Year4     3371.37  3057.95     2818.86  4915.04    sum
6:      Total repayment    11584.61 15310.94    12780.87 13484.27    sum
\end{Soutput}
\end{Schunk}
Combine descriptive statistics and produce \LaTeX  tables.


\hspace{-1cm}\begin{minipage}[t]{14cm}
\hfil\textsc{\normalsize Table \refstepcounter{table}\thetable: Descriptive statistics by arm for all households including nonparticipants\label{tab DestatByArm using ar}}\\
\setlength{\tabcolsep}{1pt}
\setlength{\baselineskip}{8pt}
\renewcommand{\arraystretch}{.55}
\hfil\begin{tikzpicture}
\node (tbl) {\input{c:/data/GUK/analysis/save/EstimationMemo/DestatByArmar.tex}};
\end{tikzpicture}\\
\renewcommand{\arraystretch}{.8}
\setlength{\tabcolsep}{1pt}
\begin{tabular}{>{\hfill\scriptsize}p{1cm}<{}>{\hfill\scriptsize}p{.25cm}<{}>{\scriptsize}p{12cm}<{\hfill}}
Source:& \multicolumn{2}{l}{\mpage{12cm}{\scriptsize Estimated with GUK administrative and survey data. Based on data \textsf{ar} which has all survey respondents. Survey respondents include nonparticipants to the experimental part of study.}}\\
Notes: & 1. & Information of original 776 households. Net saving as percentage of loan amount is a mean over loan recipients whose first disbursement is in 2013. Effective repayment is a sum of repayment and net saving. 
\end{tabular}
\end{minipage}

\hspace{-1cm}\begin{minipage}[t]{14cm}
\hfil\textsc{\normalsize Table \refstepcounter{table}\thetable: Descriptive statistics by arm for borrowers\label{tab DestatByArm using arA}}\\
\setlength{\tabcolsep}{1pt}
\setlength{\baselineskip}{8pt}
\renewcommand{\arraystretch}{.55}
\hfil\begin{tikzpicture}
\node (tbl) {\input{c:/data/GUK/analysis/save/EstimationMemo/DestatByArmarA.tex}};
\end{tikzpicture}\\
\renewcommand{\arraystretch}{.8}
\setlength{\tabcolsep}{1pt}
\begin{tabular}{>{\hfill\scriptsize}p{1cm}<{}>{\hfill\scriptsize}p{.25cm}<{}>{\scriptsize}p{12cm}<{\hfill}}
Source:& \multicolumn{2}{l}{\mpage{12cm}{\scriptsize Estimated with GUK administrative and survey data. Based on \textsf{arA} which has only borrowers and does not include nonparticipants.}}\\
Notes: & 1. & Information of borrowing members among original 776 households. Net saving as percentage of loan amount is a mean over loan recipients whose first disbursement is in 2013. Effective repayment is a sum of repayment and net saving. \\
& 2. & \textsf{Loan year} -1 is preparation period for loan disbursement when only saving is allowed. \\
\end{tabular}
\end{minipage}


\hspace{-1cm}\begin{minipage}[t]{14cm}
\hfil\textsc{\normalsize Table \refstepcounter{table}\thetable: Descriptive statistics by arm for borrowers, complete panel\label{tab DestatByArm using arACompletePanel}}\\
\setlength{\tabcolsep}{1pt}
\setlength{\baselineskip}{8pt}
\renewcommand{\arraystretch}{.55}
\hfil\begin{tikzpicture}
\node (tbl) {\input{c:/data/GUK/analysis/save/EstimationMemo/DestatByArmarACompletePanel.tex}};
\end{tikzpicture}\\
\renewcommand{\arraystretch}{.8}
\setlength{\tabcolsep}{1pt}
\begin{tabular}{>{\hfill\scriptsize}p{1cm}<{}>{\hfill\scriptsize}p{.25cm}<{}>{\scriptsize}p{12cm}<{\hfill}}
Source:& \multicolumn{2}{l}{\mpage{12cm}{\scriptsize Estimated with GUK administrative and survey data. Based on \textsf{arACompletePanel} which has only non-attriting members who were surveyed at period 2.}}\\
Notes: & 1. & Information of borrowing members among original 776 households. Net saving as percentage of loan amount is a mean over loan recipients whose first disbursement is in 2013. Effective repayment is a sum of repayment and net saving. \\
& 2. & \textsf{Loan year} -1 is preparation period for loan disbursement when only saving is allowed. \\
\end{tabular}
\end{minipage}

\hspace{-1cm}\begin{minipage}[t]{14cm}
\hfil\textsc{\normalsize Table \refstepcounter{table}\thetable: Descriptive statistics by arm for all members and borrowing members\label{tab DestatByArm using both}}\\
\setlength{\tabcolsep}{1pt}
\setlength{\baselineskip}{8pt}
\renewcommand{\arraystretch}{.55}
\hfil\begin{tikzpicture}
\node (tbl) {\input{c:/data/GUK/analysis/save/EstimationMemo/DestatByArmBoth.tex}};
\end{tikzpicture}\\
\renewcommand{\arraystretch}{.8}
\setlength{\tabcolsep}{1pt}
\begin{tabular}{>{\hfill\scriptsize}p{1cm}<{}>{\hfill\scriptsize}p{.25cm}<{}>{\scriptsize}p{12cm}<{\hfill}}
Source:& \multicolumn{2}{l}{\mpage{12cm}{\scriptsize Estimated with GUK administrative and survey data. Based on data \textsf{ar} which has all survey respondents. }}\\
Note: &  \multicolumn{2}{l}{\scriptsize All members are 776 households. Survey respondents include nonparticipants to the experimental part of study.  }
\end{tabular}
\end{minipage}


\section{Estimation using initial sample HHs}


\subsection{Repayment and net saving}

In estimating impacts on repayment and saving, we use borrower only data \textsf{arA}.

\begin{Schunk}
\begin{Soutput}
           used  (Mb) gc trigger   (Mb) limit (Mb)  max used   (Mb)
Ncells  1329364  71.0    2616520  139.8         NA   2616520  139.8
Vcells 91392612 697.3  145848282 1112.8      56320 145845570 1112.8
\end{Soutput}
\end{Schunk}
By survey rounds, in repayment and saving file, there are 28, 561, 555, 554 observations of households in rounds 1, 2, 3, 4, respectively. This is smaller than the \textsf{InitialSample} size of 776 in the survey roster file because the survey includes rejecters and residents whose houses are washed away by flood, while repayment is defined only for the borrowers.

Saving started in rd 1. Repayment and saving are more frequent than survey rounds. In repayment and saving regressions, we aggregate the data at survey rounds. This is because we have no household survey information at the monthly frequency that we can attribute the causes of monthly repayment and saving fluctuations.

\hspace{-1cm}\begin{minipage}[t]{14cm}
\hfil\textsc{\normalsize Table \refstepcounter{table}\thetable: Initial sample by arm in administrative data\label{tab IniSampByArm}}\\
\setlength{\tabcolsep}{1pt}
\setlength{\baselineskip}{8pt}
\renewcommand{\arraystretch}{.55}
%\hfil\begin{tikzpicture}
%\node (tbl) {\input{  paste0(pathsaveHere, "InitialSampleSizeByArm.tex")}};
%\end{tikzpicture}\\
\hfil\begin{tikzpicture}
\node (tbl) {\input{c:/data/GUK/analysis/save/EstimationMemo/InitialSampleSizeByArmInAr.tex}};
\end{tikzpicture}\\
\renewcommand{\arraystretch}{.8}
\setlength{\tabcolsep}{1pt}
\begin{tabular}{>{\hfill\scriptsize}p{1cm}<{}>{\hfill\scriptsize}p{.25cm}<{}>{\scriptsize}p{12cm}<{\hfill}}
Source:& \multicolumn{2}{l}{\scriptsize Estimated with GUK administrative and survey data.}\\
Notes: & 1. & Number of individuals who received a loan/cow. %within 6 months after the first loan/cow is disbursed in a group. 
Left panel are initial 800 members who were offered at the first round, including individuals who declined or left the group. Right panel also includes members who were offered on a later date.\\
\end{tabular}
\end{minipage}

\hspace{-1cm}\begin{minipage}[t]{14cm}
\hfil\textsc{\normalsize Table \refstepcounter{table}\thetable: Initial sample by arm in repayment data\label{tab IniSampByArmArA}}\\
\setlength{\tabcolsep}{1pt}
\setlength{\baselineskip}{8pt}
\renewcommand{\arraystretch}{.55}
\hfil\begin{tikzpicture}
\node (tbl) {\input{c:/data/GUK/analysis/save/EstimationMemo/InitialSampleSizeByArmInArA.tex}};
\end{tikzpicture}\\
\renewcommand{\arraystretch}{.8}
\setlength{\tabcolsep}{1pt}
\begin{tabular}{>{\hfill\scriptsize}p{1cm}<{}>{\hfill\scriptsize}p{.25cm}<{}>{\scriptsize}p{12cm}<{\hfill}}
Source:& \multicolumn{2}{l}{\scriptsize Estimated with GUK administrative and survey data.}\\
Notes: & 1. & Number of individuals who received a loan/cow. %within 6 months after the first loan/cow is disbursed in a group. 
Left panel in \textsc{\normalsize Table \ref{tab IniSampByArmArA}} is initial 800 members who were offered at the first round, including individuals who declined or left the group. Right panel also includes members who were offered on a later date.\\
\end{tabular}
\end{minipage}

\textsc{\normalsize Table \ref{tab IniSampByArm}} shows the tabuation of \textsf{InitisalSample} by arms. Left panel are \textsf{InitialSample} including borrowers, pure savers, group rejecters, flood victims, and members who left the group. Right panel includes late borrowers who were initially assigned as the control. One can see that \textsf{traditional} arm members have the highest proportion of group-rejecters and individual rejecters. This shows stronger reluctance of \textsf{traditional} arm members in borrowing the small loans. 


Read administrative meeting data attached with HH information \textsf{AllMeetingsRepayment} (\textsf{arA}). Note all binary interaction terms are demeaned and then interacted.
\begin{Schunk}
\begin{Soutput}
Error in file(filename, "r", encoding = encoding): コネクションを開くことができません
\end{Soutput}
\end{Schunk}
%NAs in \textsf{CumRepaid}.

Tabulation at rd 1 (12th month):
\begin{Schunk}
\begin{Soutput}
              Arm
Mstatus        traditional large large grace cattle Sum
  gErosion               0     0           0      0   0
  gRejection             0     0           0      0   0
  iRejection             0     0           0      0   0
  iReplacement           0     0           0      0   0
  newGroup               0     0           0      0   0
  oldMember             85   171         167    153 576
  Sum                   85   171         167    153 576
\end{Soutput}
\end{Schunk}
\begin{Schunk}
\begin{figure}

{\centering \includegraphics[width=\maxwidth]{figure/EstimationMemo/net_saving_and_cumulative_repayment-1} 

}

\caption[Weekly net saving and cumulative repayment]{Weekly net saving and cumulative repayment}\label{Figure net saving and cumulative repayment}
\end{figure}
\end{Schunk}


\hfil\textsc{\footnotesize Figure \refstepcounter{figure}\thefigure: Weekly net saving and cumulative repayment\label{fig weeklysavingrepay}}\\
\hfil\includegraphics[width = 12cm]{c:/data/GUK/analysis/program/figure/ImpactEstimationOriginal1600Memo2/CumulativeWeeklyNetSavingAndRepayment.png}\\
\renewcommand{\arraystretch}{1}
\hfil\begin{tabular}{>{\hfill\scriptsize}p{1cm}<{}>{\scriptsize}p{12cm}<{\hfill}}
Note:& Each dot represents weekly observations. Only members who received loans are shown. Each panel shows weekly net saving (saving - withdrawal) or cumulative repayment against weeks after first disbursement. Lines are smoothed lines with a penalized cubic regression spline in \textsf{ggplot2::geom\_smooth} function, originally from \textsf{mgcv::gam} with \textsf{bs=`cs'}. \\[-1ex]
\end{tabular}


\hfil\textsc{\footnotesize Figure \refstepcounter{figure}\thefigure: Cumulative weekly net repayment rates\label{fig weeklysavingrepayrate}}\\
\hfil\includegraphics{c:/data/GUK/analysis/program/figure/ImpactEstimationOriginal1600Memo2/CumulativeWeeklyRepaymentRateByPovertystatus.png}\\
\renewcommand{\arraystretch}{1}
\hfil\begin{tabular}{>{\hfill\scriptsize}p{1cm}<{}>{\scriptsize}p{12cm}<{\hfill}}
Note:& Each dot represents weekly observations. Only members who received loans are shown. Each panel shows ratios of cumulative repayment against cumulative due amount, sum of cumulative repayment and cumulative net saving (saving - withdrawal) against cumulative due amount, against weeks after first disbursement. Lines are smoothed lines with a penalized cubic regression spline in \textsf{ggplot2::geom\_smooth} function, originally from \textsf{mgcv::gam} with \textsf{bs=`cs'}. \\[-1ex]
\end{tabular}





\begin{Schunk}
\begin{Soutput}
            used   (Mb) gc trigger   (Mb) limit (Mb)  max used   (Mb)
Ncells   2538872  135.6    4521784  241.5         NA   4521784  241.5
Vcells 275675177 2103.3  436295027 3328.7      56320 363499075 2773.3
\end{Soutput}
\end{Schunk}


\hspace{-1cm}\begin{minipage}[t]{14cm}
\hfil\textsc{\normalsize Table \refstepcounter{table}\thetable: ANCOVA estimation of net saving and repayment\label{tab ANCOVA Repayment}}\\
\setlength{\tabcolsep}{1pt}
\setlength{\baselineskip}{8pt}
\renewcommand{\arraystretch}{.55}
\hspace{-.75cm}\begin{tikzpicture}
\node (tbl) {\input{c:/data/GUK/analysis/save/EstimationMemo/RepaymentANCOVAEstimationResults.tex}};
%\input{c:/dropbox/data/ramadan/save/tablecolortemplate.tex}
\end{tikzpicture}\\
\renewcommand{\arraystretch}{.8}
\setlength{\tabcolsep}{1pt}
\begin{tabular}{>{\hfill\scriptsize}p{1cm}<{}>{\hfill\scriptsize}p{.25cm}<{}>{\scriptsize}p{12cm}<{\hfill}}
Source:& \multicolumn{2}{l}{\scriptsize Estimated with GUK administrative and survey data.}\\
Notes: & 1. & ANCOVA estimates using administrative and survey data. Post treatment regressands are regressed on categorical variables, pre-treatment regressand and other covariates. \textsf{FloodInRd1} and \textsf{HeadLiterate0} are indicator variables for the presence of self reported damage by a flood at the baseline, and literacy of household head, respectively. \textsf{HHsize0} is household size at the baseline. We annotate the number of periods that a household is observed with \textsf{T}. The total number of households is shown for each values of \textsf{T}. \textsf{T=4} indicates the number of households with complete panel information, \textsf{T=3} indicates number of households observed three times, \textsf{T=2} indicates the number of households observed twice. \textsf{N} indicates total number of observations used in ANCOVA estimation, or \textsf{N$=$1$\times$(T=2)+2$\times$(T=3)+3$\times$(T=4)}.  Saving and repayment information is taken from administrative data. Net saving is saving - withdrawal. %Excess repayment is repayment - due amount. 
\textsf{LY2, LY3, LY4} are dummy variables for second, third, and fourth year into borrowing.  Repayment starts from the year 1 for \textsf{traditional} and \textsf{large} arms, from the year 2 for \textsf{large grace} and \textsf{cattle} arms. The first regression of repayment gives a mean monthly repayment for each arms. Mean monthly repayment is zero in the year 1 for \textsf{large grace} and \textsf{cattle} arms.\\
& 2. & $P$ values in percentages in parenthesises. Standard errors are clustered at group (village) level.
%${}^{***}$, ${}^{**}$, ${}^{*}$ indicate statistical significance at 1\%, 5\%, 10\%, respetively. Standard errors are clustered at group (village) level.
\end{tabular}
\end{minipage}


\hspace{-1cm}\begin{minipage}[t]{14cm}
\hfil\textsc{\normalsize Table \refstepcounter{table}\thetable: ANCOVA estimation of net saving and repayment by attributes\label{tab ANCOVA Repayment attributes}}\\
\setlength{\tabcolsep}{1pt}
\setlength{\baselineskip}{8pt}
\renewcommand{\arraystretch}{.55}
\hspace{-.75cm}\begin{tikzpicture}
\node (tbl) {\input{c:/data/GUK/analysis/save/EstimationMemo/RepaymentAttributesANCOVAEstimationResults.tex}};
%\input{c:/dropbox/data/ramadan/save/tablecolortemplate.tex}
\end{tikzpicture}\\
\renewcommand{\arraystretch}{.8}
\setlength{\tabcolsep}{1pt}
\begin{tabular}{>{\hfill\scriptsize}p{1cm}<{}>{\hfill\scriptsize}p{.25cm}<{}>{\scriptsize}p{12cm}<{\hfill}}
Source:& \multicolumn{2}{l}{\scriptsize Estimated with GUK administrative and survey data.}\\
Notes: & 1. & ANCOVA estimates using administrative and survey data. Post treatment regressands are regressed on categorical variables, pre-treatment regressand and other covariates. \textsf{FloodInRd1} and \textsf{HeadLiterate0} are indicator variables for the presence of self reported damage by a flood at the baseline, and literacy of household head, respectively. \textsf{HHsize0} is household size at the baseline. We annotate the number of periods that a household is observed with \textsf{T}. The total number of households is shown for each values of \textsf{T}. \textsf{T=4} indicates the number of households with complete panel information, \textsf{T=3} indicates number of households observed three times, \textsf{T=2} indicates the number of households observed twice. \textsf{N} indicates total number of observations used in ANCOVA estimation, or \textsf{N$=$1$\times$(T=2)+2$\times$(T=3)+3$\times$(T=4)}.  Saving and repayment information is taken from administrative data. Net saving is saving - withdrawal. %Excess repayment is repayment - due amount. 
\textsf{LY2, LY3, LY4} are dummy variables for second, third, and fourth year into borrowing.  Repayment starts from the year 2 for \textsf{WithGrace} functional attributes. The first regression of repayment gives a mean monthly repayment for each arms. Mean monthly repayment is zero in the year 1 for \textsf{WithGrace} functional attributes.\\
& 2. & $P$ values in percentages in parenthesises. Standard errors are clustered at group (village) level.
%${}^{***}$, ${}^{**}$, ${}^{*}$ indicate statistical significance at 1\%, 5\%, 10\%, respetively. Standard errors are clustered at group (village) level.
\end{tabular}
\end{minipage}

\hspace{-1cm}\begin{minipage}[t]{14cm}
\hfil\textsc{\normalsize Table \refstepcounter{table}\thetable: ANCOVA estimation of net saving and repayment, ultra poor vs. moderately poor\label{tab ANCOVA Repayment poverty}}\\
\setlength{\tabcolsep}{1pt}
\setlength{\baselineskip}{8pt}
\renewcommand{\arraystretch}{.55}
\hspace{-.75cm}\begin{tikzpicture}
\node (tbl) {\input{c:/data/GUK/analysis/save/EstimationMemo/RepaymentPovertyStatusANCOVAEstimationResults.tex}};
%\input{c:/dropbox/data/ramadan/save/tablecolortemplate.tex}
\end{tikzpicture}\\
\renewcommand{\arraystretch}{.8}
\setlength{\tabcolsep}{1pt}
\begin{tabular}{>{\hfill\scriptsize}p{1cm}<{}>{\hfill\scriptsize}p{.25cm}<{}>{\scriptsize}p{12cm}<{\hfill}}
Source:& \multicolumn{2}{l}{\scriptsize Estimated with GUK administrative and survey data.}\\
Notes: & 1. & ANCOVA estimates using administrative and survey data. Post treatment regressands are regressed on categorical variables, pre-treatment regressand and other covariates. \textsf{FloodInRd1} and \textsf{HeadLiterate0} are indicator variables for the presence of self reported damage by a flood at the baseline, and literacy of household head, respectively. \textsf{HHsize0} is household size at the baseline. We annotate the number of periods that a household is observed with \textsf{T}. The total number of households is shown for each values of \textsf{T}. \textsf{T=4} indicates the number of households with complete panel information, \textsf{T=3} indicates number of households observed three times, \textsf{T=2} indicates the number of households observed twice. \textsf{N} indicates total number of observations used in ANCOVA estimation, or \textsf{N$=$1$\times$(T=2)+2$\times$(T=3)+3$\times$(T=4)}.  \textsf{Upfront} is an indicator variable of the arm with an upfront large disbursement, \textsf{WithGrace} is an indicator variable of the arm with a grace period, \textsf{InKind} is an indicator variable of the arm which lends a heifer. \textsf{UltraPoor} is an indicator variable if the household is classified as the ultra poor. Saving and repayment information is taken from administrative data. Net saving is saving - withdrawal. %Excess repayment is repayment - due amount. 
\textsf{LY2, LY3, LY4} are dummy variables for second, third, and fourth year into borrowing.  Repayment starts from the year 2 for \textsf{WithGrace} functional attributes. The first regression of repayment gives a mean monthly repayment for each arms. Mean monthly repayment is zero in the year 1 for \textsf{WithGrace} functional attributes.\\
& 2. & $P$ values in percentages in parenthesises. Standard errors are clustered at group (village) level.
%${}^{***}$, ${}^{**}$, ${}^{*}$ indicate statistical significance at 1\%, 5\%, 10\%, respetively. Standard errors are clustered at group (village) level.
\end{tabular}
\end{minipage}

\hspace{-1cm}\begin{minipage}[t]{14cm}
\hfil\textsc{\normalsize Table \refstepcounter{table}\thetable: ANCOVA estimation of net saving and repayment, ultra poor vs. moderately poor, time varying\label{tab ANCOVA Repayment time varying poverty}}\\
\setlength{\tabcolsep}{1pt}
\setlength{\baselineskip}{8pt}
\renewcommand{\arraystretch}{.55}
\hspace{-.75cm}\begin{tikzpicture}
\node (tbl) {\input{c:/data/GUK/analysis/save/EstimationMemo/RepaymentTimeVaryingPovertyStatusANCOVAEstimationResults.tex}};
%\input{c:/dropbox/data/ramadan/save/tablecolortemplate.tex}
\end{tikzpicture}\\
\renewcommand{\arraystretch}{.8}
\setlength{\tabcolsep}{1pt}
\begin{tabular}{>{\hfill\scriptsize}p{1cm}<{}>{\hfill\scriptsize}p{.25cm}<{}>{\scriptsize}p{12cm}<{\hfill}}
Source:& \multicolumn{2}{l}{\scriptsize Estimated with GUK administrative and survey data.}\\
Notes: & 1. & ANCOVA estimates using administrative and survey data. Post treatment regressands are regressed on categorical variables, pre-treatment regressand and other covariates. \textsf{FloodInRd1} and \textsf{HeadLiterate0} are indicator variables for the presence of self reported damage by a flood at the baseline, and literacy of household head, respectively. \textsf{HHsize0} is household size at the baseline. We annotate the number of periods that a household is observed with \textsf{T}. The total number of households is shown for each values of \textsf{T}. \textsf{T=4} indicates the number of households with complete panel information, \textsf{T=3} indicates number of households observed three times, \textsf{T=2} indicates the number of households observed twice. \textsf{N} indicates total number of observations used in ANCOVA estimation, or \textsf{N$=$1$\times$(T=2)+2$\times$(T=3)+3$\times$(T=4)}.  \textsf{Upfront} is an indicator variable of the arm with an upfront large disbursement, \textsf{WithGrace} is an indicator variable of the arm with a grace period, \textsf{InKind} is an indicator variable of the arm which lends a heifer. \textsf{UltraPoor} is an indicator variable if the household is classified as the ultra poor. Saving and repayment information is taken from administrative data. Net saving is saving - withdrawal. %Excess repayment is repayment - due amount. 
\textsf{LY2, LY3, LY4} are dummy variables for second, third, and fourth year into borrowing.  Repayment starts from the year 2 for \textsf{WithGrace} functional attributes. The first regression of repayment gives a mean monthly repayment for each arms. Mean monthly repayment is zero in the year 1 for \textsf{WithGrace} functional attributes.\\
& 2. & $P$ values in percentages in parenthesises. Standard errors are clustered at group (village) level.
%${}^{***}$, ${}^{**}$, ${}^{*}$ indicate statistical significance at 1\%, 5\%, 10\%, respetively. Standard errors are clustered at group (village) level.
\end{tabular}
\end{minipage}


\begin{palepinkleftbar}
\begin{finding}
\textsc{\small Table \ref{tab ANCOVA Repayment}} shows regression results for net saving, repayment, and effective repayment (net saving + repayment) using monthly administrative data. Monthly mean repayment is given by 48 times the estimated values in colum (5). One sees that \textsf{traditional} has the lowest mean repayment. It is shown that they repaid loan year 2 and 3 

\end{finding}
\end{palepinkleftbar}


\begin{Schunk}
\begin{Soutput}
             FullyRepaid
Arm             0   1 sum
  traditional  85   0  85
  large       167   4 171
  large grace 163   4 167
  cattle      152   1 153
  sum         567   9 576
\end{Soutput}
\end{Schunk}



\begin{palepinkleftbar}
\begin{finding}
\textsc{Figure \ref{fig weeklysavingrepay}} visually presents that repayment is no different between the ultra poor and the moderately poor. The subsequent regression table econometrically confirms this (\textsc{\normalsize Table \ref{tab FD saving2 original HH}}). 
\end{finding}
\end{palepinkleftbar}

\subsection{Schooling}

	ANCOVA is a model that controls for preexisting differences by including initial values of $y$ as a covariate, traditionally for a continuous variable as a nuissance to estimated impacts of a categorical variable (treated/control). In enrollment regressions, initial enrollment is informative only for school age children at the initial period. ANCOVA estimates should be used only to school age children in 2012 who are not old enough that they may naturally stop schooling by the endline.


\begin{Schunk}
\begin{Soutput}
Dropped 902 obs due to NA.
Dropped 902 obs due to NA.
Dropped 223 obs due to T<2.
Dropped 574 obs due to NA.
\end{Soutput}
\end{Schunk}
Enrollment pattern in original schooling panel. `n' indicates NA (either attrition or not reported).
\begin{Schunk}
\begin{Soutput}
         SchPattern
ObPattern 0000 0001 000n 0011 001n 00nn 010n 0111 011n 01nn 0nnn 1000 1001 100n
     0111    0    0    0    0    0    0    0    0    2    1    2    0    0    0
     1000    0    0    0    0    0    0    0    0    0    0   32    0    0    0
     1010    0    0    0    0    0    1    0    0    0    0    2    0    0    0
     1011    0    0    0    0    0    0    0    0    0    0    0    0    0    0
     1100    0    0    0    0    0    0    0    0    0    3    1    0    0    0
     1110    0    0    5    0    2    1    0    0    3    0    2    0    0    1
     1111   21    2   21   12    2   28    1   83    8    6   68    5    1    4
         SchPattern
ObPattern 1011 101n 10nn 1100 1101 110n 1110 1111 111n 11n1 11nn 1nnn
     0111    0    0    0    0    0    0    0    0    6    0    0    1
     1000    0    0    0    0    0    0    0    0    0    0    0   22
     1010    0    0    0    0    0    0    0    0    0    0    1    2
     1011    0    0    0    0    0    0    0    0    0    0    0    0
     1100    0    0    0    0    0    0    0    0    0    0    6    0
     1110    0    1    0    0    0    0    0    0   25    0    3    0
     1111    6    3   10    8    1    9    3  397   41    1   29   42
\end{Soutput}
\end{Schunk}
Left panel is before dropping \textsf{nnn}, right panel is after: Original panel.
\begin{Schunk}
\begin{Soutput}
  traditional large large grace cattle traditional large large grace cattle
1         205   246         251    235         166   208         186    203
2         166   208         186    203         166   208         186    203
3         148   184         168    173         148   184         168    173
4         113   161         135    131         113   161         135    131
\end{Soutput}
\end{Schunk}
\textsf{sch} has 2913 rows. Drop 174 observations in \textsf{sch} with nnn in \textsf{SchPattern}. 
%and nrow(s.1x[!grepl("nnn", Spattern) & grepl("1001", EnrollPattern), ]) observations with 1001 in \textsf{EnrollPattern} because they are likely to be errors. This leaves us with nrow(s1x) rows. 

With OLS,  89, 135, 539 individuals are repeatedly observed for 2, 3, 4 times, respectively. %With ANCOVA, \textsf{sch} is reduced to rows.
Number of individuals with NAs in \textsf{Enrolled}: 0 obs for \textsf{sch}. 
%Mostly older children (round(mean(s.1[is.na(Enrolled), Age_1], na.rm = T), 1) in \textsf{s1x}, round(mean(s.2[is.na(Enrolled), Age_1], na.rm = T), 1) in \textsf{s.2}) but with a high reported enrollment rate (round(mean(s.1[is.na(Enrolled) & tee == 4, Enrolled]), 1) for \textsf{s1x}, round(mean(s.2[is.na(Enrolled) & tee == 4, Enrolled]), 1) for \textsf{s.2}) at rd 4. We will substitute relevant schooling levels to \textsf{Enrolled}.
Check missingness in junior or high school level information at baseline.
\begin{Schunk}
\begin{Sinput}
addmargins(table0(s1x[o800 == 1L & tee == 1, .(dummyJunior, dummyHigh)]))
\end{Sinput}
\begin{Soutput}
           dummyHigh
dummyJunior   0   1 Sum
        0   610  37 647
        1   116   0 116
        Sum 726  37 763
\end{Soutput}
\end{Schunk}
Drop 610 obs without school level information. (...?)

Read school initial sample data.
\begin{Schunk}
\begin{Soutput}
            used   (Mb) gc trigger   (Mb) limit (Mb)  max used   (Mb)
Ncells   2539343  135.7    4521784  241.5         NA   4521784  241.5
Vcells 273865419 2089.5  436295027 3328.7      56320 363499075 2773.3
\end{Soutput}
\end{Schunk}







% Old: Below is created in FDEstimationFile.R via ImpactEstimationOriginal1600Memo3_body3.rnw
% Below is created in FDEstimationFile.R via SchoolingOriginal1600HHsToProduce2Tables.rnw

\hspace{-1cm}\begin{minipage}[t]{14cm}
\hfil\textsc{\normalsize Table \refstepcounter{table}\thetable: FD estimation of school enrollment, round 1 vs. round 4 differences\label{tab FD enroll5 original HH}}\\
\setlength{\tabcolsep}{1pt}
\setlength{\baselineskip}{8pt}
\renewcommand{\arraystretch}{.55}
\hfil\begin{tikzpicture}
\node (tbl) {\input{c:/data/GUK/analysis/save/EstimationMemo/SchoolingRd14DiffOriginalHHsFDEstimationResults.tex}};
%\input{c:/dropbox/data/ramadan/save/tablecolortemplate.tex}
\end{tikzpicture}\\
\renewcommand{\arraystretch}{.8}
\setlength{\tabcolsep}{1pt}
\begin{tabular}{>{\hfill\scriptsize}p{1cm}<{}>{\hfill\scriptsize}p{.25cm}<{}>{\scriptsize}p{12cm}<{\hfill}}
Source:& \multicolumn{2}{l}{\scriptsize Estimated with GUK administrative and survey data.}\\
Notes: & 1. & ANCOVA estimates using administrative and survey data. Post treatment regressands are regressed on categorical variables, pre-treatment regressand and other covariates. \textsf{FloodInRd1} and \textsf{HeadLiterate0} are indicator variables for the presence of self reported damage by a flood at the baseline, and literacy of household head, respectively. \textsf{HHsize0} is household size at the baseline. We annotate the number of periods that a household is observed with \textsf{T}. The total number of households is shown for each values of \textsf{T}. \textsf{T=4} indicates the number of households with complete panel information, \textsf{T=3} indicates number of households observed three times, \textsf{T=2} indicates the number of households observed twice. \textsf{N} indicates total number of observations used in ANCOVA estimation, or \textsf{N$=$1$\times$(T=2)+2$\times$(T=3)+3$\times$(T=4)}. \\
& 2. & $P$ values in percentages in parenthesises. Standard errors are clustered at group (village) level.%
%${}^{***}$, ${}^{**}$, ${}^{*}$ indicate statistical significance at 1\%, 5\%, 10\%, respetively. Standard errors are clustered at group (village) level.
\end{tabular}
\end{minipage}

\hspace{-1cm}\begin{minipage}[t]{14cm}
\hfil\textsc{\normalsize Table \refstepcounter{table}\thetable: FD estimation of school enrollment, round 1 vs. round 4 differences by attributes\label{tab FD enroll5 attributes original HH}}\\
\setlength{\tabcolsep}{1pt}
\setlength{\baselineskip}{8pt}
\renewcommand{\arraystretch}{.55}
\hfil\begin{tikzpicture}
\node (tbl) {\input{c:/data/GUK/analysis/save/EstimationMemo/SchoolingRd14DiffAttributesOriginalHHsFDEstimationResults.tex}};
%\input{c:/dropbox/data/ramadan/save/tablecolortemplate.tex}
\end{tikzpicture}\\
\renewcommand{\arraystretch}{.8}
\setlength{\tabcolsep}{1pt}
\begin{tabular}{>{\hfill\scriptsize}p{1cm}<{}>{\hfill\scriptsize}p{.25cm}<{}>{\scriptsize}p{12cm}<{\hfill}}
Source:& \multicolumn{2}{l}{\scriptsize Estimated with GUK administrative and survey data.}\\
Notes: & 1. & ANCOVA estimates using administrative and survey data. Post treatment regressands are regressed on categorical variables, pre-treatment regressand and other covariates. \textsf{FloodInRd1} and \textsf{HeadLiterate0} are indicator variables for the presence of self reported damage by a flood at the baseline, and literacy of household head, respectively. \textsf{HHsize0} is household size at the baseline. We annotate the number of periods that a household is observed with \textsf{T}. The total number of households is shown for each values of \textsf{T}. \textsf{T=4} indicates the number of households with complete panel information, \textsf{T=3} indicates number of households observed three times, \textsf{T=2} indicates the number of households observed twice. \textsf{N} indicates total number of observations used in ANCOVA estimation, or \textsf{N$=$1$\times$(T=2)+2$\times$(T=3)+3$\times$(T=4)}.  \textsf{Upfront} is an indicator variable of the arm with an upfront large disbursement, \textsf{WithGrace} is an indicator variable of the arm with a grace period, \textsf{InKind} is an indicator variable of the arm which lends a heifer.\\
& 2. & $P$ values in percentages in parenthesises. Standard errors are clustered at group (village) level.
%${}^{***}$, ${}^{**}$, ${}^{*}$ indicate statistical significance at 1\%, 5\%, 10\%, respetively. Standard errors are clustered at group (village) level.
\end{tabular}
\end{minipage}



\hspace{-1cm}\begin{minipage}[t]{14cm}
\hfil\textsc{\normalsize Table \refstepcounter{table}\thetable: ANCOVA estimation of school enrollment\label{tab ANCOVA enroll}}\\
\setlength{\tabcolsep}{1pt}
\setlength{\baselineskip}{8pt}
\renewcommand{\arraystretch}{.525}
\hfil\begin{tikzpicture}
\node (tbl) {\input{c:/data/GUK/analysis/save/EstimationMemo/SchoolingANCOVAEstimationResults.tex}};
%\input{c:/dropbox/data/ramadan/save/tablecolortemplate.tex}
\end{tikzpicture}\\
\renewcommand{\arraystretch}{.7}
\setlength{\tabcolsep}{1pt}
\hspace{-1cm}\begin{tabular}{>{\hfill\scriptsize}p{1cm}<{}>{\hfill\scriptsize}p{.25cm}<{}>{\scriptsize}p{15cm}<{\hfill}}
Source:& \multicolumn{2}{l}{\scriptsize Estimated with GUK administrative and survey data.}\\
Notes: & 1. & ANCOVA estimates using administrative and survey data. Post treatment regressands are regressed on categorical variables, pre-treatment regressand and other covariates. \textsf{FloodInRd1} and \textsf{HeadLiterate0} are indicator variables for the presence of self reported damage by a flood at the baseline, and literacy of household head, respectively. \textsf{HHsize0} is household size at the baseline. We annotate the number of periods that a household is observed with \textsf{T}. The total number of households is shown for each values of \textsf{T}. \textsf{T=4} indicates the number of households with complete panel information, \textsf{T=3} indicates number of households observed three times, \textsf{T=2} indicates the number of households observed twice. \textsf{N} indicates total number of observations used in ANCOVA estimation, or \textsf{N$=$1$\times$(T=2)+2$\times$(T=3)+3$\times$(T=4)}.  \textsf{Large}, \textsf{LargeGrace}, \textsf{Cattle} are indicator variables of the \textsf{large}, \textsf{large grace}, and \textsf{cattle} arms, respectively. The default arm category is \textsf{traditional} arm. \textsf{Secondary} and \textsf{College} are indicator variables of secondary schooling (ages 13-15) and tertiary schooling (ages 16-18), both at the time of baseline. Default category is primary (ages 05-12). Interaction terms of dummy variables are demeaned before interacting. The first column gives mean and standard deviation (in parenthesises) of each covariates before demeaning.\\
& 2. & $P$ values in percentages in parenthesises. Standard errors are clustered at group (village) level.%
%${}^{***}$, ${}^{**}$, ${}^{*}$ indicate statistical significance at 1\%, 5\%, 10\%, respetively. Standard errors are clustered at group (village) level.
\end{tabular}
\end{minipage}

\vspace{-1cm}
\hspace{-1cm}\begin{minipage}[t]{14cm}
\hfil\textsc{\normalsize Table \refstepcounter{table}\thetable: ANCOVA estimation of school enrollment by attributes\label{tab ANCOVA enroll attributes}}\\
\setlength{\tabcolsep}{1pt}
\setlength{\baselineskip}{8pt}
\renewcommand{\arraystretch}{.525}
\hfil\begin{tikzpicture}
\node (tbl) {\input{c:/data/GUK/analysis/save/EstimationMemo/SchoolingAttributesANCOVAEstimationResults.tex}};
%\input{c:/dropbox/data/ramadan/save/tablecolortemplate.tex}
\end{tikzpicture}\\
\renewcommand{\arraystretch}{.7}
\setlength{\tabcolsep}{1pt}
\hspace{-1cm}\begin{tabular}{>{\hfill\scriptsize}p{1cm}<{}>{\hfill\scriptsize}p{.25cm}<{}>{\scriptsize}p{14cm}<{\hfill}}
Source:& \multicolumn{2}{l}{\scriptsize Estimated with GUK administrative and survey data.}\\
Notes: & 1. & \begin{minipage}[t]{14cm}ANCOVA estimates using administrative and survey data. Post treatment regressands are regressed on categorical variables, pre-treatment regressand and other covariates. \textsf{FloodInRd1} and \textsf{HeadLiterate0} are indicator variables for the presence of self reported damage by a flood at the baseline, and literacy of household head, respectively. \textsf{HHsize0} is household size at the baseline. We annotate the number of periods that a household is observed with \textsf{T}. The total number of households is shown for each values of \textsf{T}. \textsf{T=4} indicates the number of households with complete panel information, \textsf{T=3} indicates number of households observed three times, \textsf{T=2} indicates the number of households observed twice. \textsf{N} indicates total number of observations used in ANCOVA estimation, or \textsf{N$=$1$\times$(T=2)+2$\times$(T=3)+3$\times$(T=4)}.  \textsf{Upfront} is an indicator variable of the arm with an upfront large disbursement, \textsf{WithGrace} is an indicator variable of the arm with a grace period, \textsf{InKind} is an indicator variable of the arm which lends a heifer. \textsf{Secondary} and \textsf{College} are indicator variables of secondary schooling (ages 13-15) and tertiary schooling (ages 16-18), both at the time of baseline. Default category is primary (ages 05-12). Interaction terms of dummy variables are demeaned before interacting. The first column gives mean and standard deviation (in parenthesises) of each covariates before demeaning. \setlength{\baselineskip}{2pt}\end{minipage}\\
& 2. & $P$ values in percentages in parenthesises. Standard errors are clustered at group (village) level.%
%${}^{***}$, ${}^{**}$, ${}^{*}$ indicate statistical significance at 1\%, 5\%, 10\%, respetively. Standard errors are clustered at group (village) level.
\end{tabular}
\end{minipage}

$  $\\
\phantom{a}\\

\vspace{-3.0cm}
\hspace{-1cm}\begin{minipage}[t]{14cm}
\hfil\textsc{\normalsize Table \refstepcounter{table}\thetable: ANCOVA estimation of school enrollment by poverty status\label{tab ANCOVA enroll poverty}}\\
\setlength{\tabcolsep}{1pt}
\setlength{\baselineskip}{8pt}
\renewcommand{\arraystretch}{.5}
\hfil\begin{tikzpicture}
\node (tbl) {\input{c:/data/GUK/analysis/save/EstimationMemo/SchoolingPovertyStatusANCOVAEstimationResults.tex}};
%\input{c:/dropbox/data/ramadan/save/tablecolortemplate.tex}
\end{tikzpicture}\\
\renewcommand{\arraystretch}{.8}
\setlength{\tabcolsep}{1pt}
\hspace{-1cm}\begin{tabular}{>{\hfill\scriptsize}p{1cm}<{}>{\hfill\scriptsize}p{.25cm}<{}>{\scriptsize}p{14cm}<{\hfill}}
Source:& \multicolumn{2}{l}{\scriptsize Estimated with GUK administrative and survey data.}\\
Notes: & 1. & ANCOVA estimates using administrative and survey data. Post treatment regressands are regressed on categorical variables, pre-treatment regressand and other covariates. \textsf{FloodInRd1} and \textsf{HeadLiterate0} are indicator variables for the presence of self reported damage by a flood at the baseline, and literacy of household head, respectively. \textsf{HHsize0} is household size at the baseline. We annotate the number of periods that a household is observed with \textsf{T}. The total number of households is shown for each values of \textsf{T}. \textsf{T=4} indicates the number of households with complete panel information, \textsf{T=3} indicates number of households observed three times, \textsf{T=2} indicates the number of households observed twice. \textsf{N} indicates total number of observations used in ANCOVA estimation, or \textsf{N$=$1$\times$(T=2)+2$\times$(T=3)+3$\times$(T=4)}.  \textsf{Large}, \textsf{LargeGrace}, \textsf{Cattle} are indicator variables of the \textsf{large}, \textsf{large grace}, and \textsf{cattle} arms, respectively. The default arm category is \textsf{traditional} arm. \textsf{Secondary} and \textsf{College} are indicator variables of secondary schooling (ages 13-15) and tertiary schooling (ages 16-18), both at the time of baseline. Default category is primary (ages 05-12). \textsf{UltraPoor} is an indicator variable if the household is classified as the ultra poor. Interaction terms of dummy variables are demeaned before interacting. The first column gives mean and standard deviation (in parenthesises) of each covariates before demeaning.\\
& 2. & $P$ values in percentages in parenthesises. Standard errors are clustered at group (village) level.%
%${}^{***}$, ${}^{**}$, ${}^{*}$ indicate statistical significance at 1\%, 5\%, 10\%, respetively. Standard errors are clustered at group (village) level.
\end{tabular}
\end{minipage}

\hspace{-1cm}\begin{minipage}[t]{14cm}
\hfil\textsc{\normalsize Table \refstepcounter{table}\thetable: ANCOVA estimation of school enrollment by time\label{tab ANCOVA enroll time varying1}}\\
\setlength{\tabcolsep}{1pt}
\setlength{\baselineskip}{8pt}
\renewcommand{\arraystretch}{.525}
\hfil\begin{tikzpicture}
\node (tbl) {\input{c:/data/GUK/analysis/save/EstimationMemo/SchoolingTimeVaryingANCOVAEstimationResults_1.tex}};
%\input{c:/dropbox/data/ramadan/save/tablecolortemplate.tex}
\end{tikzpicture}
\end{minipage}

\addtocounter{table}{-1}
\hspace{-1cm}\begin{minipage}[t]{14cm}
\hfil\textsc{\normalsize Table \refstepcounter{table}\thetable: ANCOVA estimation of school enrollment by time (continued)\label{tab ANCOVA enroll time varying2}}\\
\setlength{\tabcolsep}{1pt}
\setlength{\baselineskip}{8pt}
\renewcommand{\arraystretch}{.525}
\hfil\begin{tikzpicture}
\node (tbl) {\input{c:/data/GUK/analysis/save/EstimationMemo/SchoolingTimeVaryingANCOVAEstimationResults_2.tex}};
%\input{c:/dropbox/data/ramadan/save/tablecolortemplate.tex}
\end{tikzpicture}\\
\renewcommand{\arraystretch}{.8}
\setlength{\tabcolsep}{1pt}
\hspace{-1cm}\begin{tabular}{>{\hfill\scriptsize}p{1cm}<{}>{\hfill\scriptsize}p{.25cm}<{}>{\scriptsize}p{14cm}<{\hfill}}
Source:& \multicolumn{2}{l}{\scriptsize Estimated with GUK administrative and survey data.}\\
Notes: & 1. & ANCOVA estimates using administrative and survey data. Post treatment regressands are regressed on categorical variables, pre-treatment regressand and other covariates. \textsf{FloodInRd1} and \textsf{HeadLiterate0} are indicator variables for the presence of self reported damage by a flood at the baseline, and literacy of household head, respectively. \textsf{HHsize0} is household size at the baseline. We annotate the number of periods that a household is observed with \textsf{T}. The total number of households is shown for each values of \textsf{T}. \textsf{T=4} indicates the number of households with complete panel information, \textsf{T=3} indicates number of households observed three times, \textsf{T=2} indicates the number of households observed twice. \textsf{N} indicates total number of observations used in ANCOVA estimation, or \textsf{N$=$1$\times$(T=2)+2$\times$(T=3)+3$\times$(T=4)}.  \textsf{Large}, \textsf{LargeGrace}, \textsf{Cattle} are indicator variables of the \textsf{large}, \textsf{large grace}, and \textsf{cattle} arms, respectively. The default arm category is \textsf{traditional} arm. \textsf{Secondary} and \textsf{College} are indicator variables of secondary schooling (ages 13-15) and tertiary schooling (ages 16-18), both at the time of baseline. Default category is primary (ages 05-12). \textsf{rd2, rd3, rd4} are dummy variables for second, third, and fourth round of survey. Interaction terms of dummy variables are demeaned before interacting. The first column gives mean and standard deviation (in parenthesises) of each covariates before demeaning.\\
& 2. & $P$ values in percentages in parenthesises. Standard errors are clustered at group (village) level.%
%${}^{***}$, ${}^{**}$, ${}^{*}$ indicate statistical significance at 1\%, 5\%, 10\%, respetively. Standard errors are clustered at group (village) level.
\end{tabular}
\end{minipage}

\hspace{-1cm}\begin{minipage}[t]{14cm}
\hfil\textsc{\normalsize Table \refstepcounter{table}\thetable: ANCOVA estimation of school enrollment by time (continued 2)\label{tab ANCOVA enroll time varying3}}\\
\setlength{\tabcolsep}{1pt}
\setlength{\baselineskip}{8pt}
\renewcommand{\arraystretch}{.525}
\hfil\begin{tikzpicture}
\node (tbl) {\input{c:/data/GUK/analysis/save/EstimationMemo/SchoolingTimeVaryingAttributesANCOVAEstimationResults_3.tex}};
%\input{c:/dropbox/data/ramadan/save/tablecolortemplate.tex}
\end{tikzpicture}\\
\renewcommand{\arraystretch}{.8}
\setlength{\tabcolsep}{1pt}
\hspace{-1cm}\begin{tabular}{>{\hfill\scriptsize}p{1cm}<{}>{\hfill\scriptsize}p{.25cm}<{}>{\scriptsize}p{14cm}<{\hfill}}
Source:& \multicolumn{2}{l}{\scriptsize Estimated with GUK administrative and survey data.}\\
Notes: & 1. & ANCOVA estimates using administrative and survey data. Post treatment regressands are regressed on categorical variables, pre-treatment regressand and other covariates. \textsf{FloodInRd1} and \textsf{HeadLiterate0} are indicator variables for the presence of self reported damage by a flood at the baseline, and literacy of household head, respectively. \textsf{HHsize0} is household size at the baseline. We annotate the number of periods that a household is observed with \textsf{T}. The total number of households is shown for each values of \textsf{T}. \textsf{T=4} indicates the number of households with complete panel information, \textsf{T=3} indicates number of households observed three times, \textsf{T=2} indicates the number of households observed twice. \textsf{N} indicates total number of observations used in ANCOVA estimation, or \textsf{N$=$1$\times$(T=2)+2$\times$(T=3)+3$\times$(T=4)}.  \textsf{Large}, \textsf{LargeGrace}, \textsf{Cattle} are indicator variables of the \textsf{large}, \textsf{large grace}, and \textsf{cattle} arms, respectively. The default arm category is \textsf{traditional} arm. \textsf{Secondary} and \textsf{College} are indicator variables of secondary schooling (ages 13-15) and tertiary schooling (ages 16-18), both at the time of baseline. Default category is primary (ages 05-12). \textsf{rd2, rd3, rd4} are dummy variables for second, third, and fourth round of survey. Interaction terms of dummy variables are demeaned before interacting. The first column gives mean and standard deviation (in parenthesises) of each covariates before demeaning.\\
& 2. & $P$ values in percentages in parenthesises. Standard errors are clustered at group (village) level.%
%${}^{***}$, ${}^{**}$, ${}^{*}$ indicate statistical significance at 1\%, 5\%, 10\%, respetively. Standard errors are clustered at group (village) level.
\end{tabular}
\end{minipage}

\hspace{-1cm}\begin{minipage}[t]{14cm}
\hfil\textsc{\normalsize Table \refstepcounter{table}\thetable: ANCOVA estimation of school enrollment by attributes and time\label{tab ANCOVA enroll time varying attributes}}\\
\setlength{\tabcolsep}{1pt}
\setlength{\baselineskip}{8pt}
\renewcommand{\arraystretch}{.525}
\hfil\begin{tikzpicture}
\node (tbl) {\input{c:/data/GUK/analysis/save/EstimationMemo/SchoolingTimeVaryingAttributesANCOVAEstimationResults_1.tex}};
%\input{c:/dropbox/data/ramadan/save/tablecolortemplate.tex}
\end{tikzpicture}
\end{minipage}

\addtocounter{table}{-1}
\hspace{-1cm}\begin{minipage}[t]{14cm}
\hfil\textsc{\normalsize Table \refstepcounter{table}\thetable: ANCOVA estimation of school enrollment by attributes and time (continued)\label{tab ANCOVA enroll time varying attributes2}}\\
\setlength{\tabcolsep}{1pt}
\setlength{\baselineskip}{8pt}
\renewcommand{\arraystretch}{.525}
\hfil\begin{tikzpicture}
\node (tbl) {\input{c:/data/GUK/analysis/save/EstimationMemo/SchoolingTimeVaryingAttributesANCOVAEstimationResults_2.tex}};
%\input{c:/dropbox/data/ramadan/save/tablecolortemplate.tex}
\end{tikzpicture}\\
\renewcommand{\arraystretch}{.8}
\setlength{\tabcolsep}{1pt}
\hspace{-1cm}\begin{tabular}{>{\hfill\scriptsize}p{1cm}<{}>{\hfill\scriptsize}p{.25cm}<{}>{\scriptsize}p{14cm}<{\hfill}}
Source:& \multicolumn{2}{l}{\scriptsize Estimated with GUK administrative and survey data.}\\
Notes: & 1. & ANCOVA estimates using administrative and survey data. Post treatment regressands are regressed on categorical variables, pre-treatment regressand and other covariates. \textsf{FloodInRd1} and \textsf{HeadLiterate0} are indicator variables for the presence of self reported damage by a flood at the baseline, and literacy of household head, respectively. \textsf{HHsize0} is household size at the baseline. We annotate the number of periods that a household is observed with \textsf{T}. The total number of households is shown for each values of \textsf{T}. \textsf{T=4} indicates the number of households with complete panel information, \textsf{T=3} indicates number of households observed three times, \textsf{T=2} indicates the number of households observed twice. \textsf{N} indicates total number of observations used in ANCOVA estimation, or \textsf{N$=$1$\times$(T=2)+2$\times$(T=3)+3$\times$(T=4)}.  \textsf{Upfront} is an indicator variable of the arm with an upfront large disbursement, \textsf{WithGrace} is an indicator variable of the arm with a grace period, \textsf{InKind} is an indicator variable of the arm which lends a heifer. \textsf{Secondary} and \textsf{College} are indicator variables of secondary schooling (ages 13-15) and tertiary schooling (ages 16-18), both at the time of baseline. Default category is primary (ages 05-12). \textsf{rd2, rd3, rd4} are dummy variables for second, third, and fourth round of survey. Interaction terms of dummy variables are demeaned before interacting. The first column gives mean and standard deviation (in parenthesises) of each covariates before demeaning.\\
& 2. & $P$ values in percentages in parenthesises. Standard errors are clustered at group (village) level.%
%${}^{***}$, ${}^{**}$, ${}^{*}$ indicate statistical significance at 1\%, 5\%, 10\%, respetively. Standard errors are clustered at group (village) level.
\end{tabular}
\end{minipage}

\hspace{-1cm}\begin{minipage}[t]{14cm}
\hfil\textsc{\normalsize Table \refstepcounter{table}\thetable: ANCOVA estimation of school enrollment by attributes and time (continued 2)\label{tab ANCOVA enroll time varying attributes3}}\\
\setlength{\tabcolsep}{1pt}
\setlength{\baselineskip}{8pt}
\renewcommand{\arraystretch}{.525}
\hfil\begin{tikzpicture}
\node (tbl) {\input{c:/data/GUK/analysis/save/EstimationMemo/SchoolingTimeVaryingAttributesANCOVAEstimationResults_3.tex}};
%\input{c:/dropbox/data/ramadan/save/tablecolortemplate.tex}
\end{tikzpicture}\\
\renewcommand{\arraystretch}{.8}
\setlength{\tabcolsep}{1pt}
\hspace{-1cm}\begin{tabular}{>{\hfill\scriptsize}p{1cm}<{}>{\hfill\scriptsize}p{.25cm}<{}>{\scriptsize}p{14cm}<{\hfill}}
Source:& \multicolumn{2}{l}{\scriptsize Estimated with GUK administrative and survey data.}\\
Notes: & 1. & ANCOVA estimates using administrative and survey data. Post treatment regressands are regressed on categorical variables, pre-treatment regressand and other covariates. \textsf{FloodInRd1} and \textsf{HeadLiterate0} are indicator variables for the presence of self reported damage by a flood at the baseline, and literacy of household head, respectively. \textsf{HHsize0} is household size at the baseline. We annotate the number of periods that a household is observed with \textsf{T}. The total number of households is shown for each values of \textsf{T}. \textsf{T=4} indicates the number of households with complete panel information, \textsf{T=3} indicates number of households observed three times, \textsf{T=2} indicates the number of households observed twice. \textsf{N} indicates total number of observations used in ANCOVA estimation, or \textsf{N$=$1$\times$(T=2)+2$\times$(T=3)+3$\times$(T=4)}.  \textsf{Upfront} is an indicator variable of the arm with an upfront large disbursement, \textsf{WithGrace} is an indicator variable of the arm with a grace period, \textsf{InKind} is an indicator variable of the arm which lends a heifer. \textsf{Secondary} and \textsf{College} are indicator variables of secondary schooling (ages 13-15) and tertiary schooling (ages 16-18), both at the time of baseline. Default category is primary (ages 05-12). \textsf{rd2, rd3, rd4} are dummy variables for second, third, and fourth round of survey. Interaction terms of dummy variables are demeaned before interacting. The first column gives mean and standard deviation (in parenthesises) of each covariates before demeaning.\\
& 2. & $P$ values in percentages in parenthesises. Standard errors are clustered at group (village) level.%
%${}^{***}$, ${}^{**}$, ${}^{*}$ indicate statistical significance at 1\%, 5\%, 10\%, respetively. Standard errors are clustered at group (village) level.
\end{tabular}
\end{minipage}
\subsection{Incomes}


Income sources are mainly labour incomes (\textsf{lab}) and farm revenues (\textsf{far}) with 849 and 53 observations at baseline, respectively. 












\hspace{-1cm}\begin{minipage}[t]{14cm}
\hfil\textsc{\normalsize Table \refstepcounter{table}\thetable: ANCOVA estimation of household labour incomes and farm incomes\label{tab ANCOVA LabourIncomes}}\\
\setlength{\tabcolsep}{.5pt}
\setlength{\baselineskip}{8pt}
\renewcommand{\arraystretch}{.55}

\vspace{2ex}
A. Labour incomes\\
\hfil\begin{tikzpicture}
\node (tbl) {\input{c:/data/GUK/analysis/save/EstimationMemo/LabourIncomeANCOVAEstimationResults.tex}};
%\input{c:/dropbox/data/ramadan/save/tablecolortemplate.tex}
\end{tikzpicture}\\
B. Farm incomes\\
\hfil\begin{tikzpicture}
\node (tbl) {\input{c:/data/GUK/analysis/save/EstimationMemo/FarmIncomeANCOVAEstimationResults.tex}};
%\input{c:/dropbox/data/ramadan/save/tablecolortemplate.tex}
\end{tikzpicture}\\

\renewcommand{\arraystretch}{.8}
\setlength{\tabcolsep}{1pt}
\begin{tabular}{>{\hfill\scriptsize}p{1cm}<{}>{\hfill\scriptsize}p{.25cm}<{}>{\scriptsize}p{12cm}<{\hfill}}
Source:& \multicolumn{2}{l}{\scriptsize Estimated with GUK administrative and survey data.}\\
Notes: & 1. & ANCOVA estimates using administrative and survey data. Post treatment regressands are regressed on categorical variables, pre-treatment regressand and other covariates. \textsf{FloodInRd1} and \textsf{HeadLiterate0} are indicator variables for the presence of self reported damage by a flood at the baseline, and literacy of household head, respectively. \textsf{HHsize0} is household size at the baseline. We annotate the number of periods that a household is observed with \textsf{T}. The total number of households is shown for each values of \textsf{T}. \textsf{T=4} indicates the number of households with complete panel information, \textsf{T=3} indicates number of households observed three times, \textsf{T=2} indicates the number of households observed twice. \textsf{N} indicates total number of observations used in ANCOVA estimation, or \textsf{N$=$1$\times$(T=2)+2$\times$(T=3)+3$\times$(T=4)}.  \textsf{Large}, \textsf{LargeGrace}, \textsf{Cattle} are indicator variables of the \textsf{large}, \textsf{large grace}, and \textsf{cattle} arms, respectively. The default arm category is \textsf{traditional} arm. Labour incomes are in 1000 Tk units and are a sum of all earned labour incomes of household members. Farm revenues are in 1000 Tk units and are a total of agricultural produce sales.  \\
& 2. & $P$ values in percentages in parenthesises. Standard errors are clustered at group (village) level.
%${}^{***}$, ${}^{**}$, ${}^{*}$ indicate statistical significance at 1\%, 5\%, 10\%, respetively. Standard errors are clustered at group (village) level.
\end{tabular}
\end{minipage}

\hspace{-1cm}\begin{minipage}[t]{14cm}
\hfil\textsc{\normalsize Table \refstepcounter{table}\thetable: ANCOVA estimation of household labour incomes and farm incomes by attributes \label{tab ANCOVA labour incomes attributes}}\\
\setlength{\tabcolsep}{.5pt}
\setlength{\baselineskip}{8pt}
\renewcommand{\arraystretch}{.55}

\vspace{2ex}
A. Labour incomes\\
\hfil\begin{tikzpicture}
\node (tbl) {\input{c:/data/GUK/analysis/save/EstimationMemo/LabourIncomeAttributesANCOVAEstimationResults.tex}};
%\input{c:/dropbox/data/ramadan/save/tablecolortemplate.tex}
\end{tikzpicture}\\
B. Farm incomes\\
\hfil\begin{tikzpicture}
\node (tbl) {\input{c:/data/GUK/analysis/save/EstimationMemo/FarmIncomeAttributesANCOVAEstimationResults.tex}};
%\input{c:/dropbox/data/ramadan/save/tablecolortemplate.tex}
\end{tikzpicture}\\
\renewcommand{\arraystretch}{.8}
\setlength{\tabcolsep}{1pt}
\begin{tabular}{>{\hfill\scriptsize}p{1cm}<{}>{\hfill\scriptsize}p{.25cm}<{}>{\scriptsize}p{12cm}<{\hfill}}
Source:& \multicolumn{2}{l}{\scriptsize Estimated with GUK administrative and survey data.}\\
Notes: & 1. & ANCOVA estimates using administrative and survey data. Post treatment regressands are regressed on categorical variables, pre-treatment regressand and other covariates. \textsf{FloodInRd1} and \textsf{HeadLiterate0} are indicator variables for the presence of self reported damage by a flood at the baseline, and literacy of household head, respectively. \textsf{HHsize0} is household size at the baseline. We annotate the number of periods that a household is observed with \textsf{T}. The total number of households is shown for each values of \textsf{T}. \textsf{T=4} indicates the number of households with complete panel information, \textsf{T=3} indicates number of households observed three times, \textsf{T=2} indicates the number of households observed twice. \textsf{N} indicates total number of observations used in ANCOVA estimation, or \textsf{N$=$1$\times$(T=2)+2$\times$(T=3)+3$\times$(T=4)}.  \textsf{Upfront} is an indicator variable of the arm with an upfront large disbursement, \textsf{WithGrace} is an indicator variable of the arm with a grace period, \textsf{InKind} is an indicator variable of the arm which lends a heifer. Labour incomes are in 1000 Tk units and are a sum of all earned labour incomes of household members. Farm revenues are in 1000 Tk units and are a total of agricultural produce sales.\\
& 2. & $P$ values in percentages in parenthesises. Standard errors are clustered at group (village) level.
%${}^{***}$, ${}^{**}$, ${}^{*}$ indicate statistical significance at 1\%, 5\%, 10\%, respetively. Standard errors are clustered at group (village) level.
\end{tabular}
\end{minipage}

\hspace{-1cm}\begin{minipage}[t]{14cm}
\hfil\textsc{\normalsize Table \refstepcounter{table}\thetable: ANCOVA estimation of household labour incomes and farm incomes by period\label{tab ANCOVA labour incomes timevarying}}\\
\setlength{\tabcolsep}{1pt}
\setlength{\baselineskip}{8pt}
\renewcommand{\arraystretch}{.55}

\vspace{2ex}
A. Labour incomes\\
\hfil\begin{tikzpicture}
\node (tbl) {\input{c:/data/GUK/analysis/save/EstimationMemo/LabourIncomeTimeVaryingANCOVAEstimationResults.tex}};
%\input{c:/dropbox/data/ramadan/save/tablecolortemplate.tex}
\end{tikzpicture}\\
B. Farm incomes\\
\hfil\begin{tikzpicture}
\node (tbl) {\input{c:/data/GUK/analysis/save/EstimationMemo/FarmIncomeTimeVaryingANCOVAEstimationResults.tex}};
%\input{c:/dropbox/data/ramadan/save/tablecolortemplate.tex}
\end{tikzpicture}\\
\renewcommand{\arraystretch}{.8}
\setlength{\tabcolsep}{1pt}
\begin{tabular}{>{\hfill\scriptsize}p{1cm}<{}>{\hfill\scriptsize}p{.25cm}<{}>{\scriptsize}p{12cm}<{\hfill}}
Source:& \multicolumn{2}{l}{\scriptsize Estimated with GUK administrative and survey data.}\\
Notes: & 1. & ANCOVA estimates using administrative and survey data. Post treatment regressands are regressed on categorical variables, pre-treatment regressand and other covariates. \textsf{FloodInRd1} and \textsf{HeadLiterate0} are indicator variables for the presence of self reported damage by a flood at the baseline, and literacy of household head, respectively. \textsf{HHsize0} is household size at the baseline. We annotate the number of periods that a household is observed with \textsf{T}. The total number of households is shown for each values of \textsf{T}. \textsf{T=4} indicates the number of households with complete panel information, \textsf{T=3} indicates number of households observed three times, \textsf{T=2} indicates the number of households observed twice. \textsf{N} indicates total number of observations used in ANCOVA estimation, or \textsf{N$=$1$\times$(T=2)+2$\times$(T=3)+3$\times$(T=4)}.  \textsf{Upfront} is an indicator variable of the arm with an upfront large disbursement, \textsf{WithGrace} is an indicator variable of the arm with a grace period, \textsf{InKind} is an indicator variable of the arm which lends a heifer. \textsf{rd2, rd3, rd4} are dummy variables for second, third, and fourth round of survey. Labour incomes are in 1000 Tk units and are a sum of all earned labour incomes of household members. Farm revenues are in 1000 Tk units and are a total of agricultural produce sales. \\
& 2. & $P$ values in percentages in parenthesises. Standard errors are clustered at group (village) level.
%${}^{***}$, ${}^{**}$, ${}^{*}$ indicate statistical significance at 1\%, 5\%, 10\%, respetively. Standard errors are clustered at group (village) level.
\end{tabular}
\end{minipage}

\hspace{-1cm}\begin{minipage}[t]{14cm}
\hfil\textsc{\normalsize Table \refstepcounter{table}\thetable: ANCOVA estimation of household labour incomes and farm incomes by attributes and period\label{tab ANCOVA labour incomes timevarying attributes}}\\
\setlength{\tabcolsep}{1pt}
\setlength{\baselineskip}{8pt}
\renewcommand{\arraystretch}{.55}

\vspace{2ex}
A. Labour incomes\\
\hfil\begin{tikzpicture}
\node (tbl) {\input{c:/data/GUK/analysis/save/EstimationMemo/LabourIncomeTimeVaryingAttributesANCOVAEstimationResults.tex}};
%\input{c:/dropbox/data/ramadan/save/tablecolortemplate.tex}
\end{tikzpicture}\\
B. Farm incomes\\
\hfil\begin{tikzpicture}
\node (tbl) {\input{c:/data/GUK/analysis/save/EstimationMemo/FarmIncomeTimeVaryingAttributesANCOVAEstimationResults.tex}};
%\input{c:/dropbox/data/ramadan/save/tablecolortemplate.tex}
\end{tikzpicture}\\
\renewcommand{\arraystretch}{.8}
\setlength{\tabcolsep}{1pt}
\begin{tabular}{>{\hfill\scriptsize}p{1cm}<{}>{\hfill\scriptsize}p{.25cm}<{}>{\scriptsize}p{12cm}<{\hfill}}
Source:& \multicolumn{2}{l}{\scriptsize Estimated with GUK administrative and survey data.}\\
Notes: & 1. & ANCOVA estimates using administrative and survey data. Post treatment regressands are regressed on categorical variables, pre-treatment regressand and other covariates. \textsf{FloodInRd1} and \textsf{HeadLiterate0} are indicator variables for the presence of self reported damage by a flood at the baseline, and literacy of household head, respectively. \textsf{HHsize0} is household size at the baseline. We annotate the number of periods that a household is observed with \textsf{T}. The total number of households is shown for each values of \textsf{T}. \textsf{T=4} indicates the number of households with complete panel information, \textsf{T=3} indicates number of households observed three times, \textsf{T=2} indicates the number of households observed twice. \textsf{N} indicates total number of observations used in ANCOVA estimation, or \textsf{N$=$1$\times$(T=2)+2$\times$(T=3)+3$\times$(T=4)}.  \textsf{Upfront} is an indicator variable of the arm with an upfront large disbursement, \textsf{WithGrace} is an indicator variable of the arm with a grace period, \textsf{InKind} is an indicator variable of the arm which lends a heifer. \textsf{rd2, rd3, rd4} are dummy variables for second, third, and fourth round of survey. Labour incomes are in 1000 Tk units and are a sum of all earned labour incomes of household members. Farm revenues are in 1000 Tk units and are a total of agricultural produce sales. \\
& 2. & $P$ values in percentages in parenthesises. Standard errors are clustered at group (village) level.
%${}^{***}$, ${}^{**}$, ${}^{*}$ indicate statistical significance at 1\%, 5\%, 10\%, respetively. Standard errors are clustered at group (village) level.
\end{tabular}
\end{minipage}
\subsection{Consumption}



Consumption is observed in rd 2-4. There are 1386 observations. 









\hspace{-1cm}\begin{minipage}[t]{14cm}
\hfil\textsc{\normalsize Table \refstepcounter{table}\thetable: ANCOVA estimation of consumption\label{tab ANCOVA consumption}}\\
\setlength{\tabcolsep}{1pt}
\setlength{\baselineskip}{8pt}
\renewcommand{\arraystretch}{.55}
\hfil\begin{tikzpicture}
\node (tbl) {\input{c:/data/GUK/analysis/save/EstimationMemo/ConsumptionANCOVAEstimationResults.tex}};
%\input{c:/dropbox/data/ramadan/save/tablecolortemplate.tex}
\end{tikzpicture}\\
\renewcommand{\arraystretch}{.8}
\setlength{\tabcolsep}{1pt}
\begin{tabular}{>{\hfill\scriptsize}p{1cm}<{}>{\hfill\scriptsize}p{.25cm}<{}>{\scriptsize}p{12cm}<{\hfill}}
Source:& \multicolumn{2}{l}{\scriptsize Estimated with GUK administrative and survey data of round 2 - 4.}\\
Notes: & 1. & ANCOVA estimates using administrative and survey data. Post treatment regressands are regressed on categorical variables, pre-treatment regressand and other covariates. \textsf{FloodInRd1} and \textsf{HeadLiterate0} are indicator variables for the presence of self reported damage by a flood at the baseline, and literacy of household head, respectively. \textsf{HHsize0} is household size at the baseline. We annotate the number of periods that a household is observed with \textsf{T}. The total number of households is shown for each values of \textsf{T}. \textsf{T=4} indicates the number of households with complete panel information, \textsf{T=3} indicates number of households observed three times, \textsf{T=2} indicates the number of households observed twice. \textsf{N} indicates total number of observations used in ANCOVA estimation, or \textsf{N$=$1$\times$(T=2)+2$\times$(T=3)+3$\times$(T=4)}.  \textsf{Large}, \textsf{LargeGrace}, \textsf{Cattle} are indicator variables of the \textsf{large}, \textsf{large grace}, and \textsf{cattle} arms, respectively. The default arm category is \textsf{traditional} arm. Consumption is annualised values. \\
& 2. & $P$ values in percentages in parenthesises. Standard errors are clustered at group (village) level.
%${}^{***}$, ${}^{**}$, ${}^{*}$ indicate statistical significance at 1\%, 5\%, 10\%, respetively. Standard errors are clustered at group (village) level.
\end{tabular}
\end{minipage}

\hspace{-1cm}\begin{minipage}[t]{14cm}
\hfil\textsc{\normalsize Table \refstepcounter{table}\thetable: ANCOVA estimation of consumption by attributes \label{tab ANCOVA consumption attributes original HH}}\\
\setlength{\tabcolsep}{1pt}
\setlength{\baselineskip}{8pt}
\renewcommand{\arraystretch}{.55}
\hfil\begin{tikzpicture}
\node (tbl) {\input{c:/data/GUK/analysis/save/EstimationMemo/ConsumptionAttributesANCOVAEstimationResults.tex}};
%\input{c:/dropbox/data/ramadan/save/tablecolortemplate.tex}
\end{tikzpicture}\\
\renewcommand{\arraystretch}{.8}
\setlength{\tabcolsep}{1pt}
\begin{tabular}{>{\hfill\scriptsize}p{1cm}<{}>{\hfill\scriptsize}p{.25cm}<{}>{\scriptsize}p{12cm}<{\hfill}}
Source:& \multicolumn{2}{l}{\scriptsize Estimated with GUK administrative and survey data.}\\
Notes: & 1. & ANCOVA estimates using administrative and survey data. Post treatment regressands are regressed on categorical variables, pre-treatment regressand and other covariates. \textsf{FloodInRd1} and \textsf{HeadLiterate0} are indicator variables for the presence of self reported damage by a flood at the baseline, and literacy of household head, respectively. \textsf{HHsize0} is household size at the baseline. We annotate the number of periods that a household is observed with \textsf{T}. The total number of households is shown for each values of \textsf{T}. \textsf{T=4} indicates the number of households with complete panel information, \textsf{T=3} indicates number of households observed three times, \textsf{T=2} indicates the number of households observed twice. \textsf{N} indicates total number of observations used in ANCOVA estimation, or \textsf{N$=$1$\times$(T=2)+2$\times$(T=3)+3$\times$(T=4)}.  \textsf{Upfront} is an indicator variable of the arm with an upfront large disbursement, \textsf{WithGrace} is an indicator variable of the arm with a grace period, \textsf{InKind} is an indicator variable of the arm which lends a heifer. Consumption is annualised values. \\
& 2. & $P$ values in percentages in parenthesises. Standard errors are clustered at group (village) level.
%${}^{***}$, ${}^{**}$, ${}^{*}$ indicate statistical significance at 1\%, 5\%, 10\%, respetively. Standard errors are clustered at group (village) level.
\end{tabular}
\end{minipage}



\hspace{-1cm}\begin{minipage}[t]{14cm}
\hfil\textsc{\normalsize Table \refstepcounter{table}\thetable: ANCOVA estimation of consumption by period\label{tab ANCOVA consumption timevarying}}\\
\setlength{\tabcolsep}{1pt}
\setlength{\baselineskip}{8pt}
\renewcommand{\arraystretch}{.55}
\hfil\begin{tikzpicture}
\node (tbl) {\input{c:/data/GUK/analysis/save/EstimationMemo/ConsumptionTimeVaryingANCOVAEstimationResults.tex}};
%\input{c:/dropbox/data/ramadan/save/tablecolortemplate.tex}
\end{tikzpicture}\\
\renewcommand{\arraystretch}{.8}
\setlength{\tabcolsep}{1pt}
\begin{tabular}{>{\hfill\scriptsize}p{1cm}<{}>{\hfill\scriptsize}p{.25cm}<{}>{\scriptsize}p{12cm}<{\hfill}}
Source:& \multicolumn{2}{l}{\scriptsize Estimated with GUK administrative and survey data.}\\
Notes: & 1. & ANCOVA estimates using administrative and survey data. Post treatment regressands are regressed on categorical variables, pre-treatment regressand and other covariates. \textsf{FloodInRd1} and \textsf{HeadLiterate0} are indicator variables for the presence of self reported damage by a flood at the baseline, and literacy of household head, respectively. \textsf{HHsize0} is household size at the baseline. We annotate the number of periods that a household is observed with \textsf{T}. The total number of households is shown for each values of \textsf{T}. \textsf{T=4} indicates the number of households with complete panel information, \textsf{T=3} indicates number of households observed three times, \textsf{T=2} indicates the number of households observed twice. \textsf{N} indicates total number of observations used in ANCOVA estimation, or \textsf{N$=$1$\times$(T=2)+2$\times$(T=3)+3$\times$(T=4)}.  \textsf{Large}, \textsf{LargeGrace}, \textsf{Cattle} are indicator variables of the \textsf{large}, \textsf{large grace}, and \textsf{cattle} arms, respectively. The default arm category is \textsf{traditional} arm. Consumption is annualised values. \\
& 2. & $P$ values in percentages in parenthesises. Standard errors are clustered at group (village) level.
%${}^{***}$, ${}^{**}$, ${}^{*}$ indicate statistical significance at 1\%, 5\%, 10\%, respetively. Standard errors are clustered at group (village) level.
\end{tabular}
\end{minipage}


\hspace{-1cm}\begin{minipage}[t]{14cm}
\hfil\textsc{\normalsize Table \refstepcounter{table}\thetable: ANCOVA estimation of consumption by attributes and period\label{tab ANCOVA consumption timevarying attributes original HH}}\\
\setlength{\tabcolsep}{1pt}
\setlength{\baselineskip}{8pt}
\renewcommand{\arraystretch}{.55}
\hfil\begin{tikzpicture}
\node (tbl) {\input{c:/data/GUK/analysis/save/EstimationMemo/ConsumptionTimeVaryingAttributesANCOVAEstimationResults.tex}};
%\input{c:/dropbox/data/ramadan/save/tablecolortemplate.tex}
\end{tikzpicture}\\
\renewcommand{\arraystretch}{.8}
\setlength{\tabcolsep}{1pt}
\begin{tabular}{>{\hfill\scriptsize}p{1cm}<{}>{\hfill\scriptsize}p{.25cm}<{}>{\scriptsize}p{12cm}<{\hfill}}
Source:& \multicolumn{2}{l}{\scriptsize Estimated with GUK administrative and survey data.}\\
Notes: & 1. & ANCOVA estimates using administrative and survey data. Post treatment regressands are regressed on categorical variables, pre-treatment regressand and other covariates. \textsf{FloodInRd1} and \textsf{HeadLiterate0} are indicator variables for the presence of self reported damage by a flood at the baseline, and literacy of household head, respectively. \textsf{HHsize0} is household size at the baseline. We annotate the number of periods that a household is observed with \textsf{T}. The total number of households is shown for each values of \textsf{T}. \textsf{T=4} indicates the number of households with complete panel information, \textsf{T=3} indicates number of households observed three times, \textsf{T=2} indicates the number of households observed twice. \textsf{N} indicates total number of observations used in ANCOVA estimation, or \textsf{N$=$1$\times$(T=2)+2$\times$(T=3)+3$\times$(T=4)}.  \textsf{Upfront} is an indicator variable of the arm with an upfront large disbursement, \textsf{WithGrace} is an indicator variable of the arm with a grace period, \textsf{InKind} is an indicator variable of the arm which lends a heifer. Consumption is annualised values. \\
& 2. & $P$ values in percentages in parenthesises. Standard errors are clustered at group (village) level.
%${}^{***}$, ${}^{**}$, ${}^{*}$ indicate statistical significance at 1\%, 5\%, 10\%, respetively. Standard errors are clustered at group (village) level.
\end{tabular}
\end{minipage}






Consumption is observed in rd 2-4. There are 1386 observations. 









\hspace{-1cm}\begin{minipage}[t]{14cm}
\hfil\textsc{\normalsize Table \refstepcounter{table}\thetable: OLS estimation of consumption\label{tab ANCOVA consumption}}\\
\setlength{\tabcolsep}{1pt}
\setlength{\baselineskip}{8pt}
\renewcommand{\arraystretch}{.55}
\hfil\begin{tikzpicture}
\node (tbl) {\input{c:/data/GUK/analysis/save/EstimationMemo/ConsumptionOLSANCOVAEstimationResults.tex}};
%\input{c:/dropbox/data/ramadan/save/tablecolortemplate.tex}
\end{tikzpicture}\\
\renewcommand{\arraystretch}{.8}
\setlength{\tabcolsep}{1pt}
\begin{tabular}{>{\hfill\scriptsize}p{1cm}<{}>{\hfill\scriptsize}p{.25cm}<{}>{\scriptsize}p{12cm}<{\hfill}}
Source:& \multicolumn{2}{l}{\scriptsize Estimated with GUK administrative and survey data of round 2 - 4.}\\
Notes: & 1. & ANCOVA estimates using administrative and survey data. Post treatment regressands are regressed on categorical variables, pre-treatment regressand and other covariates. \textsf{FloodInRd1} and \textsf{HeadLiterate0} are indicator variables for the presence of self reported damage by a flood at the baseline, and literacy of household head, respectively. \textsf{HHsize0} is household size at the baseline. We annotate the number of periods that a household is observed with \textsf{T}. The total number of households is shown for each values of \textsf{T}. \textsf{T=4} indicates the number of households with complete panel information, \textsf{T=3} indicates number of households observed three times, \textsf{T=2} indicates the number of households observed twice. \textsf{N} indicates total number of observations used in ANCOVA estimation, or \textsf{N$=$1$\times$(T=2)+2$\times$(T=3)+3$\times$(T=4)}.  \textsf{Large}, \textsf{LargeGrace}, \textsf{Cattle} are indicator variables of the \textsf{large}, \textsf{large grace}, and \textsf{cattle} arms, respectively. The default arm category is \textsf{traditional} arm. Consumption is annualised values. \\
& 2. & $P$ values in percentages in parenthesises. Standard errors are clustered at group (village) level.
%${}^{***}$, ${}^{**}$, ${}^{*}$ indicate statistical significance at 1\%, 5\%, 10\%, respetively. Standard errors are clustered at group (village) level.
\end{tabular}
\end{minipage}

\hspace{-1cm}\begin{minipage}[t]{14cm}
\hfil\textsc{\normalsize Table \refstepcounter{table}\thetable: OLS estimation of consumption by attributes \label{tab ANCOVA consumption attributes original HH}}\\
\setlength{\tabcolsep}{1pt}
\setlength{\baselineskip}{8pt}
\renewcommand{\arraystretch}{.55}
\hfil\begin{tikzpicture}
\node (tbl) {\input{c:/data/GUK/analysis/save/EstimationMemo/ConsumptionOLSAttributesANCOVAEstimationResults.tex}};
%\input{c:/dropbox/data/ramadan/save/tablecolortemplate.tex}
\end{tikzpicture}\\
\renewcommand{\arraystretch}{.8}
\setlength{\tabcolsep}{1pt}
\begin{tabular}{>{\hfill\scriptsize}p{1cm}<{}>{\hfill\scriptsize}p{.25cm}<{}>{\scriptsize}p{12cm}<{\hfill}}
Source:& \multicolumn{2}{l}{\scriptsize Estimated with GUK administrative and survey data.}\\
Notes: & 1. & ANCOVA estimates using administrative and survey data. Post treatment regressands are regressed on categorical variables, pre-treatment regressand and other covariates. \textsf{FloodInRd1} and \textsf{HeadLiterate0} are indicator variables for the presence of self reported damage by a flood at the baseline, and literacy of household head, respectively. \textsf{HHsize0} is household size at the baseline. We annotate the number of periods that a household is observed with \textsf{T}. The total number of households is shown for each values of \textsf{T}. \textsf{T=4} indicates the number of households with complete panel information, \textsf{T=3} indicates number of households observed three times, \textsf{T=2} indicates the number of households observed twice. \textsf{N} indicates total number of observations used in ANCOVA estimation, or \textsf{N$=$1$\times$(T=2)+2$\times$(T=3)+3$\times$(T=4)}.  \textsf{Upfront} is an indicator variable of the arm with an upfront large disbursement, \textsf{WithGrace} is an indicator variable of the arm with a grace period, \textsf{InKind} is an indicator variable of the arm which lends a heifer. Consumption is annualised values. \\
& 2. & $P$ values in percentages in parenthesises. Standard errors are clustered at group (village) level.
%${}^{***}$, ${}^{**}$, ${}^{*}$ indicate statistical significance at 1\%, 5\%, 10\%, respetively. Standard errors are clustered at group (village) level.
\end{tabular}
\end{minipage}



\hspace{-1cm}\begin{minipage}[t]{14cm}
\hfil\textsc{\normalsize Table \refstepcounter{table}\thetable: OLS estimation of consumption by period\label{tab ANCOVA consumption timevarying}}\\
\setlength{\tabcolsep}{1pt}
\setlength{\baselineskip}{8pt}
\renewcommand{\arraystretch}{.55}
\hfil\begin{tikzpicture}
\node (tbl) {\input{c:/data/GUK/analysis/save/EstimationMemo/ConsumptionOLSTimeVaryingANCOVAEstimationResults.tex}};
%\input{c:/dropbox/data/ramadan/save/tablecolortemplate.tex}
\end{tikzpicture}\\
\renewcommand{\arraystretch}{.8}
\setlength{\tabcolsep}{1pt}
\begin{tabular}{>{\hfill\scriptsize}p{1cm}<{}>{\hfill\scriptsize}p{.25cm}<{}>{\scriptsize}p{12cm}<{\hfill}}
Source:& \multicolumn{2}{l}{\scriptsize Estimated with GUK administrative and survey data.}\\
Notes: & 1. & ANCOVA estimates using administrative and survey data. Post treatment regressands are regressed on categorical variables, pre-treatment regressand and other covariates. \textsf{FloodInRd1} and \textsf{HeadLiterate0} are indicator variables for the presence of self reported damage by a flood at the baseline, and literacy of household head, respectively. \textsf{HHsize0} is household size at the baseline. We annotate the number of periods that a household is observed with \textsf{T}. The total number of households is shown for each values of \textsf{T}. \textsf{T=4} indicates the number of households with complete panel information, \textsf{T=3} indicates number of households observed three times, \textsf{T=2} indicates the number of households observed twice. \textsf{N} indicates total number of observations used in ANCOVA estimation, or \textsf{N$=$1$\times$(T=2)+2$\times$(T=3)+3$\times$(T=4)}.  \textsf{Large}, \textsf{LargeGrace}, \textsf{Cattle} are indicator variables of the \textsf{large}, \textsf{large grace}, and \textsf{cattle} arms, respectively. The default arm category is \textsf{traditional} arm. Consumption is annualised values. \\
& 2. & $P$ values in percentages in parenthesises. Standard errors are clustered at group (village) level.
%${}^{***}$, ${}^{**}$, ${}^{*}$ indicate statistical significance at 1\%, 5\%, 10\%, respetively. Standard errors are clustered at group (village) level.
\end{tabular}
\end{minipage}


\hspace{-1cm}\begin{minipage}[t]{14cm}
\hfil\textsc{\normalsize Table \refstepcounter{table}\thetable: OLS estimation of consumption by attributes and period\label{tab ANCOVA consumption timevarying attributes original HH}}\\
\setlength{\tabcolsep}{1pt}
\setlength{\baselineskip}{8pt}
\renewcommand{\arraystretch}{.55}
\hfil\begin{tikzpicture}
\node (tbl) {\input{c:/data/GUK/analysis/save/EstimationMemo/ConsumptionOLSTimeVaryingAttributesANCOVAEstimationResults.tex}};
%\input{c:/dropbox/data/ramadan/save/tablecolortemplate.tex}
\end{tikzpicture}\\
\renewcommand{\arraystretch}{.8}
\setlength{\tabcolsep}{1pt}
\begin{tabular}{>{\hfill\scriptsize}p{1cm}<{}>{\hfill\scriptsize}p{.25cm}<{}>{\scriptsize}p{12cm}<{\hfill}}
Source:& \multicolumn{2}{l}{\scriptsize Estimated with GUK administrative and survey data.}\\
Notes: & 1. & ANCOVA estimates using administrative and survey data. Post treatment regressands are regressed on categorical variables, pre-treatment regressand and other covariates. \textsf{FloodInRd1} and \textsf{HeadLiterate0} are indicator variables for the presence of self reported damage by a flood at the baseline, and literacy of household head, respectively. \textsf{HHsize0} is household size at the baseline. We annotate the number of periods that a household is observed with \textsf{T}. The total number of households is shown for each values of \textsf{T}. \textsf{T=4} indicates the number of households with complete panel information, \textsf{T=3} indicates number of households observed three times, \textsf{T=2} indicates the number of households observed twice. \textsf{N} indicates total number of observations used in ANCOVA estimation, or \textsf{N$=$1$\times$(T=2)+2$\times$(T=3)+3$\times$(T=4)}.  \textsf{Upfront} is an indicator variable of the arm with an upfront large disbursement, \textsf{WithGrace} is an indicator variable of the arm with a grace period, \textsf{InKind} is an indicator variable of the arm which lends a heifer. Consumption is annualised values. \\
& 2. & $P$ values in percentages in parenthesises. Standard errors are clustered at group (village) level.
%${}^{***}$, ${}^{**}$, ${}^{*}$ indicate statistical significance at 1\%, 5\%, 10\%, respetively. Standard errors are clustered at group (village) level.
\end{tabular}
\end{minipage}




\subsection{Assets}

%\subsubsection{Assets}

\subsubsection{Homestead land}


Nonzero reported residential land holding among 800 (776) HHs. Around 40-50\% of respondents report nonzero residential land holding. 
\begin{Schunk}
\begin{Soutput}
            Arm survey  NonZero
         <fctr>  <num>    <num>
 1: traditional      1 0.470588
 2: traditional      2 0.517647
 3: traditional      3 0.552941
 4: traditional      4 0.552941
 5:       large      1 0.467836
 6:       large      2 0.491228
 7:       large      3 0.514620
 8:       large      4 0.508772
 9: large grace      1 0.479042
10: large grace      2 0.508982
11: large grace      3 0.526946
12: large grace      4 0.520958
13:      cattle      1 0.379085
14:      cattle      2 0.450980
15:      cattle      3 0.470588
16:      cattle      4 0.470588
\end{Soutput}
\end{Schunk}
HHs reporting zero residential land holding are, except for the \textsf{traditional} arm, substantially poorer than HHs who report nonzero residential land holding.
\begin{Schunk}
\begin{Soutput}
           Arm ZeroLandHolding MeanNetValue
        <fctr>          <lgcl>        <num>
1: traditional           FALSE     11259.46
2: traditional            TRUE     10612.93
3:       large           FALSE     22738.46
4:       large            TRUE     17915.93
5: large grace           FALSE     15707.96
6: large grace            TRUE      9539.76
7:      cattle           FALSE     15232.56
8:      cattle            TRUE      9659.07
\end{Soutput}
\end{Schunk}

\begin{description}
\vspace{1.0ex}\setlength{\itemsep}{1.0ex}\setlength{\baselineskip}{12pt}
\item[Abu-san's email on Jan 30, 2020]	I checked the questionnaire and found that from round 2, landholding information has been included in the asset information, which made the asset data inflated from round 2. Since landholding is something that is time-invariant for the ultra-poor households, either we can add the landholding information in round 1 or create an asset holding information deleting the landholding information from round 2 onwards, to make the valid comparison. $\Rightarrow$ This is done and saved as \textsf{AmountFilled}.
\end{description}









\mpage{\linewidth}{
\hfil\textsc{\footnotesize Figure \refstepcounter{figure}\thefigure: Homestead land holding of loan recipients\label{fig HomesteadLandHolding}}\\
\hfil\includegraphics{c:/data/GUK/analysis/program/figure/EstimationMemo/HomesteadLand.png}\\
\renewcommand{\arraystretch}{1}
\hfil\begin{tabular}{>{\hfill\scriptsize}p{1cm}<{}>{\scriptsize}p{12cm}<{\hfill}}
Source: & Survey data.\\
Note:& Loan recipients only.\\[1ex]
\end{tabular}
}

\mpage{\linewidth}{
\hfil\textsc{\footnotesize Figure \refstepcounter{figure}\thefigure: Homestead land holding of loan recipients, including zero holding\label{fig LandWithZero}}\\
\hfil\includegraphics{c:/data/GUK/analysis/program/figure/EstimationMemo/LandHoldingScatterWithZero.png}\\
\renewcommand{\arraystretch}{1}
\hfil\begin{tabular}{>{\hfill\scriptsize}p{1cm}<{}>{\scriptsize}p{12cm}<{\hfill}}
Source: & Survey data.\\
Note:& Log of 1+land holding is displayed on horizontal axises. Red squares are means including zero holding for respective Arm-round. Blue squares are means excluding zero holding for respective Arm-round. Loan recipients only.\\[1ex]
\end{tabular}
}

\mpage{\linewidth}{
\hfil\textsc{\footnotesize Figure \refstepcounter{figure}\thefigure: Homestead land holding of loan recipients, excluding zero holding\label{fig LandWithoutZero}}\\
\hfil\includegraphics{c:/data/GUK/analysis/program/figure/EstimationMemo/LandHoldingScatterWithoutZero.png}\\
\renewcommand{\arraystretch}{1}
\hfil\begin{tabular}{>{\hfill\scriptsize}p{1cm}<{}>{\scriptsize}p{12cm}<{\hfill}}
Source: & Survey data.\\
Note:& Log of land holding is displayed on horizontal axises. Zero land holders are excluded. Red squares are means including zero holding for respective Arm-round. Blue squares are means excluding zero holding for respective Arm-round. Loan recipients only.\\[1ex]
\end{tabular}
}

\mpage{\linewidth}{
\hfil\textsc{\footnotesize Figure \refstepcounter{figure}\thefigure: Homestead land holding of loan recipients, exclduing zero\label{fig LandClass}}\\
\hfil\includegraphics{c:/data/GUK/analysis/program/figure/EstimationMemo/LandHoldingHistogramByRound.png}\\
\renewcommand{\arraystretch}{1}
\hfil\begin{tabular}{>{\hfill\scriptsize}p{1cm}<{}>{\scriptsize}p{12cm}<{\hfill}}
Source: & Survey data.\\
Note:&  Loan recipients only. Logarithm of land holding is displayed on horizontal axises. Zero land holding is excluded from the graph. Mean including zero holding is shown as a dotted line in each panel.\\[1ex]
\end{tabular}
}

\mpage{\linewidth}{
\hfil\textsc{\footnotesize Figure \refstepcounter{figure}\thefigure: Homestead land holding of loan recipients, exclduing zero, real values\label{fig LandClass real}}\\
\hfil\includegraphics{c:/data/GUK/analysis/program/figure/EstimationMemo/LandHoldingHistogramByRoundRealValues.png}\\
\renewcommand{\arraystretch}{1}
\hfil\begin{tabular}{>{\hfill\scriptsize}p{1cm}<{}>{\scriptsize}p{12cm}<{\hfill}}
Source: & Survey data.\\
Note:&  Loan recipients only. Land holding is displayed on horizontal axises. Zero land holding is excluded from the graph. Mean including zero holding is shown as a dotted line in each panel.\\[1ex]
\end{tabular}
}

Land holding distributions look different between arms at the baseline (\textsc{\small Figure \ref{fig LandClass}}). This can be a consequence of copying the round 2 values which was justified under the assumption that all the round 1 land acquisition is reported in round 2 when our interviewer asks about it. %Even with this caveat, \textsf{traditional} arm has more uniform distribution than the other arms. 
By round 4, distributions become more skewed to left in all arms, indicating that some members have increased their land holding, only that such a tendency is weakest among the \textsf{traditional} arm. 

%Land holding before copying values from the 2nd round shows the same picture but only from 2014 onwards (\textsc{\small Figure \ref{fig RawLandClass}}). This is because we copied values from 2014 to construct values in 2013. The assumption there was that new land acquisition is all reported. There are only 8 cases of new land acquisition, so the 2013 distributions are almost identically copied from 2014 distributions. However, the distributions in 2015 and 2017 look very different. It suggests that land holding is not invariant and we are assuming that all the variations are due to the intervention. So long as our assumption of complete and accurate new land acquisition is true, then the wealth variations after the intervention should reveal its wealth impacts.



\begin{Schunk}
\begin{Soutput}
Key: <tee>
     tee traditional large large grace cattle   Sum
   <int>       <int> <int>       <int>  <int> <int>
1:     1         174   200         199    200   773
2:     2         166   194         177    195   732
3:     3         162   191         174    188   715
4:     4         133   179         155    151   618
\end{Soutput}
\end{Schunk}
Land entries by arm and round:
\begin{Schunk}
\begin{Soutput}
           Arm   NA.1   NA.2   NA.3   NA.4 NonNA.1 NonNA.2 NonNA.3 NonNA.4
        <fctr> <char> <char> <char> <char>  <char>  <char>  <char>  <char>
1:       large     85     85     85     86     139     142     140     141
2:      cattle    108    107    107     96      84      98     100      97
3: large grace     98     78     82     81     103     106     105     103
4: traditional     85     82     81     71      83      91      89      81
5:       Total    376    352    355    334     409     437     434     422
\end{Soutput}
\end{Schunk}


\begin{Schunk}
\begin{Soutput}
[1] 6
\end{Soutput}
\end{Schunk}



\hspace{-1cm}\begin{minipage}[t]{14cm}
\hfil\textsc{\normalsize Table \refstepcounter{table}\thetable: ANCOVA estimation of land holding\label{tab ANCOVA land}}\\
\setlength{\tabcolsep}{1pt}
\setlength{\baselineskip}{8pt}
\renewcommand{\arraystretch}{.55}
\hfil\begin{tikzpicture}
\node (tbl) {\input{c:/data/GUK/analysis/save/EstimationMemo/LandANCOVAEstimationResults.tex}};
%\input{c:/dropbox/data/ramadan/save/tablecolortemplate.tex}
\end{tikzpicture}\\
\renewcommand{\arraystretch}{.8}
\setlength{\tabcolsep}{1pt}
\begin{tabular}{>{\hfill\scriptsize}p{1cm}<{}>{\hfill\scriptsize}p{.25cm}<{}>{\scriptsize}p{12cm}<{\hfill}}
Source:& \multicolumn{2}{l}{\scriptsize Estimated with GUK administrative and survey data.}\\
Notes: & 1. & ANCOVA estimates using administrative and survey data. Post treatment regressands are regressed on categorical variables, pre-treatment regressand and other covariates. \textsf{FloodInRd1} and \textsf{HeadLiterate0} are indicator variables for the presence of self reported damage by a flood at the baseline, and literacy of household head, respectively. \textsf{HHsize0} is household size at the baseline. We annotate the number of periods that a household is observed with \textsf{T}. The total number of households is shown for each values of \textsf{T}. \textsf{T=4} indicates the number of households with complete panel information, \textsf{T=3} indicates number of households observed three times, \textsf{T=2} indicates the number of households observed twice. \textsf{N} indicates total number of observations used in ANCOVA estimation, or \textsf{N$=$1$\times$(T=2)+2$\times$(T=3)+3$\times$(T=4)}.  \textsf{Large}, \textsf{LargeGrace}, \textsf{Cattle} are indicator variables of the \textsf{large}, \textsf{large grace}, and \textsf{cattle} arms, respectively. The default arm category is \textsf{traditional} arm. Interaction terms of dummy variables are demeaned before interacting. The first column gives mean and standard deviation (in parenthesises) of each covariates before demeaning.\\
& 2. & $P$ values in percentages in parenthesises. Standard errors are clustered at group (village) level.
%${}^{***}$, ${}^{**}$, ${}^{*}$ indicate statistical significance at 1\%, 5\%, 10\%, respetively. Standard errors are clustered at group (village) level.
\end{tabular}
\end{minipage}

\hspace{-1cm}\begin{minipage}[t]{14cm}
\hfil\textsc{\normalsize Table \refstepcounter{table}\thetable: ANCOVA estimation of land holding by attributes\label{tab ANCOVA land attributes}}\\
\setlength{\tabcolsep}{1pt}
\setlength{\baselineskip}{8pt}
\renewcommand{\arraystretch}{.55}
\hfil\begin{tikzpicture}
\node (tbl) {\input{c:/data/GUK/analysis/save/EstimationMemo/LandAttributesANCOVAEstimationResults.tex}};
%\input{c:/dropbox/data/ramadan/save/tablecolortemplate.tex}
\end{tikzpicture}\\
\renewcommand{\arraystretch}{.8}
\setlength{\tabcolsep}{1pt}
\begin{tabular}{>{\hfill\scriptsize}p{1cm}<{}>{\hfill\scriptsize}p{.25cm}<{}>{\scriptsize}p{12cm}<{\hfill}}
Source:& \multicolumn{2}{l}{\scriptsize Estimated with GUK administrative and survey data.}\\
Notes: & 1. & ANCOVA estimates using administrative and survey data. Post treatment regressands are regressed on categorical variables, pre-treatment regressand and other covariates. \textsf{FloodInRd1} and \textsf{HeadLiterate0} are indicator variables for the presence of self reported damage by a flood at the baseline, and literacy of household head, respectively. \textsf{HHsize0} is household size at the baseline. We annotate the number of periods that a household is observed with \textsf{T}. The total number of households is shown for each values of \textsf{T}. \textsf{T=4} indicates the number of households with complete panel information, \textsf{T=3} indicates number of households observed three times, \textsf{T=2} indicates the number of households observed twice. \textsf{N} indicates total number of observations used in ANCOVA estimation, or \textsf{N$=$1$\times$(T=2)+2$\times$(T=3)+3$\times$(T=4)}.  \textsf{Upfront} is an indicator variable of the arm with an upfront large disbursement, \textsf{WithGrace} is an indicator variable of the arm with a grace period, \textsf{InKind} is an indicator variable of the arm which lends a heifer. Interaction terms of dummy variables are demeaned before interacting. The first column gives mean and standard deviation (in parenthesises) of each covariates before demeaning.\\
& 2. & $P$ values in percentages in parenthesises. Standard errors are clustered at group (village) level.
%${}^{***}$, ${}^{**}$, ${}^{*}$ indicate statistical significance at 1\%, 5\%, 10\%, respetively. Standard errors are clustered at group (village) level.
\end{tabular}
\end{minipage}

\hspace{-1cm}\begin{minipage}[t]{14cm}
\hfil\textsc{\normalsize Table \refstepcounter{table}\thetable: ANCOVA estimation of land holding by period, arm\label{tab ANCOVA land period}}\\
\setlength{\tabcolsep}{1pt}
\setlength{\baselineskip}{8pt}
\renewcommand{\arraystretch}{.55}
\hfil\begin{tikzpicture}
\node (tbl) {\input{c:/data/GUK/analysis/save/EstimationMemo/LandTimeVaryingANCOVAEstimationResults.tex}};
%\input{c:/dropbox/data/ramadan/save/tablecolortemplate.tex}
\end{tikzpicture}\\
\renewcommand{\arraystretch}{.8}
\setlength{\tabcolsep}{1pt}
\begin{tabular}{>{\hfill\scriptsize}p{1cm}<{}>{\hfill\scriptsize}p{.25cm}<{}>{\scriptsize}p{12cm}<{\hfill}}
Source:& \multicolumn{2}{l}{\scriptsize Estimated with GUK administrative and survey data.}\\
Notes: & 1. & ANCOVA estimates using administrative and survey data. Post treatment regressands are regressed on categorical variables, pre-treatment regressand and other covariates. \textsf{FloodInRd1} and \textsf{HeadLiterate0} are indicator variables for the presence of self reported damage by a flood at the baseline, and literacy of household head, respectively. \textsf{HHsize0} is household size at the baseline. We annotate the number of periods that a household is observed with \textsf{T}. The total number of households is shown for each values of \textsf{T}. \textsf{T=4} indicates the number of households with complete panel information, \textsf{T=3} indicates number of households observed three times, \textsf{T=2} indicates the number of households observed twice. \textsf{N} indicates total number of observations used in ANCOVA estimation, or \textsf{N$=$1$\times$(T=2)+2$\times$(T=3)+3$\times$(T=4)}.  \textsf{Large}, \textsf{LargeGrace}, \textsf{Cattle} are indicator variables of the \textsf{large}, \textsf{large grace}, and \textsf{cattle} arms, respectively. The default arm category is \textsf{traditional} arm. \textsf{rd2, rd3, rd4} are dummy variables for second, third, and fourth round of survey. Interaction terms of dummy variables are demeaned before interacting. The first column gives mean and standard deviation (in parenthesises) of each covariates before demeaning.\\
& 2. & $P$ values in percentages in parenthesises. Standard errors are clustered at group (village) level.
%${}^{***}$, ${}^{**}$, ${}^{*}$ indicate statistical significance at 1\%, 5\%, 10\%, respetively. Standard errors are clustered at group (village) level.
\end{tabular}
\end{minipage}

\hspace{-1cm}\begin{minipage}[t]{14cm}
\hfil\textsc{\normalsize Table \refstepcounter{table}\thetable: ANCOVA estimation of land holding by period, arm, and poverty status\label{tab ANCOVA land period poverty}}\\
\setlength{\tabcolsep}{1pt}
\setlength{\baselineskip}{8pt}
\renewcommand{\arraystretch}{.55}
\hfil\begin{tikzpicture}
\node (tbl) {\input{c:/data/GUK/analysis/save/EstimationMemo/LandTimeVaryingPovertyStatusANCOVAEstimationResults.tex}};
%\input{c:/dropbox/data/ramadan/save/tablecolortemplate.tex}
\end{tikzpicture}\\
\renewcommand{\arraystretch}{.8}
\setlength{\tabcolsep}{1pt}
\begin{tabular}{>{\hfill\scriptsize}p{1cm}<{}>{\hfill\scriptsize}p{.25cm}<{}>{\scriptsize}p{12cm}<{\hfill}}
Source:& \multicolumn{2}{l}{\scriptsize Estimated with GUK administrative and survey data.}\\
Notes: & 1. & ANCOVA estimates using administrative and survey data. Post treatment regressands are regressed on categorical variables, pre-treatment regressand and other covariates. \textsf{FloodInRd1} and \textsf{HeadLiterate0} are indicator variables for the presence of self reported damage by a flood at the baseline, and literacy of household head, respectively. \textsf{HHsize0} is household size at the baseline. We annotate the number of periods that a household is observed with \textsf{T}. The total number of households is shown for each values of \textsf{T}. \textsf{T=4} indicates the number of households with complete panel information, \textsf{T=3} indicates number of households observed three times, \textsf{T=2} indicates the number of households observed twice. \textsf{N} indicates total number of observations used in ANCOVA estimation, or \textsf{N$=$1$\times$(T=2)+2$\times$(T=3)+3$\times$(T=4)}.  \textsf{UltraPoor} is an indicator variable if the household is classified as the ultra poor. \textsf{Large}, \textsf{LargeGrace}, \textsf{Cattle} are indicator variables of the \textsf{large}, \textsf{large grace}, and \textsf{cattle} arms, respectively. The default arm category is \textsf{traditional} arm. \textsf{rd2, rd3, rd4} are dummy variables for second, third, and fourth round of survey. Interaction terms of dummy variables are demeaned before interacting. The first column gives mean and standard deviation (in parenthesises) of each covariates before demeaning.\\
& 2. & $P$ values in percentages in parenthesises. Standard errors are clustered at group (village) level.
%${}^{***}$, ${}^{**}$, ${}^{*}$ indicate statistical significance at 1\%, 5\%, 10\%, respetively. Standard errors are clustered at group (village) level.
\end{tabular}
\end{minipage}

\hspace{-1cm}\begin{minipage}[t]{14cm}
\hfil\textsc{\normalsize Table \refstepcounter{table}\thetable: ANCOVA estimation of land holding by period and attributes\label{tab ANCOVA land period attributes}}\\
\setlength{\tabcolsep}{1pt}
\setlength{\baselineskip}{8pt}
\renewcommand{\arraystretch}{.55}
\hfil\begin{tikzpicture}
\node (tbl) {\input{c:/data/GUK/analysis/save/EstimationMemo/LandTimeVaryingAttributesANCOVAEstimationResults.tex}};
%\input{c:/dropbox/data/ramadan/save/tablecolortemplate.tex}
\end{tikzpicture}\\
\renewcommand{\arraystretch}{.8}
\setlength{\tabcolsep}{1pt}
\begin{tabular}{>{\hfill\scriptsize}p{1cm}<{}>{\hfill\scriptsize}p{.25cm}<{}>{\scriptsize}p{12cm}<{\hfill}}
Source:& \multicolumn{2}{l}{\scriptsize Estimated with GUK administrative and survey data.}\\
Notes: & 1. & ANCOVA estimates using administrative and survey data. Post treatment regressands are regressed on categorical variables, pre-treatment regressand and other covariates. \textsf{FloodInRd1} and \textsf{HeadLiterate0} are indicator variables for the presence of self reported damage by a flood at the baseline, and literacy of household head, respectively. \textsf{HHsize0} is household size at the baseline. We annotate the number of periods that a household is observed with \textsf{T}. The total number of households is shown for each values of \textsf{T}. \textsf{T=4} indicates the number of households with complete panel information, \textsf{T=3} indicates number of households observed three times, \textsf{T=2} indicates the number of households observed twice. \textsf{N} indicates total number of observations used in ANCOVA estimation, or \textsf{N$=$1$\times$(T=2)+2$\times$(T=3)+3$\times$(T=4)}.  \textsf{Upfront} is an indicator variable of the arm with an upfront large disbursement, \textsf{WithGrace} is an indicator variable of the arm with a grace period, \textsf{InKind} is an indicator variable of the arm which lends a heifer. \textsf{rd2, rd3, rd4} are dummy variables for second, third, and fourth round of survey. Interaction terms of dummy variables are demeaned before interacting. The first column gives mean and standard deviation (in parenthesises) of each covariates before demeaning.\\
& 2. & $P$ values in percentages in parenthesises. Standard errors are clustered at group (village) level.
%${}^{***}$, ${}^{**}$, ${}^{*}$ indicate statistical significance at 1\%, 5\%, 10\%, respetively. Standard errors are clustered at group (village) level.
\end{tabular}
\end{minipage}

\hspace{-1cm}\begin{minipage}[t]{14cm}
\hfil\textsc{\normalsize Table \refstepcounter{table}\thetable: ANCOVA estimation of land holding by period, attributes, and poverty status\label{tab ANCOVA land period poverty attributes}}\\
\setlength{\tabcolsep}{1pt}
\setlength{\baselineskip}{8pt}
\renewcommand{\arraystretch}{.55}
\hfil\begin{tikzpicture}
\node (tbl) {\input{c:/data/GUK/analysis/save/EstimationMemo/LandTimeVaryingPovertyStatusAttributesANCOVAEstimationResults.tex}};
%\input{c:/dropbox/data/ramadan/save/tablecolortemplate.tex}
\end{tikzpicture}\\
\renewcommand{\arraystretch}{.8}
\setlength{\tabcolsep}{1pt}
\begin{tabular}{>{\hfill\scriptsize}p{1cm}<{}>{\hfill\scriptsize}p{.25cm}<{}>{\scriptsize}p{12cm}<{\hfill}}
Source:& \multicolumn{2}{l}{\scriptsize Estimated with GUK administrative and survey data.}\\
Notes: & 1. & ANCOVA estimates using administrative and survey data. Post treatment regressands are regressed on categorical variables, pre-treatment regressand and other covariates. \textsf{FloodInRd1} and \textsf{HeadLiterate0} are indicator variables for the presence of self reported damage by a flood at the baseline, and literacy of household head, respectively. \textsf{HHsize0} is household size at the baseline. We annotate the number of periods that a household is observed with \textsf{T}. The total number of households is shown for each values of \textsf{T}. \textsf{T=4} indicates the number of households with complete panel information, \textsf{T=3} indicates number of households observed three times, \textsf{T=2} indicates the number of households observed twice. \textsf{N} indicates total number of observations used in ANCOVA estimation, or \textsf{N$=$1$\times$(T=2)+2$\times$(T=3)+3$\times$(T=4)}.  \textsf{UltraPoor} is an indicator variable if the household is classified as the ultra poor. \textsf{Upfront} is an indicator variable of the arm with an upfront large disbursement, \textsf{WithGrace} is an indicator variable of the arm with a grace period, \textsf{InKind} is an indicator variable of the arm which lends a heifer. \textsf{rd2, rd3, rd4} are dummy variables for second, third, and fourth round of survey. Interaction terms of dummy variables are demeaned before interacting. The first column gives mean and standard deviation (in parenthesises) of each covariates before demeaning.\\
& 2. & $P$ values in percentages in parenthesises. Standard errors are clustered at group (village) level.
%${}^{***}$, ${}^{**}$, ${}^{*}$ indicate statistical significance at 1\%, 5\%, 10\%, respetively. Standard errors are clustered at group (village) level.
\end{tabular}
\end{minipage}

\textsc{\normalsize Table \ref{tab ANCOVA land period}} shows that, compared to \textsf{traditional} arm, land holding is larger for \textsf{large}, \textsf{large grace}, and \textsf{cattle} arms in round 2. The difference with \textsf{traditional} arm is maintained only for \textsf{large} arm throughout the rounds, and become negligible for \textsf{large grace} and \textsf{cattle} arms. 


%\subsubsection{Farm land}

\subsubsection{Livestock, number of cattle}


\begin{Schunk}
\begin{Soutput}
             AttritIn
Arm             2   3   4   9 Sum
  traditional   7   4  20 144 175
  large         5   2   1 192 200
  large grace  12   3   3 171 189
  cattle        5   5  13 176 199
  Sum          29  14  37 683 763
     NumCows
tee      0    1    2    3    4    5    6    7    8    9 <NA>  Sum
  2     15  309  153   40   11    1    2    0    1    1  197  730
  3      5  337  175   40   16    1    2    2    1    0  110  689
  4      4  218  201   54   11    4    2    0    1    1   86  582
  Sum   24  864  529  134   38    6    6    2    3    2  393 2001
\end{Soutput}
\end{Schunk}
%  source(paste0(pathprogram, "ReadTrimLivestockANCOVA.R"))



\begin{Schunk}
\begin{Soutput}
[1] 5
\end{Soutput}
\end{Schunk}




\subsubsection{Cattle holding}

\begin{Schunk}
\begin{Soutput}
             AttritIn
Arm             2   3   4   9 Sum
  traditional   7   4  20 144 175
  large         5   2   1 192 200
  large grace  12   3   3 171 189
  cattle        5   5  13 176 199
  Sum          29  14  37 683 763
     NumCows
tee      0    1    2    3    4    5    6    7    8    9 <NA>  Sum
  2     15  309  153   40   11    1    2    0    1    1  197  730
  3      5  337  175   40   16    1    2    2    1    0  110  689
  4      4  218  201   54   11    4    2    0    1    1   86  582
  Sum   24  864  529  134   38    6    6    2    3    2  393 2001
\end{Soutput}
\end{Schunk}
\begin{Schunk}
\begin{Soutput}
Warning in `[.data.table`(lvoN, , `:=`(grepout("TotalImputed2?Value.?", : Column 'TotalImputedValue' does not exist to remove
\end{Soutput}
\end{Schunk}
\begin{Schunk}
\begin{Soutput}
[1] 5
\end{Soutput}
\end{Schunk}


\begin{Schunk}
\begin{Soutput}
   HoldingClass
tee below 1000 1000-29999 30000-49999 above 50000 Sum
  1        623        100          30          10 763
  2        211        310         153          56 730
  3        115        337         175          62 689
  4         90        218         201          73 582
\end{Soutput}
\end{Schunk}
\begin{Schunk}
\begin{figure}

{\centering \includegraphics[width=\maxwidth]{figure/EstimationMemo/Total_imputed_value_histogram_original_HHs-1} 

}

\caption{Total imputed value of livestock holding\\ {\footnotesize Livestock holding values are computed by using respective median prices of each year.\setlength{\baselineskip}{8pt}}}\label{Figure Total imputed value histogram original HHs}
\end{figure}
\end{Schunk}
\begin{Schunk}
\begin{figure}

{\centering \includegraphics[width=\maxwidth]{figure/EstimationMemo/Histogram_of_livestock_holding_classes_original_HHs-1} 

}

\caption{Histogram of livestock holding classes\\ {\footnotesize Livestock holding values are computed by using respective median prices of each year.\setlength{\baselineskip}{8pt}}}\label{Figure Histogram of livestock holding classes original HHs}
\end{figure}
\end{Schunk}
\begin{Schunk}
\begin{figure}

{\centering \includegraphics[width=\maxwidth]{figure/EstimationMemo/Histogram_of_livestock_holding_classes_by_year_original_HHs-1} 

}

\caption{Histogram of livestock holding classes by year\\ {\footnotesize Livestock holding values are computed by using respective median prices of each year.\setlength{\baselineskip}{8pt}}}\label{Figure Histogram of livestock holding classes by year original HHs}
\end{figure}
\end{Schunk}
\begin{Schunk}
\begin{Soutput}
                      povertystatus
BStatus                Ultra Poor Moderately Poor <NA>
  borrower                    411             163    0
  pure saver                    0               0    0
  individual rejection         56              33    0
  group rejection               0               0   60
  rejection by flood            0               0   40
\end{Soutput}
\end{Schunk}

\mpage{\linewidth}{
\hfil\textsc{\footnotesize Figure \refstepcounter{figure}\thefigure: Lvestock holding at baseline\label{fig LivestockBaseline}}\\
\hfil\includegraphics[width = 12cm]{c:/data/GUK/analysis/program/figure/EstimationMemo/LivestockValuesAtRd1.eps}\\
\renewcommand{\arraystretch}{1}
\hfil\begin{tabular}{>{\hfill\scriptsize}p{1cm}<{}>{\scriptsize}p{12cm}<{\hfill}}
Source: & Survey data.\\
Note:& \\[1ex]
\end{tabular}
}

\begin{itemize}
\vspace{1.0ex}\setlength{\itemsep}{1.0ex}\setlength{\baselineskip}{12pt}
\item	\textsf{cattle} reports above 20000 holding in rds 2-4 while \textsf{traditional} does not.
\end{itemize}
\begin{Schunk}
\begin{Soutput}
            Arm survey MeanImputedVal MeanNumCows     N
         <fctr>  <num>          <num>       <num> <int>
 1: traditional      1        4557.82    0.227891   294
 2: traditional      2       18965.26    1.601449   234
 3: traditional      3       21270.53    1.502703   262
 4: traditional      4       23364.52    1.591195   217
 5:       large      1        5513.78    0.275689   399
 6:       large      2       29214.50    1.979253   327
 7:       large      3       31623.09    1.798799   379
 8:       large      4       33248.21    1.882175   375
 9: large grace      1        6666.67    0.333333   399
10: large grace      2       24273.25    1.648649   302
11: large grace      3       28044.08    1.532051   341
12: large grace      4       31599.85    1.744108   328
13:      cattle      1        4360.90    0.218045   399
14:      cattle      2       22239.38    1.457031   336
15:      cattle      3       26102.42    1.474522   355
16:      cattle      4       29716.61    1.655405   330
\end{Soutput}
\end{Schunk}



\mpage{\linewidth}{
\hfil\textsc{\footnotesize Figure \refstepcounter{figure}\thefigure: Number of cows/oxen by year\label{fig Number of cowsoxen by year}}\\
\hfil\includegraphics[width = 12cm]{c:/data/GUK/analysis/program/figure/EstimationMemo/NumberOfCowsByYear.png}\\
%\hfil\includegraphics[width = 12cm]{c:/data/GUK/analysis/program/figure/EstimationMemo/Number_of_cows_by_year_original_HHs-1.eps}\\
\renewcommand{\arraystretch}{1}
\hfil\begin{tabular}{>{\hfill\scriptsize}p{1cm}<{}>{\scriptsize}p{12cm}<{\hfill}}
Source: & Survey data.\\
Note:& \\[1ex]
\end{tabular}
}

\mpage{\linewidth}{
\hfil\textsc{\footnotesize Figure \refstepcounter{figure}\thefigure: Number of cows/oxen by survey round\label{fig Number of cowsoxen by survey round}}\\
\hfil\includegraphics[width = 12cm]{c:/data/GUK/analysis/program/figure/EstimationMemo/NumberOfCowsByRound.png}\\
%\hfil\includegraphics[width = 12cm]{c:/data/GUK/analysis/program/figure/EstimationMemo/Number_of_cows_by_year_original_HHs-1.eps}\\
\renewcommand{\arraystretch}{1}
\hfil\begin{tabular}{>{\hfill\scriptsize}p{1cm}<{}>{\scriptsize}p{12cm}<{\hfill}}
Source: & Survey data.\\
Note:& \\[1ex]
\end{tabular}
}



\hspace{-1cm}\begin{minipage}[t]{14cm}
\hfil\textsc{\normalsize Table \refstepcounter{table}\thetable: ANCOVA estimation of livestock holding values\label{tab ANCOVA livestock}}\\
\setlength{\tabcolsep}{1pt}
\setlength{\baselineskip}{8pt}
\renewcommand{\arraystretch}{.55}
\hfil\begin{tikzpicture}
\node (tbl) {\input{c:/data/GUK/analysis/save/EstimationMemo/LivestockANCOVAEstimationResults.tex}};
%\input{c:/dropbox/data/ramadan/save/tablecolortemplate.tex}
\end{tikzpicture}\\
\renewcommand{\arraystretch}{.8}
\setlength{\tabcolsep}{1pt}
\begin{tabular}{>{\hfill\scriptsize}p{1cm}<{}>{\hfill\scriptsize}p{.25cm}<{}>{\scriptsize}p{12cm}<{\hfill}}
Source:& \multicolumn{2}{l}{\scriptsize Estimated with GUK administrative and survey data.}\\
Notes: & 1. & ANCOVA estimates using administrative and survey data. Post treatment regressands are regressed on categorical variables, pre-treatment regressand and other covariates. \textsf{FloodInRd1} and \textsf{HeadLiterate0} are indicator variables for the presence of self reported damage by a flood at the baseline, and literacy of household head, respectively. \textsf{HHsize0} is household size at the baseline. We annotate the number of periods that a household is observed with \textsf{T}. The total number of households is shown for each values of \textsf{T}. \textsf{T=4} indicates the number of households with complete panel information, \textsf{T=3} indicates number of households observed three times, \textsf{T=2} indicates the number of households observed twice. \textsf{N} indicates total number of observations used in ANCOVA estimation, or \textsf{N$=$1$\times$(T=2)+2$\times$(T=3)+3$\times$(T=4)}.  \textsf{Large}, \textsf{LargeGrace}, \textsf{Cattle} are indicator variables of the \textsf{large}, \textsf{large grace}, and \textsf{cattle} arms, respectively. The default arm category is \textsf{traditional} arm. Regressand is \textsf{TotalImputedValue}, a sum of all livestock holding values evaluated at respective median market prices in the same year. \\
& 2. & $P$ values in percentages in parenthesises. Standard errors are clustered at group (village) level.%
%${}^{***}$, ${}^{**}$, ${}^{*}$ indicate statistical significance at 1\%, 5\%, 10\%, respetively. 
$P$ values in parenthesises. Standard errors are clustered at group (village) level.
\end{tabular}
\end{minipage}

\hspace{-1cm}\begin{minipage}[t]{14cm}
\hfil\textsc{\normalsize Table \refstepcounter{table}\thetable: ANCOVA estimation of livestock holding values by attributes\label{tab ANCOVA livestock attributes}}\\
\setlength{\tabcolsep}{1pt}
\setlength{\baselineskip}{8pt}
\renewcommand{\arraystretch}{.55}
\hfil\begin{tikzpicture}
\node (tbl) {\input{c:/data/GUK/analysis/save/EstimationMemo/LivestockAttributesANCOVAEstimationResults.tex}};
%\input{c:/dropbox/data/ramadan/save/tablecolortemplate.tex}
\end{tikzpicture}\\
\renewcommand{\arraystretch}{.8}
\setlength{\tabcolsep}{1pt}
\begin{tabular}{>{\hfill\scriptsize}p{1cm}<{}>{\hfill\scriptsize}p{.25cm}<{}>{\scriptsize}p{12cm}<{\hfill}}
Source:& \multicolumn{2}{l}{\scriptsize Estimated with GUK administrative and survey data.}\\
Notes: & 1. & ANCOVA estimates using administrative and survey data. Post treatment regressands are regressed on categorical variables, pre-treatment regressand and other covariates. \textsf{FloodInRd1} and \textsf{HeadLiterate0} are indicator variables for the presence of self reported damage by a flood at the baseline, and literacy of household head, respectively. \textsf{HHsize0} is household size at the baseline. We annotate the number of periods that a household is observed with \textsf{T}. The total number of households is shown for each values of \textsf{T}. \textsf{T=4} indicates the number of households with complete panel information, \textsf{T=3} indicates number of households observed three times, \textsf{T=2} indicates the number of households observed twice. \textsf{N} indicates total number of observations used in ANCOVA estimation, or \textsf{N$=$1$\times$(T=2)+2$\times$(T=3)+3$\times$(T=4)}.  \textsf{Upfront} is an indicator variable of the arm with an upfront large disbursement, \textsf{WithGrace} is an indicator variable of the arm with a grace period, \textsf{InKind} is an indicator variable of the arm which lends a heifer. Regressand is \textsf{TotalImputedValue}, a sum of all livestock holding values evaluated at respective median market prices in the same year. \\
& 2. & $P$ values in percentages in parenthesises. Standard errors are clustered at group (village) level.
%${}^{***}$, ${}^{**}$, ${}^{*}$ indicate statistical significance at 1\%, 5\%, 10\%, respetively. Standard errors are clustered at group (village) level.
\end{tabular}
\end{minipage}



\hspace{-1cm}\begin{minipage}[t]{14cm}
\hfil\textsc{\normalsize Table \refstepcounter{table}\thetable: ANCOVA estimation of livestock holding values, ultra vs. moderately poor\label{tab ANCOVA livestock poor}}\\
\setlength{\tabcolsep}{1pt}
\setlength{\baselineskip}{8pt}
\renewcommand{\arraystretch}{.55}
\hfil\begin{tikzpicture}
\node (tbl) {\input{c:/data/GUK/analysis/save/EstimationMemo/LivestockPovertyStatusANCOVAEstimationResults.tex}};
%\input{c:/dropbox/data/ramadan/save/tablecolortemplate.tex}
\end{tikzpicture}\\
\renewcommand{\arraystretch}{.8}
\setlength{\tabcolsep}{1pt}
\begin{tabular}{>{\hfill\scriptsize}p{1cm}<{}>{\hfill\scriptsize}p{.25cm}<{}>{\scriptsize}p{12cm}<{\hfill}}
Source:& \multicolumn{2}{l}{\scriptsize Estimated with GUK administrative and survey data.}\\
Notes: & 1. & ANCOVA estimates using administrative and survey data. Post treatment regressands are regressed on categorical variables, pre-treatment regressand and other covariates. \textsf{FloodInRd1} and \textsf{HeadLiterate0} are indicator variables for the presence of self reported damage by a flood at the baseline, and literacy of household head, respectively. \textsf{HHsize0} is household size at the baseline. We annotate the number of periods that a household is observed with \textsf{T}. The total number of households is shown for each values of \textsf{T}. \textsf{T=4} indicates the number of households with complete panel information, \textsf{T=3} indicates number of households observed three times, \textsf{T=2} indicates the number of households observed twice. \textsf{N} indicates total number of observations used in ANCOVA estimation, or \textsf{N$=$1$\times$(T=2)+2$\times$(T=3)+3$\times$(T=4)}.  \textsf{UltraPoor} is an indicator variable if the household is classified as the ultra poor. \textsf{Large}, \textsf{LargeGrace}, \textsf{Cattle} are indicator variables of the \textsf{large}, \textsf{large grace}, and \textsf{cattle} arms, respectively. The default arm category is \textsf{traditional} arm. Regressand is \textsf{TotalImputedValue}, a sum of all livestock holding values evaluated at respective median market prices in the same year. \\
& 2. & $P$ values in percentages in parenthesises. Standard errors are clustered at group (village) level.
%${}^{***}$, ${}^{**}$, ${}^{*}$ indicate statistical significance at 1\%, 5\%, 10\%, respetively. Standard errors are clustered at group (village) level.
\end{tabular}
\end{minipage}

\hspace{-1cm}\begin{minipage}[t]{14cm}
\hfil\textsc{\normalsize Table \refstepcounter{table}\thetable: ANCOVA estimation of livestock holding values by attributes and period\label{tab ANCOVA livestock timevarying attributes}}\\
\setlength{\tabcolsep}{1pt}
\setlength{\baselineskip}{8pt}
\renewcommand{\arraystretch}{.55}
\hfil\begin{tikzpicture}
\node (tbl) {\input{c:/data/GUK/analysis/save/EstimationMemo/LivestockTimeVaryingAttributesANCOVAEstimationResults.tex}};
%\input{c:/dropbox/data/ramadan/save/tablecolortemplate.tex}
\end{tikzpicture}\\
\renewcommand{\arraystretch}{.8}
\setlength{\tabcolsep}{1pt}
\begin{tabular}{>{\hfill\scriptsize}p{1cm}<{}>{\hfill\scriptsize}p{.25cm}<{}>{\scriptsize}p{12cm}<{\hfill}}
Source:& \multicolumn{2}{l}{\scriptsize Estimated with GUK administrative and survey data.}\\
Notes: & 1. & ANCOVA estimates using administrative and survey data. Post treatment regressands are regressed on categorical variables, pre-treatment regressand and other covariates. \textsf{FloodInRd1} and \textsf{HeadLiterate0} are indicator variables for the presence of self reported damage by a flood at the baseline, and literacy of household head, respectively. \textsf{HHsize0} is household size at the baseline. We annotate the number of periods that a household is observed with \textsf{T}. The total number of households is shown for each values of \textsf{T}. \textsf{T=4} indicates the number of households with complete panel information, \textsf{T=3} indicates number of households observed three times, \textsf{T=2} indicates the number of households observed twice. \textsf{N} indicates total number of observations used in ANCOVA estimation, or \textsf{N$=$1$\times$(T=2)+2$\times$(T=3)+3$\times$(T=4)}.  \textsf{UltraPoor} is an indicator variable if the household is classified as the ultra poor. \textsf{Upfront} is an indicator variable of the arm with an upfront large disbursement, \textsf{WithGrace} is an indicator variable of the arm with a grace period, \textsf{InKind} is an indicator variable of the arm which lends a heifer. \textsf{rd2, rd3, rd4} are dummy variables for second, third, and fourth round of survey. Regressand is \textsf{TotalImputedValue}, a sum of all livestock holding values evaluated at respective median market prices in the same year. \\
& 2. & $P$ values in percentages in parenthesises. Standard errors are clustered at group (village) level.
%${}^{***}$, ${}^{**}$, ${}^{*}$ indicate statistical significance at 1\%, 5\%, 10\%, respetively. Standard errors are clustered at group (village) level.
\end{tabular}
\end{minipage}


\hspace{-1cm}\begin{minipage}[t]{14cm}
\hfil\textsc{\normalsize Table \refstepcounter{table}\thetable: ANCOVA estimation of livestock holding values using annual prices\label{tab ANCOVA livestock annual price}}\\
\setlength{\tabcolsep}{1pt}
\setlength{\baselineskip}{8pt}
\renewcommand{\arraystretch}{.55}
\hfil\begin{tikzpicture}
\node (tbl) {\input{c:/data/GUK/analysis/save/EstimationMemo/LivestockAnnualPriceANCOVAEstimationResults.tex}};
%\input{c:/dropbox/data/ramadan/save/tablecolortemplate.tex}
\end{tikzpicture}\\
\renewcommand{\arraystretch}{.8}
\setlength{\tabcolsep}{1pt}
\begin{tabular}{>{\hfill\scriptsize}p{1cm}<{}>{\hfill\scriptsize}p{.25cm}<{}>{\scriptsize}p{12cm}<{\hfill}}
Source:& \multicolumn{2}{l}{\scriptsize Estimated with GUK administrative and survey data.}\\
Notes: & 1. & ANCOVA estimates using administrative and survey data. Post treatment regressands are regressed on categorical variables, pre-treatment regressand and other covariates. \textsf{FloodInRd1} and \textsf{HeadLiterate0} are indicator variables for the presence of self reported damage by a flood at the baseline, and literacy of household head, respectively. \textsf{HHsize0} is household size at the baseline. We annotate the number of periods that a household is observed with \textsf{T}. The total number of households is shown for each values of \textsf{T}. \textsf{T=4} indicates the number of households with complete panel information, \textsf{T=3} indicates number of households observed three times, \textsf{T=2} indicates the number of households observed twice. \textsf{N} indicates total number of observations used in ANCOVA estimation, or \textsf{N$=$1$\times$(T=2)+2$\times$(T=3)+3$\times$(T=4)}.  \textsf{Large}, \textsf{LargeGrace}, \textsf{Cattle} are indicator variables of the \textsf{large}, \textsf{large grace}, and \textsf{cattle} arms, respectively. The default arm category is \textsf{traditional} arm. Regressand is \textsf{TotalImputedValue}, a sum of all livestock holding values evaluated at respective median market prices in the same year. \\
& 2. & $P$ values in percentages in parenthesises. Standard errors are clustered at group (village) level.%
%${}^{***}$, ${}^{**}$, ${}^{*}$ indicate statistical significance at 1\%, 5\%, 10\%, respetively. 
$P$ values in parenthesises. Standard errors are clustered at group (village) level.
\end{tabular}
\end{minipage}

\hspace{-1cm}\begin{minipage}[t]{14cm}
\hfil\textsc{\normalsize Table \refstepcounter{table}\thetable: ANCOVA estimation of livestock holding values using annual prices by attributes\label{tab ANCOVA livestock  annual price attributes}}\\
\setlength{\tabcolsep}{1pt}
\setlength{\baselineskip}{8pt}
\renewcommand{\arraystretch}{.55}
\hfil\begin{tikzpicture}
\node (tbl) {\input{c:/data/GUK/analysis/save/EstimationMemo/LivestockAnnualPriceAttributesANCOVAEstimationResults.tex}};
%\input{c:/dropbox/data/ramadan/save/tablecolortemplate.tex}
\end{tikzpicture}\\
\renewcommand{\arraystretch}{.8}
\setlength{\tabcolsep}{1pt}
\begin{tabular}{>{\hfill\scriptsize}p{1cm}<{}>{\hfill\scriptsize}p{.25cm}<{}>{\scriptsize}p{12cm}<{\hfill}}
Source:& \multicolumn{2}{l}{\scriptsize Estimated with GUK administrative and survey data.}\\
Notes: & 1. & ANCOVA estimates using administrative and survey data. Post treatment regressands are regressed on categorical variables, pre-treatment regressand and other covariates. \textsf{FloodInRd1} and \textsf{HeadLiterate0} are indicator variables for the presence of self reported damage by a flood at the baseline, and literacy of household head, respectively. \textsf{HHsize0} is household size at the baseline. We annotate the number of periods that a household is observed with \textsf{T}. The total number of households is shown for each values of \textsf{T}. \textsf{T=4} indicates the number of households with complete panel information, \textsf{T=3} indicates number of households observed three times, \textsf{T=2} indicates the number of households observed twice. \textsf{N} indicates total number of observations used in ANCOVA estimation, or \textsf{N$=$1$\times$(T=2)+2$\times$(T=3)+3$\times$(T=4)}.  \textsf{Upfront} is an indicator variable of the arm with an upfront large disbursement, \textsf{WithGrace} is an indicator variable of the arm with a grace period, \textsf{InKind} is an indicator variable of the arm which lends a heifer. Regressand is \textsf{TotalImputedValue}, a sum of all livestock holding values evaluated at respective median market prices in the same year. \\
& 2. & $P$ values in percentages in parenthesises. Standard errors are clustered at group (village) level.
%${}^{***}$, ${}^{**}$, ${}^{*}$ indicate statistical significance at 1\%, 5\%, 10\%, respetively. Standard errors are clustered at group (village) level.
\end{tabular}
\end{minipage}



\hspace{-1cm}\begin{minipage}[t]{14cm}
\hfil\textsc{\normalsize Table \refstepcounter{table}\thetable: ANCOVA estimation of livestock holding values using annual prices, ultra vs. moderately poor\label{tab ANCOVA livestock  annual price poor}}\\
\setlength{\tabcolsep}{1pt}
\setlength{\baselineskip}{8pt}
\renewcommand{\arraystretch}{.55}
\hfil\begin{tikzpicture}
\node (tbl) {\input{c:/data/GUK/analysis/save/EstimationMemo/LivestockAnnualPricePovertyStatusANCOVAEstimationResults.tex}};
%\input{c:/dropbox/data/ramadan/save/tablecolortemplate.tex}
\end{tikzpicture}\\
\renewcommand{\arraystretch}{.8}
\setlength{\tabcolsep}{1pt}
\begin{tabular}{>{\hfill\scriptsize}p{1cm}<{}>{\hfill\scriptsize}p{.25cm}<{}>{\scriptsize}p{12cm}<{\hfill}}
Source:& \multicolumn{2}{l}{\scriptsize Estimated with GUK administrative and survey data.}\\
Notes: & 1. & ANCOVA estimates using administrative and survey data. Post treatment regressands are regressed on categorical variables, pre-treatment regressand and other covariates. \textsf{FloodInRd1} and \textsf{HeadLiterate0} are indicator variables for the presence of self reported damage by a flood at the baseline, and literacy of household head, respectively. \textsf{HHsize0} is household size at the baseline. We annotate the number of periods that a household is observed with \textsf{T}. The total number of households is shown for each values of \textsf{T}. \textsf{T=4} indicates the number of households with complete panel information, \textsf{T=3} indicates number of households observed three times, \textsf{T=2} indicates the number of households observed twice. \textsf{N} indicates total number of observations used in ANCOVA estimation, or \textsf{N$=$1$\times$(T=2)+2$\times$(T=3)+3$\times$(T=4)}.  \textsf{UltraPoor} is an indicator variable if the household is classified as the ultra poor. \textsf{Large}, \textsf{LargeGrace}, \textsf{Cattle} are indicator variables of the \textsf{large}, \textsf{large grace}, and \textsf{cattle} arms, respectively. The default arm category is \textsf{traditional} arm. Regressand is \textsf{TotalImputedValue}, a sum of all livestock holding values evaluated at respective median market prices in the same year. \\
& 2. & $P$ values in percentages in parenthesises. Standard errors are clustered at group (village) level.
%${}^{***}$, ${}^{**}$, ${}^{*}$ indicate statistical significance at 1\%, 5\%, 10\%, respetively. Standard errors are clustered at group (village) level.
\end{tabular}
\end{minipage}

\hspace{-1cm}\begin{minipage}[t]{14cm}
\hfil\textsc{\normalsize Table \refstepcounter{table}\thetable: ANCOVA estimation of livestock holding values using annual prices by attributes and period\label{tab ANCOVA livestock  annual price timevarying attributes}}\\
\setlength{\tabcolsep}{1pt}
\setlength{\baselineskip}{8pt}
\renewcommand{\arraystretch}{.55}
\hfil\begin{tikzpicture}
\node (tbl) {\input{c:/data/GUK/analysis/save/EstimationMemo/LivestockAnnualPriceTimeVaryingAttributesANCOVAEstimationResults.tex}};
%\input{c:/dropbox/data/ramadan/save/tablecolortemplate.tex}
\end{tikzpicture}\\
\renewcommand{\arraystretch}{.8}
\setlength{\tabcolsep}{1pt}
\begin{tabular}{>{\hfill\scriptsize}p{1cm}<{}>{\hfill\scriptsize}p{.25cm}<{}>{\scriptsize}p{12cm}<{\hfill}}
Source:& \multicolumn{2}{l}{\scriptsize Estimated with GUK administrative and survey data.}\\
Notes: & 1. & ANCOVA estimates using administrative and survey data. Post treatment regressands are regressed on categorical variables, pre-treatment regressand and other covariates. \textsf{FloodInRd1} and \textsf{HeadLiterate0} are indicator variables for the presence of self reported damage by a flood at the baseline, and literacy of household head, respectively. \textsf{HHsize0} is household size at the baseline. We annotate the number of periods that a household is observed with \textsf{T}. The total number of households is shown for each values of \textsf{T}. \textsf{T=4} indicates the number of households with complete panel information, \textsf{T=3} indicates number of households observed three times, \textsf{T=2} indicates the number of households observed twice. \textsf{N} indicates total number of observations used in ANCOVA estimation, or \textsf{N$=$1$\times$(T=2)+2$\times$(T=3)+3$\times$(T=4)}.  \textsf{UltraPoor} is an indicator variable if the household is classified as the ultra poor. \textsf{Upfront} is an indicator variable of the arm with an upfront large disbursement, \textsf{WithGrace} is an indicator variable of the arm with a grace period, \textsf{InKind} is an indicator variable of the arm which lends a heifer. \textsf{rd2, rd3, rd4} are dummy variables for second, third, and fourth round of survey. Regressand is \textsf{TotalImputedValue}, a sum of all livestock holding values evaluated at respective median market prices in the same year. \\
& 2. & $P$ values in percentages in parenthesises. Standard errors are clustered at group (village) level.
%${}^{***}$, ${}^{**}$, ${}^{*}$ indicate statistical significance at 1\%, 5\%, 10\%, respetively. Standard errors are clustered at group (village) level.
\end{tabular}
\end{minipage}


\hspace{-1cm}\begin{minipage}[t]{14cm}
\hfil\textsc{\normalsize Table \refstepcounter{table}\thetable: ANCOVA estimation of livestock holding \label{tab ANCOVA livestock holding}}\\
\setlength{\tabcolsep}{1pt}
\setlength{\baselineskip}{8pt}
\renewcommand{\arraystretch}{.55}
\hfil\begin{tikzpicture}
\node (tbl) {\input{c:/data/GUK/analysis/save/EstimationMemo/NumCowsANCOVAEstimationResults.tex}};
%\input{c:/dropbox/data/ramadan/save/tablecolortemplate.tex}
\end{tikzpicture}\\
\renewcommand{\arraystretch}{.8}
\setlength{\tabcolsep}{1pt}
\begin{tabular}{>{\hfill\scriptsize}p{1cm}<{}>{\hfill\scriptsize}p{.25cm}<{}>{\scriptsize}p{12cm}<{\hfill}}
Source:& \multicolumn{2}{l}{\scriptsize Estimated with GUK administrative and survey data.}\\
Notes: & 1. & ANCOVA estimates using administrative and survey data. Post treatment regressands are regressed on categorical variables, pre-treatment regressand and other covariates. \textsf{FloodInRd1} and \textsf{HeadLiterate0} are indicator variables for the presence of self reported damage by a flood at the baseline, and literacy of household head, respectively. \textsf{HHsize0} is household size at the baseline. We annotate the number of periods that a household is observed with \textsf{T}. The total number of households is shown for each values of \textsf{T}. \textsf{T=4} indicates the number of households with complete panel information, \textsf{T=3} indicates number of households observed three times, \textsf{T=2} indicates the number of households observed twice. \textsf{N} indicates total number of observations used in ANCOVA estimation, or \textsf{N$=$1$\times$(T=2)+2$\times$(T=3)+3$\times$(T=4)}.  \textsf{UltraPoor} is an indicator variable if the household is classified as the ultra poor. \textsf{Upfront} is an indicator variable of the arm with an upfront large disbursement, \textsf{WithGrace} is an indicator variable of the arm with a grace period, \textsf{InKind} is an indicator variable of the arm which lends a heifer. Regressand is \textsf{TotalImputedValue}, a sum of all livestock holding values evaluated at respective median market prices in the same year. \\
& 2. & $P$ values in percentages in parenthesises. Standard errors are clustered at group (village) level.
%${}^{***}$, ${}^{**}$, ${}^{*}$ indicate statistical significance at 1\%, 5\%, 10\%, respetively. Standard errors are clustered at group (village) level.
\end{tabular}
\end{minipage}

\hspace{-1cm}\begin{minipage}[t]{14cm}
\hfil\textsc{\normalsize Table \refstepcounter{table}\thetable: ANCOVA estimation of livestock holding by attributes\label{tab ANCOVA livestock holding attributes}}\\
\setlength{\tabcolsep}{1pt}
\setlength{\baselineskip}{8pt}
\renewcommand{\arraystretch}{.55}
\hfil\begin{tikzpicture}
\node (tbl) {\input{c:/data/GUK/analysis/save/EstimationMemo/NumCowsAttributesANCOVAEstimationResults.tex}};
%\input{c:/dropbox/data/ramadan/save/tablecolortemplate.tex}
\end{tikzpicture}\\
\renewcommand{\arraystretch}{.8}
\setlength{\tabcolsep}{1pt}
\begin{tabular}{>{\hfill\scriptsize}p{1cm}<{}>{\hfill\scriptsize}p{.25cm}<{}>{\scriptsize}p{12cm}<{\hfill}}
Source:& \multicolumn{2}{l}{\scriptsize Estimated with GUK administrative and survey data.}\\
Notes: & 1. & ANCOVA estimates using administrative and survey data. Post treatment regressands are regressed on categorical variables, pre-treatment regressand and other covariates. \textsf{FloodInRd1} and \textsf{HeadLiterate0} are indicator variables for the presence of self reported damage by a flood at the baseline, and literacy of household head, respectively. \textsf{HHsize0} is household size at the baseline. We annotate the number of periods that a household is observed with \textsf{T}. The total number of households is shown for each values of \textsf{T}. \textsf{T=4} indicates the number of households with complete panel information, \textsf{T=3} indicates number of households observed three times, \textsf{T=2} indicates the number of households observed twice. \textsf{N} indicates total number of observations used in ANCOVA estimation, or \textsf{N$=$1$\times$(T=2)+2$\times$(T=3)+3$\times$(T=4)}.  \textsf{UltraPoor} is an indicator variable if the household is classified as the ultra poor. \textsf{Upfront} is an indicator variable of the arm with an upfront large disbursement, \textsf{WithGrace} is an indicator variable of the arm with a grace period, \textsf{InKind} is an indicator variable of the arm which lends a heifer. Regressand is \textsf{TotalImputedValue}, a sum of all livestock holding values evaluated at respective median market prices in the same year. \\
& 2. & $P$ values in percentages in parenthesises. Standard errors are clustered at group (village) level.
%${}^{***}$, ${}^{**}$, ${}^{*}$ indicate statistical significance at 1\%, 5\%, 10\%, respetively. Standard errors are clustered at group (village) level.
\end{tabular}
\end{minipage}

\hspace{-1cm}\begin{minipage}[t]{14cm}
\hfil\textsc{\normalsize Table \refstepcounter{table}\thetable: ANCOVA estimation of livestock holding, ultra vs. moderately poor\label{tab ANCOVA livestock holding poor}}\\
\setlength{\tabcolsep}{1pt}
\setlength{\baselineskip}{8pt}
\renewcommand{\arraystretch}{.55}
\hfil\begin{tikzpicture}
\node (tbl) {\input{c:/data/GUK/analysis/save/EstimationMemo/NumCowsPovertyStatusANCOVAEstimationResults.tex}};
%\input{c:/dropbox/data/ramadan/save/tablecolortemplate.tex}
\end{tikzpicture}\\
\renewcommand{\arraystretch}{.8}
\setlength{\tabcolsep}{1pt}
\begin{tabular}{>{\hfill\scriptsize}p{1cm}<{}>{\hfill\scriptsize}p{.25cm}<{}>{\scriptsize}p{12cm}<{\hfill}}
Source:& \multicolumn{2}{l}{\scriptsize Estimated with GUK administrative and survey data.}\\
Notes: & 1. & ANCOVA estimates using administrative and survey data. Post treatment regressands are regressed on categorical variables, pre-treatment regressand and other covariates. \textsf{FloodInRd1} and \textsf{HeadLiterate0} are indicator variables for the presence of self reported damage by a flood at the baseline, and literacy of household head, respectively. \textsf{HHsize0} is household size at the baseline. We annotate the number of periods that a household is observed with \textsf{T}. The total number of households is shown for each values of \textsf{T}. \textsf{T=4} indicates the number of households with complete panel information, \textsf{T=3} indicates number of households observed three times, \textsf{T=2} indicates the number of households observed twice. \textsf{N} indicates total number of observations used in ANCOVA estimation, or \textsf{N$=$1$\times$(T=2)+2$\times$(T=3)+3$\times$(T=4)}.  \textsf{UltraPoor} is an indicator variable if the household is classified as the ultra poor. \textsf{Large}, \textsf{LargeGrace}, \textsf{Cattle} are indicator variables of the \textsf{large}, \textsf{large grace}, and \textsf{cattle} arms, respectively. The default arm category is \textsf{traditional} arm. Regressand is \textsf{TotalImputedValue}, a sum of all livestock holding values evaluated at respective median market prices in the same year. \\
& 2. & $P$ values in percentages in parenthesises. Standard errors are clustered at group (village) level.
%${}^{***}$, ${}^{**}$, ${}^{*}$ indicate statistical significance at 1\%, 5\%, 10\%, respetively. Standard errors are clustered at group (village) level.
\end{tabular}
\end{minipage}

\hspace{-1cm}\begin{minipage}[t]{14cm}
\hfil\textsc{\normalsize Table \refstepcounter{table}\thetable: ANCOVA estimation of livestock holding by time\label{tab ANCOVA livestock holding time varying}}\\
\setlength{\tabcolsep}{1pt}
\setlength{\baselineskip}{8pt}
\renewcommand{\arraystretch}{.55}
\hfil\begin{tikzpicture}
\node (tbl) {\input{c:/data/GUK/analysis/save/EstimationMemo/NumCowsTimeVaryingANCOVAEstimationResults.tex}};
%\input{c:/dropbox/data/ramadan/save/tablecolortemplate.tex}
\end{tikzpicture}\\
\renewcommand{\arraystretch}{.8}
\setlength{\tabcolsep}{1pt}
\begin{tabular}{>{\hfill\scriptsize}p{1cm}<{}>{\hfill\scriptsize}p{.25cm}<{}>{\scriptsize}p{12cm}<{\hfill}}
Source:& \multicolumn{2}{l}{\scriptsize Estimated with GUK administrative and survey data.}\\
Notes: & 1. & ANCOVA estimates using administrative and survey data. Post treatment regressands are regressed on categorical variables, pre-treatment regressand and other covariates. \textsf{FloodInRd1} and \textsf{HeadLiterate0} are indicator variables for the presence of self reported damage by a flood at the baseline, and literacy of household head, respectively. \textsf{HHsize0} is household size at the baseline. We annotate the number of periods that a household is observed with \textsf{T}. The total number of households is shown for each values of \textsf{T}. \textsf{T=4} indicates the number of households with complete panel information, \textsf{T=3} indicates number of households observed three times, \textsf{T=2} indicates the number of households observed twice. \textsf{N} indicates total number of observations used in ANCOVA estimation, or \textsf{N$=$1$\times$(T=2)+2$\times$(T=3)+3$\times$(T=4)}.  \textsf{UltraPoor} is an indicator variable if the household is classified as the ultra poor. \textsf{Upfront} is an indicator variable of the arm with an upfront large disbursement, \textsf{WithGrace} is an indicator variable of the arm with a grace period, \textsf{InKind} is an indicator variable of the arm which lends a heifer. \textsf{rd2, rd3, rd4} are dummy variables for second, third, and fourth round of survey. Regressand is \textsf{TotalImputedValue}, a sum of all livestock holding values evaluated at respective median market prices in the same year. \\
& 2. & $P$ values in percentages in parenthesises. Standard errors are clustered at group (village) level.
%${}^{***}$, ${}^{**}$, ${}^{*}$ indicate statistical significance at 1\%, 5\%, 10\%, respetively. Standard errors are clustered at group (village) level.
\end{tabular}
\end{minipage}

\hspace{-1cm}\begin{minipage}[t]{14cm}
\hfil\textsc{\normalsize Table \refstepcounter{table}\thetable: ANCOVA estimation of livestock holding by timeand attributes\label{tab ANCOVA livestock holding time varying attributes}}\\
\setlength{\tabcolsep}{1pt}
\setlength{\baselineskip}{8pt}
\renewcommand{\arraystretch}{.55}
\hfil\begin{tikzpicture}
\node (tbl) {\input{c:/data/GUK/analysis/save/EstimationMemo/NumCowsTimeVaryingAttributesANCOVAEstimationResults.tex}};
%\input{c:/dropbox/data/ramadan/save/tablecolortemplate.tex}
\end{tikzpicture}\\
\renewcommand{\arraystretch}{.8}
\setlength{\tabcolsep}{1pt}
\begin{tabular}{>{\hfill\scriptsize}p{1cm}<{}>{\hfill\scriptsize}p{.25cm}<{}>{\scriptsize}p{12cm}<{\hfill}}
Source:& \multicolumn{2}{l}{\scriptsize Estimated with GUK administrative and survey data.}\\
Notes: & 1. & ANCOVA estimates using administrative and survey data. Post treatment regressands are regressed on categorical variables, pre-treatment regressand and other covariates. \textsf{FloodInRd1} and \textsf{HeadLiterate0} are indicator variables for the presence of self reported damage by a flood at the baseline, and literacy of household head, respectively. \textsf{HHsize0} is household size at the baseline. We annotate the number of periods that a household is observed with \textsf{T}. The total number of households is shown for each values of \textsf{T}. \textsf{T=4} indicates the number of households with complete panel information, \textsf{T=3} indicates number of households observed three times, \textsf{T=2} indicates the number of households observed twice. \textsf{N} indicates total number of observations used in ANCOVA estimation, or \textsf{N$=$1$\times$(T=2)+2$\times$(T=3)+3$\times$(T=4)}.  \textsf{UltraPoor} is an indicator variable if the household is classified as the ultra poor. \textsf{Upfront} is an indicator variable of the arm with an upfront large disbursement, \textsf{WithGrace} is an indicator variable of the arm with a grace period, \textsf{InKind} is an indicator variable of the arm which lends a heifer. \textsf{rd2, rd3, rd4} are dummy variables for second, third, and fourth round of survey. Regressand is \textsf{TotalImputedValue}, a sum of all livestock holding values evaluated at respective median market prices in the same year. \\
& 2. & $P$ values in percentages in parenthesises. Standard errors are clustered at group (village) level.
%${}^{***}$, ${}^{**}$, ${}^{*}$ indicate statistical significance at 1\%, 5\%, 10\%, respetively. Standard errors are clustered at group (village) level.
\end{tabular}
\end{minipage}



\begin{palepinkleftbar}
\begin{finding}
\textsc{Figure \ref{Figure Histogram of livestock holding classes by year original HHs}} shows increasing livestock accumulation in all arms but \textsf{traditional}. 
\textsc{Figure \ref{fig Number of cowsoxen by year}} shows increasng cow ownership relative to \textsf{traditional} in the bottom panel while the holding per owner is similar across all arms. This is evidence of an acceleration of becoming a large livestock owner for the large sized arms relative to the small size arm. Given that the number of cows per owner remains the similar, it does not provide evidence for accelerated growth of livestock after becoming an owner.
\end{finding}
\end{palepinkleftbar}




\subsubsection{Productive assets}



\begin{Schunk}
\begin{Soutput}


Number of obs by Arm and attrition
             AttritIn
Arm             2   3   4   9 Sum
  traditional   6   4  20 144 174
  large         5   2   1 192 200
  large grace  22   3   3 171 199
  cattle        5   5  13 177 200
  Sum          38  14  37 684 773


Number of obs by membership status and attrition
                      AttritIn
BStatus                  2   3   4   9 Sum
  borrower               8   6   8 578 600
  pure saver             0   0   0   0   0
  individual rejection   9   4   1  75  89
  group rejection        9   4   0  55  68
  rejection by flood    12   0  28   0  40
  Sum                   38  14  37 708 797
\end{Soutput}
\end{Schunk}








\mpage{\linewidth}{
\hfil\textsc{\footnotesize Figure \refstepcounter{figure}\thefigure: Productive asset holding\label{fig prod asset survey round}}\\
\hfil\includegraphics[width = 12cm]{c:/data/GUK/analysis/program/figure/EstimationMemo/ProdAssets.png}\\
\renewcommand{\arraystretch}{1}
\hfil\begin{tabular}{>{\hfill\scriptsize}p{1cm}<{}>{\scriptsize}p{12cm}<{\hfill}}
Source: & Survey data.\\
Note:& \\[1ex]
\end{tabular}
}



\hspace{-1cm}\begin{minipage}[t]{14cm}
\hfil\textsc{\normalsize Table \refstepcounter{table}\thetable: ANCOVA estimation of productive assets\label{tab ANCOVA productive assets}}\\
\setlength{\tabcolsep}{1pt}
\setlength{\baselineskip}{8pt}
\renewcommand{\arraystretch}{.55}
\hfil\begin{tikzpicture}
\node (tbl) {\input{c:/data/GUK/analysis/save/EstimationMemo/ProdAssetANCOVAEstimationResults.tex}};
%\input{c:/dropbox/data/ramadan/save/tablecolortemplate.tex}
\end{tikzpicture}\\
\renewcommand{\arraystretch}{.8}
\setlength{\tabcolsep}{1pt}
\begin{tabular}{>{\hfill\scriptsize}p{1cm}<{}>{\hfill\scriptsize}p{.25cm}<{}>{\scriptsize}p{12cm}<{\hfill}}
Source:& \multicolumn{2}{l}{\scriptsize Estimated with GUK administrative and survey data.}\\
Notes: & 1. & ANCOVA estimates using administrative and survey data. Post treatment regressands are regressed on categorical variables, pre-treatment regressand and other covariates. \textsf{FloodInRd1} and \textsf{HeadLiterate0} are indicator variables for the presence of self reported damage by a flood at the baseline, and literacy of household head, respectively. \textsf{HHsize0} is household size at the baseline. \textsf{Large}, \textsf{LargeGrace}, \textsf{Cattle} are indicator variables of the \textsf{large}, \textsf{large grace}, and \textsf{cattle} arms, respectively. The default arm category is \textsf{traditional} arm. Only households that are observed for all 4 rounds are used. Households are continuing members and replacing members of early rejecters and received loans prior to 2015 Janunary. Productive assets do not include livestock. Regressions (1)-(3), (5)-(6) use only arm and calendar information. (4) and (7) use previous six month repayment and saving information which is lacking in rd 1, hence starts from rd 2.\\
& 2. & $P$ values in percentages in parenthesises. Standard errors are clustered at group (village) level.
%${}^{***}$, ${}^{**}$, ${}^{*}$ indicate statistical significance at 1\%, 5\%, 10\%, respetively. Standard errors are clustered at group (village) level.
\end{tabular}
\end{minipage}

\hspace{-1cm}\begin{minipage}[t]{14cm}
\hfil\textsc{\normalsize Table \refstepcounter{table}\thetable: ANCOVA estimation of productive assets by attributes\label{tab ANCOVA productive assets attributes}}\\
\setlength{\tabcolsep}{1pt}
\setlength{\baselineskip}{8pt}
\renewcommand{\arraystretch}{.55}
\hfil\begin{tikzpicture}
\node (tbl) {\input{c:/data/GUK/analysis/save/EstimationMemo/ProdAssetAttributesANCOVAEstimationResults.tex}};
%\input{c:/dropbox/data/ramadan/save/tablecolortemplate.tex}
\end{tikzpicture}\\
\renewcommand{\arraystretch}{.8}
\setlength{\tabcolsep}{1pt}
\begin{tabular}{>{\hfill\scriptsize}p{1cm}<{}>{\hfill\scriptsize}p{.25cm}<{}>{\scriptsize}p{12cm}<{\hfill}}
Source:& \multicolumn{2}{l}{\scriptsize Estimated with GUK administrative and survey data.}\\
Notes: & 1. & ANCOVA estimates using administrative and survey data. Post treatment regressands are regressed on categorical variables, pre-treatment regressand and other covariates. \textsf{FloodInRd1} and \textsf{HeadLiterate0} are indicator variables for the presence of self reported damage by a flood at the baseline, and literacy of household head, respectively. \textsf{HHsize0} is household size at the baseline. \textsf{Large}, \textsf{LargeGrace}, \textsf{Cattle} are indicator variables of the \textsf{large}, \textsf{large grace}, and \textsf{cattle} arms, respectively. The default arm category is \textsf{traditional} arm. Only households that are observed for all 4 rounds are used. Households are continuing members and replacing members of early rejecters and received loans prior to 2015 Janunary. Productive assets do not include livestock. Regressions (1)-(3), (5)-(6) use only arm and calendar information. (4) and (7) use previous six month repayment and saving information which is lacking in rd 1, hence starts from rd 2.\\
& 2. & $P$ values in percentages in parenthesises. Standard errors are clustered at group (village) level.
%${}^{***}$, ${}^{**}$, ${}^{*}$ indicate statistical significance at 1\%, 5\%, 10\%, respetively. Standard errors are clustered at group (village) level.
\end{tabular}
\end{minipage}

\hspace{-1cm}\begin{minipage}[t]{14cm}
\hfil\textsc{\normalsize Table \refstepcounter{table}\thetable: ANCOVA estimation of broad productive assets by period\label{tab ANCOVA productive assets timevarying}}\\
\setlength{\tabcolsep}{1pt}
\setlength{\baselineskip}{8pt}
\renewcommand{\arraystretch}{.55}
\hfil\begin{tikzpicture}
\node (tbl) {\input{c:/data/GUK/analysis/save/EstimationMemo/ProdAssetTimeVaryingANCOVAEstimationResults.tex}};
%\input{c:/dropbox/data/ramadan/save/tablecolortemplate.tex}
\end{tikzpicture}\\
\renewcommand{\arraystretch}{.8}
\setlength{\tabcolsep}{1pt}
\begin{tabular}{>{\hfill\scriptsize}p{1cm}<{}>{\hfill\scriptsize}p{.25cm}<{}>{\scriptsize}p{12cm}<{\hfill}}
Source:& \multicolumn{2}{l}{\scriptsize Estimated with GUK administrative and survey data.}\\
Notes: & 1. & ANCOVA estimates using administrative and survey data. Post treatment regressands are regressed on categorical variables, pre-treatment regressand and other covariates. \textsf{FloodInRd1} and \textsf{HeadLiterate0} are indicator variables for the presence of self reported damage by a flood at the baseline, and literacy of household head, respectively. \textsf{HHsize0} is household size at the baseline. \textsf{Large}, \textsf{LargeGrace}, \textsf{Cattle} are indicator variables of the \textsf{large}, \textsf{large grace}, and \textsf{cattle} arms, respectively. The default arm category is \textsf{traditional} arm. \textsf{rd2, rd3, rd4} are dummy variables for second, third, and fourth round of survey. Only households that are observed for all 4 rounds are used. Households are continuing members and replacing members of early rejecters and received loans prior to 2015 Janunary. Productive assets do not include livestock. Regressions (1)-(3), (5)-(6) use only arm and calendar information. (4) and (7) use previous six month repayment and saving information which is lacking in rd 1, hence starts from rd 2.\\
& 2. & $P$ values in percentages in parenthesises. Standard errors are clustered at group (village) level.
%${}^{***}$, ${}^{**}$, ${}^{*}$ indicate statistical significance at 1\%, 5\%, 10\%, respetively. Standard errors are clustered at group (village) level.
\end{tabular}
\end{minipage}

\hspace{-1cm}\begin{minipage}[t]{14cm}
\hfil\textsc{\normalsize Table \refstepcounter{table}\thetable: ANCOVA estimation of broad productive assets by attributes and period\label{tab ANCOVA productive assets timevarying attributes}}\\
\setlength{\tabcolsep}{1pt}
\setlength{\baselineskip}{8pt}
\renewcommand{\arraystretch}{.55}
\hfil\begin{tikzpicture}
\node (tbl) {\input{c:/data/GUK/analysis/save/EstimationMemo/ProdAssetTimeVaryingAttributesANCOVAEstimationResults.tex}};
%\input{c:/dropbox/data/ramadan/save/tablecolortemplate.tex}
\end{tikzpicture}\\
\renewcommand{\arraystretch}{.8}
\setlength{\tabcolsep}{1pt}
\begin{tabular}{>{\hfill\scriptsize}p{1cm}<{}>{\hfill\scriptsize}p{.25cm}<{}>{\scriptsize}p{12cm}<{\hfill}}
Source:& \multicolumn{2}{l}{\scriptsize Estimated with GUK administrative and survey data.}\\
Notes: & 1. & ANCOVA estimates using administrative and survey data. Post treatment regressands are regressed on categorical variables, pre-treatment regressand and other covariates. \textsf{FloodInRd1} and \textsf{HeadLiterate0} are indicator variables for the presence of self reported damage by a flood at the baseline, and literacy of household head, respectively. \textsf{HHsize0} is household size at the baseline. \textsf{Large}, \textsf{LargeGrace}, \textsf{Cattle} are indicator variables of the \textsf{large}, \textsf{large grace}, and \textsf{cattle} arms, respectively. The default arm category is \textsf{traditional} arm. \textsf{rd2, rd3, rd4} are dummy variables for second, third, and fourth round of survey. Only households that are observed for all 4 rounds are used. Households are continuing members and replacing members of early rejecters and received loans prior to 2015 Janunary.  Productive assets do not include livestock. Regressions (1)-(3), (5)-(6) use only arm and calendar information. (4) and (7) use previous six month repayment and saving information which is lacking in rd 1, hence starts from rd 2.\\
& 2. & $P$ values in percentages in parenthesises. Standard errors are clustered at group (village) level.
%${}^{***}$, ${}^{**}$, ${}^{*}$ indicate statistical significance at 1\%, 5\%, 10\%, respetively. Standard errors are clustered at group (village) level.
\end{tabular}
\end{minipage}


\hspace{-1cm}\begin{minipage}[t]{14cm}
\hfil\textsc{\normalsize Table \refstepcounter{table}\thetable: ANCOVA estimation of broad productive assets by attributes, poverty status, and period\label{tab ANCOVA productive assets timevarying poverty status attributes}}\\
\setlength{\tabcolsep}{1pt}
\setlength{\baselineskip}{8pt}
\renewcommand{\arraystretch}{.55}
\hfil\begin{tikzpicture}
\node (tbl) {\input{c:/data/GUK/analysis/save/EstimationMemo/ProdAssetTimeVaryingPovertyStatusAttributesANCOVAEstimationResults.tex}};
%\input{c:/dropbox/data/ramadan/save/tablecolortemplate.tex}
\end{tikzpicture}\\
\renewcommand{\arraystretch}{.8}
\setlength{\tabcolsep}{1pt}
\begin{tabular}{>{\hfill\scriptsize}p{1cm}<{}>{\hfill\scriptsize}p{.25cm}<{}>{\scriptsize}p{12cm}<{\hfill}}
Source:& \multicolumn{2}{l}{\scriptsize Estimated with GUK administrative and survey data.}\\
Notes: & 1. & ANCOVA estimates using administrative and survey data. Post treatment regressands are regressed on categorical variables, pre-treatment regressand and other covariates. \textsf{FloodInRd1} and \textsf{HeadLiterate0} are indicator variables for the presence of self reported damage by a flood at the baseline, and literacy of household head, respectively. \textsf{HHsize0} is household size at the baseline. \textsf{Large}, \textsf{LargeGrace}, \textsf{Cattle} are indicator variables of the \textsf{large}, \textsf{large grace}, and \textsf{cattle} arms, respectively. The default arm category is \textsf{traditional} arm. \textsf{rd2, rd3, rd4} are dummy variables for second, third, and fourth round of survey. Only households that are observed for all 4 rounds are used. Households are continuing members and replacing members of early rejecters and received loans prior to 2015 Janunary.  Productive assets do not include livestock. Regressions (1)-(3), (5)-(6) use only arm and calendar information. (4) and (7) use previous six month repayment and saving information which is lacking in rd 1, hence starts from rd 2.\\
& 2. & $P$ values in percentages in parenthesises. Standard errors are clustered at group (village) level.
%${}^{***}$, ${}^{**}$, ${}^{*}$ indicate statistical significance at 1\%, 5\%, 10\%, respetively. Standard errors are clustered at group (village) level.
\end{tabular}
\end{minipage}



\subsubsection{Narrow productive assets}


Narrow productive assets are productive assets that are reported in all rounds. They are bees-box, brooder, cage incubator, country boat, dheki, ginning machine, gola (grain storage), hand pump, husking machine, jata, ladder(moi), sickle/dao/axe/spade, spray, weeder



\begin{Schunk}
\begin{Soutput}


Number of obs by Arm and attrition
             AttritIn
Arm             2   3   4   9 Sum
  traditional   6   4  20 144 174
  large         5   2   1 192 200
  large grace  22   3   3 171 199
  cattle        5   5  13 177 200
  Sum          38  14  37 684 773


Number of obs by membership status and attrition
                      AttritIn
BStatus                  2   3   4   9 Sum
  borrower               8   6   8 578 600
  pure saver             0   0   0   0   0
  individual rejection   9   4   1  75  89
  group rejection        9   4   0  55  68
  rejection by flood    12   0  28   0  40
  Sum                   38  14  37 708 797
\end{Soutput}
\end{Schunk}





\begin{Schunk}
\begin{Soutput}
Error in `geom_boxplot()`:
! Problem while computing aesthetics.
i Error occurred in the 1st layer.
Caused by error:
! オブジェクト 'NarrowPAssetAmount' がありません
\end{Soutput}
\begin{Soutput}
Error in `geom_boxplot()`:
! Problem while computing aesthetics.
i Error occurred in the 1st layer.
Caused by error:
! オブジェクト 'NarrowPAssetAmount' がありません
\end{Soutput}
\end{Schunk}


\mpage{\linewidth}{
\hfil\textsc{\footnotesize Figure \refstepcounter{figure}\thefigure: Narrow productive asset holding\label{fig narrow prod asset survey round}}\\
\hfil\includegraphics[width = 12cm]{c:/data/GUK/analysis/program/figure/EstimationMemo/NarrowProdAssets.pdf}\\
\renewcommand{\arraystretch}{1}
\hfil\begin{tabular}{>{\hfill\scriptsize}p{1cm}<{}>{\scriptsize}p{12cm}<{\hfill}}
Source: & Survey data.\\
Note:& Narrow productive assets are productive assets that are reported in all rounds. They are bees-box, brooder, cage incubator, country boat, dheki, ginning machine, gola (grain storage), hand pump, husking machine, jata, ladder(moi), sickle/dao/axe/spade, spray, weeder. \\[1ex]
\end{tabular}
}



\hspace{-1cm}\begin{minipage}[t]{14cm}
\hfil\textsc{\normalsize Table \refstepcounter{table}\thetable: ANCOVA estimation of narrow productive assets\label{tab ANCOVA narrow productive assets}}\\
\setlength{\tabcolsep}{1pt}
\setlength{\baselineskip}{8pt}
\renewcommand{\arraystretch}{.55}
\hfil\begin{tikzpicture}
\node (tbl) {\input{c:/data/GUK/analysis/save/EstimationMemo/NarrowProdAssetANCOVAEstimationResults.tex}};
%\input{c:/dropbox/data/ramadan/save/tablecolortemplate.tex}
\end{tikzpicture}\\
\renewcommand{\arraystretch}{.8}
\setlength{\tabcolsep}{1pt}
\begin{tabular}{>{\hfill\scriptsize}p{1cm}<{}>{\hfill\scriptsize}p{.25cm}<{}>{\scriptsize}p{12cm}<{\hfill}}
Source:& \multicolumn{2}{l}{\scriptsize Estimated with GUK administrative and survey data.}\\
Notes: & 1. & ANCOVA estimates using administrative and survey data. Post treatment regressands are regressed on categorical variables, pre-treatment regressand and other covariates. \textsf{FloodInRd1} and \textsf{HeadLiterate0} are indicator variables for the presence of self reported damage by a flood at the baseline, and literacy of household head, respectively. \textsf{HHsize0} is household size at the baseline. We annotate the number of periods that a household is observed with \textsf{T}. The total number of households is shown for each values of \textsf{T}. \textsf{T=4} indicates the number of households with complete panel information, \textsf{T=3} indicates number of households observed three times, \textsf{T=2} indicates the number of households observed twice. \textsf{N} indicates total number of observations used in ANCOVA estimation, or \textsf{N$=$1$\times$(T=2)+2$\times$(T=3)+3$\times$(T=4)}.  \textsf{Large}, \textsf{LargeGrace}, \textsf{Cattle} are indicator variables of the \textsf{large}, \textsf{large grace}, and \textsf{cattle} arms, respectively. The default arm category is \textsf{traditional} arm. Sample is continuing members and replacing members of early rejecters and received loans prior to 2015 Janunary. Productive assets do not include livestock. Regressions (1)-(3), (5)-(6) use only arm and calendar information. (4) and (7) use previous six month repayment and saving information which is lacking in rd 1, hence starts from rd 2.\\
& 2. & $P$ values in percentages in parenthesises. Standard errors are clustered at group (village) level.
%${}^{***}$, ${}^{**}$, ${}^{*}$ indicate statistical significance at 1\%, 5\%, 10\%, respetively. Standard errors are clustered at group (village) level.
\end{tabular}
\end{minipage}

\hspace{-1cm}\begin{minipage}[t]{14cm}
\hfil\textsc{\normalsize Table \refstepcounter{table}\thetable: ANCOVA estimation of narrow productive assets by attributes\label{tab ANCOVA narrow productive assets attributes}}\\
\setlength{\tabcolsep}{1pt}
\setlength{\baselineskip}{8pt}
\renewcommand{\arraystretch}{.55}
\hfil\begin{tikzpicture}
\node (tbl) {\input{c:/data/GUK/analysis/save/EstimationMemo/NarrowProdAssetAttributesANCOVAEstimationResults.tex}};
%\input{c:/dropbox/data/ramadan/save/tablecolortemplate.tex}
\end{tikzpicture}\\
\renewcommand{\arraystretch}{.8}
\setlength{\tabcolsep}{1pt}
\begin{tabular}{>{\hfill\scriptsize}p{1cm}<{}>{\hfill\scriptsize}p{.25cm}<{}>{\scriptsize}p{12cm}<{\hfill}}
Source:& \multicolumn{2}{l}{\scriptsize Estimated with GUK administrative and survey data.}\\
Notes: & 1. & ANCOVA estimates using administrative and survey data. Post treatment regressands are regressed on categorical variables, pre-treatment regressand and other covariates. \textsf{FloodInRd1} and \textsf{HeadLiterate0} are indicator variables for the presence of self reported damage by a flood at the baseline, and literacy of household head, respectively. \textsf{HHsize0} is household size at the baseline. We annotate the number of periods that a household is observed with \textsf{T}. The total number of households is shown for each values of \textsf{T}. \textsf{T=4} indicates the number of households with complete panel information, \textsf{T=3} indicates number of households observed three times, \textsf{T=2} indicates the number of households observed twice. \textsf{N} indicates total number of observations used in ANCOVA estimation, or \textsf{N$=$1$\times$(T=2)+2$\times$(T=3)+3$\times$(T=4)}.  \textsf{Upfront} is an indicator variable of the arm with an upfront large disbursement, \textsf{WithGrace} is an indicator variable of the arm with a grace period, \textsf{InKind} is an indicator variable of the arm which lends a heifer. Sample is continuing members and replacing members of early rejecters and received loans prior to 2015 Janunary. Productive assets do not include livestock. Regressions (1)-(3), (5)-(6) use only arm and calendar information. (4) and (7) use previous six month repayment and saving information which is lacking in rd 1, hence starts from rd 2.\\
& 2. & $P$ values in percentages in parenthesises. Standard errors are clustered at group (village) level.
%${}^{***}$, ${}^{**}$, ${}^{*}$ indicate statistical significance at 1\%, 5\%, 10\%, respetively. Standard errors are clustered at group (village) level.
\end{tabular}
\end{minipage}

\hspace{-1cm}\begin{minipage}[t]{14cm}
\hfil\textsc{\normalsize Table \refstepcounter{table}\thetable: ANCOVA estimation of narrow productive assets by period\label{tab ANCOVA narrow productive assets timevarying}}\\
\setlength{\tabcolsep}{1pt}
\setlength{\baselineskip}{8pt}
\renewcommand{\arraystretch}{.55}
\hfil\begin{tikzpicture}
\node (tbl) {\input{c:/data/GUK/analysis/save/EstimationMemo/NarrowProdAssetTimeVaryingANCOVAEstimationResults.tex}};
%\input{c:/dropbox/data/ramadan/save/tablecolortemplate.tex}
\end{tikzpicture}\\
\renewcommand{\arraystretch}{.8}
\setlength{\tabcolsep}{1pt}
\begin{tabular}{>{\hfill\scriptsize}p{1cm}<{}>{\hfill\scriptsize}p{.25cm}<{}>{\scriptsize}p{12cm}<{\hfill}}
Source:& \multicolumn{2}{l}{\scriptsize Estimated with GUK administrative and survey data.}\\
Notes: & 1. & ANCOVA estimates using administrative and survey data. Post treatment regressands are regressed on categorical variables, pre-treatment regressand and other covariates. \textsf{FloodInRd1} and \textsf{HeadLiterate0} are indicator variables for the presence of self reported damage by a flood at the baseline, and literacy of household head, respectively. \textsf{HHsize0} is household size at the baseline. We annotate the number of periods that a household is observed with \textsf{T}. The total number of households is shown for each values of \textsf{T}. \textsf{T=4} indicates the number of households with complete panel information, \textsf{T=3} indicates number of households observed three times, \textsf{T=2} indicates the number of households observed twice. \textsf{N} indicates total number of observations used in ANCOVA estimation, or \textsf{N$=$1$\times$(T=2)+2$\times$(T=3)+3$\times$(T=4)}.  \textsf{Large}, \textsf{LargeGrace}, \textsf{Cattle} are indicator variables of the \textsf{large}, \textsf{large grace}, and \textsf{cattle} arms, respectively. The default arm category is \textsf{traditional} arm. \textsf{rd2, rd3, rd4} are dummy variables for second, third, and fourth round of survey.  Productive assets do not include livestock. Regressions (1)-(3), (5)-(6) use only arm and calendar information. (4) and (7) use previous six month repayment and saving information which is lacking in rd 1, hence starts from rd 2.\\
& 2. & $P$ values in percentages in parenthesises. Standard errors are clustered at group (village) level.
%${}^{***}$, ${}^{**}$, ${}^{*}$ indicate statistical significance at 1\%, 5\%, 10\%, respetively. Standard errors are clustered at group (village) level.
\end{tabular}
\end{minipage}

\hspace{-1cm}\begin{minipage}[t]{14cm}
\hfil\textsc{\normalsize Table \refstepcounter{table}\thetable: ANCOVA estimation of narrow productive assets by attributes and period\label{tab ANCOVA narrow productive assets timevarying attributes}}\\
\setlength{\tabcolsep}{1pt}
\setlength{\baselineskip}{8pt}
\renewcommand{\arraystretch}{.55}
\hfil\begin{tikzpicture}
\node (tbl) {\input{c:/data/GUK/analysis/save/EstimationMemo/NarrowProdAssetTimeVaryingAttributesANCOVAEstimationResults.tex}};
%\input{c:/dropbox/data/ramadan/save/tablecolortemplate.tex}
\end{tikzpicture}\\
\renewcommand{\arraystretch}{.8}
\setlength{\tabcolsep}{1pt}
\begin{tabular}{>{\hfill\scriptsize}p{1cm}<{}>{\hfill\scriptsize}p{.25cm}<{}>{\scriptsize}p{12cm}<{\hfill}}
Source:& \multicolumn{2}{l}{\scriptsize Estimated with GUK administrative and survey data.}\\
Notes: & 1. & ANCOVA estimates using administrative and survey data. Post treatment regressands are regressed on categorical variables, pre-treatment regressand and other covariates. \textsf{FloodInRd1} and \textsf{HeadLiterate0} are indicator variables for the presence of self reported damage by a flood at the baseline, and literacy of household head, respectively. \textsf{HHsize0} is household size at the baseline. We annotate the number of periods that a household is observed with \textsf{T}. The total number of households is shown for each values of \textsf{T}. \textsf{T=4} indicates the number of households with complete panel information, \textsf{T=3} indicates number of households observed three times, \textsf{T=2} indicates the number of households observed twice. \textsf{N} indicates total number of observations used in ANCOVA estimation, or \textsf{N$=$1$\times$(T=2)+2$\times$(T=3)+3$\times$(T=4)}.  \textsf{Upfront} is an indicator variable of the arm with an upfront large disbursement, \textsf{WithGrace} is an indicator variable of the arm with a grace period, \textsf{InKind} is an indicator variable of the arm which lends a heifer. \textsf{rd2, rd3, rd4} are dummy variables for second, third, and fourth round of survey. Productive assets do not include livestock. Regressions (1)-(3), (5)-(6) use only arm and calendar information. (4) and (7) use previous six month repayment and saving information which is lacking in rd 1, hence starts from rd 2.\\
& 2. & $P$ values in percentages in parenthesises. Standard errors are clustered at group (village) level.
%${}^{***}$, ${}^{**}$, ${}^{*}$ indicate statistical significance at 1\%, 5\%, 10\%, respetively. Standard errors are clustered at group (village) level.
\end{tabular}
\end{minipage}


\hspace{-1cm}\begin{minipage}[t]{14cm}
\hfil\textsc{\normalsize Table \refstepcounter{table}\thetable: ANCOVA estimation of narrow productive assets by attributes, poverty status, and period\label{tab ANCOVA narrow productive assets timevarying poverty status attributes}}\\
\setlength{\tabcolsep}{1pt}
\setlength{\baselineskip}{8pt}
\renewcommand{\arraystretch}{.55}
\hfil\begin{tikzpicture}
\node (tbl) {\input{c:/data/GUK/analysis/save/EstimationMemo/NarrowProdAssetTimeVaryingPovertyStatusAttributesANCOVAEstimationResults.tex}};
%\input{c:/dropbox/data/ramadan/save/tablecolortemplate.tex}
\end{tikzpicture}\\
\renewcommand{\arraystretch}{.8}
\setlength{\tabcolsep}{1pt}
\begin{tabular}{>{\hfill\scriptsize}p{1cm}<{}>{\hfill\scriptsize}p{.25cm}<{}>{\scriptsize}p{12cm}<{\hfill}}
Source:& \multicolumn{2}{l}{\scriptsize Estimated with GUK administrative and survey data.}\\
Notes: & 1. & ANCOVA estimates using administrative and survey data. Post treatment regressands are regressed on categorical variables, pre-treatment regressand and other covariates. \textsf{FloodInRd1} and \textsf{HeadLiterate0} are indicator variables for the presence of self reported damage by a flood at the baseline, and literacy of household head, respectively. \textsf{HHsize0} is household size at the baseline. We annotate the number of periods that a household is observed with \textsf{T}. The total number of households is shown for each values of \textsf{T}. \textsf{T=4} indicates the number of households with complete panel information, \textsf{T=3} indicates number of households observed three times, \textsf{T=2} indicates the number of households observed twice. \textsf{N} indicates total number of observations used in ANCOVA estimation, or \textsf{N$=$1$\times$(T=2)+2$\times$(T=3)+3$\times$(T=4)}.  \textsf{UltraPoor} is an indicator variable if the household is classified as the ultra poor. \textsf{Upfront} is an indicator variable of the arm with an upfront large disbursement, \textsf{WithGrace} is an indicator variable of the arm with a grace period, \textsf{InKind} is an indicator variable of the arm which lends a heifer. \textsf{rd2, rd3, rd4} are dummy variables for second, third, and fourth round of survey. Productive assets do not include livestock. Regressions (1)-(3), (5)-(6) use only arm and calendar information. (4) and (7) use previous six month repayment and saving information which is lacking in rd 1, hence starts from rd 2.\\
& 2. & $P$ values in percentages in parenthesises. Standard errors are clustered at group (village) level.
%${}^{***}$, ${}^{**}$, ${}^{*}$ indicate statistical significance at 1\%, 5\%, 10\%, respetively. Standard errors are clustered at group (village) level.
\end{tabular}
\end{minipage}



\subsubsection{Productive assets+livestock}


\begin{Schunk}
\begin{Soutput}


Number of obs by Arm and attrition
             AttritIn
Arm             2   3   4   9 Sum
  traditional   6   4  20 144 174
  large         5   2   1 192 200
  large grace  22   3   3 171 199
  cattle        5   5  13 177 200
  Sum          38  14  37 684 773


Number of obs by membership status and attrition
                      AttritIn
BStatus                  2   3   4   9 Sum
  borrower               8   6   8 578 600
  pure saver             0   0   0   0   0
  individual rejection   9   4   1  75  89
  group rejection        9   4   0  55  68
  rejection by flood    12   0  28   0  40
  Sum                   38  14  37 708 797
\end{Soutput}
\end{Schunk}









\hspace{-1cm}\begin{minipage}[t]{14cm}
\hfil\textsc{\normalsize Table \refstepcounter{table}\thetable: ANCOVA estimation of productive and livestock assets\label{tab ANCOVA productive and livestock assets}}\\
\setlength{\tabcolsep}{1pt}
\setlength{\baselineskip}{8pt}
\renewcommand{\arraystretch}{.55}
\hfil\begin{tikzpicture}
\node (tbl) {\input{c:/data/GUK/analysis/save/EstimationMemo/ProdAssetLivestockANCOVAEstimationResults.tex}};
%\input{c:/dropbox/data/ramadan/save/tablecolortemplate.tex}
\end{tikzpicture}\\
\renewcommand{\arraystretch}{.8}
\setlength{\tabcolsep}{1pt}
\begin{tabular}{>{\hfill\scriptsize}p{1cm}<{}>{\hfill\scriptsize}p{.25cm}<{}>{\scriptsize}p{12cm}<{\hfill}}
Source:& \multicolumn{2}{l}{\scriptsize Estimated with GUK administrative and survey data.}\\
Notes: & 1. & ANCOVA estimates using administrative and survey data. Post treatment regressands are regressed on categorical variables, pre-treatment regressand and other covariates. \textsf{FloodInRd1} and \textsf{HeadLiterate0} are indicator variables for the presence of self reported damage by a flood at the baseline, and literacy of household head, respectively. \textsf{HHsize0} is household size at the baseline. We annotate the number of periods that a household is observed with \textsf{T}. The total number of households is shown for each values of \textsf{T}. \textsf{T=4} indicates the number of households with complete panel information, \textsf{T=3} indicates number of households observed three times, \textsf{T=2} indicates the number of households observed twice. \textsf{N} indicates total number of observations used in ANCOVA estimation, or \textsf{N$=$1$\times$(T=2)+2$\times$(T=3)+3$\times$(T=4)}.  \textsf{Large}, \textsf{LargeGrace}, \textsf{Cattle} are indicator variables of the \textsf{large}, \textsf{large grace}, and \textsf{cattle} arms, respectively. The default arm category is \textsf{traditional} arm. Sample is continuing members and replacing members of early rejecters and received loans prior to 2015 Janunary. Household assets do not include livestock. Regressions (1)-(3), (5)-(6) use only arm and calendar information. (4) and (7) use previous six month repayment and saving information which is lacking in rd 1, hence starts from rd 2.\\
& 2. & $P$ values in percentages in parenthesises. Standard errors are clustered at group (village) level.
%${}^{***}$, ${}^{**}$, ${}^{*}$ indicate statistical significance at 1\%, 5\%, 10\%, respetively. Standard errors are clustered at group (village) level.
\end{tabular}
\end{minipage}

\hspace{-1cm}\begin{minipage}[t]{14cm}
\hfil\textsc{\normalsize Table \refstepcounter{table}\thetable: ANCOVA estimation of productive and livestock assets by attributes\label{tab ANCOVA productive and livestock assets attributes}}\\
\setlength{\tabcolsep}{1pt}
\setlength{\baselineskip}{8pt}
\renewcommand{\arraystretch}{.55}
\hfil\begin{tikzpicture}
\node (tbl) {\input{c:/data/GUK/analysis/save/EstimationMemo/ProdAssetLivestockAttributesANCOVAEstimationResults.tex}};
%\input{c:/dropbox/data/ramadan/save/tablecolortemplate.tex}
\end{tikzpicture}\\
\renewcommand{\arraystretch}{.8}
\setlength{\tabcolsep}{1pt}
\begin{tabular}{>{\hfill\scriptsize}p{1cm}<{}>{\hfill\scriptsize}p{.25cm}<{}>{\scriptsize}p{12cm}<{\hfill}}
Source:& \multicolumn{2}{l}{\scriptsize Estimated with GUK administrative and survey data.}\\
Notes: & 1. & ANCOVA estimates using administrative and survey data. Post treatment regressands are regressed on categorical variables, pre-treatment regressand and other covariates. \textsf{FloodInRd1} and \textsf{HeadLiterate0} are indicator variables for the presence of self reported damage by a flood at the baseline, and literacy of household head, respectively. \textsf{HHsize0} is household size at the baseline. We annotate the number of periods that a household is observed with \textsf{T}. The total number of households is shown for each values of \textsf{T}. \textsf{T=4} indicates the number of households with complete panel information, \textsf{T=3} indicates number of households observed three times, \textsf{T=2} indicates the number of households observed twice. \textsf{N} indicates total number of observations used in ANCOVA estimation, or \textsf{N$=$1$\times$(T=2)+2$\times$(T=3)+3$\times$(T=4)}.  \textsf{Upfront} is an indicator variable of the arm with an upfront large disbursement, \textsf{WithGrace} is an indicator variable of the arm with a grace period, \textsf{InKind} is an indicator variable of the arm which lends a heifer. Sample is continuing members and replacing members of early rejecters and received loans prior to 2015 Janunary. Household assets do not include livestock. Regressions (1)-(3), (5)-(6) use only arm and calendar information. (4) and (7) use previous six month repayment and saving information which is lacking in rd 1, hence starts from rd 2.\\
& 2. & $P$ values in percentages in parenthesises. Standard errors are clustered at group (village) level.
%${}^{***}$, ${}^{**}$, ${}^{*}$ indicate statistical significance at 1\%, 5\%, 10\%, respetively. Standard errors are clustered at group (village) level.
\end{tabular}
\end{minipage}

\hspace{-1cm}\begin{minipage}[t]{14cm}
\hfil\textsc{\normalsize Table \refstepcounter{table}\thetable: ANCOVA estimation of livestock and productive assets by period\label{tab ANCOVA productive assets timevarying}}\\
\setlength{\tabcolsep}{1pt}
\setlength{\baselineskip}{8pt}
\renewcommand{\arraystretch}{.55}
\hfil\begin{tikzpicture}
\node (tbl) {\input{c:/data/GUK/analysis/save/EstimationMemo/ProdAssetLivestockTimeVaryingANCOVAEstimationResults.tex}};
%\input{c:/dropbox/data/ramadan/save/tablecolortemplate.tex}
\end{tikzpicture}\\
\renewcommand{\arraystretch}{.8}
\setlength{\tabcolsep}{1pt}
\begin{tabular}{>{\hfill\scriptsize}p{1cm}<{}>{\hfill\scriptsize}p{.25cm}<{}>{\scriptsize}p{12cm}<{\hfill}}
Source:& \multicolumn{2}{l}{\scriptsize Estimated with GUK administrative and survey data.}\\
Notes: & 1. & ANCOVA estimates using administrative and survey data. Post treatment regressands are regressed on categorical variables, pre-treatment regressand and other covariates. \textsf{FloodInRd1} and \textsf{HeadLiterate0} are indicator variables for the presence of self reported damage by a flood at the baseline, and literacy of household head, respectively. \textsf{HHsize0} is household size at the baseline. We annotate the number of periods that a household is observed with \textsf{T}. The total number of households is shown for each values of \textsf{T}. \textsf{T=4} indicates the number of households with complete panel information, \textsf{T=3} indicates number of households observed three times, \textsf{T=2} indicates the number of households observed twice. \textsf{N} indicates total number of observations used in ANCOVA estimation, or \textsf{N$=$1$\times$(T=2)+2$\times$(T=3)+3$\times$(T=4)}.  \textsf{Large}, \textsf{LargeGrace}, \textsf{Cattle} are indicator variables of the \textsf{large}, \textsf{large grace}, and \textsf{cattle} arms, respectively. The default arm category is \textsf{traditional} arm. \textsf{rd2, rd3, rd4} are dummy variables for second, third, and fourth round of survey.  Productive assets do not include livestock. Regressions (1)-(3), (5)-(6) use only arm and calendar information. (4) and (7) use previous six month repayment and saving information which is lacking in rd 1, hence starts from rd 2.\\
& 2. & $P$ values in percentages in parenthesises. Standard errors are clustered at group (village) level.
%${}^{***}$, ${}^{**}$, ${}^{*}$ indicate statistical significance at 1\%, 5\%, 10\%, respetively. Standard errors are clustered at group (village) level.
\end{tabular}
\end{minipage}

\hspace{-1cm}\begin{minipage}[t]{14cm}
\hfil\textsc{\normalsize Table \refstepcounter{table}\thetable: ANCOVA estimation of livestock and productive assets by attributes and period\label{tab ANCOVA productive assets timevarying attributes}}\\
\setlength{\tabcolsep}{1pt}
\setlength{\baselineskip}{8pt}
\renewcommand{\arraystretch}{.55}
\hfil\begin{tikzpicture}
\node (tbl) {\input{c:/data/GUK/analysis/save/EstimationMemo/ProdAssetLivestockTimeVaryingAttributesANCOVAEstimationResults.tex}};
%\input{c:/dropbox/data/ramadan/save/tablecolortemplate.tex}
\end{tikzpicture}\\
\renewcommand{\arraystretch}{.8}
\setlength{\tabcolsep}{1pt}
\begin{tabular}{>{\hfill\scriptsize}p{1cm}<{}>{\hfill\scriptsize}p{.25cm}<{}>{\scriptsize}p{12cm}<{\hfill}}
Source:& \multicolumn{2}{l}{\scriptsize Estimated with GUK administrative and survey data.}\\
Notes: & 1. & ANCOVA estimates using administrative and survey data. Post treatment regressands are regressed on categorical variables, pre-treatment regressand and other covariates. \textsf{FloodInRd1} and \textsf{HeadLiterate0} are indicator variables for the presence of self reported damage by a flood at the baseline, and literacy of household head, respectively. \textsf{HHsize0} is household size at the baseline. We annotate the number of periods that a household is observed with \textsf{T}. The total number of households is shown for each values of \textsf{T}. \textsf{T=4} indicates the number of households with complete panel information, \textsf{T=3} indicates number of households observed three times, \textsf{T=2} indicates the number of households observed twice. \textsf{N} indicates total number of observations used in ANCOVA estimation, or \textsf{N$=$1$\times$(T=2)+2$\times$(T=3)+3$\times$(T=4)}.  \textsf{Upfront} is an indicator variable of the arm with an upfront large disbursement, \textsf{WithGrace} is an indicator variable of the arm with a grace period, \textsf{InKind} is an indicator variable of the arm which lends a heifer. \textsf{rd2, rd3, rd4} are dummy variables for second, third, and fourth round of survey. Productive assets do not include livestock. Regressions (1)-(3), (5)-(6) use only arm and calendar information. (4) and (7) use previous six month repayment and saving information which is lacking in rd 1, hence starts from rd 2.\\
& 2. & $P$ values in percentages in parenthesises. Standard errors are clustered at group (village) level.
%${}^{***}$, ${}^{**}$, ${}^{*}$ indicate statistical significance at 1\%, 5\%, 10\%, respetively. Standard errors are clustered at group (village) level.
\end{tabular}
\end{minipage}


\hspace{-1cm}\begin{minipage}[t]{14cm}
\hfil\textsc{\normalsize Table \refstepcounter{table}\thetable: ANCOVA estimation of livestock and productive assets by attributes, poverty status, and period\label{tab ANCOVA productive assets timevarying poverty status attributes}}\\
\setlength{\tabcolsep}{1pt}
\setlength{\baselineskip}{8pt}
\renewcommand{\arraystretch}{.55}
\hfil\begin{tikzpicture}
\node (tbl) {\input{c:/data/GUK/analysis/save/EstimationMemo/ProdAssetLivestockTimeVaryingPovertyStatusAttributesANCOVAEstimationResults.tex}};
%\input{c:/dropbox/data/ramadan/save/tablecolortemplate.tex}
\end{tikzpicture}\\
\renewcommand{\arraystretch}{.8}
\setlength{\tabcolsep}{1pt}
\begin{tabular}{>{\hfill\scriptsize}p{1cm}<{}>{\hfill\scriptsize}p{.25cm}<{}>{\scriptsize}p{12cm}<{\hfill}}
Source:& \multicolumn{2}{l}{\scriptsize Estimated with GUK administrative and survey data.}\\
Notes: & 1. & ANCOVA estimates using administrative and survey data. Post treatment regressands are regressed on categorical variables, pre-treatment regressand and other covariates. \textsf{FloodInRd1} and \textsf{HeadLiterate0} are indicator variables for the presence of self reported damage by a flood at the baseline, and literacy of household head, respectively. \textsf{HHsize0} is household size at the baseline. We annotate the number of periods that a household is observed with \textsf{T}. The total number of households is shown for each values of \textsf{T}. \textsf{T=4} indicates the number of households with complete panel information, \textsf{T=3} indicates number of households observed three times, \textsf{T=2} indicates the number of households observed twice. \textsf{N} indicates total number of observations used in ANCOVA estimation, or \textsf{N$=$1$\times$(T=2)+2$\times$(T=3)+3$\times$(T=4)}.  \textsf{UltraPoor} is an indicator variable if the household is classified as the ultra poor. \textsf{Upfront} is an indicator variable of the arm with an upfront large disbursement, \textsf{WithGrace} is an indicator variable of the arm with a grace period, \textsf{InKind} is an indicator variable of the arm which lends a heifer. \textsf{rd2, rd3, rd4} are dummy variables for second, third, and fourth round of survey. Productive assets do not include livestock. Regressions (1)-(3), (5)-(6) use only arm and calendar information. (4) and (7) use previous six month repayment and saving information which is lacking in rd 1, hence starts from rd 2.\\
& 2. & $P$ values in percentages in parenthesises. Standard errors are clustered at group (village) level.
%${}^{***}$, ${}^{**}$, ${}^{*}$ indicate statistical significance at 1\%, 5\%, 10\%, respetively. Standard errors are clustered at group (village) level.
\end{tabular}
\end{minipage}


\mpage{\linewidth}{
\hfil\textsc{\footnotesize Figure \refstepcounter{figure}\thefigure: Livestock and productive asset holding\label{fig livestock prod asset survey round}}\\
\hfil\includegraphics[width = 12cm]{c:/data/GUK/analysis/program/figure/EstimationMemo/ProdAssetLivestock.png}\\
\renewcommand{\arraystretch}{1}
\hfil\begin{tabular}{>{\hfill\scriptsize}p{1cm}<{}>{\scriptsize}p{12cm}<{\hfill}}
Source: & Survey data.\\
Note:& \\[1ex]
\end{tabular}
}


% \subsubsection{Assets+Livestock}

% \subsubsection{GUK broad net assets: Broad assets+Livestock-GUK Debt}

\subsubsection{Net broad assets: Broad assets+Livestock-GUK Debt-Other Debts}


Net broad assets = Broad assets + net saving - debt to GUK - debts to relatives and money lenders.

\begin{Schunk}
\begin{Soutput}


Number of obs by Arm and attrition
             AttritIn
Arm             2   3   4   9 Sum
  traditional   6   4  20 144 174
  large         5   2   1 192 200
  large grace  22   3   3 171 199
  cattle        5   5  13 177 200
  Sum          38  14  37 684 773


Number of obs by membership status and attrition
                      AttritIn
BStatus                  2   3   4   9 Sum
  borrower               8   6   8 578 600
  pure saver             0   0   0   0   0
  individual rejection   9   4   1  75  89
  group rejection        9   4   0  55  68
  rejection by flood    12   0  28   0  40
  Sum                   38  14  37 708 797
\end{Soutput}
\end{Schunk}














\mpage{\linewidth}{
\hfil\textsc{\footnotesize Figure \refstepcounter{figure}\thefigure: Total and net broad asset values\label{fig TotalAndNetAssetValues}}\\
\hfil\includegraphics[height = 6cm]{c:/data/GUK/analysis/program/figure/EstimationMemo/NetBroadAssets.pdf}\\
\renewcommand{\arraystretch}{1}
\hfil\begin{tabular}{>{\hfill\scriptsize}p{1cm}<{}>{\scriptsize}p{12cm}<{\hfill}}
Source: & Survey data.\\
Note:& Top panel shows total gross asset values. Bottom panel shows total net broad asset values = total gross broad asset values - debt outstanding. Debt outstanding takes the value of the month immediately after the respective survey round interview. \\[1ex]
\end{tabular}
}

\mpage{\linewidth}{
\hfil\textsc{\footnotesize Figure \refstepcounter{figure}\thefigure: Broad net asset values at round 1\label{fig NetAssetValuesAtRd1}}\\
\hfil\includegraphics[width = 10cm]{c:/data/GUK/analysis/program/figure/EstimationMemo/NetBroadAssetsAtRd1.pdf}\\
\renewcommand{\arraystretch}{1}
\hfil\begin{tabular}{>{\hfill\scriptsize}p{1cm}<{}>{\scriptsize}p{12cm}<{\hfill}}
Source: & Survey data.\\
Note:& Broad net asset values = total gross broad asset values - debt outstanding. Debt outstanding takes the value of the month immediately after the respective survey round interview. \\[1ex]
\end{tabular}
}

\mpage{\linewidth}{
\hfil\textsc{\footnotesize Figure \refstepcounter{figure}\thefigure: Total broad asset dynamics of nonborrowers\label{fig NonborrowerTotalAssetDynamics}}\\
%\hfil\includegraphics{  paste0(pathprogram, "figure/EstimationMemo/NonborrowerGrossBroadAssetsDynamicsByPovertyStatus.png")}\\
\renewcommand{\arraystretch}{1}
\hfil\begin{tabular}{>{\hfill\scriptsize}p{1cm}<{}>{\scriptsize}p{12cm}<{\hfill}}
Source: & Survey data.\\
Note:& Only for nonborrowers. Scatter plots contrast $t$ vs. $t+1$ comparison where $t$ and $t+1$ are given in strip ribbons of each panel. \\[1ex]
\end{tabular}
}




\hspace{-1cm}\begin{minipage}[t]{14cm}
\hfil\textsc{\normalsize Table \refstepcounter{table}\thetable: ANCOVA estimation of net broad assets\label{tab ANCOVA net assets}}\\
\setlength{\tabcolsep}{1pt}
\setlength{\baselineskip}{8pt}
\renewcommand{\arraystretch}{.55}
\hfil\begin{tikzpicture}
\node (tbl) {\input{c:/data/GUK/analysis/save/EstimationMemo/NetBroadAssetsANCOVAEstimationResults.tex}};
%\input{c:/dropbox/data/ramadan/save/tablecolortemplate.tex}
\end{tikzpicture}\\
\renewcommand{\arraystretch}{.8}
\setlength{\tabcolsep}{1pt}
\begin{tabular}{>{\hfill\scriptsize}p{1cm}<{}>{\hfill\scriptsize}p{.25cm}<{}>{\scriptsize}p{12cm}<{\hfill}}
Source:& \multicolumn{2}{l}{\scriptsize Estimated with GUK administrative and survey data.}\\
Notes: & 1. & ANCOVA estimates using administrative and survey data. Post treatment regressands are regressed on categorical variables, pre-treatment regressand and other covariates. \textsf{FloodInRd1} and \textsf{HeadLiterate0} are indicator variables for the presence of self reported damage by a flood at the baseline, and literacy of household head, respectively. \textsf{HHsize0} is household size at the baseline. We annotate the number of periods that a household is observed with \textsf{T}. The total number of households is shown for each values of \textsf{T}. \textsf{T=4} indicates the number of households with complete panel information, \textsf{T=3} indicates number of households observed three times, \textsf{T=2} indicates the number of households observed twice. \textsf{N} indicates total number of observations used in ANCOVA estimation, or \textsf{N$=$1$\times$(T=2)+2$\times$(T=3)+3$\times$(T=4)}.  \textsf{Large}, \textsf{LargeGrace}, \textsf{Cattle} are indicator variables of the \textsf{large}, \textsf{large grace}, and \textsf{cattle} arms, respectively. The default arm category is \textsf{traditional} arm. Household assets do not include livestock. Regressions (1)-(3), (5)-(6) use only arm and calendar information. (4) and (7) use previous six month repayment and saving information which is lacking in rd 1, hence starts from rd 2.\\
& 2. & $P$ values in percentages in parenthesises. Standard errors are clustered at group (village) level.
%${}^{***}$, ${}^{**}$, ${}^{*}$ indicate statistical significance at 1\%, 5\%, 10\%, respetively. Standard errors are clustered at group (village) level.
\end{tabular}
\end{minipage}

\hspace{-1cm}\begin{minipage}[t]{14cm}
\hfil\textsc{\normalsize Table \refstepcounter{table}\thetable: ANCOVA estimation of net broad assets by attributes\label{tab ANCOVA net broad assets attributes}}\\
\setlength{\tabcolsep}{1pt}
\setlength{\baselineskip}{8pt}
\renewcommand{\arraystretch}{.55}
\hfil\begin{tikzpicture}
\node (tbl) {\input{c:/data/GUK/analysis/save/EstimationMemo/NetBroadAssetsAttributesANCOVAEstimationResults.tex}};
%\input{c:/dropbox/data/ramadan/save/tablecolortemplate.tex}
\end{tikzpicture}\\
\renewcommand{\arraystretch}{.8}
\setlength{\tabcolsep}{1pt}
\begin{tabular}{>{\hfill\scriptsize}p{1cm}<{}>{\hfill\scriptsize}p{.25cm}<{}>{\scriptsize}p{12cm}<{\hfill}}
Source:& \multicolumn{2}{l}{\scriptsize Estimated with GUK administrative and survey data.}\\
Notes: & 1. & ANCOVA estimates using administrative and survey data. Post treatment regressands are regressed on categorical variables, pre-treatment regressand and other covariates. \textsf{FloodInRd1} and \textsf{HeadLiterate0} are indicator variables for the presence of self reported damage by a flood at the baseline, and literacy of household head, respectively. \textsf{HHsize0} is household size at the baseline. We annotate the number of periods that a household is observed with \textsf{T}. The total number of households is shown for each values of \textsf{T}. \textsf{T=4} indicates the number of households with complete panel information, \textsf{T=3} indicates number of households observed three times, \textsf{T=2} indicates the number of households observed twice. \textsf{N} indicates total number of observations used in ANCOVA estimation, or \textsf{N$=$1$\times$(T=2)+2$\times$(T=3)+3$\times$(T=4)}.  \textsf{Upfront} is an indicator variable of the arm with an upfront large disbursement, \textsf{WithGrace} is an indicator variable of the arm with a grace period, \textsf{InKind} is an indicator variable of the arm which lends a heifer. Household assets do not include livestock. Regressions (1)-(3), (5)-(6) use only arm and calendar information. (4) and (7) use previous six month repayment and saving information which is lacking in rd 1, hence starts from rd 2.\\
& 2. & $P$ values in percentages in parenthesises. Standard errors are clustered at group (village) level.
%${}^{***}$, ${}^{**}$, ${}^{*}$ indicate statistical significance at 1\%, 5\%, 10\%, respetively. Standard errors are clustered at group (village) level.
\end{tabular}
\end{minipage}

\hspace{-1cm}\begin{minipage}[t]{14cm}
\hfil\textsc{\normalsize Table \refstepcounter{table}\thetable: ANCOVA estimation of net broad assets by period\label{tab ANCOVA net broad assets timevarying}}\\
\setlength{\tabcolsep}{1pt}
\setlength{\baselineskip}{8pt}
\renewcommand{\arraystretch}{.55}
\hfil\begin{tikzpicture}
\node (tbl) {\input{c:/data/GUK/analysis/save/EstimationMemo/NetBroadAssetsTimeVaryingANCOVAEstimationResults.tex}};
%\input{c:/dropbox/data/ramadan/save/tablecolortemplate.tex}
\end{tikzpicture}\\
\renewcommand{\arraystretch}{.8}
\setlength{\tabcolsep}{1pt}
\begin{tabular}{>{\hfill\scriptsize}p{1cm}<{}>{\hfill\scriptsize}p{.25cm}<{}>{\scriptsize}p{12cm}<{\hfill}}
Source:& \multicolumn{2}{l}{\scriptsize Estimated with GUK administrative and survey data.}\\
Notes: & 1. & ANCOVA estimates using administrative and survey data. Post treatment regressands are regressed on categorical variables, pre-treatment regressand and other covariates. \textsf{FloodInRd1} and \textsf{HeadLiterate0} are indicator variables for the presence of self reported damage by a flood at the baseline, and literacy of household head, respectively. \textsf{HHsize0} is household size at the baseline. We annotate the number of periods that a household is observed with \textsf{T}. The total number of households is shown for each values of \textsf{T}. \textsf{T=4} indicates the number of households with complete panel information, \textsf{T=3} indicates number of households observed three times, \textsf{T=2} indicates the number of households observed twice. \textsf{N} indicates total number of observations used in ANCOVA estimation, or \textsf{N$=$1$\times$(T=2)+2$\times$(T=3)+3$\times$(T=4)}.  \textsf{Large}, \textsf{LargeGrace}, \textsf{Cattle} are indicator variables of the \textsf{large}, \textsf{large grace}, and \textsf{cattle} arms, respectively. The default arm category is \textsf{traditional} arm. \textsf{rd2, rd3, rd4} are dummy variables for second, third, and fourth round of survey.  Household assets do not include livestock. Regressions (1)-(3), (5)-(6) use only arm and calendar information. (4) and (7) use previous six month repayment and saving information which is lacking in rd 1, hence starts from rd 2.\\
& 2. & $P$ values in percentages in parenthesises. Standard errors are clustered at group (village) level.
%${}^{***}$, ${}^{**}$, ${}^{*}$ indicate statistical significance at 1\%, 5\%, 10\%, respetively. Standard errors are clustered at group (village) level.
\end{tabular}
\end{minipage}

\hspace{-1cm}\begin{minipage}[t]{14cm}
\hfil\textsc{\normalsize Table \refstepcounter{table}\thetable: ANCOVA estimation of net broad assets by attributes and period\label{tab ANCOVA net broad assets timevarying attributes}}\\
\setlength{\tabcolsep}{1pt}
\setlength{\baselineskip}{8pt}
\renewcommand{\arraystretch}{.55}
\hfil\begin{tikzpicture}
\node (tbl) {\input{c:/data/GUK/analysis/save/EstimationMemo/NetBroadAssetsTimeVaryingAttributesANCOVAEstimationResults.tex}};
%\input{c:/dropbox/data/ramadan/save/tablecolortemplate.tex}
\end{tikzpicture}\\
\renewcommand{\arraystretch}{.8}
\setlength{\tabcolsep}{1pt}
\begin{tabular}{>{\hfill\scriptsize}p{1cm}<{}>{\hfill\scriptsize}p{.25cm}<{}>{\scriptsize}p{12cm}<{\hfill}}
Source:& \multicolumn{2}{l}{\scriptsize Estimated with GUK administrative and survey data.}\\
Notes: & 1. & ANCOVA estimates using administrative and survey data. Post treatment regressands are regressed on categorical variables, pre-treatment regressand and other covariates. \textsf{FloodInRd1} and \textsf{HeadLiterate0} are indicator variables for the presence of self reported damage by a flood at the baseline, and literacy of household head, respectively. \textsf{HHsize0} is household size at the baseline. We annotate the number of periods that a household is observed with \textsf{T}. The total number of households is shown for each values of \textsf{T}. \textsf{T=4} indicates the number of households with complete panel information, \textsf{T=3} indicates number of households observed three times, \textsf{T=2} indicates the number of households observed twice. \textsf{N} indicates total number of observations used in ANCOVA estimation, or \textsf{N$=$1$\times$(T=2)+2$\times$(T=3)+3$\times$(T=4)}.  \textsf{Upfront} is an indicator variable of the arm with an upfront large disbursement, \textsf{WithGrace} is an indicator variable of the arm with a grace period, \textsf{InKind} is an indicator variable of the arm which lends a heifer. \textsf{rd2, rd3, rd4} are dummy variables for second, third, and fourth round of survey. Household assets do not include livestock. Regressions (1)-(3), (5)-(6) use only arm and calendar information. (4) and (7) use previous six month repayment and saving information which is lacking in rd 1, hence starts from rd 2.\\
& 2. & $P$ values in percentages in parenthesises. Standard errors are clustered at group (village) level.
%${}^{***}$, ${}^{**}$, ${}^{*}$ indicate statistical significance at 1\%, 5\%, 10\%, respetively. Standard errors are clustered at group (village) level.
\end{tabular}
\end{minipage}


\hspace{-1cm}\begin{minipage}[t]{14cm}
\hfil\textsc{\normalsize Table \refstepcounter{table}\thetable: ANCOVA estimation of net broad assets by arm, poverty status, and period\label{tab ANCOVA net broad assets timevarying poverty status}}\\
\setlength{\tabcolsep}{1pt}
\setlength{\baselineskip}{8pt}
\renewcommand{\arraystretch}{.55}
\hfil\begin{tikzpicture}
\node (tbl) {\input{c:/data/GUK/analysis/save/EstimationMemo/NetBroadAssetsTimeVaryingPovertyStatusANCOVAEstimationResults.tex}};
%\input{c:/dropbox/data/ramadan/save/tablecolortemplate.tex}
\end{tikzpicture}\\
\renewcommand{\arraystretch}{.8}
\setlength{\tabcolsep}{1pt}
\begin{tabular}{>{\hfill\scriptsize}p{1cm}<{}>{\hfill\scriptsize}p{.25cm}<{}>{\scriptsize}p{12cm}<{\hfill}}
Source:& \multicolumn{2}{l}{\scriptsize Estimated with GUK administrative and survey data.}\\
Notes: & 1. & ANCOVA estimates using administrative and survey data. Post treatment regressands are regressed on categorical variables, pre-treatment regressand and other covariates. \textsf{FloodInRd1} and \textsf{HeadLiterate0} are indicator variables for the presence of self reported damage by a flood at the baseline, and literacy of household head, respectively. \textsf{HHsize0} is household size at the baseline. We annotate the number of periods that a household is observed with \textsf{T}. The total number of households is shown for each values of \textsf{T}. \textsf{T=4} indicates the number of households with complete panel information, \textsf{T=3} indicates number of households observed three times, \textsf{T=2} indicates the number of households observed twice. \textsf{N} indicates total number of observations used in ANCOVA estimation, or \textsf{N$=$1$\times$(T=2)+2$\times$(T=3)+3$\times$(T=4)}.  \textsf{UltraPoor} is an indicator variable if the household is classified as the ultra poor. \textsf{Large}, \textsf{LargeGrace}, \textsf{Cattle} are indicator variables of the \textsf{large}, \textsf{large grace}, and \textsf{cattle} arms, respectively. The default arm category is \textsf{traditional} arm. \textsf{rd2, rd3, rd4} are dummy variables for second, third, and fourth round of survey. Household assets do not include livestock. Regressions (1)-(3), (5)-(6) use only arm and calendar information. (4) and (7) use previous six month repayment and saving information which is lacking in rd 1, hence starts from rd 2.\\
& 2. & $P$ values in percentages in parenthesises. Standard errors are clustered at group (village) level.
%${}^{***}$, ${}^{**}$, ${}^{*}$ indicate statistical significance at 1\%, 5\%, 10\%, respetively. Standard errors are clustered at group (village) level.
\end{tabular}
\end{minipage}

\hspace{-1cm}\begin{minipage}[t]{14cm}
\hfil\textsc{\normalsize Table \refstepcounter{table}\thetable: ANCOVA estimation of net broad assets by attributes, poverty status, and period\label{tab ANCOVA net broad assets timevarying poverty status attributes}}\\
\setlength{\tabcolsep}{1pt}
\setlength{\baselineskip}{8pt}
\renewcommand{\arraystretch}{.55}
\hfil\begin{tikzpicture}
\node (tbl) {\input{c:/data/GUK/analysis/save/EstimationMemo/NetBroadAssetsTimeVaryingPovertyStatusAttributesANCOVAEstimationResults.tex}};
%\input{c:/dropbox/data/ramadan/save/tablecolortemplate.tex}
\end{tikzpicture}\\
\renewcommand{\arraystretch}{.8}
\setlength{\tabcolsep}{1pt}
\begin{tabular}{>{\hfill\scriptsize}p{1cm}<{}>{\hfill\scriptsize}p{.25cm}<{}>{\scriptsize}p{12cm}<{\hfill}}
Source:& \multicolumn{2}{l}{\scriptsize Estimated with GUK administrative and survey data.}\\
Notes: & 1. & ANCOVA estimates using administrative and survey data. Post treatment regressands are regressed on categorical variables, pre-treatment regressand and other covariates. \textsf{FloodInRd1} and \textsf{HeadLiterate0} are indicator variables for the presence of self reported damage by a flood at the baseline, and literacy of household head, respectively. \textsf{HHsize0} is household size at the baseline. We annotate the number of periods that a household is observed with \textsf{T}. The total number of households is shown for each values of \textsf{T}. \textsf{T=4} indicates the number of households with complete panel information, \textsf{T=3} indicates number of households observed three times, \textsf{T=2} indicates the number of households observed twice. \textsf{N} indicates total number of observations used in ANCOVA estimation, or \textsf{N$=$1$\times$(T=2)+2$\times$(T=3)+3$\times$(T=4)}.  \textsf{UltraPoor} is an indicator variable if the household is classified as the ultra poor. \textsf{Upfront} is an indicator variable of the arm with an upfront large disbursement, \textsf{WithGrace} is an indicator variable of the arm with a grace period, \textsf{InKind} is an indicator variable of the arm which lends a heifer. \textsf{rd2, rd3, rd4} are dummy variables for second, third, and fourth round of survey. Household assets do not include livestock. Regressions (1)-(3), (5)-(6) use only arm and calendar information. (4) and (7) use previous six month repayment and saving information which is lacking in rd 1, hence starts from rd 2.\\
& 2. & $P$ values in percentages in parenthesises. Standard errors are clustered at group (village) level.
%${}^{***}$, ${}^{**}$, ${}^{*}$ indicate statistical significance at 1\%, 5\%, 10\%, respetively. Standard errors are clustered at group (village) level.
\end{tabular}
\end{minipage}



%\subsubsection{Narrow net assets: NarrowAssets+Livestock-GUK Debt-Other Debts}

\subsubsection{Net assets: Assets+Livestock-GUK Debt-Other Debts}

Net assets = Assets + net saving - debt to GUK - debts to relatives and money lenders. Assets use only items observed for all 4 rounds for household assets \textit{including} radios and cassette players (which have possibly large errors). 


\begin{Schunk}
\begin{Soutput}


Number of obs by Arm and attrition
             AttritIn
Arm             2   3   4   9 Sum
  traditional   6   4  20 144 174
  large         5   2   1 192 200
  large grace  22   3   3 171 199
  cattle        5   5  13 177 200
  Sum          38  14  37 684 773


Number of obs by membership status and attrition
                      AttritIn
BStatus                  2   3   4   9 Sum
  borrower               8   6   8 578 600
  pure saver             0   0   0   0   0
  individual rejection   9   4   1  75  89
  group rejection        9   4   0  55  68
  rejection by flood    12   0  28   0  40
  Sum                   38  14  37 708 797
\end{Soutput}
\end{Schunk}


\begin{Schunk}
\begin{Soutput}
      dummyLarge dummyLargeSize dummyLargeGrace dummyWithGrace dummyCattle
           <num>          <num>           <num>          <num>       <num>
   1:          1              1               0              0           0
   2:          1              1               0              0           0
   3:          1              1               0              0           0
   4:          1              1               0              0           0
   5:          1              1               0              0           0
  ---                                                                     
4243:          0              1               0              1           1
4244:          0              1               0              1           1
4245:          0              1               0              1           1
4246:          0              1               0              1           1
4247:          0              1               0              1           1
      dummyInKind
            <num>
   1:           0
   2:           0
   3:           0
   4:           0
   5:           0
  ---            
4243:           1
4244:           1
4245:           1
4246:           1
4247:           1
\end{Soutput}
\begin{Soutput}
[1] TRUE
\end{Soutput}
\begin{Soutput}
[1] TRUE
\end{Soutput}
\begin{Soutput}
[1] TRUE
\end{Soutput}
\end{Schunk}









\mpage{\linewidth}{
\hfil\textsc{\footnotesize Figure \refstepcounter{figure}\thefigure: Total and net asset values\label{fig TotalAndNetAssetValues}}\\
\hfil\includegraphics[height = 6cm]{c:/data/GUK/analysis/program/figure/EstimationMemo/NetAssets.pdf}\\
\renewcommand{\arraystretch}{1}
\hfil\begin{tabular}{>{\hfill\scriptsize}p{1cm}<{}>{\scriptsize}p{12cm}<{\hfill}}
Source: & Survey data.\\
Note:& Top panel shows total gross asset values. Bottom panel shows total net asset values = total gross asset values - debt outstanding. Debt outstanding takes the value of the month immediately after the respective survey round interview. Net assets uses only assets observed for all 4 rounds in household assets.\\[1ex]
\end{tabular}
}

\mpage{\linewidth}{
\hfil\textsc{\footnotesize Figure \refstepcounter{figure}\thefigure: Net asset values at round 1\label{fig NetAssetValuesAtRd1}}\\
\hfil\includegraphics[width = 10cm]{c:/data/GUK/analysis/program/figure/EstimationMemo/NetAssetsAtRd1.pdf}\\
\renewcommand{\arraystretch}{1}
\hfil\begin{tabular}{>{\hfill\scriptsize}p{1cm}<{}>{\scriptsize}p{12cm}<{\hfill}}
Source: & Survey data.\\
Note:& Net asset values = total gross asset values - debt outstanding. Debt outstanding takes the value of the month immediately after the respective survey round interview. Net assets uses only assets observed for all 4 rounds in household assets.\\[1ex]
\end{tabular}
}



\hspace{-1cm}\begin{minipage}[t]{14cm}
\hfil\textsc{\normalsize Table \refstepcounter{table}\thetable: ANCOVA estimation of net assets\label{tab ANCOVA net assets}}\\
\setlength{\tabcolsep}{1pt}
\setlength{\baselineskip}{8pt}
\renewcommand{\arraystretch}{.55}
\hfil\begin{tikzpicture}
\node (tbl) {\input{c:/data/GUK/analysis/save/EstimationMemo/NetAssetsANCOVAEstimationResults.tex}};
%\input{c:/dropbox/data/ramadan/save/tablecolortemplate.tex}
\end{tikzpicture}\\
\renewcommand{\arraystretch}{.8}
\setlength{\tabcolsep}{1pt}
\begin{tabular}{>{\hfill\scriptsize}p{1cm}<{}>{\hfill\scriptsize}p{.25cm}<{}>{\scriptsize}p{12cm}<{\hfill}}
Source:& \multicolumn{2}{l}{\scriptsize Estimated with GUK administrative and survey data.}\\
Notes: & 1. & ANCOVA estimates using administrative and survey data. Post treatment regressands are regressed on categorical variables, pre-treatment regressand and other covariates. \textsf{FloodInRd1} and \textsf{HeadLiterate0} are indicator variables for the presence of self reported damage by a flood at the baseline, and literacy of household head, respectively. \textsf{HHsize0} is household size at the baseline. We annotate the number of periods that a household is observed with \textsf{T}. The total number of households is shown for each values of \textsf{T}. \textsf{T=4} indicates the number of households with complete panel information, \textsf{T=3} indicates number of households observed three times, \textsf{T=2} indicates the number of households observed twice. \textsf{N} indicates total number of observations used in ANCOVA estimation, or \textsf{N$=$1$\times$(T=2)+2$\times$(T=3)+3$\times$(T=4)}.  \textsf{Large}, \textsf{LargeGrace}, \textsf{Cattle} are indicator variables of the \textsf{large}, \textsf{large grace}, and \textsf{cattle} arms, respectively. The default arm category is \textsf{traditional} arm. Net assets use only assets observed for all 4 rounds in household assets. Household assets do not include livestock. Regressions (1)-(3), (5)-(6) use only arm and calendar information. (4) and (7) use previous six month repayment and saving information which is lacking in rd 1, hence starts from rd 2.\\
& 2. & $P$ values in percentages in parenthesises. Standard errors are clustered at group (village) level.
%${}^{***}$, ${}^{**}$, ${}^{*}$ indicate statistical significance at 1\%, 5\%, 10\%, respetively. Standard errors are clustered at group (village) level.
\end{tabular}
\end{minipage}

\hspace{-1cm}\begin{minipage}[t]{14cm}
\hfil\textsc{\normalsize Table \refstepcounter{table}\thetable: ANCOVA estimation of net assets by attributes\label{tab ANCOVA narrow net assets attributes}}\\
\setlength{\tabcolsep}{1pt}
\setlength{\baselineskip}{8pt}
\renewcommand{\arraystretch}{.55}
\hfil\begin{tikzpicture}
\node (tbl) {\input{c:/data/GUK/analysis/save/EstimationMemo/NetAssetsAttributesANCOVAEstimationResults.tex}};
%\input{c:/dropbox/data/ramadan/save/tablecolortemplate.tex}
\end{tikzpicture}\\
\renewcommand{\arraystretch}{.8}
\setlength{\tabcolsep}{1pt}
\begin{tabular}{>{\hfill\scriptsize}p{1cm}<{}>{\hfill\scriptsize}p{.25cm}<{}>{\scriptsize}p{12cm}<{\hfill}}
Source:& \multicolumn{2}{l}{\scriptsize Estimated with GUK administrative and survey data.}\\
Notes: & 1. & ANCOVA estimates using administrative and survey data. Post treatment regressands are regressed on categorical variables, pre-treatment regressand and other covariates. \textsf{FloodInRd1} and \textsf{HeadLiterate0} are indicator variables for the presence of self reported damage by a flood at the baseline, and literacy of household head, respectively. \textsf{HHsize0} is household size at the baseline. We annotate the number of periods that a household is observed with \textsf{T}. The total number of households is shown for each values of \textsf{T}. \textsf{T=4} indicates the number of households with complete panel information, \textsf{T=3} indicates number of households observed three times, \textsf{T=2} indicates the number of households observed twice. \textsf{N} indicates total number of observations used in ANCOVA estimation, or \textsf{N$=$1$\times$(T=2)+2$\times$(T=3)+3$\times$(T=4)}.  \textsf{Upfront} is an indicator variable of the arm with an upfront large disbursement, \textsf{WithGrace} is an indicator variable of the arm with a grace period, \textsf{InKind} is an indicator variable of the arm which lends a heifer. Net assets use only assets observed for all 4 rounds in household assets. Household assets do not include livestock. Regressions (1)-(3), (5)-(6) use only arm and calendar information. (4) and (7) use previous six month repayment and saving information which is lacking in rd 1, hence starts from rd 2.\\
& 2. & $P$ values in percentages in parenthesises. Standard errors are clustered at group (village) level.
%${}^{***}$, ${}^{**}$, ${}^{*}$ indicate statistical significance at 1\%, 5\%, 10\%, respetively. Standard errors are clustered at group (village) level.
\end{tabular}
\end{minipage}

\hspace{-1cm}\begin{minipage}[t]{14cm}
\hfil\textsc{\normalsize Table \refstepcounter{table}\thetable: ANCOVA estimation of net assets by period\label{tab ANCOVA narrow net assets timevarying}}\\
\setlength{\tabcolsep}{1pt}
\setlength{\baselineskip}{8pt}
\renewcommand{\arraystretch}{.55}
\hfil\begin{tikzpicture}
\node (tbl) {\input{c:/data/GUK/analysis/save/EstimationMemo/NetAssetsTimeVaryingANCOVAEstimationResults.tex}};
%\input{c:/dropbox/data/ramadan/save/tablecolortemplate.tex}
\end{tikzpicture}\\
\renewcommand{\arraystretch}{.8}
\setlength{\tabcolsep}{1pt}
\begin{tabular}{>{\hfill\scriptsize}p{1cm}<{}>{\hfill\scriptsize}p{.25cm}<{}>{\scriptsize}p{12cm}<{\hfill}}
Source:& \multicolumn{2}{l}{\scriptsize Estimated with GUK administrative and survey data.}\\
Notes: & 1. & ANCOVA estimates using administrative and survey data. Post treatment regressands are regressed on categorical variables, pre-treatment regressand and other covariates. \textsf{FloodInRd1} and \textsf{HeadLiterate0} are indicator variables for the presence of self reported damage by a flood at the baseline, and literacy of household head, respectively. \textsf{HHsize0} is household size at the baseline. We annotate the number of periods that a household is observed with \textsf{T}. The total number of households is shown for each values of \textsf{T}. \textsf{T=4} indicates the number of households with complete panel information, \textsf{T=3} indicates number of households observed three times, \textsf{T=2} indicates the number of households observed twice. \textsf{N} indicates total number of observations used in ANCOVA estimation, or \textsf{N$=$1$\times$(T=2)+2$\times$(T=3)+3$\times$(T=4)}.  \textsf{Large}, \textsf{LargeGrace}, \textsf{Cattle} are indicator variables of the \textsf{large}, \textsf{large grace}, and \textsf{cattle} arms, respectively. The default arm category is \textsf{traditional} arm. \textsf{rd2, rd3, rd4} are dummy variables for second, third, and fourth round of survey. Net assets use only assets observed for all 4 rounds in household assets. Household assets do not include livestock. Regressions (1)-(3), (5)-(6) use only arm and calendar information. (4) and (7) use previous six month repayment and saving information which is lacking in rd 1, hence starts from rd 2.\\
& 2. & $P$ values in percentages in parenthesises. Standard errors are clustered at group (village) level.
%${}^{***}$, ${}^{**}$, ${}^{*}$ indicate statistical significance at 1\%, 5\%, 10\%, respetively. Standard errors are clustered at group (village) level.
\end{tabular}
\end{minipage}

\hspace{-1cm}\begin{minipage}[t]{14cm}
\hfil\textsc{\normalsize Table \refstepcounter{table}\thetable: ANCOVA estimation of net assets by attributes and period\label{tab ANCOVA narrow net assets timevarying attributes}}\\
\setlength{\tabcolsep}{1pt}
\setlength{\baselineskip}{8pt}
\renewcommand{\arraystretch}{.55}
\hfil\begin{tikzpicture}
\node (tbl) {\input{c:/data/GUK/analysis/save/EstimationMemo/NetAssetsTimeVaryingAttributesANCOVAEstimationResults.tex}};
%\input{c:/dropbox/data/ramadan/save/tablecolortemplate.tex}
\end{tikzpicture}\\
\renewcommand{\arraystretch}{.8}
\setlength{\tabcolsep}{1pt}
\begin{tabular}{>{\hfill\scriptsize}p{1cm}<{}>{\hfill\scriptsize}p{.25cm}<{}>{\scriptsize}p{12cm}<{\hfill}}
Source:& \multicolumn{2}{l}{\scriptsize Estimated with GUK administrative and survey data.}\\
Notes: & 1. & ANCOVA estimates using administrative and survey data. Post treatment regressands are regressed on categorical variables, pre-treatment regressand and other covariates. \textsf{FloodInRd1} and \textsf{HeadLiterate0} are indicator variables for the presence of self reported damage by a flood at the baseline, and literacy of household head, respectively. \textsf{HHsize0} is household size at the baseline. We annotate the number of periods that a household is observed with \textsf{T}. The total number of households is shown for each values of \textsf{T}. \textsf{T=4} indicates the number of households with complete panel information, \textsf{T=3} indicates number of households observed three times, \textsf{T=2} indicates the number of households observed twice. \textsf{N} indicates total number of observations used in ANCOVA estimation, or \textsf{N$=$1$\times$(T=2)+2$\times$(T=3)+3$\times$(T=4)}.  \textsf{Upfront} is an indicator variable of the arm with an upfront large disbursement, \textsf{WithGrace} is an indicator variable of the arm with a grace period, \textsf{InKind} is an indicator variable of the arm which lends a heifer. \textsf{rd2, rd3, rd4} are dummy variables for second, third, and fourth round of survey. Net assets use only assets observed for all 4 rounds in household assets. Household assets do not include livestock. Regressions (1)-(3), (5)-(6) use only arm and calendar information. (4) and (7) use previous six month repayment and saving information which is lacking in rd 1, hence starts from rd 2.\\
& 2. & $P$ values in percentages in parenthesises. Standard errors are clustered at group (village) level.
%${}^{***}$, ${}^{**}$, ${}^{*}$ indicate statistical significance at 1\%, 5\%, 10\%, respetively. Standard errors are clustered at group (village) level.
\end{tabular}
\end{minipage}


\begin{figure}
\renewcommand{\arraystretch}{.6}
\mpage{\linewidth}{
\hfil\textsc{\footnotesize Figure \refstepcounter{figure}\thefigure: Impacts on net assets and various other measures of assets\label{fig NetAssetsNetBroadAssetsNEAssetsCattleEffects}}\\
\hfil\includegraphics[width = 14cm]{c:/data/GUK/analysis/program/figure/EstimationMemo/NetAssetsNetBroadAssetsNEAssetsCattleEffects.pdf}\\
\renewcommand{\arraystretch}{.6}
\hfil\begin{tabular}{>{\hfill\scriptsize}p{1cm}<{}>{\scriptsize}p{12cm}<{\hfill}}
Source: & Estimated with survey data.\\
Note:&  See the footnote of \textsc{\footnotesize Figure \ref{fig AssetRelativeToCumulativeConcurrentTradEffectsByExperience}}. \\[1ex]
\end{tabular}
}
\end{figure}



\hspace{-1cm}\begin{minipage}[t]{14cm}
\hfil\textsc{\normalsize Table \refstepcounter{table}\thetable: ANCOVA estimation of broad net assets using annual prices \label{tab ANCOVA net assets}}\\
\setlength{\tabcolsep}{1pt}
\setlength{\baselineskip}{8pt}
\renewcommand{\arraystretch}{.55}
\hfil\begin{tikzpicture}
\node (tbl) {\input{c:/data/GUK/analysis/save/EstimationMemo/NetAssetsAnnualPricesANCOVAEstimationResults.tex}};
%\input{c:/dropbox/data/ramadan/save/tablecolortemplate.tex}
\end{tikzpicture}\\
\renewcommand{\arraystretch}{.8}
\setlength{\tabcolsep}{1pt}
\begin{tabular}{>{\hfill\scriptsize}p{1cm}<{}>{\hfill\scriptsize}p{.25cm}<{}>{\scriptsize}p{12cm}<{\hfill}}
Source:& \multicolumn{2}{l}{\scriptsize Estimated with GUK administrative and survey data.}\\
Notes: & 1. & ANCOVA estimates using administrative and survey data. Post treatment regressands are regressed on categorical variables, pre-treatment regressand and other covariates. \textsf{FloodInRd1} and \textsf{HeadLiterate0} are indicator variables for the presence of self reported damage by a flood at the baseline, and literacy of household head, respectively. \textsf{HHsize0} is household size at the baseline. We annotate the number of periods that a household is observed with \textsf{T}. The total number of households is shown for each values of \textsf{T}. \textsf{T=4} indicates the number of households with complete panel information, \textsf{T=3} indicates number of households observed three times, \textsf{T=2} indicates the number of households observed twice. \textsf{N} indicates total number of observations used in ANCOVA estimation, or \textsf{N$=$1$\times$(T=2)+2$\times$(T=3)+3$\times$(T=4)}.  \textsf{Large}, \textsf{LargeGrace}, \textsf{Cattle} are indicator variables of the \textsf{large}, \textsf{large grace}, and \textsf{cattle} arms, respectively. The default arm category is \textsf{traditional} arm. Sample is continuing members and replacing members of early rejecters and received loans prior to 2015 Janunary. Household assets do not include livestock. Regressions (1)-(3), (5)-(6) use only arm and calendar information. (4) and (7) use previous six month repayment and saving information which is lacking in rd 1, hence starts from rd 2.\\
& 2. & $P$ values in percentages in parenthesises. Standard errors are clustered at group (village) level.
%${}^{***}$, ${}^{**}$, ${}^{*}$ indicate statistical significance at 1\%, 5\%, 10\%, respetively. Standard errors are clustered at group (village) level.
\end{tabular}
\end{minipage}

\hspace{-1cm}\begin{minipage}[t]{14cm}
\hfil\textsc{\normalsize Table \refstepcounter{table}\thetable: ANCOVA estimation of broad net assets using annual prices by attributes\label{tab ANCOVA net assets using annual prices attributes}}\\
\setlength{\tabcolsep}{1pt}
\setlength{\baselineskip}{8pt}
\renewcommand{\arraystretch}{.55}
\hfil\begin{tikzpicture}
\node (tbl) {\input{c:/data/GUK/analysis/save/EstimationMemo/NetAssetsAnnualPricesAttributesANCOVAEstimationResults.tex}};
%\input{c:/dropbox/data/ramadan/save/tablecolortemplate.tex}
\end{tikzpicture}\\
\renewcommand{\arraystretch}{.8}
\setlength{\tabcolsep}{1pt}
\begin{tabular}{>{\hfill\scriptsize}p{1cm}<{}>{\hfill\scriptsize}p{.25cm}<{}>{\scriptsize}p{12cm}<{\hfill}}
Source:& \multicolumn{2}{l}{\scriptsize Estimated with GUK administrative and survey data.}\\
Notes: & 1. & ANCOVA estimates using administrative and survey data. Post treatment regressands are regressed on categorical variables, pre-treatment regressand and other covariates. \textsf{FloodInRd1} and \textsf{HeadLiterate0} are indicator variables for the presence of self reported damage by a flood at the baseline, and literacy of household head, respectively. \textsf{HHsize0} is household size at the baseline. We annotate the number of periods that a household is observed with \textsf{T}. The total number of households is shown for each values of \textsf{T}. \textsf{T=4} indicates the number of households with complete panel information, \textsf{T=3} indicates number of households observed three times, \textsf{T=2} indicates the number of households observed twice. \textsf{N} indicates total number of observations used in ANCOVA estimation, or \textsf{N$=$1$\times$(T=2)+2$\times$(T=3)+3$\times$(T=4)}.  \textsf{Upfront} is an indicator variable of the arm with an upfront large disbursement, \textsf{WithGrace} is an indicator variable of the arm with a grace period, \textsf{InKind} is an indicator variable of the arm which lends a heifer. Sample is continuing members and replacing members of early rejecters and received loans prior to 2015 Janunary. Household assets do not include livestock. Regressions (1)-(3), (5)-(6) use only arm and calendar information. (4) and (7) use previous six month repayment and saving information which is lacking in rd 1, hence starts from rd 2.\\
& 2. & $P$ values in percentages in parenthesises. Standard errors are clustered at group (village) level.
%${}^{***}$, ${}^{**}$, ${}^{*}$ indicate statistical significance at 1\%, 5\%, 10\%, respetively. Standard errors are clustered at group (village) level.
\end{tabular}
\end{minipage}

\hspace{-1cm}\begin{minipage}[t]{14cm}
\hfil\textsc{\normalsize Table \refstepcounter{table}\thetable: ANCOVA estimation of broad net assets using annual prices by period\label{tab ANCOVA net assets using annual prices timevarying}}\\
\setlength{\tabcolsep}{1pt}
\setlength{\baselineskip}{8pt}
\renewcommand{\arraystretch}{.55}
\hfil\begin{tikzpicture}
\node (tbl) {\input{c:/data/GUK/analysis/save/EstimationMemo/NetAssetsAnnualPricesTimeVaryingANCOVAEstimationResults.tex}};
%\input{c:/dropbox/data/ramadan/save/tablecolortemplate.tex}
\end{tikzpicture}\\
\renewcommand{\arraystretch}{.8}
\setlength{\tabcolsep}{1pt}
\begin{tabular}{>{\hfill\scriptsize}p{1cm}<{}>{\hfill\scriptsize}p{.25cm}<{}>{\scriptsize}p{12cm}<{\hfill}}
Source:& \multicolumn{2}{l}{\scriptsize Estimated with GUK administrative and survey data.}\\
Notes: & 1. & ANCOVA estimates using administrative and survey data. Post treatment regressands are regressed on categorical variables, pre-treatment regressand and other covariates. \textsf{FloodInRd1} and \textsf{HeadLiterate0} are indicator variables for the presence of self reported damage by a flood at the baseline, and literacy of household head, respectively. \textsf{HHsize0} is household size at the baseline. We annotate the number of periods that a household is observed with \textsf{T}. The total number of households is shown for each values of \textsf{T}. \textsf{T=4} indicates the number of households with complete panel information, \textsf{T=3} indicates number of households observed three times, \textsf{T=2} indicates the number of households observed twice. \textsf{N} indicates total number of observations used in ANCOVA estimation, or \textsf{N$=$1$\times$(T=2)+2$\times$(T=3)+3$\times$(T=4)}.  \textsf{Large}, \textsf{LargeGrace}, \textsf{Cattle} are indicator variables of the \textsf{large}, \textsf{large grace}, and \textsf{cattle} arms, respectively. The default arm category is \textsf{traditional} arm. Sample is continuing members and replacing members of early rejecters and received loans prior to 2015 Janunary. Household assets do not include livestock. Regressions (1)-(3), (5)-(6) use only arm and calendar information. (4) and (7) use previous six month repayment and saving information which is lacking in rd 1, hence starts from rd 2.\\
& 2. & $P$ values in percentages in parenthesises. Standard errors are clustered at group (village) level.
%${}^{***}$, ${}^{**}$, ${}^{*}$ indicate statistical significance at 1\%, 5\%, 10\%, respetively. Standard errors are clustered at group (village) level.
\end{tabular}
\end{minipage}

\hspace{-1cm}\begin{minipage}[t]{14cm}
\hfil\textsc{\normalsize Table \refstepcounter{table}\thetable: ANCOVA estimation of broad net assets using annual prices by attributes and period\label{tab ANCOVA net assets using annual prices timevarying attributes}}\\
\setlength{\tabcolsep}{1pt}
\setlength{\baselineskip}{8pt}
\renewcommand{\arraystretch}{.55}
\hfil\begin{tikzpicture}
\node (tbl) {\input{c:/data/GUK/analysis/save/EstimationMemo/NetAssetsAnnualPricesTimeVaryingAttributesANCOVAEstimationResults.tex}};
%\input{c:/dropbox/data/ramadan/save/tablecolortemplate.tex}
\end{tikzpicture}\\
\renewcommand{\arraystretch}{.8}
\setlength{\tabcolsep}{1pt}
\begin{tabular}{>{\hfill\scriptsize}p{1cm}<{}>{\hfill\scriptsize}p{.25cm}<{}>{\scriptsize}p{12cm}<{\hfill}}
Source:& \multicolumn{2}{l}{\scriptsize Estimated with GUK administrative and survey data.}\\
Notes: & 1. & ANCOVA estimates using administrative and survey data. Post treatment regressands are regressed on categorical variables, pre-treatment regressand and other covariates. \textsf{FloodInRd1} and \textsf{HeadLiterate0} are indicator variables for the presence of self reported damage by a flood at the baseline, and literacy of household head, respectively. \textsf{HHsize0} is household size at the baseline. We annotate the number of periods that a household is observed with \textsf{T}. The total number of households is shown for each values of \textsf{T}. \textsf{T=4} indicates the number of households with complete panel information, \textsf{T=3} indicates number of households observed three times, \textsf{T=2} indicates the number of households observed twice. \textsf{N} indicates total number of observations used in ANCOVA estimation, or \textsf{N$=$1$\times$(T=2)+2$\times$(T=3)+3$\times$(T=4)}.  \textsf{Upfront} is an indicator variable of the arm with an upfront large disbursement, \textsf{WithGrace} is an indicator variable of the arm with a grace period, \textsf{InKind} is an indicator variable of the arm which lends a heifer. Sample is continuing members and replacing members of early rejecters and received loans prior to 2015 Janunary. Household assets do not include livestock. Regressions (1)-(3), (5)-(6) use only arm and calendar information. (4) and (7) use previous six month repayment and saving information which is lacking in rd 1, hence starts from rd 2.\\
& 2. & $P$ values in percentages in parenthesises. Standard errors are clustered at group (village) level.
%${}^{***}$, ${}^{**}$, ${}^{*}$ indicate statistical significance at 1\%, 5\%, 10\%, respetively. Standard errors are clustered at group (village) level.
\end{tabular}
\end{minipage}


\hspace{-1cm}\begin{minipage}[t]{14cm}
\hfil\textsc{\normalsize Table \refstepcounter{table}\thetable: ANCOVA estimation of broad net assets using annual prices by arm, poverty status, and period\label{tab ANCOVA net assets using annual prices timevarying poverty status}}\\
\setlength{\tabcolsep}{1pt}
\setlength{\baselineskip}{8pt}
\renewcommand{\arraystretch}{.55}
\hfil\begin{tikzpicture}
\node (tbl) {\input{c:/data/GUK/analysis/save/EstimationMemo/NetAssetsAnnualPricesTimeVaryingPovertyStatusANCOVAEstimationResults.tex}};
%\input{c:/dropbox/data/ramadan/save/tablecolortemplate.tex}
\end{tikzpicture}\\
\renewcommand{\arraystretch}{.8}
\setlength{\tabcolsep}{1pt}
\begin{tabular}{>{\hfill\scriptsize}p{1cm}<{}>{\hfill\scriptsize}p{.25cm}<{}>{\scriptsize}p{12cm}<{\hfill}}
Source:& \multicolumn{2}{l}{\scriptsize Estimated with GUK administrative and survey data.}\\
Notes: & 1. & ANCOVA estimates using administrative and survey data. Post treatment regressands are regressed on categorical variables, pre-treatment regressand and other covariates. \textsf{FloodInRd1} and \textsf{HeadLiterate0} are indicator variables for the presence of self reported damage by a flood at the baseline, and literacy of household head, respectively. \textsf{HHsize0} is household size at the baseline. We annotate the number of periods that a household is observed with \textsf{T}. The total number of households is shown for each values of \textsf{T}. \textsf{T=4} indicates the number of households with complete panel information, \textsf{T=3} indicates number of households observed three times, \textsf{T=2} indicates the number of households observed twice. \textsf{N} indicates total number of observations used in ANCOVA estimation, or \textsf{N$=$1$\times$(T=2)+2$\times$(T=3)+3$\times$(T=4)}.  \textsf{UltraPoor} is an indicator variable if the household is classified as the ultra poor. \textsf{Large}, \textsf{LargeGrace}, \textsf{Cattle} are indicator variables of the \textsf{large}, \textsf{large grace}, and \textsf{cattle} arms, respectively. The default arm category is \textsf{traditional} arm. Sample is continuing members and replacing members of early rejecters and received loans prior to 2015 Janunary. Household assets do not include livestock. Regressions (1)-(3), (5)-(6) use only arm and calendar information. (4) and (7) use previous six month repayment and saving information which is lacking in rd 1, hence starts from rd 2.\\
& 2. & $P$ values in percentages in parenthesises. Standard errors are clustered at group (village) level.
%${}^{***}$, ${}^{**}$, ${}^{*}$ indicate statistical significance at 1\%, 5\%, 10\%, respetively. Standard errors are clustered at group (village) level.
\end{tabular}
\end{minipage}

\hspace{-1cm}\begin{minipage}[t]{14cm}
\hfil\textsc{\normalsize Table \refstepcounter{table}\thetable: ANCOVA estimation of broad net assets using annual prices by attributes, poverty status, and period\label{tab ANCOVA net assets using annual prices timevarying poverty status attributes}}\\
\setlength{\tabcolsep}{1pt}
\setlength{\baselineskip}{8pt}
\renewcommand{\arraystretch}{.55}
\hfil\begin{tikzpicture}
\node (tbl) {\input{c:/data/GUK/analysis/save/EstimationMemo/NetAssetsAnnualPricesTimeVaryingPovertyStatusAttributesANCOVAEstimationResults.tex}};
%\input{c:/dropbox/data/ramadan/save/tablecolortemplate.tex}
\end{tikzpicture}\\
\renewcommand{\arraystretch}{.8}
\setlength{\tabcolsep}{1pt}
\begin{tabular}{>{\hfill\scriptsize}p{1cm}<{}>{\hfill\scriptsize}p{.25cm}<{}>{\scriptsize}p{12cm}<{\hfill}}
Source:& \multicolumn{2}{l}{\scriptsize Estimated with GUK administrative and survey data.}\\
Notes: & 1. & ANCOVA estimates using administrative and survey data. Post treatment regressands are regressed on categorical variables, pre-treatment regressand and other covariates. \textsf{FloodInRd1} and \textsf{HeadLiterate0} are indicator variables for the presence of self reported damage by a flood at the baseline, and literacy of household head, respectively. \textsf{HHsize0} is household size at the baseline. We annotate the number of periods that a household is observed with \textsf{T}. The total number of households is shown for each values of \textsf{T}. \textsf{T=4} indicates the number of households with complete panel information, \textsf{T=3} indicates number of households observed three times, \textsf{T=2} indicates the number of households observed twice. \textsf{N} indicates total number of observations used in ANCOVA estimation, or \textsf{N$=$1$\times$(T=2)+2$\times$(T=3)+3$\times$(T=4)}.  \textsf{UltraPoor} is an indicator variable if the household is classified as the ultra poor. \textsf{Upfront} is an indicator variable of the arm with an upfront large disbursement, \textsf{WithGrace} is an indicator variable of the arm with a grace period, \textsf{InKind} is an indicator variable of the arm which lends a heifer. Sample is continuing members and replacing members of early rejecters and received loans prior to 2015 Janunary. Household assets do not include livestock. Regressions (1)-(3), (5)-(6) use only arm and calendar information. (4) and (7) use previous six month repayment and saving information which is lacking in rd 1, hence starts from rd 2.\\
& 2. & $P$ values in percentages in parenthesises. Standard errors are clustered at group (village) level.
%${}^{***}$, ${}^{**}$, ${}^{*}$ indicate statistical significance at 1\%, 5\%, 10\%, respetively. Standard errors are clustered at group (village) level.
\end{tabular}
\end{minipage}

%\subsubsection{Revised net assets: Winsorised Assets+Livestock-GUK Debt-Other Debts}

%\subsubsection{Broad net non-livestock assets: Broad non-livestock assets-GUK Debt-Other Debts}

\subsubsection{Net non-livestock assets: Non-livestock assets-GUK Debt-Other Debts}

Net non-livestock assets = Non livestock assets + net saving - debt to GUK - debts to relatives and money lenders. 


\begin{Schunk}
\begin{Soutput}


Number of obs by Arm and attrition
             AttritIn
Arm             2   3   4   9 Sum
  traditional   6   4  20 144 174
  large         5   2   1 192 200
  large grace  22   3   3 171 199
  cattle        5   5  13 177 200
  Sum          38  14  37 684 773


Number of obs by membership status and attrition
                      AttritIn
BStatus                  2   3   4   9 Sum
  borrower               8   6   8 578 600
  pure saver             0   0   0   0   0
  individual rejection   9   4   1  75  89
  group rejection        9   4   0  55  68
  rejection by flood    12   0  28   0  40
  Sum                   38  14  37 708 797
\end{Soutput}
\end{Schunk}










\hspace{-1cm}\begin{minipage}[t]{14cm}
\hfil\textsc{\normalsize Table \refstepcounter{table}\thetable: ANCOVA estimation of net non-livestock assets\label{tab ANCOVA NetNLAssets}}\\
\setlength{\tabcolsep}{1pt}
\setlength{\baselineskip}{8pt}
\renewcommand{\arraystretch}{.55}
\hfil\begin{tikzpicture}
\node (tbl) {\input{c:/data/GUK/analysis/save/EstimationMemo/NetNLAssetsANCOVAEstimationResults.tex}};
%\input{c:/dropbox/data/ramadan/save/tablecolortemplate.tex}
\end{tikzpicture}\\
\renewcommand{\arraystretch}{.8}
\setlength{\tabcolsep}{1pt}
\begin{tabular}{>{\hfill\scriptsize}p{1cm}<{}>{\hfill\scriptsize}p{.25cm}<{}>{\scriptsize}p{12cm}<{\hfill}}
Source:& \multicolumn{2}{l}{\scriptsize Estimated with GUK administrative and survey data.}\\
Notes: & 1. & ANCOVA estimates using administrative and survey data. Post treatment regressands are regressed on categorical variables, pre-treatment regressand and other covariates. \textsf{FloodInRd1} and \textsf{HeadLiterate0} are indicator variables for the presence of self reported damage by a flood at the baseline, and literacy of household head, respectively. \textsf{HHsize0} is household size at the baseline. We annotate the number of periods that a household is observed with \textsf{T}. The total number of households is shown for each values of \textsf{T}. \textsf{T=4} indicates the number of households with complete panel information, \textsf{T=3} indicates number of households observed three times, \textsf{T=2} indicates the number of households observed twice. \textsf{N} indicates total number of observations used in ANCOVA estimation, or \textsf{N$=$1$\times$(T=2)+2$\times$(T=3)+3$\times$(T=4)}.  \textsf{Large}, \textsf{LargeGrace}, \textsf{Cattle} are indicator variables of the \textsf{large}, \textsf{large grace}, and \textsf{cattle} arms, respectively. The default arm category is \textsf{traditional} arm. Non-livestock assets do not include livestock. \\
& 2. & $P$ values in percentages in parenthesises. Standard errors are clustered at group (village) level.
%${}^{***}$, ${}^{**}$, ${}^{*}$ indicate statistical significance at 1\%, 5\%, 10\%, respetively. Standard errors are clustered at group (village) level.
\end{tabular}
\end{minipage}

\hspace{-1cm}\begin{minipage}[t]{14cm}
\hfil\textsc{\normalsize Table \refstepcounter{table}\thetable: ANCOVA estimation of net non-livestock assets by attributes\label{tab ANCOVA NetNLAssets attributes}}\\
\setlength{\tabcolsep}{1pt}
\setlength{\baselineskip}{8pt}
\renewcommand{\arraystretch}{.55}
\hfil\begin{tikzpicture}
\node (tbl) {\input{c:/data/GUK/analysis/save/EstimationMemo/NetNLAssetsAttributesANCOVAEstimationResults.tex}};
%\input{c:/dropbox/data/ramadan/save/tablecolortemplate.tex}
\end{tikzpicture}\\
\renewcommand{\arraystretch}{.8}
\setlength{\tabcolsep}{1pt}
\begin{tabular}{>{\hfill\scriptsize}p{1cm}<{}>{\hfill\scriptsize}p{.25cm}<{}>{\scriptsize}p{12cm}<{\hfill}}
Source:& \multicolumn{2}{l}{\scriptsize Estimated with GUK administrative and survey data.}\\
Notes: & 1. & ANCOVA estimates using administrative and survey data. Post treatment regressands are regressed on categorical variables, pre-treatment regressand and other covariates. \textsf{FloodInRd1} and \textsf{HeadLiterate0} are indicator variables for the presence of self reported damage by a flood at the baseline, and literacy of household head, respectively. \textsf{HHsize0} is household size at the baseline. We annotate the number of periods that a household is observed with \textsf{T}. The total number of households is shown for each values of \textsf{T}. \textsf{T=4} indicates the number of households with complete panel information, \textsf{T=3} indicates number of households observed three times, \textsf{T=2} indicates the number of households observed twice. \textsf{N} indicates total number of observations used in ANCOVA estimation, or \textsf{N$=$1$\times$(T=2)+2$\times$(T=3)+3$\times$(T=4)}.  \textsf{Large}, \textsf{LargeGrace}, \textsf{Cattle} are indicator variables of the \textsf{large}, \textsf{large grace}, and \textsf{cattle} arms, respectively. The default arm category is \textsf{traditional} arm. Net non-livestockassets do not include livestock. \\
& 2. & $P$ values in percentages in parenthesises. Standard errors are clustered at group (village) level.
%${}^{***}$, ${}^{**}$, ${}^{*}$ indicate statistical significance at 1\%, 5\%, 10\%, respetively. Standard errors are clustered at group (village) level.
\end{tabular}
\end{minipage}

\hspace{-1cm}\begin{minipage}[t]{14cm}
\hfil\textsc{\normalsize Table \refstepcounter{table}\thetable: ANCOVA estimation of net non-livestock assets by period\label{tab ANCOVA NetNLAssets timevarying}}\\
\setlength{\tabcolsep}{1pt}
\setlength{\baselineskip}{8pt}
\renewcommand{\arraystretch}{.55}
\hfil\begin{tikzpicture}
\node (tbl) {\input{c:/data/GUK/analysis/save/EstimationMemo/NetNLAssetsTimeVaryingANCOVAEstimationResults.tex}};
%\input{c:/dropbox/data/ramadan/save/tablecolortemplate.tex}
\end{tikzpicture}\\
\renewcommand{\arraystretch}{.8}
\setlength{\tabcolsep}{1pt}
\begin{tabular}{>{\hfill\scriptsize}p{1cm}<{}>{\hfill\scriptsize}p{.25cm}<{}>{\scriptsize}p{12cm}<{\hfill}}
Source:& \multicolumn{2}{l}{\scriptsize Estimated with GUK administrative and survey data.}\\
Notes: & 1. & ANCOVA estimates using administrative and survey data. Post treatment regressands are regressed on categorical variables, pre-treatment regressand and other covariates. \textsf{FloodInRd1} and \textsf{HeadLiterate0} are indicator variables for the presence of self reported damage by a flood at the baseline, and literacy of household head, respectively. \textsf{HHsize0} is household size at the baseline. We annotate the number of periods that a household is observed with \textsf{T}. The total number of households is shown for each values of \textsf{T}. \textsf{T=4} indicates the number of households with complete panel information, \textsf{T=3} indicates number of households observed three times, \textsf{T=2} indicates the number of households observed twice. \textsf{N} indicates total number of observations used in ANCOVA estimation, or \textsf{N$=$1$\times$(T=2)+2$\times$(T=3)+3$\times$(T=4)}.  \textsf{Large}, \textsf{LargeGrace}, \textsf{Cattle} are indicator variables of the \textsf{large}, \textsf{large grace}, and \textsf{cattle} arms, respectively. The default arm category is \textsf{traditional} arm. Net non-livestockassets do not include livestock. \\
& 2. & $P$ values in percentages in parenthesises. Standard errors are clustered at group (village) level.
%${}^{***}$, ${}^{**}$, ${}^{*}$ indicate statistical significance at 1\%, 5\%, 10\%, respetively. Standard errors are clustered at group (village) level.
\end{tabular}
\end{minipage}

\hspace{-1cm}\begin{minipage}[t]{14cm}
\hfil\textsc{\normalsize Table \refstepcounter{table}\thetable: ANCOVA estimation of net non-livestock assets by attributes and period\label{tab ANCOVA NetNLAssets timevarying attributes}}\\
\setlength{\tabcolsep}{1pt}
\setlength{\baselineskip}{8pt}
\renewcommand{\arraystretch}{.55}
\hfil\begin{tikzpicture}
\node (tbl) {\input{c:/data/GUK/analysis/save/EstimationMemo/NetNLAssetsTimeVaryingAttributesANCOVAEstimationResults.tex}};
%\input{c:/dropbox/data/ramadan/save/tablecolortemplate.tex}
\end{tikzpicture}\\
\renewcommand{\arraystretch}{.8}
\setlength{\tabcolsep}{1pt}
\begin{tabular}{>{\hfill\scriptsize}p{1cm}<{}>{\hfill\scriptsize}p{.25cm}<{}>{\scriptsize}p{12cm}<{\hfill}}
Source:& \multicolumn{2}{l}{\scriptsize Estimated with GUK administrative and survey data.}\\
Notes: & 1. & ANCOVA estimates using administrative and survey data. Post treatment regressands are regressed on categorical variables, pre-treatment regressand and other covariates. \textsf{FloodInRd1} and \textsf{HeadLiterate0} are indicator variables for the presence of self reported damage by a flood at the baseline, and literacy of household head, respectively. \textsf{HHsize0} is household size at the baseline. We annotate the number of periods that a household is observed with \textsf{T}. The total number of households is shown for each values of \textsf{T}. \textsf{T=4} indicates the number of households with complete panel information, \textsf{T=3} indicates number of households observed three times, \textsf{T=2} indicates the number of households observed twice. \textsf{N} indicates total number of observations used in ANCOVA estimation, or \textsf{N$=$1$\times$(T=2)+2$\times$(T=3)+3$\times$(T=4)}.  \textsf{Large}, \textsf{LargeGrace}, \textsf{Cattle} are indicator variables of the \textsf{large}, \textsf{large grace}, and \textsf{cattle} arms, respectively. The default arm category is \textsf{traditional} arm. Net non-livestockassets do not include livestock. \\
2.\\
& 2. & $P$ values in percentages in parenthesises. Standard errors are clustered at group (village) level.
%${}^{***}$, ${}^{**}$, ${}^{*}$ indicate statistical significance at 1\%, 5\%, 10\%, respetively. Standard errors are clustered at group (village) level.
\end{tabular}
\end{minipage}


\hspace{-1cm}\begin{minipage}[t]{14cm}
\hfil\textsc{\normalsize Table \refstepcounter{table}\thetable: ANCOVA estimation of net non-livestock assets by arm, poverty status, and period\label{tab ANCOVA NetNLAssets timevarying poverty status}}\\
\setlength{\tabcolsep}{1pt}
\setlength{\baselineskip}{8pt}
\renewcommand{\arraystretch}{.55}
\hfil\begin{tikzpicture}
\node (tbl) {\input{c:/data/GUK/analysis/save/EstimationMemo/NetNLAssetsTimeVaryingPovertyStatusANCOVAEstimationResults.tex}};
%\input{c:/dropbox/data/ramadan/save/tablecolortemplate.tex}
\end{tikzpicture}\\
\renewcommand{\arraystretch}{.8}
\setlength{\tabcolsep}{1pt}
\begin{tabular}{>{\hfill\scriptsize}p{1cm}<{}>{\hfill\scriptsize}p{.25cm}<{}>{\scriptsize}p{12cm}<{\hfill}}
Source:& \multicolumn{2}{l}{\scriptsize Estimated with GUK administrative and survey data.}\\
Notes: & 1. & ANCOVA estimates using administrative and survey data. Post treatment regressands are regressed on categorical variables, pre-treatment regressand and other covariates. \textsf{FloodInRd1} and \textsf{HeadLiterate0} are indicator variables for the presence of self reported damage by a flood at the baseline, and literacy of household head, respectively. \textsf{HHsize0} is household size at the baseline. We annotate the number of periods that a household is observed with \textsf{T}. The total number of households is shown for each values of \textsf{T}. \textsf{T=4} indicates the number of households with complete panel information, \textsf{T=3} indicates number of households observed three times, \textsf{T=2} indicates the number of households observed twice. \textsf{N} indicates total number of observations used in ANCOVA estimation, or \textsf{N$=$1$\times$(T=2)+2$\times$(T=3)+3$\times$(T=4)}.  \textsf{Large}, \textsf{LargeGrace}, \textsf{Cattle} are indicator variables of the \textsf{large}, \textsf{large grace}, and \textsf{cattle} arms, respectively. The default arm category is \textsf{traditional} arm. Net non-livestockassets do not include livestock. \\
2.\\
& 2. & $P$ values in percentages in parenthesises. Standard errors are clustered at group (village) level.
%${}^{***}$, ${}^{**}$, ${}^{*}$ indicate statistical significance at 1\%, 5\%, 10\%, respetively. Standard errors are clustered at group (village) level.
\end{tabular}
\end{minipage}

\hspace{-1cm}\begin{minipage}[t]{14cm}
\hfil\textsc{\normalsize Table \refstepcounter{table}\thetable: ANCOVA estimation of net non-livestock assets by attributes, poverty status, and period\label{tab ANCOVA NetNLAssets timevarying poverty status attributes}}\\
\setlength{\tabcolsep}{1pt}
\setlength{\baselineskip}{8pt}
\renewcommand{\arraystretch}{.55}
\hfil\begin{tikzpicture}
\node (tbl) {\input{c:/data/GUK/analysis/save/EstimationMemo/NetNLAssetsTimeVaryingPovertyStatusAttributesANCOVAEstimationResults.tex}};
%\input{c:/dropbox/data/ramadan/save/tablecolortemplate.tex}
\end{tikzpicture}\\
\renewcommand{\arraystretch}{.8}
\setlength{\tabcolsep}{1pt}
\begin{tabular}{>{\hfill\scriptsize}p{1cm}<{}>{\hfill\scriptsize}p{.25cm}<{}>{\scriptsize}p{12cm}<{\hfill}}
Source:& \multicolumn{2}{l}{\scriptsize Estimated with GUK administrative and survey data.}\\
Notes: & 1. & ANCOVA estimates using administrative and survey data. Post treatment regressands are regressed on categorical variables, pre-treatment regressand and other covariates. \textsf{FloodInRd1} and \textsf{HeadLiterate0} are indicator variables for the presence of self reported damage by a flood at the baseline, and literacy of household head, respectively. \textsf{HHsize0} is household size at the baseline. We annotate the number of periods that a household is observed with \textsf{T}. The total number of households is shown for each values of \textsf{T}. \textsf{T=4} indicates the number of households with complete panel information, \textsf{T=3} indicates number of households observed three times, \textsf{T=2} indicates the number of households observed twice. \textsf{N} indicates total number of observations used in ANCOVA estimation, or \textsf{N$=$1$\times$(T=2)+2$\times$(T=3)+3$\times$(T=4)}.  \textsf{Large}, \textsf{LargeGrace}, \textsf{Cattle} are indicator variables of the \textsf{large}, \textsf{large grace}, and \textsf{cattle} arms, respectively. The default arm category is \textsf{traditional} arm. Net non-livestockassets do not include livestock. \\
& 2. & $P$ values in percentages in parenthesises. Standard errors are clustered at group (village) level.
%${}^{***}$, ${}^{**}$, ${}^{*}$ indicate statistical significance at 1\%, 5\%, 10\%, respetively. Standard errors are clustered at group (village) level.
\end{tabular}
\end{minipage}


%\subsubsection{Narrow net non-livestock assets: Non-livestock assets-GUK Debt-Other Debts}

\subsubsection{Cattle holding}



\begin{Schunk}
\begin{Soutput}
             AttritIn
Arm             2   3   4   9 Sum
  traditional   7   4  20 144 175
  large         5   2   1 192 200
  large grace  12   3   3 171 189
  cattle        5   5  13 176 199
  Sum          29  14  37 683 763
     NumCows
tee      0    1    2    3    4    5    6    7    8    9 <NA>  Sum
  2     15  309  153   40   11    1    2    0    1    1  197  730
  3      5  337  175   40   16    1    2    2    1    0  110  689
  4      4  218  201   54   11    4    2    0    1    1   86  582
  Sum   24  864  529  134   38    6    6    2    3    2  393 2001
\end{Soutput}
\end{Schunk}
%  source(paste0(pathprogram, "ReadTrimLivestockByExperienceANCOVA.R"))
\begin{Schunk}
\begin{Soutput}
      NumCows0
Cattle   0   1   2   3   4   5 Sum
  Adi  108   0   0   0   0   0 108
  None 484   0   0   0   0   0 484
  Own    0  99  30   5   3   1 138
  Sum  592  99  30   5   3   1 730
\end{Soutput}
\end{Schunk}




\begin{Schunk}
\begin{Soutput}


[1]
 ~  + dummyLarge + dummyLargeGrace + dummyCattle

 + NumCows0
FloodInRd1 + HHsize0 + HeadLiteracy0 + NA

 + dummyHadCows
 + TotalImputed2Value0 

[2]
 ~  + dummyUltraPoor + dummyLargeSize + dummyWithGrace
 + dummyWithGrace + dummyInKind + UDdummyUltraPoor
 + dummyLargeSize.UltraPoor + UDdummyLargeSize.UltraPoor + dummyWithGrace.UltraPoor
 + UDdummyWithGrace.UltraPoor + dummyInKind.UltraPoor + UDdummyInKind.UltraPoor
 + 

 + NumCows0
FloodInRd1 + HHsize0 + HeadLiteracy0 + NA

 + dummyHadCows
 + TotalImputed2Value0 

[3]
 ~  + dummyLargeSize + dummyWithGrace + dummyInKind

 + NumCows0
FloodInRd1 + HHsize0 + HeadLiteracy0 + NA

 + dummyHadCows
 + TotalImputed2Value0 

[4]
 ~  + Time.3 + Time.4 + dummyLarge
 + dummyLarge + dummyLargeGrace + dummyCattle
 + dummyLarge.Time3 + dummyLargeGrace.Time3 + dummyCattle.Time3
 + dummyLarge.Time4 + dummyLargeGrace.Time4 + dummyCattle.Time4
 + 

 + NumCows0
FloodInRd1 + HHsize0 + HeadLiteracy0 + NA

 + dummyHadCows
 + TotalImputed2Value0 

[5]
 ~  + Time.3 + Time.4 + dummyLargeSize
 + dummyLargeSize + dummyWithGrace + dummyInKind
 + dummyLargeSize.Time3 + dummyWithGrace.Time3 + dummyInKind.Time3
 + dummyLargeSize.Time4 + dummyWithGrace.Time4 + dummyInKind.Time4
 + 

 + NumCows0
FloodInRd1 + HHsize0 + HeadLiteracy0 + NA

 + dummyHadCows
 + TotalImputed2Value0 

[6]
 ~  + Time.3 + Time.4 + dummyLarge
 + dummyLarge + dummyLargeGrace + dummyCattle
 + dummyUltraPoor + dummyLarge.Time3 + dummyLargeGrace.Time3
 + dummyCattle.Time3 + dummyUltraPoor.Time3 + dummyLarge.Time4
 + dummyLargeGrace.Time4 + dummyCattle.Time4 + dummyUltraPoor.Time4
 + dummyLarge.UltraPoor + dummyLargeGrace.UltraPoor + dummyCattle.UltraPoor
 + dummyLarge.UltraPoor.Time3 + dummyLarge.UltraPoor.Time4 + dummyLargeGrace.UltraPoor.Time3
 + dummyLargeGrace.UltraPoor.Time4 + dummyCattle.UltraPoor.Time3 + dummyCattle.UltraPoor.Time4
 + 

 + NumCows0
FloodInRd1 + HHsize0 + HeadLiteracy0 + NA

 + dummyHadCows
 + TotalImputed2Value0 

[7]
 ~  + Time.3 + Time.4 + dummyUltraPoor
 + dummyUltraPoor + dummyLargeSize + dummyWithGrace
 + dummyInKind + dummyUltraPoor.Time3 + dummyLargeSize.Time3
 + dummyWithGrace.Time3 + dummyInKind.Time3 + dummyUltraPoor.Time4
 + dummyLargeSize.Time4 + dummyWithGrace.Time4 + dummyInKind.Time4
 + dummyLargeSize.UltraPoor + dummyWithGrace.UltraPoor + dummyInKind.UltraPoor
 + dummyLargeSize.UltraPoor.Time3 + dummyLargeSize.UltraPoor.Time4 + dummyWithGrace.UltraPoor.Time3
 + dummyWithGrace.UltraPoor.Time4 + dummyInKind.UltraPoor.Time3 + dummyInKind.UltraPoor.Time4
 + 

 + NumCows0
FloodInRd1 + HHsize0 + HeadLiteracy0 + NA

 + dummyHadCows
 + TotalImputed2Value0 
\end{Soutput}
\end{Schunk}



\hspace{-1cm}\begin{minipage}[t]{14cm}
\hfil\textsc{\normalsize Table \refstepcounter{table}\thetable: ANCOVA estimation of cattle holding\label{tab ANCOVA cow}}\\
\setlength{\tabcolsep}{1pt}
\setlength{\baselineskip}{8pt}
\renewcommand{\arraystretch}{.55}
\hfil\begin{tikzpicture}
\node (tbl) {\input{c:/data/GUK/analysis/save/EstimationMemo/NumCowsANCOVAEstimationResults.tex}};
%\input{c:/dropbox/data/ramadan/save/tablecolortemplate.tex}
\end{tikzpicture}\\
\renewcommand{\arraystretch}{.8}
\setlength{\tabcolsep}{1pt}
\begin{tabular}{>{\hfill\scriptsize}p{1cm}<{}>{\hfill\scriptsize}p{.25cm}<{}>{\scriptsize}p{12cm}<{\hfill}}
Source:& \multicolumn{2}{l}{\scriptsize Estimated with GUK administrative and survey data.}\\
Notes: & 1. & ANCOVA estimates using administrative and survey data. Post treatment regressands are regressed on categorical variables, pre-treatment regressand and other covariates. \textsf{FloodInRd1} and \textsf{HeadLiterate0} are indicator variables for the presence of self reported damage by a flood at the baseline, and literacy of household head, respectively. \textsf{HHsize0} is household size at the baseline. We annotate the number of periods that a household is observed with \textsf{T}. The total number of households is shown for each values of \textsf{T}. \textsf{T=4} indicates the number of households with complete panel information, \textsf{T=3} indicates number of households observed three times, \textsf{T=2} indicates the number of households observed twice. \textsf{N} indicates total number of observations used in ANCOVA estimation, or \textsf{N$=$1$\times$(T=2)+2$\times$(T=3)+3$\times$(T=4)}.  \textsf{Upfront} is an indicator variable of the arm with an upfront large disbursement, \textsf{WithGrace} is an indicator variable of the arm with a grace period, \textsf{InKind} is an indicator variable of the arm which lends a heifer. Regressand is \textsf{NumCows}, number of cattle holding. \\
& 2. & $P$ values in percentages in parenthesises. Standard errors are clustered at group (village) level.
%${}^{***}$, ${}^{**}$, ${}^{*}$ indicate statistical significance at 1\%, 5\%, 10\%, respetively. Standard errors are clustered at group (village) level.
\end{tabular}
\end{minipage}

\hspace{-1cm}\begin{minipage}[t]{14cm}
\hfil\textsc{\normalsize Table \refstepcounter{table}\thetable: ANCOVA estimation of cattle holding by attributes\label{tab ANCOVA cow attributes}}\\
\setlength{\tabcolsep}{1pt}
\setlength{\baselineskip}{8pt}
\renewcommand{\arraystretch}{.55}
\hfil\begin{tikzpicture}
\node (tbl) {\input{c:/data/GUK/analysis/save/EstimationMemo/NumCowsAttributesANCOVAEstimationResults.tex}};
%\input{c:/dropbox/data/ramadan/save/tablecolortemplate.tex}
\end{tikzpicture}\\
\renewcommand{\arraystretch}{.8}
\setlength{\tabcolsep}{1pt}
\begin{tabular}{>{\hfill\scriptsize}p{1cm}<{}>{\hfill\scriptsize}p{.25cm}<{}>{\scriptsize}p{12cm}<{\hfill}}
Source:& \multicolumn{2}{l}{\scriptsize Estimated with GUK administrative and survey data.}\\
Notes: & 1. & ANCOVA estimates using administrative and survey data. Post treatment regressands are regressed on categorical variables, pre-treatment regressand and other covariates. \textsf{FloodInRd1} and \textsf{HeadLiterate0} are indicator variables for the presence of self reported damage by a flood at the baseline, and literacy of household head, respectively. \textsf{HHsize0} is household size at the baseline. We annotate the number of periods that a household is observed with \textsf{T}. The total number of households is shown for each values of \textsf{T}. \textsf{T=4} indicates the number of households with complete panel information, \textsf{T=3} indicates number of households observed three times, \textsf{T=2} indicates the number of households observed twice. \textsf{N} indicates total number of observations used in ANCOVA estimation, or \textsf{N$=$1$\times$(T=2)+2$\times$(T=3)+3$\times$(T=4)}.  \textsf{Upfront} is an indicator variable of the arm with an upfront large disbursement, \textsf{WithGrace} is an indicator variable of the arm with a grace period, \textsf{InKind} is an indicator variable of the arm which lends a heifer. Regressand is \textsf{NumCows}, number of cattle holding. \\
& 2. & $P$ values in percentages in parenthesises. Standard errors are clustered at group (village) level.
%${}^{***}$, ${}^{**}$, ${}^{*}$ indicate statistical significance at 1\%, 5\%, 10\%, respetively. Standard errors are clustered at group (village) level.
\end{tabular}
\end{minipage}

\hspace{-1cm}\begin{minipage}[t]{14cm}
\hfil\textsc{\normalsize Table \refstepcounter{table}\thetable: ANCOVA estimation of cattle holding, ultra vs. moderately poor\label{tab ANCOVA NumCows poor}}\\
\setlength{\tabcolsep}{1pt}
\setlength{\baselineskip}{8pt}
\renewcommand{\arraystretch}{.55}
\hfil\begin{tikzpicture}
\node (tbl) {\input{c:/data/GUK/analysis/save/EstimationMemo/NumCowsPovertyStatusANCOVAEstimationResults.tex}};
\end{tikzpicture}\\
\renewcommand{\arraystretch}{.8}
\setlength{\tabcolsep}{1pt}
\begin{tabular}{>{\hfill\scriptsize}p{1cm}<{}>{\hfill\scriptsize}p{.25cm}<{}>{\scriptsize}p{12cm}<{\hfill}}
Source:& \multicolumn{2}{l}{\scriptsize Estimated with GUK administrative and survey data.}\\
Notes: & 1. & ANCOVA estimates using administrative and survey data. Post treatment regressands are regressed on categorical variables, pre-treatment regressand and other covariates. \textsf{FloodInRd1} and \textsf{HeadLiterate0} are indicator variables for the presence of self reported damage by a flood at the baseline, and literacy of household head, respectively. \textsf{HHsize0} is household size at the baseline. We annotate the number of periods that a household is observed with \textsf{T}. The total number of households is shown for each values of \textsf{T}. \textsf{T=4} indicates the number of households with complete panel information, \textsf{T=3} indicates number of households observed three times, \textsf{T=2} indicates the number of households observed twice. \textsf{N} indicates total number of observations used in ANCOVA estimation, or \textsf{N$=$1$\times$(T=2)+2$\times$(T=3)+3$\times$(T=4)}.  \textsf{Large}, \textsf{LargeGrace}, \textsf{Cattle} are indicator variables of the \textsf{large}, \textsf{large grace}, and \textsf{cattle} arms, respectively. The default arm category is \textsf{traditional} arm. \textsf{UltraPoor} is an indicator variable if the household is classified as the ultra poor. Regressand is \textsf{NumCows}, number of cattle holding. \\
& 2. & $P$ values in percentages in parenthesises. Standard errors are clustered at group (village) level.
%${}^{***}$, ${}^{**}$, ${}^{*}$ indicate statistical significance at 1\%, 5\%, 10\%, respetively. Standard errors are clustered at group (village) level.
\end{tabular}
\end{minipage}

\hspace{-1cm}\begin{minipage}[t]{14cm}
\hfil\textsc{\normalsize Table \refstepcounter{table}\thetable: ANCOVA estimation of cattle holding by arm and period\label{tab ANCOVA cow time varying}}\\
\setlength{\tabcolsep}{1pt}
\setlength{\baselineskip}{8pt}
\renewcommand{\arraystretch}{.55}
\hfil\begin{tikzpicture}
\node (tbl) {\input{c:/data/GUK/analysis/save/EstimationMemo/NumCowsTimeVaryingANCOVAEstimationResults.tex}};
\end{tikzpicture}\\
\renewcommand{\arraystretch}{.8}
\setlength{\tabcolsep}{1pt}
\begin{tabular}{>{\hfill\scriptsize}p{1cm}<{}>{\hfill\scriptsize}p{.25cm}<{}>{\scriptsize}p{12cm}<{\hfill}}
Source:& \multicolumn{2}{l}{\scriptsize Estimated with GUK administrative and survey data.}\\
Notes: & 1. & ANCOVA estimates using administrative and survey data. Post treatment regressands are regressed on categorical variables, pre-treatment regressand and other covariates. \textsf{FloodInRd1} and \textsf{HeadLiterate0} are indicator variables for the presence of self reported damage by a flood at the baseline, and literacy of household head, respectively. \textsf{HHsize0} is household size at the baseline. We annotate the number of periods that a household is observed with \textsf{T}. The total number of households is shown for each values of \textsf{T}. \textsf{T=4} indicates the number of households with complete panel information, \textsf{T=3} indicates number of households observed three times, \textsf{T=2} indicates the number of households observed twice. \textsf{N} indicates total number of observations used in ANCOVA estimation, or \textsf{N$=$1$\times$(T=2)+2$\times$(T=3)+3$\times$(T=4)}.  \textsf{Large}, \textsf{LargeGrace}, \textsf{Cattle} are indicator variables of the \textsf{large}, \textsf{large grace}, and \textsf{cattle} arms, respectively. The default arm category is \textsf{traditional} arm. \textsf{rd2, rd3, rd4} are dummy variables for second, third, and fourth round of survey. Sample is continuing members and replacing members of early rejecters and received loans prior to 2015 Janunary. Regressand is \textsf{NumCows}, number of cattle holding. \\
& 2. & $P$ values in percentages in parenthesises. Standard errors are clustered at group (village) level.
%${}^{***}$, ${}^{**}$, ${}^{*}$ indicate statistical significance at 1\%, 5\%, 10\%, respetively. Standard errors are clustered at group (village) level.
\end{tabular}
\end{minipage}

\hspace{-1cm}\begin{minipage}[t]{14cm}
\hfil\textsc{\normalsize Table \refstepcounter{table}\thetable: ANCOVA estimation of cattle holding by attributes and period\label{tab ANCOVA cow time varying attributes}}\\
\setlength{\tabcolsep}{1pt}
\setlength{\baselineskip}{8pt}
\renewcommand{\arraystretch}{.55}
\hfil\begin{tikzpicture}
\node (tbl) {\input{c:/data/GUK/analysis/save/EstimationMemo/NumCowsTimeVaryingAttributesANCOVAEstimationResults.tex}};
\end{tikzpicture}\\
\renewcommand{\arraystretch}{.8}
\setlength{\tabcolsep}{1pt}
\begin{tabular}{>{\hfill\scriptsize}p{1cm}<{}>{\hfill\scriptsize}p{.25cm}<{}>{\scriptsize}p{12cm}<{\hfill}}
Source:& \multicolumn{2}{l}{\scriptsize Estimated with GUK administrative and survey data.}\\
Notes: & 1. & ANCOVA estimates using administrative and survey data. Post treatment regressands are regressed on categorical variables, pre-treatment regressand and other covariates. \textsf{FloodInRd1} and \textsf{HeadLiterate0} are indicator variables for the presence of self reported damage by a flood at the baseline, and literacy of household head, respectively. \textsf{HHsize0} is household size at the baseline. We annotate the number of periods that a household is observed with \textsf{T}. The total number of households is shown for each values of \textsf{T}. \textsf{T=4} indicates the number of households with complete panel information, \textsf{T=3} indicates number of households observed three times, \textsf{T=2} indicates the number of households observed twice. \textsf{N} indicates total number of observations used in ANCOVA estimation, or \textsf{N$=$1$\times$(T=2)+2$\times$(T=3)+3$\times$(T=4)}.  \textsf{Upfront} is an indicator variable of the arm with an upfront large disbursement, \textsf{WithGrace} is an indicator variable of the arm with a grace period, \textsf{InKind} is an indicator variable of the arm which lends a heifer. \textsf{rd2, rd3, rd4} are dummy variables for second, third, and fourth round of survey. Sample is continuing members and replacing members of early rejecters and received loans prior to 2015 Janunary. Regressand is \textsf{NumCows}, number of cattle holding. \\
& 2. & $P$ values in percentages in parenthesises. Standard errors are clustered at group (village) level.
%${}^{***}$, ${}^{**}$, ${}^{*}$ indicate statistical significance at 1\%, 5\%, 10\%, respetively. Standard errors are clustered at group (village) level.
\end{tabular}
\end{minipage}

\hspace{-1cm}\begin{minipage}[t]{14cm}
\hfil\textsc{\normalsize Table \refstepcounter{table}\thetable: ANCOVA estimation of cattle holding by arm, period, and poverty class\label{tab NumCowsExperienceTimeVaryingPovertyStatusANCOVA}}\\
\setlength{\tabcolsep}{1pt}
\setlength{\baselineskip}{8pt}
\renewcommand{\arraystretch}{.55}
\hfil\begin{tikzpicture}
\node (tbl) {\input{c:/data/GUK/analysis/save/EstimationMemo/NumCowsExperienceTimeVaryingPovertyStatusANCOVAEstimationResults_1.tex}};
\end{tikzpicture}
\end{minipage}

\hspace{-1cm}\begin{minipage}[t]{14cm}
\hfil\textsc{\normalsize Table \refstepcounter{table}\thetable: ANCOVA estimation of cattle holding by arm, period, and poverty class (continued)\label{tab ANCOVA cow time varying poverty2}}\\
\setlength{\tabcolsep}{1pt}
\setlength{\baselineskip}{8pt}
\renewcommand{\arraystretch}{.55}
\hfil\begin{tikzpicture}
\node (tbl) {\input{c:/data/GUK/analysis/save/EstimationMemo/NumCowsExperienceTimeVaryingPovertyStatusANCOVAEstimationResults_2.tex}};
\end{tikzpicture}\\
\renewcommand{\arraystretch}{.8}
\setlength{\tabcolsep}{1pt}
\begin{tabular}{>{\hfill\scriptsize}p{1cm}<{}>{\hfill\scriptsize}p{.25cm}<{}>{\scriptsize}p{12cm}<{\hfill}}
Source:& \multicolumn{2}{l}{\scriptsize Estimated with GUK administrative and survey data.}\\
Notes: & 1. & ANCOVA estimates using administrative and survey data. Post treatment regressands are regressed on categorical variables, pre-treatment regressand and other covariates. \textsf{FloodInRd1} and \textsf{HeadLiterate0} are indicator variables for the presence of self reported damage by a flood at the baseline, and literacy of household head, respectively. \textsf{HHsize0} is household size at the baseline. We annotate the number of periods that a household is observed with \textsf{T}. The total number of households is shown for each values of \textsf{T}. \textsf{T=4} indicates the number of households with complete panel information, \textsf{T=3} indicates number of households observed three times, \textsf{T=2} indicates the number of households observed twice. \textsf{N} indicates total number of observations used in ANCOVA estimation, or \textsf{N$=$1$\times$(T=2)+2$\times$(T=3)+3$\times$(T=4)}.  \textsf{Large}, \textsf{LargeGrace}, \textsf{Cattle} are indicator variables of the \textsf{large}, \textsf{large grace}, and \textsf{cattle} arms, respectively. The default arm category is \textsf{traditional} arm. \textsf{rd2, rd3, rd4} are dummy variables for second, third, and fourth round of survey. Sample is continuing members and replacing members of early rejecters and received loans prior to 2015 Janunary. Regressand is \textsf{NumCows}, number of cattle holding. \\
& 2. & $P$ values in percentages in parenthesises. Standard errors are clustered at group (village) level.
%${}^{***}$, ${}^{**}$, ${}^{*}$ indicate statistical significance at 1\%, 5\%, 10\%, respetively. Standard errors are clustered at group (village) level.
\end{tabular}
\end{minipage}

\subsubsection{Net assets, experienced vs. inexperienced}



\begin{Schunk}
\begin{Soutput}
        LeaseInCattle
NumCows0   0   1 <NA> Sum
     0   539  94    0 633
     1   100   0    1 101
     2    30   0    0  30
     3     6   0    0   6
     4     2   0    1   3
     5     1   0    0   1
     Sum 678  94    2 774
\end{Soutput}
\begin{Soutput}
                 LeaseInCattle
YearsSinceLastAdi   0   1 <NA> Sum
             1      3   0    0   3
             2     13   0    0  13
             3      8   0    0   8
             <NA> 654  94    2 750
             Sum  678  94    2 774
\end{Soutput}
\end{Schunk}
Even LeaseInCattle $==$ 0 \& OwnCattle $==$ 0, some had Adi experiences.
\begin{Schunk}
\begin{Soutput}
                 OwnCattle
YearsSinceLastAdi   0   1 Sum
             1      1   2   3
             2     11   2  13
             3      6   2   8
             <NA> 483 171 654
             Sum  501 177 678
\end{Soutput}
\end{Schunk}
Recode to Adi if OwnCattle == 0 \& $!$is.na(YearsSinceLastAdi). This means, \textsf{LeaseInCattle} == 1 if OwnCattle = 0 but has experience of Adi in last 3 years.

If we (and we will) assume that the \textsf{NumCows0} (computed from \textsf{"abu\_livestockownershipupdated.dta}) as the truth, there are 52 false positives in \textsf{OwnCattle} (falsely reporting cattle ownership at baseline). 
\begin{Schunk}
\begin{Soutput}
        OwnCattle
NumCows0   0   1 <NA> Sum
     0   581  52    0 633
     1     0 100    1 101
     2     0  30    0  30
     3     0   6    0   6
     4     0   2    1   3
     5     0   1    0   1
     Sum 581 191    2 774
\end{Soutput}
\end{Schunk}
NumCows are computed in \textsf{read\_clean\_data.rnw} by:
\begin{screen}
\verb+xloL[, NumCows := as.integer(NA)]+\\
\verb+xloL[grepl("ow", LVcode), NumCows := as.integer(number_owned)]+\\
\verb+xloL[, NumCows := NumCows[grepl("ow", LVcode)], by = .(hhid, survey)]+
\end{screen}
\textsf{xloL} is the raw data file \textsf{"abu\_livestockownershipupdated.dta}. 

	We will correct Table 8a baseline data such that it becomes consistent with \textsf{"abu\_livestockownershipupdated.dta}. 
\begin{screen}
\verb+lvLv[NumCows0 == 0 & OwnCattle == 1L, OwnCattle := 0L]+\\
\verb+lvLv[NumCows0 > 0 & is.na(OwnCattle), OwnCattle := 1L]+\\
\verb+lvLv[, NumCows0 := NULL]+
\end{screen}
We will also correct \textsf{LeaseInCattle} from NA to 0, if \textsf{NumCows0} $>0$ (2 cases). Save it as \textsf{lvLv}, and use it in estimation.
\begin{Schunk}
\begin{Soutput}
                 OwnCattle
YearsSinceLastAdi   0   1 Sum
             1      0   2   2
             2      0   2   2
             3      0   2   2
             <NA> 521 135 656
             Sum  521 141 662
\end{Soutput}
\begin{Soutput}
             OwnCattle
LeaseInCattle   0   1 Sum
          0   521 141 662
          1   112   0 112
          Sum 633 141 774
\end{Soutput}
\end{Schunk}



\begin{Schunk}
\begin{Soutput}
             AttritIn
Arm             2   3   4   9 Sum
  traditional   6   4  20 144 174
  large         5   2   1 192 200
  large grace  22   3   3 171 199
  cattle        5   5  13 177 200
  Sum          38  14  37 684 773
                      AttritIn
BStatus                  2   3   4   9 Sum
  borrower               8   6   8 578 600
  pure saver             0   0   0   0   0
  individual rejection   9   4   1  75  89
  group rejection        9   4   0  55  68
  rejection by flood    12   0  28   0  40
  Sum                   38  14  37 708 797
         AttritIn
TradGroup   2   3   4   9 Sum
  planned   0   0   1  83  84
  twice     0   0   0  24  24
  double    0   0   0   0   0
  <NA>     38  14  36 601 689
  Sum      38  14  37 708 797
             AttritIn
Arm             2   3   4   9 Sum
  traditional   6   4  20 168 198
  large         5   2   1 192 200
  large grace  22   3   3 171 199
  cattle        5   5  13 177 200
  Sum          38  14  37 708 797
             AttritIn
Arm             2   3   4   9 Sum
  traditional   6   4  20 144 174
  large         5   2   1 192 200
  large grace  22   3   3 171 199
  cattle        5   5  13 177 200
  Sum          38  14  37 684 773


Number of obs based on assets
             tee
Arm              1    2    3    4  Sum
  traditional  174  166  162  133  635
  large        200  194  191  179  764
  large grace  199  177  174  155  705
  cattle       200  195  188  151  734
  Sum          773  732  715  618 2838
             AttritIn
Arm             2   3   4   9 Sum
  traditional   6   4  20 144 174
  large         5   2   1 192 200
  large grace  22   3   3 171 199
  cattle        5   5  13 177 200
  Sum          38  14  37 684 773


Number of obs based on roster
             AttritIn
Arm             2   3   4   9 Sum
  traditional   6   4  20 144 174
  large         5   2   1 192 200
  large grace  22   3   3 171 199
  cattle        5   5  13 177 200
  Sum          38  14  37 684 773


Number of nonattriting obs but with lacking 4 entries in assets
             ObPattern
Arm           0111 1111 <NA> Sum
  traditional    1    1    9  11
  large          3    0   10  13
  large grace    5    1   10  16
  cattle         4    1   21  26
  Sum           13    3   50  66
\end{Soutput}
\end{Schunk}










\hspace{-1cm}\begin{minipage}[t]{14cm}
\hfil\textsc{\normalsize Table \refstepcounter{table}\thetable: ANCOVA estimation of net assets, cattle rearing experiences\label{tab ANCOVA net assets experience}}\\
\setlength{\tabcolsep}{1pt}
\setlength{\baselineskip}{8pt}
\renewcommand{\arraystretch}{.52}
\hfil\begin{tikzpicture}
\node (tbl) {\input{c:/data/GUK/analysis/save/EstimationMemo/NetAssetsExperienceANCOVAEstimationResults.tex}};
%\input{c:/dropbox/data/ramadan/save/tablecolortemplate.tex}
\end{tikzpicture}\\
\renewcommand{\arraystretch}{.8}
\setlength{\tabcolsep}{1pt}
\begin{tabular}{>{\hfill\scriptsize}p{1cm}<{}>{\hfill\scriptsize}p{.25cm}<{}>{\scriptsize}p{12cm}<{\hfill}}
Source:& \multicolumn{2}{l}{\scriptsize Estimated with GUK administrative and survey data.}\\
Notes: & 1. & ANCOVA estimates using administrative and survey data. Post treatment regressands are regressed on categorical variables, pre-treatment regressand and other covariates. \textsf{FloodInRd1} and \textsf{HeadLiterate0} are indicator variables for the presence of self reported damage by a flood at the baseline, and literacy of household head, respectively. \textsf{HHsize0} is household size at the baseline. We annotate the number of periods that a household is observed with \textsf{T}. The total number of households is shown for each values of \textsf{T}. \textsf{T=4} indicates the number of households with complete panel information, \textsf{T=3} indicates number of households observed three times, \textsf{T=2} indicates the number of households observed twice. \textsf{N} indicates total number of observations used in ANCOVA estimation, or \textsf{N$=$1$\times$(T=2)+2$\times$(T=3)+3$\times$(T=4)}.  \textsf{Large}, \textsf{LargeGrace}, \textsf{Cattle} are indicator variables of the \textsf{large}, \textsf{large grace}, and \textsf{cattle} arms, respectively. The default arm category is \textsf{traditional} arm. Net assets uses only assets observed for all 4 rounds in household assets. Household assets do not include livestock. \textsf{OwnCattle0} is an indicator if a household owned cattle at the baseline. \textsf{AdiCattle0} is an indicator if a household engaged in the cattle lease-in contract at the baseline.  \\
& 2. & $P$ values in percentages in parenthesises. Standard errors are clustered at group (village) level.
%${}^{***}$, ${}^{**}$, ${}^{*}$ indicate statistical significance at 1\%, 5\%, 10\%, respetively. Standard errors are clustered at group (village) level.
\end{tabular}
\end{minipage}

\vspace{-1cm}\hspace{-1cm}\begin{minipage}[t]{14cm}
\hfil\textsc{\normalsize Table \refstepcounter{table}\thetable: ANCOVA estimation of net assets by attributes, cattle rearing experiences\label{tab ANCOVA narrow net assets Experience attributes}}\\
\setlength{\tabcolsep}{1pt}
\setlength{\baselineskip}{8pt}
\renewcommand{\arraystretch}{.52}
\hfil\begin{tikzpicture}
\node (tbl) {\input{c:/data/GUK/analysis/save/EstimationMemo/NetAssetsExperienceAttributesANCOVAEstimationResults.tex}};
%\input{c:/dropbox/data/ramadan/save/tablecolortemplate.tex}
\end{tikzpicture}\\
\renewcommand{\arraystretch}{.8}
\setlength{\tabcolsep}{1pt}
\begin{tabular}{>{\hfill\scriptsize}p{1cm}<{}>{\hfill\scriptsize}p{.25cm}<{}>{\scriptsize}p{12cm}<{\hfill}}
Source:& \multicolumn{2}{l}{\scriptsize Estimated with GUK administrative and survey data.}\\
Notes: & 1. & ANCOVA estimates using administrative and survey data. Post treatment regressands are regressed on categorical variables, pre-treatment regressand and other covariates. \textsf{FloodInRd1} and \textsf{HeadLiterate0} are indicator variables for the presence of self reported damage by a flood at the baseline, and literacy of household head, respectively. \textsf{HHsize0} is household size at the baseline. We annotate the number of periods that a household is observed with \textsf{T}. The total number of households is shown for each values of \textsf{T}. \textsf{T=4} indicates the number of households with complete panel information, \textsf{T=3} indicates number of households observed three times, \textsf{T=2} indicates the number of households observed twice. \textsf{N} indicates total number of observations used in ANCOVA estimation, or \textsf{N$=$1$\times$(T=2)+2$\times$(T=3)+3$\times$(T=4)}.  \textsf{Upfront} is an indicator variable of the arm with an upfront large disbursement, \textsf{WithGrace} is an indicator variable of the arm with a grace period, \textsf{InKind} is an indicator variable of the arm which lends a heifer. Net assets uses only assets observed for all 4 rounds in household assets. Household assets do not include livestock. \textsf{OwnCattle0} is an indicator if a household owned cattle at the baseline. \textsf{AdiCattle0} is an indicator if a household engaged in the cattle lease-in contract at the baseline.  \\
& 2. & $P$ values in percentages in parenthesises. Standard errors are clustered at group (village) level.
%${}^{***}$, ${}^{**}$, ${}^{*}$ indicate statistical significance at 1\%, 5\%, 10\%, respetively. Standard errors are clustered at group (village) level.
\end{tabular}
\end{minipage}

\vspace{-1cm}\hspace{-1cm}\begin{minipage}[t]{14cm}
\hfil\textsc{\normalsize Table \refstepcounter{table}\thetable: ANCOVA estimation of net assets by period, cattle rearing experiences\label{tab ANCOVA narrow net assets Experience timevarying}}\\
\setlength{\tabcolsep}{1pt}
\setlength{\baselineskip}{8pt}
\renewcommand{\arraystretch}{.52}
\hfil\begin{tikzpicture}
\node (tbl) {\input{c:/data/GUK/analysis/save/EstimationMemo/NetAssetsExperienceTimeVaryingANCOVAEstimationResults.tex}};
%\input{c:/dropbox/data/ramadan/save/tablecolortemplate.tex}
\end{tikzpicture}\\
\renewcommand{\arraystretch}{.8}
\setlength{\tabcolsep}{1pt}
\begin{tabular}{>{\hfill\scriptsize}p{1cm}<{}>{\hfill\scriptsize}p{.25cm}<{}>{\scriptsize}p{12cm}<{\hfill}}
Source:& \multicolumn{2}{l}{\scriptsize Estimated with GUK administrative and survey data.}\\
Notes: & 1. & ANCOVA estimates using administrative and survey data. Post treatment regressands are regressed on categorical variables, pre-treatment regressand and other covariates. \textsf{FloodInRd1} and \textsf{HeadLiterate0} are indicator variables for the presence of self reported damage by a flood at the baseline, and literacy of household head, respectively. \textsf{HHsize0} is household size at the baseline. We annotate the number of periods that a household is observed with \textsf{T}. The total number of households is shown for each values of \textsf{T}. \textsf{T=4} indicates the number of households with complete panel information, \textsf{T=3} indicates number of households observed three times, \textsf{T=2} indicates the number of households observed twice. \textsf{N} indicates total number of observations used in ANCOVA estimation, or \textsf{N$=$1$\times$(T=2)+2$\times$(T=3)+3$\times$(T=4)}.  \textsf{Large}, \textsf{LargeGrace}, \textsf{Cattle} are indicator variables of the \textsf{large}, \textsf{large grace}, and \textsf{cattle} arms, respectively. The default arm category is \textsf{traditional} arm. \textsf{rd2, rd3, rd4} are dummy variables for second, third, and fourth round of survey. Net assets uses only assets observed for all 4 rounds in household assets. Household assets do not include livestock. \textsf{OwnCattle0} is an indicator if a household owned cattle at the baseline. \textsf{AdiCattle0} is an indicator if a household engaged in the cattle lease-in contract at the baseline.  \\
& 2. & $P$ values in percentages in parenthesises. Standard errors are clustered at group (village) level.
%${}^{***}$, ${}^{**}$, ${}^{*}$ indicate statistical significance at 1\%, 5\%, 10\%, respetively. Standard errors are clustered at group (village) level.
\end{tabular}
\end{minipage}

\vspace{-1cm}\hspace{-1cm}\begin{minipage}[t]{14cm}
\hfil\textsc{\normalsize Table \refstepcounter{table}\thetable: ANCOVA estimation of net assets by attributes and period, cattle rearing experiences\label{tab ANCOVA narrow net assets Experience timevarying attributes}}\\
\setlength{\tabcolsep}{1pt}
\setlength{\baselineskip}{8pt}
\renewcommand{\arraystretch}{.52}
\hfil\begin{tikzpicture}
\node (tbl) {\input{c:/data/GUK/analysis/save/EstimationMemo/NetAssetsExperienceTimeVaryingAttributesANCOVAEstimationResults.tex}};
%\input{c:/dropbox/data/ramadan/save/tablecolortemplate.tex}
\end{tikzpicture}\\
\renewcommand{\arraystretch}{.8}
\setlength{\tabcolsep}{1pt}
\begin{tabular}{>{\hfill\scriptsize}p{1cm}<{}>{\hfill\scriptsize}p{.25cm}<{}>{\scriptsize}p{12cm}<{\hfill}}
Source:& \multicolumn{2}{l}{\scriptsize Estimated with GUK administrative and survey data.}\\
Notes: & 1. & ANCOVA estimates using administrative and survey data. Post treatment regressands are regressed on categorical variables, pre-treatment regressand and other covariates. \textsf{FloodInRd1} and \textsf{HeadLiterate0} are indicator variables for the presence of self reported damage by a flood at the baseline, and literacy of household head, respectively. \textsf{HHsize0} is household size at the baseline. We annotate the number of periods that a household is observed with \textsf{T}. The total number of households is shown for each values of \textsf{T}. \textsf{T=4} indicates the number of households with complete panel information, \textsf{T=3} indicates number of households observed three times, \textsf{T=2} indicates the number of households observed twice. \textsf{N} indicates total number of observations used in ANCOVA estimation, or \textsf{N$=$1$\times$(T=2)+2$\times$(T=3)+3$\times$(T=4)}.  \textsf{Upfront} is an indicator variable of the arm with an upfront large disbursement, \textsf{WithGrace} is an indicator variable of the arm with a grace period, \textsf{InKind} is an indicator variable of the arm which lends a heifer. \textsf{rd2, rd3, rd4} are dummy variables for second, third, and fourth round of survey. Net assets uses only assets observed for all 4 rounds in household assets. Household assets do not include livestock. \textsf{OwnCattle0} is an indicator if a household owned cattle at the baseline. \textsf{AdiCattle0} is an indicator if a household engaged in the cattle lease-in contract at the baseline.  \\
& 2. & $P$ values in percentages in parenthesises. Standard errors are clustered at group (village) level.
%${}^{***}$, ${}^{**}$, ${}^{*}$ indicate statistical significance at 1\%, 5\%, 10\%, respetively. Standard errors are clustered at group (village) level.
\end{tabular}
\end{minipage}


\vspace{-1cm}\hspace{-1cm}\begin{minipage}[t]{14cm}
\hfil\textsc{\normalsize Table \refstepcounter{table}\thetable: ANCOVA estimation of net assets by arm, poverty status, and period, cattle rearing experiences\label{tab ANCOVA narrow net assets Experience timevarying poverty status}}\\
\setlength{\tabcolsep}{1pt}
\setlength{\baselineskip}{8pt}
\renewcommand{\arraystretch}{.52}
\hfil\begin{tikzpicture}
\node (tbl) {\input{c:/data/GUK/analysis/save/EstimationMemo/NetAssetsExperienceTimeVaryingPovertyStatusANCOVAEstimationResults.tex}};
%\input{c:/dropbox/data/ramadan/save/tablecolortemplate.tex}
\end{tikzpicture}\\
\renewcommand{\arraystretch}{.8}
\setlength{\tabcolsep}{1pt}
\begin{tabular}{>{\hfill\scriptsize}p{1cm}<{}>{\hfill\scriptsize}p{.25cm}<{}>{\scriptsize}p{12cm}<{\hfill}}
Source:& \multicolumn{2}{l}{\scriptsize Estimated with GUK administrative and survey data.}\\
Notes: & 1. & ANCOVA estimates using administrative and survey data. Post treatment regressands are regressed on categorical variables, pre-treatment regressand and other covariates. \textsf{FloodInRd1} and \textsf{HeadLiterate0} are indicator variables for the presence of self reported damage by a flood at the baseline, and literacy of household head, respectively. \textsf{HHsize0} is household size at the baseline. We annotate the number of periods that a household is observed with \textsf{T}. The total number of households is shown for each values of \textsf{T}. \textsf{T=4} indicates the number of households with complete panel information, \textsf{T=3} indicates number of households observed three times, \textsf{T=2} indicates the number of households observed twice. \textsf{N} indicates total number of observations used in ANCOVA estimation, or \textsf{N$=$1$\times$(T=2)+2$\times$(T=3)+3$\times$(T=4)}.  \textsf{UltraPoor} is an indicator variable if the household is classified as the ultra poor. \textsf{Large}, \textsf{LargeGrace}, \textsf{Cattle} are indicator variables of the \textsf{large}, \textsf{large grace}, and \textsf{cattle} arms, respectively. The default arm category is \textsf{traditional} arm. \textsf{rd2, rd3, rd4} are dummy variables for second, third, and fourth round of survey. Net assets uses only assets observed for all 4 rounds in household assets. Household assets do not include livestock. \textsf{OwnCattle0} is an indicator if a household owned cattle at the baseline. \textsf{AdiCattle0} is an indicator if a household engaged in the cattle lease-in contract at the baseline.  \\
& 2. & $P$ values in percentages in parenthesises. Standard errors are clustered at group (village) level.
%${}^{***}$, ${}^{**}$, ${}^{*}$ indicate statistical significance at 1\%, 5\%, 10\%, respetively. Standard errors are clustered at group (village) level.
\end{tabular}
\end{minipage}

\vspace{-1cm}\hspace{-1cm}\begin{minipage}[t]{14cm}
\hfil\textsc{\normalsize Table \refstepcounter{table}\thetable: ANCOVA estimation of net assets by attributes, poverty status, and period, cattle rearing experiences\label{tab ANCOVA narrow net assets Experience timevarying poverty status attributes}}\\
\setlength{\tabcolsep}{1pt}
\setlength{\baselineskip}{8pt}
\renewcommand{\arraystretch}{.52}
\hfil\begin{tikzpicture}
\node (tbl) {\input{c:/data/GUK/analysis/save/EstimationMemo/NetAssetsExperienceTimeVaryingPovertyStatusAttributesANCOVAEstimationResults.tex}};
%\input{c:/dropbox/data/ramadan/save/tablecolortemplate.tex}
\end{tikzpicture}\\
\renewcommand{\arraystretch}{.8}
\setlength{\tabcolsep}{1pt}
\begin{tabular}{>{\hfill\scriptsize}p{1cm}<{}>{\hfill\scriptsize}p{.25cm}<{}>{\scriptsize}p{12cm}<{\hfill}}
Source:& \multicolumn{2}{l}{\scriptsize Estimated with GUK administrative and survey data.}\\
Notes: & 1. & ANCOVA estimates using administrative and survey data. Post treatment regressands are regressed on categorical variables, pre-treatment regressand and other covariates. \textsf{FloodInRd1} and \textsf{HeadLiterate0} are indicator variables for the presence of self reported damage by a flood at the baseline, and literacy of household head, respectively. \textsf{HHsize0} is household size at the baseline. We annotate the number of periods that a household is observed with \textsf{T}. The total number of households is shown for each values of \textsf{T}. \textsf{T=4} indicates the number of households with complete panel information, \textsf{T=3} indicates number of households observed three times, \textsf{T=2} indicates the number of households observed twice. \textsf{N} indicates total number of observations used in ANCOVA estimation, or \textsf{N$=$1$\times$(T=2)+2$\times$(T=3)+3$\times$(T=4)}.  \textsf{UltraPoor} is an indicator variable if the household is classified as the ultra poor. \textsf{Upfront} is an indicator variable of the arm with an upfront large disbursement, \textsf{WithGrace} is an indicator variable of the arm with a grace period, \textsf{InKind} is an indicator variable of the arm which lends a heifer. \textsf{rd2, rd3, rd4} are dummy variables for second, third, and fourth round of survey.  Net assets uses only assets observed for all 4 rounds in household assets. Household assets do not include livestock. \textsf{OwnCattle0} is an indicator if a household owned cattle at the baseline. \textsf{AdiCattle0} is an indicator if a household engaged in the cattle lease-in contract at the baseline.  \\
& 2. & $P$ values in percentages in parenthesises. Standard errors are clustered at group (village) level.
%${}^{***}$, ${}^{**}$, ${}^{*}$ indicate statistical significance at 1\%, 5\%, 10\%, respetively. Standard errors are clustered at group (village) level.
\end{tabular}
\end{minipage}



\vspace{-1cm}\hspace{-1cm}\begin{minipage}[t]{14cm}
\hfil\textsc{\normalsize Table \refstepcounter{table}\thetable: ANCOVA estimation of net assets, cattle rearing experiences\label{tab ANCOVA net assets experience 1}}\\
\setlength{\tabcolsep}{1pt}
\setlength{\baselineskip}{8pt}
\renewcommand{\arraystretch}{.52}
\hfil\begin{tikzpicture}
\node (tbl) {\input{c:/data/GUK/analysis/save/EstimationMemo/NetAssetsByExperience1ANCOVAEstimationResults.tex}};
%\input{c:/dropbox/data/ramadan/save/tablecolortemplate.tex}
\end{tikzpicture}\\
\end{minipage}

\addtocounter{table}{-1}
\hspace{-1cm}\begin{minipage}[t]{14cm}
\hfil\textsc{\normalsize Table \refstepcounter{table}\thetable: ANCOVA estimation of net assets, cattle rearing experiences (continued)\label{tab ANCOVA net assets experience 2}}\\
\setlength{\tabcolsep}{1pt}
\setlength{\baselineskip}{8pt}
\renewcommand{\arraystretch}{.52}
\hfil\begin{tikzpicture}
\node (tbl) {\input{c:/data/GUK/analysis/save/EstimationMemo/NetAssetsByExperience2ANCOVAEstimationResults.tex}};
%\input{c:/dropbox/data/ramadan/save/tablecolortemplate.tex}
\end{tikzpicture}\\
\renewcommand{\arraystretch}{.8}
\setlength{\tabcolsep}{1pt}
\begin{tabular}{>{\hfill\scriptsize}p{1cm}<{}>{\hfill\scriptsize}p{.25cm}<{}>{\scriptsize}p{12cm}<{\hfill}}
Source:& \multicolumn{2}{l}{\scriptsize Estimated with GUK administrative and survey data.}\\
Notes: & 1. & ANCOVA estimates using administrative and survey data. Post treatment regressands are regressed on categorical variables, pre-treatment regressand and other covariates. \textsf{FloodInRd1} and \textsf{HeadLiterate0} are indicator variables for the presence of self reported damage by a flood at the baseline, and literacy of household head, respectively. \textsf{HHsize0} is household size at the baseline. We annotate the number of periods that a household is observed with \textsf{T}. The total number of households is shown for each values of \textsf{T}. \textsf{T=4} indicates the number of households with complete panel information, \textsf{T=3} indicates number of households observed three times, \textsf{T=2} indicates the number of households observed twice. \textsf{N} indicates total number of observations used in ANCOVA estimation, or \textsf{N$=$1$\times$(T=2)+2$\times$(T=3)+3$\times$(T=4)}.  \textsf{Large}, \textsf{LargeGrace}, \textsf{Cattle} are indicator variables of the \textsf{large}, \textsf{large grace}, and \textsf{cattle} arms, respectively. The default arm category is \textsf{traditional} arm. Net assets uses only assets observed for all 4 rounds in household assets. Household assets do not include livestock. \textsf{OwnCattle0} is an indicator if a household owned cattle at the baseline. \textsf{AdiCattle0} is an indicator if a household engaged in the cattle lease-in contract at the baseline.  \\
& 2. & $P$ values in percentages in parenthesises. Standard errors are clustered at group (village) level.
%${}^{***}$, ${}^{**}$, ${}^{*}$ indicate statistical significance at 1\%, 5\%, 10\%, respetively. Standard errors are clustered at group (village) level.
\end{tabular}
\end{minipage}



\vspace{-1cm}\hspace{-1cm}\begin{minipage}[t]{14cm}
\hfil\textsc{\normalsize Table \refstepcounter{table}\thetable: ANCOVA estimation of net assets by attributes, cattle rearing experiences\label{tab ANCOVA narrow net assets Experience attributes 1}}\\
\setlength{\tabcolsep}{1pt}
\setlength{\baselineskip}{8pt}
\renewcommand{\arraystretch}{.52}
\hfil\begin{tikzpicture}
\node (tbl) {\input{c:/data/GUK/analysis/save/EstimationMemo/NetAssetsByExperience1AttributesANCOVAEstimationResults.tex}};
%\input{c:/dropbox/data/ramadan/save/tablecolortemplate.tex}
\end{tikzpicture}\\
\end{minipage}

\addtocounter{table}{-1}
\hspace{-1cm}\begin{minipage}[t]{14cm}
\hfil\textsc{\normalsize Table \refstepcounter{table}\thetable: ANCOVA estimation of net assets by attributes, cattle rearing experiences (continued)\label{tab ANCOVA narrow net assets Experience attributes 2}}\\
\setlength{\tabcolsep}{1pt}
\setlength{\baselineskip}{8pt}
\renewcommand{\arraystretch}{.52}
\hfil\begin{tikzpicture}
\node (tbl) {\input{c:/data/GUK/analysis/save/EstimationMemo/NetAssetsByExperience2AttributesANCOVAEstimationResults.tex}};
%\input{c:/dropbox/data/ramadan/save/tablecolortemplate.tex}
\end{tikzpicture}\\
\renewcommand{\arraystretch}{.8}
\setlength{\tabcolsep}{1pt}
\begin{tabular}{>{\hfill\scriptsize}p{1cm}<{}>{\hfill\scriptsize}p{.25cm}<{}>{\scriptsize}p{12cm}<{\hfill}}
Source:& \multicolumn{2}{l}{\scriptsize Estimated with GUK administrative and survey data.}\\
Notes: & 1. & ANCOVA estimates using administrative and survey data. Post treatment regressands are regressed on categorical variables, pre-treatment regressand and other covariates. \textsf{FloodInRd1} and \textsf{HeadLiterate0} are indicator variables for the presence of self reported damage by a flood at the baseline, and literacy of household head, respectively. \textsf{HHsize0} is household size at the baseline. We annotate the number of periods that a household is observed with \textsf{T}. The total number of households is shown for each values of \textsf{T}. \textsf{T=4} indicates the number of households with complete panel information, \textsf{T=3} indicates number of households observed three times, \textsf{T=2} indicates the number of households observed twice. \textsf{N} indicates total number of observations used in ANCOVA estimation, or \textsf{N$=$1$\times$(T=2)+2$\times$(T=3)+3$\times$(T=4)}.  \textsf{Upfront} is an indicator variable of the arm with an upfront large disbursement, \textsf{WithGrace} is an indicator variable of the arm with a grace period, \textsf{InKind} is an indicator variable of the arm which lends a heifer. Net assets uses only assets observed for all 4 rounds in household assets. Household assets do not include livestock. \\
& 2. & $P$ values in percentages in parenthesises. Standard errors are clustered at group (village) level.
%${}^{***}$, ${}^{**}$, ${}^{*}$ indicate statistical significance at 1\%, 5\%, 10\%, respetively. Standard errors are clustered at group (village) level.
\end{tabular}
\end{minipage}



\vspace{-1cm}\hspace{-1cm}\begin{minipage}[t]{14cm}
\hfil\textsc{\normalsize Table \refstepcounter{table}\thetable: ANCOVA estimation of net assets by period, cattle rearing experiences\label{tab ANCOVA narrow net assets Experience timevarying 1}}\\
\setlength{\tabcolsep}{1pt}
\setlength{\baselineskip}{8pt}
\renewcommand{\arraystretch}{.52}
\hfil\begin{tikzpicture}
\node (tbl) {\input{c:/data/GUK/analysis/save/EstimationMemo/NetAssetsByExperience1TimeVaryingANCOVAEstimationResults.tex}};
%\input{c:/dropbox/data/ramadan/save/tablecolortemplate.tex}
\end{tikzpicture}
\end{minipage}

\addtocounter{table}{-1}
\hspace{-1cm}\begin{minipage}[t]{14cm}
\hfil\textsc{\normalsize Table \refstepcounter{table}\thetable: ANCOVA estimation of net assets by period, cattle rearing experiences (continued) \label{tab ANCOVA narrow net assets Experience timevarying 2}}\\
\setlength{\tabcolsep}{1pt}
\setlength{\baselineskip}{8pt}
\renewcommand{\arraystretch}{.52}
\hfil\begin{tikzpicture}
\node (tbl) {\input{c:/data/GUK/analysis/save/EstimationMemo/NetAssetsByExperience2TimeVaryingANCOVAEstimationResults.tex}};
%\input{c:/dropbox/data/ramadan/save/tablecolortemplate.tex}
\end{tikzpicture}\\
\renewcommand{\arraystretch}{.8}
\setlength{\tabcolsep}{1pt}
\begin{tabular}{>{\hfill\scriptsize}p{1cm}<{}>{\hfill\scriptsize}p{.25cm}<{}>{\scriptsize}p{12cm}<{\hfill}}
Source:& \multicolumn{2}{l}{\scriptsize Estimated with GUK administrative and survey data.}\\
Notes: & 1. & ANCOVA estimates using administrative and survey data. Post treatment regressands are regressed on categorical variables, pre-treatment regressand and other covariates. \textsf{FloodInRd1} and \textsf{HeadLiterate0} are indicator variables for the presence of self reported damage by a flood at the baseline, and literacy of household head, respectively. \textsf{HHsize0} is household size at the baseline. We annotate the number of periods that a household is observed with \textsf{T}. The total number of households is shown for each values of \textsf{T}. \textsf{T=4} indicates the number of households with complete panel information, \textsf{T=3} indicates number of households observed three times, \textsf{T=2} indicates the number of households observed twice. \textsf{N} indicates total number of observations used in ANCOVA estimation, or \textsf{N$=$1$\times$(T=2)+2$\times$(T=3)+3$\times$(T=4)}.  \textsf{Large}, \textsf{LargeGrace}, \textsf{Cattle} are indicator variables of the \textsf{large}, \textsf{large grace}, and \textsf{cattle} arms, respectively. The default arm category is \textsf{traditional} arm. \textsf{rd2, rd3, rd4} are dummy variables for second, third, and fourth round of survey. Net assets uses only assets observed for all 4 rounds in household assets. Household assets do not include livestock. \\
& 2. & $P$ values in percentages in parenthesises. Standard errors are clustered at group (village) level.
%${}^{***}$, ${}^{**}$, ${}^{*}$ indicate statistical significance at 1\%, 5\%, 10\%, respetively. Standard errors are clustered at group (village) level.
\end{tabular}
\end{minipage}



\textsc{\normalsize Table \ref{tab ANCOVA narrow net assets Experience timevarying 1}} and \textsc{\normalsize Table \ref{tab ANCOVA narrow net assets Experience timevarying 2} (continued)} show estimation after dividing into three subsamples: Individuals who had a cattle lease contract (\textsf{Adi}, $n=92$) at the baseline, individuals who owned cattle at the baseline (\textsf{Own}, $n=137$), and individuals who had neither (\textsf{None}, $n=505$). The total of 734 is fewer than baseline sample size of 774 as we lost 40 observations by round 2. The number of individuals with \textsf{Adi} is small that makes the estimates inprecise. Specifications (2) - (4) show that the individuals of \textsf{Own} responded well to the non-\textsf{Traditional} lending by round 2, particularly so under \textsf{Large} and \textsf{Large grace} arms. Individuals of \textsf{None} have the smallest net asset holding under \textsf{Traditional} as indicated by the intercept terms. They have excess positive returns under all non-\textsf{Traditional} arms relative to the \textsf{Traditional} arm. Among the individuals of \textsf{None}, or who had no prior cattle rearing experience at the baseline, the \textsf{Cattle} arm gives the higher mean returns than the \textsf{Large grace} arm. As argued in the main text, it strongly suggests that the effective difference of the two arms, the managerial support program, resulted in a higher return for the \textsf{None} group.
%It is interesting that they had smaller excess returns to \textsf{Large} and \textsf{Large grace} arms relative to \textsf{Own} group, yet they had a greater excess returns to \textsf{Cattle} arm. In fact, returns to \textsf{Traditional} arm in \textsf{Own} group by round 2 is 17273.5 while the return to \textsf{Cattle} in \textsf{None} group is 10886+9666.7. 

The household size is positively correlated with the net asset values in \textsf{Adi} and \textsf{None} groups while not in \textsf{Own} group. This implies that there may be selection into cattle ownership at the baseline that requires a certain household size, either labour and/or a barn, and \textsf{Own} group may already have a way to acquire them as they become necessary. Returns to baseline net asset holding is meaningful only among the \textsf{Own} group, and estimates on other groups are less precise. \textsf{Adi} group has a large point estimate, although the $p$ value is greater than .1, which is consistent with a conjecture that the skills acquired through Adi have high returns but they are cash constrained.


\hspace{-1cm}\begin{minipage}[t]{14cm}
\hfil\textsc{\normalsize Table \refstepcounter{table}\thetable: ANCOVA estimation of narrow net assets by attributes and period, cattle rearing experiences\label{tab ANCOVA narrow net assets Experience timevarying attributes 1}}\\
\setlength{\tabcolsep}{1pt}
\setlength{\baselineskip}{8pt}
\renewcommand{\arraystretch}{.52}
\hfil\begin{tikzpicture}
\node (tbl) {\input{c:/data/GUK/analysis/save/EstimationMemo/NetAssetsByExperience1TimeVaryingAttributesANCOVAEstimationResults.tex}};
%\input{c:/dropbox/data/ramadan/save/tablecolortemplate.tex}
\end{tikzpicture}\\
\end{minipage}

\addtocounter{table}{-1}
\vspace{-1cm}\hspace{-1cm}\begin{minipage}[t]{14cm}
\hfil\textsc{\normalsize Table \refstepcounter{table}\thetable: ANCOVA estimation of net assets by attributes and period, cattle rearing experiences (continued)\label{tab ANCOVA narrow net assets Experience timevarying attributes 2}}\\
\setlength{\tabcolsep}{1pt}
\setlength{\baselineskip}{8pt}
\renewcommand{\arraystretch}{.52}
\hfil\begin{tikzpicture}
\node (tbl) {\input{c:/data/GUK/analysis/save/EstimationMemo/NetAssetsByExperience2TimeVaryingAttributesANCOVAEstimationResults.tex}};
%\input{c:/dropbox/data/ramadan/save/tablecolortemplate.tex}
\end{tikzpicture}\\
\renewcommand{\arraystretch}{.8}
\setlength{\tabcolsep}{1pt}
\begin{tabular}{>{\hfill\scriptsize}p{1cm}<{}>{\hfill\scriptsize}p{.25cm}<{}>{\scriptsize}p{12cm}<{\hfill}}
Source:& \multicolumn{2}{l}{\scriptsize Estimated with GUK administrative and survey data.}\\
Notes: & 1. & ANCOVA estimates using administrative and survey data. Post treatment regressands are regressed on categorical variables, pre-treatment regressand and other covariates. \textsf{FloodInRd1} and \textsf{HeadLiterate0} are indicator variables for the presence of self reported damage by a flood at the baseline, and literacy of household head, respectively. \textsf{HHsize0} is household size at the baseline. We annotate the number of periods that a household is observed with \textsf{T}. The total number of households is shown for each values of \textsf{T}. \textsf{T=4} indicates the number of households with complete panel information, \textsf{T=3} indicates number of households observed three times, \textsf{T=2} indicates the number of households observed twice. \textsf{N} indicates total number of observations used in ANCOVA estimation, or \textsf{N$=$1$\times$(T=2)+2$\times$(T=3)+3$\times$(T=4)}.  \textsf{Upfront} is an indicator variable of the arm with an upfront large disbursement, \textsf{WithGrace} is an indicator variable of the arm with a grace period, \textsf{InKind} is an indicator variable of the arm which lends a heifer. \textsf{rd2, rd3, rd4} are dummy variables for second, third, and fourth round of survey. Net assets uses only assets observed for all 4 rounds in household assets. Household assets do not include livestock. \\
& 2. & $P$ values in percentages in parenthesises. Standard errors are clustered at group (village) level.
%${}^{***}$, ${}^{**}$, ${}^{*}$ indicate statistical significance at 1\%, 5\%, 10\%, respetively. Standard errors are clustered at group (village) level.
\end{tabular}
\end{minipage}


\vspace{-1cm}\hspace{-1cm}\begin{minipage}[t]{14cm}
\hfil\textsc{\normalsize Table \refstepcounter{table}\thetable: ANCOVA estimation of net assets by arm, poverty status, and period, cattle rearing experiences\label{tab ANCOVA narrow net assets Experience timevarying poverty status 1}}\\
\setlength{\tabcolsep}{1pt}
\setlength{\baselineskip}{8pt}
\renewcommand{\arraystretch}{.52}
\hfil\begin{tikzpicture}
\node (tbl) {\input{c:/data/GUK/analysis/save/EstimationMemo/NetAssetsByExperience1TimeVaryingPovertyStatusANCOVAEstimationResults.tex}};
%\input{c:/dropbox/data/ramadan/save/tablecolortemplate.tex}
\end{tikzpicture}\\
\end{minipage}

\addtocounter{table}{-1}
\vspace{-1cm}\hspace{-1cm}\begin{minipage}[t]{14cm}
\hfil\textsc{\normalsize Table \refstepcounter{table}\thetable: ANCOVA estimation of net assets by arm, poverty status, and period, cattle rearing experiences (continued)\label{tab ANCOVA narrow net assets Experience timevarying poverty status 2}}\\
\setlength{\tabcolsep}{1pt}
\setlength{\baselineskip}{8pt}
\renewcommand{\arraystretch}{.52}
\hfil\begin{tikzpicture}
\node (tbl) {\input{c:/data/GUK/analysis/save/EstimationMemo/NetAssetsByExperience2TimeVaryingPovertyStatusANCOVAEstimationResults.tex}};
%\input{c:/dropbox/data/ramadan/save/tablecolortemplate.tex}
\end{tikzpicture}\\
\renewcommand{\arraystretch}{.8}
\setlength{\tabcolsep}{1pt}
\begin{tabular}{>{\hfill\scriptsize}p{1cm}<{}>{\hfill\scriptsize}p{.25cm}<{}>{\scriptsize}p{12cm}<{\hfill}}
Source:& \multicolumn{2}{l}{\scriptsize Estimated with GUK administrative and survey data.}\\
Notes: & 1. & ANCOVA estimates using administrative and survey data. Post treatment regressands are regressed on categorical variables, pre-treatment regressand and other covariates. \textsf{FloodInRd1} and \textsf{HeadLiterate0} are indicator variables for the presence of self reported damage by a flood at the baseline, and literacy of household head, respectively. \textsf{HHsize0} is household size at the baseline. We annotate the number of periods that a household is observed with \textsf{T}. The total number of households is shown for each values of \textsf{T}. \textsf{T=4} indicates the number of households with complete panel information, \textsf{T=3} indicates number of households observed three times, \textsf{T=2} indicates the number of households observed twice. \textsf{N} indicates total number of observations used in ANCOVA estimation, or \textsf{N$=$1$\times$(T=2)+2$\times$(T=3)+3$\times$(T=4)}.  \textsf{UltraPoor} is an indicator variable if the household is classified as the ultra poor. \textsf{Large}, \textsf{LargeGrace}, \textsf{Cattle} are indicator variables of the \textsf{large}, \textsf{large grace}, and \textsf{cattle} arms, respectively. The default arm category is \textsf{traditional} arm. \textsf{rd2, rd3, rd4} are dummy variables for second, third, and fourth round of survey. Net assets uses only assets observed for all 4 rounds in household assets. Household assets do not include livestock. \\
& 2. & $P$ values in percentages in parenthesises. Standard errors are clustered at group (village) level.
%${}^{***}$, ${}^{**}$, ${}^{*}$ indicate statistical significance at 1\%, 5\%, 10\%, respetively. Standard errors are clustered at group (village) level.
\end{tabular}
\end{minipage}


\vspace{-1cm}\hspace{-1cm}\begin{minipage}[t]{14cm}
\hfil\textsc{\normalsize Table \refstepcounter{table}\thetable: ANCOVA estimation of net assets by attributes, poverty status, and period, cattle rearing experiences\label{tab ANCOVA narrow net assets Experience timevarying poverty status attributes 1}}\\
\setlength{\tabcolsep}{1pt}
\setlength{\baselineskip}{8pt}
\renewcommand{\arraystretch}{.52}
\hfil\begin{tikzpicture}
\node (tbl) {\input{c:/data/GUK/analysis/save/EstimationMemo/NetAssetsByExperience1TimeVaryingPovertyStatusAttributesANCOVAEstimationResults.tex}};
%\input{c:/dropbox/data/ramadan/save/tablecolortemplate.tex}
\end{tikzpicture}\\
\end{minipage}

\addtocounter{table}{-1}
\vspace{-1cm}\hspace{-1cm}\begin{minipage}[t]{14cm}
\hfil\textsc{\normalsize Table \refstepcounter{table}\thetable: ANCOVA estimation of net assets by attributes, poverty status, and period, cattle rearing experiences (continued) \label{tab ANCOVA narrow net assets Experience timevarying poverty status attributes 2}}\\
\setlength{\tabcolsep}{1pt}
\setlength{\baselineskip}{8pt}
\renewcommand{\arraystretch}{.52}
\hfil\begin{tikzpicture}
\node (tbl) {\input{c:/data/GUK/analysis/save/EstimationMemo/NetAssetsByExperience2TimeVaryingPovertyStatusAttributesANCOVAEstimationResults.tex}};
%\input{c:/dropbox/data/ramadan/save/tablecolortemplate.tex}
\end{tikzpicture}\\
\renewcommand{\arraystretch}{.8}
\setlength{\tabcolsep}{1pt}
\begin{tabular}{>{\hfill\scriptsize}p{1cm}<{}>{\hfill\scriptsize}p{.25cm}<{}>{\scriptsize}p{12cm}<{\hfill}}
Source:& \multicolumn{2}{l}{\scriptsize Estimated with GUK administrative and survey data.}\\
Notes: & 1. & ANCOVA estimates using administrative and survey data. Post treatment regressands are regressed on categorical variables, pre-treatment regressand and other covariates. \textsf{FloodInRd1} and \textsf{HeadLiterate0} are indicator variables for the presence of self reported damage by a flood at the baseline, and literacy of household head, respectively. \textsf{HHsize0} is household size at the baseline. We annotate the number of periods that a household is observed with \textsf{T}. The total number of households is shown for each values of \textsf{T}. \textsf{T=4} indicates the number of households with complete panel information, \textsf{T=3} indicates number of households observed three times, \textsf{T=2} indicates the number of households observed twice. \textsf{N} indicates total number of observations used in ANCOVA estimation, or \textsf{N$=$1$\times$(T=2)+2$\times$(T=3)+3$\times$(T=4)}.  \textsf{UltraPoor} is an indicator variable if the household is classified as the ultra poor. \textsf{Upfront} is an indicator variable of the arm with an upfront large disbursement, \textsf{WithGrace} is an indicator variable of the arm with a grace period, \textsf{InKind} is an indicator variable of the arm which lends a heifer. \textsf{rd2, rd3, rd4} are dummy variables for second, third, and fourth round of survey. Net assets uses only assets observed for all 4 rounds in household assets. Household assets do not include livestock. \\
& 2. & $P$ values in percentages in parenthesises. Standard errors are clustered at group (village) level.
%${}^{***}$, ${}^{**}$, ${}^{*}$ indicate statistical significance at 1\%, 5\%, 10\%, respetively. Standard errors are clustered at group (village) level.
\end{tabular}
\end{minipage}



\subsubsection{Livestock, experienced vs. inexperienced}


\begin{Schunk}
\begin{Soutput}
             AttritIn
Arm             2   3   4   9 Sum
  traditional   7   4  20 144 175
  large         5   2   1 192 200
  large grace  12   3   3 171 189
  cattle        5   5  13 176 199
  Sum          29  14  37 683 763
     NumCows
tee      0    1    2    3    4    5    6    7    8    9 <NA>  Sum
  2     15  309  153   40   11    1    2    0    1    1  197  730
  3      5  337  175   40   16    1    2    2    1    0  110  689
  4      4  218  201   54   11    4    2    0    1    1   86  582
  Sum   24  864  529  134   38    6    6    2    3    2  393 2001
\end{Soutput}
\end{Schunk}
%  source(paste0(pathprogram, "ReadTrimLivestockByExperienceANCOVA.R"))





\begin{Schunk}
\begin{Soutput}


[1]
 ~  + dummyLarge + dummyLargeGrace + dummyCattle

 + TotalImputedValue0
 + dummyAdiCattle0
FloodInRd1 + HHsize0 + HeadLiteracy0 + NA

dummyAdiCattle0.Large + dummyAdiCattle0.LargeGrace + dummyAdiCattle0.Cattle + NA

 

[2]
 ~  + dummyUltraPoor + dummyLargeSize + dummyWithGrace
 + dummyWithGrace + dummyInKind + UDdummyUltraPoor
 + dummyLargeSize.UltraPoor + UDdummyLargeSize.UltraPoor + dummyWithGrace.UltraPoor
 + UDdummyWithGrace.UltraPoor + dummyInKind.UltraPoor + UDdummyInKind.UltraPoor
 + 

 + TotalImputedValue0
 + dummyAdiCattle0
FloodInRd1 + HHsize0 + HeadLiteracy0 + NA

dummyAdiCattle0.LargeSize + dummyAdiCattle0.WithGrace + dummyAdiCattle0.InKind + NA

 

[3]
 ~  + dummyLargeSize + dummyWithGrace + dummyInKind

 + TotalImputedValue0
 + dummyAdiCattle0
FloodInRd1 + HHsize0 + HeadLiteracy0 + NA

dummyAdiCattle0.LargeSize + dummyAdiCattle0.WithGrace + dummyAdiCattle0.InKind + NA

 

[4]
 ~  + Time.3 + Time.4 + dummyLarge
 + dummyLarge + dummyLargeGrace + dummyCattle
 + dummyLarge.Time3 + dummyLargeGrace.Time3 + dummyCattle.Time3
 + dummyLarge.Time4 + dummyLargeGrace.Time4 + dummyCattle.Time4
 + 

 + TotalImputedValue0
 + dummyAdiCattle0
FloodInRd1 + HHsize0 + HeadLiteracy0 + NA

dummyAdiCattle0.Large + dummyAdiCattle0.Time3 + dummyAdiCattle0.Large.Time3 + dummyAdiCattle0.Time4
 + dummyAdiCattle0.Time4 + dummyAdiCattle0.Large.Time4 + dummyAdiCattle0.LargeGrace
 + dummyAdiCattle0.LargeGrace.Time3 + dummyAdiCattle0.LargeGrace.Time4 + dummyAdiCattle0.Cattle
 + dummyAdiCattle0.Cattle.Time3 + dummyAdiCattle0.Cattle.Time4

 

[5]
 ~  + Time.3 + Time.4 + dummyLargeSize
 + dummyLargeSize + dummyWithGrace + dummyInKind
 + dummyLargeSize.Time3 + dummyWithGrace.Time3 + dummyInKind.Time3
 + dummyLargeSize.Time4 + dummyWithGrace.Time4 + dummyInKind.Time4
 + 

 + TotalImputedValue0
 + dummyAdiCattle0
FloodInRd1 + HHsize0 + HeadLiteracy0 + NA

dummyAdiCattle0.Time3 + dummyAdiCattle0.Time4 + dummyAdiCattle0.LargeSize + dummyAdiCattle0.LargeSize.Time3
 + dummyAdiCattle0.LargeSize.Time3 + dummyAdiCattle0.LargeSize.Time4 + dummyAdiCattle0.WithGrace
 + dummyAdiCattle0.WithGrace.Time3 + dummyAdiCattle0.WithGrace.Time4 + dummyAdiCattle0.InKind
 + dummyAdiCattle0.InKind.Time3 + dummyAdiCattle0.InKind.Time4

 

[6]
 ~  + Time.3 + Time.4 + dummyLarge
 + dummyLarge + dummyLargeGrace + dummyCattle
 + dummyUltraPoor + dummyLarge.Time3 + dummyLargeGrace.Time3
 + dummyCattle.Time3 + dummyUltraPoor.Time3 + dummyLarge.Time4
 + dummyLargeGrace.Time4 + dummyCattle.Time4 + dummyUltraPoor.Time4
 + dummyLarge.UltraPoor + dummyLargeGrace.UltraPoor + dummyCattle.UltraPoor
 + dummyLarge.UltraPoor.Time3 + dummyLarge.UltraPoor.Time4 + dummyLargeGrace.UltraPoor.Time3
 + dummyLargeGrace.UltraPoor.Time4 + dummyCattle.UltraPoor.Time3 + dummyCattle.UltraPoor.Time4
 + 

 + TotalImputedValue0
 + dummyAdiCattle0
FloodInRd1 + HHsize0 + HeadLiteracy0 + NA

dummyAdiCattle0.Large + dummyAdiCattle0.Time3 + dummyAdiCattle0.Large.Time3 + dummyAdiCattle0.Time4
 + dummyAdiCattle0.Time4 + dummyAdiCattle0.Large.Time4 + dummyAdiCattle0.LargeGrace
 + dummyAdiCattle0.LargeGrace.Time3 + dummyAdiCattle0.LargeGrace.Time4 + dummyAdiCattle0.Cattle
 + dummyAdiCattle0.Cattle.Time3 + dummyAdiCattle0.Cattle.Time4

 

[7]
 ~  + Time.3 + Time.4 + dummyUltraPoor
 + dummyUltraPoor + dummyLargeSize + dummyWithGrace
 + dummyInKind + dummyUltraPoor.Time3 + dummyLargeSize.Time3
 + dummyWithGrace.Time3 + dummyInKind.Time3 + dummyUltraPoor.Time4
 + dummyLargeSize.Time4 + dummyWithGrace.Time4 + dummyInKind.Time4
 + dummyLargeSize.UltraPoor + dummyWithGrace.UltraPoor + dummyInKind.UltraPoor
 + dummyLargeSize.UltraPoor.Time3 + dummyLargeSize.UltraPoor.Time4 + dummyWithGrace.UltraPoor.Time3
 + dummyWithGrace.UltraPoor.Time4 + dummyInKind.UltraPoor.Time3 + dummyInKind.UltraPoor.Time4
 + 

 + TotalImputedValue0
 + dummyAdiCattle0
FloodInRd1 + HHsize0 + HeadLiteracy0 + NA

dummyAdiCattle0.Time3 + dummyAdiCattle0.Time4 + dummyAdiCattle0.LargeSize + dummyAdiCattle0.LargeSize.Time3
 + dummyAdiCattle0.LargeSize.Time3 + dummyAdiCattle0.LargeSize.Time4 + dummyAdiCattle0.WithGrace
 + dummyAdiCattle0.WithGrace.Time3 + dummyAdiCattle0.WithGrace.Time4 + dummyAdiCattle0.InKind
 + dummyAdiCattle0.InKind.Time3 + dummyAdiCattle0.InKind.Time4

 
\end{Soutput}
\end{Schunk}





\hspace{-1cm}\begin{minipage}[t]{14cm}
\hfil\textsc{\normalsize Table \refstepcounter{table}\thetable: ANCOVA estimation of livestock values, cattle rearing experiences\label{tab ANCOVA net assets experience}}\\
\setlength{\tabcolsep}{1pt}
\setlength{\baselineskip}{8pt}
\renewcommand{\arraystretch}{.52}
\hfil\begin{tikzpicture}
\node (tbl) {\input{c:/data/GUK/analysis/save/EstimationMemo/LivestockExperienceANCOVAEstimationResults.tex}};
%\input{c:/dropbox/data/ramadan/save/tablecolortemplate.tex}
\end{tikzpicture}\\
\renewcommand{\arraystretch}{.8}
\setlength{\tabcolsep}{1pt}
\begin{tabular}{>{\hfill\scriptsize}p{1cm}<{}>{\hfill\scriptsize}p{.25cm}<{}>{\scriptsize}p{12cm}<{\hfill}}
Source:& \multicolumn{2}{l}{\scriptsize Estimated with GUK administrative and survey data.}\\
Notes: & 1. & ANCOVA estimates using administrative and survey data. Post treatment regressands are regressed on categorical variables, pre-treatment regressand and other covariates. \textsf{FloodInRd1} and \textsf{HeadLiterate0} are indicator variables for the presence of self reported damage by a flood at the baseline, and literacy of household head, respectively. \textsf{HHsize0} is household size at the baseline. We annotate the number of periods that a household is observed with \textsf{T}. The total number of households is shown for each values of \textsf{T}. \textsf{T=4} indicates the number of households with complete panel information, \textsf{T=3} indicates number of households observed three times, \textsf{T=2} indicates the number of households observed twice. \textsf{N} indicates total number of observations used in ANCOVA estimation, or \textsf{N$=$1$\times$(T=2)+2$\times$(T=3)+3$\times$(T=4)}.  \textsf{Large}, \textsf{LargeGrace}, \textsf{Cattle} are indicator variables of the \textsf{large}, \textsf{large grace}, and \textsf{cattle} arms, respectively. The default arm category is \textsf{traditional} arm. Net assets uses only assets observed for all 4 rounds in household assets. Household assets do not include livestock. \textsf{OwnCattle0} is an indicator if a household owned cattle at the baseline. \textsf{AdiCattle0} is an indicator if a household engaged in the cattle lease-in contract at the baseline.  \\
& 2. & $P$ values in percentages in parenthesises. Standard errors are clustered at group (village) level.
%${}^{***}$, ${}^{**}$, ${}^{*}$ indicate statistical significance at 1\%, 5\%, 10\%, respetively. Standard errors are clustered at group (village) level.
\end{tabular}
\end{minipage}

\vspace{-1cm}\hspace{-1cm}\begin{minipage}[t]{14cm}
\hfil\textsc{\normalsize Table \refstepcounter{table}\thetable: ANCOVA estimation of livestock values by attributes, cattle rearing experiences\label{tab ANCOVA Livestock Experience attributes}}\\
\setlength{\tabcolsep}{1pt}
\setlength{\baselineskip}{8pt}
\renewcommand{\arraystretch}{.52}
\hfil\begin{tikzpicture}
\node (tbl) {\input{c:/data/GUK/analysis/save/EstimationMemo/LivestockExperienceAttributesANCOVAEstimationResults.tex}};
%\input{c:/dropbox/data/ramadan/save/tablecolortemplate.tex}
\end{tikzpicture}\\
\renewcommand{\arraystretch}{.8}
\setlength{\tabcolsep}{1pt}
\begin{tabular}{>{\hfill\scriptsize}p{1cm}<{}>{\hfill\scriptsize}p{.25cm}<{}>{\scriptsize}p{12cm}<{\hfill}}
Source:& \multicolumn{2}{l}{\scriptsize Estimated with GUK administrative and survey data.}\\
Notes: & 1. & ANCOVA estimates using administrative and survey data. Post treatment regressands are regressed on categorical variables, pre-treatment regressand and other covariates. \textsf{FloodInRd1} and \textsf{HeadLiterate0} are indicator variables for the presence of self reported damage by a flood at the baseline, and literacy of household head, respectively. \textsf{HHsize0} is household size at the baseline. We annotate the number of periods that a household is observed with \textsf{T}. The total number of households is shown for each values of \textsf{T}. \textsf{T=4} indicates the number of households with complete panel information, \textsf{T=3} indicates number of households observed three times, \textsf{T=2} indicates the number of households observed twice. \textsf{N} indicates total number of observations used in ANCOVA estimation, or \textsf{N$=$1$\times$(T=2)+2$\times$(T=3)+3$\times$(T=4)}.  \textsf{Upfront} is an indicator variable of the arm with an upfront large disbursement, \textsf{WithGrace} is an indicator variable of the arm with a grace period, \textsf{InKind} is an indicator variable of the arm which lends a heifer. Net assets uses only assets observed for all 4 rounds in household assets. Household assets do not include livestock. \textsf{OwnCattle0} is an indicator if a household owned cattle at the baseline. \textsf{AdiCattle0} is an indicator if a household engaged in the cattle lease-in contract at the baseline.  \\
& 2. & $P$ values in percentages in parenthesises. Standard errors are clustered at group (village) level.
%${}^{***}$, ${}^{**}$, ${}^{*}$ indicate statistical significance at 1\%, 5\%, 10\%, respetively. Standard errors are clustered at group (village) level.
\end{tabular}
\end{minipage}

\vspace{-1cm}\hspace{-1cm}\begin{minipage}[t]{14cm}
\hfil\textsc{\normalsize Table \refstepcounter{table}\thetable: ANCOVA estimation of livestock values by period, cattle rearing experiences\label{tab ANCOVA Livestock Experience timevarying}}\\
\setlength{\tabcolsep}{1pt}
\setlength{\baselineskip}{8pt}
\renewcommand{\arraystretch}{.52}
\hfil\begin{tikzpicture}
\node (tbl) {\input{c:/data/GUK/analysis/save/EstimationMemo/LivestockExperienceTimeVaryingANCOVAEstimationResults.tex}};
%\input{c:/dropbox/data/ramadan/save/tablecolortemplate.tex}
\end{tikzpicture}\\
\renewcommand{\arraystretch}{.8}
\setlength{\tabcolsep}{1pt}
\begin{tabular}{>{\hfill\scriptsize}p{1cm}<{}>{\hfill\scriptsize}p{.25cm}<{}>{\scriptsize}p{12cm}<{\hfill}}
Source:& \multicolumn{2}{l}{\scriptsize Estimated with GUK administrative and survey data.}\\
Notes: & 1. & ANCOVA estimates using administrative and survey data. Post treatment regressands are regressed on categorical variables, pre-treatment regressand and other covariates. \textsf{FloodInRd1} and \textsf{HeadLiterate0} are indicator variables for the presence of self reported damage by a flood at the baseline, and literacy of household head, respectively. \textsf{HHsize0} is household size at the baseline. We annotate the number of periods that a household is observed with \textsf{T}. The total number of households is shown for each values of \textsf{T}. \textsf{T=4} indicates the number of households with complete panel information, \textsf{T=3} indicates number of households observed three times, \textsf{T=2} indicates the number of households observed twice. \textsf{N} indicates total number of observations used in ANCOVA estimation, or \textsf{N$=$1$\times$(T=2)+2$\times$(T=3)+3$\times$(T=4)}.  \textsf{rd2, rd3, rd4} are dummy variables for second, third, and fourth round of survey. Net assets uses only assets observed for all 4 rounds in household assets. Household assets do not include livestock. \textsf{OwnCattle0} is an indicator if a household owned cattle at the baseline. \textsf{AdiCattle0} is an indicator if a household engaged in the cattle lease-in contract at the baseline.  \\
& 2. & $P$ values in percentages in parenthesises. Standard errors are clustered at group (village) level.
%${}^{***}$, ${}^{**}$, ${}^{*}$ indicate statistical significance at 1\%, 5\%, 10\%, respetively. Standard errors are clustered at group (village) level.
\end{tabular}
\end{minipage}


\vspace{0cm}\hspace{-1cm}\begin{minipage}[t]{14cm}
\hfil\textsc{\normalsize Table \refstepcounter{table}\thetable: ANCOVA estimation of livestock values by period, attributes, cattle rearing experiences\label{tab ANCOVA Livestock Experience attributes timevarying 1}}\\
\setlength{\tabcolsep}{1pt}
\setlength{\baselineskip}{8pt}
\renewcommand{\arraystretch}{.52}
\hfil\begin{tikzpicture}
\node (tbl) {\input{c:/data/GUK/analysis/save/EstimationMemo/LivestockExperienceTimeVaryingAttributesANCOVAEstimationResults.tex}};
%\input{c:/dropbox/data/ramadan/save/tablecolortemplate.tex}
\end{tikzpicture}\\
\renewcommand{\arraystretch}{.8}
\setlength{\tabcolsep}{1pt}
\begin{tabular}{>{\hfill\scriptsize}p{1cm}<{}>{\hfill\scriptsize}p{.25cm}<{}>{\scriptsize}p{12cm}<{\hfill}}
%${}^{***}$, ${}^{**}$, ${}^{*}$ indicate statistical significance at 1\%, 5\%, 10\%, respetively. Standard errors are clustered at group (village) level.
Source:& \multicolumn{2}{l}{\scriptsize Estimated with GUK administrative and survey data.}\\
Notes: & 1. & ANCOVA estimates using administrative and survey data. Post treatment regressands are regressed on categorical variables, pre-treatment regressand and other covariates. \textsf{FloodInRd1} and \textsf{HeadLiterate0} are indicator variables for the presence of self reported damage by a flood at the baseline, and literacy of household head, respectively. \textsf{HHsize0} is household size at the baseline. We annotate the number of periods that a household is observed with \textsf{T}. The total number of households is shown for each values of \textsf{T}. \textsf{T=4} indicates the number of households with complete panel information, \textsf{T=3} indicates number of households observed three times, \textsf{T=2} indicates the number of households observed twice. \textsf{N} indicates total number of observations used in ANCOVA estimation, or \textsf{N$=$1$\times$(T=2)+2$\times$(T=3)+3$\times$(T=4)}.  \textsf{Upfront} is an indicator variable of the arm with an upfront large disbursement, \textsf{WithGrace} is an indicator variable of the arm with a grace period, \textsf{InKind} is an indicator variable of the arm which lends a heifer. \textsf{rd2, rd3, rd4} are dummy variables for second, third, and fourth round of survey. Net assets uses only assets observed for all 4 rounds in household assets. Household assets do not include livestock. \textsf{OwnCattle0} is an indicator if a household owned cattle at the baseline. \textsf{AdiCattle0} is an indicator if a household engaged in the cattle lease-in contract at the baseline.  \\
& 2. & $P$ values in percentages in parenthesises. Standard errors are clustered at group (village) level.
\end{tabular}
\end{minipage}



\vspace{-1cm}\hspace{-1cm}\begin{minipage}[t]{14cm}
\hfil\textsc{\normalsize Table \refstepcounter{table}\thetable: ANCOVA estimation of livestock values by arm, poverty status, and period, cattle rearing experiences\label{tab ANCOVA Livestock Experience timevarying poverty status}}\\
\setlength{\tabcolsep}{1pt}
\setlength{\baselineskip}{8pt}
\renewcommand{\arraystretch}{.52}
\hfil\begin{tikzpicture}
\node (tbl) {\input{c:/data/GUK/analysis/save/EstimationMemo/LivestockExperienceTimeVaryingPovertyStatusANCOVAEstimationResults.tex}};
%\input{c:/dropbox/data/ramadan/save/tablecolortemplate.tex}
\end{tikzpicture}\\
\renewcommand{\arraystretch}{.8}
\setlength{\tabcolsep}{1pt}
\begin{tabular}{>{\hfill\scriptsize}p{1cm}<{}>{\hfill\scriptsize}p{.25cm}<{}>{\scriptsize}p{12cm}<{\hfill}}
Source:& \multicolumn{2}{l}{\scriptsize Estimated with GUK administrative and survey data.}\\
Notes: & 1. & ANCOVA estimates using administrative and survey data. Post treatment regressands are regressed on categorical variables, pre-treatment regressand and other covariates. \textsf{FloodInRd1} and \textsf{HeadLiterate0} are indicator variables for the presence of self reported damage by a flood at the baseline, and literacy of household head, respectively. \textsf{HHsize0} is household size at the baseline. We annotate the number of periods that a household is observed with \textsf{T}. The total number of households is shown for each values of \textsf{T}. \textsf{T=4} indicates the number of households with complete panel information, \textsf{T=3} indicates number of households observed three times, \textsf{T=2} indicates the number of households observed twice. \textsf{N} indicates total number of observations used in ANCOVA estimation, or \textsf{N$=$1$\times$(T=2)+2$\times$(T=3)+3$\times$(T=4)}.  \textsf{UltraPoor} is an indicator variable if the household is classified as the ultra poor. \textsf{Large}, \textsf{LargeGrace}, \textsf{Cattle} are indicator variables of the \textsf{large}, \textsf{large grace}, and \textsf{cattle} arms, respectively. The default arm category is \textsf{traditional} arm. \textsf{rd2, rd3, rd4} are dummy variables for second, third, and fourth round of survey. Net assets uses only assets observed for all 4 rounds in household assets. Household assets do not include livestock. \textsf{OwnCattle0} is an indicator if a household owned cattle at the baseline. \textsf{AdiCattle0} is an indicator if a household engaged in the cattle lease-in contract at the baseline.  \\
& 2. & $P$ values in percentages in parenthesises. Standard errors are clustered at group (village) level.
%${}^{***}$, ${}^{**}$, ${}^{*}$ indicate statistical significance at 1\%, 5\%, 10\%, respetively. Standard errors are clustered at group (village) level.
\end{tabular}
\end{minipage}

\vspace{-1cm}\hspace{-1cm}\begin{minipage}[t]{14cm}
\hfil\textsc{\normalsize Table \refstepcounter{table}\thetable: ANCOVA estimation of livestock values by attributes, poverty status, and period, cattle rearing experiences\label{tab ANCOVA Livestock Experience timevarying poverty status attributes}}\\
\setlength{\tabcolsep}{1pt}
\setlength{\baselineskip}{8pt}
\renewcommand{\arraystretch}{.52}
\hfil\begin{tikzpicture}
\node (tbl) {\input{c:/data/GUK/analysis/save/EstimationMemo/LivestockExperienceTimeVaryingPovertyStatusAttributesANCOVAEstimationResults.tex}};
%\input{c:/dropbox/data/ramadan/save/tablecolortemplate.tex}
\end{tikzpicture}\\
\renewcommand{\arraystretch}{.8}
\setlength{\tabcolsep}{1pt}
\begin{tabular}{>{\hfill\scriptsize}p{1cm}<{}>{\hfill\scriptsize}p{.25cm}<{}>{\scriptsize}p{12cm}<{\hfill}}
Source:& \multicolumn{2}{l}{\scriptsize Estimated with GUK administrative and survey data.}\\
Notes: & 1. & ANCOVA estimates using administrative and survey data. Post treatment regressands are regressed on categorical variables, pre-treatment regressand and other covariates. \textsf{FloodInRd1} and \textsf{HeadLiterate0} are indicator variables for the presence of self reported damage by a flood at the baseline, and literacy of household head, respectively. \textsf{HHsize0} is household size at the baseline. We annotate the number of periods that a household is observed with \textsf{T}. The total number of households is shown for each values of \textsf{T}. \textsf{T=4} indicates the number of households with complete panel information, \textsf{T=3} indicates number of households observed three times, \textsf{T=2} indicates the number of households observed twice. \textsf{N} indicates total number of observations used in ANCOVA estimation, or \textsf{N$=$1$\times$(T=2)+2$\times$(T=3)+3$\times$(T=4)}.  \textsf{UltraPoor} is an indicator variable if the household is classified as the ultra poor. \textsf{Upfront} is an indicator variable of the arm with an upfront large disbursement, \textsf{WithGrace} is an indicator variable of the arm with a grace period, \textsf{InKind} is an indicator variable of the arm which lends a heifer. \textsf{rd2, rd3, rd4} are dummy variables for second, third, and fourth round of survey.  Net assets uses only assets observed for all 4 rounds in household assets. Household assets do not include livestock. \textsf{OwnCattle0} is an indicator if a household owned cattle at the baseline. \textsf{AdiCattle0} is an indicator if a household engaged in the cattle lease-in contract at the baseline.  \\
& 2. & $P$ values in percentages in parenthesises. Standard errors are clustered at group (village) level.
%${}^{***}$, ${}^{**}$, ${}^{*}$ indicate statistical significance at 1\%, 5\%, 10\%, respetively. Standard errors are clustered at group (village) level.
\end{tabular}
\end{minipage}

\hspace{-1cm}\begin{minipage}[t]{14cm}
\hfil\textsc{\normalsize Table \refstepcounter{table}\thetable: ANCOVA estimation of livestock values, cattle rearing experiences\label{tab ANCOVA net assets experience 1}}\\
\setlength{\tabcolsep}{1pt}
\setlength{\baselineskip}{8pt}
\renewcommand{\arraystretch}{.52}
\hfil\begin{tikzpicture}
\node (tbl) {\input{c:/data/GUK/analysis/save/EstimationMemo/LivestockByExperience1ANCOVAEstimationResults.tex}};
%\input{c:/dropbox/data/ramadan/save/tablecolortemplate.tex}
\end{tikzpicture}\\
\end{minipage}

\addtocounter{table}{-1}
\vspace{-1cm}\hspace{-1cm}\begin{minipage}[t]{14cm}
\hfil\textsc{\normalsize Table \refstepcounter{table}\thetable: ANCOVA estimation of livestock values, cattle rearing experiences (continued)\label{tab ANCOVA net assets experience 2}}\\
\setlength{\tabcolsep}{1pt}
\setlength{\baselineskip}{8pt}
\renewcommand{\arraystretch}{.52}
\hfil\begin{tikzpicture}
\node (tbl) {\input{c:/data/GUK/analysis/save/EstimationMemo/LivestockByExperience2ANCOVAEstimationResults.tex}};
%\input{c:/dropbox/data/ramadan/save/tablecolortemplate.tex}
\end{tikzpicture}\\
\renewcommand{\arraystretch}{.8}
\setlength{\tabcolsep}{1pt}
\begin{tabular}{>{\hfill\scriptsize}p{1cm}<{}>{\hfill\scriptsize}p{.25cm}<{}>{\scriptsize}p{12cm}<{\hfill}}
Source:& \multicolumn{2}{l}{\scriptsize Estimated with GUK administrative and survey data.}\\
Notes: & 1. & ANCOVA estimates using administrative and survey data. Post treatment regressands are regressed on categorical variables, pre-treatment regressand and other covariates. \textsf{FloodInRd1} and \textsf{HeadLiterate0} are indicator variables for the presence of self reported damage by a flood at the baseline, and literacy of household head, respectively. \textsf{HHsize0} is household size at the baseline. We annotate the number of periods that a household is observed with \textsf{T}. The total number of households is shown for each values of \textsf{T}. \textsf{T=4} indicates the number of households with complete panel information, \textsf{T=3} indicates number of households observed three times, \textsf{T=2} indicates the number of households observed twice. \textsf{N} indicates total number of observations used in ANCOVA estimation, or \textsf{N$=$1$\times$(T=2)+2$\times$(T=3)+3$\times$(T=4)}.  \textsf{Large}, \textsf{LargeGrace}, \textsf{Cattle} are indicator variables of the \textsf{large}, \textsf{large grace}, and \textsf{cattle} arms, respectively. The default arm category is \textsf{traditional} arm. Narrow net assets uses only assets observed for all 4 rounds in household assets. Household assets do not include livestock. \textsf{OwnCattle0} is an indicator if a household owned cattle at the baseline. \textsf{AdiCattle0} is an indicator if a household engaged in the cattle lease-in contract at the baseline.  \textsf{Adi} and \textsf{None} subgroups do not own cattle at the baseline. We used net asset values at the baseline \textsf{NetAssets0} in place of \textsf{NumCows0} in ANCOVA estimation. \\
& 2. & $P$ values in percentages in parenthesises. Standard errors are clustered at group (village) level.
%${}^{***}$, ${}^{**}$, ${}^{*}$ indicate statistical significance at 1\%, 5\%, 10\%, respetively. Standard errors are clustered at group (village) level.
\end{tabular}
\end{minipage}



\vspace{-1cm}\hspace{-1cm}\begin{minipage}[t]{14cm}
\hfil\textsc{\normalsize Table \refstepcounter{table}\thetable: ANCOVA estimation of livestock values by attributes, cattle rearing experiences\label{tab ANCOVA livestock values Experience attributes}}\\
\setlength{\tabcolsep}{1pt}
\setlength{\baselineskip}{8pt}
\renewcommand{\arraystretch}{.52}
\hfil\begin{tikzpicture}
\node (tbl) {\input{c:/data/GUK/analysis/save/EstimationMemo/LivestockByExperience1AttributesANCOVAEstimationResults.tex}};
%\input{c:/dropbox/data/ramadan/save/tablecolortemplate.tex}
\end{tikzpicture}\\
\end{minipage}

\addtocounter{table}{-1}
\hspace{-1cm}\begin{minipage}[t]{14cm}
\hfil\textsc{\normalsize Table \refstepcounter{table}\thetable: ANCOVA estimation of livestock values by attributes, cattle rearing experiences (continued)\label{tab ANCOVA livestock values Experience attributes 2}}\\
\setlength{\tabcolsep}{1pt}
\setlength{\baselineskip}{8pt}
\renewcommand{\arraystretch}{.52}
\hfil\begin{tikzpicture}
\node (tbl) {\input{c:/data/GUK/analysis/save/EstimationMemo/LivestockByExperience2AttributesANCOVAEstimationResults.tex}};
%\input{c:/dropbox/data/ramadan/save/tablecolortemplate.tex}
\end{tikzpicture}\\
\renewcommand{\arraystretch}{.8}
\setlength{\tabcolsep}{1pt}
\begin{tabular}{>{\hfill\scriptsize}p{1cm}<{}>{\hfill\scriptsize}p{.25cm}<{}>{\scriptsize}p{12cm}<{\hfill}}
Source:& \multicolumn{2}{l}{\scriptsize Estimated with GUK administrative and survey data.}\\
Notes: & 1. & ANCOVA estimates using administrative and survey data. Post treatment regressands are regressed on categorical variables, pre-treatment regressand and other covariates. \textsf{FloodInRd1} and \textsf{HeadLiterate0} are indicator variables for the presence of self reported damage by a flood at the baseline, and literacy of household head, respectively. \textsf{HHsize0} is household size at the baseline. We annotate the number of periods that a household is observed with \textsf{T}. The total number of households is shown for each values of \textsf{T}. \textsf{T=4} indicates the number of households with complete panel information, \textsf{T=3} indicates number of households observed three times, \textsf{T=2} indicates the number of households observed twice. \textsf{N} indicates total number of observations used in ANCOVA estimation, or \textsf{N$=$1$\times$(T=2)+2$\times$(T=3)+3$\times$(T=4)}.  \textsf{Upfront} is an indicator variable of the arm with an upfront large disbursement, \textsf{WithGrace} is an indicator variable of the arm with a grace period, \textsf{InKind} is an indicator variable of the arm which lends a heifer. Narrow net assets uses only assets observed for all 4 rounds in household assets. Household assets do not include livestock. \textsf{OwnCattle0} is an indicator if a household owned cattle at the baseline. \textsf{AdiCattle0} is an indicator if a household engaged in the cattle lease-in contract at the baseline.  \textsf{Adi} and \textsf{None} subgroups do not own cattle at the baseline. We used net asset values at the baseline \textsf{NetAssets0} in place of \textsf{NumCows0} in ANCOVA estimation.\\
& 2. & $P$ values in percentages in parenthesises. Standard errors are clustered at group (village) level.
%${}^{***}$, ${}^{**}$, ${}^{*}$ indicate statistical significance at 1\%, 5\%, 10\%, respetively. Standard errors are clustered at group (village) level.
\end{tabular}
\end{minipage}



\vspace{-1cm}\hspace{-1cm}\begin{minipage}[t]{14cm}
\hfil\textsc{\normalsize Table \refstepcounter{table}\thetable: ANCOVA estimation of livestock values by period, cattle rearing experiences\label{tab ANCOVA livestock values Experience timevarying}}\\
\setlength{\tabcolsep}{1pt}
\setlength{\baselineskip}{8pt}
\renewcommand{\arraystretch}{.52}
\hfil\begin{tikzpicture}
\node (tbl) {\input{c:/data/GUK/analysis/save/EstimationMemo/LivestockByExperience1TimeVaryingANCOVAEstimationResults.tex}};
%\input{c:/dropbox/data/ramadan/save/tablecolortemplate.tex}
\end{tikzpicture}
\end{minipage}

\addtocounter{table}{-1}
\hspace{-1cm}\begin{minipage}[t]{14cm}
\hfil\textsc{\normalsize Table \refstepcounter{table}\thetable: ANCOVA estimation of livestock values by period, cattle rearing experiences (continued) \label{tab ANCOVA livestock values Experience timevarying 2}}\\
\setlength{\tabcolsep}{1pt}
\setlength{\baselineskip}{8pt}
\renewcommand{\arraystretch}{.52}
\hfil\begin{tikzpicture}
\node (tbl) {\input{c:/data/GUK/analysis/save/EstimationMemo/LivestockByExperience2TimeVaryingANCOVAEstimationResults.tex}};
%\input{c:/dropbox/data/ramadan/save/tablecolortemplate.tex}
\end{tikzpicture}\\
\renewcommand{\arraystretch}{.8}
\setlength{\tabcolsep}{1pt}
\begin{tabular}{>{\hfill\scriptsize}p{1cm}<{}>{\hfill\scriptsize}p{.25cm}<{}>{\scriptsize}p{12cm}<{\hfill}}
Source:& \multicolumn{2}{l}{\scriptsize Estimated with GUK administrative and survey data.}\\
Notes: & 1. & ANCOVA estimates using administrative and survey data. Post treatment regressands are regressed on categorical variables, pre-treatment regressand and other covariates. \textsf{FloodInRd1} and \textsf{HeadLiterate0} are indicator variables for the presence of self reported damage by a flood at the baseline, and literacy of household head, respectively. \textsf{HHsize0} is household size at the baseline. We annotate the number of periods that a household is observed with \textsf{T}. The total number of households is shown for each values of \textsf{T}. \textsf{T=4} indicates the number of households with complete panel information, \textsf{T=3} indicates number of households observed three times, \textsf{T=2} indicates the number of households observed twice. \textsf{N} indicates total number of observations used in ANCOVA estimation, or \textsf{N$=$1$\times$(T=2)+2$\times$(T=3)+3$\times$(T=4)}.  \textsf{Large}, \textsf{LargeGrace}, \textsf{Cattle} are indicator variables of the \textsf{large}, \textsf{large grace}, and \textsf{cattle} arms, respectively. The default arm category is \textsf{traditional} arm. \textsf{rd2, rd3, rd4} are dummy variables for second, third, and fourth round of survey. Narrow net assets uses only assets observed for all 4 rounds in household assets. Household assets do not include livestock. \textsf{OwnCattle0} is an indicator if a household owned cattle at the baseline. \textsf{AdiCattle0} is an indicator if a household engaged in the cattle lease-in contract at the baseline.  \textsf{Adi} and \textsf{None} subgroups do not own cattle at the baseline. We used net asset values at the baseline \textsf{NetAssets0} in place of \textsf{NumCows0} in ANCOVA estimation.\\
& 2. & $P$ values in percentages in parenthesises. Standard errors are clustered at group (village) level.
%${}^{***}$, ${}^{**}$, ${}^{*}$ indicate statistical significance at 1\%, 5\%, 10\%, respetively. Standard errors are clustered at group (village) level.
\end{tabular}
\end{minipage}




\vspace{-1cm}\hspace{-1cm}\begin{minipage}[t]{14cm}
\hfil\textsc{\normalsize Table \refstepcounter{table}\thetable: ANCOVA estimation of livestock values by attributes and period, cattle rearing experiences\label{tab ANCOVA livestock values Experience timevarying attributes}}\\
\setlength{\tabcolsep}{1pt}
\setlength{\baselineskip}{8pt}
\renewcommand{\arraystretch}{.52}
\hfil\begin{tikzpicture}
\node (tbl) {\input{c:/data/GUK/analysis/save/EstimationMemo/LivestockByExperience1TimeVaryingAttributesANCOVAEstimationResults.tex}};
%\input{c:/dropbox/data/ramadan/save/tablecolortemplate.tex}
\end{tikzpicture}\\
\end{minipage}

\addtocounter{table}{-1}
\hspace{-1cm}\begin{minipage}[t]{14cm}
\hfil\textsc{\normalsize Table \refstepcounter{table}\thetable: ANCOVA estimation of livestock values by attributes and period, cattle rearing experiences (continued)\label{tab ANCOVA livestock values Experience timevarying attributes 2}}\\
\setlength{\tabcolsep}{1pt}
\setlength{\baselineskip}{8pt}
\renewcommand{\arraystretch}{.52}
\hfil\begin{tikzpicture}
\node (tbl) {\input{c:/data/GUK/analysis/save/EstimationMemo/LivestockByExperience2TimeVaryingAttributesANCOVAEstimationResults.tex}};
%\input{c:/dropbox/data/ramadan/save/tablecolortemplate.tex}
\end{tikzpicture}\\
\renewcommand{\arraystretch}{.8}
\setlength{\tabcolsep}{1pt}
\begin{tabular}{>{\hfill\scriptsize}p{1cm}<{}>{\hfill\scriptsize}p{.25cm}<{}>{\scriptsize}p{12cm}<{\hfill}}
Source:& \multicolumn{2}{l}{\scriptsize Estimated with GUK administrative and survey data.}\\
Notes: & 1. & ANCOVA estimates using administrative and survey data. Post treatment regressands are regressed on categorical variables, pre-treatment regressand and other covariates. \textsf{FloodInRd1} and \textsf{HeadLiterate0} are indicator variables for the presence of self reported damage by a flood at the baseline, and literacy of household head, respectively. \textsf{HHsize0} is household size at the baseline. We annotate the number of periods that a household is observed with \textsf{T}. The total number of households is shown for each values of \textsf{T}. \textsf{T=4} indicates the number of households with complete panel information, \textsf{T=3} indicates number of households observed three times, \textsf{T=2} indicates the number of households observed twice. \textsf{N} indicates total number of observations used in ANCOVA estimation, or \textsf{N$=$1$\times$(T=2)+2$\times$(T=3)+3$\times$(T=4)}.  \textsf{Upfront} is an indicator variable of the arm with an upfront large disbursement, \textsf{WithGrace} is an indicator variable of the arm with a grace period, \textsf{InKind} is an indicator variable of the arm which lends a heifer. \textsf{rd2, rd3, rd4} are dummy variables for second, third, and fourth round of survey. Narrow net assets uses only assets observed for all 4 rounds in household assets. Household assets do not include livestock. \textsf{OwnCattle0} is an indicator if a household owned cattle at the baseline. \textsf{AdiCattle0} is an indicator if a household engaged in the cattle lease-in contract at the baseline.  \textsf{Adi} and \textsf{None} subgroups do not own cattle at the baseline. We used net asset values at the baseline \textsf{NetAssets0} in place of \textsf{NumCows0} in ANCOVA estimation.\\
& 2. & $P$ values in percentages in parenthesises. Standard errors are clustered at group (village) level.
%${}^{***}$, ${}^{**}$, ${}^{*}$ indicate statistical significance at 1\%, 5\%, 10\%, respetively. Standard errors are clustered at group (village) level.
\end{tabular}
\end{minipage}



\subsubsection{Cattle holding, experienced vs. inexperienced}



\begin{Schunk}
\begin{Soutput}
             AttritIn
Arm             2   3   4   9 Sum
  traditional   7   4  20 144 175
  large         5   2   1 192 200
  large grace  12   3   3 171 189
  cattle        5   5  13 176 199
  Sum          29  14  37 683 763
     NumCows
tee      0    1    2    3    4    5    6    7    8    9 <NA>  Sum
  2     15  309  153   40   11    1    2    0    1    1  197  730
  3      5  337  175   40   16    1    2    2    1    0  110  689
  4      4  218  201   54   11    4    2    0    1    1   86  582
  Sum   24  864  529  134   38    6    6    2    3    2  393 2001
\end{Soutput}
\end{Schunk}
%  source(paste0(pathprogram, "ReadTrimLivestockByExperienceANCOVA.R"))
\begin{Schunk}
\begin{Soutput}
      NumCows0
Cattle   0   1   2   3   4   5 Sum
  Adi  108   0   0   0   0   0 108
  None 484   0   0   0   0   0 484
  Own    0  99  30   5   3   1 138
  Sum  592  99  30   5   3   1 730
\end{Soutput}
\end{Schunk}




\begin{Schunk}
\begin{Soutput}


[1]
 ~  + dummyLarge + dummyLargeGrace + dummyCattle

 + NumCows0
 + dummyAdiCattle0
FloodInRd1 + HHsize0 + HeadLiteracy0 + NA

dummyAdiCattle0.Large + dummyAdiCattle0.LargeGrace + dummyAdiCattle0.Cattle + NA

 

[2]
 ~  + dummyUltraPoor + dummyLargeSize + dummyWithGrace
 + dummyWithGrace + dummyInKind + UDdummyUltraPoor
 + dummyLargeSize.UltraPoor + UDdummyLargeSize.UltraPoor + dummyWithGrace.UltraPoor
 + UDdummyWithGrace.UltraPoor + dummyInKind.UltraPoor + UDdummyInKind.UltraPoor
 + 

 + NumCows0
 + dummyAdiCattle0
FloodInRd1 + HHsize0 + HeadLiteracy0 + NA

dummyAdiCattle0.LargeSize + dummyAdiCattle0.WithGrace + dummyAdiCattle0.InKind + NA

 

[3]
 ~  + dummyLargeSize + dummyWithGrace + dummyInKind

 + NumCows0
 + dummyAdiCattle0
FloodInRd1 + HHsize0 + HeadLiteracy0 + NA

dummyAdiCattle0.LargeSize + dummyAdiCattle0.WithGrace + dummyAdiCattle0.InKind + NA

 

[4]
 ~  + Time.3 + Time.4 + dummyLarge
 + dummyLarge + dummyLargeGrace + dummyCattle
 + dummyLarge.Time3 + dummyLargeGrace.Time3 + dummyCattle.Time3
 + dummyLarge.Time4 + dummyLargeGrace.Time4 + dummyCattle.Time4
 + 

 + NumCows0
 + dummyAdiCattle0
FloodInRd1 + HHsize0 + HeadLiteracy0 + NA

dummyAdiCattle0.Large + dummyAdiCattle0.Time3 + dummyAdiCattle0.Large.Time3 + dummyAdiCattle0.Time4
 + dummyAdiCattle0.Time4 + dummyAdiCattle0.Large.Time4 + dummyAdiCattle0.LargeGrace
 + dummyAdiCattle0.LargeGrace.Time3 + dummyAdiCattle0.LargeGrace.Time4 + dummyAdiCattle0.Cattle
 + dummyAdiCattle0.Cattle.Time3 + dummyAdiCattle0.Cattle.Time4

 

[5]
 ~  + Time.3 + Time.4 + dummyLargeSize
 + dummyLargeSize + dummyWithGrace + dummyInKind
 + dummyLargeSize.Time3 + dummyWithGrace.Time3 + dummyInKind.Time3
 + dummyLargeSize.Time4 + dummyWithGrace.Time4 + dummyInKind.Time4
 + 

 + NumCows0
 + dummyAdiCattle0
FloodInRd1 + HHsize0 + HeadLiteracy0 + NA

dummyAdiCattle0.Time3 + dummyAdiCattle0.Time4 + dummyAdiCattle0.LargeSize + dummyAdiCattle0.LargeSize.Time3
 + dummyAdiCattle0.LargeSize.Time3 + dummyAdiCattle0.LargeSize.Time4 + dummyAdiCattle0.WithGrace
 + dummyAdiCattle0.WithGrace.Time3 + dummyAdiCattle0.WithGrace.Time4 + dummyAdiCattle0.InKind
 + dummyAdiCattle0.InKind.Time3 + dummyAdiCattle0.InKind.Time4

 

[6]
 ~  + Time.3 + Time.4 + dummyLarge
 + dummyLarge + dummyLargeGrace + dummyCattle
 + dummyUltraPoor + dummyLarge.Time3 + dummyLargeGrace.Time3
 + dummyCattle.Time3 + dummyUltraPoor.Time3 + dummyLarge.Time4
 + dummyLargeGrace.Time4 + dummyCattle.Time4 + dummyUltraPoor.Time4
 + dummyLarge.UltraPoor + dummyLargeGrace.UltraPoor + dummyCattle.UltraPoor
 + dummyLarge.UltraPoor.Time3 + dummyLarge.UltraPoor.Time4 + dummyLargeGrace.UltraPoor.Time3
 + dummyLargeGrace.UltraPoor.Time4 + dummyCattle.UltraPoor.Time3 + dummyCattle.UltraPoor.Time4
 + 

 + NumCows0
 + dummyAdiCattle0
FloodInRd1 + HHsize0 + HeadLiteracy0 + NA

dummyAdiCattle0.Large + dummyAdiCattle0.Time3 + dummyAdiCattle0.Large.Time3 + dummyAdiCattle0.Time4
 + dummyAdiCattle0.Time4 + dummyAdiCattle0.Large.Time4 + dummyAdiCattle0.LargeGrace
 + dummyAdiCattle0.LargeGrace.Time3 + dummyAdiCattle0.LargeGrace.Time4 + dummyAdiCattle0.Cattle
 + dummyAdiCattle0.Cattle.Time3 + dummyAdiCattle0.Cattle.Time4

 

[7]
 ~  + Time.3 + Time.4 + dummyUltraPoor
 + dummyUltraPoor + dummyLargeSize + dummyWithGrace
 + dummyInKind + dummyUltraPoor.Time3 + dummyLargeSize.Time3
 + dummyWithGrace.Time3 + dummyInKind.Time3 + dummyUltraPoor.Time4
 + dummyLargeSize.Time4 + dummyWithGrace.Time4 + dummyInKind.Time4
 + dummyLargeSize.UltraPoor + dummyWithGrace.UltraPoor + dummyInKind.UltraPoor
 + dummyLargeSize.UltraPoor.Time3 + dummyLargeSize.UltraPoor.Time4 + dummyWithGrace.UltraPoor.Time3
 + dummyWithGrace.UltraPoor.Time4 + dummyInKind.UltraPoor.Time3 + dummyInKind.UltraPoor.Time4
 + 

 + NumCows0
 + dummyAdiCattle0
FloodInRd1 + HHsize0 + HeadLiteracy0 + NA

dummyAdiCattle0.Time3 + dummyAdiCattle0.Time4 + dummyAdiCattle0.LargeSize + dummyAdiCattle0.LargeSize.Time3
 + dummyAdiCattle0.LargeSize.Time3 + dummyAdiCattle0.LargeSize.Time4 + dummyAdiCattle0.WithGrace
 + dummyAdiCattle0.WithGrace.Time3 + dummyAdiCattle0.WithGrace.Time4 + dummyAdiCattle0.InKind
 + dummyAdiCattle0.InKind.Time3 + dummyAdiCattle0.InKind.Time4

 
\end{Soutput}
\end{Schunk}





\hspace{-1cm}\begin{minipage}[t]{14cm}
\hfil\textsc{\normalsize Table \refstepcounter{table}\thetable: ANCOVA estimation of cattle holding, cattle rearing experiences\label{tab ANCOVA net assets experience}}\\
\setlength{\tabcolsep}{1pt}
\setlength{\baselineskip}{8pt}
\renewcommand{\arraystretch}{.52}
\hfil\begin{tikzpicture}
\node (tbl) {\input{c:/data/GUK/analysis/save/EstimationMemo/NumCowsExperienceANCOVAEstimationResults.tex}};
%\input{c:/dropbox/data/ramadan/save/tablecolortemplate.tex}
\end{tikzpicture}\\
\renewcommand{\arraystretch}{.8}
\setlength{\tabcolsep}{1pt}
\begin{tabular}{>{\hfill\scriptsize}p{1cm}<{}>{\hfill\scriptsize}p{.25cm}<{}>{\scriptsize}p{12cm}<{\hfill}}
Source:& \multicolumn{2}{l}{\scriptsize Estimated with GUK administrative and survey data.}\\
Notes: & 1. & ANCOVA estimates using administrative and survey data. Post treatment regressands are regressed on categorical variables, pre-treatment regressand and other covariates. \textsf{FloodInRd1} and \textsf{HeadLiterate0} are indicator variables for the presence of self reported damage by a flood at the baseline, and literacy of household head, respectively. \textsf{HHsize0} is household size at the baseline. We annotate the number of periods that a household is observed with \textsf{T}. The total number of households is shown for each values of \textsf{T}. \textsf{T=4} indicates the number of households with complete panel information, \textsf{T=3} indicates number of households observed three times, \textsf{T=2} indicates the number of households observed twice. \textsf{N} indicates total number of observations used in ANCOVA estimation, or \textsf{N$=$1$\times$(T=2)+2$\times$(T=3)+3$\times$(T=4)}.  \textsf{Large}, \textsf{LargeGrace}, \textsf{Cattle} are indicator variables of the \textsf{large}, \textsf{large grace}, and \textsf{cattle} arms, respectively. The default arm category is \textsf{traditional} arm. Net assets uses only assets observed for all 4 rounds in household assets. Household assets do not include livestock. \textsf{OwnCattle0} is an indicator if a household owned cattle at the baseline. \textsf{AdiCattle0} is an indicator if a household engaged in the cattle lease-in contract at the baseline.  \\
& 2. & $P$ values in percentages in parenthesises. Standard errors are clustered at group (village) level.
%${}^{***}$, ${}^{**}$, ${}^{*}$ indicate statistical significance at 1\%, 5\%, 10\%, respetively. Standard errors are clustered at group (village) level.
\end{tabular}
\end{minipage}

\vspace{-1cm}\hspace{-1cm}\begin{minipage}[t]{14cm}
\hfil\textsc{\normalsize Table \refstepcounter{table}\thetable: ANCOVA estimation of cattle holding by attributes, cattle rearing experiences\label{tab ANCOVA NumCows Experience attributes}}\\
\setlength{\tabcolsep}{1pt}
\setlength{\baselineskip}{8pt}
\renewcommand{\arraystretch}{.52}
\hfil\begin{tikzpicture}
\node (tbl) {\input{c:/data/GUK/analysis/save/EstimationMemo/NumCowsExperienceAttributesANCOVAEstimationResults.tex}};
%\input{c:/dropbox/data/ramadan/save/tablecolortemplate.tex}
\end{tikzpicture}\\
\renewcommand{\arraystretch}{.8}
\setlength{\tabcolsep}{1pt}
\begin{tabular}{>{\hfill\scriptsize}p{1cm}<{}>{\hfill\scriptsize}p{.25cm}<{}>{\scriptsize}p{12cm}<{\hfill}}
Source:& \multicolumn{2}{l}{\scriptsize Estimated with GUK administrative and survey data.}\\
Notes: & 1. & ANCOVA estimates using administrative and survey data. Post treatment regressands are regressed on categorical variables, pre-treatment regressand and other covariates. \textsf{FloodInRd1} and \textsf{HeadLiterate0} are indicator variables for the presence of self reported damage by a flood at the baseline, and literacy of household head, respectively. \textsf{HHsize0} is household size at the baseline. We annotate the number of periods that a household is observed with \textsf{T}. The total number of households is shown for each values of \textsf{T}. \textsf{T=4} indicates the number of households with complete panel information, \textsf{T=3} indicates number of households observed three times, \textsf{T=2} indicates the number of households observed twice. \textsf{N} indicates total number of observations used in ANCOVA estimation, or \textsf{N$=$1$\times$(T=2)+2$\times$(T=3)+3$\times$(T=4)}.  \textsf{Upfront} is an indicator variable of the arm with an upfront large disbursement, \textsf{WithGrace} is an indicator variable of the arm with a grace period, \textsf{InKind} is an indicator variable of the arm which lends a heifer. Net assets uses only assets observed for all 4 rounds in household assets. Household assets do not include livestock. \textsf{OwnCattle0} is an indicator if a household owned cattle at the baseline. \textsf{AdiCattle0} is an indicator if a household engaged in the cattle lease-in contract at the baseline.  \\
& 2. & $P$ values in percentages in parenthesises. Standard errors are clustered at group (village) level.
%${}^{***}$, ${}^{**}$, ${}^{*}$ indicate statistical significance at 1\%, 5\%, 10\%, respetively. Standard errors are clustered at group (village) level.
\end{tabular}
\end{minipage}

\vspace{-1cm}\hspace{-1cm}\begin{minipage}[t]{14cm}
\hfil\textsc{\normalsize Table \refstepcounter{table}\thetable: ANCOVA estimation of cattle holding by period, cattle rearing experiences\label{tab ANCOVA NumCows Experience timevarying}}\\
\setlength{\tabcolsep}{1pt}
\setlength{\baselineskip}{8pt}
\renewcommand{\arraystretch}{.52}
\hfil\begin{tikzpicture}
\node (tbl) {\input{c:/data/GUK/analysis/save/EstimationMemo/NumCowsExperienceTimeVaryingANCOVAEstimationResults.tex}};
%\input{c:/dropbox/data/ramadan/save/tablecolortemplate.tex}
\end{tikzpicture}\\
\renewcommand{\arraystretch}{.8}
\setlength{\tabcolsep}{1pt}
\begin{tabular}{>{\hfill\scriptsize}p{1cm}<{}>{\hfill\scriptsize}p{.25cm}<{}>{\scriptsize}p{12cm}<{\hfill}}
Source:& \multicolumn{2}{l}{\scriptsize Estimated with GUK administrative and survey data.}\\
Notes: & 1. & ANCOVA estimates using administrative and survey data. Post treatment regressands are regressed on categorical variables, pre-treatment regressand and other covariates. \textsf{FloodInRd1} and \textsf{HeadLiterate0} are indicator variables for the presence of self reported damage by a flood at the baseline, and literacy of household head, respectively. \textsf{HHsize0} is household size at the baseline. We annotate the number of periods that a household is observed with \textsf{T}. The total number of households is shown for each values of \textsf{T}. \textsf{T=4} indicates the number of households with complete panel information, \textsf{T=3} indicates number of households observed three times, \textsf{T=2} indicates the number of households observed twice. \textsf{N} indicates total number of observations used in ANCOVA estimation, or \textsf{N$=$1$\times$(T=2)+2$\times$(T=3)+3$\times$(T=4)}.  \textsf{rd2, rd3, rd4} are dummy variables for second, third, and fourth round of survey. Net assets uses only assets observed for all 4 rounds in household assets. Household assets do not include livestock. \textsf{OwnCattle0} is an indicator if a household owned cattle at the baseline. \textsf{AdiCattle0} is an indicator if a household engaged in the cattle lease-in contract at the baseline.  \\
& 2. & $P$ values in percentages in parenthesises. Standard errors are clustered at group (village) level.
%${}^{***}$, ${}^{**}$, ${}^{*}$ indicate statistical significance at 1\%, 5\%, 10\%, respetively. Standard errors are clustered at group (village) level.
\end{tabular}
\end{minipage}


\vspace{0cm}\hspace{-1cm}\begin{minipage}[t]{14cm}
\hfil\textsc{\normalsize Table \refstepcounter{table}\thetable: ANCOVA estimation of cattle holding by period, attributes, cattle rearing experiences\label{tab ANCOVA NumCows Experience attributes timevarying 1}}\\
\setlength{\tabcolsep}{1pt}
\setlength{\baselineskip}{8pt}
\renewcommand{\arraystretch}{.52}
\hfil\begin{tikzpicture}
\node (tbl) {\input{c:/data/GUK/analysis/save/EstimationMemo/NumCowsExperienceTimeVaryingAttributesANCOVAEstimationResults_1.tex}};
%\input{c:/dropbox/data/ramadan/save/tablecolortemplate.tex}
\end{tikzpicture}\\
\renewcommand{\arraystretch}{.8}
\setlength{\tabcolsep}{1pt}
\begin{tabular}{>{\hfill\scriptsize}p{1cm}<{}>{\hfill\scriptsize}p{.25cm}<{}>{\scriptsize}p{12cm}<{\hfill}}
%${}^{***}$, ${}^{**}$, ${}^{*}$ indicate statistical significance at 1\%, 5\%, 10\%, respetively. Standard errors are clustered at group (village) level.
\end{tabular}
\end{minipage}

\addtocounter{table}{-1}
\vspace{0cm}\hspace{-1cm}\begin{minipage}[t]{14cm}
\hfil\textsc{\normalsize Table \refstepcounter{table}\thetable: ANCOVA estimation of cattle holding by period, attributes, cattle rearing experiences (continued)\label{tab ANCOVA NumCows Experience attributes timevarying 2}}\\
\setlength{\tabcolsep}{1pt}
\setlength{\baselineskip}{8pt}
\renewcommand{\arraystretch}{.52}
\hfil\begin{tikzpicture}
\node (tbl) {\input{c:/data/GUK/analysis/save/EstimationMemo/NumCowsExperienceTimeVaryingAttributesANCOVAEstimationResults_2.tex}};
%\input{c:/dropbox/data/ramadan/save/tablecolortemplate.tex}
\end{tikzpicture}\\
\renewcommand{\arraystretch}{.8}
\setlength{\tabcolsep}{1pt}
\begin{tabular}{>{\hfill\scriptsize}p{1cm}<{}>{\hfill\scriptsize}p{.25cm}<{}>{\scriptsize}p{12cm}<{\hfill}}
%${}^{***}$, ${}^{**}$, ${}^{*}$ indicate statistical significance at 1\%, 5\%, 10\%, respetively. Standard errors are clustered at group (village) level.
Source:& \multicolumn{2}{l}{\scriptsize Estimated with GUK administrative and survey data.}\\
Notes: & 1. & ANCOVA estimates using administrative and survey data. Post treatment regressands are regressed on categorical variables, pre-treatment regressand and other covariates. \textsf{FloodInRd1} and \textsf{HeadLiterate0} are indicator variables for the presence of self reported damage by a flood at the baseline, and literacy of household head, respectively. \textsf{HHsize0} is household size at the baseline. We annotate the number of periods that a household is observed with \textsf{T}. The total number of households is shown for each values of \textsf{T}. \textsf{T=4} indicates the number of households with complete panel information, \textsf{T=3} indicates number of households observed three times, \textsf{T=2} indicates the number of households observed twice. \textsf{N} indicates total number of observations used in ANCOVA estimation, or \textsf{N$=$1$\times$(T=2)+2$\times$(T=3)+3$\times$(T=4)}.  \textsf{Upfront} is an indicator variable of the arm with an upfront large disbursement, \textsf{WithGrace} is an indicator variable of the arm with a grace period, \textsf{InKind} is an indicator variable of the arm which lends a heifer. \textsf{rd2, rd3, rd4} are dummy variables for second, third, and fourth round of survey. Net assets uses only assets observed for all 4 rounds in household assets. Household assets do not include livestock. \textsf{OwnCattle0} is an indicator if a household owned cattle at the baseline. \textsf{AdiCattle0} is an indicator if a household engaged in the cattle lease-in contract at the baseline.  \\
& 2. & $P$ values in percentages in parenthesises. Standard errors are clustered at group (village) level.
\end{tabular}
\end{minipage}



\vspace{-1cm}\hspace{-1cm}\begin{minipage}[t]{14cm}
\hfil\textsc{\normalsize Table \refstepcounter{table}\thetable: ANCOVA estimation of cattle holding by arm, poverty status, and period, cattle rearing experiences\label{tab ANCOVA NumCows Experience timevarying poverty status}}\\
\setlength{\tabcolsep}{1pt}
\setlength{\baselineskip}{8pt}
\renewcommand{\arraystretch}{.52}
\hfil\begin{tikzpicture}
\node (tbl) {\input{c:/data/GUK/analysis/save/EstimationMemo/NumCowsExperienceTimeVaryingPovertyStatusANCOVAEstimationResults.tex}};
%\input{c:/dropbox/data/ramadan/save/tablecolortemplate.tex}
\end{tikzpicture}\\
\renewcommand{\arraystretch}{.8}
\setlength{\tabcolsep}{1pt}
\begin{tabular}{>{\hfill\scriptsize}p{1cm}<{}>{\hfill\scriptsize}p{.25cm}<{}>{\scriptsize}p{12cm}<{\hfill}}
Source:& \multicolumn{2}{l}{\scriptsize Estimated with GUK administrative and survey data.}\\
Notes: & 1. & ANCOVA estimates using administrative and survey data. Post treatment regressands are regressed on categorical variables, pre-treatment regressand and other covariates. \textsf{FloodInRd1} and \textsf{HeadLiterate0} are indicator variables for the presence of self reported damage by a flood at the baseline, and literacy of household head, respectively. \textsf{HHsize0} is household size at the baseline. We annotate the number of periods that a household is observed with \textsf{T}. The total number of households is shown for each values of \textsf{T}. \textsf{T=4} indicates the number of households with complete panel information, \textsf{T=3} indicates number of households observed three times, \textsf{T=2} indicates the number of households observed twice. \textsf{N} indicates total number of observations used in ANCOVA estimation, or \textsf{N$=$1$\times$(T=2)+2$\times$(T=3)+3$\times$(T=4)}.  \textsf{UltraPoor} is an indicator variable if the household is classified as the ultra poor. \textsf{Large}, \textsf{LargeGrace}, \textsf{Cattle} are indicator variables of the \textsf{large}, \textsf{large grace}, and \textsf{cattle} arms, respectively. The default arm category is \textsf{traditional} arm. \textsf{rd2, rd3, rd4} are dummy variables for second, third, and fourth round of survey. Net assets uses only assets observed for all 4 rounds in household assets. Household assets do not include livestock. \textsf{OwnCattle0} is an indicator if a household owned cattle at the baseline. \textsf{AdiCattle0} is an indicator if a household engaged in the cattle lease-in contract at the baseline.  \\
& 2. & $P$ values in percentages in parenthesises. Standard errors are clustered at group (village) level.
%${}^{***}$, ${}^{**}$, ${}^{*}$ indicate statistical significance at 1\%, 5\%, 10\%, respetively. Standard errors are clustered at group (village) level.
\end{tabular}
\end{minipage}

\vspace{-1cm}\hspace{-1cm}\begin{minipage}[t]{14cm}
\hfil\textsc{\normalsize Table \refstepcounter{table}\thetable: ANCOVA estimation of cattle holding by attributes, poverty status, and period, cattle rearing experiences\label{tab ANCOVA NumCows Experience timevarying poverty status attributes}}\\
\setlength{\tabcolsep}{1pt}
\setlength{\baselineskip}{8pt}
\renewcommand{\arraystretch}{.52}
\hfil\begin{tikzpicture}
\node (tbl) {\input{c:/data/GUK/analysis/save/EstimationMemo/NumCowsExperienceTimeVaryingPovertyStatusAttributesANCOVAEstimationResults.tex}};
%\input{c:/dropbox/data/ramadan/save/tablecolortemplate.tex}
\end{tikzpicture}\\
\renewcommand{\arraystretch}{.8}
\setlength{\tabcolsep}{1pt}
\begin{tabular}{>{\hfill\scriptsize}p{1cm}<{}>{\hfill\scriptsize}p{.25cm}<{}>{\scriptsize}p{12cm}<{\hfill}}
Source:& \multicolumn{2}{l}{\scriptsize Estimated with GUK administrative and survey data.}\\
Notes: & 1. & ANCOVA estimates using administrative and survey data. Post treatment regressands are regressed on categorical variables, pre-treatment regressand and other covariates. \textsf{FloodInRd1} and \textsf{HeadLiterate0} are indicator variables for the presence of self reported damage by a flood at the baseline, and literacy of household head, respectively. \textsf{HHsize0} is household size at the baseline. We annotate the number of periods that a household is observed with \textsf{T}. The total number of households is shown for each values of \textsf{T}. \textsf{T=4} indicates the number of households with complete panel information, \textsf{T=3} indicates number of households observed three times, \textsf{T=2} indicates the number of households observed twice. \textsf{N} indicates total number of observations used in ANCOVA estimation, or \textsf{N$=$1$\times$(T=2)+2$\times$(T=3)+3$\times$(T=4)}.  \textsf{UltraPoor} is an indicator variable if the household is classified as the ultra poor. \textsf{Upfront} is an indicator variable of the arm with an upfront large disbursement, \textsf{WithGrace} is an indicator variable of the arm with a grace period, \textsf{InKind} is an indicator variable of the arm which lends a heifer. \textsf{rd2, rd3, rd4} are dummy variables for second, third, and fourth round of survey.  Net assets uses only assets observed for all 4 rounds in household assets. Household assets do not include livestock. \textsf{OwnCattle0} is an indicator if a household owned cattle at the baseline. \textsf{AdiCattle0} is an indicator if a household engaged in the cattle lease-in contract at the baseline.  \\
& 2. & $P$ values in percentages in parenthesises. Standard errors are clustered at group (village) level.
%${}^{***}$, ${}^{**}$, ${}^{*}$ indicate statistical significance at 1\%, 5\%, 10\%, respetively. Standard errors are clustered at group (village) level.
\end{tabular}
\end{minipage}

\hspace{-1cm}\begin{minipage}[t]{14cm}
\hfil\textsc{\normalsize Table \refstepcounter{table}\thetable: ANCOVA estimation of livestock holding, subsamles by cattle rearing experiences\label{tab ANCOVA net assets experience 1}}\\
\setlength{\tabcolsep}{1pt}
\setlength{\baselineskip}{8pt}
\renewcommand{\arraystretch}{.52}
\hfil\begin{tikzpicture}
\node (tbl) {\input{c:/data/GUK/analysis/save/EstimationMemo/NumCowsByExperience1ANCOVAEstimationResults.tex}};
%\input{c:/dropbox/data/ramadan/save/tablecolortemplate.tex}
\end{tikzpicture}\\
\end{minipage}

\addtocounter{table}{-1}
\vspace{-1cm}\hspace{-1cm}\begin{minipage}[t]{14cm}
\hfil\textsc{\normalsize Table \refstepcounter{table}\thetable: ANCOVA estimation of livestock holding, subsamles by cattle rearing experiences (continued)\label{tab ANCOVA net assets experience 2}}\\
\setlength{\tabcolsep}{1pt}
\setlength{\baselineskip}{8pt}
\renewcommand{\arraystretch}{.52}
\hfil\begin{tikzpicture}
\node (tbl) {\input{c:/data/GUK/analysis/save/EstimationMemo/NumCowsByExperience2ANCOVAEstimationResults.tex}};
%\input{c:/dropbox/data/ramadan/save/tablecolortemplate.tex}
\end{tikzpicture}\\
\renewcommand{\arraystretch}{.8}
\setlength{\tabcolsep}{1pt}
\begin{tabular}{>{\hfill\scriptsize}p{1cm}<{}>{\hfill\scriptsize}p{.25cm}<{}>{\scriptsize}p{12cm}<{\hfill}}
Source:& \multicolumn{2}{l}{\scriptsize Estimated with GUK administrative and survey data.}\\
Notes: & 1. & ANCOVA estimates using administrative and survey data. Post treatment regressands are regressed on categorical variables, pre-treatment regressand and other covariates. \textsf{FloodInRd1} and \textsf{HeadLiterate0} are indicator variables for the presence of self reported damage by a flood at the baseline, and literacy of household head, respectively. \textsf{HHsize0} is household size at the baseline. We annotate the number of periods that a household is observed with \textsf{T}. The total number of households is shown for each values of \textsf{T}. \textsf{T=4} indicates the number of households with complete panel information, \textsf{T=3} indicates number of households observed three times, \textsf{T=2} indicates the number of households observed twice. \textsf{N} indicates total number of observations used in ANCOVA estimation, or \textsf{N$=$1$\times$(T=2)+2$\times$(T=3)+3$\times$(T=4)}.  \textsf{Large}, \textsf{LargeGrace}, \textsf{Cattle} are indicator variables of the \textsf{large}, \textsf{large grace}, and \textsf{cattle} arms, respectively. The default arm category is \textsf{traditional} arm. Narrow net assets uses only assets observed for all 4 rounds in household assets. Household assets do not include livestock. \textsf{OwnCattle0} is an indicator if a household owned cattle at the baseline. \textsf{AdiCattle0} is an indicator if a household engaged in the cattle lease-in contract at the baseline.  \textsf{Adi} and \textsf{None} subgroups do not own cattle at the baseline. We used net asset values at the baseline \textsf{NetAssets0} in place of \textsf{NumCows0} in ANCOVA estimation. \\
& 2. & $P$ values in percentages in parenthesises. Standard errors are clustered at group (village) level.
%${}^{***}$, ${}^{**}$, ${}^{*}$ indicate statistical significance at 1\%, 5\%, 10\%, respetively. Standard errors are clustered at group (village) level.
\end{tabular}
\end{minipage}



\vspace{-1cm}\hspace{-1cm}\begin{minipage}[t]{14cm}
\hfil\textsc{\normalsize Table \refstepcounter{table}\thetable: ANCOVA estimation of livestock holding by attributes, subsamles by cattle rearing experiences\label{tab ANCOVA livestock holding Experience attributes}}\\
\setlength{\tabcolsep}{1pt}
\setlength{\baselineskip}{8pt}
\renewcommand{\arraystretch}{.52}
\hfil\begin{tikzpicture}
\node (tbl) {\input{c:/data/GUK/analysis/save/EstimationMemo/NumCowsByExperience1AttributesANCOVAEstimationResults.tex}};
%\input{c:/dropbox/data/ramadan/save/tablecolortemplate.tex}
\end{tikzpicture}\\
\end{minipage}

\addtocounter{table}{-1}
\hspace{-1cm}\begin{minipage}[t]{14cm}
\hfil\textsc{\normalsize Table \refstepcounter{table}\thetable: ANCOVA estimation of livestock holding by attributes, subsamles by cattle rearing experiences (continued)\label{tab ANCOVA livestock holding Experience attributes 2}}\\
\setlength{\tabcolsep}{1pt}
\setlength{\baselineskip}{8pt}
\renewcommand{\arraystretch}{.52}
\hfil\begin{tikzpicture}
\node (tbl) {\input{c:/data/GUK/analysis/save/EstimationMemo/NumCowsByExperience2AttributesANCOVAEstimationResults.tex}};
%\input{c:/dropbox/data/ramadan/save/tablecolortemplate.tex}
\end{tikzpicture}\\
\renewcommand{\arraystretch}{.8}
\setlength{\tabcolsep}{1pt}
\begin{tabular}{>{\hfill\scriptsize}p{1cm}<{}>{\hfill\scriptsize}p{.25cm}<{}>{\scriptsize}p{12cm}<{\hfill}}
Source:& \multicolumn{2}{l}{\scriptsize Estimated with GUK administrative and survey data.}\\
Notes: & 1. & ANCOVA estimates using administrative and survey data. Post treatment regressands are regressed on categorical variables, pre-treatment regressand and other covariates. \textsf{FloodInRd1} and \textsf{HeadLiterate0} are indicator variables for the presence of self reported damage by a flood at the baseline, and literacy of household head, respectively. \textsf{HHsize0} is household size at the baseline. We annotate the number of periods that a household is observed with \textsf{T}. The total number of households is shown for each values of \textsf{T}. \textsf{T=4} indicates the number of households with complete panel information, \textsf{T=3} indicates number of households observed three times, \textsf{T=2} indicates the number of households observed twice. \textsf{N} indicates total number of observations used in ANCOVA estimation, or \textsf{N$=$1$\times$(T=2)+2$\times$(T=3)+3$\times$(T=4)}.  \textsf{Upfront} is an indicator variable of the arm with an upfront large disbursement, \textsf{WithGrace} is an indicator variable of the arm with a grace period, \textsf{InKind} is an indicator variable of the arm which lends a heifer. Narrow net assets uses only assets observed for all 4 rounds in household assets. Household assets do not include livestock. \textsf{OwnCattle0} is an indicator if a household owned cattle at the baseline. \textsf{AdiCattle0} is an indicator if a household engaged in the cattle lease-in contract at the baseline.  \textsf{Adi} and \textsf{None} subgroups do not own cattle at the baseline. We used net asset values at the baseline \textsf{NetAssets0} in place of \textsf{NumCows0} in ANCOVA estimation.\\
& 2. & $P$ values in percentages in parenthesises. Standard errors are clustered at group (village) level.
%${}^{***}$, ${}^{**}$, ${}^{*}$ indicate statistical significance at 1\%, 5\%, 10\%, respetively. Standard errors are clustered at group (village) level.
\end{tabular}
\end{minipage}



\vspace{-1cm}\hspace{-1cm}\begin{minipage}[t]{14cm}
\hfil\textsc{\normalsize Table \refstepcounter{table}\thetable: ANCOVA estimation of livestock holding by period, subsamles by cattle rearing experiences\label{tab ANCOVA livestock holding Experience timevarying}}\\
\setlength{\tabcolsep}{1pt}
\setlength{\baselineskip}{8pt}
\renewcommand{\arraystretch}{.52}
\hfil\begin{tikzpicture}
\node (tbl) {\input{c:/data/GUK/analysis/save/EstimationMemo/NumCowsByExperience1TimeVaryingANCOVAEstimationResults.tex}};
%\input{c:/dropbox/data/ramadan/save/tablecolortemplate.tex}
\end{tikzpicture}
\end{minipage}

\addtocounter{table}{-1}
\hspace{-1cm}\begin{minipage}[t]{14cm}
\hfil\textsc{\normalsize Table \refstepcounter{table}\thetable: ANCOVA estimation of livestock holding by period, subsamles by cattle rearing experiences (continued) \label{tab ANCOVA livestock holding Experience timevarying 2}}\\
\setlength{\tabcolsep}{1pt}
\setlength{\baselineskip}{8pt}
\renewcommand{\arraystretch}{.52}
\hfil\begin{tikzpicture}
\node (tbl) {\input{c:/data/GUK/analysis/save/EstimationMemo/NumCowsByExperience2TimeVaryingANCOVAEstimationResults.tex}};
%\input{c:/dropbox/data/ramadan/save/tablecolortemplate.tex}
\end{tikzpicture}\\
\renewcommand{\arraystretch}{.8}
\setlength{\tabcolsep}{1pt}
\begin{tabular}{>{\hfill\scriptsize}p{1cm}<{}>{\hfill\scriptsize}p{.25cm}<{}>{\scriptsize}p{12cm}<{\hfill}}
Source:& \multicolumn{2}{l}{\scriptsize Estimated with GUK administrative and survey data.}\\
Notes: & 1. & ANCOVA estimates using administrative and survey data. Post treatment regressands are regressed on categorical variables, pre-treatment regressand and other covariates. \textsf{FloodInRd1} and \textsf{HeadLiterate0} are indicator variables for the presence of self reported damage by a flood at the baseline, and literacy of household head, respectively. \textsf{HHsize0} is household size at the baseline. We annotate the number of periods that a household is observed with \textsf{T}. The total number of households is shown for each values of \textsf{T}. \textsf{T=4} indicates the number of households with complete panel information, \textsf{T=3} indicates number of households observed three times, \textsf{T=2} indicates the number of households observed twice. \textsf{N} indicates total number of observations used in ANCOVA estimation, or \textsf{N$=$1$\times$(T=2)+2$\times$(T=3)+3$\times$(T=4)}.  \textsf{Large}, \textsf{LargeGrace}, \textsf{Cattle} are indicator variables of the \textsf{large}, \textsf{large grace}, and \textsf{cattle} arms, respectively. The default arm category is \textsf{traditional} arm. \textsf{rd2, rd3, rd4} are dummy variables for second, third, and fourth round of survey. Narrow net assets uses only assets observed for all 4 rounds in household assets. Household assets do not include livestock. \textsf{OwnCattle0} is an indicator if a household owned cattle at the baseline. \textsf{AdiCattle0} is an indicator if a household engaged in the cattle lease-in contract at the baseline.  \textsf{Adi} and \textsf{None} subgroups do not own cattle at the baseline. We used net asset values at the baseline \textsf{NetAssets0} in place of \textsf{NumCows0} in ANCOVA estimation.\\
& 2. & $P$ values in percentages in parenthesises. Standard errors are clustered at group (village) level.
%${}^{***}$, ${}^{**}$, ${}^{*}$ indicate statistical significance at 1\%, 5\%, 10\%, respetively. Standard errors are clustered at group (village) level.
\end{tabular}
\end{minipage}




\vspace{-1cm}\hspace{-1cm}\begin{minipage}[t]{14cm}
\hfil\textsc{\normalsize Table \refstepcounter{table}\thetable: ANCOVA estimation of livestock holding by attributes and period, subsamles by cattle rearing experiences\label{tab ANCOVA livestock holding Experience timevarying attributes}}\\
\setlength{\tabcolsep}{1pt}
\setlength{\baselineskip}{8pt}
\renewcommand{\arraystretch}{.52}
\hfil\begin{tikzpicture}
\node (tbl) {\input{c:/data/GUK/analysis/save/EstimationMemo/NumCowsByExperience1TimeVaryingAttributesANCOVAEstimationResults.tex}};
%\input{c:/dropbox/data/ramadan/save/tablecolortemplate.tex}
\end{tikzpicture}\\
\end{minipage}

\addtocounter{table}{-1}
\hspace{-1cm}\begin{minipage}[t]{14cm}
\hfil\textsc{\normalsize Table \refstepcounter{table}\thetable: ANCOVA estimation of livestock holding by attributes and period, subsamles by cattle rearing experiences (continued)\label{tab ANCOVA livestock holding Experience timevarying attributes 2}}\\
\setlength{\tabcolsep}{1pt}
\setlength{\baselineskip}{8pt}
\renewcommand{\arraystretch}{.52}
\hfil\begin{tikzpicture}
\node (tbl) {\input{c:/data/GUK/analysis/save/EstimationMemo/NumCowsByExperience2TimeVaryingAttributesANCOVAEstimationResults.tex}};
%\input{c:/dropbox/data/ramadan/save/tablecolortemplate.tex}
\end{tikzpicture}\\
\renewcommand{\arraystretch}{.8}
\setlength{\tabcolsep}{1pt}
\begin{tabular}{>{\hfill\scriptsize}p{1cm}<{}>{\hfill\scriptsize}p{.25cm}<{}>{\scriptsize}p{12cm}<{\hfill}}
Source:& \multicolumn{2}{l}{\scriptsize Estimated with GUK administrative and survey data.}\\
Notes: & 1. & ANCOVA estimates using administrative and survey data. Post treatment regressands are regressed on categorical variables, pre-treatment regressand and other covariates. \textsf{FloodInRd1} and \textsf{HeadLiterate0} are indicator variables for the presence of self reported damage by a flood at the baseline, and literacy of household head, respectively. \textsf{HHsize0} is household size at the baseline. We annotate the number of periods that a household is observed with \textsf{T}. The total number of households is shown for each values of \textsf{T}. \textsf{T=4} indicates the number of households with complete panel information, \textsf{T=3} indicates number of households observed three times, \textsf{T=2} indicates the number of households observed twice. \textsf{N} indicates total number of observations used in ANCOVA estimation, or \textsf{N$=$1$\times$(T=2)+2$\times$(T=3)+3$\times$(T=4)}.  \textsf{Upfront} is an indicator variable of the arm with an upfront large disbursement, \textsf{WithGrace} is an indicator variable of the arm with a grace period, \textsf{InKind} is an indicator variable of the arm which lends a heifer. \textsf{rd2, rd3, rd4} are dummy variables for second, third, and fourth round of survey. Narrow net assets uses only assets observed for all 4 rounds in household assets. Household assets do not include livestock. \textsf{OwnCattle0} is an indicator if a household owned cattle at the baseline. \textsf{AdiCattle0} is an indicator if a household engaged in the cattle lease-in contract at the baseline.  \textsf{Adi} and \textsf{None} subgroups do not own cattle at the baseline. We used net asset values at the baseline \textsf{NetAssets0} in place of \textsf{NumCows0} in ANCOVA estimation.\\
& 2. & $P$ values in percentages in parenthesises. Standard errors are clustered at group (village) level.
%${}^{***}$, ${}^{**}$, ${}^{*}$ indicate statistical significance at 1\%, 5\%, 10\%, respetively. Standard errors are clustered at group (village) level.
\end{tabular}
\end{minipage}





\section{Estimation using complete panel HHs in household assets}


This section uses subsample limited to households which gives complete panel of household assets. 

\subsection{Assets}

\subsubsection{Productive assets}


Productive assets are sorveyed consistently across rounds, except hand pumps that were asked only in round 1. Major productive assets (above 300 entries) are bees-box, cage incubator, dheki, fishing net, ginning machine, hand pump, sickle/dao/axe/spade. Bee boxes have increased dramatically from round 2. Sickles/dao/axes/spades and fishing nets have decreased dramatically since round 2. These indicate that household production may have shifted to more domestic-oriented tasks. There is no indication that productive asset holding related to cattle rearing has increased.


\begin{Schunk}
\begin{Soutput}
Error in file(filename, "r", encoding = encoding): コネクションを開くことができません
\end{Soutput}
\end{Schunk}




\begin{Schunk}
\begin{Soutput}
[1] excl
[[1]]
PAssetAmount ~ dummyLarge + dummyLargeGrace + dummyCattle

[[2]]
PAssetAmount ~ dummyLarge + dummyLargeGrace + dummyCattle + PAssetAmount0

[[3]]
PAssetAmount ~ FloodInRd1 + dummyLarge + dummyLargeGrace + dummyCattle + 
    HHsize0 + HeadLiteracy0 + PAssetAmount0

[[4]]
PAssetAmount ~ FloodInRd1 + dummyLarge + dummyLargeGrace + dummyCattle + 
    dummyHadCows + HHsize0 + HeadLiteracy0 + PAssetAmount0 + 
    dummyHadCows.Large + dummyHadCows.LargeGrace + dummyHadCows.Cattle

[[5]]
PAssetAmount ~ FloodInRd1 + dummyLarge + dummyLargeGrace + dummyCattle + 
    HHsize0 + HeadLiteracy0 + PAssetAmount0 + NumCows0

[[6]]
PAssetAmount ~ FloodInRd1 + dummyLarge + dummyLargeGrace + dummyCattle + 
    dummyHadCows + HHsize0 + HeadLiteracy0 + PAssetAmount0 + 
    NumCows0 + dummyHadCows.Large + dummyHadCows.LargeGrace + 
    dummyHadCows.Cattle

[1] exclP
[[1]]
PAssetAmount ~ dummyUltraPoor + dummyLargeSize + dummyWithGrace + 
    dummyInKind + dummyLargeSize.UltraPoor + dummyWithGrace.UltraPoor + 
    dummyInKind.UltraPoor

[[2]]
PAssetAmount ~ dummyUltraPoor + dummyLargeSize + dummyWithGrace + 
    dummyInKind + PAssetAmount0 + dummyLargeSize.UltraPoor + 
    dummyWithGrace.UltraPoor + dummyInKind.UltraPoor

[[3]]
PAssetAmount ~ FloodInRd1 + dummyUltraPoor + dummyLargeSize + 
    dummyWithGrace + dummyInKind + HHsize0 + HeadLiteracy0 + 
    PAssetAmount0 + dummyLargeSize.UltraPoor + dummyWithGrace.UltraPoor + 
    dummyInKind.UltraPoor

[[4]]
PAssetAmount ~ FloodInRd1 + dummyUltraPoor + dummyLargeSize + 
    dummyWithGrace + dummyInKind + dummyHadCows + HHsize0 + HeadLiteracy0 + 
    PAssetAmount0 + dummyLargeSize.UltraPoor + dummyWithGrace.UltraPoor + 
    dummyInKind.UltraPoor + dummyHadCows.LargeSize + dummyHadCows.WithGrace + 
    dummyHadCows.InKind

[[5]]
PAssetAmount ~ FloodInRd1 + dummyUltraPoor + dummyLargeSize + 
    dummyWithGrace + dummyInKind + HHsize0 + HeadLiteracy0 + 
    PAssetAmount0 + NumCows0 + dummyLargeSize.UltraPoor + dummyWithGrace.UltraPoor + 
    dummyInKind.UltraPoor

[[6]]
PAssetAmount ~ FloodInRd1 + dummyUltraPoor + dummyLargeSize + 
    dummyWithGrace + dummyInKind + dummyHadCows + HHsize0 + HeadLiteracy0 + 
    PAssetAmount0 + NumCows0 + dummyLargeSize.UltraPoor + dummyWithGrace.UltraPoor + 
    dummyInKind.UltraPoor + dummyHadCows.LargeSize + dummyHadCows.WithGrace + 
    dummyHadCows.InKind

[1] excla
[[1]]
PAssetAmount ~ dummyLargeSize + dummyWithGrace + dummyInKind

[[2]]
PAssetAmount ~ dummyLargeSize + dummyWithGrace + dummyInKind + 
    PAssetAmount0

[[3]]
PAssetAmount ~ FloodInRd1 + dummyLargeSize + dummyWithGrace + 
    dummyInKind + HHsize0 + HeadLiteracy0 + PAssetAmount0

[[4]]
PAssetAmount ~ FloodInRd1 + dummyLargeSize + dummyWithGrace + 
    dummyInKind + dummyHadCows + HHsize0 + HeadLiteracy0 + PAssetAmount0 + 
    dummyHadCows.LargeSize + dummyHadCows.WithGrace + dummyHadCows.InKind

[[5]]
PAssetAmount ~ FloodInRd1 + dummyLargeSize + dummyWithGrace + 
    dummyInKind + HHsize0 + HeadLiteracy0 + PAssetAmount0 + NumCows0

[[6]]
PAssetAmount ~ FloodInRd1 + dummyLargeSize + dummyWithGrace + 
    dummyInKind + dummyHadCows + HHsize0 + HeadLiteracy0 + PAssetAmount0 + 
    NumCows0 + dummyHadCows.LargeSize + dummyHadCows.WithGrace + 
    dummyHadCows.InKind

[1] exclT
[[1]]
PAssetAmount ~ Time.3 + Time.4 + dummyLarge + dummyLargeGrace + 
    dummyCattle + dummyLarge.Time3 + dummyLargeGrace.Time3 + 
    dummyCattle.Time3 + dummyLarge.Time4 + dummyLargeGrace.Time4 + 
    dummyCattle.Time4

[[2]]
PAssetAmount ~ Time.3 + Time.4 + dummyLarge + dummyLargeGrace + 
    dummyCattle + dummyLarge.Time3 + dummyLargeGrace.Time3 + 
    dummyCattle.Time3 + dummyLarge.Time4 + dummyLargeGrace.Time4 + 
    dummyCattle.Time4 + PAssetAmount0

[[3]]
PAssetAmount ~ FloodInRd1 + Time.3 + Time.4 + dummyLarge + dummyLargeGrace + 
    dummyCattle + dummyLarge.Time3 + dummyLargeGrace.Time3 + 
    dummyCattle.Time3 + dummyLarge.Time4 + dummyLargeGrace.Time4 + 
    dummyCattle.Time4 + HHsize0 + HeadLiteracy0 + PAssetAmount0

[[4]]
PAssetAmount ~ FloodInRd1 + Time.3 + Time.4 + dummyLarge + dummyLargeGrace + 
    dummyCattle + dummyLarge.Time3 + dummyLargeGrace.Time3 + 
    dummyCattle.Time3 + dummyLarge.Time4 + dummyLargeGrace.Time4 + 
    dummyCattle.Time4 + dummyHadCows + HHsize0 + HeadLiteracy0 + 
    PAssetAmount0 + dummyHadCows.Large + dummyHadCows.Time3 + 
    dummyHadCows.Large.Time3 + dummyHadCows.Time4 + dummyHadCows.Large.Time4 + 
    dummyHadCows.LargeGrace + dummyHadCows.LargeGrace.Time3 + 
    dummyHadCows.LargeGrace.Time4 + dummyHadCows.Cattle + dummyHadCows.Cattle.Time3 + 
    dummyHadCows.Cattle.Time4

[[5]]
PAssetAmount ~ FloodInRd1 + Time.3 + Time.4 + dummyLarge + dummyLargeGrace + 
    dummyCattle + dummyLarge.Time3 + dummyLargeGrace.Time3 + 
    dummyCattle.Time3 + dummyLarge.Time4 + dummyLargeGrace.Time4 + 
    dummyCattle.Time4 + HHsize0 + HeadLiteracy0 + PAssetAmount0 + 
    NumCows0

[[6]]
PAssetAmount ~ FloodInRd1 + Time.3 + Time.4 + dummyLarge + dummyLargeGrace + 
    dummyCattle + dummyLarge.Time3 + dummyLargeGrace.Time3 + 
    dummyCattle.Time3 + dummyLarge.Time4 + dummyLargeGrace.Time4 + 
    dummyCattle.Time4 + dummyHadCows + HHsize0 + HeadLiteracy0 + 
    PAssetAmount0 + NumCows0 + dummyHadCows.Large + dummyHadCows.Time3 + 
    dummyHadCows.Large.Time3 + dummyHadCows.Time4 + dummyHadCows.Large.Time4 + 
    dummyHadCows.LargeGrace + dummyHadCows.LargeGrace.Time3 + 
    dummyHadCows.LargeGrace.Time4 + dummyHadCows.Cattle + dummyHadCows.Cattle.Time3 + 
    dummyHadCows.Cattle.Time4

[1] exclTa
[[1]]
PAssetAmount ~ Time.3 + Time.4 + dummyLargeSize + dummyWithGrace + 
    dummyInKind + dummyLargeSize.Time3 + dummyWithGrace.Time3 + 
    dummyInKind.Time3 + dummyLargeSize.Time4 + dummyWithGrace.Time4 + 
    dummyInKind.Time4

[[2]]
PAssetAmount ~ Time.3 + Time.4 + dummyLargeSize + dummyWithGrace + 
    dummyInKind + dummyLargeSize.Time3 + dummyWithGrace.Time3 + 
    dummyInKind.Time3 + dummyLargeSize.Time4 + dummyWithGrace.Time4 + 
    dummyInKind.Time4 + PAssetAmount0

[[3]]
PAssetAmount ~ FloodInRd1 + Time.3 + Time.4 + dummyLargeSize + 
    dummyWithGrace + dummyInKind + dummyLargeSize.Time3 + dummyWithGrace.Time3 + 
    dummyInKind.Time3 + dummyLargeSize.Time4 + dummyWithGrace.Time4 + 
    dummyInKind.Time4 + HHsize0 + HeadLiteracy0 + PAssetAmount0

[[4]]
PAssetAmount ~ FloodInRd1 + Time.3 + Time.4 + dummyLargeSize + 
    dummyWithGrace + dummyInKind + dummyLargeSize.Time3 + dummyWithGrace.Time3 + 
    dummyInKind.Time3 + dummyLargeSize.Time4 + dummyWithGrace.Time4 + 
    dummyInKind.Time4 + dummyHadCows + HHsize0 + HeadLiteracy0 + 
    PAssetAmount0 + dummyHadCows.Time3 + dummyHadCows.Time4 + 
    dummyHadCows.LargeSize + dummyHadCows.LargeSize.Time3 + dummyHadCows.LargeSize.Time4 + 
    dummyHadCows.WithGrace + dummyHadCows.WithGrace.Time3 + dummyHadCows.WithGrace.Time4 + 
    dummyHadCows.InKind + dummyHadCows.InKind.Time3 + dummyHadCows.InKind.Time4

[[5]]
PAssetAmount ~ FloodInRd1 + Time.3 + Time.4 + dummyLargeSize + 
    dummyWithGrace + dummyInKind + dummyLargeSize.Time3 + dummyWithGrace.Time3 + 
    dummyInKind.Time3 + dummyLargeSize.Time4 + dummyWithGrace.Time4 + 
    dummyInKind.Time4 + HHsize0 + HeadLiteracy0 + PAssetAmount0 + 
    NumCows0

[[6]]
PAssetAmount ~ FloodInRd1 + Time.3 + Time.4 + dummyLargeSize + 
    dummyWithGrace + dummyInKind + dummyLargeSize.Time3 + dummyWithGrace.Time3 + 
    dummyInKind.Time3 + dummyLargeSize.Time4 + dummyWithGrace.Time4 + 
    dummyInKind.Time4 + dummyHadCows + HHsize0 + HeadLiteracy0 + 
    PAssetAmount0 + NumCows0 + dummyHadCows.Time3 + dummyHadCows.Time4 + 
    dummyHadCows.LargeSize + dummyHadCows.LargeSize.Time3 + dummyHadCows.LargeSize.Time4 + 
    dummyHadCows.WithGrace + dummyHadCows.WithGrace.Time3 + dummyHadCows.WithGrace.Time4 + 
    dummyHadCows.InKind + dummyHadCows.InKind.Time3 + dummyHadCows.InKind.Time4

[1] exclTPa
[[1]]
PAssetAmount ~ Time.3 + Time.4 + dummyUltraPoor + dummyLargeSize + 
    dummyWithGrace + dummyInKind + dummyUltraPoor.Time3 + dummyLargeSize.Time3 + 
    dummyWithGrace.Time3 + dummyInKind.Time3 + dummyUltraPoor.Time4 + 
    dummyLargeSize.Time4 + dummyWithGrace.Time4 + dummyInKind.Time4 + 
    dummyLargeSize.UltraPoor + dummyWithGrace.UltraPoor + dummyInKind.UltraPoor + 
    dummyLargeSize.UltraPoor.Time3 + dummyLargeSize.UltraPoor.Time4 + 
    dummyWithGrace.UltraPoor.Time3 + dummyWithGrace.UltraPoor.Time4 + 
    dummyInKind.UltraPoor.Time3 + dummyInKind.UltraPoor.Time4

[[2]]
PAssetAmount ~ Time.3 + Time.4 + dummyUltraPoor + dummyLargeSize + 
    dummyWithGrace + dummyInKind + dummyUltraPoor.Time3 + dummyLargeSize.Time3 + 
    dummyWithGrace.Time3 + dummyInKind.Time3 + dummyUltraPoor.Time4 + 
    dummyLargeSize.Time4 + dummyWithGrace.Time4 + dummyInKind.Time4 + 
    PAssetAmount0 + dummyLargeSize.UltraPoor + dummyWithGrace.UltraPoor + 
    dummyInKind.UltraPoor + dummyLargeSize.UltraPoor.Time3 + 
    dummyLargeSize.UltraPoor.Time4 + dummyWithGrace.UltraPoor.Time3 + 
    dummyWithGrace.UltraPoor.Time4 + dummyInKind.UltraPoor.Time3 + 
    dummyInKind.UltraPoor.Time4

[[3]]
PAssetAmount ~ FloodInRd1 + Time.3 + Time.4 + dummyUltraPoor + 
    dummyLargeSize + dummyWithGrace + dummyInKind + dummyUltraPoor.Time3 + 
    dummyLargeSize.Time3 + dummyWithGrace.Time3 + dummyInKind.Time3 + 
    dummyUltraPoor.Time4 + dummyLargeSize.Time4 + dummyWithGrace.Time4 + 
    dummyInKind.Time4 + HHsize0 + HeadLiteracy0 + PAssetAmount0 + 
    dummyLargeSize.UltraPoor + dummyWithGrace.UltraPoor + dummyInKind.UltraPoor + 
    dummyLargeSize.UltraPoor.Time3 + dummyLargeSize.UltraPoor.Time4 + 
    dummyWithGrace.UltraPoor.Time3 + dummyWithGrace.UltraPoor.Time4 + 
    dummyInKind.UltraPoor.Time3 + dummyInKind.UltraPoor.Time4

[[4]]
PAssetAmount ~ FloodInRd1 + Time.3 + Time.4 + dummyUltraPoor + 
    dummyLargeSize + dummyWithGrace + dummyInKind + dummyUltraPoor.Time3 + 
    dummyLargeSize.Time3 + dummyWithGrace.Time3 + dummyInKind.Time3 + 
    dummyUltraPoor.Time4 + dummyLargeSize.Time4 + dummyWithGrace.Time4 + 
    dummyInKind.Time4 + dummyHadCows + HHsize0 + HeadLiteracy0 + 
    PAssetAmount0 + dummyLargeSize.UltraPoor + dummyWithGrace.UltraPoor + 
    dummyInKind.UltraPoor + dummyLargeSize.UltraPoor.Time3 + 
    dummyLargeSize.UltraPoor.Time4 + dummyWithGrace.UltraPoor.Time3 + 
    dummyWithGrace.UltraPoor.Time4 + dummyInKind.UltraPoor.Time3 + 
    dummyInKind.UltraPoor.Time4 + dummyHadCows.Time3 + dummyHadCows.Time4 + 
    dummyHadCows.LargeSize + dummyHadCows.LargeSize.Time3 + dummyHadCows.LargeSize.Time4 + 
    dummyHadCows.WithGrace + dummyHadCows.WithGrace.Time3 + dummyHadCows.WithGrace.Time4 + 
    dummyHadCows.InKind + dummyHadCows.InKind.Time3 + dummyHadCows.InKind.Time4

[[5]]
PAssetAmount ~ FloodInRd1 + Time.3 + Time.4 + dummyUltraPoor + 
    dummyLargeSize + dummyWithGrace + dummyInKind + dummyUltraPoor.Time3 + 
    dummyLargeSize.Time3 + dummyWithGrace.Time3 + dummyInKind.Time3 + 
    dummyUltraPoor.Time4 + dummyLargeSize.Time4 + dummyWithGrace.Time4 + 
    dummyInKind.Time4 + HHsize0 + HeadLiteracy0 + PAssetAmount0 + 
    NumCows0 + dummyLargeSize.UltraPoor + dummyWithGrace.UltraPoor + 
    dummyInKind.UltraPoor + dummyLargeSize.UltraPoor.Time3 + 
    dummyLargeSize.UltraPoor.Time4 + dummyWithGrace.UltraPoor.Time3 + 
    dummyWithGrace.UltraPoor.Time4 + dummyInKind.UltraPoor.Time3 + 
    dummyInKind.UltraPoor.Time4

[[6]]
PAssetAmount ~ FloodInRd1 + Time.3 + Time.4 + dummyUltraPoor + 
    dummyLargeSize + dummyWithGrace + dummyInKind + dummyUltraPoor.Time3 + 
    dummyLargeSize.Time3 + dummyWithGrace.Time3 + dummyInKind.Time3 + 
    dummyUltraPoor.Time4 + dummyLargeSize.Time4 + dummyWithGrace.Time4 + 
    dummyInKind.Time4 + dummyHadCows + HHsize0 + HeadLiteracy0 + 
    PAssetAmount0 + NumCows0 + dummyLargeSize.UltraPoor + dummyWithGrace.UltraPoor + 
    dummyInKind.UltraPoor + dummyLargeSize.UltraPoor.Time3 + 
    dummyLargeSize.UltraPoor.Time4 + dummyWithGrace.UltraPoor.Time3 + 
    dummyWithGrace.UltraPoor.Time4 + dummyInKind.UltraPoor.Time3 + 
    dummyInKind.UltraPoor.Time4 + dummyHadCows.Time3 + dummyHadCows.Time4 + 
    dummyHadCows.LargeSize + dummyHadCows.LargeSize.Time3 + dummyHadCows.LargeSize.Time4 + 
    dummyHadCows.WithGrace + dummyHadCows.WithGrace.Time3 + dummyHadCows.WithGrace.Time4 + 
    dummyHadCows.InKind + dummyHadCows.InKind.Time3 + dummyHadCows.InKind.Time4
\end{Soutput}
\end{Schunk}
\begin{Schunk}
\begin{Soutput}
Error in `geom_boxplot()`:
! Problem while computing aesthetics.
i Error occurred in the 1st layer.
Caused by error:
! オブジェクト 'PAssetAmount' がありません
\end{Soutput}
\begin{Soutput}
Error in `geom_boxplot()`:
! Problem while computing aesthetics.
i Error occurred in the 1st layer.
Caused by error:
! オブジェクト 'PAssetAmount' がありません
\end{Soutput}
\end{Schunk}


\mpage{\linewidth}{
\hfil\textsc{\footnotesize Figure \refstepcounter{figure}\thefigure: Productive asset holding\label{fig prod asset survey round}}\\
\hfil\includegraphics[width = 12cm]{c:/data/GUK/analysis/program/figure/EstimationMemo/CompletePanelProdAssets.pdf}\\
\renewcommand{\arraystretch}{1}
\hfil\begin{tabular}{>{\hfill\scriptsize}p{1cm}<{}>{\scriptsize}p{12cm}<{\hfill}}
Source: & Survey data.\\
Note:& Productive assets are bees-box, brooder, cage incubator, country boat, deep tube well, dheki, done/swing basket, engine boat, fishing net, ginning machine, gola (grain storage), hand pump, husking machine, jata, ladder(moi), other, specify, plough and yoke, power pump, power tiller, rickshaw, rower pump, saw, sewing machine, shallow tube well, sickle/dao/axe/spade, spray, thresher, tractor, treddle pump, weeder. \\[1ex]
\end{tabular}
}



\hspace{-1cm}\begin{minipage}[t]{14cm}
\hfil\textsc{\normalsize Table \refstepcounter{table}\thetable: ANCOVA estimation of productive assets\label{tab ANCOVA productive assets}}\\
\setlength{\tabcolsep}{1pt}
\setlength{\baselineskip}{8pt}
\renewcommand{\arraystretch}{.55}
\hfil\begin{tikzpicture}
\node (tbl) {\input{c:/data/GUK/analysis/save/EstimationMemo/ProdAssetANCOVAEstimationResults.tex}};
%\input{c:/dropbox/data/ramadan/save/tablecolortemplate.tex}
\end{tikzpicture}\\
\renewcommand{\arraystretch}{.8}
\setlength{\tabcolsep}{1pt}
\begin{tabular}{>{\hfill\scriptsize}p{1cm}<{}>{\hfill\scriptsize}p{.25cm}<{}>{\scriptsize}p{12cm}<{\hfill}}
Source:& \multicolumn{2}{l}{\scriptsize Estimated with GUK administrative and survey data.}\\
Notes: & 1. & ANCOVA estimates using administrative and survey data. Post treatment regressands are regressed on categorical variables, pre-treatment regressand and other covariates. \textsf{FloodInRd1} and \textsf{HeadLiterate0} are indicator variables for the presence of self reported damage by a flood at the baseline, and literacy of household head, respectively. \textsf{HHsize0} is household size at the baseline. \textsf{Large}, \textsf{LargeGrace}, \textsf{Cattle} are indicator variables of the \textsf{large}, \textsf{large grace}, and \textsf{cattle} arms, respectively. The default arm category is \textsf{traditional} arm. Only households that are observed for all 4 rounds are used. Households are continuing members and replacing members of early rejecters and received loans prior to 2015 Janunary. Productive assets do not include livestock. Regressions (1)-(3), (5)-(6) use only arm and calendar information. (4) and (7) use previous six month repayment and saving information which is lacking in rd 1, hence starts from rd 2.\\
& 2. & $P$ values in percentages in parenthesises. Standard errors are clustered at group (village) level.
%${}^{***}$, ${}^{**}$, ${}^{*}$ indicate statistical significance at 1\%, 5\%, 10\%, respetively. Standard errors are clustered at group (village) level.
\end{tabular}
\end{minipage}

\hspace{-1cm}\begin{minipage}[t]{14cm}
\hfil\textsc{\normalsize Table \refstepcounter{table}\thetable: ANCOVA estimation of productive assets by attributes\label{tab ANCOVA productive assets attributes}}\\
\setlength{\tabcolsep}{1pt}
\setlength{\baselineskip}{8pt}
\renewcommand{\arraystretch}{.55}
\hfil\begin{tikzpicture}
\node (tbl) {\input{c:/data/GUK/analysis/save/EstimationMemo/ProdAssetAttributesANCOVAEstimationResults.tex}};
%\input{c:/dropbox/data/ramadan/save/tablecolortemplate.tex}
\end{tikzpicture}\\
\renewcommand{\arraystretch}{.8}
\setlength{\tabcolsep}{1pt}
\begin{tabular}{>{\hfill\scriptsize}p{1cm}<{}>{\hfill\scriptsize}p{.25cm}<{}>{\scriptsize}p{12cm}<{\hfill}}
Source:& \multicolumn{2}{l}{\scriptsize Estimated with GUK administrative and survey data.}\\
Notes: & 1. & ANCOVA estimates using administrative and survey data. Post treatment regressands are regressed on categorical variables, pre-treatment regressand and other covariates. \textsf{FloodInRd1} and \textsf{HeadLiterate0} are indicator variables for the presence of self reported damage by a flood at the baseline, and literacy of household head, respectively. \textsf{HHsize0} is household size at the baseline. \textsf{Large}, \textsf{LargeGrace}, \textsf{Cattle} are indicator variables of the \textsf{large}, \textsf{large grace}, and \textsf{cattle} arms, respectively. The default arm category is \textsf{traditional} arm. Only households that are observed for all 4 rounds are used. Households are continuing members and replacing members of early rejecters and received loans prior to 2015 Janunary. Productive assets do not include livestock. Regressions (1)-(3), (5)-(6) use only arm and calendar information. (4) and (7) use previous six month repayment and saving information which is lacking in rd 1, hence starts from rd 2.\\
& 2. & $P$ values in percentages in parenthesises. Standard errors are clustered at group (village) level.
%${}^{***}$, ${}^{**}$, ${}^{*}$ indicate statistical significance at 1\%, 5\%, 10\%, respetively. Standard errors are clustered at group (village) level.
\end{tabular}
\end{minipage}

\hspace{-1cm}\begin{minipage}[t]{14cm}
\hfil\textsc{\normalsize Table \refstepcounter{table}\thetable: ANCOVA estimation of broad productive assets by period\label{tab ANCOVA productive assets timevarying}}\\
\setlength{\tabcolsep}{1pt}
\setlength{\baselineskip}{8pt}
\renewcommand{\arraystretch}{.55}
\hfil\begin{tikzpicture}
\node (tbl) {\input{c:/data/GUK/analysis/save/EstimationMemo/ProdAssetTimeVaryingANCOVAEstimationResults.tex}};
%\input{c:/dropbox/data/ramadan/save/tablecolortemplate.tex}
\end{tikzpicture}\\
\renewcommand{\arraystretch}{.8}
\setlength{\tabcolsep}{1pt}
\begin{tabular}{>{\hfill\scriptsize}p{1cm}<{}>{\hfill\scriptsize}p{.25cm}<{}>{\scriptsize}p{12cm}<{\hfill}}
Source:& \multicolumn{2}{l}{\scriptsize Estimated with GUK administrative and survey data.}\\
Notes: & 1. & ANCOVA estimates using administrative and survey data. Post treatment regressands are regressed on categorical variables, pre-treatment regressand and other covariates. \textsf{FloodInRd1} and \textsf{HeadLiterate0} are indicator variables for the presence of self reported damage by a flood at the baseline, and literacy of household head, respectively. \textsf{HHsize0} is household size at the baseline. \textsf{Large}, \textsf{LargeGrace}, \textsf{Cattle} are indicator variables of the \textsf{large}, \textsf{large grace}, and \textsf{cattle} arms, respectively. The default arm category is \textsf{traditional} arm. \textsf{rd2, rd3, rd4} are dummy variables for second, third, and fourth round of survey. Only households that are observed for all 4 rounds are used. Households are continuing members and replacing members of early rejecters and received loans prior to 2015 Janunary. Productive assets do not include livestock. Regressions (1)-(3), (5)-(6) use only arm and calendar information. (4) and (7) use previous six month repayment and saving information which is lacking in rd 1, hence starts from rd 2.\\
& 2. & $P$ values in percentages in parenthesises. Standard errors are clustered at group (village) level.
%${}^{***}$, ${}^{**}$, ${}^{*}$ indicate statistical significance at 1\%, 5\%, 10\%, respetively. Standard errors are clustered at group (village) level.
\end{tabular}
\end{minipage}

\hspace{-1cm}\begin{minipage}[t]{14cm}
\hfil\textsc{\normalsize Table \refstepcounter{table}\thetable: ANCOVA estimation of broad productive assets by attributes and period\label{tab ANCOVA productive assets timevarying attributes}}\\
\setlength{\tabcolsep}{1pt}
\setlength{\baselineskip}{8pt}
\renewcommand{\arraystretch}{.55}
\hfil\begin{tikzpicture}
\node (tbl) {\input{c:/data/GUK/analysis/save/EstimationMemo/ProdAssetTimeVaryingAttributesANCOVAEstimationResults.tex}};
%\input{c:/dropbox/data/ramadan/save/tablecolortemplate.tex}
\end{tikzpicture}\\
\renewcommand{\arraystretch}{.8}
\setlength{\tabcolsep}{1pt}
\begin{tabular}{>{\hfill\scriptsize}p{1cm}<{}>{\hfill\scriptsize}p{.25cm}<{}>{\scriptsize}p{12cm}<{\hfill}}
Source:& \multicolumn{2}{l}{\scriptsize Estimated with GUK administrative and survey data.}\\
Notes: & 1. & ANCOVA estimates using administrative and survey data. Post treatment regressands are regressed on categorical variables, pre-treatment regressand and other covariates. \textsf{FloodInRd1} and \textsf{HeadLiterate0} are indicator variables for the presence of self reported damage by a flood at the baseline, and literacy of household head, respectively. \textsf{HHsize0} is household size at the baseline. \textsf{Large}, \textsf{LargeGrace}, \textsf{Cattle} are indicator variables of the \textsf{large}, \textsf{large grace}, and \textsf{cattle} arms, respectively. The default arm category is \textsf{traditional} arm. \textsf{rd2, rd3, rd4} are dummy variables for second, third, and fourth round of survey. Only households that are observed for all 4 rounds are used. Households are continuing members and replacing members of early rejecters and received loans prior to 2015 Janunary.  Productive assets do not include livestock. Regressions (1)-(3), (5)-(6) use only arm and calendar information. (4) and (7) use previous six month repayment and saving information which is lacking in rd 1, hence starts from rd 2.\\
& 2. & $P$ values in percentages in parenthesises. Standard errors are clustered at group (village) level.
%${}^{***}$, ${}^{**}$, ${}^{*}$ indicate statistical significance at 1\%, 5\%, 10\%, respetively. Standard errors are clustered at group (village) level.
\end{tabular}
\end{minipage}


\hspace{-1cm}\begin{minipage}[t]{14cm}
\hfil\textsc{\normalsize Table \refstepcounter{table}\thetable: ANCOVA estimation of broad productive assets by attributes, poverty status, and period\label{tab ANCOVA productive assets timevarying poverty status attributes}}\\
\setlength{\tabcolsep}{1pt}
\setlength{\baselineskip}{8pt}
\renewcommand{\arraystretch}{.55}
\hfil\begin{tikzpicture}
\node (tbl) {\input{c:/data/GUK/analysis/save/EstimationMemo/ProdAssetTimeVaryingPovertyStatusAttributesANCOVAEstimationResults.tex}};
%\input{c:/dropbox/data/ramadan/save/tablecolortemplate.tex}
\end{tikzpicture}\\
\renewcommand{\arraystretch}{.8}
\setlength{\tabcolsep}{1pt}
\begin{tabular}{>{\hfill\scriptsize}p{1cm}<{}>{\hfill\scriptsize}p{.25cm}<{}>{\scriptsize}p{12cm}<{\hfill}}
Source:& \multicolumn{2}{l}{\scriptsize Estimated with GUK administrative and survey data.}\\
Notes: & 1. & ANCOVA estimates using administrative and survey data. Post treatment regressands are regressed on categorical variables, pre-treatment regressand and other covariates. \textsf{FloodInRd1} and \textsf{HeadLiterate0} are indicator variables for the presence of self reported damage by a flood at the baseline, and literacy of household head, respectively. \textsf{HHsize0} is household size at the baseline. \textsf{Large}, \textsf{LargeGrace}, \textsf{Cattle} are indicator variables of the \textsf{large}, \textsf{large grace}, and \textsf{cattle} arms, respectively. The default arm category is \textsf{traditional} arm. \textsf{rd2, rd3, rd4} are dummy variables for second, third, and fourth round of survey. Only households that are observed for all 4 rounds are used. Households are continuing members and replacing members of early rejecters and received loans prior to 2015 Janunary.  Productive assets do not include livestock. Regressions (1)-(3), (5)-(6) use only arm and calendar information. (4) and (7) use previous six month repayment and saving information which is lacking in rd 1, hence starts from rd 2.\\
& 2. & $P$ values in percentages in parenthesises. Standard errors are clustered at group (village) level.
%${}^{***}$, ${}^{**}$, ${}^{*}$ indicate statistical significance at 1\%, 5\%, 10\%, respetively. Standard errors are clustered at group (village) level.
\end{tabular}
\end{minipage}




\subsubsection{Net assets: Assets+Livestock-GUK Debt-Other Debts}

Keep households with baseline household asset information. For productive assets, all households have baseline information but ownership is spattered. Net assets = Assets + net saving - debt to GUK - debts to relatives and money lenders. Assets use only items observed for all 4 rounds for household assets \textit{including} radios and cassette players (which have possibly large errors). 


\begin{Schunk}
\begin{Soutput}


Number of obs by Arm and attrition
             AttritIn
Arm             2   3   4   9 Sum
  traditional   6   4  20 144 174
  large         5   2   1 192 200
  large grace  22   3   3 171 199
  cattle        5   5  13 177 200
  Sum          38  14  37 684 773


Number of obs by membership status and attrition
                      AttritIn
BStatus                  2   3   4   9 Sum
  borrower               8   6   8 578 600
  pure saver             0   0   0   0   0
  individual rejection   9   4   1  75  89
  group rejection        9   4   0  55  68
  rejection by flood    12   0  28   0  40
  Sum                   38  14  37 708 797
\end{Soutput}
\end{Schunk}


\begin{Schunk}
\begin{Soutput}
NeA1R2
\end{Soutput}
\begin{Soutput}
     NonNA
tee   FALSE TRUE  Sum
  1      14 1474 1488
  2       7 1391 1398
  3       8 1359 1367
  4       6 1178 1184
  Sum    35 5402 5437
\end{Soutput}
\begin{Soutput}
     NonNA
tee   FALSE TRUE  Sum
  1     548  940 1488
  2     137 1261 1398
  3      35 1332 1367
  4      12 1172 1184
  Sum   732 4705 5437
\end{Soutput}
\begin{Soutput}
             tee
Arm              2    3    4  Sum
  traditional   58   58   58  174
  large        131  131  131  393
  large grace  118  118  118  354
  cattle       118  118  118  354
  Sum          425  425  425 1275
\end{Soutput}
\begin{Soutput}
             tee
Arm              2    3    4  Sum
  traditional   58   58   58  174
  large        131  131  131  393
  large grace  118  118  118  354
  cattle       118  118  118  354
  Sum          425  425  425 1275
\end{Soutput}
\end{Schunk}









\mpage{\linewidth}{
\hfil\textsc{\footnotesize Figure \refstepcounter{figure}\thefigure: Net asset values using only complete panel households\label{fig CPNetAssetValues}}\\
\hfil\includegraphics[height = 6cm]{c:/data/GUK/analysis/program/figure/EstimationMemo/CompletePanelNetAssets.eps}\\
\renewcommand{\arraystretch}{1}
\hfil\begin{tabular}{>{\hfill\scriptsize}p{1cm}<{}>{\scriptsize}p{12cm}<{\hfill}}
Source: & Survey data.\\
Note:& Net asset values = total gross asset values - debt outstanding. Debt outstanding takes the value of the month immediately after the respective survey round interview. Net assets uses only assets observed for all 4 rounds in household assets.\\[1ex]
\end{tabular}
}

\mpage{\linewidth}{
\hfil\textsc{\footnotesize Figure \refstepcounter{figure}\thefigure: Net asset values at round 1 using only complete panel households\label{fig CPNetAssetValuesAtRd1}}\\
\hfil\includegraphics[width = 10cm]{c:/data/GUK/analysis/program/figure/EstimationMemo/CompletePanelNetAssetsAtRd1.pdf}\\
\renewcommand{\arraystretch}{1}
\hfil\begin{tabular}{>{\hfill\scriptsize}p{1cm}<{}>{\scriptsize}p{12cm}<{\hfill}}
Source: & Survey data.\\
Note:& Net asset values = total gross asset values - debt outstanding. Debt outstanding takes the value of the month immediately after the respective survey round interview. Net assets uses only assets observed for all 4 rounds in household assets.\\[1ex]
\end{tabular}
}



\hspace{-1cm}\begin{minipage}[t]{14cm}
\hfil\textsc{\normalsize Table \refstepcounter{table}\thetable: ANCOVA estimation of complete panel net assets\label{tab ANCOVA complete panel net assets}}\\
\setlength{\tabcolsep}{1pt}
\setlength{\baselineskip}{8pt}
\renewcommand{\arraystretch}{.55}
\hfil\begin{tikzpicture}
\node (tbl) {\input{c:/data/GUK/analysis/save/EstimationMemo/CompletePanelNetAssetsANCOVAEstimationResults.tex}};
%\input{c:/dropbox/data/ramadan/save/tablecolortemplate.tex}
\end{tikzpicture}\\
\renewcommand{\arraystretch}{.8}
\setlength{\tabcolsep}{1pt}
\begin{tabular}{>{\hfill\scriptsize}p{1cm}<{}>{\hfill\scriptsize}p{.25cm}<{}>{\scriptsize}p{12cm}<{\hfill}}
Source:& \multicolumn{2}{l}{\scriptsize Estimated with GUK administrative and survey data.}\\
Notes: & 1. & ANCOVA estimates using administrative and survey data. Post treatment regressands are regressed on categorical variables, pre-treatment regressand and other covariates. \textsf{FloodInRd1} and \textsf{HeadLiterate0} are indicator variables for the presence of self reported damage by a flood at the baseline, and literacy of household head, respectively. \textsf{HHsize0} is household size at the baseline. We annotate the number of periods that a household is observed with \textsf{T}. The total number of households is shown for each values of \textsf{T}. \textsf{T=4} indicates the number of households with complete panel information, \textsf{T=3} indicates number of households observed three times, \textsf{T=2} indicates the number of households observed twice. \textsf{N} indicates total number of observations used in ANCOVA estimation, or \textsf{N$=$1$\times$(T=2)+2$\times$(T=3)+3$\times$(T=4)}.  \textsf{Large}, \textsf{LargeGrace}, \textsf{Cattle} are indicator variables of the \textsf{large}, \textsf{large grace}, and \textsf{cattle} arms, respectively. The default arm category is \textsf{traditional} arm. Net assets use only assets observed for all 4 rounds in household assets. Household assets do not include livestock. Regressions (1)-(3), (5)-(6) use only arm and calendar information. (4) and (7) use previous six month repayment and saving information which is lacking in rd 1, hence starts from rd 2.\\
& 2. & $P$ values in percentages in parenthesises. Standard errors are clustered at group (village) level.
%${}^{***}$, ${}^{**}$, ${}^{*}$ indicate statistical significance at 1\%, 5\%, 10\%, respetively. Standard errors are clustered at group (village) level.
\end{tabular}
\end{minipage}

\hspace{-1cm}\begin{minipage}[t]{14cm}
\hfil\textsc{\normalsize Table \refstepcounter{table}\thetable: ANCOVA estimation of complete panel net assets by attributes\label{tab ANCOVA narrow complete panel net assets attributes}}\\
\setlength{\tabcolsep}{1pt}
\setlength{\baselineskip}{8pt}
\renewcommand{\arraystretch}{.55}
\hfil\begin{tikzpicture}
\node (tbl) {\input{c:/data/GUK/analysis/save/EstimationMemo/CompletePanelNetAssetsAttributesANCOVAEstimationResults.tex}};
%\input{c:/dropbox/data/ramadan/save/tablecolortemplate.tex}
\end{tikzpicture}\\
\renewcommand{\arraystretch}{.8}
\setlength{\tabcolsep}{1pt}
\begin{tabular}{>{\hfill\scriptsize}p{1cm}<{}>{\hfill\scriptsize}p{.25cm}<{}>{\scriptsize}p{12cm}<{\hfill}}
Source:& \multicolumn{2}{l}{\scriptsize Estimated with GUK administrative and survey data.}\\
Notes: & 1. & ANCOVA estimates using administrative and survey data. Post treatment regressands are regressed on categorical variables, pre-treatment regressand and other covariates. \textsf{FloodInRd1} and \textsf{HeadLiterate0} are indicator variables for the presence of self reported damage by a flood at the baseline, and literacy of household head, respectively. \textsf{HHsize0} is household size at the baseline. We annotate the number of periods that a household is observed with \textsf{T}. The total number of households is shown for each values of \textsf{T}. \textsf{T=4} indicates the number of households with complete panel information, \textsf{T=3} indicates number of households observed three times, \textsf{T=2} indicates the number of households observed twice. \textsf{N} indicates total number of observations used in ANCOVA estimation, or \textsf{N$=$1$\times$(T=2)+2$\times$(T=3)+3$\times$(T=4)}.  \textsf{Upfront} is an indicator variable of the arm with an upfront large disbursement, \textsf{WithGrace} is an indicator variable of the arm with a grace period, \textsf{InKind} is an indicator variable of the arm which lends a heifer. Net assets use only assets observed for all 4 rounds in household assets. Household assets do not include livestock. Regressions (1)-(3), (5)-(6) use only arm and calendar information. (4) and (7) use previous six month repayment and saving information which is lacking in rd 1, hence starts from rd 2.\\
& 2. & $P$ values in percentages in parenthesises. Standard errors are clustered at group (village) level.
%${}^{***}$, ${}^{**}$, ${}^{*}$ indicate statistical significance at 1\%, 5\%, 10\%, respetively. Standard errors are clustered at group (village) level.
\end{tabular}
\end{minipage}

\hspace{-1cm}\begin{minipage}[t]{14cm}
\hfil\textsc{\normalsize Table \refstepcounter{table}\thetable: ANCOVA estimation of complete panel net assets by period\label{tab ANCOVA narrow complete panel net assets timevarying}}\\
\setlength{\tabcolsep}{1pt}
\setlength{\baselineskip}{8pt}
\renewcommand{\arraystretch}{.55}
\hfil\begin{tikzpicture}
\node (tbl) {\input{c:/data/GUK/analysis/save/EstimationMemo/CompletePanelNetAssetsTimeVaryingANCOVAEstimationResults.tex}};
%\input{c:/dropbox/data/ramadan/save/tablecolortemplate.tex}
\end{tikzpicture}\\
\renewcommand{\arraystretch}{.8}
\setlength{\tabcolsep}{1pt}
\begin{tabular}{>{\hfill\scriptsize}p{1cm}<{}>{\hfill\scriptsize}p{.25cm}<{}>{\scriptsize}p{12cm}<{\hfill}}
Source:& \multicolumn{2}{l}{\scriptsize Estimated with GUK administrative and survey data.}\\
Notes: & 1. & ANCOVA estimates using administrative and survey data. Post treatment regressands are regressed on categorical variables, pre-treatment regressand and other covariates. \textsf{FloodInRd1} and \textsf{HeadLiterate0} are indicator variables for the presence of self reported damage by a flood at the baseline, and literacy of household head, respectively. \textsf{HHsize0} is household size at the baseline. We annotate the number of periods that a household is observed with \textsf{T}. The total number of households is shown for each values of \textsf{T}. \textsf{T=4} indicates the number of households with complete panel information, \textsf{T=3} indicates number of households observed three times, \textsf{T=2} indicates the number of households observed twice. \textsf{N} indicates total number of observations used in ANCOVA estimation, or \textsf{N$=$1$\times$(T=2)+2$\times$(T=3)+3$\times$(T=4)}.  \textsf{Large}, \textsf{LargeGrace}, \textsf{Cattle} are indicator variables of the \textsf{large}, \textsf{large grace}, and \textsf{cattle} arms, respectively. The default arm category is \textsf{traditional} arm. \textsf{rd2, rd3, rd4} are dummy variables for second, third, and fourth round of survey. Net assets use only assets observed for all 4 rounds in household assets. Household assets do not include livestock. Regressions (1)-(3), (5)-(6) use only arm and calendar information. (4) and (7) use previous six month repayment and saving information which is lacking in rd 1, hence starts from rd 2.\\
& 2. & $P$ values in percentages in parenthesises. Standard errors are clustered at group (village) level.
%${}^{***}$, ${}^{**}$, ${}^{*}$ indicate statistical significance at 1\%, 5\%, 10\%, respetively. Standard errors are clustered at group (village) level.
\end{tabular}
\end{minipage}

\hspace{-1cm}\begin{minipage}[t]{14cm}
\hfil\textsc{\normalsize Table \refstepcounter{table}\thetable: ANCOVA estimation of complete panel net assets by attributes and period\label{tab ANCOVA narrow complete panel net assets timevarying attributes}}\\
\setlength{\tabcolsep}{1pt}
\setlength{\baselineskip}{8pt}
\renewcommand{\arraystretch}{.55}
\hfil\begin{tikzpicture}
\node (tbl) {\input{c:/data/GUK/analysis/save/EstimationMemo/CompletePanelNetAssetsTimeVaryingAttributesANCOVAEstimationResults.tex}};
%\input{c:/dropbox/data/ramadan/save/tablecolortemplate.tex}
\end{tikzpicture}\\
\renewcommand{\arraystretch}{.8}
\setlength{\tabcolsep}{1pt}
\begin{tabular}{>{\hfill\scriptsize}p{1cm}<{}>{\hfill\scriptsize}p{.25cm}<{}>{\scriptsize}p{12cm}<{\hfill}}
Source:& \multicolumn{2}{l}{\scriptsize Estimated with GUK administrative and survey data.}\\
Notes: & 1. & ANCOVA estimates using administrative and survey data. Post treatment regressands are regressed on categorical variables, pre-treatment regressand and other covariates. \textsf{FloodInRd1} and \textsf{HeadLiterate0} are indicator variables for the presence of self reported damage by a flood at the baseline, and literacy of household head, respectively. \textsf{HHsize0} is household size at the baseline. We annotate the number of periods that a household is observed with \textsf{T}. The total number of households is shown for each values of \textsf{T}. \textsf{T=4} indicates the number of households with complete panel information, \textsf{T=3} indicates number of households observed three times, \textsf{T=2} indicates the number of households observed twice. \textsf{N} indicates total number of observations used in ANCOVA estimation, or \textsf{N$=$1$\times$(T=2)+2$\times$(T=3)+3$\times$(T=4)}.  \textsf{Upfront} is an indicator variable of the arm with an upfront large disbursement, \textsf{WithGrace} is an indicator variable of the arm with a grace period, \textsf{InKind} is an indicator variable of the arm which lends a heifer. \textsf{rd2, rd3, rd4} are dummy variables for second, third, and fourth round of survey. Net assets use only assets observed for all 4 rounds in household assets. Household assets do not include livestock. Regressions (1)-(3), (5)-(6) use only arm and calendar information. (4) and (7) use previous six month repayment and saving information which is lacking in rd 1, hence starts from rd 2.\\
& 2. & $P$ values in percentages in parenthesises. Standard errors are clustered at group (village) level.
%${}^{***}$, ${}^{**}$, ${}^{*}$ indicate statistical significance at 1\%, 5\%, 10\%, respetively. Standard errors are clustered at group (village) level.
\end{tabular}
\end{minipage}


\hspace{-1cm}\begin{minipage}[t]{14cm}
\hfil\textsc{\normalsize Table \refstepcounter{table}\thetable: ANCOVA estimation of complete panel net assets by arm, poverty status, and period\label{tab ANCOVA narrow complete panel net assets timevarying poverty status}}\\
\setlength{\tabcolsep}{1pt}
\setlength{\baselineskip}{8pt}
\renewcommand{\arraystretch}{.55}
\hfil\begin{tikzpicture}
\node (tbl) {\input{c:/data/GUK/analysis/save/EstimationMemo/CompletePanelNetAssetsTimeVaryingPovertyStatusANCOVAEstimationResults.tex}};
%\input{c:/dropbox/data/ramadan/save/tablecolortemplate.tex}
\end{tikzpicture}\\
\renewcommand{\arraystretch}{.8}
\setlength{\tabcolsep}{1pt}
\begin{tabular}{>{\hfill\scriptsize}p{1cm}<{}>{\hfill\scriptsize}p{.25cm}<{}>{\scriptsize}p{12cm}<{\hfill}}
Source:& \multicolumn{2}{l}{\scriptsize Estimated with GUK administrative and survey data.}\\
Notes: & 1. & ANCOVA estimates using administrative and survey data. Post treatment regressands are regressed on categorical variables, pre-treatment regressand and other covariates. \textsf{FloodInRd1} and \textsf{HeadLiterate0} are indicator variables for the presence of self reported damage by a flood at the baseline, and literacy of household head, respectively. \textsf{HHsize0} is household size at the baseline. We annotate the number of periods that a household is observed with \textsf{T}. The total number of households is shown for each values of \textsf{T}. \textsf{T=4} indicates the number of households with complete panel information, \textsf{T=3} indicates number of households observed three times, \textsf{T=2} indicates the number of households observed twice. \textsf{N} indicates total number of observations used in ANCOVA estimation, or \textsf{N$=$1$\times$(T=2)+2$\times$(T=3)+3$\times$(T=4)}.  \textsf{UltraPoor} is an indicator variable if the household is classified as the ultra poor. \textsf{Large}, \textsf{LargeGrace}, \textsf{Cattle} are indicator variables of the \textsf{large}, \textsf{large grace}, and \textsf{cattle} arms, respectively. The default arm category is \textsf{traditional} arm. \textsf{rd2, rd3, rd4} are dummy variables for second, third, and fourth round of survey. Net assets use only assets observed for all 4 rounds in household assets. Household assets do not include livestock. Regressions (1)-(3), (5)-(6) use only arm and calendar information. (4) and (7) use previous six month repayment and saving information which is lacking in rd 1, hence starts from rd 2.\\
& 2. & $P$ values in percentages in parenthesises. Standard errors are clustered at group (village) level.
%${}^{***}$, ${}^{**}$, ${}^{*}$ indicate statistical significance at 1\%, 5\%, 10\%, respetively. Standard errors are clustered at group (village) level.
\end{tabular}
\end{minipage}

\hspace{-1cm}\begin{minipage}[t]{14cm}
\hfil\textsc{\normalsize Table \refstepcounter{table}\thetable: ANCOVA estimation of complete panel net assets by attributes, poverty status, and period\label{tab ANCOVA narrow complete panel net assets timevarying poverty status attributes}}\\
\setlength{\tabcolsep}{1pt}
\setlength{\baselineskip}{8pt}
\renewcommand{\arraystretch}{.55}
\hfil\begin{tikzpicture}
\node (tbl) {\input{c:/data/GUK/analysis/save/EstimationMemo/CompletePanelNetAssetsTimeVaryingPovertyStatusAttributesANCOVAEstimationResults.tex}};
%\input{c:/dropbox/data/ramadan/save/tablecolortemplate.tex}
\end{tikzpicture}\\
\renewcommand{\arraystretch}{.8}
\setlength{\tabcolsep}{1pt}
\begin{tabular}{>{\hfill\scriptsize}p{1cm}<{}>{\hfill\scriptsize}p{.25cm}<{}>{\scriptsize}p{12cm}<{\hfill}}
Source:& \multicolumn{2}{l}{\scriptsize Estimated with GUK administrative and survey data.}\\
Notes: & 1. & ANCOVA estimates using administrative and survey data. Post treatment regressands are regressed on categorical variables, pre-treatment regressand and other covariates. \textsf{FloodInRd1} and \textsf{HeadLiterate0} are indicator variables for the presence of self reported damage by a flood at the baseline, and literacy of household head, respectively. \textsf{HHsize0} is household size at the baseline. We annotate the number of periods that a household is observed with \textsf{T}. The total number of households is shown for each values of \textsf{T}. \textsf{T=4} indicates the number of households with complete panel information, \textsf{T=3} indicates number of households observed three times, \textsf{T=2} indicates the number of households observed twice. \textsf{N} indicates total number of observations used in ANCOVA estimation, or \textsf{N$=$1$\times$(T=2)+2$\times$(T=3)+3$\times$(T=4)}.  \textsf{UltraPoor} is an indicator variable if the household is classified as the ultra poor. \textsf{Upfront} is an indicator variable of the arm with an upfront large disbursement, \textsf{WithGrace} is an indicator variable of the arm with a grace period, \textsf{InKind} is an indicator variable of the arm which lends a heifer. \textsf{rd2, rd3, rd4} are dummy variables for second, third, and fourth round of survey. Net assets use only assets observed for all 4 rounds in household assets. Household assets do not include livestock. Regressions (1)-(3), (5)-(6) use only arm and calendar information. (4) and (7) use previous six month repayment and saving information which is lacking in rd 1, hence starts from rd 2.\\
& 2. & $P$ values in percentages in parenthesises. Standard errors are clustered at group (village) level.
%${}^{***}$, ${}^{**}$, ${}^{*}$ indicate statistical significance at 1\%, 5\%, 10\%, respetively. Standard errors are clustered at group (village) level.
\end{tabular}
\end{minipage}


%\subsubsection{Narrow net assets: NarrowAssets+Livestock-GUK Debt-Other Debts}



\section{Summarising results}


\subsection{Counting observations used in ANCOVA estimation}

\begin{Schunk}
\begin{Soutput}
    survey         Arm              BStatus   Num     N
     <num>      <fctr>               <fctr> <int> <int>
 1:      1 traditional             borrower     1   109
 2:      2 traditional             borrower     1   108
 3:      3 traditional             borrower     1   108
 4:      4 traditional             borrower     1   107
 5:      1 traditional individual rejection     1    31
 6:      2 traditional individual rejection     1    26
 7:      3 traditional individual rejection     1    26
 8:      4 traditional individual rejection     1    25
 9:      1 traditional      group rejection     1    40
10:      2 traditional      group rejection     1    39
11:      3 traditional      group rejection     1    36
12:      4 traditional      group rejection     1    36
13:      1 traditional   rejection by flood     1    20
14:      2 traditional   rejection by flood     1    17
15:      3 traditional   rejection by flood     1    18
16:      1       large             borrower     1   171
17:      2       large             borrower     1   163
18:      3       large             borrower     1   165
19:      4       large             borrower     1   164
20:      1       large individual rejection     1     9
21:      2       large individual rejection     1     8
22:      3       large individual rejection     1     9
23:      4       large individual rejection     1     9
24:      1       large      group rejection     1    20
25:      2       large      group rejection     1    20
26:      3       large      group rejection     1    19
27:      4       large      group rejection     1    19
28:      1 large grace             borrower     1   167
29:      2 large grace             borrower     1   163
30:      3 large grace             borrower     1   163
31:      4 large grace             borrower     1   160
32:      1 large grace individual rejection     1    13
33:      2 large grace individual rejection     1     9
34:      3 large grace individual rejection     1    11
35:      4 large grace individual rejection     1    11
36:      1 large grace      group rejection     1    10
37:      1 large grace   rejection by flood     1    10
38:      1      cattle             borrower     1   153
39:      2      cattle             borrower     1   151
40:      3      cattle             borrower     1   150
41:      4      cattle             borrower     1   147
42:      1      cattle individual rejection     1    37
43:      2      cattle individual rejection     1    29
44:      3      cattle individual rejection     1    30
45:      4      cattle individual rejection     1    30
46:      1      cattle   rejection by flood     1    10
47:      2      cattle   rejection by flood     1    10
48:      3      cattle   rejection by flood     1    10
    survey         Arm              BStatus   Num     N
\end{Soutput}
\begin{Soutput}
      Arm    hhid   tee MaxTee AttritIn              BStatus creditstatus
   <fctr>   <num> <int>  <int>    <int>               <fctr>       <fctr>
1: cattle 7054319     1      3        9 individual rejection           No
      Mgroup
      <fctr>
1: drop outs
\end{Soutput}
\begin{Soutput}
          Arm      TradGroup                 BStatus       hhid          survey
 traditional:5   planned:0   borrower            :1   Min.   : 7031513   1:4   
 large      :0   twice  :0   pure saver          :0   1st Qu.: 7054408   3:1   
 large grace:0   double :0   individual rejection:0   Median : 7054413         
 cattle     :0   NA's   :5   group rejection     :0   Mean   :36912148         
                             rejection by flood  :4   3rd Qu.:81710203         
                                                      Max.   :81710203         
                                                                               
 NLAssetAmount 
 Min.   :1960  
 1st Qu.:2780  
 Median :3600  
 Mean   :4040  
 3rd Qu.:5080  
 Max.   :6560  
 NA's   :2     
\end{Soutput}
\begin{Soutput}
           Arm            BStatus     hhid survey NumCows
        <fctr>             <fctr>    <num>  <num>   <int>
1: traditional           borrower  7031513      1       1
2: traditional rejection by flood  7054408      1       0
3: traditional rejection by flood  7054413      1       0
4: traditional rejection by flood 81710203      1       2
5: traditional rejection by flood 81710203      3       2
\end{Soutput}
\begin{Soutput}
Key: <hhid, tee>
Empty data.table (0 rows and 3 cols): BStatus,hhid,tee
\end{Soutput}
\begin{Soutput}
    survey              BStatus   Num     N
     <num>               <fctr> <int> <int>
 1:      1             borrower     1   102
 2:      2             borrower     1   106
 3:      3             borrower     1   108
 4:      4             borrower     1   107
 5:      1 individual rejection     1    28
 6:      2 individual rejection     1    26
 7:      3 individual rejection     1    26
 8:      4 individual rejection     1    25
 9:      1      group rejection     1    35
10:      2      group rejection     1    39
11:      3      group rejection     1    36
12:      4      group rejection     1    36
13:      1   rejection by flood     1    19
14:      2   rejection by flood     1    17
15:      3   rejection by flood     1    18
\end{Soutput}
\begin{Soutput}
            used   (Mb) gc trigger   (Mb) limit (Mb)  max used   (Mb)
Ncells   2843586  151.9    4521784  241.5         NA   4521784  241.5
Vcells 375189468 2862.5  628440838 4794.7      56320 413311950 3153.4
\end{Soutput}
\begin{Soutput}
[1] 1
\end{Soutput}
\begin{Soutput}
[1] 10
\end{Soutput}
\begin{Soutput}
Warning in rbind(c("", "&", rbind(paste0("\\makebox[", hcenter, unit, "]{", : number of columns of result is not a multiple of vector length (arg 1)
\end{Soutput}
\begin{Soutput}
Warning in rbind(c("", "&", rbind(paste0("\\makebox[", hcenter, unit, "]{", : number of columns of result is not a multiple of vector length (arg 1)
\end{Soutput}
\begin{Soutput}
Warning in rbind(c("", "&", rbind(paste0("\\makebox[", hcenter, unit, "]{", : number of columns of result is not a multiple of vector length (arg 1)
\end{Soutput}
\end{Schunk}

\renewcommand{\arraystretch}{.55}
\mpage{\linewidth}{
\hfil\textsc{\footnotesize Table \refstepcounter{table}\thetable: Number of observations by borrower status and arm\label{tab NumObsByBStatusArmFile}}\\
\hfil\input{c:/data/GUK/analysis/program/table/EstimationMemo/NumObsByBStatusArmFile.tex}\\
\renewcommand{\arraystretch}{1}
\hfil\begin{tabular}{>{\hfill\scriptsize}p{1cm}<{}>{\scriptsize}p{12cm}<{\hfill}}
Source: & Survey data.\\
Note:&  \\[1ex]
\end{tabular}
}

\renewcommand{\arraystretch}{.55}
\mpage{\linewidth}{
\hfil\textsc{\footnotesize Table \refstepcounter{table}\thetable: Number of observations used in estimation by borrower status and arm at period 1\label{tab NumObsByBStatusArmRegUsedMin}}\\
\hfil\input{c:/data/GUK/analysis/program/table/EstimationMemo/NumObsByBStatusArmRegUsedMin.tex}\\
\renewcommand{\arraystretch}{1}
\hfil\begin{tabular}{>{\hfill\scriptsize}p{1cm}<{}>{\scriptsize}p{12cm}<{\hfill}}
Source: & Survey data.\\
Note:&  \\[1ex]
\end{tabular}
}

\renewcommand{\arraystretch}{.55}
\mpage{\linewidth}{
\hfil\textsc{\footnotesize Table \refstepcounter{table}\thetable: Number of observations used in estimation by borrower status and arm at last period\label{tab NumObsByBStatusArmRegUsedMax}}\\
\hfil\input{c:/data/GUK/analysis/program/table/EstimationMemo/NumObsByBStatusArmRegUsedMax.tex}\\
\renewcommand{\arraystretch}{1}
\hfil\begin{tabular}{>{\hfill\scriptsize}p{1cm}<{}>{\scriptsize}p{12cm}<{\hfill}}
Source: & Survey data.\\
Note:&  \\[1ex]
\end{tabular}
}


\subsection{IGA}

IGA info is from \textsf{\scriptsize c:/data/GUK/received/cleaned\_by\_RA/GUKAdminstrativeData.dta}. 




In \textsf{traditional} arm, there are 33 borrowing members who report cattle as their first IGA, and 76 borrowing members (69.72\%) who report other than cattle as their first IGA. This contrasts with the non-\textsf{traditional} arms that 466 borrowing members who report cattle as their first IGA and 25 borrowing members (5.09\%) other than cattle as their first IGA. 









\mpage{\linewidth}{
\hfil\textsc{\footnotesize Figure \refstepcounter{figure}\thefigure: First IGA choices \label{fig IGAChoices}}\\
\hfil\includegraphics[height = 5cm]{c:/data/GUK/analysis/program/figure/EstimationMemo/FirstIGAChoices.pdf}\\
\renewcommand{\arraystretch}{1}
\hfil\begin{tabular}{>{\hfill\scriptsize}p{1cm}<{}>{\scriptsize}p{12cm}<{\hfill}}
Source: & Survey data.\\
Note:&  The first income generating activity (IGA) choices are plotted. The rows headed by `$n=1, 2, 3$' indicate there are $n$ project(s) owned by the household, and displayed type of project on the horizontal axis shows the contents of first project that was invested. \\[1ex]
\end{tabular}
}

\mpage{\linewidth}{
\hfil\textsc{\footnotesize Figure \refstepcounter{figure}\thefigure: All IGA choices \label{fig All IGAChoices}}\\
\hfil\includegraphics[height = 5cm]{c:/data/GUK/analysis/program/figure/EstimationMemo/AllIGAChoices.pdf}\\
\renewcommand{\arraystretch}{1}
\hfil\begin{tabular}{>{\hfill\scriptsize}p{1cm}<{}>{\scriptsize}p{12cm}<{\hfill}}
Source: & Survey data.\\
Note:&  \\[1ex]
\end{tabular}
}

\mpage{\linewidth}{
\hfil\textsc{\footnotesize Figure \refstepcounter{figure}\thefigure: All IGA choices (collapsed view)\label{fig All IGAChoicesCollapsed}}\\
\hfil\includegraphics[height = 5cm]{c:/data/GUK/analysis/program/figure/EstimationMemo/AllIGAChoicesCollapsed.pdf}\\
\renewcommand{\arraystretch}{1}
\hfil\begin{tabular}{>{\hfill\scriptsize}p{1cm}<{}>{\scriptsize}p{12cm}<{\hfill}}
Source: & Survey data.\\
Note:&  \\[1ex]
\end{tabular}
}


\subsection{Graphs}

Cumulative changes of non-traditional arm up to $t$ is given by $\mbox{(Intercept)}+b_{\mbox{\scriptsize Arm}}+b_{\mbox{\scriptsize $t$}}+b_{\mbox{\scriptsize Arm*$t$}}.$  This is given by \textsf{Intercept+Arm+TimeX+Arm.TimeX}. For the traditional arm, it is given by $\mbox{(Intercept)}+b_{\mbox{\scriptsize $t$}}.$

Time-varying impacts relative to the traditional arm is given by $b_{\mbox{\scriptsize Arm}}+b_{\mbox{\scriptsize Arm*$t$}}.$ This is given by \textsf{Arm+Arm.TimeX}. 


Need to run construct confi manually and run EstimationMemo.rnw again to draw error bar charts. To compute linear functions of estimated parameters, we use a vector \textsf{hv} giving linear combinations, covariance matrix of the regression \textsf{thisV}, and run $Wald$ tests with:\\
{\verb+  glht(model=thisreg, linfct = matrix(hv, byrow = T, nrow=1),+ \\
\verb+    alternative="two.sided", vcov.=thisV)+}



\renewcommand{\arraystretch}{.8}
\hfil\begin{tabular}{>{\footnotesize}p{1.75cm}<{}
>{\footnotesize}p{6cm}<{}
>{\footnotesize}p{5cm}<{}}
\rowcolor{gray90}
\hfil Object & \hfil What it does & \hfil Note\\
 hvT0 &  \mpage{6cm}{Picks covariates to test overall change.\\
  \texttt{"\textbackslash\textbackslash(Intercept\textbackslash\textbackslash)"\setlength{\baselineskip}{8pt}} }& \\
 hvN0 &  \mpage{6cm}{Overall mean impact of each non-traditional arm.\\
  \texttt{"\textbackslash\textbackslash(Intercept\textbackslash\textbackslash)", "dummyInKind"}\setlength{\baselineskip}{8pt}} & \\
 hvN1 &  \mpage{6cm}{Difference of period 2 Arm relative to period 2 trad.\\
  \texttt{"dummyInKind"}\setlength{\baselineskip}{8pt}}& \\
 hvTinT &  \mpage{6cm}{Picks covariates to test changes in period t relative to period 2.\\
  \texttt{"Time.4"}\setlength{\baselineskip}{8pt}}  & \\
  hv & \mpage{6cm}{Collects all coefficients by far to compute changes.\\
 \texttt{\textbackslash\textbackslash(Intercept\textbackslash\textbackslash) + Time.T}\setlength{\baselineskip}{8pt}} & hv $<-$ hvT0 + hvTinT\\
 hvNinT &  \mpage{6cm}{Picks covariates to test changes in period t relative to period 2 trad.\\ 
  \texttt{"Time.4", "dummyInKind.Time4"}\setlength{\baselineskip}{8pt}} &  Use this if period 2 trad is the reference.\\
 dhvNinT &  \mpage{6cm}{Difference relative to concurrent trad.\\
 \texttt{"dummyInKind.Time4"}\setlength{\baselineskip}{8pt}} &  Marginal difference between g and trad in period T.\setlength{\baselineskip}{8pt}\\
 cumNrelativeT &  \mpage{6cm}{Cumulative difference relative to concurrent trad.\\
  \texttt{"dummyInKind.Time2"+"dummyInKind.Time3"}\\\texttt{+"dummyInKind.Time4"}\setlength{\baselineskip}{8pt}} &  \textsf{cumstrings} adds dummyInKind.TimeX as period loops goes, with paste(cumstrings, paste0("\^", covadd.nontrad[[i]][2], "\$"), sep = "|")\setlength{\baselineskip}{8pt}\\
 periNrelativeT &  \mpage{6cm}{Periodwise difference relative to concurrent trad.\\
  \texttt{"dummyInKind"+"dummyInKind.TimeX"}\setlength{\baselineskip}{8pt}} &  Total difference between g and trad in time X. Period X effects relative to trad in perid X.  "dummyInKind" is stored in peristrings at hvN1\setlength{\baselineskip}{8pt}\\
 hvN2 &  \mpage{6cm}{Nontrad gross mean in period t.\\
 \texttt{\textbackslash\textbackslash(Intercept\textbackslash\textbackslash)+TimeX+TimeX.Arm}\\\texttt{=hvT0 + hvNinT}\setlength{\baselineskip}{8pt}} & Baseline trad + change relative to period 2 trad.\setlength{\baselineskip}{8pt}
\end{tabular}





\begin{Schunk}
\begin{Soutput}
            used   (Mb) gc trigger   (Mb) limit (Mb)  max used   (Mb)
Ncells   2846584  152.1    4521784  241.5         NA   4521784  241.5
Vcells 375589631 2865.6  756217989 5769.5      56320 669203001 5105.7
\end{Soutput}
\end{Schunk}
\begin{verbatim}
cumstrings0 <- peristrings0 <- paste0("^", covadd.nontrad[[1]][2], "$") & 
peristrings2 <- paste(peristrings0, paste0("^", covadd.nontrad[[i]][2], "$"), sep = "|")
\end{verbatim}

\hfil\begin{tabular}{
>{\scriptsize }p{1.25cm}<{}
>{\scriptsize }p{3.5cm}<{}
>{\scriptsize }p{5cm}<{}
>{\scriptsize }p{4cm}<{}
}
\rowcolor{gray90}
\hfil Object & \hfil What it does & \hfil Typical terms & \hfil Code\\
hvT0 & picks covariates to test overall change & \verb+[[1]]"\\(Intercept\\)"+ & covadd.trad[[1]]\\ 
hvN0 & Arm & \verb+[[1]]"\\(Intercept\\)", "dummyInKind"+ & covadd.nontrad[[1]]\\ 
hvN1 & Arm - trad, in period 2 & \verb+[[1]][2] "dummyInKind"+ & covadd.nontrad[[1]][2] \\ 
hvTinT & trad in each period - trad in period 2 & \verb+[[2]] "Time.4"+ & covadd.trad[[i]] \\ 
hv & trad in each period & \verb|intercept + Time.T| & hv = hvT0 + hvTinT\\ 
hvNinT & Arm in each period - nontrad in period 2. For period 2, period 2 level of Arm is reurned. & \verb+[[2]] "Time.4", "dummyInKind.Time4"+ & covadd.nontrad[[i]][c(1, 2)]\\ 
dhvNinT & Difference = 0 (of Arm g and trad in time X) & \verb+[[2]][1] "dummyInKind.Time4"+ & covadd.nontrad[[i]][2]\\
hvNinT2 & Arm in each period - trad in period 2 & \verb+[[2]] Arm + TimeX + Arm.TimeX+ & hvN1+covadd.nontrad[[i]][c(1, 2)]\\ 
periNrelativeT & Cumulative difference = 0 (of arm g and trad in time X) & "dummyInKind"+"dummyInKind.TimeX" for cumulative effects relative to trad in time X & periNrelativeT=hvN1+dhvNinT\\
hvN2 & nontrad gross mean in period t = cumulative trad + relative to concurrent trad = 0 & \verb|Intercept+TimeX| \verb|+Arm+TimeX.Arm| & hvT0 + hvNinT \\
\multicolumn{4}{l}{\scriptsize Impacts by baseline experience \texttt{j} }\\
dhvJ0 & Average difference = 0 (of experienced j and trad) & \verb+dummyAdiCattle0+ & j\\
dhvJinT & Difference = 0 (of experienced j and trad in period X) & \verb+dummyAdiCattle0.TimeX+ & \verb+paste0("^", j, ".Time.$")+\\
hvJinT & Cumulative difference = 0 (of experience j and trad in period X) & \verb|dummyAdiCattle0 + dummyAdiCattle0.TimeX| & dhvJ0+dhvJinT\\
hvJG0 & Average difference = 0 (of experience*arm j*g and arm g) & \verb+dummyAdiCattle0.Large+ & \verb+paste0("^", j, "$")+\\
dhvJGinT & Difference = 0 (of experience*arm j*g and arm g in period X) & \verb+dummyAdiCattle0.Large.TimeX+ & paste0(``\^'', j, ``.'', covadd.nontrad[[i]][2])\\
hvJGinT & Cumulative difference = 0 (of experience*arm j*g and arm g in period X) & \verb|dummyAdiCattle0.Large| \verb|+ dummyAdiCattle0.Large.TimeX| & hvJG0 + dhvJGinT\\
periJGinT & Cumulative difference = 0 (of experience*arm j*g and trad in time X) &  \verb|dummyLarge + dummyLarge.TimeX| \verb|+ dummyAdiCattle0 + dummyAdiCattle0.TimeX| \verb|+ dummyAdiCattle0.Large| \verb|+ dummyAdiCattle0.Large.TimeX| &  periNrelativeT+hvJinT+hvJGinT
\end{tabular}




\begin{Schunk}
\begin{Soutput}
要求されたパッケージ carData をロード中です
\end{Soutput}
\begin{Soutput}
要求されたパッケージ mvtnorm をロード中です
\end{Soutput}
\begin{Soutput}
要求されたパッケージ survival をロード中です
\end{Soutput}
\begin{Soutput}
要求されたパッケージ TH.data をロード中です
\end{Soutput}
\begin{Soutput}
要求されたパッケージ MASS をロード中です
\end{Soutput}
\begin{Soutput}

次のパッケージを付け加えます: 'TH.data'
\end{Soutput}
\begin{Soutput}
以下のオブジェクトは 'package:MASS' からマスクされています:

    geyser
\end{Soutput}
\end{Schunk}
\hfil\begin{tabular}{
>{\scriptsize }p{1.25cm}<{}
>{\scriptsize }p{2.25cm}<{}
>{\scriptsize }p{7cm}<{}
>{\scriptsize }p{2.5cm}<{}
}
\rowcolor{gray90}
\hfil Object & \hfil What it does & \hfil Formula & \hfil Code\\
hvMofTA & average change = 0 (of males in trad school i) & Intercept + School = dMofT & addcovaMofT[c(1, i)]\\
hvFofTA & average change = 0 (of females in trad school i) & Intercept + School + Female + School.Female = hvMofTA + dFofT & hvMofTA + addcovaFofT[c(1, i)]\\
hvMofNA & average change = 0 (of nontrad arm g at school i) & intercept + Arm + School + Arm.School = hvMofTA + dMofNA & hvMofTA + addcovaMofN[c(1, i)]\\
hvFofNA & average change = 0 (of nontrad Arm g at School i for females) & Intercept + Arm + School + Female + Arm.School + Arm.Female + School.Female + Arm.School.Female = Intercept + Arm + School + Arm.School (hvMofNA) + Female + School.Female (dFofT) + Arm.Female + Arm.School.Female (dFofNA) = hvMofNA + dFofT + dFofNA& hvMofNA + dFofT + addcovaFofN[c(1, i)]\\
hvMofN & average difference = 0 (of nontrad Arm g relative to trad, at School i) & hvMofNA - hvMofTA = Arm + Arm.School & dMofNA\\
hvFofN & difference = 0 (of nontrad Arm g females to trad females, at School i) & hvFofNA - hvFofTA =  Arm + School + Female + Arm.School + Arm.Female + School.Female + Arm.School.Female - (School + Female + School.Female) = Arm + Arm.School + Arm.Female + Arm.School.Female = hvMofNA + dFofNA & hvMofNA + dFofNA\\
hvMofTinT & difference = 0 (of trad in timeX relative to period 2, at School i) & School + TimeX + School.TimeX - School = TimeX + School.TimeX & addteeMofTinT[c(1 ,i)]\\
hvFofTinT & difference = 0 (of female trad in timeX relative to period 2, at School i) & School + TimeX + Female + School.TimeX + School.Female + Female.TimeX + School.Female.TimeX - (School + Female + School.Female)= TimeX + School.TimeX + Female.TimeX + School.Female.TimeX  = hvMofTinT + FofTinT & hvMofTinT + addteeFofTinT[c(1 ,i)]\\
hvMofTinTL & cumulative change = 0 (of trad at school i in period X) & Intercept + School + TimeX + School.TimeX = hvMofTA + hvMofTinT & \\
hvFofTinTL & cumulative change = 0 (of female trad at school i in period X)  & Intercept + School + Female + TimeX + School.TimeX + School.Female + Female.TimeX + School.Female.TimeX = hvFofTA + hvFofTinT & \\
dMofNinT & diff = 0 (of nontrad change relative to concurrent trad change, at school iin period X) & 
TimeX + Arm.TimeX + School.TimeX + Arm.School.TimeX - (TimeX + School.TimeX) = Arm.TimeX + Arm.School.TimeX & addteeMofNinT[c(1 ,i)]\\
dFofNinT& diff = 0 (of female nontrad change relative to concurrent female trad change, at school i in period X) & 
TimeX + Arm.TimeX + Female.TimeX + School.TimeX
+ Arm.School.TimeX + Arm.Female.TimeX + Female.School.TimeX 
+ Arm.School.Female.TimeX 
- (TimeX + Female.TimeX + School.TimeX + Female.School.TimeX)
= Arm.TimeX + Arm.School.TimeX + Arm.Female.TimeX
+ Arm.School.Female.TimeX 
= dMofNinT + Arm.Female.TimeX + Arm.School.Female.TimeX
= dMofNinT + dFofNinT0 & addteeFofNinT[c(1 ,i)]\\
hvMofNinT & difference = 0 (of nontrad relative to concurrent trad, at school iin period X) & Arm + School + TimeX + Arm.School + Arm.TimeX + School.TimeX + Arm.School.TimeX - (School + TimeX + School.TimeX)= Arm + Arm.School + Arm.TimeX + Arm.School.TimeX = hvMofN + dMofNinT & hvMofN + dMofNinT\\
hvFofNinT & difference = 0 (of female nontrad relative to concurrent female trad, at school iin period X) & 
Arm + School + Female + TimeX + Arm.School + Arm.Female + Arm.TimeX + School.Female + School.TimeX + Female.TimeX + Arm.School.Female + Arm.School.TimeX + Arm.Female.TimeX + School.Female.TimeX + Arm.School.Female.TimeX - (School + Female + TimeX + School.Female + School.TimeX + Female.TimeX) = Arm + Arm.School  (MofN) + Arm.Female + Arm.School.Female  (dFofN)  + Arm.TimeX + Arm.School.TimeX (dMofNinT)+ Arm.Female.TimeX + Arm.School.Female.TimeX  (dFofNinT0) = hvMofN + dFofN + dMofNinT + dFofNinT0  = hvMofN + dFofN + dFofNinT & hvMofN + dFofNA + dFofNinT\\
hvMofNinTL & cumulative change = 0 (of nontrad school i in period X) & (intercept) + Arm + School + TimeX + Arm.School + Arm.TimeX + School.TimeX + Arm.School.TimeX = hvMofTinTL +  hvMofNinT & \\
&& (intercept) + School + TimeX + School.TimeX  (hvMofTinTL)
+ Arm + Arm.School (hvMofN)
+ Arm.TimeX + Arm.School.TimeX (dMofNinT) & \\
&& (intercept) + Arm + School + Arm.School (MofNA)
TimeX + School.TimeX (MofTinT)
Arm.TimeX + Arm.School.TimeX (dMofNinT) & \\
hvFofNinTL & cumulative change = 0 (of female nontrad school i in period X) & 
(intercept) + Arm + School + Female + TimeX + 
Arm.School + Arm.Female + Arm.TimeX + 
School.Female  + School.TimeX + Female.TimeX + 
Arm.School.Female + Arm.School.TimeX + Arm.Female.TimeX + School.Female.TimeX + 
Arm.School.Female.TimeX &\\
 &  & 
hvMofNinTL  
+ Female + School.Female  (dFofT)
+ Female.TimeX + School.Female.TimeX (dFofTinT)
+ Arm.Female + Arm.School.Female  (dFofNA)
+  Arm.Female.TimeX + Arm.School.Female.TimeX  (dFofNinT)
= hvMofNinTL + dFofT + dFofTinT + dFofNA + dFofNinT
& 
\end{tabular}




dMofT=addcovaMofT[c(1, i)] (Intercept), dummyJunior, dummyHigh

dFofT=addcovaFofT[c(1, i)] Female, dummyJunior.Female, dummyHigh.Female

dMofNA=addcovaMofN[c(1, i)] dummyInKind, dummyInKind.dummyJunior, dummyInKind.dummyHigh

dFofNA=addcovaFofN[c(1, i)] dummyInKind.Female, dummyInKind.dummyJunior.Female, dummyInKind.dummyHigh.Female

hvMofTinT=addteeMofTinT[c(1 ,i)] Time.4, dummyJunior.Time4, dummyHigh.Time4

dFofTinT=addteeFofTinT[c(1 ,i)] Female.Time4, dummyJunior.Female.Time4, dummyHigh.Female.Time4

dMofNinT=addteeMofNinT[c(1 ,i)] dummyInKind.Time4, dummyInKind.dummyJunior.Time4, dummyInKind.dummyHigh.Time4

dFofNinT=addteeFofNinT[c(1 ,i)] dummyInKind.Female.Time4, dummyInKind.dummyJunior.Female.Time4, dummyInKind.dummyHigh.Female.Time4

\begin{Schunk}
\begin{Soutput}
                        num
FileName                   1   2   3   4   5   6
  Consumption              0 257 257   0   0   0
  ConsumptionOLS           0 361 361 361 361 361
  LabourIncome             0 361 361 361 361 361
  Land                     0 361 361 361 361 361
  Livestock                0 137 137 137   0   0
  NetAssets                0 361 361 361 361 361
  NetAssetsAnnualPrices    0 361 361 361 361 361
  NetAssetsByExperiencea   0 253 253 253   0   0
  NetAssetsByExperiencen   0 253 253 253   0   0
  NetAssetsByExperienceo   0 253 253 253   0   0
  NetAssetsExperience      0 541 541 541 541   0
  NetBroadAssets           0 355 355 355 355 355
  NetNLAssets              0 361 361 361 361 361
  NumCows                361 361 361 361   0   0
  NumCowsByExperiencea   137 137 137   0   0   0
  NumCowsByExperiencen   137 137 137   0   0   0
  NumCowsByExperienceo   137 137 137   0   0   0
  NumCowsExperience        0 541 541 541 541   0
\end{Soutput}
\begin{Soutput}
          FileName regtype   num attributes experience
            <fctr>  <fctr> <num>     <fctr>     <fctr>
 1:    Consumption       T     2      Large       None
 2:    Consumption       T     2      Large       None
 3:    Consumption       T     3      Large       None
 4:    Consumption       T     3      Large       None
 5: ConsumptionOLS       T     2      Large       None
 6: ConsumptionOLS       T     2      Large       None
 7: ConsumptionOLS       T     2      Large       None
 8: ConsumptionOLS       T     3      Large       None
 9: ConsumptionOLS       T     3      Large       None
10: ConsumptionOLS       T     3      Large       None
11: ConsumptionOLS       T     4      Large       None
12: ConsumptionOLS       T     4      Large       None
13: ConsumptionOLS       T     4      Large       None
14: ConsumptionOLS       T     5      Large       None
15: ConsumptionOLS       T     5      Large       None
16: ConsumptionOLS       T     5      Large       None
17: ConsumptionOLS       T     6      Large       None
18: ConsumptionOLS       T     6      Large       None
19: ConsumptionOLS       T     6      Large       None
                                 ImpactType             hv period estimate
                                     <fctr>         <fctr>  <num>    <num>
 1: sum of (nontrad - trad, in each period) periNrelativeT      2  61.1449
 2: sum of (nontrad - trad, in each period) periNrelativeT      3  85.2276
 3: sum of (nontrad - trad, in each period) periNrelativeT      2  94.2623
 4: sum of (nontrad - trad, in each period) periNrelativeT      3 120.0333
 5: sum of (nontrad - trad, in each period) periNrelativeT      1  30.0386
 6: sum of (nontrad - trad, in each period) periNrelativeT      2 152.9208
 7: sum of (nontrad - trad, in each period) periNrelativeT      3 134.1695
 8: sum of (nontrad - trad, in each period) periNrelativeT      1  44.5824
 9: sum of (nontrad - trad, in each period) periNrelativeT      2 158.4445
10: sum of (nontrad - trad, in each period) periNrelativeT      3 133.0717
11: sum of (nontrad - trad, in each period) periNrelativeT      1 293.1541
12: sum of (nontrad - trad, in each period) periNrelativeT      2 682.2070
13: sum of (nontrad - trad, in each period) periNrelativeT      3 294.7481
14: sum of (nontrad - trad, in each period) periNrelativeT      1 -28.5149
15: sum of (nontrad - trad, in each period) periNrelativeT      2 402.8235
16: sum of (nontrad - trad, in each period) periNrelativeT      3 106.0482
17: sum of (nontrad - trad, in each period) periNrelativeT      1  38.3050
18: sum of (nontrad - trad, in each period) periNrelativeT      2 438.6247
19: sum of (nontrad - trad, in each period) periNrelativeT      3 139.7646
           lb       ub   pvalue
        <num>    <num>    <num>
 1:  -79.7129  202.003 0.394613
 2:  -90.6428  261.098 0.341953
 3:  -48.9110  237.436 0.196734
 4:  -37.3321  277.399 0.134801
 5:  -93.3558  153.433 0.633197
 6: -116.6002  422.442 0.266042
 7:  -46.7863  315.125 0.146121
 8:  -41.8740  131.039 0.312084
 9: -118.4655  435.355 0.262012
10:  -74.0044  340.148 0.207780
11: -448.2723 1034.580 0.438275
12: -499.4741 1863.888 0.257759
13: -533.4245 1122.921 0.485365
14: -517.8424  460.813 0.909046
15: -594.5254 1400.172 0.428491
16: -495.8928  707.989 0.729808
17: -255.4579  332.068 0.798238
18: -615.9475 1493.197 0.414865
19: -640.7970  920.326 0.725569
\end{Soutput}
\begin{Soutput}
          FileName regtype   num attributes experience
            <fctr>  <fctr> <num>     <fctr>     <fctr>
 1: ConsumptionOLS      Ta     2     InKind       None
 2:    Consumption      Ta     2     InKind       None
 3: ConsumptionOLS      Ta     2     InKind       None
 4:    Consumption      Ta     2     InKind       None
 5: ConsumptionOLS      Ta     2     InKind       None
 6:    Consumption      Ta     2     InKind       None
 7: ConsumptionOLS      Ta     2     InKind       None
 8: ConsumptionOLS      Ta     2     InKind       None
 9:    Consumption      Ta     2     InKind       None
10: ConsumptionOLS      Ta     2     InKind       None
11:    Consumption      Ta     2     InKind       None
12: ConsumptionOLS      Ta     2     InKind       None
13:    Consumption      Ta     2     InKind       None
14: ConsumptionOLS      Ta     2     InKind       None
15: ConsumptionOLS      Ta     2     InKind       None
16:    Consumption      Ta     2     InKind       None
17: ConsumptionOLS      Ta     2     InKind       None
18:    Consumption      Ta     2     InKind       None
19: ConsumptionOLS      Ta     2     InKind       None
20: ConsumptionOLS      Ta     2     InKind       None
21:    Consumption      Ta     2     InKind       None
22: ConsumptionOLS      Ta     2     InKind       None
23:    Consumption      Ta     2     InKind       None
24: ConsumptionOLS      Ta     2     InKind       None
25: ConsumptionOLS      Ta     2     InKind       None
26:    Consumption      Ta     2     InKind       None
27: ConsumptionOLS      Ta     2     InKind       None
28:    Consumption      Ta     2     InKind       None
29: ConsumptionOLS      Ta     2     InKind       None
30: ConsumptionOLS      Ta     2  LargeSize       None
31:    Consumption      Ta     2  LargeSize       None
32: ConsumptionOLS      Ta     2  LargeSize       None
33:    Consumption      Ta     2  LargeSize       None
34: ConsumptionOLS      Ta     2  LargeSize       None
35:    Consumption      Ta     2  LargeSize       None
36: ConsumptionOLS      Ta     2  LargeSize       None
37: ConsumptionOLS      Ta     2  LargeSize       None
38:    Consumption      Ta     2  LargeSize       None
39: ConsumptionOLS      Ta     2  LargeSize       None
40:    Consumption      Ta     2  LargeSize       None
41: ConsumptionOLS      Ta     2  LargeSize       None
42:    Consumption      Ta     2  LargeSize       None
43: ConsumptionOLS      Ta     2  LargeSize       None
44: ConsumptionOLS      Ta     2  LargeSize       None
45:    Consumption      Ta     2  LargeSize       None
46: ConsumptionOLS      Ta     2  LargeSize       None
47:    Consumption      Ta     2  LargeSize       None
48: ConsumptionOLS      Ta     2  LargeSize       None
49: ConsumptionOLS      Ta     2  LargeSize       None
50:    Consumption      Ta     2  LargeSize       None
51: ConsumptionOLS      Ta     2  LargeSize       None
52:    Consumption      Ta     2  LargeSize       None
53: ConsumptionOLS      Ta     2  LargeSize       None
54: ConsumptionOLS      Ta     2  LargeSize       None
55:    Consumption      Ta     2  LargeSize       None
56: ConsumptionOLS      Ta     2  LargeSize       None
57:    Consumption      Ta     2  LargeSize       None
58: ConsumptionOLS      Ta     2  LargeSize       None
59: ConsumptionOLS      Ta     2       trad       None
60:    Consumption      Ta     2       trad       None
61: ConsumptionOLS      Ta     2       trad       None
62:    Consumption      Ta     2       trad       None
63: ConsumptionOLS      Ta     2       trad       None
64:    Consumption      Ta     2       trad       None
65: ConsumptionOLS      Ta     2       trad       None
66: ConsumptionOLS      Ta     2       trad       None
67:    Consumption      Ta     2       trad       None
68: ConsumptionOLS      Ta     2       trad       None
69:    Consumption      Ta     2       trad       None
70: ConsumptionOLS      Ta     2       trad       None
71: ConsumptionOLS      Ta     2  WithGrace       None
72:    Consumption      Ta     2  WithGrace       None
73: ConsumptionOLS      Ta     2  WithGrace       None
74:    Consumption      Ta     2  WithGrace       None
75: ConsumptionOLS      Ta     2  WithGrace       None
76:    Consumption      Ta     2  WithGrace       None
77: ConsumptionOLS      Ta     2  WithGrace       None
78: ConsumptionOLS      Ta     2  WithGrace       None
79:    Consumption      Ta     2  WithGrace       None
80: ConsumptionOLS      Ta     2  WithGrace       None
81:    Consumption      Ta     2  WithGrace       None
82: ConsumptionOLS      Ta     2  WithGrace       None
83:    Consumption      Ta     2  WithGrace       None
84: ConsumptionOLS      Ta     2  WithGrace       None
85: ConsumptionOLS      Ta     2  WithGrace       None
86:    Consumption      Ta     2  WithGrace       None
87: ConsumptionOLS      Ta     2  WithGrace       None
88:    Consumption      Ta     2  WithGrace       None
89: ConsumptionOLS      Ta     2  WithGrace       None
90: ConsumptionOLS      Ta     2  WithGrace       None
91:    Consumption      Ta     2  WithGrace       None
92: ConsumptionOLS      Ta     2  WithGrace       None
93:    Consumption      Ta     2  WithGrace       None
94: ConsumptionOLS      Ta     2  WithGrace       None
95: ConsumptionOLS      Ta     2  WithGrace       None
96:    Consumption      Ta     2  WithGrace       None
97: ConsumptionOLS      Ta     2  WithGrace       None
98:    Consumption      Ta     2  WithGrace       None
99: ConsumptionOLS      Ta     2  WithGrace       None
          FileName regtype   num attributes experience
                                      ImpactType             hv period
                                          <fctr>         <fctr>  <num>
 1:                   level of reference nontrad             N0      2
 2:                   level of reference nontrad             N0      3
 3:              level of nontrad in each period             N2      2
 4:              level of nontrad in each period             N2      3
 5:              level of nontrad in each period             N2      3
 6:              level of nontrad in each period             N2      4
 7:              level of nontrad in each period             N2      4
 8:           reference nontrad - reference trad             N1      2
 9:           reference nontrad - reference trad             N1      3
10:    nontrad in each period - trad in period 2          NinT2      2
11:    nontrad in each period - trad in period 2          NinT2      3
12:    nontrad in each period - trad in period 2          NinT2      3
13:    nontrad in each period - trad in period 2          NinT2      4
14:    nontrad in each period - trad in period 2          NinT2      4
15: nontrad in each period - nontrad in period 2           NinT      2
16: nontrad in each period - nontrad in period 2           NinT      3
17: nontrad in each period - nontrad in period 2           NinT      3
18: nontrad in each period - nontrad in period 2           NinT      4
19: nontrad in each period - nontrad in period 2           NinT      4
20:               nontrad - trad, in each period          dNinT      2
21:               nontrad - trad, in each period          dNinT      3
22:               nontrad - trad, in each period          dNinT      3
23:               nontrad - trad, in each period          dNinT      4
24:               nontrad - trad, in each period          dNinT      4
25:      sum of (nontrad - trad, in each period) periNrelativeT      2
26:      sum of (nontrad - trad, in each period) periNrelativeT      3
27:      sum of (nontrad - trad, in each period) periNrelativeT      3
28:      sum of (nontrad - trad, in each period) periNrelativeT      4
29:      sum of (nontrad - trad, in each period) periNrelativeT      4
30:                   level of reference nontrad             N0      2
31:                   level of reference nontrad             N0      3
32:              level of nontrad in each period             N2      2
33:              level of nontrad in each period             N2      3
34:              level of nontrad in each period             N2      3
35:              level of nontrad in each period             N2      4
36:              level of nontrad in each period             N2      4
37:           reference nontrad - reference trad             N1      2
38:           reference nontrad - reference trad             N1      3
39:    nontrad in each period - trad in period 2          NinT2      2
40:    nontrad in each period - trad in period 2          NinT2      3
41:    nontrad in each period - trad in period 2          NinT2      3
42:    nontrad in each period - trad in period 2          NinT2      4
43:    nontrad in each period - trad in period 2          NinT2      4
44: nontrad in each period - nontrad in period 2           NinT      2
45: nontrad in each period - nontrad in period 2           NinT      3
46: nontrad in each period - nontrad in period 2           NinT      3
47: nontrad in each period - nontrad in period 2           NinT      4
48: nontrad in each period - nontrad in period 2           NinT      4
49:               nontrad - trad, in each period          dNinT      2
50:               nontrad - trad, in each period          dNinT      3
51:               nontrad - trad, in each period          dNinT      3
52:               nontrad - trad, in each period          dNinT      4
53:               nontrad - trad, in each period          dNinT      4
54:      sum of (nontrad - trad, in each period) periNrelativeT      2
55:      sum of (nontrad - trad, in each period) periNrelativeT      3
56:      sum of (nontrad - trad, in each period) periNrelativeT      3
57:      sum of (nontrad - trad, in each period) periNrelativeT      4
58:      sum of (nontrad - trad, in each period) periNrelativeT      4
59:                      level of reference trad             T0      2
60:                      level of reference trad             T0      3
61:                 level of trad in each period             TL      2
62:                 level of trad in each period             TL      3
63:                 level of trad in each period             TL      3
64:                 level of trad in each period             TL      4
65:                 level of trad in each period             TL      4
66:       trad in each period - trad in period 2           TinT      2
67:       trad in each period - trad in period 2           TinT      3
68:       trad in each period - trad in period 2           TinT      3
69:       trad in each period - trad in period 2           TinT      4
70:       trad in each period - trad in period 2           TinT      4
71:                   level of reference nontrad             N0      2
72:                   level of reference nontrad             N0      3
73:              level of nontrad in each period             N2      2
74:              level of nontrad in each period             N2      3
75:              level of nontrad in each period             N2      3
76:              level of nontrad in each period             N2      4
77:              level of nontrad in each period             N2      4
78:           reference nontrad - reference trad             N1      2
79:           reference nontrad - reference trad             N1      3
80:    nontrad in each period - trad in period 2          NinT2      2
81:    nontrad in each period - trad in period 2          NinT2      3
82:    nontrad in each period - trad in period 2          NinT2      3
83:    nontrad in each period - trad in period 2          NinT2      4
84:    nontrad in each period - trad in period 2          NinT2      4
85: nontrad in each period - nontrad in period 2           NinT      2
86: nontrad in each period - nontrad in period 2           NinT      3
87: nontrad in each period - nontrad in period 2           NinT      3
88: nontrad in each period - nontrad in period 2           NinT      4
89: nontrad in each period - nontrad in period 2           NinT      4
90:               nontrad - trad, in each period          dNinT      2
91:               nontrad - trad, in each period          dNinT      3
92:               nontrad - trad, in each period          dNinT      3
93:               nontrad - trad, in each period          dNinT      4
94:               nontrad - trad, in each period          dNinT      4
95:      sum of (nontrad - trad, in each period) periNrelativeT      2
96:      sum of (nontrad - trad, in each period) periNrelativeT      3
97:      sum of (nontrad - trad, in each period) periNrelativeT      3
98:      sum of (nontrad - trad, in each period) periNrelativeT      4
99:      sum of (nontrad - trad, in each period) periNrelativeT      4
                                      ImpactType             hv period
      estimate         lb         ub      pvalue
         <num>      <num>      <num>       <num>
 1: 3152.55051 2931.12329 3373.97773 0.00000e+00
 2: 2125.83261 1844.52633 2407.13889 0.00000e+00
 3: 6331.89035 5964.05166 6699.72903 0.00000e+00
 4: 2032.45751 1802.78924 2262.12577 0.00000e+00
 5: 3956.03085 3713.84557 4198.21613 0.00000e+00
 6: 1942.82634 1636.56277 2249.08990 0.00000e+00
 7: 3849.25119 3653.49379 4045.00860 0.00000e+00
 8:  -26.78933 -152.18021   98.60155 6.75337e-01
 9:   93.37511  -47.91065  234.66086 1.95031e-01
10: 3125.76118 2804.79123 3446.73113 0.00000e+00
11:   93.37511  -47.91065  234.66086 1.95031e-01
12:  749.90168  507.84002  991.96334 1.36517e-09
13:    3.74394 -182.97028  190.45815 9.68629e-01
14:  643.12203  496.63702  789.60703 0.00000e+00
15: 3152.55051 2931.12329 3373.97773 0.00000e+00
16:    0.00000    0.00000    0.00000          NA
17:  776.69101  573.38117  980.00085 8.43769e-14
18:  -89.63117 -340.81682  161.55449 4.84049e-01
19:  669.91135  506.36908  833.45363 1.33227e-15
20:  -26.78933 -152.18021   98.60155 6.75337e-01
21:    0.00000    0.00000    0.00000          NA
22:  224.38574   33.39158  415.37990 2.13117e-02
23: -181.55385 -438.70913   75.60144 1.66286e-01
24:   10.87422 -152.10162  173.85006 8.95929e-01
25:  -26.78933 -152.18021   98.60155 6.75337e-01
26:   93.37511  -47.91065  234.66086 1.95031e-01
27:  197.59641  -37.90248  433.09529 1.00046e-01
28:  -88.17874 -275.04293   98.68546 3.54767e-01
29:  -15.91511 -149.05264  117.22243 8.14716e-01
30: 3209.37844 3034.88269 3383.87418 0.00000e+00
31: 2093.60243 1855.30397 2331.90089 0.00000e+00
32: 6388.71828 6074.64903 6702.78752 0.00000e+00
33: 2032.45751 1802.78924 2262.12577 0.00000e+00
34: 3854.52730 3577.88105 4131.17354 0.00000e+00
35: 2148.46284 1827.59827 2469.32742 0.00000e+00
36: 3942.50787 3683.50920 4201.50654 0.00000e+00
37:   30.03860  -93.35580  153.43300 6.33197e-01
38:   61.14492  -79.71293  202.00277 3.94613e-01
39: 3239.41704 2984.72936 3494.10472 0.00000e+00
40:   61.14492  -79.71293  202.00277 3.94613e-01
41:  705.22606  412.76832  997.68380 2.34917e-06
42:  177.15026  -20.07717  374.37768 7.82921e-02
43:  793.20663  594.01172  992.40154 7.43849e-15
44: 3209.37844 3034.88269 3383.87418 0.00000e+00
45:    0.00000    0.00000    0.00000          NA
46:  675.18746  435.69601  914.67891 3.45784e-08
47:  116.00534 -115.21702  347.22769 3.25196e-01
48:  763.16803  564.94641  961.38965 5.41789e-14
49:   30.03860  -93.35580  153.43300 6.33197e-01
50:    0.00000    0.00000    0.00000          NA
51:  122.88219  -89.34666  335.11103 2.56369e-01
52:   24.08266 -186.09489  234.26021 8.22188e-01
53:  104.13090  -70.49236  278.75415 2.42428e-01
54:   30.03860  -93.35580  153.43300 6.33197e-01
55:   61.14492  -79.71293  202.00277 3.94613e-01
56:  152.92079 -116.60017  422.44175 2.66042e-01
57:   85.22758  -90.64283  261.09798 3.41953e-01
58:  134.16950  -46.78632  315.12532 1.46121e-01
59: 3179.33984 3016.60057 3342.07911 0.00000e+00
60: 2032.45751 1802.78924 2262.12577 0.00000e+00
61: 6358.67968 6033.20114 6684.15821 0.00000e+00
62: 2032.45751 1802.78924 2262.12577 0.00000e+00
63: 3731.64511 3561.36684 3901.92338 0.00000e+00
64: 2124.38018 1876.34097 2372.41940 0.00000e+00
65: 3838.37697 3682.25682 3994.49712 0.00000e+00
66: 3179.33984 3016.60057 3342.07911 0.00000e+00
67:    0.00000    0.00000    0.00000          NA
68:  552.30527  479.66994  624.94060 0.00000e+00
69:   91.92268    7.66953  176.17583 3.25090e-02
70:  659.03713  598.24425  719.83002 0.00000e+00
71: 3170.47099 2976.42379 3364.51819 0.00000e+00
72: 1970.98336 1693.06823 2248.89850 0.00000e+00
73: 6349.81083 6018.12013 6681.50153 0.00000e+00
74: 2032.45751 1802.78924 2262.12577 0.00000e+00
75: 3527.02988 3259.65088 3794.40888 0.00000e+00
76: 2160.25425 1769.65304 2550.85547 0.00000e+00
77: 3700.63560 3449.41085 3951.86035 0.00000e+00
78:   -8.86885 -143.99120  126.25351 8.97616e-01
79:  -61.47414 -215.44427   92.49599 4.33630e-01
80: 3161.60214 2869.47090 3453.73339 0.00000e+00
81:  -61.47414 -215.44427   92.49599 4.33630e-01
82:  338.82120   67.00354  610.63885 1.45748e-02
83:   66.32260 -166.08767  298.73288 5.75704e-01
84:  512.42691  356.44170  668.41213 1.32984e-10
85: 3170.47099 2976.42379 3364.51819 0.00000e+00
86:    0.00000    0.00000    0.00000          NA
87:  347.69004  135.34136  560.03872 1.33722e-03
88:  127.79675 -158.88000  414.47349 3.82001e-01
89:  521.29576  329.21932  713.37220 1.08847e-07
90:   -8.86885 -143.99120  126.25351 8.97616e-01
91:    0.00000    0.00000    0.00000          NA
92: -204.61523 -416.21589    6.98543 5.80539e-02
93:   35.87407 -217.70597  289.45411 7.81422e-01
94: -137.74137 -324.08301   48.60026 1.47355e-01
95:   -8.86885 -143.99120  126.25351 8.97616e-01
96:  -61.47414 -215.44427   92.49599 4.33630e-01
97: -213.48408 -485.40253   58.43438 1.23825e-01
98:  -25.60007 -224.77407  173.57392 8.00972e-01
99: -146.61022 -302.77540    9.55496 6.57542e-02
      estimate         lb         ub      pvalue
\end{Soutput}
\end{Schunk}
\begin{Schunk}
\begin{Soutput}
     FileName regtype   num  attributes gender  school              ImpactType
       <char>  <char> <num>      <char> <char>  <char>                  <char>
 1: Schooling       T     2 traditional    all primary level of reference trad
 2: Schooling       T     2 traditional    all  junior level of reference trad
 3: Schooling       T     2 traditional    all    high level of reference trad
 4: Schooling       T     3 traditional    all primary level of reference trad
 5: Schooling       T     3 traditional    all  junior level of reference trad
 6: Schooling       T     3 traditional    all    high level of reference trad
 7: Schooling       T     4 traditional    all primary level of reference trad
 8: Schooling       T     4 traditional    all  junior level of reference trad
 9: Schooling       T     4 traditional    all    high level of reference trad
10: Schooling      Ta     2 traditional    all primary level of reference trad
11: Schooling      Ta     2 traditional    all  junior level of reference trad
12: Schooling      Ta     2 traditional    all    high level of reference trad
13: Schooling      Ta     3 traditional    all primary level of reference trad
14: Schooling      Ta     3 traditional    all  junior level of reference trad
15: Schooling      Ta     3 traditional    all    high level of reference trad
16: Schooling      Ta     4 traditional    all primary level of reference trad
17: Schooling      Ta     4 traditional    all  junior level of reference trad
18: Schooling      Ta     4 traditional    all    high level of reference trad
    period     hv estimate       lb       ub      pvalue
     <num> <char>    <num>    <num>    <num>       <num>
 1:     NA  MofTA 0.654996 0.570694 0.739298 0.00000e+00
 2:     NA  MofTA 0.654996 0.570694 0.739298 0.00000e+00
 3:     NA  MofTA 0.654996 0.570694 0.739298 0.00000e+00
 4:     NA  MofTA 0.703637 0.628657 0.778617 0.00000e+00
 5:     NA  MofTA 0.558788 0.463473 0.654103 0.00000e+00
 6:     NA  MofTA 0.465110 0.375061 0.555160 0.00000e+00
 7:     NA  MofTA 0.816777 0.679928 0.953626 0.00000e+00
 8:     NA  MofTA 0.698685 0.547135 0.850235 0.00000e+00
 9:     NA  MofTA 0.606556 0.447721 0.765391 1.08358e-13
10:     NA  MofTA 0.654996 0.570694 0.739298 0.00000e+00
11:     NA  MofTA 0.654996 0.570694 0.739298 0.00000e+00
12:     NA  MofTA 0.654996 0.570694 0.739298 0.00000e+00
13:     NA  MofTA 0.703637 0.628657 0.778617 0.00000e+00
14:     NA  MofTA 0.558788 0.463473 0.654103 0.00000e+00
15:     NA  MofTA 0.465110 0.375061 0.555160 0.00000e+00
16:     NA  MofTA 0.816777 0.679928 0.953626 0.00000e+00
17:     NA  MofTA 0.698685 0.547135 0.850235 0.00000e+00
18:     NA  MofTA 0.606556 0.447721 0.765391 1.08358e-13
\end{Soutput}
\begin{Soutput}
    FileName regtype   num attributes
      <fctr>  <fctr> <num>     <fctr>
1: Schooling       T     5      Large
2: Schooling       T     5      Large
3: Schooling       T     5      Large
4: Schooling       T     5      Large
5: Schooling       T     5      Large
6: Schooling       T     5      Large
                                                ImpactType period        lb
                                                    <fctr>  <num>     <num>
1: female nontrad - female trad, in each period, at school      2 -0.407138
2: female nontrad - female trad, in each period, at school      3 -0.260304
3: female nontrad - female trad, in each period, at school      4 -0.392079
4:               nontrad - trad, in each period, at school      2 -0.140731
5:               nontrad - trad, in each period, at school      3 -0.194737
6:               nontrad - trad, in each period, at school      4 -0.243397
       estimate        ub gender school      hv   pvalue
          <num>     <num> <fctr> <fctr>  <fctr>    <num>
1: -0.172937656 0.0612629 female junior FofNinT 0.147728
2:  0.035773671 0.3318515 female junior FofNinT 0.812710
3: -0.095465521 0.2011480 female junior FofNinT 0.527973
4: -0.000432567 0.1398662   male junior MofNinT 0.995176
5:  0.001223819 0.1971844   male junior MofNinT 0.990229
6:  0.009168600 0.2617341   male junior MofNinT 0.943250
                           AtType
                           <fctr>
1: Arms (relative to Traditional)
2: Arms (relative to Traditional)
3: Arms (relative to Traditional)
4: Arms (relative to Traditional)
5: Arms (relative to Traditional)
6: Arms (relative to Traditional)
\end{Soutput}
\end{Schunk}


\begin{Schunk}
\begin{Soutput}
Error in factor(variables, labels = c("Broad net assets", "Net assets", : 無効な 'labels' です; 長さ 4 は 1 または 3 であるべきです
\end{Soutput}
\end{Schunk}
\begin{Schunk}
\begin{Soutput}
Error in factor(variables, labels = c("Broad net assets", "Net assets", : 無効な 'labels' です; 長さ 4 は 1 または 3 であるべきです
\end{Soutput}
\end{Schunk}


\begin{Schunk}
\begin{Soutput}
               regressand
attributes      land net non-livestock assets net assets cattle
  Large/Upfront   15                       15         15     12
  LargeGrace      15                       15         15     12
  Cattle          15                       15         15     12
  WithGrace       15                       15         15     12
  InKind          15                       15         15     12
\end{Soutput}
\end{Schunk}
\begin{Schunk}
\begin{Soutput}
Error in factor(regressand, labels = c("Net assets (BDT)", "Net non-livestock assets (BDT)", : 無効な 'labels' です; 長さ 3 は 1 または 2 であるべきです
\end{Soutput}
\begin{Soutput}
               regressand
attributes      net assets net non livestock assets cattle
  Large/Upfront         15                        0     12
  LargeGrace            15                        0     12
  Cattle                15                        0     12
  WithGrace             15                        0     12
  InKind                15                        0     12
\end{Soutput}
\end{Schunk}



\begin{Schunk}
\begin{Soutput}
             num
SubGroup        1   2   3   4 Sum
  All members   9   9   9   9  36
  Owner         9   9   9   0  27
  Adi           9   9   9   0  27
  None          9   9   9   0  27
  Sum          36  36  36   9 117
\end{Soutput}
\end{Schunk}
\begin{Schunk}
\begin{Soutput}
            regressand
attributes   Net assets,\nannual price (BDT) Net non-livestock\n assets (BDT)
  Large                                   15                               15
  LargeGrace                              15                               15
  Cattle                                  15                               15
            regressand
attributes   Net broad assets\n (BDT) Cattle (counts)
  Large                            15               9
  LargeGrace                       15               9
  Cattle                           15               9
\end{Soutput}
\end{Schunk}
\begin{Schunk}
\begin{Soutput}
            regressand
attributes   Net assets (BDT) Net non-livestock assets\n(BDT) Cattle (counts)
  Large                    15                              15              12
  LargeGrace               15                              15              12
  Cattle                   15                              15              12
\end{Soutput}
\end{Schunk}


\begin{Schunk}
\begin{Soutput}
                                                         attributes
ImpactType                                                Traditional Large
  level of trad in each period at school                           18     0
  level of female trad in each period at school                    18     0
  nontrad - trad, in each period, at school                         0    18
  female nontrad - female trad, in each period, at school           0    18
                                                         attributes
ImpactType                                                LargeGrace Cattle
  level of trad in each period at school                           0      0
  level of female trad in each period at school                    0      0
  nontrad - trad, in each period, at school                       18     18
  female nontrad - female trad, in each period, at school         18     18
\end{Soutput}
\end{Schunk}
\begin{Schunk}
\begin{Soutput}
                                                         attributes
ImpactType                                                Traditional Upfront
  level of trad in each period at school                           18       0
  level of female trad in each period at school                    18       0
  nontrad - trad, in each period, at school                         0      18
  female nontrad - female trad, in each period, at school           0      18
                                                         attributes
ImpactType                                                WithGrace InKind
  level of trad in each period at school                          0      0
  level of female trad in each period at school                   0      0
  nontrad - trad, in each period, at school                      18     18
  female nontrad - female trad, in each period, at school        18     18
\end{Soutput}
\end{Schunk}
\begin{Schunk}
\begin{Soutput}
                                                                       attributes
ImpactType                                                              Traditional
  level of trad in each period at school                                         18
  level of female trad in each period at school                                  18
  nontrad change - trad change, in each period, at school                         0
  female nontrad change - female trad change, in each period, at school           0
                                                                       attributes
ImpactType                                                              Large
  level of trad in each period at school                                    0
  level of female trad in each period at school                             0
  nontrad change - trad change, in each period, at school                  18
  female nontrad change - female trad change, in each period, at school    18
                                                                       attributes
ImpactType                                                              LargeGrace
  level of trad in each period at school                                         0
  level of female trad in each period at school                                  0
  nontrad change - trad change, in each period, at school                       18
  female nontrad change - female trad change, in each period, at school         18
                                                                       attributes
ImpactType                                                              Cattle
  level of trad in each period at school                                     0
  level of female trad in each period at school                              0
  nontrad change - trad change, in each period, at school                   18
  female nontrad change - female trad change, in each period, at school     18
\end{Soutput}
\end{Schunk}
\begin{Schunk}
\begin{Soutput}
                                                                       attributes
ImpactType                                                              Traditional
  level of trad in each period at school                                         18
  level of female trad in each period at school                                  18
  nontrad change - trad change, in each period, at school                         0
  female nontrad change - female trad change, in each period, at school           0
                                                                       attributes
ImpactType                                                              Upfront
  level of trad in each period at school                                      0
  level of female trad in each period at school                               0
  nontrad change - trad change, in each period, at school                    18
  female nontrad change - female trad change, in each period, at school      18
                                                                       attributes
ImpactType                                                              WithGrace
  level of trad in each period at school                                        0
  level of female trad in each period at school                                 0
  nontrad change - trad change, in each period, at school                      18
  female nontrad change - female trad change, in each period, at school        18
                                                                       attributes
ImpactType                                                              InKind
  level of trad in each period at school                                     0
  level of female trad in each period at school                              0
  nontrad change - trad change, in each period, at school                   18
  female nontrad change - female trad change, in each period, at school     18
\end{Soutput}
\end{Schunk}
\begin{Schunk}
\begin{Soutput}
         OwnCattle
AdiCattle   0   1 <NA> Sum
     0    519 141    0 660
     1    112   0    0 112
     <NA>   0   0    1   1
     Sum  631 141    1 773
\end{Soutput}
\end{Schunk}

\newpage
\renewcommand{\arraystretch}{.6}
\mpage{\linewidth}{
\hfil\textsc{\footnotesize Figure \refstepcounter{figure}\thefigure: Assets by period\label{fig AssetCumRelativeToConcurrentTradEffects}}\\
\hfil\includegraphics[width = 14cm]{c:/data/GUK/analysis/program/figure/EstimationMemo/AssetsByPeriod.pdf}\\
\renewcommand{\arraystretch}{1}
\hfil\begin{tabular}{>{\hfill\scriptsize}p{1cm}<{}>{\scriptsize}p{12cm}<{\hfill}}
Source: & Tabulated with survey data.\\
Note:&  Red squares are means of respective data. Asset values are expressed in BDT. \textsf{Net assets}=total assets - debts. Debts include outstanding loaned amount of the experiment. Total assets use items observed in all 4 rounds of household surveys. \textsf{Net non livestock assets}=\textsf{net assets}-livestock asset values. \textsf{Number of cattle} is a headcount of cattle holding.  All net assets are in logarithms, number of cattle is in natural numbers. \\[1ex]
\end{tabular}
}

\newpage
\renewcommand{\arraystretch}{.6}
\mpage{\linewidth}{
\hfil\textsc{\footnotesize Figure \refstepcounter{figure}\thefigure: Assets by period among out of sample members\label{fig AssetCumRelativeToConcurrentTradEffectsNon800}}\\
\hfil\includegraphics[width = 14cm]{c:/data/GUK/analysis/program/figure/EstimationMemo/AssetsByPeriodNon800.pdf}\\
\renewcommand{\arraystretch}{1}
\hfil\begin{tabular}{>{\hfill\scriptsize}p{1cm}<{}>{\scriptsize}p{12cm}<{\hfill}}
Source: & Tabulated with survey data. Out of sample members are households who were not a part of 800 members and treated with the same intervention arms as in our experiment.\\
Note:&  Red squares are means of respective data. Asset values are expressed in BDT. \textsf{Net assets}=total assets - debts. Debts include outstanding loaned amount of the experiment. Total assets use items observed in all 4 rounds of household surveys. \textsf{Net non livestock assets}=\textsf{net assets}-livestock asset values. \textsf{Number of cattle} is a headcount of cattle holding.  All net assets are in logarithms, number of cattle is in natural numbers. \\[1ex]
\end{tabular}
}

\newpage
\renewcommand{\arraystretch}{.6}
\mpage{\linewidth}{
\hfil\textsc{\footnotesize Figure \refstepcounter{figure}\thefigure: Net assets by period\label{fig NetAssets}}\\
\hfil\includegraphics[width = 12cm]{c:/data/GUK/analysis/program/figure/EstimationMemo/NetAssetsByPeriod.pdf}\\
\renewcommand{\arraystretch}{1}
\hfil\begin{tabular}{>{\hfill\scriptsize}p{1cm}<{}>{\scriptsize}p{12cm}<{\hfill}}
Source: & Tabulated with survey data.\\
Note:&  Red squares are means of respective data. Net assets are in logarithms. \\[1ex]
\end{tabular}
}

\renewcommand{\arraystretch}{.6}
\mpage{\linewidth}{
\hfil\textsc{\footnotesize Figure \refstepcounter{figure}\thefigure: Impacts on net assets relative to concurrent traditional arm\label{fig NetAssetsCumRelativeToConcurrentTradEffects}}\\
\hfil\includegraphics[width = 12cm]{c:/data/GUK/analysis/program/figure/EstimationMemo/NetAssetsNLAssetsCattleEffects.pdf}\\
\renewcommand{\arraystretch}{1}
\hfil\begin{tabular}{>{\hfill\scriptsize}p{1cm}<{}>{\scriptsize}p{12cm}<{\hfill}}
Source: & Estimated with survey data.\\
Note:&  Cumulative impacts on net assets. \textsf{Large/Upfront}, \textsf{Large grace}, \textsf{Cattle} are impacts relative to \textsf{Traditional} arm. \textsf{WithGrace} and \textsf{InKind} are the impacts of respective marginal functional attributes. Panels show cumulative impacts of respective arm or attributes \textsf{k} relative to \textsf{tradiotional} arm which are obtained by $\mbox{2nd period}=b_{2k}, \mbox{3rd period}=b_{2k}+b_{3k}$, $\mbox{4th period}=b_{2k}+b_{4k}$ in the estimating equation $y_{it}=b_{1}y_{i1}+b_{2}+\bfb'_{2}\bfdee_{i}+b_{3}c_{3t}+\bfb'_{3}\bfdee_{i}c_{3t}+b_{4}c_{4t}+\bfb'_{4}\bfdee_{i}c_{4t}+e_{it}, \ t=2, 3, 4$, where $y_{it}$ is the outcome measure of member $i$ in period $t$, $\bfdee_{i}$ is a vector of arms or functional attributes, $c_{3t}, c_{4t}$ are indicator variables of period 3 and 4. Bars show 95\% confidence intervals using cluster robust standard errors. Asset values are expressed in Taka. \textsf{Net assets}=total assets - debts. Debts include outstanding loaned amount of the experiment. Total assets use items observed in all 4 rounds of household surveys. \\[1ex]
\end{tabular}
}

\begin{itemize}
\vspace{1.0ex}\setlength{\itemsep}{1.0ex}\setlength{\baselineskip}{12pt}
\item	All non-\textsf{Traditional} arms achieve larger net assets than \textsf{Traditional} arm by period 4.
\item	This is achieved through increases in both livestock and non-livestock assets relative to \textsf{Traditional} arm.
\item	\textsf{Large} arm shows an earlier increase in all non-\textsf{Traditional} arms. This indicates the borrowers of this arm may be better prepared than other non-\textsf{Traditional} arms as they had to build up cash holding before the loan is disbursed. The impacts in period 4 are similar in all non-\textsf{Traditional} arms. It implies that better preparation may affect the time course of impacts, but not their size (in midium term). 
\item	\textsf{Cattle} arm confidence intervals are tightest among the all non-\textsf{Traditional} arms. While its sample size is largest ($n=199$), it is the same as the \textsf{Large} arm and rejection and flood caused attrition are greater (47 vs. 29). This hints that the limited room of discretionary decision making under this arm may have resulted in smaller variations in project returns. 
\end{itemize}

\renewcommand{\arraystretch}{.6}
\mpage{\linewidth}{
\hfil\textsc{\footnotesize Figure \refstepcounter{figure}\thefigure: Cumulative impacts on various assets relative to concurrent traditional arm\label{fig NetAssetLandCattleCumRelativeToConcurrentTradEffects}}\\
\hfil\includegraphics[width = 14cm]{c:/data/GUK/analysis/program/figure/EstimationMemo/NetAssetLandCattleEffects.pdf}\\
\renewcommand{\arraystretch}{1}
\hfil\begin{tabular}{>{\hfill\scriptsize}p{1cm}<{}>{\scriptsize}p{12cm}<{\hfill}}
Source: & Estimated with survey data.\\
Note:&  Cumulative impacts on various asset measures. \textsf{Large/Upfront}, \textsf{Large grace}, \textsf{Cattle} are impacts relative to \textsf{Traditional} arm. \textsf{WithGrace} and \textsf{InKind} are the impacts of respective marginal functional attributes. Panels show cumulative impacts of respective arm or attributes \textsf{k} relative to \textsf{tradiotional} arm which are obtained by $\mbox{2nd period}=b_{2k}, \mbox{3rd period}=b_{2k}+b_{3k}$, $\mbox{4th period}=b_{2k}+b_{4k}$ in the estimating equation $y_{it}=b_{1}y_{i1}+b_{2}+\bfb'_{2}\bfdee_{i}+b_{3}c_{3t}+\bfb'_{3}\bfdee_{i}c_{3t}+b_{4}c_{4t}+\bfb'_{4}\bfdee_{i}c_{4t}+e_{it}, \ t=2, 3, 4$, where $y_{it}$ is the outcome measure of member $i$ in period $t$, $\bfdee_{i}$ is a vector of arms or functional attributes, $c_{3t}, c_{4t}$ are indicator variables of period 3 and 4. Bars show 95\% confidence intervals using cluster robust standard errors. Asset values are expressed in BDT. \textsf{Net assets}=total assets - debts. Debts include outstanding loaned amount of the experiment. Total assets use items observed in all 4 rounds of household surveys. \textsf{Net non livestock assets}=\textsf{net assets}-livestock asset values. \textsf{Number of cattle} is a headcount of cattle holding. \\[1ex]
\end{tabular}
}

\renewcommand{\arraystretch}{.6}
\mpage{\linewidth}{
\hfil\textsc{\footnotesize Figure \refstepcounter{figure}\thefigure: Cumulative impacts on net assets relative to concurrent traditional arm\label{fig NetAssetRelativeToConcurrentTradEffects}}\\
\hfil\includegraphics[width = 14cm]{c:/data/GUK/analysis/program/figure/EstimationMemo/NetAssetsEffects.pdf}\\
\renewcommand{\arraystretch}{1}
\hfil\begin{tabular}{>{\hfill\scriptsize}p{1cm}<{}>{\scriptsize}p{12cm}<{\hfill}}
Source: & Estimated with survey data.\\
Note:&  Cumulative impacts on net assets of non-\textsf{Traditional} arms relative to \textsf{Traditional} arm. \textsf{Large/Upfront}, \textsf{Large grace}, \textsf{Cattle} are impacts relative to \textsf{Traditional} arm. \textsf{WithGrace} and \textsf{InKind} are the impacts of respective marginal functional attributes. Panels show cumulative impacts of respective arm or attributes \textsf{k} relative to \textsf{tradiotional} arm which are obtained by $\mbox{2nd period}=b_{2k}, \mbox{3rd period}=b_{2k}+b_{3k}$, $\mbox{4th period}=b_{2k}+b_{4k}$ in the estimating equation $y_{it}=b_{1}y_{i1}+b_{2}+\bfb'_{2}\bfdee_{i}+b_{3}c_{3t}+\bfb'_{3}\bfdee_{i}c_{3t}+b_{4}c_{4t}+\bfb'_{4}\bfdee_{i}c_{4t}+e_{it}, \ t=2, 3, 4$, where $y_{it}$ is the outcome measure of member $i$ in period $t$, $\bfdee_{i}$ is a vector of arms or functional attributes, $c_{3t}, c_{4t}$ are indicator variables of period 3 and 4. Bars show 95\% confidence intervals using cluster robust standard errors., Asset values are expressed in BDT. \textsf{Net assets}=total assets - debts. Debts include outstanding loaned amount of the experiment. Total assets use items observed in all 4 rounds of household surveys. \textsf{Net non livestock assets}=\textsf{net assets}-livestock asset values. \textsf{Number of cattle} is a headcount of cattle holding. \\[1ex]
\end{tabular}
}

Results of land holding is similar to net assets, as it is a part of net assets, but the gap widens as period progresses. This is seen in the point estimates of non-\textsf{traditional} arms that are positive, yet most of estimates are imprecise and have their 95\% confidence intervals crossing zero. Among all three assets, land holding may be most reliable indicator of wealth for fewer missingness. Net assets are defined as total assets less debt outstanding, yet we have smaller coverage of asset items in the first period which inflates the increasing trend.\footnote{This change in coverage is common to all arms, and given randomisation, this should not affect identification of imapcts by ANCOVA estimator as it is captured in the estimates of \textsf{traditional} arm, although it adds an extra noise. }


\renewcommand{\arraystretch}{.6}
\mpage{\linewidth}{
\hfil\textsc{\footnotesize Figure \refstepcounter{figure}\thefigure: Cumulative impacts on net assets relative to traditional arm by experience\label{fig AssetRelativeToCumulativeConcurrentTradEffectsByExperience}}\\
\hfil\includegraphics[width = 14cm]{c:/data/GUK/analysis/program/figure/EstimationMemo/NetAssetsByExperienceEffects.pdf}\\
\renewcommand{\arraystretch}{.8}
\hfil\begin{tabular}{>{\hfill\scriptsize}p{1cm}<{}>{\scriptsize}p{12cm}<{\hfill}}
Source: & Estimated with survey data.\\
Note:&  Asset values are expressed in BDT. \textsf{Net assets}=total assets - debts. Debts include outstanding loaned amount of the experiment. Total assets use items observed in all 4 rounds of household surveys. \textsf{Net non livestock assets}=\textsf{net assets}-livestock asset values. \textsf{Number of cattle} is a headcount of cattle holding.  \textsf{Adi} is a group who has an experience of lease-in cattle contract at the period 2, \textsf{Own} is a group who holds cattle at the period 2, and \textsf{None} are all other individuals. There are 141 members who owned cattle at the period 2, 112 members who ever practiced \textsf{Adi} at the period 2, and  523 members who have no experience in cattle rearing.
\end{tabular}
}

\renewcommand{\arraystretch}{.6}
\mpage{\linewidth}{
\hfil\textsc{\footnotesize Figure \refstepcounter{figure}\thefigure: Cumulative impacts on cattle holding relative to traditional arm by experience\label{fig NumCowsRelativeToCumulativeConcurrentTradEffectsByExperience}}\\
%\hfil\includegraphics[width = 14cm]{    paste0(pathprogram,   "figure/EstimationMemo/NumCowsByExperienceEffects.pdf")}\\
\renewcommand{\arraystretch}{.8}
\hfil\begin{tabular}{>{\hfill\scriptsize}p{1cm}<{}>{\scriptsize}p{12cm}<{\hfill}}
Source: & Estimated with survey data.\\
Note:&  Asset values are expressed in BDT. \textsf{Net assets}=total assets - debts. Debts include outstanding loaned amount of the experiment. Total assets use items observed in all 4 rounds of household surveys. \textsf{Net non livestock assets}=\textsf{net assets}-livestock asset values. \textsf{Number of cattle} is a headcount of cattle holding.  \textsf{Adi} is a group who has an experience of lease-in cattle contract at the period 2, \textsf{Own} is a group who holds cattle at the period 2, and \textsf{None} are all other individuals. There are 141 members who owned cattle at the period 2, 112 members who ever practiced \textsf{Adi} at the period 2, and  523 members who have no experience in cattle rearing. 
\end{tabular}
}


\begin{figure}
\mpage{12cm}{
\hfil\textsc{\footnotesize Figure \refstepcounter{figure}\thefigure: Cumulative effects on labour income and per capita consumption\label{fig IncomeConsumptionEffects}}\\

\vspace{2ex}
\hspace{-2em}\includegraphics[height = 8cm, width = 14cm]{c:/data/GUK/analysis/program/figure/EstimationMemo/IncomeConsumptionEffects.pdf}\\
\renewcommand{\arraystretch}{1}
\setlength{\tabcolsep}{1pt}
\hfil\begin{tabular}{>{\hfill\scriptsize}p{1cm}<{}>{\scriptsize}p{12.5cm}<{\hfill}}
Source: & Constructed from ANCOVA estimation results 
\textsc{Table \ref{tab ANCOVA consumption timevarying}}, \textsc{Table \ref{tab ANCOVA consumption timevarying attributes original HH}}, \textsc{Table \ref{tab ANCOVA labour incomes timevarying}}, \textsc{Table \ref{tab ANCOVA labour incomes timevarying attributes}}.\\
Note:& Style and placement of panels follow the \textsc{\footnotesize Figure \ref{fig NetAssetEffects}}. \textsf{Large/Upfront}, \textsf{Large grace}, \textsf{Cattle} are impacts relative to \textsf{Traditional} arm. \textsf{WithGrace} and \textsf{InKind} are the impacts of respective marginal functional attributes. Panels show cumulative impacts of respective arm or attributes \textsf{k} relative to \textsf{tradiotional} arm which are obtained by $\mbox{2nd period}=b_{2k}, \mbox{3rd period}=b_{2k}+b_{3k}$, $\mbox{4th period}=b_{2k}+b_{4k}$ in the estimating equation $y_{it}=b_{1}y_{i1}+b_{2}+\bfb'_{2}\bfdee_{i}+b_{3}c_{3t}+\bfb'_{3}\bfdee_{i}c_{3t}+b_{4}c_{4t}+\bfb'_{4}\bfdee_{i}c_{4t}+e_{it}, \ t=2, 3, 4$, where $y_{it}$ is the outcome measure of member $i$ in period $t$, $\bfdee_{i}$ is a vector of arms or functional attributes, $c_{3t}, c_{4t}$ are indicator variables of period 3 and 4. Bars show 95\% confidence intervals using cluster robust standard errors. \textsf{Per capita consumption} is a total of food, hygiene, social, and energy expenditure divided by the number of household members, expressed as the annuaslied values in BDT. In-kind consumption of home made products is imputed at median prices. \textsf{Labour income} is labour incomes of household in 1000 BDT units. \\[1ex]
\end{tabular}
}
\end{figure}

\begin{figure}
\mpage{13cm}{
\hfil\textsc{\footnotesize Figure \refstepcounter{figure}\thefigure: Effects on child schooling by arm\label{fig SchEffectsByArm}}\\

\vspace{2ex}
\hfil\includegraphics[height = 12cm]{c:/data/GUK/analysis/program/figure/EstimationMemo/SchoolingEffectsWithTradByArm.pdf}\\
\renewcommand{\arraystretch}{1}
\setlength{\tabcolsep}{1pt}
\hfil\begin{tabular}{>{\hfill\scriptsize}p{1cm}<{}>{\scriptsize}p{12cm}<{\hfill}}
Source: & Constructed from ANCOVA estimation results of \textsc{Table \ref{tab ANCOVA Sch}}.\\
Note:& The left most column shows schooling level of \textsf{Traditional arm}. The right three columns show marginal impacts of each arms relative to the \textsf{Traditional arm}. Each rows are grouped into primary, junior, and high school levels. 
\textsf{Large/Upfront}, \textsf{Large grace}, \textsf{Cattle} are impacts relative to \textsf{Traditional} arm. \textsf{WithGrace} and \textsf{InKind} are the impacts of respective marginal functional attributes. Panels show cumulative impacts of respective arm or attributes \textsf{k} relative to \textsf{tradiotional} arm which are obtained by $\mbox{2nd period}=b_{2k}, \mbox{3rd period}=b_{2k}+b_{3k}$, $\mbox{4th period}=b_{2k}+b_{4k}$ in the estimating equation $y_{it}=b_{1}y_{i1}+b_{2}+\bfb'_{2}\bfdee_{i}+b_{3}c_{3t}+\bfb'_{3}\bfdee_{i}c_{3t}+b_{4}c_{4t}+\bfb'_{4}\bfdee_{i}c_{4t}+e_{it}, \ t=2, 3, 4$, where $y_{it}$ is the outcome measure of member $i$ in period $t$, $\bfdee_{i}$ is a vector of arms or functional attributes, $c_{3t}, c_{4t}$ are indicator variables of period 3 and 4. Bars show 95\% confidence intervals using cluster robust standard errors.  \\[1ex]
\end{tabular}
}
\end{figure}

\begin{figure}
\mpage{13cm}{
\hfil\textsc{\footnotesize Figure \refstepcounter{figure}\thefigure: Effects on child schooling by functional attribute\label{fig SchEffectsByFunAttribute}}\\

\vspace{2ex}
\hfil\includegraphics[height = 12cm]{c:/data/GUK/analysis/program/figure/EstimationMemo/SchoolingEffectsWithTradByFunAttribute.pdf}\\
\renewcommand{\arraystretch}{1}
\setlength{\tabcolsep}{1pt}
\hfil\begin{tabular}{>{\hfill\scriptsize}p{1cm}<{}>{\scriptsize}p{12cm}<{\hfill}}
Source: & Constructed from ANCOVA estimation results of \textsc{Table \ref{tab ANCOVA Sch}}.\\
Note:& The left most column shows schooling level of \textsf{Traditional arm}. The right three columns show marginal impacts of each functional attributes. \textsf{Upfront} shows impacts relative to the \textsf{Traditional arm}, \textsf{With grace} shows impacts relative to \textsf{Traditional arm} and \textsf{Upfront}, and \textsf{In Kind} shows impacts relative to \textsf{Traditional arm}, \textsf{Upfront}, and \textsf{With grace}. Each rows are grouped into primary, junior, and high school levels. 
\textsf{Large/Upfront}, \textsf{Large grace}, \textsf{Cattle} are impacts relative to \textsf{Traditional} arm. \textsf{WithGrace} and \textsf{InKind} are the impacts of respective marginal functional attributes. Panels show cumulative impacts of respective arm or attributes \textsf{k} relative to \textsf{tradiotional} arm which are obtained by $\mbox{2nd period}=b_{2k}, \mbox{3rd period}=b_{2k}+b_{3k}$, $\mbox{4th period}=b_{2k}+b_{4k}$ in the estimating equation $y_{it}=b_{1}y_{i1}+b_{2}+\bfb'_{2}\bfdee_{i}+b_{3}c_{3t}+\bfb'_{3}\bfdee_{i}c_{3t}+b_{4}c_{4t}+\bfb'_{4}\bfdee_{i}c_{4t}+e_{it}, \ t=2, 3, 4$, where $y_{it}$ is the outcome measure of member $i$ in period $t$, $\bfdee_{i}$ is a vector of arms or functional attributes, $c_{3t}, c_{4t}$ are indicator variables of period 3 and 4. Bars show 95\% confidence intervals using cluster robust standard errors.  \\[1ex]
\end{tabular}
}
\end{figure}

\begin{figure}
\mpage{13cm}{
\hfil\textsc{\footnotesize Figure \refstepcounter{figure}\thefigure: Concurrent effects on child schooling by functional attribute\label{fig SchEffectsConcurrentByFunAttribute}}\\

\vspace{2ex}
\hfil\includegraphics[height = 12cm]{c:/data/GUK/analysis/program/figure/EstimationMemo/SchoolingEffectsConcurrentWithTradByFunAttribute.pdf}\\
\renewcommand{\arraystretch}{1}
\setlength{\tabcolsep}{1pt}
\hfil\begin{tabular}{>{\hfill\scriptsize}p{1cm}<{}>{\scriptsize}p{12cm}<{\hfill}}
Source: & Constructed from ANCOVA estimation results of \textsc{Table \ref{tab ANCOVA Sch}}.\\
Note:& The left most column shows schooling level of \textsf{Traditional arm}. The right three columns show marginal impacts of each functional attributes. \textsf{Upfront} shows impacts relative to the \textsf{Traditional arm}, \textsf{With grace} and \textsf{In Kind} show impacts relative to \textsf{Upfront}. Each rows are grouped into primary, junior, and high school levels. Impacts are per period effects $b_{2t}$ relative to concurrent respective comparison group, not the total effects $b_{0}+b_{2t}$. 
  \\[1ex]
\end{tabular}
}
\end{figure}


\begin{itemize}
\vspace{1.0ex}\setlength{\itemsep}{1.0ex}\setlength{\baselineskip}{12pt}
\item	\textsc{\footnotesize Figure \ref{fig SchEffectsByArm}} shows negative impacts of nontraditional arm among boys of primary school. Impacts on the boys are always negative among non-traditional arms for all school levels. But only primary school impacts are estimated precisely enough to be statistically distinct from zero.
\item	\textsc{\footnotesize Figure \ref{fig SchEffectsByFunAttribute}} shows the negative impacts on primary school aged boys are due to the upfront nature of the leding. 
\item	Given that the \textsf{Upfront} nature of the lending causes borrowers to purchase a heiffer, the negative impacts on the boy's schooling is most likely to be due to heiffer related labour. Impacts in period 4 are 
-0.038646, -0.034893, -0.052939 percentage points for \textsf{Large}, \textsf{LargeGrace}, and \textsf{Cattle} arms, respectively.

\end{itemize}


\subsection{Project cycle}


There are issues with the project cycle data. 
\begin{itemize}
\vspace{1.0ex}\setlength{\itemsep}{1.0ex}\setlength{\baselineskip}{12pt}
\item  There are 94 members who report multiple entries (rows). This is the intended way of reporting multiple projects. However, 12 members report IGAs (\textsf{iga1\_1st}, etc.) that do not match with respective \textsf{project\_type}. Among all members, \textsf{project\_type} is less in details (``cow'') and IGAs are more detailed (``cow, trade, goat''). In the majority cases, the contents in the former is a subset of the contents of the latter. In other cases, they simply differ: There are 96 unmatching members of which 60 with NAs in \textsf{project\_type}. Given that there are (a relatively small number of) 36 cases of nonNAs in project type and detailed IGAs, I will use information only in \textsf{igaX\_Y} and ignore \textsf{project\_type}. 
\item	There is one piece of information that may not to be dropped with \textsf{project\_type} where 0 members report ox in their project while IGAs report cows. I will overwrite cow as IGA with ox.
\item	\textsf{igaX\_Y} supposedly indicates \textsf{X}-th income generating activity in \textsf{Y}-th most recent project. But \textsf{year\_Y} shows that \textsf{igaX\_Y} is \textsf{Y}-th oldest project. \textsf{year\_2nd} (all 2014), \textsf{year\_3rd} (all 2015) are reported only for \textsf{traditional} indicates that \textsf{year\_Y} refers to disbursement years, not necessarily the project starting year. This is further supported by no \textsf{year\_2nd} is recorded for other arms. Information exists in \textsf{iga1\_1st}, \textsf{iga1\_2nd}, \textsf{iga1\_3rd} (most, 2nd most, 3rd most recent igas), but not in \textsf{iga2\_1st}, \textsf{iga2\_2nd}, \textsf{iga2\_3rd}, \textsf{iga3\_1st}, \textsf{iga3\_2nd}, \textsf{iga3\_3rd}. 
\end{itemize}
\begin{Schunk}
\begin{Soutput}
                  Project
IGAs               cow  ox goat/sheep business/trade land sum
  2 cows,goat        0   0          2              0    0   2
  2 cows,land        6   0          0              0    0   6
  2 cows,trade       5   0          0              3    0   8
  2 goats,cow        3   0          4              0    0   7
  2 goats,trade      0   0          3              2    0   5
  2 trades,cow       2   0          0              2    0   4
  2 trades,goat      0   0          0              1    0   1
  cow              327   0          0              0    0 327
  cow,goat,land      1   0          0              0    0   1
  cow,goat,trade     4   0          7              2    0  13
  cow,land,nutcorn   9   0          0              0    0   9
  cow,land,trade     3   0          0              0    0   3
  land               0   0          0              0    2   2
  ox                 0   1          0              0    0   1
  trade              0   0          0              1    0   1
  sum              360   1         16             11    2 390
\end{Soutput}
\begin{Soutput}
                  Project
IGAs               cow  ox goat/sheep business/trade land <NA> sum
  2 cows,goat        0   3          0              0    0    0   3
  2 cows,land        0   4          1              0    0    0   5
  2 cows,nutcorn     0   1          0              0    0    0   1
  2 cows,trade       0   5          3              0    0    3  11
  2 goats,cow        0   5          0              0    0    0   5
  2 goats,trade      2   1          0              0    0    7  10
  2 trades,cow       0   0          3              0    0    4   7
  2 trades,goat      0   1          0              0    0    2   3
  cow                0 179          5              1    1   34 220
  cow,goat,trade     0   5          0              0    0    1   6
  cow,land,nutcorn   0   8          0              0    0    1   9
  cow,land,trade     0   1          0              0    0    2   3
  goat               0   0          0              0    0    1   1
  house              0   0          0              0    0    1   1
  land               5   1          0              0    0    4  10
  ox                 1   0          0              0    0    0   1
  trade              6   5          1              0    0    0  12
  sum               14 219         13              1    1   60 308
\end{Soutput}
\begin{Soutput}
        year_2nd
year_1st  0 2014
    2013 27   95
\end{Soutput}
\begin{Soutput}
        year_3rd
year_1st  0 2015
    2013 27   95
\end{Soutput}
\begin{Soutput}
          Arm         BStatus                 IGAs    Project  
 traditional: 0   borrower:27   2 cows,land     : 8   cow :14  
 large      : 0                 2 cows,nutcorn  : 1   ox  :12  
 large grace:22                 cow,land,nutcorn:18   NA's: 1  
 cow        : 0                                                
 NA's       : 5                                                
\end{Soutput}
\begin{Soutput}
          Arm         BStatus               IGAs              Project   year_2nd 
 traditional:95   borrower:95   2 cows,trade  :19   cow           :21   2014:95  
                                cow,goat,trade:19   ox            :22            
                                2 goats,trade :15   goat/sheep    :23            
                                2 goats,cow   :12   business/trade:10            
                                2 trades,cow  :11   NA's          :19            
                                cow,land,trade: 6                                
                                (Other)       :13                                
 year_3rd 
 2015:95  
          
          
          
          
          
          
\end{Soutput}
\begin{Soutput}
          Arm         BStatus                 IGAs    Project   year_3rd
 large grace:22   borrower:27   2 cows,land     : 8   cow :14   0:27    
 NA's       : 5                 2 cows,nutcorn  : 1   ox  :12           
                                cow,land,nutcorn:18   NA's: 1           
\end{Soutput}
\end{Schunk}
Tabulation of loan projects shows that there is no member invested all in goats and goats are not the members' main assets. Among the 85 \textsf{tradtional} loan recipients who report their loan projects, there are 27 members who report to have purchased a goat twice and 15 who have invested in a retail trade twice. It is also puzzling that, among \textsf{traditional} arm members, 27 report to have invested in a cow twice, which seems unlikely with their purchasing powers. 
\begin{Schunk}
\begin{Soutput}
x
   2 cows,goat    2 cows,land   2 cows,trade    2 goats,cow  2 goats,trade   2 trades,cow 
             5              3             19             12             15             11 
 2 trades,goat  cow,goat,land cow,goat,trade cow,land,trade 
             4              1             19              6 
\end{Soutput}
\end{Schunk}
%\textsf{project\_type} and \textsf{project\_type\_others} give primary projects.
Number of reported IGAs by arm shows that \textsf{traditional} members report a project everytime they receive a loan, hence all have 3 IGAs. Interestingly, none has three goats.
\begin{Schunk}
\begin{Soutput}
                 1      3 sum
traditional   0.00 100.00  95
large       100.00   0.00 217
large grace  88.83  11.17 197
cow            NaN    NaN   0
<NA>         97.35   2.65 189
\end{Soutput}
\begin{Soutput}
x
   2 cows,goat    2 cows,land   2 cows,trade    2 goats,cow  2 goats,trade   2 trades,cow 
             5              3             19             12             15             11 
 2 trades,goat  cow,goat,land cow,goat,trade cow,land,trade 
             4              1             19              6 
\end{Soutput}
\end{Schunk}
%Proportion of single project reporting: Traditional vs. other arms.

Goat holding size and total holding increase by the final round but the number of holders is decreasing, indicating a limited number of expansion in goat holding. Interestingly, it is only \textsf{traditional} arm holding that are increasing while all ther arms reduce the goat holding size.

\verb|addmargins(table0(lvo[o800==1L & tee == 1, .(Arm, Num)]))|
\begin{Schunk}
\begin{Soutput}
             Num
Arm             1   2   3   4 Sum
  traditional  13   9  39 114 175
  large         6   6  22 166 200
  large grace  14   7  26 142 189
  cattle       11   8  20 160 199
  Sum          44  30 107 582 763
\end{Soutput}
\begin{Soutput}
          Arm          hhid          survey  NumOwned.goatsheep NumOwned.chickenduck
 traditional:20   Min.   : 7010103   1:116   0:100              0      :63          
 large      :14   1st Qu.: 7021186           1:  6              2      :19          
 large grace:51   Median : 7036864           2:  7              4      :16          
 cattle     :31   Mean   : 7818279           4:  3              3      : 6          
                  3rd Qu.: 7096233                              5      : 5          
                  Max.   :81710316                              1      : 3          
                                                                (Other): 4          
 NumCows ObPattern
 0:104   0111: 1  
 1:  8   1000:91  
 2:  3   1010: 1  
 3:  1   1011: 0  
         1100: 8  
         1110: 1  
         1111:14  
\end{Soutput}
\end{Schunk}
Cattle ownership at rd 1.
\begin{Schunk}
\begin{Soutput}
             NumCows
Arm             0   1   2   3   4   5 Sum
  traditional 147  20   6   2   0   0 175
  large       156  31   8   2   2   0 199
  large grace 163  25   9   1   0   1 199
  cattle      167  24   7   1   0   0 199
  Sum         633 100  30   6   2   1 772
\end{Soutput}
\end{Schunk}
Cattle ownership of attriters (at round 4) at rd 1.
\begin{Schunk}
\begin{Soutput}
             NumCows
Arm            0  1  2  3 Sum
  traditional 18  1  1  0  20
  large        1  0  0  0   1
  large grace  3  0  0  0   3
  cattle       7  2  3  1  13
  Sum         29  3  4  1  37
\end{Soutput}
\end{Schunk}
Cattle ownership at rd 4 
\begin{Schunk}
\begin{Soutput}
             NumCows
Arm             0   1   2   3   4   5   6   8   9 <NA> Sum
  traditional   2  59  30   8   2   0   0   0   0   31 132
  large         0  62  67  21   4   3   2   0   1   29 189
  large grace   1  61  58  11   5   1   0   1   0   24 162
  cattle        1  68  61  16   2   0   0   0   0   22 170
  Sum           4 250 216  56  13   4   2   1   1  106 653
\end{Soutput}
\begin{Soutput}
            Arm survey     N MeanNumCow MedianNumCow
         <fctr>  <num> <int>      <num>        <num>
 1: traditional      1   175   0.217143            0
 2: traditional      2   140   1.542169            1
 3: traditional      3   157   1.440678            1
 4: traditional      4   132   1.495050            1
 5:       large      1   199   0.306533            0
 6:       large      2   172   1.953125            2
 7:       large      3   188   1.784810            2
 8:       large      4   189   1.943750            2
 9: large grace      1   199   0.256281            0
10: large grace      2   154   1.530435            1
11: large grace      3   170   1.496599            1
12: large grace      4   162   1.760870            2
13:      cattle      1   199   0.206030            0
14:      cattle      2   177   1.365517            1
15:      cattle      3   181   1.436709            1
16:      cattle      4   170   1.662162            2
\end{Soutput}
\end{Schunk}
Last observed round.
\begin{Schunk}
\begin{Soutput}
                      LastObservedRound
BStatus                  1   2   3   4 sum
  borrower              11   7  19 538 575
  pure saver             0   0   0   0   0
  individual rejection  16   3   4  66  89
  group rejection       15   2   4  49  70
  rejection by flood    13   1  26   0  40
  sum                   55  13  53 653 774
\end{Soutput}
\end{Schunk}
Attach 0 cattle ownership when nothing is reported.
\begin{Schunk}
\begin{Soutput}
             NumCows
Arm             0   1   2   3   4   5 sum
  traditional 147  20   6   2   0   0 175
  large       156  31   8   2   2   0 199
  large grace 163  25   9   1   0   1 199
  cattle      167  24   7   1   0   0 199
  sum         633 100  30   6   2   1 772
\end{Soutput}
\end{Schunk}
Number of cattle in round 4.
\begin{Schunk}
\begin{Soutput}
             NumCows
Arm             0   1   2   3   4   5   6   8   9 sum
  traditional  33  59  30   8   2   0   0   0   0 132
  large        29  62  67  21   4   3   2   0   1 189
  large grace  25  61  58  11   5   1   0   1   0 162
  cattle       23  68  61  16   2   0   0   0   0 170
  sum         110 250 216  56  13   4   2   1   1 653
\end{Soutput}
\end{Schunk}
There are 5 members in \textsf{cattle} arm who report not to own cattle at least once after receiving cattle. Total holding size and holders may be too low. Below gives holding size of cattle among nonattriting members in \textsf{cattle} arm.
\begin{Schunk}
\begin{Soutput}
      NumOwned.cowox
survey   0   1   2   3   4 <NA> Sum
     1 150  22   4   0   0    0 176
     2   2  93  28  10   1   29 163
     3   2  97  36   9   3   22 169
     4   1  68  61  16   2   22 170
\end{Soutput}
\end{Schunk}
Members of \textsf{traditional} arm have the smallest cattle holding. In \textsc{\small Table \ref{table anova CattleHoldingArm}}, ANOVA and Kruskal-Wallis tests indicate that means of cattle holding are different between arms in 2017. Tukey HST gives test results that account for multiple testing and shows that there is a difference between \textsf{traditional} and \textsf{large}, and other arms are in between yet their standard errors are too large to be considered statistically different from both extremes.

\begin{Schunk}
\begin{Soutput}
             NumCows
Arm             0   1   2   3   4   5   6   8   9 sum
  Traditional  33  59  30   8   2   0   0   0   0 132
  Large        29  62  67  21   4   3   2   0   1 189
  Large grace  25  61  58  11   5   1   0   1   0 162
  Cattle       23  68  61  16   2   0   0   0   0 170
  sum         110 250 216  56  13   4   2   1   1 653
\end{Soutput}
\end{Schunk}
Cattle arm: add a cow for borrowers if NumCows is NA or zero in rd 2 onwards.
\begin{Schunk}
\begin{Soutput}
             NumCows
Arm             0   1   2   3   4   5   6   8   9 sum
  Traditional  33  59  30   8   2   0   0   0   0 132
  Large        29  62  67  21   4   3   2   0   1 189
  Large grace  25  61  58  11   5   1   0   1   0 162
  Cattle       11  80  61  16   2   0   0   0   0 170
  sum          98 262 216  56  13   4   2   1   1 653
\end{Soutput}
\begin{Soutput}
Margins computed over dimensions
in the following order:
1: Arm
2: groupid
\end{Soutput}
\begin{Soutput}
             groupid
Arm           70203 70206 70210 70538 70962 sum
  Traditional     0     0     0     0     0   0
  Large           1     0     0     1     4   6
  Large grace     0     1     1     0     0   2
  Cattle          0     0     0     0     0   0
  sum             1     1     1     1     4   8
\end{Soutput}
\end{Schunk}

\mpage{\linewidth}{
\renewcommand{\arraystretch}{.6}
\hfil\textsc{\footnotesize Table \refstepcounter{table}\thetable: Anova results for cattle holding equality by arm\label{table anova CattleHoldingArm}}\\
\hfil\input{c:/data/GUK/analysis/program/table/EstimationMemo/anovaCowResults.tex}\\
\renewcommand{\arraystretch}{1}
\hfil\begin{tabular}{>{\hfill\scriptsize}p{1cm}<{}>{\scriptsize}p{12cm}<{\hfill}}
Source: & Survey data.\\
Note:& Each column uses respective year cattle ownership information. For ANOVA and Kruskal-Wallis, each entry indicates $p$ values. ANOVA tests for the null of equality of means under normality. Kruskal-Wallis tests for the null of no stochastic dominance among samples without using the normality assumption. Tukey's honest significant tests show difference in means and $p$ values in parenthesis that account for multiple testing under normality. In column 2, we edited data by assigning 1 to members of \textsf{cattle} arm at dates after disbursement if reported holding is NA or zero. \\[1ex]
\end{tabular}}


%Drop these households from livestock data.

\begin{Schunk}
\begin{Soutput}
   1  2  3  4 5 6 7 sum total HoldingSize
1 39 44 14 33 3 6 1 140   359        2.56
2  0  0  0  0 0 0 0   0     0         NaN
3  0  0  0  0 0 0 0   0     0         NaN
4  0  0  0  0 0 0 0   0     0         NaN
\end{Soutput}
\begin{Soutput}
Warning: Invalid .internal.selfref detected and fixed by taking a (shallow) copy of the data.table so that := can add this new column by reference. At an earlier point, this data.table has been copied by R (or was created manually using structure() or similar). Avoid names<- and attr<- which in R currently (and oddly) may copy the whole data.table. Use set* syntax instead to avoid copying: ?set, ?setnames and ?setattr. If this message doesn't help, please report your use case to the data.table issue tracker so the root cause can be fixed or this message improved.
\end{Soutput}
\end{Schunk}
\begin{Schunk}
\begin{Soutput}
             InitialOwner
Arm             0   1 Sum
  Traditional 147  28 175
  Large       156  43 199
  Large grace 163  36 199
  Cattle      167  32 199
  Sum         633 139 772
\end{Soutput}
\begin{Soutput}
Warning: Invalid .internal.selfref detected and fixed by taking a (shallow) copy of the data.table so that := can add this new column by reference. At an earlier point, this data.table has been copied by R (or was created manually using structure() or similar). Avoid names<- and attr<- which in R currently (and oddly) may copy the whole data.table. Use set* syntax instead to avoid copying: ?set, ?setnames and ?setattr. If this message doesn't help, please report your use case to the data.table issue tracker so the root cause can be fixed or this message improved.
\end{Soutput}
\end{Schunk}
Given the misreporting in large loans arms, the power may get affected and only \textsf{large} seems to stand out from all other arms, while \textsf{large grace, cattle} are not different in terms of cattle ownership against \textsf{traditional}. 








\mpage{\linewidth}{
\hfil\textsc{\footnotesize Figure \refstepcounter{figure}\thefigure: Cattle holding by arm and borrower status\label{fig CattleHoldingArmBStatus}}\\
\hfil\includegraphics{c:/data/GUK/analysis/program/figure/EstimationMemo/CowHoldingByArmBStatus.jpg}\\
\renewcommand{\arraystretch}{1}
\hfil\begin{tabular}{>{\hfill\scriptsize}p{1cm}<{}>{\scriptsize}p{12cm}<{\hfill}}
Source: & Survey data.\\
Note:& Numbers of loan recipients are 85, 170, 166, 152, numbers of reported livestock holding are 85, 170, 166, 152 for \textsf{traditional, large, large grace, cattle} arms, respectively. Red horizontal lines indicate number of loan recipients. \\[1ex]
\end{tabular}
}


\mpage{\linewidth}{
\hfil\textsc{\footnotesize Figure \refstepcounter{figure}\thefigure: Cattle holding by arm\label{fig CattleHoldingArm}}\\
\hfil\includegraphics{c:/data/GUK/analysis/program/figure/EstimationMemo/CattleHoldingByArm.jpg}\\
\renewcommand{\arraystretch}{1}
\hfil\begin{tabular}{>{\hfill\scriptsize}p{1cm}<{}>{\scriptsize}p{12cm}<{\hfill}}
Source: & Survey data.\\
Note:& Numbers of survey participants are 175, 199, 199, 199 for \textsf{traditional, large, large grace, cattle} arms in round 1, respectively. Holders rates are the number of cattle owners per arm size, holding size is average holding per owner, initial owner holding are average holding per owner who held cattle at period 2, and per capita holding is cattle owned per arm member. Initial owner holding and holder rates show impacts on the intensive and extensive margins, respectively. Per capita holding shows the total impacts on cattle holding.\\[1ex]
\end{tabular}
}

\mpage{\linewidth}{
\hfil\textsc{\footnotesize Figure \refstepcounter{figure}\thefigure: Project choices\label{fig ProjectChoices}}\\
\hfil\includegraphics{c:/data/GUK/analysis/program/figure/ImpactEstimationOriginal1600Memo2/ProjectChoices.jpg}\\
\renewcommand{\arraystretch}{1}
\hfil\begin{tabular}{>{\hfill\scriptsize}p{1cm}<{}>{\scriptsize}p{12cm}<{\hfill}}
Source: & Survey data.\\
Note:& Ratios of reported project choices using the lending to total number of projects in \textsf{InitialSample}. NAs include nonresponse to the question and dropped out individuals.  \\[1ex]
\end{tabular}
}

\mpage{\linewidth}{
\hfil\textsc{\footnotesize Figure \refstepcounter{figure}\thefigure: Largest fixed investment amount\label{fig FixedInvest}}\\
\hfil\includegraphics{c:/data/GUK/analysis/program/figure/ImpactEstimationOriginal1600Memo2/FixedInvestmentAmount.jpg}\\
\renewcommand{\arraystretch}{1}
\hfil\begin{tabular}{>{\hfill\scriptsize}p{1cm}<{}>{\scriptsize}p{12cm}<{\hfill}}
Source: & Survey data.\\
Note:& Reported largest one-off investment amounts of the lending. \\[1ex]
\end{tabular}
}

\mpage{\linewidth}{
\hfil\textsc{\footnotesize Figure \refstepcounter{figure}\thefigure: Fixed investment sequence and amounts\label{fig first2ndFixedInvest}}\\
%\hfil\includegraphics{  paste0(pathprogram, "figure/ImpactEstimationOriginal1600Memo2/FirstFixedInvestmentAmountByYear.png")}\\
\hfil\includegraphics{c:/data/GUK/analysis/program/figure/ImpactEstimationOriginal1600Memo2/FixedInvestmentAmountBySequence.png}\\
\renewcommand{\arraystretch}{1}
\hfil\begin{tabular}{>{\hfill\scriptsize}p{1cm}<{}>{\scriptsize}p{12cm}<{\hfill}}
Source: & Survey data.\\
Note:& Reported largest one-off investment amounts of the lending. Top figure is the first investments reported by year, bottom figure is later investments reported by the sequence of investment projects. \\[1ex]
\end{tabular}
}


\footnotesize\bibliographystyle{aer}
\setlength{\baselineskip}{10pt}
\bibliography{c:/seiro/settings/TeX/seiro}


\begin{Schunk}
\begin{Soutput}
       objectName memorySize
                Z  113.52 MB
             adw2   93.18 MB
              arA   77.64 MB
          estlist   76.85 MB
           linhyp   61.49 MB
           svTP11   53.09 MB
            svTP5   52.27 MB
           svTP10   52.27 MB
            svTP4   50.33 MB
            svTP9   50.33 MB
            svTP2   49.51 MB
            svTP7   49.51 MB
            svTP3   48.69 MB
            svTP8   48.69 MB
            svTP1   47.87 MB
            svTP6   47.87 MB
             arA2    46.4 MB
              aob   45.51 MB
 arACompletePanel   45.51 MB
            svP11   44.08 MB
\end{Soutput}
\end{Schunk}

\end{document}
