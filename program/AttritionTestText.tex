	\textsc{\normalsize Table \ref{\Sexpr{TabLabel1[1]}}} shows results from tests of independence between attriters and nonattriters. Attrition is defined as attrition from household surveys, not from the lending program. We see the moderate rate of attrition is not correlated with household level characteristics%, except less risk torelance in \textsf{RiskPrefVal}, 
	at the conventional $p$ value level. Productive asset amounts seem to differ between attriters and nonattriters at $p=.105$, with the former being larger than the latter. This positive attrition selection can cause underestimation of impacts, if the asset values are positively correlated with entrepreneurial capacity. We also see that the attriters are less risk tolerant in terms of minimum expected payoff to choose a risky option in \textsf{RiskPrefVal}. \textsc{\normalsize Table \ref{\Sexpr{TabLabel1[grep("of attri.* traditional arm$", TabLabel1)]}}} shows attrition in the \textsf{traditional} arm. Household heads of attriters are relatively less literate than nonattriters. We observe the \textsf{traditional} arm attriters are less risk tolerant the nonattriters.
	\textsc{\normalsize Table \ref{\Sexpr{TabLabel1[grep("of attri.* non-traditional arm$", TabLabel1)]}}} compares attriters and nonattriters in the non-\textsf{traditional} arm. Unlike \textsf{traditional} arm attriters, non-\textsf{traditional} arm attriters have more literate household heads, have a larger household size, are more exposed to floods, and have larger productive assets. The \textsf{traditional} arm attriters may be less entrepreneurial, if anything, so their attrition may upwardly bias the positive gains of the arm, hence understate the impacts of non-\textsf{traditional} arm. These are explicitly shown in \textsc{\normalsize Table \ref{\Sexpr{TabLabel1[grep("of attri.* of", TabLabel1)]}}} where we compare attriters of \textsf{traditional} and non-\textsf{traditional} arms. Overall, attrition may have attenuated the impacts but is not likely to have inflated them.\footnote{So one can employ the Lee bounds for stronger results, but doing so will give us less precision and require more assumptions. We will not use the Lee bounds \textcolor{red}{[we can show them if necessary]}. } We observe the non-\textsf{traditional} arm attriters are also less risk tolerant than the nonattriters.

	For the microfinance institutions (MFIs), attrition of the loan receiving members poses a threat to their business continuation. Financial institutions often use observable characteristics, such as collateralisable assets, and easily surveyed chracteristics, such as job experiences and schooling of borrowers, and are likely to lend if the assets levels are greater and the borrowers have relevant job experiences and more schooling. We first examine if such screening variables have any predictive power in terms of loan rejection or borrower attrition under our lending. \textsc{\normalsize Table \ref{\Sexpr{TabLabel1[grep("active status$", TabLabel1)]}}} compares potential MFI targets (nonattriting borrowers, noted as \textsf{Active}) vs. non-targets (attriting borrowers or loan rejecters, noted as \textsf{NonSurvived}) in all arms. It shows potential targets at the baseline have larger values in livestock and greater number of cattle, and are less affected by the flood, which conforms the conventional wisdom of lenders in using these aspects in their loan decisions. We also see that more risk torelant members are likely to be borrowers and do not attrit. Next, we examine if the relationship of having ``less favourable'' values in these characteristics and attrition is mitigated under various loan characteristics. In \textsc{\normalsize Table \ref{\Sexpr{TabLabel1[grep("active members of cattle and large grace", TabLabel1)]}}}, we restrict our attention to the potential MFI targets, or the nonattriting borrowers, and compare between \textsf{cattle} and \textsf{large grace} arms, whose difference is efffectively the presence of managerial supports that the former provides. \label{PageOfAttrition}%Comparing the nonattriting borrower, characteristics are similar except that the \textsf{traditional} members are more exposed to the flood than the non-\textsf{traditional} members. 
	Comparing against the \textsf{large grace} arm, nonattriting borrowers of the \textsf{cattle} arm are more exposed to the flood ($p=.055$), have less productive assets ($p=.003$), have lower net asset values ($p=.046$), and have fewer livestock ($p=.139$). This shows that the smaller livestock holders or individuals with less experienced in livestock are encouraged to participate and continue to operate in the \textsf{cattle} arm that has a managerial support program, with all other features being equal. This is consistent with our analysis of participation in \textsc{\normalsize Table \ref{\Sexpr{TabLabel1[grep("bo.*non-ca", TabLabel1)]}}} which weakly hints that the \textsf{cattle} arm's managerial support programs may have encouraged participation of inexperienced or lower asset holders. This also underscores our interpretation that the current impact estimates may be downwardly biased, if any, as people who would otherwise attrit or reject in the \textsf{cattle} arm stayed on. This result is confirmed with lower $p$ values due to a larger sample size when we compare the nonattriting borrowers between \textsf{cattle} arm with all other arms in \textsc{\normalsize Table \ref{\Sexpr{TabLabel1[grep("Permutation test results of active members of cattle and all", TabLabel1)]}}}. At the baseline, \textsf{cattle} arm nonattriting borrowers have smaller baseline livestock holding ($p$ value = .016) and smaller baseline net asset holding ($p$ value = .007) than other arms' nonattriting borrowers. 
