% Set the overall layout of the tree
\tikzstyle{level 1}=[level distance = 1.75cm, sibling distance = 2.1cm]
\tikzstyle{level 2}=[level distance = 3.00cm, sibling distance = 2.0cm]
\tikzstyle{level 3}=[level distance = 1.5cm, sibling distance = 0.85cm]
\tikzstyle{level 4}=[level distance = 2.5cm, sibling distance = 1cm]
\tikzstyle{level 5}=[level distance = 3cm, sibling distance = 2cm]
\tikzstyle{level 6}=[level distance = 3cm, sibling distance = 2cm]
\tikzstyle{level 7}=[level distance = 3cm, sibling distance = 2cm]
\tikzstyle{level 8}=[level distance = 2.25cm, sibling distance = 2cm]
% Define styles for bags and leafs
\tikzstyle{bag} = [text width = 5em, text centered, thick]
\tikzstyle{testbag} = [minimum width=2.5em, text centered, thick, top color = darkgreen, bottom color = green]
\tikzstyle{Hbbag} = [text width=3em, text centered, thick, top color = blue!40, bottom color = blue!40, color = black] 
\tikzstyle{govbag} = [minimum width=2.5em, text centered, thick, top color = gray!30, bottom color = gray!10, color = black]
%\tikzstyle{end} = [circle, minimum width = 3pt, fill, inner sep = 0pt]
\tikzstyle{end} = [text width = 4em, text centered, thick]
\tikzstyle{endbox} = [draw, very thick, fill = white, rectangle, 
	minimum height = 3em, minimum width=7em, 
	node distance = 8em, font = {\sffamily\bfseries}]
\tikzstyle{endboxEdgePortion} = [black, thick]
\tikzstyle{endboxEdge} = [endboxEdgePortion, ->]
\tikzstyle{edgeLabel} = [pos = 0.5, text centered, font = {\sffamily\small}]
\tikzstyle{bracebag} = [decoration = {brace}, decorate]
\tikzstyle{surveybag} = [edgeLabel, xshift = 0em, shading = ball,
				top color = green, bottom color = green, color = black]
\tikzstyle{floodbag} = [edgeLabel, xshift = 0em, shading = ball,
				top color = paleblue, bottom color = blue, color = black]
\tikzstyle{nobackground} = [text centered, thick, top color =yellow!30, 
				bottom color = yellow!30, opacity = 50, color = black]
% The sloped option gives rotated edge labels. Personally
% I find sloped labels a bit difficult to read. Remove the sloped options
% to get horizontal labels. 
\hspace{-2cm}\begin{tikzpicture}
[	grow = right, sloped, 
%	every node/.style={top color = blue, bottom color = blue!30, rounded corners, text = white}
	every node/.style = {shading = ball, rounded corners, text = white},
	%background rectangle/.style = {fill = gray, opacity = .2}, 
	%framed
	decoration={brace, amplitude = 2pt},
]

\node (randomize) [testbag] {randomize}
child 
{
	%	start packaged loans arm
	node (packaged) {}
		child
		{
			node (ps) {}
				child 
				{
					node (pt1) [Hbbag] {cows}
					child { 
							node (pt2) {}
								child { node (pt3) {}
									child { node (ptend) {} }
									}
						}
				}
			edge from parent
		}
	edge from parent
}
child 
{
	%	start large loans with grace arm
	node (grace) {}
		child 
		{
			node (gs) {}
				child 
				{
					node (gt1) [Hbbag] {large}
					child { 
							node (gt2) {}
								child { node (gt3) {}
									child { node (gtend) {} }
									}
						}
				}
			edge from parent
		}
	edge from parent
}
child 
{
	%	start large loans arm
	node (large) {}
		child 
		{
			node (ls) {}
				child 
				{
					node (lt1) [Hbbag] {large}
					child { 
							node (lt2) {}
								child { node (lt3) {}
									child { node (ltend) {} }
									}
						}
				}
			edge from parent
		}
	edge from parent
}
child 
{
	%	start small loans arm
	node (traditional) {}
		child 
		{
			node (ts) {}
				child 
				{
					node (tt1) [Hbbag] {small}
					child { 
							node (tt2) {}
								child { node (tt3) {}
									child { node (ttend) {} }
									}
						}
				}
			edge from parent
		}
	edge from parent
}
;

% draw time line at the bottom
%\node (calendar1) at ($(baseline) + (0, -5cm)$){};
%\node (calendar2) at ($(f3) + (0, -5cm)$){};
%\draw (calendar1) -- (calendar2);
\setcounter{armindex}{2011}
\def\ytimeline{-5.35}
\def\ytimelinetwo{1}
\def\ytimelinethree{8}
\draw [thick] (1cm, \ytimeline cm) -- (15cm, \ytimeline cm);
\foreach \x in {1, 7, 10.35, 13.35}
{
	\stepcounter{armindex}
	\draw [thick] (\x cm, \ytimeline) -- ($(\x cm, \ytimeline) + (0, .5cm)$);
	\draw ($(\x cm, \ytimeline) + (0, .75cm)$) node [text centered] {\thearmindex};
	\ifthenelse{\equal{\thearmindex}{12}}{\setcounter{armindex}{0}}{};
	\draw [thick] (\x cm, \ytimeline) -- ($(\x cm, \ytimeline) + (0, .25cm)$);
};

%  target
\node (peri) at ($(randomize) + (0, 2cm)$)
	{\hfil\begin{minipage}[t]{1.75cm}
	\hfil villages\\
	\hfil on chars\setlength{\baselineskip}{12pt}
	\end{minipage}};
\draw[->, >= latex', thick] (peri.south) -- (randomize.north);

%	%  curly braced texts
%		%	mirror: turn brace to opposite side
%	\path (cp1.west |- cn.north)+(0, .5ex) node (phase1topleft) {};
%	\path (tp2.east |- cn.north)+(0, .5ex) node (phase1topright) {};
%	\path (tp4.west |- cn.north)+(0, .5ex) node (phase2topleft) {};
%	\path (tp5.east |- cn.north)+(0, .5ex) node (phase2topright) {};
%	\draw[bracebag] let \p1=(phase1topleft), \p2=(phase1topright) in
%		($(\x1, \y1)$) -- ($(\x2, \y1)$) 
%		node (ph1) [midway, govbag, above = 2pt] {Phase 1};
%	\draw[bracebag] let \p1=(phase2topleft), \p2=(phase2topright) in
%		($(\x1, \y1)$) -- ($(\x2, \y1)$) 
%		node (ph2) [midway, govbag, above = 2pt] {Phase 2};

% composition of arms
% \pause;
\node (armcomp)[govbag] at ($(tt1.north |- tt1.east) + (10em, 10ex)$)
	{\mpage{3.5cm}{7 ultra poor\hfill\\ 3 moderately poor\hfill\setlength{\baselineskip}{10pt}}\mpage{2cm}{per village}};
\draw (armcomp.west) edge[out = 180, in = 90, ->, >= latex', thick] 
	node[midway, above, nobackground] {} (tt1.north);
% \pause;
\node (armsize)[govbag] at ($(tt1.north |- tt1.east) + (-10em, 10ex)$)
	{\mpage{3.75cm}{20 HHs$\times$20 villages\hfill\\ = 400 HHs\hfill\setlength{\baselineskip}{10pt}} \mpage{2cm}{per arm}};
\draw (armsize.east) edge[out = 0, in = 90, ->, >= latex', thick] 
	node[midway, above, nobackground] {} (tt1.north);
%\node (armcomp)[govbag] at ($(tt1.south) + (1em, -3ex)$){400 HHs};
%\node (armcomp)[govbag] at ($(gt1.south) + (1em, -3ex)$){400 HHs};
%\node (armcomp)[govbag] at ($(lt1.south) + (1em, -3ex)$){400 HHs};
%\node (armcomp)[govbag] at ($(pt1.south) + (1em, -3ex)$){400 HHs};

%	common events
% \pause;
\draw[->, >= latex', thick, dashed]
	($(large) + (3ex, -12.0ex)$)
	edge[endboxEdge] node (listing) [edgeLabel, xshift = 0em, shading = ball,
				top color = orange, bottom color = orange, color = black]
		{\mpage{9cm}{\hfil household census and gradation}} 
	($(large) + (3ex, 3.5ex)$);
% \pause;
\draw[->, >= latex', thick, dashed]
	($(large) + (7.5ex, -12.0ex)$)
	edge[endboxEdge] node (baseline) [surveybag]
		{\mpage{9cm}{\hfil baseline survey}} 
	($(large) + (7.5ex, 3.5ex)$);
% \pause;
\draw[->, >= latex', thick, dashed]
	($(large) + (12ex, -12.0ex)$)
	edge[endboxEdge] node (selection) [edgeLabel, xshift = 0em, shading = ball,
				top color = purple, bottom color = purple]
		{\mpage{9cm}{\hfil randomized selection}} 
	($(large) + (12ex, 3.5ex)$);
% \pause;
\draw[->, >= latex', thick, dashed]
	($(large) + (16ex, -12.0ex)$)
	edge[endboxEdge] node (training) [edgeLabel, xshift = 0em, shading = ball,
				top color = darkpurple, bottom color = purple]
		{\mpage{9cm}{\hfil training}} 
	($(large) + (16ex, 3.5ex)$);
%	\draw[->, >= latex', thick, dashed]
%		($(large) + (30ex, -0.5ex)$)
%		edge[endboxEdge] node (selection) [edgeLabel, xshift = 0em, shading = ball,
%					top color = green, bottom color = green]
%			{\mpage{8cm}{\hfil follow up survey 1}} 
%		($(large) + (30ex, 1.5ex)$);

% traditinal loans arm
% \pause;
\draw[fill = red, opacity = .15] 
	($(tt1) + (-1cm, -1.4cm)$) rectangle 
	($(ttend) + (.5cm, .75cm)$);
% \pause;
\node (trepay1) at ($(tt1.north |- tt1.east) + (3.5em, -2ex)$){repay};
% \pause;
\node (trad1) at ($(tt2.north |- tt2.east) + (3.5em, 2ex)$){loan 2};
% \pause;
\node (trepay2) at ($(tt2.north |- tt2.east) + (3.5em, -2ex)$){repay};
% \pause;
\node (trad2) at ($(tt3.north |- tt3.east) + (3.5em, 2ex)$){loan 3};
% \pause;
\node (trepay3) at ($(tt3.north |- tt3.east) + (3.5em, -2ex)$){repay};
% large loans arm
% \pause;
\draw[fill = green, opacity = .15] 
	($(lt1) + (-1cm, -1.4cm)$) rectangle 
	($(ltend) + (.5cm, .6cm)$);
% \pause;
\node (lrepay1) at ($(lt1.north |- lt1.east) + (3.5em, -2ex)$){repay};
% \pause;
\node (lrepay2) at ($(lt2.north |- lt2.east) + (3.5em, -2ex)$){repay};
% \pause;
\node (lrepay3) at ($(lt3.north |- lt3.east) + (3.5em, -2ex)$){repay};
% large loans with grace arm
% \pause;
\draw[fill = red, opacity = .15] 
	($(gt1) + (-1cm, -1.4cm)$) rectangle 
	($(gtend) + (.5cm, .6cm)$);
% \pause;
\node (lrepay2) at ($(gt2.north |- gt2.east) + (3.5em, -2ex)$){repay};
% \pause;
\node (lrepay3) at ($(gt3.north |- gt3.east) + (3.5em, -2ex)$){repay};
% packaged loans arm
% \pause;
\draw[fill = yellow, opacity = .15] 
	($(pt1) + (-1cm, -1.4cm)$) rectangle 
	($(ptend) + (.5cm, .6cm)$);
% \pause;
\node (prepay2) at ($(pt2.north |- pt2.east) + (3.5em, -2ex)$){repay};
% \pause;
\node (prepay3) at ($(pt3.north |- pt3.east) + (3.5em, -2ex)$){repay};

% surveys
% \pause;
\draw[->, >= latex', thick, dashed]
	($(large) + (7.0cm, -12.0ex)$)
	edge[endboxEdge] node (f1) [surveybag]
		{\mpage{9cm}{\hfil follow up survey 1}} 
	($(large) + (7.0cm, 3.5ex)$);
% \pause;
\draw[->, >= latex', thick, dashed]
	($(large) + (10.0cm, -12.0ex)$)
	edge[endboxEdge] node (f2) [surveybag]
		{\mpage{9cm}{\hfil follow up survey 2}} 
	($(large) + (10.0cm, 3.5ex)$);
% \pause;
\draw[->, >= latex', thick, dashed]
	($(large) + (13.0cm, -12.0ex)$)
	edge[endboxEdge] node (f3) [surveybag]
		{\mpage{9cm}{\hfil follow up survey 3}} 
	($(large) + (13.0cm, 3.5ex)$);

% Overwrite with flood
\draw[->, >= latex', thick, dashed]
	($(large) + (7.0cm, -12.0ex)$)
% pauase;
	edge[endboxEdge] node (f1) [floodbag]
		{\mpage{9cm}{\hfil Flood}} 
	($(large) + (7.0cm, 3.5ex)$);
% pauase;
\draw[->, >= latex', thick, dashed]
	($(large) + (10.0cm, -12.0ex)$)
	edge[endboxEdge] node (f2) [surveybag]
		{\mpage{9cm}{\hfil follow up survey 1}} 
	($(large) + (10.0cm, 3.5ex)$);
% pauase;
\draw[->, >= latex', thick, dashed]
	($(large) + (13.0cm, -12.0ex)$)
	edge[endboxEdge] node (f3) [surveybag]
		{\mpage{9cm}{\hfil follow up survey 2}} 
	($(large) + (13.0cm, 3.5ex)$);


% 	% Draw the background
% \begin{pgfonlayer}{background}
% 	\path (peri.west |- peri.north)+(-0.5em, 8.5em) 
% 		node (topleft1) {};
% 	\path (pcend.east |- pcend.south)+(0.5cm, -4.5em) 
% 		node (bottomright1) {};
% 	\path[fill = yellow!30, rounded corners]
% 		(topleft1) rectangle (bottomright1);
% \end{pgfonlayer}

\end{tikzpicture}
