%  path0 <- "c:/data/GUK/"; path <- paste0(path0, "analysis/"); setwd(pathprogram <- paste0(path, "program/")); system("recycle c:/data/GUK/analysis/program/cache/PermutationTests/"); library(knitr); knit("PermutationTests.rnw", "PermutationTests.tex"); system("platex PermutationTests"); system("dvipdfmx PermutationTests")

\input{c:/migrate/R/knitrPreamble/knitr_preamble.rnw}
\renewcommand\Routcolor{\color{gray30}}
\newtheorem{finding}{Finding}[section]
\makeatletter
\g@addto@macro{\UrlBreaks}{\UrlOrds}
\newcommand\gobblepars{%
    \@ifnextchar\par%
        {\expandafter\gobblepars\@gobble}%
        {}}
\newenvironment{lightgrayleftbar}{%
  \def\FrameCommand{\textcolor{lightgray}{\vrule width 1zw} \hspace{10pt}}% 
  \MakeFramed {\advance\hsize-\width \FrameRestore}}%
{\endMakeFramed}
\newenvironment{palepinkleftbar}{%
  \def\FrameCommand{\textcolor{palepink}{\vrule width 1zw} \hspace{10pt}}% 
  \MakeFramed {\advance\hsize-\width \FrameRestore}}%
{\endMakeFramed}
\makeatother
\usepackage{caption}
\usepackage{setspace}
\usepackage{framed}
\captionsetup[figure]{font={stretch=.6}} 
\def\pgfsysdriver{pgfsys-dvipdfm.def}
\usepackage{tikz}
\usetikzlibrary{calc, arrows, decorations, decorations.pathreplacing, backgrounds}
\usepackage{adjustbox}
\tikzstyle{toprow} =
[
top color = gray!20, bottom color = gray!50, thick
]
\tikzstyle{maintable} =
[
top color = blue!1, bottom color = blue!20, draw = white
%top color = green!1, bottom color = green!20, draw = white
]
\tikzset{
%Define standard arrow tip
>=stealth',
%Define style for different line styles
help lines/.style={dashed, thick},
axis/.style={<->},
important line/.style={thick},
connection/.style={thick, dotted},
}


\begin{document}
\setlength{\baselineskip}{12pt}









\hfil Permutation tests using \textsf{o800} == 1\\

\hfil\MonthDY\\
\hfil{\footnotesize\currenttime}\\

\hfil Seiro Ito

\setcounter{tocdepth}{3}
%\tableofcontents

\setlength{\parindent}{1em}
\vspace{2ex}




Group level mean as a unit of observation. Use \textsf{coin} package's \textsf{independence\_test}: Approximate permutation tests by randomly resampling 100000 times.


\hspace{-1.0cm}\begin{minipage}[t]{15cm}
\hfil\textsc{\normalsize Table \refstepcounter{table}\thetable: Permutation test results\label{tab perm}}\\
\setlength{\tabcolsep}{.5pt}
\setlength{\baselineskip}{8pt}
\renewcommand{\arraystretch}{.50}
\hfil\begin{tikzpicture}
\node (tbl) {\input{c:/data/GUK/analysis/save/EstimationMemo/PermutationTestResults.tex}};
%\input{c:/dropbox/data/ramadan/save/tablecolortemplate.tex}
\end{tikzpicture}\\
\renewcommand{\arraystretch}{.8}
\setlength{\tabcolsep}{1pt}
\begin{tabular}{>{\hfill\scriptsize}p{1cm}<{}>{\scriptsize}p{13cm}<{\hfill}}
Source:& \scriptsize Estimated with GUK administrative and survey data.\\
Notes: & \textsf{R}'s package \textsf{coin} is used on the baseline group mean covariates to conduct approximate permutation tests at the group level. Number of repetition is set to 100000. Number of groups is 72. Holm's step-down method is used to adjust for multiple testing of a multi-factor grouping variable. For example, for the \textsf{traditional} arm, comparisons are made against other three arms, while the same comparisons are made in other arms against \textsf{traditional}. Specifically, \textsf{independence\_test} function is used for permutation tests and \textsf{pvalue} function is applied with an option \textsf{method = ``step-down''}. \textsf{HeadLiteracy} is an indicator variable of household head literacy. \textsf{HeadAge} is age of household head. \textsf{HHsize} is total number of household members. \textsf{FloodInRd1} is an indicator variable of flood exposure. \textsf{HAssetAmount} and \textsf{PAssetAmount} are amount of household and productive assets, respectively, in BDT, \textsf{NumCows} is cattle holding per household. \textsf{NetValue} is net asset values in BDT per housheold using asset items observed in all 4 rounds. \textsf{BroadNetValue} is net asset values in BDT per housheold for all asset items. \textsf{Attrited} indicates attrition rates in the household survey, and \textsf{GRejected} and \textsf{IRejected} show group rejection rates and individual rejection rates to the lending program. \textsf{Non-attriting borrowers} indicates the ratio of non-attriting borrowers to all borrowers. Because attrition and rejection are separate events, a household can reject and attrit, so non-attrited borrowers $\geqslant$ total - (rejected members + attrited members). USD 1 is about BDT 80. \textsf{RiskPrefIndex} is an index where a larger number is associated with more risk tolerance. \textsf{TimePref1Index} is an index where a larger number is associated with greater future discounting in a 3 month time frame, and \textsf{TimePref2Index} is in a 1 year and 3 month time frame. \textsf{RiskPrefVal} is the minimum expected monetary value of the risky option that a respondent chose over a certainty option. Lower values indicate a greater risk tolerance. \textsf{TimePref1val} is the respondent's choice of the minimum expected monetary value of future benefits in a 3 month time frame over the present monetary benefit, and \textsf{TimePref2Val} is in a 1 year and 3 month time frame. Lower values indicate a greater patience. If a respondent's \textsf{TimePref1val} is greater than \textsf{TimePref2val}, the respondent is considered to be present-biased. \textsf{PresentBias} is an indicator function that takes the value of 1 if the respondent is considered to be present-biased, 0 otherwise. \setlength{\baselineskip}{7pt}
\end{tabular}
\end{minipage}

Household as a unit of observation. 


\hspace{-1.0cm}\begin{minipage}[t]{14cm}
\hfil\textsc{\normalsize Table \refstepcounter{table}\thetable: Permutation test results at household level\label{tab permHH}}\\
\setlength{\tabcolsep}{.5pt}
\setlength{\baselineskip}{8pt}
\renewcommand{\arraystretch}{.50}
\hfil\begin{tikzpicture}
\node (tbl) {\input{c:/data/GUK/analysis/save/EstimationMemo/HHLevelPermutationTestResults.tex}};
%\input{c:/dropbox/data/ramadan/save/tablecolortemplate.tex}
\end{tikzpicture}\\
\renewcommand{\arraystretch}{.8}
\setlength{\tabcolsep}{1pt}
\begin{tabular}{>{\hfill\scriptsize}p{1cm}<{}>{\scriptsize}p{13cm}<{\hfill}}
Source:& stimated with GUK administrative and survey data.\\
Notes: & \textsf{R}'s package \textsf{coin} is used on baseline covariates to conduct approximate permutation tests. For each variable, the first row indicates means and the second row indicates $p$ values in percentages of permutation tests. Number of repetition is set to 100000. Step-down method is used to adjust for multiple testing of a multi-factor grouping variable. $P$ values for observed test statistic and the null distribution are expressed in per centage units.
\end{tabular}
\end{minipage}






\end{document}
